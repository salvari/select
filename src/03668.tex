
%(BEGIN_QUESTION)
% Copyright 2005, Tony R. Kuphaldt, released under the Creative Commons Attribution License (v 1.0)
% This means you may do almost anything with this work of mine, so long as you give me proper credit

Assume that the switch in this circuit is toggled (switched positions) once every 5 seconds, beginning in the "up" (charge) position at time $t = 0$, and that the capacitor begins in a fully discharged state at that time.  Determine the capacitor voltage at each switch toggle:

$$\epsfbox{03668x01.eps}$$

% No blank lines allowed between lines of an \halign structure!
% I use comments (%) instead, so that TeX doesn't choke.

$$\vbox{\offinterlineskip
\halign{\strut
\vrule \quad\hfil # \ \hfil & 
\vrule \quad\hfil # \ \hfil & 
\vrule \quad\hfil # \ \hfil \vrule \cr
\noalign{\hrule}
%
% First row
Time & Switch motion & $V_C$ (volts) \cr
%
\noalign{\hrule}
%
0 s & discharge $\to$ charge & 0 volts \cr
%
\noalign{\hrule}
%
5 s & charge $\to$ discharge &  \cr
%
\noalign{\hrule}
%
10 s & discharge $\to$ charge &  \cr
%
\noalign{\hrule}
%
15 s & charge $\to$ discharge &  \cr
%
\noalign{\hrule}
%
20 s & discharge $\to$ charge &  \cr
%
\noalign{\hrule}
%
25 s & charge $\to$ discharge &  \cr
%
\noalign{\hrule}
} % End of \halign 
}$$ % End of \vbox

\underbar{file 03668}
%(END_QUESTION)





%(BEGIN_ANSWER)

% No blank lines allowed between lines of an \halign structure!
% I use comments (%) instead, so that TeX doesn't choke.

$$\vbox{\offinterlineskip
\halign{\strut
\vrule \quad\hfil # \ \hfil & 
\vrule \quad\hfil # \ \hfil & 
\vrule \quad\hfil # \ \hfil \vrule \cr
\noalign{\hrule}
%
% First row
Time & Switch motion & $V_C$ (volts) \cr
%
\noalign{\hrule}
%
0 s & discharge $\to$ charge & 0 volts \cr
%
\noalign{\hrule}
%
5 s & charge $\to$ discharge & 6.549 volts \cr
%
\noalign{\hrule}
%
10 s & discharge $\to$ charge & 2.260 volts \cr
%
\noalign{\hrule}
%
15 s & charge $\to$ discharge & 7.329 volts \cr
%
\noalign{\hrule}
%
20 s & discharge $\to$ charge & 2.529 volts \cr
%
\noalign{\hrule}
%
25 s & charge $\to$ discharge & 7.422 volts \cr
%
\noalign{\hrule}
} % End of \halign 
}$$ % End of \vbox

%(END_ANSWER)





%(BEGIN_NOTES)

Be sure to have your students share their problem-solving techniques (how they determined which equation to use, etc.) in class.  See how many of them notice that the exponential portion of the equation ($e^{t \over \tau}$) is the same for each calculation, and if they find an easy way to manage the calculations by storing charge/discharge percentages in their calculator memories!

%INDEX% Time constant calculation, RC circuit

%(END_NOTES)


