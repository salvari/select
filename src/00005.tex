
%(BEGIN_QUESTION)
% Copyright 2003, Tony R. Kuphaldt, released under the Creative Commons Attribution License (v 1.0)
% This means you may do almost anything with this work of mine, so long as you give me proper credit

If you scuff your feet across a carpeted surface on a dry day, you will produce an electric potential resulting from a static electric charge that may range in the order of tens of thousands of volts!  Can this pose a danger to you, at least in principle?

\underbar{file 00005}
%(END_QUESTION)





%(BEGIN_ANSWER)

Static electric charges rarely pose any shock hazard, because the actual charge quantity (measured in {\it coulombs}) is so small that the resulting current upon discharge can only last a very brief moment in time.

%(END_ANSWER)





%(BEGIN_NOTES)

This question is a good starting point for a discussion on {\it time} as a variable in determining electric shock hazard.  There is more to determining hazard to the human body than a simple assessment of volts, amps, and ohms!

%INDEX% Electric shock
%INDEX% Ohm's Law, conceptual

%(END_NOTES)


