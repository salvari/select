
%(BEGIN_QUESTION)
% Copyright 2005, Tony R. Kuphaldt, released under the Creative Commons Attribution License (v 1.0)
% This means you may do almost anything with this work of mine, so long as you give me proper credit

Suppose you owned a scientific calculator with two broken buttons: the power ($y^x$) and root ($\root x \of y$).  Demonstrate how you could solve this simple root problem using only logarithms, division, and antilogarithms (powers):

$$\root 3 \of 8 = \hbox{???}$$

The answer to this problem was easy enough for you to figure out without a calculator at all, so here are some more practice problems for you to try:

\medskip
\goodbreak
\item{$\bullet$} $\root 4 \of {13}$ = 
\vskip 5pt
\item{$\bullet$} $\root 5 \of {209}$ = 
\vskip 5pt
\item{$\bullet$} $\root 2.5 \of {9935}$ = 
\vskip 5pt
\item{$\bullet$} $\root 9.2 \of {0.15}$ = 
\medskip

\underbar{file 02689}
%(END_QUESTION)





%(BEGIN_ANSWER)

Here I will show you the steps to using logarithms to solve the first multiplication problem:

$$\root 3 \of 8 = \hbox{???}$$

$$\root 3 \of 8 = 10^{\left( {1 \over 3} \log 8 \right)}$$

$$\root 3 \of 8 = 10^{\left( {1 \over 3} (0.9031) \right)}$$

$$\root 3 \of 8 = 10^{0.3010}$$

$$\root 3 \of 8 = 2$$

Since the others are easy enough for you to check (with your non-broken calculator!), I'll leave their solutions in your capable hands.

%(END_ANSWER)





%(BEGIN_NOTES)

Incidentally, there is nothing special about the common logarithm to warrant its exclusive use in this problem.  We could have just as easily applied the natural logarithm function with the same (final) result:

$$\root 3 \of 8 = \hbox{???}$$

$$\root 3 \of 8 = e^{\left( {1 \over 3} \ln 8 \right)}$$

$$\root 3 \of 8 = e^{\left( {1 \over 3} (2.0794) \right)}$$

$$\root 3 \of 8 = e^{0.6931}$$

$$\root 3 \of 8 = 2$$

%INDEX% Logarithms, used to transform a root problem into a division problem

%(END_NOTES)


