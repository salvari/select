
%(BEGIN_QUESTION)
% Copyright 2003, Tony R. Kuphaldt, released under the Creative Commons Attribution License (v 1.0)
% This means you may do almost anything with this work of mine, so long as you give me proper credit

For the following electronic components, determine whether they are more likely to fail {\it open} or fail {\it shorted} (this includes partial, or high-resistance, shorts):

\medskip
\item{$\bullet$} Resistors:
\item{$\bullet$} Capacitors:
\item{$\bullet$} Inductors:
\item{$\bullet$} Switches:
\item{$\bullet$} Transformers:
\item{$\bullet$} Diodes:
\item{$\bullet$} Bipolar transistors:
\item{$\bullet$} Field-effect transistors:
\item{$\bullet$} Crystals:
\medskip

I encourage you to research information on these devices' failure modes, as well as glean from your own experiences building and troubleshooting electronic circuits.

\underbar{file 01573}
%(END_QUESTION)





%(BEGIN_ANSWER)

Remember that each of these answers merely represents the {\it most likely} of the two failure modes, either open or shorted, and that probabilities may shift with operating conditions (i.e. switches may be more prone to failing shorted due to welded contacts if they are routinely abused with excessive current upon closure).

\medskip
\item{$\bullet$} Resistors: {\bf open}
\item{$\bullet$} Capacitors: {\bf shorted}
\item{$\bullet$} Inductors: {\bf open or short equally probable}
\item{$\bullet$} Switches: {\bf open}
\item{$\bullet$} Transformers: {\bf open or short equally probable}
\item{$\bullet$} Diodes: {\bf shorted}
\item{$\bullet$} Bipolar transistors: {\bf shorted}
\item{$\bullet$} Field-effect transistors: {\bf shorted}
\item{$\bullet$} Crystals: {\bf open}
\medskip

%(END_ANSWER)





%(BEGIN_NOTES)

Emphasize to your students how a good understanding of common failure modes is important to efficient troubleshooting technique.  Knowing which way a particular component is more likely to fail under normal operating conditions enables the troubleshooter to make better judgments when assessing the most probable cause of a system failure.  

Of course, proper troubleshooting technique should always reveal the source of trouble, whether or not the troubleshooter has any experience with the failure modes of particular devices.  However, possessing a detailed knowledge of failure probabilities allows one to check the most likely sources of trouble first, which generally leads to faster repairs.

A division of the United States Department of Defense (DoD) as the {\it Reliability Analysis Center}, or {\it RAC}, publishes detailed analyses of failure modes for a wide variety of components, electronic as well as non-electronic.  They may be contacted at 201 Mill Street, Rome, New York, 13440-6916.  Data for this question was gleaned from the RAC's 1997 publication, {\it Failure Mode / Mechanism Distributions} (FMD-97).

%INDEX% Resistor, probable failure mode
%INDEX% Capacitor, probable failure mode
%INDEX% Inductor, probable failure mode
%INDEX% Switch, probable failure mode
%INDEX% Transformer, probable failure mode
%INDEX% Diode, probable failure mode
%INDEX% Transistor, probable failure mode
%INDEX% Crystal, probable failure mode

%(END_NOTES)


