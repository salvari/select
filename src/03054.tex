
%(BEGIN_QUESTION)
% Copyright 2005, Tony R. Kuphaldt, released under the Creative Commons Attribution License (v 1.0)
% This means you may do almost anything with this work of mine, so long as you give me proper credit

Observe the following equivalence:

$$4^3 \times 4^2 = (4 \times 4 \times 4) \times (4 \times 4)$$

Since all operations are the same (multiplication) and reversible, the parentheses are not needed.  Therefore, we may write the expression like this:

$$4 \times 4 \times 4 \times 4 \times 4$$

Of course, the simplest way to write this is $4^5$, since there are five 4's multiplied together.

\vskip 10pt

Expand each of these expressions so that there are no exponents either:

\medskip
\goodbreak
\item{$\bullet$} $3^5 \times 3^2 = $
\vskip 5pt
\item{$\bullet$} $10^4 \times 10^3 = $
\vskip 5pt
\item{$\bullet$} $8^2 \times 8^3 = $
\vskip 5pt
\item{$\bullet$} $20^1 \times 20^2 = $
\medskip

After expanding each of these expressions, re-write each one in simplest form: one number to a power, just like the final form of the example given ($4^5$).  From these examples, what pattern do you see with exponents of products.  In other words, what is the general solution to the following expression?

$$a^m \times a^n = $$

\underbar{file 03054}
%(END_QUESTION)





%(BEGIN_ANSWER)

$$a^m \times a^n = a^{m+n}$$

%(END_ANSWER)





%(BEGIN_NOTES)

I have found that students who cannot fathom the general rule ($a^m \times a^n = a^{m+n}$) often understand for the first time when they see concrete examples.

%INDEX% Algebra, exponents
%INDEX% Exponents, algebra

%(END_NOTES)


