
%(BEGIN_QUESTION)
% Copyright 2005, Tony R. Kuphaldt, released under the Creative Commons Attribution License (v 1.0)
% This means you may do almost anything with this work of mine, so long as you give me proper credit

Predict how all transistor currents ($I_B$, $I_C$, and $I_E$) and the output voltage signal will be affected as a result of the following faults.  Consider each fault independently (i.e. one at a time, no multiple faults):

$$\epsfbox{03739x01.eps}$$

\medskip
\item{$\bullet$} Capacitor $C_{out}$ fails open:
\vskip 5pt
\item{$\bullet$} Solder bridge (short) past resistor $R_1$:
\vskip 5pt
\item{$\bullet$} Resistor $R_1$ fails open:
\vskip 5pt
\item{$\bullet$} Resistor $R_C$ fails open:
\vskip 5pt
\item{$\bullet$} Resistor $R_E$ fails open:
\vskip 5pt
\item{$\bullet$} Capacitor $C_{bypass}$ fails shorted:
\medskip

For each of these conditions, explain {\it why} the resulting effects will occur.

\underbar{file 03739}
%(END_QUESTION)





%(BEGIN_ANSWER)

\medskip
\item{$\bullet$} Capacitor $C_{out}$ fails open: {\it Transistor currents unaffected, no output signal.}
\vskip 5pt
\item{$\bullet$} Solder bridge (short) past resistor $R_1$: {\it Transistor saturates (large increase in all currents), no output signal.}
\vskip 5pt
\item{$\bullet$} Resistor $R_1$ fails open: {\it All transistor currents fall to zero (transistor in complete cutoff mode), no output signal.}
\vskip 5pt
\item{$\bullet$} Resistor $R_C$ fails open: {\it Transistor base current will decrease, zero collector current, greatly decreased emitter current, no output signal.}
\vskip 5pt
\item{$\bullet$} Resistor $R_E$ fails open: {\it All transistor currents fall to zero (transistor in complete cutoff mode), no output signal.}
\vskip 5pt
\item{$\bullet$} Capacitor $C_{bypass}$ fails shorted: {\it All transistor currents fall to zero (transistor in complete cutoff mode), no output signal.}
\medskip

%(END_ANSWER)





%(BEGIN_NOTES)

The purpose of this question is to approach the domain of circuit troubleshooting from a perspective of knowing what the fault is, rather than only knowing what the symptoms are.  Although this is not necessarily a realistic perspective, it helps students build the foundational knowledge necessary to diagnose a faulted circuit from empirical data.  Questions such as this should be followed (eventually) by other questions asking students to identify likely faults based on measurements.

%INDEX% Troubleshooting, predicting effects of fault in common-base amplifier circuit

%(END_NOTES)


