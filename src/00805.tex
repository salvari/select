
%(BEGIN_QUESTION)
% Copyright 2003, Tony R. Kuphaldt, released under the Creative Commons Attribution License (v 1.0)
% This means you may do almost anything with this work of mine, so long as you give me proper credit

\centerline{\bf Don't just sit there!  Build something!!}

\vskip 10pt

Learning to analyze digital circuits requires much study and practice.  Typically, students practice by working through lots of sample problems and checking their answers against those provided by the textbook or the instructor.  While this is good, there is a much better way.

You will learn much more by actually {\it building and analyzing real circuits}, letting your test equipment provide the "answers" instead of a book or another person.  For successful circuit-building exercises, follow these steps:

\medskip
\item{1.} Draw the schematic diagram for the digital circuit to be analyzed.
\item{2.} Carefully build this circuit on a breadboard or other convenient medium.
\item{3.} Check the accuracy of the circuit's construction, following each wire to each connection point, and verifying these elements one-by-one on the diagram.
\item{4.} Analyze the circuit, determining all output logic states for given input conditions.
\item{5.} Carefully measure those logic states, to verify the accuracy of your analysis.
\item{6.} If there are any errors, carefully check your circuit's construction against the diagram, then carefully re-analyze the circuit and re-measure.
\medskip

Always be sure that the power supply voltage levels are within specification for the logic circuits you plan to use.  If TTL, the power supply {\it must} be a 5-volt regulated supply, adjusted to a value as close to 5.0 volts DC as possible.

One way you can save time and reduce the possibility of error is to begin with a very simple circuit and incrementally add components to increase its complexity after each analysis, rather than building a whole new circuit for each practice problem.  Another time-saving technique is to re-use the same components in a variety of different circuit configurations.  This way, you won't have to measure any component's value more than once.

\underbar{file 00805}
%(END_QUESTION)





%(BEGIN_ANSWER)

Let the electrons themselves give you the answers to your own "practice problems"!

%(END_ANSWER)





%(BEGIN_NOTES)

It has been my experience that students require much practice with circuit analysis to become proficient.  To this end, instructors usually provide their students with lots of practice problems to work through, and provide answers for students to check their work against.  While this approach makes students proficient in circuit theory, it fails to fully educate them.

Students don't just need mathematical practice.  They also need real, hands-on practice building circuits and using test equipment.  So, I suggest the following alternative approach: students should {\it build} their own "practice problems" with real components, and try to predict the various logic states.  This way, the digital theory "comes alive," and students gain practical proficiency they wouldn't gain merely by solving Boolean equations or simplifying Karnaugh maps.

Another reason for following this method of practice is to teach students {\it scientific method}: the process of testing a hypothesis (in this case, logic state predictions) by performing a real experiment.  Students will also develop real troubleshooting skills as they occasionally make circuit construction errors.

Spend a few moments of time with your class to review some of the "rules" for building circuits before they begin.  Discuss these issues with your students in the same Socratic manner you would normally discuss the worksheet questions, rather than simply telling them what they should and should not do.  I never cease to be amazed at how poorly students grasp instructions when presented in a typical lecture (instructor monologue) format!

I highly recommend CMOS logic circuitry for at-home experiments, where students may not have access to a 5-volt regulated power supply.  Modern CMOS circuitry is far more rugged with regard to static discharge than the first CMOS circuits, so fears of students harming these devices by not having a "proper" laboratory set up at home are largely unfounded.

\vskip 10pt

A note to those instructors who may complain about the "wasted" time required to have students build real circuits instead of just mathematically analyzing theoretical circuits:

\vskip 10pt

\hskip 1in
{\it What is the purpose of students taking your course?}

\vskip 10pt

If your students will be working with real circuits, then they should learn on real circuits whenever possible.  If your goal is to educate theoretical physicists, then stick with abstract analysis, by all means!  But most of us plan for our students to do something in the real world with the education we give them.  The "wasted" time spent building real circuits will pay huge dividends when it comes time for them to apply their knowledge to practical problems.

Furthermore, having students build their own practice problems teaches them how to perform {\it primary research}, thus empowering them to continue their electrical/electronics education autonomously.

In most sciences, realistic experiments are much more difficult and expensive to set up than electrical circuits.  Nuclear physics, biology, geology, and chemistry professors would just love to be able to have their students apply advanced mathematics to real experiments posing no safety hazard and costing less than a textbook.  They can't, but you can.  Exploit the convenience inherent to your science, and {\it get those students of yours practicing their math on lots of real circuits!}

%INDEX% Don't just sit there -- Build something!! (digital circuits)

%(END_NOTES)


