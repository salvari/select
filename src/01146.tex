
%(BEGIN_QUESTION)
% Copyright 2003, Tony R. Kuphaldt, released under the Creative Commons Attribution License (v 1.0)
% This means you may do almost anything with this work of mine, so long as you give me proper credit

Define the following terms, as they relate to phase-locked loop circuits:

\medskip
\item{$\bullet$} Center frequency
\item{$\bullet$} Lock range 
\item{$\bullet$} Capture range
\item{$\bullet$} Lock-up time
\medskip

\underbar{file 01146}
%(END_QUESTION)





%(BEGIN_ANSWER)

\medskip
\item{$\bullet$} Center frequency: the "free-running" frequency of a PLL's voltage-controlled oscillator.
\item{$\bullet$} Lock range: the range of input signal frequencies that a PLL is able to remain "locked in" to.
\item{$\bullet$} Capture range: the range of input signal frequencies that a PLL is able to lock in to, from an unlocked state.
\item{$\bullet$} Lock-up time: the amount of time required for a PLL to "lock in" to a given input frequency, from a cold start.
\medskip

%(END_ANSWER)





%(BEGIN_NOTES)

Ask your students to explain which range -- lock or capture -- is narrower, and why.

%(END_NOTES)


