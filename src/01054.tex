
%(BEGIN_QUESTION)
% Copyright 2003, Tony R. Kuphaldt, released under the Creative Commons Attribution License (v 1.0)
% This means you may do almost anything with this work of mine, so long as you give me proper credit

Why are polarity marks ({\bf +} and {\bf -}) shown at the terminals of the components in this AC network?

$$\epsfbox{01054x01.eps}$$

Are these polarity markings really necessary?  Do they make any sense at all, given the fact that AC by its very nature has no fixed polarity (because polarity alternates over time)?  Explain your answer.

\underbar{file 01054}
%(END_QUESTION)





%(BEGIN_ANSWER)

The polarity markings provide a frame of reference for the phase angles of the voltage drops.

%(END_ANSWER)





%(BEGIN_NOTES)

Ask your students why polarity markings need to be provided in DC electrical networks, as an essential part of the voltage figures.  Why is an answer for a voltage drop incomplete if not accompanied by polarity markings in a DC circuit?  Discuss this with your students, then ask them to extrapolate this principle to AC circuits.  When we are accounting for the {\it phase shift} of a voltage drop in our answer, does the "polarity" of the voltage drop matter?

%(END_NOTES)


