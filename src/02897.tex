
%(BEGIN_QUESTION)
% Copyright 2005, Tony R. Kuphaldt, released under the Creative Commons Attribution License (v 1.0)
% This means you may do almost anything with this work of mine, so long as you give me proper credit

When studying latch circuits, you will come across many references to {\it set} and {\it reset} logic states.  Give a simple definition for each of these terms in the context of latch and flip-flop circuits.

\underbar{file 02897}
%(END_QUESTION)





%(BEGIN_ANSWER)

A latch is considered {\it set} when its output ($Q$) is high, and {\it reset} when its output ($Q$) is low.

%(END_ANSWER)





%(BEGIN_NOTES)

Having a consistent definition for "set" and "reset" is important, especially as students study multiple latch circuit topologies and active-low inputs!

%INDEX% Reset state, defined for digital latch circuit
%INDEX% Set state, defined for digital latch circuit

%(END_NOTES)


