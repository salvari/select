
%(BEGIN_QUESTION)
% Copyright 2005, Tony R. Kuphaldt, released under the Creative Commons Attribution License (v 1.0)
% This means you may do almost anything with this work of mine, so long as you give me proper credit

A common term used to describe the internal workings of a programmable logic device is a {\it macrocell}.  What, exactly, is a macrocell?

\underbar{file 03051}
%(END_QUESTION)





%(BEGIN_ANSWER)

A macrocell is a collection of logic gates and a flip-flop, lumped together in one unit.  PLDs usually have many macrocells, which may be interconnected to form a variety of synchronous logic functions.

%(END_ANSWER)





%(BEGIN_NOTES)

Ask your students to show you where they found their information, and if they were able to determine how many macrocells are in a typical PLD.

%INDEX% Macrocell (programmable logic device), defined

%(END_NOTES)


