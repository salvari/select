
%(BEGIN_QUESTION)
% Copyright 2005, Tony R. Kuphaldt, released under the Creative Commons Attribution License (v 1.0)
% This means you may do almost anything with this work of mine, so long as you give me proper credit

An {\it electric arc welder} is a low-voltage, high-current power source designed to supply enough electric current to sustain an arc capable of welding metal with its high temperature:

$$\epsfbox{03292x01.eps}$$

It is possible to derive a Norton equivalent circuit for an arc welder based on empirical measurements of voltage and current.  Take for example these measurements, under {\it loaded} and {\it no-load} conditions:

$$\epsfbox{03292x02.eps}$$

$$\epsfbox{03292x03.eps}$$

Based on these measurements, draw a Norton equivalent circuit for the arc welder.

\underbar{file 03292}
%(END_QUESTION)





%(BEGIN_ANSWER)

$$\epsfbox{03292x04.eps}$$

%(END_ANSWER)





%(BEGIN_NOTES)

This practical scenario shows how Norton's theorem may be used to "model" a complex device as two simple components (current source and resistor).  Of course, we must make certain assumptions when modeling in this fashion: we assume, for instance, that the arc welder is a linear device, which may or may not be true.

Incidentally, there is such a thing as a DC-measuring clamp-on ammeter as shown in the illustrations, in case any one of your students ask.  AC clamp-on meters are simpler, cheaper, and thus more popularly known, but devices using the Hall effect are capable of inferring DC current by the strength of an unchanging magnetic field, and these Hall-effect devices are available at modest expense.

%INDEX% Norton's theorem, experimentally determining Norton current and Norton resistance of network

%(END_NOTES)


