
%(BEGIN_QUESTION)
% Copyright 2003, Tony R. Kuphaldt, released under the Creative Commons Attribution License (v 1.0)
% This means you may do almost anything with this work of mine, so long as you give me proper credit

Explain what this graph means, and how it represents both {\it saturation} and {\it hysteresis} as magnetic phenomena:

$$\epsfbox{00472x01.eps}$$

\underbar{file 00472}
%(END_QUESTION)





%(BEGIN_ANSWER)

This is a {\it B-H} curve, plotting the magnetic flux density ($B$) against the magnetic field intensity ($H$) of an electromagnet.  The arrowheads represent the directions of increase and decrease in the variables.

"Saturation" is when $B$ changes little for substantial changes in $H$.  There are two regions on the B-H curve where saturation is evident.

%(END_ANSWER)





%(BEGIN_NOTES)

This question is worthy of much discussion.  It is one thing to recognize this curve as being a $B-H$ curve, and quite another to explain exactly what it means.  Ask your students to show on the curve, for instance, what happens when an electromagnet is fully energized with DC, and then the current is shut off, leaving a residual flux in the core.  What is necessary to de-magnetize the core once again?

Also be sure to discuss saturation in detail.  This is a very important magnetic phenomenon, without a direct analogy in electric circuits (it is not as though wires "saturate" when carrying too much current!).

%INDEX% B-H curve, for ferrous material
%INDEX% Saturation, magnetic

%(END_NOTES)


