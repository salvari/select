
%(BEGIN_QUESTION)
% Copyright 2003, Tony R. Kuphaldt, released under the Creative Commons Attribution License (v 1.0)
% This means you may do almost anything with this work of mine, so long as you give me proper credit

An audio power amplifier with an internal impedance of 8 $\Omega$ needs to power a set of speakers with a combined total impedance of 1 $\Omega$.  We know that connecting this speaker array directly to the amplifier's output will not result in optimum power transfer, because of the impedance mismatch.

Someone suggests using a transformer to match the two disparate impedances, but what turns ratio does this transformer need to have?  Should it be used in a step-up configuration, or a step-down configuration?  Explain your answers.

\underbar{file 00663}
%(END_QUESTION)





%(BEGIN_ANSWER)

2.83:1 winding ratio, step-down.

%(END_ANSWER)





%(BEGIN_NOTES)

Students should know at this point how to calculate the impedance transformation ratio from a transformer's winding ratio.  In this question, they are challenged to calculate "backwards" to find the winding ratio from the impedance ratio.

%INDEX% Impedance transformation

%(END_NOTES)


