
%(BEGIN_QUESTION)
% Copyright 2003, Tony R. Kuphaldt, released under the Creative Commons Attribution License (v 1.0)
% This means you may do almost anything with this work of mine, so long as you give me proper credit

There is a lot of incorrect terminology and information in the world of high-fidelity audio equipment, due primarily to a large consumer base lacking technical knowledge.  One of the common mis-statements of audio amplifier performance is {\it power}, expressed in "watts peak" and "watts RMS".  While the term "peak power" is not necessarily incorrect, "RMS power" most definitely is.  What is wrong with the latter term, and what do you think audio equipment manufacturers mean when they specify an amplifier's power rating in "watts RMS"?

\underbar{file 00761}
%(END_QUESTION)





%(BEGIN_ANSWER)

"RMS" is a means of expressing an AC voltage or current in equivalent terms with DC voltage or current, based on an equality of power between the two.  There is no such thing as "RMS power," only "RMS voltage" or "RMS current."

%(END_ANSWER)





%(BEGIN_NOTES)

Having dispensed with the oxymoronic notion of "RMS power," we now proceed to the next question: what do amplifier manufacturers mean by this rating?  Your students should have done some research on amplifier power ratings, and seen what the manufacturers say about them.  Based on that research, what do your students think the manufacturers are trying to state when they talk about "RMS power"?  Is there a better way to say this?

%(END_NOTES)


