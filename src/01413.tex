
%(BEGIN_QUESTION)
% Copyright 2003, Tony R. Kuphaldt, released under the Creative Commons Attribution License (v 1.0)
% This means you may do almost anything with this work of mine, so long as you give me proper credit

The circuit shown here is a four-bit analog-to-digital converter (ADC).  Specifically, it is a {\it flash} converter, so named because of its high speed:

$$\epsfbox{01413x01.eps}$$

Explain why we must use a {\it priority} encoder to encode the comparator outputs into a four-bit binary code, and not a regular encoder.  What problem(s) would we have if we were to use a non-priority encoder in this ADC circuit?

\underbar{file 01413}
%(END_QUESTION)





%(BEGIN_ANSWER)

I won't directly answer this question, but instead pose a "thought experiment."  Suppose the analog input voltage ($V_{in}$) were slowly increased from 0 volts to the reference voltage ($V_{ref}$).  What do the outputs of the comparators do, one at a time, as the analog input voltage increases?  What input conditions does the encoder see?  How would a primitive "diode network" type of encoder (which we know does {\it not} encode based on priority) interpret the comparator outputs?

%(END_ANSWER)





%(BEGIN_NOTES)

Here, I show students a very practical application of a priority encoder, in which the necessity of priority encoding should be apparent after some analysis of the circuit.

%INDEX% ADC
%INDEX% Analog-to-digital conversion
%INDEX% Priority encoder

%(END_NOTES)


