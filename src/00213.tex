
%(BEGIN_QUESTION)
% Copyright 2003, Tony R. Kuphaldt, released under the Creative Commons Attribution License (v 1.0)
% This means you may do almost anything with this work of mine, so long as you give me proper credit

Suppose a length of resistive material (such as {\it nichrome} wire) had three points of electrical contact: one at each end (points 1 and 3), plus a movable metal "wiper" making contact at some point between the two ends (point 2):

$$\epsfbox{00213x01.eps}$$

Describe what happens to the amount of electrical resistance between the following points, as the wiper is moved toward the left end of the resistive element (toward point 1)?  State your answers in terms of "increase," "decrease," or "remains the same," and explain why each answer is so.
 
$$\epsfbox{00213x02.eps}$$

\medskip
\item{$\bullet$} Between points 1 and 2, resistance . . .
\item{$\bullet$} Between points 2 and 3, resistance . . .
\item{$\bullet$} Between points 1 and 3, resistance . . .
\medskip

\underbar{file 00213}
%(END_QUESTION)





%(BEGIN_ANSWER)

As the wiper moves to the left (toward point 1):

\medskip
\item{$\bullet$} Between points 1 and 2, resistance {\it decreases}
\item{$\bullet$} Between points 2 and 3, resistance {\it increases}
\item{$\bullet$} Between points 1 and 3, resistance remains the same
\medskip

%(END_ANSWER)





%(BEGIN_NOTES)

The purpose of this question is to get students to comprehend the function of a {\it potentiometer}, before they have ever seen one.

%INDEX% Potentiometer function

%(END_NOTES)


