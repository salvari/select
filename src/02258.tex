
%(BEGIN_QUESTION)
% Copyright 2004, Tony R. Kuphaldt, released under the Creative Commons Attribution License (v 1.0)
% This means you may do almost anything with this work of mine, so long as you give me proper credit

Ideally, a sinusoidal oscillator will output a signal consisting of a single (fundamental) frequency, with no harmonics.  Realistically, though, sine-wave oscillators always exhibit some degree of distortion, and are therefore never completely harmonic-free.

Describe what the display of a spectrum analyzer would look like when connected to the output of a {\it perfect} sinusoidal oscillator.  Then, describe what the same instrument's display would look like if the oscillator exhibited substantial distortion.

\underbar{file 02258}
%(END_QUESTION)





%(BEGIN_ANSWER)

I'll let you figure out the answer to this question on your own.

%(END_ANSWER)





%(BEGIN_NOTES)

The purpose of this question is to get students to think about how a spectrum analyzer would be used in a practical scenario, and what the spectrum would look like for a couple of different scenarios.  Really, it focuses more on the harmonic analysis instrument (the spectrum analyzer) more than the oscillator circuit.

%INDEX% Spectrum analysis, of oscillator circuits

%(END_NOTES)


