
%(BEGIN_QUESTION)
% Copyright 2003, Tony R. Kuphaldt, released under the Creative Commons Attribution License (v 1.0)
% This means you may do almost anything with this work of mine, so long as you give me proper credit

Explain why Gray code is often used in rotary encoders rather than binary coding.  What difference does it make what type of code we use to mark the sectors of an encoder disk, so long as each sector possesses a unique number?

\underbar{file 01238}
%(END_QUESTION)





%(BEGIN_ANSWER)

Gray code markings are more tolerant of sensor misalignment than binary markings, because there is no need for perfect synchronization of multiple bit transitions between sectors.

%(END_ANSWER)





%(BEGIN_NOTES)

This is perhaps the most important reason for using Gray code in encoder marking, but it is not necessarily obvious {\it why} to the new student.  I found that making a physical mock-up of a binary-coded wheel versus a Gray-coded wheel helped me better present this concept to students.  Those students with better visualization/spatial relations skills will grasp this concept faster than the others, so you might want to solicit their help in explaining it to the rest of the class.

%INDEX% Gray code, used for electromechanical encoders

%(END_NOTES)


