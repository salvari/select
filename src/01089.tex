
%(BEGIN_QUESTION)
% Copyright 2003, Tony R. Kuphaldt, released under the Creative Commons Attribution License (v 1.0)
% This means you may do almost anything with this work of mine, so long as you give me proper credit

All thyristor devices exhibit the property of {\it hysteresis}.  From an electrical perspective, what is "hysteresis"?  How does this behavior differ from that of "normal" active semiconductor components such as bipolar or field-effect transistors?

\underbar{file 01089}
%(END_QUESTION)





%(BEGIN_ANSWER)

Once turned on, a thyristor tends to remain in the "on" state, and visa-versa.

%(END_ANSWER)





%(BEGIN_NOTES)

The hysteretic action of thyristors is often referred to as {\it latching}.  Ask your students to relate this term to the action of a thyristor.  Why is "latching" an appropriate term for this behavior?  Can your students think of any applications for such a device?

%INDEX% Hysteresis, in relation to thyristors

%(END_NOTES)


