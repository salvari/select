
%(BEGIN_QUESTION)
% Copyright 2003, Tony R. Kuphaldt, released under the Creative Commons Attribution License (v 1.0)
% This means you may do almost anything with this work of mine, so long as you give me proper credit

A very important measure of a transistor's behavior is its {\it characteristic curves}, a set of graphs showing collector current over a wide range of collector-emitter voltage drops, for a given amount of base current.  The following plot is a typical curve for a bipolar transistor with a fixed value of base current:

$$\epsfbox{00940x01.eps}$$

A "test circuit" for collecting data to make this graph looks like this:

$$\epsfbox{00940x02.eps}$$

Identify three different regions on this graph: {\it saturation}, {\it active}, and {\it breakdown}, and explain what each of these terms mean.  Also, identify which part of this curve the transistor acts most like a current-regulating device.

\underbar{file 00940}
%(END_QUESTION)





%(BEGIN_ANSWER)

$$\epsfbox{00940x03.eps}$$

The transistor's best current-regulation behavior occurs in its "active" region.

\vskip 10pt

Follow-up question: what might the characteristic curves look like for a transistor that is failed {\it shorted} between its collector and emitter terminals?  What about the curves for a transistor that is failed {\it open}?

%(END_ANSWER)





%(BEGIN_NOTES)

Ask your students what a perfect current-regulating curve would look like.  How does this perfect curve compare with the characteristic curve shown in this question for a typical transistor?

A word of caution is in order: I do not recommend that a test circuit such as the one shown in the question be built for collecting curve data.  If the transistor dissipates power for any substantial amount of time, it will heat up and its curves will change dramatically.  Real transistor curves are generated by a piece of test equipment called a "curve tracer," which sweeps the collector-emitter voltage and steps the base current very rapidly (fast enough to "paint" all curves on an oscilloscope screen before the phosphor stops glowing).

%INDEX% Characteristic curve, BJT

%(END_NOTES)


