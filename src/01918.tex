
%(BEGIN_QUESTION)
% Copyright 2003, Tony R. Kuphaldt, released under the Creative Commons Attribution License (v 1.0)
% This means you may do almost anything with this work of mine, so long as you give me proper credit

Suppose a metal-detecting sensor were connected to a strobe light, so that the light flashed every time a fan blade passed by the sensor.  How would this setup differ in operation from one where the strobe light is free-running?

$$\epsfbox{01918x01.eps}$$

\underbar{file 01918}
%(END_QUESTION)





%(BEGIN_ANSWER)

In this system, the fan would {\it always} appear to "stand still" in a position where a fan blade is near the sensor.

\vskip 10pt

Follow-up question: how would the strobe light respond if the fan speed were changed?  Explain your answer in detail.

%(END_ANSWER)





%(BEGIN_NOTES)

This question previews the concept of oscilloscope triggering: waiting until an event occurs before plotting the shape of a moving waveform.  Often I find new students relate better to such mechanical analogies than directly to electronic abstractions when first learning oscilloscope operation.

An important detail to note in this scenario is that the strobe will flash {\it four times} per fan rotation!

%INDEX% Strobe light, as analogy to oscilloscope sweep
%INDEX% Oscilloscope sweep, analogous to strobe light used to "freeze" rotational motion

%(END_NOTES)


