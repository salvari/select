
%(BEGIN_QUESTION)
% Copyright 2003, Tony R. Kuphaldt, released under the Creative Commons Attribution License (v 1.0)
% This means you may do almost anything with this work of mine, so long as you give me proper credit

In solitary atoms, electrons are free to inhabit only certain, discrete energy states.  However, in solid materials where there are many atoms in close proximity to each other, {\it bands} of energy states form.  Explain what it means for there to be an energy "band" in a solid material, and why these "bands" form.

\underbar{file 00901}
%(END_QUESTION)





%(BEGIN_ANSWER)

Pauli's Exclusion Principle states that "No two electrons in close proximity may inhabit the exact same quantum state."  Therefore, when lots of atoms are packed together in close proximity, their individual electron states shift energy levels slightly to become continuous {\it bands} of energy levels.

%(END_ANSWER)





%(BEGIN_NOTES)

Ask your students to think of analogies to illustrate this principle.  Where else do we see multiple, individual entities joining together to form a larger (continuous) whole?

%INDEX% Bands, electron
%INDEX% Electron bands
%INDEX% Quantum physics

%(END_NOTES)


