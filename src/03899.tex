
%(BEGIN_QUESTION)
% Copyright 2006, Tony R. Kuphaldt, released under the Creative Commons Attribution License (v 1.0)
% This means you may do almost anything with this work of mine, so long as you give me proper credit

The zener diodes shown in the schematic are there to absorb transient voltages resulting when the MOSFETs turn off.  Explain where these transients originate from, and what might happen if the zener diodes were not there.

\underbar{file 03899}
%(END_QUESTION)





%(BEGIN_ANSWER)

Energy stored in the inductance of the primary winding must go somewhere when the MOSFETs turn off and the magnetic field collapses.  Zener diodes provide a safe way to dissipate this stored energy.

\vskip 10pt

Follow-up question: when the inverter circuit runs unloaded (no AC loads connected to the secondary of the transformer), the zener diodes may become warm.  Interestingly, these same diodes will cool off when an AC load is connected.  Explain why this is.

%(END_ANSWER)





%(BEGIN_NOTES)

Energy storage and transfer are vitally important concepts to grasp for power conversion circuits such as this.  The follow-up question was the result of actually testing this inverter circuit and monitoring its performance loaded and unloaded.

%(END_NOTES)


