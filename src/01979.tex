
%(BEGIN_QUESTION)
% Copyright 2003, Tony R. Kuphaldt, released under the Creative Commons Attribution License (v 1.0)
% This means you may do almost anything with this work of mine, so long as you give me proper credit

$$\epsfbox{01979x01.eps}$$

\underbar{file 01979}
\vfil \eject
%(END_QUESTION)





%(BEGIN_ANSWER)

Use circuit simulation software to verify your predicted and measured parameter values.

%(END_ANSWER)





%(BEGIN_NOTES)

Any diodes will work for this, so long as the source frequency is not too high.  I recommend students set the volts/division controls on both channels to the exact same range, so that the slope of the clipped wave near zero-crossing may seen to be exactly the same as the slope of the input sine wave at the same points.  This makes it absolutely clear that the output waveform is nothing more than the input waveform with the tops and bottoms cut off.

An extension of this exercise is to incorporate troubleshooting questions.  Whether using this exercise as a performance assessment or simply as a concept-building lab, you might want to follow up your students' results by asking them to predict the consequences of certain circuit faults.

%INDEX% Assessment, performance-based (Diode clipper circuit)

%(END_NOTES)


