
%(BEGIN_QUESTION)
% Copyright 2003, Tony R. Kuphaldt, released under the Creative Commons Attribution License (v 1.0)
% This means you may do almost anything with this work of mine, so long as you give me proper credit

Suppose you are finishing a maintenance project where an electric motor was locked out and tagged, and now the work is complete.  Your lock is the last one to be removed from the circuit breaker, everyone else already having taken their locks and tags off.  What should you do before removing your lock and turning the circuit breaker back on?

\underbar{file 00575}
%(END_QUESTION)





%(BEGIN_ANSWER)

You should check the equipment site to be sure no one is still working on it, unaware of the impending startup.

%(END_ANSWER)





%(BEGIN_NOTES)

Real-life story here: I was once asked to place an electric motor back in service after it had been locked out for a few days, for routine maintenance.  I removed my lock and tag, and was just about to turn the breaker back on, when better judgment prevailed and I decided to first check the job site.  Lo and behold, there, still working on the motor coupling, were two contract employees completely oblivious to the situation.  They had not been told there was a circuit breaker to secure power to that electric motor, nor were they aware that they needed to lock it out in addition to everyone else on the project!  Had I turned that circuit breaker back on, the motor could have started up and severely injured at least one of them!

Lesson to be learned: if you are performing work on a piece of equipment, {\it you} need to have {\it your} lock and {\it your} tag securing energy to that equipment.  Never, ever trust someone else to lock-out and tag-out a circuit breaker for you!

%INDEX% Lock-out / tag-out

%(END_NOTES)


