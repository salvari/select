
%(BEGIN_QUESTION)
% Copyright 2003, Tony R. Kuphaldt, released under the Creative Commons Attribution License (v 1.0)
% This means you may do almost anything with this work of mine, so long as you give me proper credit

A common saying about electricity is that "it always takes the path of least resistance."  Explain how this proverb relates to the following circuit, where electric current from the battery encounters two alternate paths, one being less resistive than the other:

$$\epsfbox{00085x01.eps}$$

\underbar{file 00085}
%(END_QUESTION)





%(BEGIN_ANSWER)

The 250 $\Omega$ resistor will experience a current of 40 mA, while the 800 $\Omega$ resistor will experience a current of 12.5 mA.

%(END_ANSWER)





%(BEGIN_NOTES)

As an instructor, I was very surprised to hear many beginning students claim that {\it all} current would go through the lesser resistor, and {\it none} through the greater resistor!  The proverb about "takes the path of least resistance" really should be understood as "{\it proportionately} taking paths of lesser resistance."  People new to the study of electricity often misunderstand such basic principles, their errors usually based on folk wisdom like this.  It is imperative to break through these myths with hard fact.  In this case, Ohm's Law serves as a mathematical tool we can use to dispel false ideas.

Of course, a circuit as simple as this may be readily assembled and tested in class, so that all may see the truth for themselves.

%INDEX% Ohm's Law, conceptual
%INDEX% Electricity "taking the path of least resistance"

%(END_NOTES)


