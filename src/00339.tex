
%(BEGIN_QUESTION)
% Copyright 2003, Tony R. Kuphaldt, released under the Creative Commons Attribution License (v 1.0)
% This means you may do almost anything with this work of mine, so long as you give me proper credit

What would you expect the voltmeter in the following circuit to do when the potentiometer wiper is moved to the right?

$$\epsfbox{00339x01.eps}$$

What would you expect the voltmeter to register if the wiper were set precisely at the 50\% (half-way) position?

\underbar{file 00339}
%(END_QUESTION)





%(BEGIN_ANSWER)

As the wiper is moved to the right, we should expect to see the voltmeter register an increasing voltage, ranging between 0 and 12 volts DC.

\vskip 10pt

Follow-up question: a useful problem-solving method is to imagine multiple "test case" scenarios, sometimes referred to as {\it thought experiments}, for the purpose of identifying a trend.  For instance, in this circuit you might imagine what the voltmeter would register with the wiper moved all the way to the left, and then with the wiper moved all the way to the right.  Identify the voltmeter's readings in these two scenarios, and then explain why the analysis of "test cases" like these are useful in problem-solving.

%(END_ANSWER)





%(BEGIN_NOTES)

Ask your students to describe possible applications for a circuit like this.  Where might we wish to utilize a potentiometer in such a way that it outputs a variable voltage, from a constant voltage source?

%INDEX% Potentiometer, as variable voltage divider

%(END_NOTES)


