
%(BEGIN_QUESTION)
% Copyright 2003, Tony R. Kuphaldt, released under the Creative Commons Attribution License (v 1.0)
% This means you may do almost anything with this work of mine, so long as you give me proper credit

Sadly, many introductory textbooks oversimplify the definition of a {\it semiconductor} by declaring them to be substances whose atoms contain four valence-shell (outer level) electrons.  Silicon and germanium are traditionally given as the two major semiconductor materials used.

However, there is more to a "semiconductor" than this simple definition.  Take for instance the element carbon, which also has four valence electrons just like atoms of silicon and germanium.  But not all forms of carbon are semiconducting: diamond is (at high temperatures), but graphite is not, and microscopic tubes known as "carbon nanotubes" may be made either conducting or semiconducting just by varying their diameter and "twist rate."

Provide a more accurate definition of what makes a "semiconductor," based on electron bands.  Also, name some other semiconducting substances.

\underbar{file 02002}
%(END_QUESTION)





%(BEGIN_ANSWER)

Semiconducting substances are defined by the size of the gap between the valence and conduction bands.  In elemental substances, this definition is generally met in crystalline materials having four valence electrons.  However, other materials also meet the band gap criterion and thus are also semiconductors.  A few are listed here:

\medskip
\item{$\bullet$} Gallium arsenide (GaAs)
\item{$\bullet$} Gallium nitride
\item{$\bullet$} Silicon carbide
\item{$\bullet$} Some plastics (!)
\medskip

While Gallium Arsenide is broadly used at the time of this writing (2004), the others are mostly in developmental stages.  However, some of them show great promise, especially gallium nitride and silicon carbide in applications of high power, high temperature, and/or high frequency.

%(END_ANSWER)





%(BEGIN_NOTES)

I find it frustrating how many introductory electronics texts butcher the subject of semiconductor physics in an effort to "dumb it down" for technician consumption, when in fact these inaccuracies really obfuscate the subject.  Furthermore, I have yet to read (October 2004) an introductory text that even bothers to mention substances other than silicon and germanium as semiconductors, despite a great deal of research and development taking place in the field of semiconductor materials

Thankfully, the internet provides a wealth of up-to-date information on the subject, much of it simple enough for beginning students to understand.  This question is designed to get students researching sources other than their (poorly written) textbooks.

%INDEX% Bands, electron
%INDEX% Semiconductor, defined

%(END_NOTES)


