
%(BEGIN_QUESTION)
% Copyright 2003, Tony R. Kuphaldt, released under the Creative Commons Attribution License (v 1.0)
% This means you may do almost anything with this work of mine, so long as you give me proper credit

Suppose that a variable-voltage AC source is adjusted until it dissipates the exact same amount of power in a standard load resistance as a DC voltage source with an output of 120 volts:

$$\epsfbox{00402x01.eps}$$

In this condition of equal power dissipation, how much voltage is the AC power supply outputting?  Be as specific as you can in your answer.

\underbar{file 00402}
%(END_QUESTION)





%(BEGIN_ANSWER)

120 volts AC RMS, {\it by definition}.

%(END_ANSWER)





%(BEGIN_NOTES)

Ask your students, "how much {\it peak} voltage is the AC power source outputting?  More or less than 120 volts?"

If one of your students claims to have calculated the peak voltage as 169.7 volts, ask them how they arrived at that answer.  Then ask if that answer depends on the shape of the waveform (it does!).  Note that the question did not specify a "sinusoidal" wave shape.  Realistically, an adjustable-voltage AC power supply of substantial power output will likely be sinusoidal, being powered from utility AC power, but it {\it could} be a different wave-shape, depending on the nature of the source!

%INDEX% RMS (AC), defined

%(END_NOTES)


