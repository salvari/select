
%(BEGIN_QUESTION)
% Copyright 2005, Tony R. Kuphaldt, released under the Creative Commons Attribution License (v 1.0)
% This means you may do almost anything with this work of mine, so long as you give me proper credit

Determine the amount of time needed after switch closure for the capacitor voltage ($V_C$) to reach the specified levels:

$$\epsfbox{03118x01.eps}$$

% No blank lines allowed between lines of an \halign structure!
% I use comments (%) instead, so that TeX doesn't choke.

$$\vbox{\offinterlineskip
\halign{\strut
\vrule \quad\hfil # \ \hfil & 
\vrule \quad\hfil # \ \hfil \vrule \cr
\noalign{\hrule}
%
% First row
$V_C$ & Time  \cr
%
\noalign{\hrule}
%
% Second row
0 volts &  \cr
%
\noalign{\hrule}
%
% Third row
-5 volts &  \cr
%
\noalign{\hrule}
%
% Fourth row
-10 volts &  \cr
%
\noalign{\hrule}
%
% Fifth row
-15 volts &  \cr
%
\noalign{\hrule}
%
% Sixth row
-19 volts &  \cr
%
\noalign{\hrule}
} % End of \halign 
}$$ % End of \vbox

Trace the direction of current in the circuit while the capacitor is charging, and be sure to denote whether you are using electron or conventional flow notation.

\vskip 10pt

Note: the voltages are specified as negative quantities because they are negative with respect to (positive) ground in this particular circuit.

\underbar{file 03118}
%(END_QUESTION)





%(BEGIN_ANSWER)

% No blank lines allowed between lines of an \halign structure!
% I use comments (%) instead, so that TeX doesn't choke.

$$\vbox{\offinterlineskip
\halign{\strut
\vrule \quad\hfil # \ \hfil & 
\vrule \quad\hfil # \ \hfil \vrule \cr
\noalign{\hrule}
%
% First row
$V_C$ & Time  \cr
%
\noalign{\hrule}
%
% Second row
0 volts & 0 ms \cr
%
\noalign{\hrule}
%
% Third row
-5 volts & 29.75 ms \cr
%
\noalign{\hrule}
%
% Fourth row
-10 volts & 71.67 ms \cr
%
\noalign{\hrule}
%
% Fifth row
-15 volts & 143.3 ms \cr
%
\noalign{\hrule}
%
% Sixth row
-19 volts & 309.8 ms \cr
%
\noalign{\hrule}
} % End of \halign 
}$$ % End of \vbox

While the capacitor is charging, electron flow moves clockwise and conventional flow moves counter-clockwise.

%(END_ANSWER)





%(BEGIN_NOTES)

Ask your students to show how they algebraically solved the standard time constant equation for $t$ using logarithms.

%INDEX% Time constant calculation, RC circuit (calculating time required to charge to specified amount)

%(END_NOTES)


