
%(BEGIN_QUESTION)
% Copyright 2004, Tony R. Kuphaldt, released under the Creative Commons Attribution License (v 1.0)
% This means you may do almost anything with this work of mine, so long as you give me proper credit

Compared to common-collector and common-emitter amplifiers, {\it common-base} circuits have few practical applications.  Explain why.

\underbar{file 02234}
%(END_QUESTION)





%(BEGIN_ANSWER)

Common-base amplifier circuits are typified by sub-unity current gains and very low input impedances.

%(END_ANSWER)





%(BEGIN_NOTES)

Perhaps the most frequent application of the common-base amplifier topology is the so-called {\it cascode} amplifier circuit, where a common-emitter stage acts as a "front-end" buffer to the common-base stage to provide reasonable input impedance and current gain.  The grounded-base configuration of the final output stage virtually eliminates the undesirable effects of Miller (collector-to-base) capacitance, resulting in an amplifier capable of high-frequency operation with little or no neutralization required.

%INDEX% Common-base amplifier, characteristics of

%(END_NOTES)


