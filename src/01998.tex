
%(BEGIN_QUESTION)
% Copyright 2003, Tony R. Kuphaldt, released under the Creative Commons Attribution License (v 1.0)
% This means you may do almost anything with this work of mine, so long as you give me proper credit

$$\epsfbox{01998x01.eps}$$

\underbar{file 01998}
\vfil \eject
%(END_QUESTION)





%(BEGIN_ANSWER)

Use circuit simulation software to verify your predicted and measured parameter values.

%(END_ANSWER)





%(BEGIN_NOTES)

Use a power transistor for this circuit, as general-purpose signal transistors may not have sufficient power dissipation ratings to survive the loading students may put them through!  I recommend a small DC motor as a load.  An electric motor offers an easy way to increase electrical loading by placing a mechanical load on the shaft.  By doing this, students can see for themselves how well the circuit maintains load voltage (resisting voltage "sag" under increasing load current).

I have found that this circuit is excellent for getting students to understand how negative feedback really works.  Here, the opamp adjusts the power transistor's base voltage to {\it whatever it needs to be} in order to maintain the load voltage at the same level as the reference set by the zener diode.  Any sort of loss incurred by the transistor (most notably $V_{BE}$) is automatically compensated for by the opamp.

An extension of this exercise is to incorporate troubleshooting questions.  Whether using this exercise as a performance assessment or simply as a concept-building lab, you might want to follow up your students' results by asking them to predict the consequences of certain circuit faults.

%INDEX% Assessment, performance-based (Linear voltage regulator circuit)

%(END_NOTES)


