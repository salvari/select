
%(BEGIN_QUESTION)
% Copyright 2003, Tony R. Kuphaldt, released under the Creative Commons Attribution License (v 1.0)
% This means you may do almost anything with this work of mine, so long as you give me proper credit

One method of achieving reduced-voltage starting for large electric motors is to insert series resistances into each of the motor's power conductors.  When starting, all power must go through the resistors.  After the motor has had time to speed up, another set of "starter" contacts bypass line power around the resistors, directly to the motor windings.

Draw a diagram showing how this could be done for a single-phase electric motor, using two starter contacts: "R" for "run" and "S" for "start".  Hint: you only need two contacts and one resistor!

\underbar{file 00842}
%(END_QUESTION)





%(BEGIN_ANSWER)

None of the control circuitry (start switch, overload contact, starter coil, etc.) is shown in this diagram:

$$\epsfbox{00842x01.eps}$$

%(END_ANSWER)





%(BEGIN_NOTES)

If students have studied the autotransformer method of reduced-voltage starting, ask them to compare this method against that.  Certainly, the resistive method is simpler, but does the autotransformer method have its own advantage(s)?

%(END_NOTES)


