
%(BEGIN_QUESTION)
% Copyright 2005, Tony R. Kuphaldt, released under the Creative Commons Attribution License (v 1.0)
% This means you may do almost anything with this work of mine, so long as you give me proper credit

$$\epsfbox{02466x01.eps}$$

\underbar{file 02466}
\vfil \eject
%(END_QUESTION)





%(BEGIN_ANSWER)

Use circuit simulation software to verify your predicted and measured parameter values.

%(END_ANSWER)





%(BEGIN_NOTES)

Use a dual-voltage, regulated power supply to supply power to the opamp.  Specify all four resistors as equal value, between 1 k$\Omega$ and 100 k$\Omega$ (1k5, 2k2, 2k7, 3k3, 4k7, 5k1, 6k8, 10k, 22k, 33k, 39k 47k, 68k, etc.).  This will ensure a differential voltage gain of unity.  If you {\it want} to have a different voltage gain, then by all means specify these resistor values however you see fit!

Differential gain is calculated by averaging the quotients of each measured $V_{out}$ value with its respective $V_{in(+)} - V_{in(-)}$ differential input voltage.  Common-mode gain is calculated by dividing the difference in output voltages ($\Delta V_{out}$) by the difference in common-mode input voltages ($\Delta V_{in}$).

An extension of this exercise is to incorporate troubleshooting questions.  Whether using this exercise as a performance assessment or simply as a concept-building lab, you might want to follow up your students' results by asking them to predict the consequences of certain circuit faults.

%INDEX% Assessment, performance-based (Opamp difference amplifier)

%(END_NOTES)


