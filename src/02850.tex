
%(BEGIN_QUESTION)
% Copyright 2005, Tony R. Kuphaldt, released under the Creative Commons Attribution License (v 1.0)
% This means you may do almost anything with this work of mine, so long as you give me proper credit

An arithmetic trick often used when working with the metric system is multiplication-by-ten and division-by-ten via shifting of the decimal point.  A similar "trick" may be applied to binary numbers, with similar results.

Determine what sort of multiplication or division is accomplished when the "binary point" is shifted in a binary number.  Research the datasheet of an arithmetic logic unit (ALU) circuit to see if and how this function is implemented.

\underbar{file 02850}
%(END_QUESTION)





%(BEGIN_ANSWER)

Shifting the "binary point" results in either multiplication or division by two.  A multiplicative shift is performed by the 74AS181 ALU by arithmetic function selection $1100_2$ (C$_{16}$).

\vskip 10pt

Challenge question: explain how multiplication or division by {\it any} binary quantity may be accomplished using successive bit-shifts and additions.  For example, show what steps you could take to multiply any binary number by five ($101_2$), using only "binary point" shifting and addition(s).

%(END_ANSWER)





%(BEGIN_NOTES)

A lot of arithmetic tricks existing in the decimal numeration system are applicable, with slight revision, in the binary numeration system as well.  This is a popular one, and often used by shrewd computer programmers to execute fast multiply-by-two or divide-by-two operations when "conventional" multiplication commands take more time.

%INDEX% Arithmetic Logic Unit (ALU), model 74AS181
%INDEX% Binary multiplication via bitwise shift
%INDEX% Multiplication (binary) via bitwise shift

%(END_NOTES)


