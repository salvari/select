
%(BEGIN_QUESTION)
% Copyright 2005, Tony R. Kuphaldt, released under the Creative Commons Attribution License (v 1.0)
% This means you may do almost anything with this work of mine, so long as you give me proper credit

An interesting and useful property in mathematics is the {\it transitive property}:

$$\hbox{If } a = b \hbox{ and } b = c \hbox{ , then } a = c$$

Simply stated, two variables must be equal to one another if they are both equal to a common (third) variable.  While not particularly profound or breathtaking in scope, this property is nevertheless useful in solving certain mathematical problems.

\vskip 10pt

Suppose you were given the following two equations and asked to find solutions for $x$ and $y$ that will satisfy {\it both} at the same time:

$$y + x = 8$$

$$y - x = 3$$

Manipulate both of these equations to solve for $y$, and then explain how you could apply the transitive principle to solve for $x$.

\underbar{file 03192}
%(END_QUESTION)





%(BEGIN_ANSWER)

If $8 - x = y$ and $3 + x = y$, then $8 - x$ must equal $3 + x$:

$$8 - x = 3 + x$$

Solutions for $x$ and $y$:

$$x = 2.5 \hbox{ and } y = 5.5$$

%(END_ANSWER)





%(BEGIN_NOTES)

This method of solving for a two-variable set of simultaneous equations is really nothing more than substitution in disguise.  Some students find it easier to grasp than straight substitution, though.

%INDEX% Algebra, substitution
%INDEX% Simultaneous equations
%INDEX% Systems of linear equations
%INDEX% Transitive principle, algebra

%(END_NOTES)


