
%(BEGIN_QUESTION)
% Copyright 2003, Tony R. Kuphaldt, released under the Creative Commons Attribution License (v 1.0)
% This means you may do almost anything with this work of mine, so long as you give me proper credit

The cross-sectional dimensions of a copper "busbar" measure 8 cm by 2.5 cm.  How much resistance would this busbar have, measured end-to-end, if its length is 10 meters?  Assume a temperature of 20$^{o}$ Celsius.

$$\epsfbox{00486x01.eps}$$

\underbar{file 00486}
%(END_QUESTION)





%(BEGIN_ANSWER)

83.9 $\mu \Omega$

%(END_ANSWER)





%(BEGIN_NOTES)

This question is a good review of the metric system, relating centimeters to meters, and such.  It may also be a good review of unit conversions, if students choose to do their resistance calculations using English units (cmils, or square inches) rather than metric.

Students may be surprised at the low resistance figure, but remind them that they are dealing with a solid bar of copper, over 3 square inches in cross-sectional area.  {\it This is one big conductor!}

%INDEX% Conductor resistance calculation

%(END_NOTES)


