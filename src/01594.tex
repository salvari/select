
%(BEGIN_QUESTION)
% Copyright 2003, Tony R. Kuphaldt, released under the Creative Commons Attribution License (v 1.0)
% This means you may do almost anything with this work of mine, so long as you give me proper credit

The likelihood that a given component will fail in the "open" mode is quite often not the same as the likelihood that it will fail "shorted."  Based on the research you do and your own personal experience with troubleshooting electronic circuits, determine whether the following components are more likely to fail {\it open} or fail {\it shorted} (this includes partial, or high-resistance, shorts):

\medskip
\item{$\bullet$} Resistors:
\item{$\bullet$} Capacitors:
\item{$\bullet$} Inductors:
\item{$\bullet$} Transformers:
\item{$\bullet$} Bipolar transistors:
\medskip

I encourage you to research information on these devices' failure modes, as well as glean from your own experiences building and troubleshooting electronic circuits.

\underbar{file 01594}
%(END_QUESTION)





%(BEGIN_ANSWER)

Remember that each of these answers merely represents the {\it most likely} of the two failure modes, either open or shorted, and that probabilities may shift with operating conditions (i.e. switches may be more prone to failing shorted due to welded contacts if they are routinely abused with excessive current upon closure).

\medskip
\item{$\bullet$} Resistors: {\bf open}
\item{$\bullet$} Capacitors: {\bf shorted}
\item{$\bullet$} Inductors: {\bf open or short equally probable}
\item{$\bullet$} Transformers: {\bf open or short equally probable}
\item{$\bullet$} Bipolar transistors: {\bf shorted}
\medskip

Follow-up question: When bipolar transistors fail shorted, the short is usually apparent between the collector and emitter terminals (although sometimes all three terminals may register shorted, as though the transistor were nothing more than a junction between three wires).  Why do you suppose this is?  What is it about the base terminal that makes it less likely to "fuse" with the other terminals?

%(END_ANSWER)





%(BEGIN_NOTES)

Emphasize to your students how a good understanding of common failure modes is important to efficient troubleshooting technique.  Knowing which way a particular component is more likely to fail under normal operating conditions enables the troubleshooter to make better judgments when assessing the most probable cause of a system failure.  

Of course, proper troubleshooting technique should always reveal the source of trouble, whether or not the troubleshooter has any experience with the failure modes of particular devices.  However, possessing a detailed knowledge of failure probabilities allows one to check the most likely sources of trouble first, which generally leads to faster repairs.

An organization known as the {\it Reliability Analysis Center}, or {\it RAC}, publishes detailed analyses of failure modes for a wide variety of components, electronic as well as non-electronic.  They may be contacted at 201 Mill Street, Rome, New York, 13440-6916.  Data for this question was gleaned from the RAC's publication, {\it Part Failure Mode Distributions}.

%INDEX% Resistor, probable failure mode
%INDEX% Capacitor, probable failure mode
%INDEX% Inductor, probable failure mode
%INDEX% Transformer, probable failure mode
%INDEX% Transistor, probable failure mode

%(END_NOTES)


