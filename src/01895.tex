
%(BEGIN_QUESTION)
% Copyright 2003, Tony R. Kuphaldt, released under the Creative Commons Attribution License (v 1.0)
% This means you may do almost anything with this work of mine, so long as you give me proper credit

Describe what happens to the capacitor voltage in this circuit over time as it is charged by the constant-current source:

$$\epsfbox{01895x01.eps}$$

Now, determine the ideal values for $V$ and $R$ that will result in similar behavior in a capacitor circuit powered by a {\it voltage source} rather than a current source:

$$\epsfbox{01895x02.eps}$$

Your answers will, of course, be qualitative rather than quantitative.  Explain whether or not the time constant for the voltage-source-powered circuit ought to be large or small, and why.

\underbar{file 01895}
%(END_QUESTION)





%(BEGIN_ANSWER)

Both $V$ and $R$ should be extremely large values in order to mimic the behavior of a current source.

%(END_ANSWER)





%(BEGIN_NOTES)

In this question, I ask students to identify the behavior of a true integrator circuit, and then contrast it with the behavior of what is more accurately known as a first-order lag circuit (the RC circuit powered by the voltage source).  Of course, the two circuits do not behave the same, but through judicious choices of $V$ and $C$, one can make the "lag" circuit closely mimic the true integrator circuit over a practical range of capacitor voltage.

%INDEX% Passive integrator circuit, mimicking ideal integrator circuit

%(END_NOTES)


