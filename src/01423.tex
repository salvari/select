
%(BEGIN_QUESTION)
% Copyright 2003, Tony R. Kuphaldt, released under the Creative Commons Attribution License (v 1.0)
% This means you may do almost anything with this work of mine, so long as you give me proper credit

Find one or two real integrated circuits and bring them with you to class discussion.  Identify as much information as you can about your ICs prior to discussion:

\medskip
\item{$\bullet$} Manufacturer
\item{$\bullet$} Part number
\item{$\bullet$} Function
\item{$\bullet$} Datasheet
\item{$\bullet$} Date of manufacture
\medskip

\underbar{file 01423}
%(END_QUESTION)





%(BEGIN_ANSWER)

You will find a wealth of information on manufacturers' websites, on the internet!

%(END_ANSWER)





%(BEGIN_NOTES)

The purpose of this question is to get students to kinesthetically interact with the subject matter.  It may seem silly to have students engage in a "show and tell" exercise, but I have found that activities such as this greatly help some students.  For those learners who are kinesthetic in nature, it is a great help to actually {\it touch} real components while they're learning about their function.  Of course, this question also provides an excellent opportunity for them to practice interpreting component markings, use a multimeter, access datasheets, etc.

%INDEX% Integrated circuit identification
%INDEX% IC "chip" identification

%(END_NOTES)


