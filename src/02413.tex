
%(BEGIN_QUESTION)
% Copyright 2005, Tony R. Kuphaldt, released under the Creative Commons Attribution License (v 1.0)
% This means you may do almost anything with this work of mine, so long as you give me proper credit

Identify each type of JFET (whether it is N-channel or P-channel), label the terminals, and determine whether the JFET in each of these circuits will be turned {\it on} or {\it off}:

$$\epsfbox{02413x01.eps}$$

Additionally, identify which of these four circuits places unnecessary stress on the transistor.  There is one circuit among these four where the transistor is operated in a state that might lead to premature failure.

\underbar{file 02413}
%(END_QUESTION)





%(BEGIN_ANSWER)

$$\epsfbox{02413x02.eps}$$

The upper-right circuit places unnecessary stress on the JFET.

%(END_ANSWER)





%(BEGIN_NOTES)

It is very important for your students to understand what factor(s) in a circuit force a JFET to turn on or off.  Be sure to ask your students to explain their reasoning for each transistor's status.  What factor, or combination of factors, is necessary to turn a JFET on, versus off?  One point of this question is to emphasize the non-importance of $V_{DD}$'s polarity when there is an external biasing voltage applied directly between gate and source.

Discuss with your students precisely what is wrong with the upper-right JFET circuit.  Why is the transistor being stressed?  How do we avoid such a problem?

%INDEX% JFET, determining conduction status of

%(END_NOTES)


