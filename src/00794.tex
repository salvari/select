
%(BEGIN_QUESTION)
% Copyright 2003, Tony R. Kuphaldt, released under the Creative Commons Attribution License (v 1.0)
% This means you may do almost anything with this work of mine, so long as you give me proper credit

Suppose you suspected a failed-open diode in this power supply circuit.  Describe how you could detect its presence without using an oscilloscope:

$$\epsfbox{00794x01.eps}$$

Incidentally, the "low voltage AC power supply" is nothing more than a step-down transformer with a center-tapped secondary winding.

\underbar{file 00794}
%(END_QUESTION)





%(BEGIN_ANSWER)

"Remove all diodes from the circuit and test them individually" is not an acceptable answer to this question.  Think of a way that they could be checked while in-circuit (ideally, without having to shut off power to the circuit).

%(END_ANSWER)





%(BEGIN_NOTES)

A common tendency for students is to troubleshoot using the "shotgun approach," which is to remove each component one-by-one and test it.  This is a very time-intensive and inefficient method of troubleshooting.  Instead, students need to develop diagnostic procedures not requiring removal of components from the circuit.  At the very least, there should be some way we can narrow the range of possibilities using in-circuit tests prior to removing components.

%INDEX% Rectifier circuit, diode failure in

%(END_NOTES)


