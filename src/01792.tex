
%(BEGIN_QUESTION)
% Copyright 2003, Tony R. Kuphaldt, released under the Creative Commons Attribution License (v 1.0)
% This means you may do almost anything with this work of mine, so long as you give me proper credit

Calculate the necessary resistance values to give this multi-range voltmeter the ranges indicated by the selector switch positions:

$$\epsfbox{01792x01.eps}$$

\underbar{file 01792}
%(END_QUESTION)





%(BEGIN_ANSWER)

\item{$\bullet$} $R_1 = 99 \hbox{ k} \Omega$
\item{$\bullet$} $R_2 = 300 \hbox{ k} \Omega$
\item{$\bullet$} $R_3 = 600 \hbox{ k} \Omega$
\item{$\bullet$} $R_4 = 1 \hbox{ M} \Omega$
\item{$\bullet$} $R_5 = 3 \hbox{ M} \Omega$

\vskip 10pt

Hint: if you need help getting started in this problem, begin with calculating the value of $R_1$.

%(END_ANSWER)





%(BEGIN_NOTES)

This is really nothing more than a set of simple series circuit problems, although the context of it being a voltmeter seems to confuse some students.  If you find a large percentage of your class not understanding where to begin in a problem such as this, it means they really don't understand series circuits -- all they learned to do when studying series resistor circuits before is to follow an easy sequence of steps to find voltages and currents in series resistor circuits.  They did not learn the concepts well enough to abstract to something that looks just a little bit different.

You should point out to your students how the series arrangement of the range resistors lends itself to more common resistance values, as opposed to having a separate range resistor for each range.  There is a downside to this design, however: reliability.  Discuss with your students the consequences of "open" resistor faults in both types of voltmeter designs.

%INDEX% Voltmeter, range resistor sizing for multiple ranges

%(END_NOTES)


