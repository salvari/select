
%(BEGIN_QUESTION)
% Copyright 2003, Tony R. Kuphaldt, released under the Creative Commons Attribution License (v 1.0)
% This means you may do almost anything with this work of mine, so long as you give me proper credit

There are two well-known formulae for calculating the total resistance of parallel-connected resistances.  One of these works only for two resistances, while the other works for any number of parallel resistances.  Write these two formulae, and give examples of their use.

\underbar{file 01733}
%(END_QUESTION)





%(BEGIN_ANSWER)

$$R_{parallel} = {{R_1 R_2} \over {R_1 + R_2}}$$

$$R_{parallel} = {1 \over { {1 \over R_1} + {1 \over R_2} + \cdots {1 \over R_n} } }$$

%(END_ANSWER)





%(BEGIN_NOTES)

Although I typically use the lower formula exclusively in my teaching, the upper formula is often useful for situations where a calculator is not handy, and you must estimate parallel resistance.

%INDEX% Parallel resistance, calculating

%(END_NOTES)


