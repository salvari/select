
%(BEGIN_QUESTION)
% Copyright 2005, Tony R. Kuphaldt, released under the Creative Commons Attribution License (v 1.0)
% This means you may do almost anything with this work of mine, so long as you give me proper credit

Calculate the approximate amount of current this current mirror circuit will try to maintain through $R_{load}$, assuming silicon transistors (0.7 volts forward base-emitter junction drop):

$$\epsfbox{02434x01.eps}$$

Also, calculate the approximate power dissipation of both transistors.

\underbar{file 02434}
%(END_QUESTION)





%(BEGIN_ANSWER)

$I_{load} \approx$ 8.63 mA \hskip 30pt $P_{Q1} \approx$ 6.041 mW \hskip 30pt $P_{Q2} \approx $ 95.41 mW

\vskip 10pt

Follow-up question: what do the two power dissipation figures tell us about the relative power of two transistors handling the exact same currents?  Explain why this is important in a current mirror circuit, and why it is customary to thermally bond $Q_1$ and $Q_2$ together in discrete-component current mirrors.

%(END_ANSWER)





%(BEGIN_NOTES)

Ask your students to explain how they obtained the answer to this question, step by step.

The follow-up question is an important one for a few reasons.  First, students must be aware that transistor power dissipation is determined by more than just collector current.  Secondly, the disparate dissipations of these two transistors will lead to inaccuracies in regulated current in a current mirror circuit if measures are not taken to equalize their temperatures.

%INDEX% Current mirror circuit, BJT

%(END_NOTES)


