
%(BEGIN_QUESTION)
% Copyright 2005, Tony R. Kuphaldt, released under the Creative Commons Attribution License (v 1.0)
% This means you may do almost anything with this work of mine, so long as you give me proper credit

Trace the paths of {\it injection}, {\it diffusion}, and {\it collection} currents in this energy diagram for a PNP transistor as it is conducting:

$$\epsfbox{02481x01.eps}$$

\underbar{file 02481}
%(END_QUESTION)





%(BEGIN_ANSWER)

$$\epsfbox{02481x02.eps}$$

%(END_ANSWER)





%(BEGIN_NOTES)

A picture is worth a thousand words, they say.  For me, this illustration is the one that finally made transistors make sense to me.  By forward-biasing the emitter-base junction, minority carriers are injected into the base (holes in the "N" type material, in the case of a PNP transistor), which then rise easily into the collector region.  This energy diagram is invaluable for explaining why collector current can flow even when the base-collector junction is reverse biased.

When looking at energy diagrams, it is helpful to think of natural hole motion as air bubbles in a liquid, trying to rise as high as possible within their designated band.

%INDEX% Collection current, BJT
%INDEX% Diffusion current, BJT
%INDEX% Energy diagram, conducting BJT
%INDEX% Injection current, BJT

%(END_NOTES)


