
%(BEGIN_QUESTION)
% Copyright 2003, Tony R. Kuphaldt, released under the Creative Commons Attribution License (v 1.0)
% This means you may do almost anything with this work of mine, so long as you give me proper credit

Write the transfer function (input/output equation) for an operational amplifier with an open-loop voltage gain of 100,000, and the inverting input connected to a voltage divider on its output terminal (so the inverting input receives exactly one-half the output voltage).  In other words, write an equation describing the output voltage of this op-amp ($V_{out}$) for any given input voltage at the noninverting input ($V_{in(+)}$):

$$\epsfbox{00928x01.eps}$$

Then, once you have an equation written, solve for the output voltage if the noninverting input voltage is -2.4 volts.

\underbar{file 00928}
%(END_QUESTION)





%(BEGIN_ANSWER)

$V_{out} = 100,000(V_{in(+)} - {1 \over 2}V_{out})$

\vskip 10pt

(I've left it up to you to perform the algebraic simplification here!)

\vskip 20pt

For an input voltage of -2.4 volts, the output voltage will be -4.7999 volts.

\vskip 10pt

Follow-up question: what do you notice about the output voltage in this circuit?  What value is it very close to being, in relation to the input voltage?  Does this pattern hold true for other input voltages as well?

%(END_ANSWER)





%(BEGIN_NOTES)

Your students should see a definite pattern here as they calculate the output voltage for several different input voltage levels.  Discuss this phenomenon with your students, asking them to explain it as best they can.

%INDEX% Opamp, voltage amplifier circuit (transfer function and precise output voltage calculation)

%(END_NOTES)


