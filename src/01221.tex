
%(BEGIN_QUESTION)
% Copyright 2003, Tony R. Kuphaldt, released under the Creative Commons Attribution License (v 1.0)
% This means you may do almost anything with this work of mine, so long as you give me proper credit

Counting practice: count from zero to thirty-one in binary, octal, and hexadecimal:

$$\epsfbox{01221x01.eps}$$

\underbar{file 01221}
%(END_QUESTION)





%(BEGIN_ANSWER)

No answers given here -- compare with your classmates!

%(END_ANSWER)





%(BEGIN_NOTES)

In order to familiarize students with these "strange" numeration systems, I like to begin each day of digital circuit instruction with counting practice.  Students need to be {\it fluent} in these numeration systems by the time they are finished studying digital circuits!

One suggestion I give to students to help them see patterns in the count sequences is "pad" the numbers with leading zeroes so that all numbers have the same number of characters.  For example, instead of writing "10" for the binary number two, write "00010".  This way, the patterns of character cycling (especially binary, where each successively higher-valued bit has half the frequency of the one before it) become more evident to see.

%INDEX% Binary counting practice
%INDEX% Counting practice, binary

%(END_NOTES)


