
%(BEGIN_QUESTION)
% Copyright 2006, Tony R. Kuphaldt, released under the Creative Commons Attribution License (v 1.0)
% This means you may do almost anything with this work of mine, so long as you give me proper credit

The output of the following gate circuit is always high, no matter what states the input switches are in.  Assume that CMOS logic gates are being used here:

$$\epsfbox{02894x01.eps}$$

Identify which of these possibilities could account for the output always being high:

\medskip
\goodbreak
\item{$\bullet$} Output of $U_1$ stuck in a high state
\item{$\bullet$} Output of $U_2$ stuck in a high state
\item{$\bullet$} $R_1$ failed open
\item{$\bullet$} $R_2$ failed shorted
\item{$\bullet$} $R_3$ failed shorted
\item{$\bullet$} Switch A failed open
\item{$\bullet$} Switch B failed shorted
\item{$\bullet$} Switch C failed shorted
\medskip

\underbar{file 02894}
%(END_QUESTION)





%(BEGIN_ANSWER)

Only these three possibilities could account for the output always being high:

\medskip
\goodbreak
\item{$\bullet$} $R_1$ failed open
\item{$\bullet$} $R_3$ failed shorted
\item{$\bullet$} Switch B failed shorted
\medskip

%(END_ANSWER)





%(BEGIN_NOTES)

Questions like this help students hone their troubleshooting skills by forcing them to think through the consequences of each possibility.  This is an essential step in troubleshooting, and it requires a firm understanding of circuit function.

%INDEX% Troubleshooting, logic gate circuit

%(END_NOTES)


