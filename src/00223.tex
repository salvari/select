
%(BEGIN_QUESTION)
% Copyright 2003, Tony R. Kuphaldt, released under the Creative Commons Attribution License (v 1.0)
% This means you may do almost anything with this work of mine, so long as you give me proper credit

In the late 1700's, an Italian professor of anatomy, Luigi Galvani, discovered that the leg muscles of a recently deceased frog could be made to twitch when subjected to an electric current.  What phenomenon is suggested by Galvani's discovery?  In other words, what does this tell us about the operation of muscle fibers in living creatures?  More importantly, what practical importance does this have for people working near electric circuits?

\underbar{file 00223}
%(END_QUESTION)





%(BEGIN_ANSWER)

Essentially, muscle fibers are "activated" by electrical signals.  I'll let you figure out what practical importance this effect has for you!

%(END_ANSWER)





%(BEGIN_NOTES)

This question presents an excellent opportunity to discuss one of the important aspects of electrical safety: involuntary muscle contraction.

%INDEX% Electricity, effect on muscle tissue
%INDEX% Safety, electrical

%(END_NOTES)


