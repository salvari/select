
%(BEGIN_QUESTION)
% Copyright 2005, Tony R. Kuphaldt, released under the Creative Commons Attribution License (v 1.0)
% This means you may do almost anything with this work of mine, so long as you give me proper credit

Predict how the operation of this AC lamp dimmer circuit will be affected as a result of the following faults.  Consider each fault independently (i.e. one at a time, no multiple faults):

$$\epsfbox{03731x01.eps}$$

\medskip
\item{$\bullet$} Potentiometer $R_{pot}$ fails open:
\vskip 5pt
\item{$\bullet$} Capacitor $C_1$ fails shorted:
\vskip 5pt
\item{$\bullet$} Capacitor $C_1$ fails open:
\vskip 5pt
\item{$\bullet$} DIAC fails open:
\vskip 5pt
\item{$\bullet$} TRIAC fails shorted:
\medskip

For each of these conditions, explain {\it why} the resulting effects will occur.

\underbar{file 03731}
%(END_QUESTION)





%(BEGIN_ANSWER)

\medskip
\item{$\bullet$} Potentiometer $R_{pot}$ fails open: {\it Lamp remains off.}
\vskip 5pt
\item{$\bullet$} Capacitor $C_1$ fails shorted: {\it Lamp remains off.}
\vskip 5pt
\item{$\bullet$} Capacitor $C_1$ fails open: {\it Range of lamp brightness control extends from 100\% to 50\%, and any attempt to make it dimmer results in the lamp just turning all the way off.}
\vskip 5pt
\item{$\bullet$} DIAC fails open: {\it Lamp remains off.}
\vskip 5pt
\item{$\bullet$} TRIAC fails shorted: {\it Lamp remains on at 100\% brightness.}
\medskip

%(END_ANSWER)





%(BEGIN_NOTES)

The purpose of this question is to approach the domain of circuit troubleshooting from a perspective of knowing what the fault is, rather than only knowing what the symptoms are.  Although this is not necessarily a realistic perspective, it helps students build the foundational knowledge necessary to diagnose a faulted circuit from empirical data.  Questions such as this should be followed (eventually) by other questions asking students to identify likely faults based on measurements.

%INDEX% Troubleshooting, predicting effects of fault in TRIAC-based lamp dimmer circuit

%(END_NOTES)


