
%(BEGIN_QUESTION)
% Copyright 2005, Tony R. Kuphaldt, released under the Creative Commons Attribution License (v 1.0)
% This means you may do almost anything with this work of mine, so long as you give me proper credit

{\it Protective relays} are special power-sensing devices whose job it is to automatically open or close circuit breakers in large electric power systems.  Some protective relays are designed to be used directly with large electric motors to provide sophisticated monitoring, shut-down, and start-up control.

One of the features of these motor-oriented protective relays is {\it start-up lockout}.  What this means is the relay will prevent someone from attempting too many successive re-starts of a large electric motor.  If the motor is started and stopped several times over a short period of time, the relay will prevent the person from starting it again until a sufficient "rest" time has passed.

Explain why a large electric motor would need to "rest" after several successive start-up events.  If electric motors are perfectly capable of running continuously at full load for years on end, why would a few start-ups be worthy of automatic lock-out?

\underbar{file 03131}
%(END_QUESTION)





%(BEGIN_ANSWER)

I won't give you a direct answer here, but I will provide a big hint: {\it inrush current}.

%(END_ANSWER)





%(BEGIN_NOTES)

Inrush current is a factor with {\it every} motor type, AC or DC.  It is easy to forget just how substantially larger a typical motor's inrush current is compared to its normal full-load current.  When students consider the magnitude of the currents involved, and also the fact that most electric motors are fan-cooled and therefore lacking in cooling during the initial moments of a start-up, the reason for automatic lock-out after several successive start-up events becomes obvious.

%INDEX% Inrush current, electric motor
%INDEX% Protective relay, for induction motor start control

%(END_NOTES)


