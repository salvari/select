
%(BEGIN_QUESTION)
% Copyright 2003, Tony R. Kuphaldt, released under the Creative Commons Attribution License (v 1.0)
% This means you may do almost anything with this work of mine, so long as you give me proper credit

We know that connecting a sensitive meter movement directly in series with a high-current circuit is a Bad Thing.  So, I want you to determine what other component(s) must be connected to the meter movement to limit the current through its coil, so that connecting the circuit in series with a 1-amp circuit results in the meter's needle moving exactly to the full-scale position.

$$\epsfbox{00727x01.eps}$$

In your diagram, show both the extra component(s) and the manner in which the meter assembly will be connected to the battery/resistor circuit to measure current.

\underbar{file 00727}
%(END_QUESTION)





%(BEGIN_ANSWER)

$$\epsfbox{00727x02.eps}$$

\vskip 10pt

Follow-up question: given the 0 to 1 amp range of the ammeter created by the 0.4004 $\Omega$ "shunt" resistor, how much current will the meter actually register when connected in series with the 6 volt battery and 6 ohm resistor?  Is this the real current, or is the meter giving us a false indication?

%(END_ANSWER)





%(BEGIN_NOTES)

Beginning students sometimes feel "lost" when trying to answer a question like this.  They may know how to apply Ohm's Law to a circuit, but they do not know how to design a circuit that makes use of Ohm's Law for a specific purpose.  If this is the case, you may direct their understanding through a series of questions such as this:

\medskip
\item{$\bullet$} Why does the meter movement "peg" if directly connected to the battery?
\item{$\bullet$} What type of electrical component could be used to direct current "away" from the movement, without limiting the measured current?
\item{$\bullet$} How might we connect this component to the meter (series or parallel)?  (Draw both configurations and let the student determine for themselves which connection pattern fulfills the goal of limiting current to the meter.)
\medskip

The follow-up question is quite interesting, and causes students to carefully evaluate the performance of the ammeter they've "created".  At root, the problem is similar to that of voltmeter loading, except of course that we're dealing with ammeters here rather than voltmeters.

%(END_NOTES)


