
%(BEGIN_QUESTION)
% Copyright 2003, Tony R. Kuphaldt, released under the Creative Commons Attribution License (v 1.0)
% This means you may do almost anything with this work of mine, so long as you give me proper credit

A student sets up a circuit that looks like this, to gather data for characterizing a diode:

$$\epsfbox{01923x01.eps}$$

Measuring diode voltage and diode current in this circuit, the student generates the following table of data:

% No blank lines allowed between lines of an \halign structure!
% I use comments (%) instead, so that TeX doesn't choke.

$$\vbox{\offinterlineskip
\halign{\strut
\vrule \quad\hfil # \ \hfil & 
\vrule \quad\hfil # \ \hfil \vrule \cr
\noalign{\hrule}
%
% First row
$V_{diode}$ & $I_{diode}$ \cr
%
\noalign{\hrule}
%
% Second row
0.600 V & 1.68 mA \cr
%
\noalign{\hrule}
%
% Third row
0.625 V & 2.88 mA \cr
%
\noalign{\hrule}
%
% Fourth row
0.650 V & 5.00 mA \cr
%
\noalign{\hrule}
%
% Fifth row
0.675 V & 8.68 mA \cr
%
\noalign{\hrule}
%
% Sixth row
0.700 V & 14.75 mA \cr
%
\noalign{\hrule}
%
% Seventh row
0.725 V & 27.25 mA \cr
%
\noalign{\hrule}
%
% Eighth row
0.750 V & 48.2 mA \cr
%
\noalign{\hrule}
} % End of \halign 
}$$ % End of \vbox

This student knows that the behavior of a PN junction follows Shockley's diode equation, and that the equation may be simplified to the following form:

$$I_{diode} = I_S (e^{V_{diode} \over K} - 1)$$

\noindent
Where,

$K = $ a constant incorporating both the thermal voltage and the nonideality coefficient

\vskip 10pt

The goal of this experiment is to calculate $K$ and $I_S$, so that the diode's current may be predicted for any arbitrary value of voltage drop.  However, the equation must be simplified a bit before the student can proceed.

At substantial levels of current, the exponential term is very much larger than unity ($e^{V_{diode} \over K} >> 1$), so the equation may be simplified as such:

$$I_{diode} \approx I_S (e^{V_{diode} \over K})$$

From this equation, determine how the student would calculate $K$ and $I_S$ from the data shown in the table.  Also, explain how this student may verify the accuracy of these calculated values.

\underbar{file 01923}
%(END_QUESTION)





%(BEGIN_ANSWER)

$K \approx 0.04516$

$I_S \approx 2.869$ nA

\vskip 10pt

Hint: this may be a difficult problem to solve if you are unfamiliar with the algebraic technique of dividing one equation by another.  Here is the technique shown in general terms:

$$\hbox{Given: \hskip 10pt} y_1 = ax_1 \hbox{\hskip 20pt} y_2 = ax_2$$

$${y_1 \over y_2} = {ax_1 \over ax_2}$$

From here, it may be possible to perform simplifications impossible before.  I suggest using this technique to solve for $K$ first.

\vskip 10pt

Follow-up question: explain how this student knew it was "safe" to simplify the Shockley diode equation by eliminating the "- 1" term.  Is this sort of elimination always permissible?  Why or why not?

%(END_ANSWER)





%(BEGIN_NOTES)

The algebraic technique used to solve for $K$ is very useful for certain types of problems.

Discuss the follow-up question with your students.  It is important in the realm of technical mathematics to have a good sense of the relative values of equation terms, so that one may "safely" eliminate terms as a simplifying technique without incurring significant errors.  In the Shockley diode equation it is easy to show that the exponential term is {\it enormous} compared to 1 for the values of $V_{diode}$ shown in the table (assuming a typical value for thermal voltage), and so the "- 1" part is very safe to eliminate. 

Also discuss the idea of verifying the calculated values of $K$ and $I_S$ with your students, to help them cultivate a scientifically critical point of view in their study of electronics.

\vskip 10pt

Incidentally, the data in this table came from a real experiment, set up exactly as shown by the schematic diagram in the question.  Care was taken to avoid diode heating by turning the potentiometer to maximum resistance between readings.

%INDEX% Algebra, manipulating equations
%INDEX% Diode equation
%INDEX% Shockley's diode equation

%(END_NOTES)


