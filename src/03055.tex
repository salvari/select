
%(BEGIN_QUESTION)
% Copyright 2005, Tony R. Kuphaldt, released under the Creative Commons Attribution License (v 1.0)
% This means you may do almost anything with this work of mine, so long as you give me proper credit

Observe the following equivalence:

$${4^3 \over 4^2} = {{4 \times 4 \times 4} \over {4 \times 4}}$$

It should be readily apparent that we may cancel out two quantities from both top and bottom of the fraction, so in the end we are left with this:

$$4 \over 1$$

Re-writing this using exponents, we get $4^1$.

\vskip 10pt

Expand each of these expressions so that there are no exponents either:

\vskip 10pt
\goodbreak
\item{$\bullet$} ${3^5 \over 3^2} = $
\vskip 10pt
\item{$\bullet$} ${10^6 \over 10^4} = $
\vskip 10pt
\item{$\bullet$} ${8^7 \over 8^3} = $
\vskip 10pt
\item{$\bullet$} ${20^5 \over 20^4} = $
\vskip 10pt

After expanding each of these expressions, re-write each one in simplest form: one number to a power, just like the final form of the example given ($4^1$).  From these examples, what pattern do you see with exponents of products.  In other words, what is the general solution to the following expression?

$${a^m \over a^n} = $$

\underbar{file 03055}
%(END_QUESTION)





%(BEGIN_ANSWER)

$${a^m \over a^n} = a^{m-n}$$

%(END_ANSWER)





%(BEGIN_NOTES)

I have found that students who cannot fathom the general rule (${a^m \over a^n} = a^{m-n}$) often understand for the first time when they see concrete examples.

%INDEX% Algebra, exponents
%INDEX% Exponents, algebra

%(END_NOTES)


