
%(BEGIN_QUESTION)
% Copyright 2006, Tony R. Kuphaldt, released under the Creative Commons Attribution License (v 1.0)
% This means you may do almost anything with this work of mine, so long as you give me proper credit

This Johnson counter circuit is special.  It outputs three square-wave signals, shifted 120$^{o}$ from one another:

$$\epsfbox{03907x01.eps}$$

Suppose the middle flip-flop's $Q$ output fails in the "high" state.  Plot the new output waveforms for signals {\bf A}, {\bf B}, and {\bf C}.  Assume all $Q$ outputs begin in the "low" state (except for the middle flip-flop, of course):

$$\epsfbox{03907x02.eps}$$

\underbar{file 03907}
%(END_QUESTION)





%(BEGIN_ANSWER)

$$\epsfbox{03907x03.eps}$$

%(END_ANSWER)





%(BEGIN_NOTES)

The purpose of this question is to approach the domain of circuit troubleshooting from a perspective of knowing what the fault is, rather than only knowing what the symptoms are.  Although this is not necessarily a realistic perspective, it helps students build the foundational knowledge necessary to diagnose a faulted circuit from empirical data.  Questions such as this should be followed (eventually) by other questions asking students to identify likely faults based on measurements.

%INDEX% Troubleshooting, predicting effects of fault in Johnson counter circuit

%(END_NOTES)


