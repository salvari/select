
%(BEGIN_QUESTION)
% Copyright 2003, Tony R. Kuphaldt, released under the Creative Commons Attribution License (v 1.0)
% This means you may do almost anything with this work of mine, so long as you give me proper credit

This {\it transistor} circuit is powered by two different voltage sources, one that outputs 6 volts, and the other that is variable.

$$\epsfbox{01044x01.eps}$$

Transistors naturally act as current-regulating devices, and are often analyzed as though they were current {\it sources}.  Suppose that this transistor happened to be regulating current at a value of 3.5 mA:

$$\epsfbox{01044x02.eps}$$

How high does the voltage of the variable source have to be adjusted, until no current is drawn from the 6-volt battery?

\vskip 10pt

Hint: simultaneous equations are not needed to solve this problem!

\underbar{file 01044}
%(END_QUESTION)





%(BEGIN_ANSWER)

$E =$ 9.5 V

%(END_ANSWER)





%(BEGIN_NOTES)

The purpose of this question is to get students to apply what they know of basic circuit "laws" (Ohm's Law, KVL, KCL) to the solution of a single voltage value.  As usual, the {\it method} of solution is far more valuable than the answer.

If some students are completely confused regarding how to solve for this voltage, suggest that they "plug" the given answer into the circuit and determine currents and voltage drops.  What do they notice when they do this?  What unusual condition(s) stand out with the variable source at 9.5 volts?  Are any of these conditions things they could have (or should have) known prior to knowing the variable source's voltage, given the condition of ". . . no current [drawn] from the 6-volt battery"?

%(END_NOTES)


