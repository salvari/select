
%(BEGIN_QUESTION)
% Copyright 2005, Tony R. Kuphaldt, released under the Creative Commons Attribution License (v 1.0)
% This means you may do almost anything with this work of mine, so long as you give me proper credit

The following equation is a general model for Hall effect devices, describing the voltage produced by a Hall element given the current through it ($I$), the magnetic flux density ($B$) passing through perpendicular to the current, and the thickness ($x$) of the Hall element.  The variable $K$ is a constant accounting for variations in material composition and temperature, and for any combination of units represented by the other variables:

$$V_{Hall} = K {IB \over x}$$

$$\epsfbox{03672x01.eps}$$

Explain what this equation means with regard to the effect of each variable ($I$, $B$, and $x$) on the Hall voltage generated.  Identify whether each of the variables has a {\it direct} or an {\it inverse} effect on the output voltage.

\underbar{file 03672}
%(END_QUESTION)





%(BEGIN_ANSWER)

$I$ = direct effect

$B$ = direct effect

$x$ = inverse effect

%(END_ANSWER)





%(BEGIN_NOTES)

Discuss with your students what it means for a variable to have a {\it direct} or {\it inverse} effect on another, and ask them to provide examples other than the Hall effect equation.  If they cannot think of any, suggest Ohm's Law.

%INDEX% Hall effect, equation describing strength of

%(END_NOTES)


