
%(BEGIN_QUESTION)
% Copyright 2003, Tony R. Kuphaldt, released under the Creative Commons Attribution License (v 1.0)
% This means you may do almost anything with this work of mine, so long as you give me proper credit

Bipolar junction transistor (BJT) function is usually considered in terms of currents: a relatively small current through one of the transistor's terminals exerts control over a much larger current.  Draw the directions of all currents for these two transistors (one NPN and one PNP), clearly identifying which of the currents is {\it doing} the control, and which of the currents is {\it being} controlled:

$$\epsfbox{02037x01.eps}$$

\underbar{file 02037}
%(END_QUESTION)





%(BEGIN_ANSWER)

$$\epsfbox{02037x02.eps}$$

%(END_ANSWER)





%(BEGIN_NOTES)

I have heard questions of this sort asked on technician job interviews.  Knowing which way currents go through a BJT is considered a very fundamental aspect of electronics technician knowledge, and for good reason.  It is impossible to understand the function of many transistor circuits without a firm grasp on which signal exerts control over which other signal in a circuit.

%INDEX% BJT, current directions
%INDEX% Directions of current through a BJT

%(END_NOTES)


