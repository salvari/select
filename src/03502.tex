
%(BEGIN_QUESTION)
% Copyright 2005, Tony R. Kuphaldt, released under the Creative Commons Attribution License (v 1.0)
% This means you may do almost anything with this work of mine, so long as you give me proper credit

Sketch the approximate waveform of this circuit's output signal ($V_{out}$) on the screen of the oscilloscope:

$$\epsfbox{03502x01.eps}$$

Hint: use the Superposition Theorem!

\vskip 10pt

\underbar{file 03502}
%(END_QUESTION)





%(BEGIN_ANSWER)

$$\epsfbox{03502x02.eps}$$

\vskip 10pt

Follow-up question: what would the oscilloscope display look like if the coupling switch for channel A had been set to "AC" instead of "DC"?

%(END_ANSWER)





%(BEGIN_NOTES)

Note that the capacitor size has been chosen for negligible capacitive reactance ($X_C$) at the specific frequency, such that the 10 k$\Omega$ DC bias resistors present negligible loading to the coupled AC signal.  This is typical for this type of biasing circuit.

Aside from giving students an excuse to apply the Superposition Theorem, this question previews a circuit topology that is extremely common in transistor amplifiers.

%INDEX% Superposition theorem, applied to DC bias circuit

%(END_NOTES)


