
%(BEGIN_QUESTION)
% Copyright 2005, Tony R. Kuphaldt, released under the Creative Commons Attribution License (v 1.0)
% This means you may do almost anything with this work of mine, so long as you give me proper credit

A special type of component called a {\it varactor diode} is used to electronically tune resonant circuits, given its ability to change capacitance with the application of a DC voltage.  In the following schematic, the diode, capacitor, and inductor form a resonant circuit; the potentiometer creates a variable DC voltage to impress across the varactor diode (to change its capacitance); and the radio-frequency choke ("RFC") inductor acts as a low-pass filter to block AC from getting to the potentiometer:

$$\epsfbox{03632x01.eps}$$

The equation for calculating the capacitance of a varactor diode is as follows:

$$C_{diode} = {C_o \over {\sqrt{2V + 1}}}$$

\noindent
Where,

$C_{diode} =$ Diode capacitance

$C_o =$ Natural capacitance of diode with zero applied DC voltage

$V = $ Applied DC voltage 

\vskip 10pt

Based on this information, determine which direction the potentiometer must be moved to {\it increase} the resonant frequency of the tuning circuit shown, and also explain why this is so.

\vskip 30pt

\underbar{file 03632}
%(END_QUESTION)





%(BEGIN_ANSWER)

Moving the pot wiper {\it up} increases the DC voltage across the varactor, which decreases its capacitance, which increases the resonant frequency of the LC circuit.

%(END_ANSWER)





%(BEGIN_NOTES)

{\bf This question is intended for exams only and not worksheets!}.

%(END_NOTES)


