
%(BEGIN_QUESTION)
% Copyright 2003, Tony R. Kuphaldt, released under the Creative Commons Attribution License (v 1.0)
% This means you may do almost anything with this work of mine, so long as you give me proper credit

$$\epsfbox{02010x01.eps}$$

The $R_{load}$ (max) and $R_{load}$ (min) parameters are the maximum and minimum resistance settings that $R_{load}$ may be adjusted to with the regulator circuit maintaining constant load voltage.  $V_{load}$ (nominal) is simply the regulated voltage output of the circuit under normal conditions.

\underbar{file 2010}
\vfil \eject
%(END_QUESTION)





%(BEGIN_ANSWER)

Use circuit simulation software to verify your predicted and measured parameter values.

%(END_ANSWER)





%(BEGIN_NOTES)

Use a variable-voltage, regulated power supply to supply any amount of DC voltage below 30 volts.

I highly recommend specifying a large value for $R_{series}$ and/or a high-wattage rated transistor and variable load resistor, so that students do not dissipate excessive power at either the transistor or the load as they test for $R_{load}$ (min).  {\it Do not} use a decade resistance box for $R_{load}$ unless you have made sure its power dissipation will not be exceeded under any circuit condition!

An extension of this exercise is to incorporate troubleshooting questions.  Whether using this exercise as a performance assessment or simply as a concept-building lab, you might want to follow up your students' results by asking them to predict the consequences of certain circuit faults.

%INDEX% Assessment, performance-based (Darlington-buffered zener diode voltage regulator)

%(END_NOTES)


