
%(BEGIN_QUESTION)
% Copyright 2003, Tony R. Kuphaldt, released under the Creative Commons Attribution License (v 1.0)
% This means you may do almost anything with this work of mine, so long as you give me proper credit

Suppose you are working on the power wiring inside of a home, and are wondering whether or not the home is supplied with 3-phase or single-phase power.  You do not have a voltmeter available to measure voltage, but you do have plenty of light bulbs, switches, wires, and other standard residential wiring components available for use.

The two possibilities for this home's power source are shown here, the coils representing secondary windings of the utility power transformer:

$$\epsfbox{00413x01.eps}$$

An experienced electrician suggests you build the following circuit to test whether or not the home's power is supplied by a 3-phase source or a single-phase source:

$$\epsfbox{00413x02.eps}$$

The electrician tells you to open and close the switch, and observe the brightness of the light bulbs.  This will indicate whether or not the system is 3-phase.

Explain how this circuit works.  What sort of light bulb behavior would indicate a 3-phase source?  What sort of light bulb behavior would indicate a single-phase source?

\underbar{file 00413}
%(END_QUESTION)





%(BEGIN_ANSWER)

On a single-phase system, there should be no change in light bulb brightness as the switch is opened and closed.  On a three-phase system, there will be a change in brightness between the two different switch states (I'll let you figure out which switch state makes the light bulbs glow brighter!).

%(END_ANSWER)





%(BEGIN_NOTES)

The subject of this question is closely related to the subject of "single-phasing" a polyphase load as the result of a line conductor opening.  In this case it is the neutral conductor we are intentionally opening, but the results are similar.

%INDEX% Polyphase versus single-phase
%INDEX% Single-phase versus polyphase

%(END_NOTES)


