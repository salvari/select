
%(BEGIN_QUESTION)
% Copyright 2008, Tony R. Kuphaldt, released under the Creative Commons Attribution License (v 1.0)
% This means you may do almost anything with this work of mine, so long as you give me proper credit

Using a B-H curve obtained from a reference book, determine the amount of magnetizing force ($H$) required to establish a magnetic flux density of 0.2 T in a cast iron torus with a cross-sectional area of $7 \times 10^{-4}$ square meters.

$$\epsfbox{00686x01.eps}$$

Calculate the amount of current necessary in the wire coil to establish this amount of flux, if the coil has 250 turns, and the torus has an average flux path length of 45 cm.  Also, calculate the amount of magnetic flux ($\Phi$) inside the torus.

\underbar{file 00686}
%(END_QUESTION)





%(BEGIN_ANSWER)

$H =$ 400 At/m

$I =$ 720 mA

$\Phi =$ 0.14 mWb

%(END_ANSWER)





%(BEGIN_NOTES)

I obtained the magnetizing force figure of 400 At/m for 0.2 T of flux density, from Robert L. Boylestad's 9th edition of \underbar{Introductory Circuit Analysis}, page 437.

%INDEX% B-H curve, for ferrous material
%INDEX% Magnetic flux calculation, using B-H curve

%(END_NOTES)


