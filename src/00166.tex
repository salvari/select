
%(BEGIN_QUESTION)
% Copyright 2003, Tony R. Kuphaldt, released under the Creative Commons Attribution License (v 1.0)
% This means you may do almost anything with this work of mine, so long as you give me proper credit

What size (gauge) of copper wire is needed in this circuit to ensure the load receives at least 110 volts?

$$\epsfbox{00166x01.eps}$$

\underbar{file 00166}
%(END_QUESTION)





%(BEGIN_ANSWER)

\#6 gauge copper wire comes close, but is not quite large enough.  \#5 gauge or larger will suffice.

%(END_ANSWER)





%(BEGIN_NOTES)

Several steps are necessary to solve this problem: Ohm's Law, algebraic manipulation of the specific resistance equation, and research into wire sizes.  Be sure to spend adequate time discussing this problem with your students!

The concept of a generic "load" is any component or device that dissipates electrical power in a circuit.  Often, generic loads are symbolized by a resistor symbol (a zig-zag line), even though they might not really be a resistor.

%INDEX% Conductor resistance calculation
%INDEX% Ohm's Law, quantitative

%(END_NOTES)


