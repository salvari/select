
%(BEGIN_QUESTION)
% Copyright 2003, Tony R. Kuphaldt, released under the Creative Commons Attribution License (v 1.0)
% This means you may do almost anything with this work of mine, so long as you give me proper credit

A student just learning to use oscilloscopes connects one directly to the output of a signal generator, with these results:

$$\epsfbox{01854x01.eps}$$

As you can see, the function generator is configured to output a square wave, but the oscilloscope does not register a square wave.  Perplexed, the student takes the function generator to a different oscilloscope.  At the second oscilloscope, the student sees a proper square wave on the screen:

$$\epsfbox{01854x02.eps}$$

It is then that the student realizes the first oscilloscope has its "coupling" control set to AC, while the second oscilloscope was set to DC.  Now the student is really confused!  The signal is obviously AC, as it oscillates above and below the centerline of the screen, but yet the "DC" setting appears to give the most accurate results: a true-to-form square wave.

How would you explain what is happening to this student, and also describe the appropriate uses of the "AC" and "DC" coupling settings so he or she knows better how to use it in the future?

\underbar{file 01854}
%(END_QUESTION)





%(BEGIN_ANSWER)

"DC" does {\it not} imply that the oscilloscope can only show DC signals and not AC signals, as many beginning students think.  Rather, the "DC" setting is the one that should be first used to measure all signals, with the "AC" setting engaged only as needed.

%(END_ANSWER)





%(BEGIN_NOTES)

The answer I give here is correct, but does not address {\it why} the coupling control does what it does, nor does it describe why the square wave signal appears all distorted on the first oscilloscope's screen.  I leave this for your students to research and for you and your students to discuss together in class.

%INDEX% Oscilloscope, AC/DC coupling controls and low-frequency square-wave signals

%(END_NOTES)


