
%(BEGIN_QUESTION)
% Copyright 2005, Tony R. Kuphaldt, released under the Creative Commons Attribution License (v 1.0)
% This means you may do almost anything with this work of mine, so long as you give me proper credit

When evaluating (calculating) a mathematical expression, what order should you do the various expressions in?  In other words, which comes first: multiplication, division, addition, subtraction, powers, roots, parentheses, etc.; and then what comes after that, and after that?

\underbar{file 03052}
%(END_QUESTION)





%(BEGIN_ANSWER)

Do what is inside parentheses first (the furthest "inside" parentheses if there are multiple layers of parentheses), powers and roots, functions (trig, log, etc.), multiplication/division, and finally addition/subtraction.

%(END_ANSWER)





%(BEGIN_NOTES)

Order of operations is extremely important, as it becomes critical to recognize proper order of evaluation when "stripping" an expression down to isolate a particular variable.  In essence, the normal order of operations is reversed when "undoing" an expression, so students must recognize what the proper order of operations is.

%INDEX% Arithmetic, order of operations
%INDEX% Order of operations, arithmetic

%(END_NOTES)


