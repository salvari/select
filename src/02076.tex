
%(BEGIN_QUESTION)
% Copyright 2003, Tony R. Kuphaldt, released under the Creative Commons Attribution License (v 1.0)
% This means you may do almost anything with this work of mine, so long as you give me proper credit

The {\it impedance triangle} is often used to graphically relate $Z$, $R$, and $X$ in a series circuit:

$$\epsfbox{02076x01.eps}$$

Unfortunately, many students do not grasp the significance of this triangle, but rather memorize it as a "trick" used to calculate one of the three variables given the other two.  Explain {\it why} a right triangle is an appropriate form to relate these variables, and what each side of the triangle actually represents.

\underbar{file 02076}
%(END_QUESTION)





%(BEGIN_ANSWER)

Each side of the impedance triangle is actually a {\it phasor} (a vector representing impedance with magnitude and direction):

$$\epsfbox{02076x02.eps}$$

Since the phasor for resistive impedance ($Z_R$) has an angle of zero degrees and the phasor for reactive impedance ($Z_C$ or $Z_L$) either has an angle of +90 or -90 degrees, the {\it phasor sum} representing total series impedance will form the hypotenuse of a right triangle when the first to phasors are added (tip-to-tail).

\vskip 10pt

Follow-up question: as a review, explain why resistive impedance phasors always have an angle of zero degrees, and why reactive impedance phasors always have angles of either +90 degrees or -90 degrees.

%(END_ANSWER)





%(BEGIN_NOTES)

The question is sufficiently open-ended that many students may not realize exactly what is being asked until they read the answer.  This is okay, as it is difficult to phrase the question in a more specific manner without giving away the answer!

%INDEX% Impedance triangle

%(END_NOTES)


