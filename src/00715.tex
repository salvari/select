
%(BEGIN_QUESTION)
% Copyright 2003, Tony R. Kuphaldt, released under the Creative Commons Attribution License (v 1.0)
% This means you may do almost anything with this work of mine, so long as you give me proper credit

When "P" and "N" type semiconductor pieces are brought into close contact, free electrons from the "N" piece will rush over to fill holes in the "P" piece, creating a zone on both sides of the contact region devoid of charge carriers.  What is this zone called, and what are its electrical characteristics?

\underbar{file 00715}
%(END_QUESTION)





%(BEGIN_ANSWER)

This is called the {\it depletion region}, and it is essentially an insulator at room temperatures.

%(END_ANSWER)





%(BEGIN_NOTES)

Students should know that both "N" and "P" type semiconductors are electrically conductive.  So, when a depletion region forms in the contact zone between two differing semiconductor types, the conductivity from end-to-end must be affected.  Ask your students what this effect is, and what factors may influence it.

%INDEX% Depletion region, defined

%(END_NOTES)


