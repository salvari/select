
%(BEGIN_QUESTION)
% Copyright 2003, Tony R. Kuphaldt, released under the Creative Commons Attribution License (v 1.0)
% This means you may do almost anything with this work of mine, so long as you give me proper credit

Calculate the output voltages of these two voltage divider circuits ($V_A$ and $V_B$):

$$\epsfbox{01725x01.eps}$$

Now, calculate the voltage between points {\bf A} (red lead) and {\bf B} (black lead) ($V_{AB}$).

\underbar{file 01725}
%(END_QUESTION)





%(BEGIN_ANSWER)

$V_A =$ + 65.28 V

$V_B =$ + 23.26 V

$V_{AB} =$ + 42.02 V (point {\bf A} being positive relative to point {\bf B})

\vskip 10pt

Challenge question: what would change if the wire connecting the two voltage divider circuits together were removed?

$$\epsfbox{01725x02.eps}$$

%(END_ANSWER)





%(BEGIN_NOTES)

In this question, I want students to see how the voltage between the two dividers' output terminals is the difference between their individual output voltages.  I also want students to see the notation used to denote the voltages (use of subscripts, with an applied reference point of ground).  Although voltage is {\it always and forever} a quantity between two points, it is appropriate to speak of voltage being "at" a single point in a circuit if there is an implied point of reference (ground).

It is possible to solve for $V_{AB}$ without formally appealing to Kirchhoff's Voltage Law.  One way I've found helpful to students is to envision the two voltages ($V_A$ and $V_B$) as heights of objects, asking the question of "How much height {\it difference} is there between the two objects?"

$$\epsfbox{01725x03.eps}$$

The height of each object is analogous to the voltage dropped across each of the lower resistors in the voltage divider circuits.  Like voltage, height is a quantity measured {\it between two points} (the top of the object and ground level).  Also like the voltage $V_{AB}$, the difference in height between the two objects is a measurement taken between two points, and it is also found by subtraction.

%INDEX% Voltage divider

%(END_NOTES)


