
%(BEGIN_QUESTION)
% Copyright 2003, Tony R. Kuphaldt, released under the Creative Commons Attribution License (v 1.0)
% This means you may do almost anything with this work of mine, so long as you give me proper credit

We know that connecting a sensitive meter movement directly across the terminals of a substantial voltage source (such as a battery) is a Bad Thing.  So, I want you to determine what other component(s) must be connected to the meter movement to limit the current through its coil, so that connecting the circuit to a 6-volt battery results in the meter's needle moving exactly to the full-scale position:

$$\epsfbox{00720x01.eps}$$

\underbar{file 00720}
%(END_QUESTION)





%(BEGIN_ANSWER)

$$\epsfbox{00720x02.eps}$$

%(END_ANSWER)





%(BEGIN_NOTES)

Beginning students sometimes feel "lost" when trying to answer a question like this.  They may know how to apply Ohm's Law to a circuit, but they do not know how to design a circuit that makes use of Ohm's Law for a specific purpose.  If this is the case, you may direct their understanding through a series of questions such as this:

\medskip
\item{$\bullet$} Why does the meter movement "peg" if directly connected to the battery?
\item{$\bullet$} What type of electrical component is good at limiting current?
\item{$\bullet$} How might we connect this component to the meter (series or parallel)?  (Draw both configurations and let the student determine for themselves which connection pattern fulfills the goal of limiting current to the meter.)
\medskip

The math is simple enough in this question to allow solution without the use of a calculator.  Whenever possible, I challenge students during discussion time to perform any necessary arithmetic "mentally" (i.e. without using a calculator), even if only to estimate the answer.  I find many American high school graduates unable to do even very simple arithmetic without a calculator, and this lack of skill causes them no small amount of trouble.  Not only are these students helpless without a calculator, but they lack the ability to mentally check their calculator-derived answers, so when they do use a calculator they have no idea whether their answer is even close to being correct.

%INDEX% Voltmeter, range resistor sizing

%(END_NOTES)


