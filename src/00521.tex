
%(BEGIN_QUESTION)
% Copyright 2003, Tony R. Kuphaldt, released under the Creative Commons Attribution License (v 1.0)
% This means you may do almost anything with this work of mine, so long as you give me proper credit

An oft-repeated claim is that lead-acid batteries self-discharge at a greater rate while sitting on a concrete floor, than when installed in an automobile, or on a shelf, or in some other place where there is no direct contact with a concrete surface.  Is this claim true, or not?

Design an experiment to test the veracity of this claim, and design it in such a way that a single instance of a defective battery does not skew the results of the test.

\underbar{file 00521}
%(END_QUESTION)





%(BEGIN_ANSWER)

This is a fun topic to discuss!  Do your research, and come to discussion time prepared with ideas on how to test this claim.

%(END_ANSWER)





%(BEGIN_NOTES)

I have heard this claim many, many times, but I have not yet been in such a position where I could test it myself.  Even if I had tested it, I still wouldn't give away the answer!  There are many practical matters that must be addressed in the setup and testing of batteries for self-discharge.  This question provides an excellent opportunity to discuss scientific method with your students, including such principles as {\it double-blind} testing, {\it test} and {\it control} groups, and statistical correlation.

%INDEX% Battery discharging

%(END_NOTES)


