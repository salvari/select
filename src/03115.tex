
%(BEGIN_QUESTION)
% Copyright 2005, Tony R. Kuphaldt, released under the Creative Commons Attribution License (v 1.0)
% This means you may do almost anything with this work of mine, so long as you give me proper credit

Suppose you were given two components and told one was an inductor while the other was a capacitor.  Both components are unmarked, and impossible to visually distinguish or identify.  Explain how you could use an ohmmeter to distinguish one from the other, based on each component's response to direct current (DC).

Then, explain how you could approximately measure the value of each component using nothing more than a sine-wave signal generator and an AC meter capable only of precise AC voltage and current measurements across a wide frequency range (no direct capacitance or inductance measurement capability), and show how the reactance equation for each component (L and C) would be used in your calculations.

\underbar{file 03115}
%(END_QUESTION)





%(BEGIN_ANSWER)

Did you really think I would give you the answers to a question like this?

\vskip 10pt

Challenge question: suppose the only test equipment you had available was a 6-volt battery and an old analog volt-milliammeter (with no resistance check function).  How could you use this primitive gear to identify which component was the inductor and which was the capacitor?

%(END_ANSWER)





%(BEGIN_NOTES)

This is an excellent opportunity to brainstorm as a group and experiment on real components.  The purpose of this question is to make the reactance equations more "real" to students by having them apply the equations to a realistic scenario.  The ohmmeter test is based on DC component response, which may be thought of in terms of reactance at a frequency at or near zero.  The multimeter/generator test is based on AC response, and will require algebraic manipulation to convert the canonical forms of these equations to versions appropriate for calculating L and C.

%INDEX% Algebra, manipulating equations
%INDEX% Component identification (L or C)

%(END_NOTES)


