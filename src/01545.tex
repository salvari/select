
%(BEGIN_QUESTION)
% Copyright 2003, Tony R. Kuphaldt, released under the Creative Commons Attribution License (v 1.0)
% This means you may do almost anything with this work of mine, so long as you give me proper credit

What is the difference between {\it white noise} and {\it pink noise}?

\vskip 10pt

White noise in the audio range is generally considered to be pleasing to the human ear.  Identify one source of "white" noise that anyone can readily experience for themselves.

\underbar{file 01545}
%(END_QUESTION)





%(BEGIN_ANSWER)

The difference has to do with the distribution of noise amplitude compared to noise frequency.  Hint: compare red-colored light with "white" light.

\vskip 10pt

With regard to white noise, {\it everyone} has heard it at least one in their lives, but most people don't know what to call it!

%(END_ANSWER)





%(BEGIN_NOTES)

Explain to your students that other colors are used to categorize spectral distributions of different noise types:

\medskip
\item{$\bullet$} Purple = $f^2$ (directly proportional to square of frequency)
\item{$\bullet$} Blue = $f$ (directly proportional to frequency)
\item{$\bullet$} White = 1 (constant amplitude)
\item{$\bullet$} Pink = $1/f$ (inversely proportional to frequency)
\item{$\bullet$} Red/Brown = $1/{f^2}$ (inversely proportional to square of frequency)
\medskip

And, of course, there are such noises that do not neatly fall into any of these categories.

For a good overview of electrical noise, consult Texas Instruments' online manual, {\it Op Amps For Everyone}, sections 10-1 through 10-12.

%INDEX% Noise types, defined

%(END_NOTES)


