
%(BEGIN_QUESTION)
% Copyright 2003, Tony R. Kuphaldt, released under the Creative Commons Attribution License (v 1.0)
% This means you may do almost anything with this work of mine, so long as you give me proper credit

What is the {\it complement} of a Boolean number?  How do we represent the complement of a Boolean variable, and what logic circuit function performs the complementation function?

\underbar{file 01300}
%(END_QUESTION)





%(BEGIN_ANSWER)

A Boolean "complement" is the {\it opposite} value of a given number.  This is represented either by overbars or prime marks next to the variable (i.e. the complement of $A$ may be written as either $\overline{A}$ or $A'$):

$$\epsfbox{01300x01.eps}$$

%(END_ANSWER)





%(BEGIN_NOTES)

Students need to be able to readily associate fundamental Boolean operations with logic circuits.  If they can see the relationship between the "strange" rules of Boolean arithmetic and something they are already familiar with (i.e. truth tables), the association is made much easier.

%INDEX% Boolean algebra, definition of a complement

%(END_NOTES)


