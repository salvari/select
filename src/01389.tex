
%(BEGIN_QUESTION)
% Copyright 2003, Tony R. Kuphaldt, released under the Creative Commons Attribution License (v 1.0)
% This means you may do almost anything with this work of mine, so long as you give me proper credit

In analyzing circuits with inductors, we often take the luxury of assuming the constituent inductors to be perfect; i.e., purely inductive, with no "stray" properties such as winding resistance or inter-winding capacitance.

Real life is not so generous.  With real inductors, we have to consider these factors.  One measure often used to express the "purity" of an inductor is its so-called {\it Q rating}, or {\it quality factor}.

Write the formula for calculating the quality factor (Q) of a coil, and describe some of the operational parameters that may affect this number.

\underbar{file 01389}
%(END_QUESTION)





%(BEGIN_ANSWER)

$$Q_{coil} = {X_L \over R}$$

%(END_ANSWER)





%(BEGIN_NOTES)

Your students should be able to immediately understand that $Q$ is not a static property of an inductor.  Let them explain what makes $Q$ vary, based on their knowledge of inductive reactance.

%INDEX% Quality factor of inductor, defined

%(END_NOTES)


