
%(BEGIN_QUESTION)
% Copyright 2003, Tony R. Kuphaldt, released under the Creative Commons Attribution License (v 1.0)
% This means you may do almost anything with this work of mine, so long as you give me proper credit

Suppose we were building a circuit that required an adjustable resistance with a range of 1500 $\Omega$ to 4500 $\Omega$.  The only potentiometer we have on hand is a 10 k$\Omega$ unit.  Of course, we could simply connect the potentiometer as-is and have an adjustable range of 0 $\Omega$ to 10,000 $\Omega$, but that would be too "coarse" of an adjustment for our application.

Explain how we could connect other resistors to this 10 k$\Omega$ potentiometer in order to achieve the desired adjustable resistance range.

\underbar{file 00470}
%(END_QUESTION)





%(BEGIN_ANSWER)

I'll give you a hint:

$$\epsfbox{00470x01.eps}$$

%(END_ANSWER)





%(BEGIN_NOTES)

Knowing how to limit the adjustable range of a potentiometer is a very useful skill when designing and building circuits where precision of adjustment is important.  The only drawback to building such a sub-circuit is that the adjustment becomes nonlinear (i.e., setting the potentiometer to the half-way position does not result in the total resistance being 50\% of the way between the lower and upper range values).

%INDEX% Potentiometer, custom resistance range

%(END_NOTES)


