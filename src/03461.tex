
%(BEGIN_QUESTION)
% Copyright 2005, Tony R. Kuphaldt, released under the Creative Commons Attribution License (v 1.0)
% This means you may do almost anything with this work of mine, so long as you give me proper credit

We know at this point that any circuit comprised of inductance ($L$) and capacitance ($C$) is capable of {\it resonating}: attaining large values of AC voltage and current if "excited" at the proper frequency.  The so-called {\it tank circuit} is the simplest example of this:

$$\epsfbox{03461x01.eps}$$

The less resistance ($R$) such a circuit has, the better its ability to resonate.

\vskip 10pt

We also know that any piece of wire contains both inductance and capacitance, distributed along its length.  These properties are not necessarily intentional -- they exist whether we would want them to or not:

$$\epsfbox{03461x02.eps}$$

Given that the electrical resistance of a continuous piece of metal wire is usually quite low, describe what these natural properties of inductance and capacitance mean with regard to that wire's function as an electrical element.

\underbar{file 03461}
%(END_QUESTION)





%(BEGIN_ANSWER)

The fact that an piece of wire contains both inductance and capacitance means that it has the ability to resonate just like any tank circuit!

\vskip 10pt

Follow-up question: qualitatively estimate the frequency you suppose a length of wire would resonate at.  Do you think $f_r$ would be a very low value (tens of hertz), a very high value (thousands, millions, or billions of hertz), or somewhere in between?  Keep in mind the equation for resonant frequency:

$$f_r = {1 \over {2 \pi \sqrt{LC}}}$$

%(END_ANSWER)





%(BEGIN_NOTES)

If your students have difficulty knowing where to start with the follow-up question, ask them to qualitatively estimate the distributed $L$ and $C$ for a piece of wire, say, 10 feet long.  Given the lack of any high-permeability core material and the lack of any high-permittivity dielectric (just air), the answers for both should be "very small."  Then, ask them again how they would qualitatively rate the wire's resonant frequency.

%INDEX% Antenna, modeled as an LC resonant circuit
%INDEX% Resonance, antenna

%(END_NOTES)


