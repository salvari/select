
%(BEGIN_QUESTION)
% Copyright 2003, Tony R. Kuphaldt, released under the Creative Commons Attribution License (v 1.0)
% This means you may do almost anything with this work of mine, so long as you give me proper credit

The amount of voltage applied to a permanent-magnet DC motor, and the amount of current going through the armature windings of a permanent-magnet DC motor, are related to two mechanical quantities: maximum speed, and torque output (twisting force).

Which electrical quantity relates to which mechanical quantity?  Is it voltage that relates to speed and current to torque, or visa-versa?  Explain your answer.

\underbar{file 00399}
%(END_QUESTION)





%(BEGIN_ANSWER)

The amount of voltage applied to a permanent-magnet DC motor determines its no-load speed, while the amount of current through the armature windings is indicative of the torque output.

%(END_ANSWER)





%(BEGIN_NOTES)

This question asks students to relate concepts of electromagnetism and electromagnetic induction together with voltage and current.  While the permanent-magnet style of DC motor exhibits almost linear relationships between these variables, all DC electric motors exhibit the same general pattern: more voltage, more speed; more current, more torque; all other variables being equal.

%INDEX% DC electric motor, speed and torque
%INDEX% Speed and torque, DC electric motor

%(END_NOTES)


