
%(BEGIN_QUESTION)
% Copyright 2003, Tony R. Kuphaldt, released under the Creative Commons Attribution License (v 1.0)
% This means you may do almost anything with this work of mine, so long as you give me proper credit

Where does the electricity come from that powers your home, or your school, or the streetlights along roads, or the many business establishments in your city?  You will find that there are many different sources and types of sources of electrical power.  In each case, try to determine where the {\it ultimate} source of that energy is.

For example, in a hydroelectric dam, the electricity is generated when falling water spins a turbine, which turns an electromechanical generator.  But what continually drives the water to its "uphill" location so that the process is continuous?  What is the {\it ultimate} source of energy that is being harnessed by the dam?

\underbar{file 00024}
%(END_QUESTION)





%(BEGIN_ANSWER)

Some sources of electrical power:

\medskip
\item{$\bullet$} Hydroelectric dams
\item{$\bullet$} Nuclear power plants
\item{$\bullet$} Coal and oil fired power plants
\item{$\bullet$} Natural gas fired power plants
\item{$\bullet$} Wood fired power plants
\item{$\bullet$} Geothermal power plants
\item{$\bullet$} Solar power plants
\item{$\bullet$} Tidal/wave power plants
\item{$\bullet$} Windmills
\medskip

%(END_ANSWER)





%(BEGIN_NOTES)

A great point of conversation here is that almost all "sources" of energy have a common origin.  The different "sources" are merely variant expressions of the same true source (with exceptions, of course!).

%INDEX% Sources of electricity
%INDEX% Energy, conceptual

%(END_NOTES)


