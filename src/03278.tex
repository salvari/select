
%(BEGIN_QUESTION)
% Copyright 2005, Tony R. Kuphaldt, released under the Creative Commons Attribution License (v 1.0)
% This means you may do almost anything with this work of mine, so long as you give me proper credit

Suppose two people work together to slide a large box across the floor, one pushing with a force of 400 newtons and the other pulling with a force of 300 newtons:

$$\epsfbox{03278x01.eps}$$

The resultant force from these two persons' efforts on the box will, quite obviously, be the sum of their forces: 700 newtons (to the right).

\vskip 10pt

What if the person pulling decides to change position and push {\it sideways} on the box in relation to the first person, so the 400 newton force and the 300 newton force will be perpendicular to each other (the 300 newton force facing into the page, away from you)?  What will the resultant force on the box be then?

$$\epsfbox{03278x02.eps}$$

\underbar{file 03278}
%(END_QUESTION)





%(BEGIN_ANSWER)

The resultant force on the box will be 500 newtons.

%(END_ANSWER)





%(BEGIN_NOTES)

This is a non-electrical application of vector summation, to prepare students for the concept of using vectors to add voltages that are out-of-phase.  Note how I chose to use multiples of 3, 4, and 5 for the vector magnitudes.

%INDEX% Phasor diagram

%(END_NOTES)


