
%(BEGIN_QUESTION)
% Copyright 2006, Tony R. Kuphaldt, released under the Creative Commons Attribution License (v 1.0)
% This means you may do almost anything with this work of mine, so long as you give me proper credit

A very common accessory for oscilloscopes is a $\times$10 probe, which effectively acts as a 10:1 voltage divider for any measured signals.  Thus, an oscilloscope showing a waveform with a peak-to-peak amplitude of 4 divisions, with a vertical sensitivity setting of 1 volt per division, using a $\times$10 probe, would actually be measuring a signal of {\it 40 volts} peak-peak:

$$\epsfbox{03985x01.eps}$$

Obviously, one use for a $\times$10 probe is measuring voltages beyond the normal range of an oscilloscope.  However, there is another application that is less obvious, and it regards the input impedance of the oscilloscope.  A $\times$10 probe gives the oscilloscope 10 times more input impedance (as seen from the probe tip to ground).  Typically this means an input impedance of 10 M$\Omega$ (with the $\times$10 probe) rather than 1 M$\Omega$ (with a normal 1:1 probe).  Identify an application where this feature could be useful.

\underbar{file 03985}
%(END_QUESTION)





%(BEGIN_ANSWER)

I won't give away an answer here, but I will provide a hint in the form of another question: why is it generally a good thing for voltmeters to have high input impedance?  Or conversely, what bad things might happen if you tried to use a low-impedance voltmeter to measure voltages?

%(END_ANSWER)





%(BEGIN_NOTES)

Increased input impedance is often a more common reason for choosing $\times$10 probes, as opposed to increased voltage measurement range.  The answer to this question is more readily grasped by students after they have worked with loading-sensitive electronic circuits.

%INDEX% 10x probes, oscilloscope
%INDEX% Probe (x10), uses for

%(END_NOTES)


