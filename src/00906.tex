
%(BEGIN_QUESTION)
% Copyright 2003, Tony R. Kuphaldt, released under the Creative Commons Attribution License (v 1.0)
% This means you may do almost anything with this work of mine, so long as you give me proper credit

Shown here are two energy diagrams: one for a "P" type semiconducting material and another for an "N" type.

$$\epsfbox{00906x01.eps}$$

Next is an energy diagram showing the {\it initial} state when these two pieces of semiconducting material are brought into contact with each other.  This is known as a {\it flatband diagram}:

$$\epsfbox{00906x02.eps}$$

The state represented by the "flatband" diagram is most definitely a temporary one.  The two different Fermi levels are incompatible with one another in the absence of an external electric field.

Draw a new energy diagram representing the final energy states after the two Fermi levels have equalized.

\vskip 10pt

{\it Note: $E_f$ represents the Fermi energy level, and not a voltage.  In physics, $E$ always stands for energy and $V$ for electric potential (voltage).} 

\underbar{file 00906}
%(END_QUESTION)





%(BEGIN_ANSWER)

$$\epsfbox{00906x03.eps}$$

Electrons from the N-piece rushed over to fill holes in the P-piece {\it in order to achieve a lower energy state} and equalize the two Fermi levels.  This displacement of charge carriers created an electric field which accounts for the sloped energy bands in the middle region.

\vskip 10pt

Follow-up question: what is this middle region called?

%(END_ANSWER)





%(BEGIN_NOTES)

This is one of those concepts I just couldn't understand when I had no comprehension of the quantum nature of electrons.  In the "planetary" atomic model, there is no reason whatsoever for electrons to move from the N-piece to the P-piece unless there was an electric field pushing them in that direction.  And conversely, once an electric field was created by the imbalance of electrons, the free-wheeling planetary theory would have predicted that the electrons move right back where they came from in order to neutralize the field.

Once you grasp the significance of quantized energy states, and the principle that particles do not "hold on" to unnecessary energy and therefore remain in high states when they could move down to a lower level, the concept becomes much clearer.

%INDEX% Diagrams, electron band
%INDEX% Flatband diagram, PN junction
%INDEX% Fermi level, for semiconducting substances
%INDEX% Energy diagram, equilibrium state for PN junction
%INDEX% PN junction
%INDEX% Junction, PN

%(END_NOTES)


