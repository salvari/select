
%(BEGIN_QUESTION)
% Copyright 2003, Tony R. Kuphaldt, released under the Creative Commons Attribution License (v 1.0)
% This means you may do almost anything with this work of mine, so long as you give me proper credit

A common feature of oscilloscopes is the $X-Y$ mode, where the vertical and horizontal plot directions are driven by external signals, rather than only the vertical direction being driven by a measured signal and the horizontal being driven by the oscilloscope's internal sweep circuitry:

$$\epsfbox{01480x01.eps}$$

The oval pattern shown in the right-hand oscilloscope display of the above illustration is typical for two sinusoidal waveforms of the same frequency, but slightly out of phase with one another.  The technical name for this type of $X-Y$ plot is a {\it Lissajous figure}.

What should the Lissajous figure look like for two sinusoidal waveforms that are at exactly the same frequency, and exactly the same phase (0 degrees phase shift between the two)?  What should the Lissajous figure look like for two sinusoidal waveforms that are exactly 90 degrees out of phase?

A good way to answer each of these questions is to plot the specified waveforms over time on graph paper, then determine their instantaneous amplitudes at equal time intervals, and then determine where that would place the "dot" on the oscilloscope screen at those points in time, in $X-Y$ mode.  To help you, I'll provide two blank oscilloscope displays for you to draw the Lissajous figures on:

$$\epsfbox{01480x02.eps}$$

\underbar{file 01480}
%(END_QUESTION)





%(BEGIN_ANSWER)

$$\epsfbox{01480x03.eps}$$

\vskip 10pt

Challenge question: what kind of Lissajous figures would be plotted by the oscilloscope if the signals were non-sinusoidal?  Perhaps the simplest example of this would be two square waves instead of two sine waves.

%(END_ANSWER)





%(BEGIN_NOTES)

Many students seem to have trouble grasping how Lissajous figures are formed.  One of the demonstrations I use to overcome this conceptual barrier is an analog oscilloscope and two signal generators set to very low frequencies, so students can see the "dot" being swept across the screen by both waveforms in slow-motion.  Then, I speed up the signals and let them see how the Lissajous pattern becomes more "solid" with persistence of vision and the inherent phosphor delay of the screen.

%INDEX% Lissajous figures

%(END_NOTES)


