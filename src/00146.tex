
%(BEGIN_QUESTION)
% Copyright 2003, Tony R. Kuphaldt, released under the Creative Commons Attribution License (v 1.0)
% This means you may do almost anything with this work of mine, so long as you give me proper credit

{\it Lightning} is a natural, electrical phenomenon.  It is caused by the accumulation of a large electrical charge over time resulting from air, dust, and water droplets transporting small electrical charges.

Explain how the terms {\it voltage}, {\it current}, and {\it resistance} relate to the process of lightning.  In other words, use these three terms to explain the cycle of charge accumulation and lightning discharge.
 
\underbar{file 00146}
%(END_QUESTION)





%(BEGIN_ANSWER)

As electric charges accumulate between clouds and earth, the {\bf voltage} between these points increase.  Air under normal conditions is a good insulator of electricity: that is, it possesses very high electrical {\bf resistance}.  So, at first there is no {\bf current} resulting from the rise in voltage between clouds and earth.  However, when the {\bf voltage} exceeds the air's "ionization" potential, the air becomes a good conductor of electricity (its electrical {\bf resistance} decreases dramatically), resulting in a transient {\bf current} as the accumulated electric charge dissipates in the form of a lightning bolt.

%(END_ANSWER)





%(BEGIN_NOTES)

I usually avoid spending a lot of time on technical definitions, because undue emphasis on the definitions of words tends to reinforce rote memorization rather than true comprehension.  If students must master certain definitions, though, it is best to develop that mastery in the context of application: ask the students to {\it use} their new vocabulary, not just recite it.

%INDEX% Lightning
%INDEX% Voltage, as it relates to lightning
%INDEX% Current, as it relates to lightning
%INDEX% Resistance, as it relates to lightning

%(END_NOTES)


