
%(BEGIN_QUESTION)
% Copyright 2003, Tony R. Kuphaldt, released under the Creative Commons Attribution License (v 1.0)
% This means you may do almost anything with this work of mine, so long as you give me proper credit

Suppose you need a memory array with 1k $\times$ 8 organization, but all you have on hand are 1k $\times$ 4 memory chips.  Show how you could connect two of them to form the desired array:

$$\epsfbox{01446x01.eps}$$

\underbar{file 01446}
%(END_QUESTION)





%(BEGIN_ANSWER)

$$\epsfbox{01446x02.eps}$$

\vskip 10pt

Follow-up question: a common mistake made by students when they "expand" the data bus width of a memory array is to parallel the output lines (in the same way that the address lines are shown paralleled here).  Why would this be wrong to do?  What might happen to the memory chip(s) if their data lines were paralleled?

%(END_ANSWER)





%(BEGIN_NOTES)

Be sure to spend some time discussing the common mistake referenced in the follow-up question.  This is something I've seen more than once, and it reveals a fundamental gap in understanding on the part of the mistaken student.  What students are prone to do is try to {\it memorize} the sequence of connections rather than really understand {\it why} memory array expansion works, which leads to errors such as this.

Note that the answer to "what might happen" depends on whether the first operation is a {\it read}, or a {\it write}.

%INDEX% Memory expansion, data
%INDEX% Data expansion, memory

%(END_NOTES)


