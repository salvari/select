
%(BEGIN_QUESTION)
% Copyright 2004, Tony R. Kuphaldt, released under the Creative Commons Attribution License (v 1.0)
% This means you may do almost anything with this work of mine, so long as you give me proper credit

A doorbell ringer has a solenoid with an inductance of 63 mH connected in parallel with a lamp (for visual indication) having a resistance of 150 ohms:

$$\epsfbox{02105x01.eps}$$

Calculate the phase shift of the total current (in units of degrees) in relation to the total supply voltage, when the doorbell switch is actuated.

\underbar{file 02105}
%(END_QUESTION)





%(BEGIN_ANSWER)

$\Theta$ = 81 degrees

\vskip 10pt

Suppose the lamp turned on whenever the pushbutton switch was actuated, but the doorbell refused to ring.  Identify what you think to be the most likely fault which could account for this problem.

%(END_ANSWER)





%(BEGIN_NOTES)

This would be an excellent question to have students present methods of solution for.  Sometimes I have students present nothing but their solution steps on the board in front of class (no arithmetic at all), in order to generate a discussion on problem-solving strategies.  The important part of their education here is not to arrive at the correct answer or to memorize an algorithm for solving this type of problem, but rather how to {\it think} like a problem-solver, and how to methodically apply the math they know to the problem(s) at hand.

%INDEX% Phase shift calculation, parallel LR circuit

%(END_NOTES)


