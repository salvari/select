
%(BEGIN_QUESTION)
% Copyright 2003, Tony R. Kuphaldt, released under the Creative Commons Attribution License (v 1.0)
% This means you may do almost anything with this work of mine, so long as you give me proper credit

Special types of vectors called {\it phasors} are often used to depict the magnitude and phase-shifts of sinusoidal AC voltages and currents.  Suppose that the following phasors represent the series summation of two AC voltages, one with a magnitude of 3 volts and the other with a magnitude of 4 volts:

$$\epsfbox{01559x01.eps}$$

Explain what each of the following phasor diagrams represents, in electrical terms:

$$\epsfbox{01559x02.eps}$$

Also explain the significance of these sums: that we may obtain three {\it different} values of total voltage (7 volts, 1 volt, or 5 volts) from the same series-connected AC voltages.  What does this mean for us as we prepare to analyze AC circuits using the rules we learned for DC circuits?

\underbar{file 01559}
%(END_QUESTION)





%(BEGIN_ANSWER)

Each of the phasor diagrams represents two AC voltages being added together.  The dotted phasor represents the sum of the 3-volt and 4-volt signals, for different conditions of phase shift between them.

Please note that these three possibilities are not exhaustive!  There are a multitude of other possible total voltages that the series-connected 3 volt and 4 volt sources may create.

\vskip 10pt

Follow-up question: in DC circuits, it is permissible to connect multiple voltage sources in parallel, so long as the voltages (magnitudes) and polarities are the same.  Is this also true for AC?  Why or why not?

%(END_ANSWER)





%(BEGIN_NOTES)

Be sure to discuss with your students that these three conditions shown are not the only conditions possible!  I simply chose 0$^{o}$, 180$^{o}$, and 90$^{o}$ because they all resulted in round sums for the given quantities.

The follow-up question previews an important subject concerning AC phase: the necessary {\it synchronization} or paralleled AC voltage sources.

%INDEX% Addition of two sine waves, phasor representation
%INDEX% Phasor diagram

%(END_NOTES)


