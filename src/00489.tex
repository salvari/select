
%(BEGIN_QUESTION)
% Copyright 2003, Tony R. Kuphaldt, released under the Creative Commons Attribution License (v 1.0)
% This means you may do almost anything with this work of mine, so long as you give me proper credit

An electric {\it arc welder} is a power conversion device, used to step utility power voltage (usually 240 or 480 volts AC) down to a low voltage, and conversely step up the current (to 100 amps or more), to generate a very hot arc used to weld metal pieces together:

$$\epsfbox{00489x01.eps}$$

The simplest designs of arc welder are nothing more than a large step-down transformer.  To achieve different power intensities for welding different thicknesses of metal, some of these arc welders are equipped with taps on the secondary winding:

$$\epsfbox{00489x02.eps}$$

Some arc welder designs achieve continuous variability by moving a magnetic "shunt" in and out of the transformer core structure:

$$\epsfbox{00489x03.eps}$$

Explain how this shunt works.  Which way does it need to be moved in order to increase the intensity of the welding arc?  What advantages does this method of arc power control have over a "tapped" secondary winding?

\underbar{file 00489}
%(END_QUESTION)





%(BEGIN_ANSWER)

As the shunt is pulled further away from the core (up, in the illustration), the welding arc intensity increases.

\vskip 10pt

Challenge question: why would it not be a good idea to achieve the same continuously-variable arc control by varying the reluctance ($\Re$) of the transformer's magnetic circuit, like this?

$$\epsfbox{00489x04.eps}$$

%(END_ANSWER)





%(BEGIN_NOTES)

This question illustrates an application of the coupling (k) factor between mutual inductors.  There are a few advantages of controlling the arc welder's output in this manner, as compared to using winding taps, so be sure to discuss this with your students.

As to the challenge question, controlling the transformer output in this manner would also affect the magnetizing inductance of the primary winding, which would have detrimental effects at low settings (what would happen to the "excitation" current of the primary winding as its inductance decreases?).

%INDEX% Transformer, used for an electric arc welder

%(END_NOTES)


