
\centerline{\bf ELTR 105 (DC 2), section 3} \bigskip 
 
\vskip 10pt

\noindent
{\bf Recommended schedule}

\vskip 5pt

%%%%%%%%%%%%%%%
\hrule \vskip 5pt
\noindent
\underbar{Day 1}

\hskip 10pt Topics: {\it Inductance and inductors}
 
\hskip 10pt Questions: {\it 1 through 20}
 
\hskip 10pt Lab Exercises: {\it Series inductances (question 81) and parallel inductances (question 82)}
 
%INSTRUCTOR \hskip 10pt {\bf Demo: Light up a neon bulb using an inductor and a low-voltage battery}

%INSTRUCTOR \hskip 10pt {\bf Demo: Show picture of a substation inductor (fault current limiter)}

\vskip 10pt
%%%%%%%%%%%%%%%
\hrule \vskip 5pt
\noindent
\underbar{Day 2}

\hskip 10pt Topics: {\it Capacitance and capacitors}
 
\hskip 10pt Questions: {\it 21 through 40}
 
\hskip 10pt Lab Exercises: {\it Series capacitances (question 83) and parallel capacitances (question 84)}
 
%INSTRUCTOR \hskip 10pt {\bf MIT 8.02 video clip: Disk 1, Lecture 2; Electric field lines shown 41:00 to 45:45}

%INSTRUCTOR \hskip 10pt {\bf MIT 8.02 video clip: Disk 1, Lecture 3; Electric field between plates 47:52 to end}

%INSTRUCTOR \hskip 10pt {\bf MIT 6.002 video clip: Disk 2, Lecture 13; Capacitor discharge demo 24:55 to 27:20}

%INSTRUCTOR \hskip 10pt {\bf Demo: Charge a long two-wire cable, then measure stored voltage}

\vskip 10pt
%%%%%%%%%%%%%%%
\hrule \vskip 5pt
\noindent
\underbar{Day 3}

\hskip 10pt Topics: {\it Time constants}
 
\hskip 10pt Questions: {\it 41 through 60}
 
\hskip 10pt Lab Exercise: {\it RC discharge circuit (question 85)}
 
%INSTRUCTOR \hskip 10pt {\bf MIT 6.002 video clip: Disk 2, Lecture 13; RC time constant demo 47:03 to end}

%INSTRUCTOR \hskip 10pt {\bf MIT 8.02 video clip: Disk 3, Lecture 20; L/R time constant demo 26:27 to 30:30}

\vskip 10pt
%%%%%%%%%%%%%%%
\hrule \vskip 5pt
\noindent
\underbar{Day 4}

\hskip 10pt Topics: {\it Time constant circuits}
 
\hskip 10pt Questions: {\it 61 through 80}
 
\hskip 10pt Lab Exercise: {\it Time-delay relay (question 86)}
 
\vskip 10pt
%%%%%%%%%%%%%%%
\hrule \vskip 5pt
\noindent
\underbar{Day 5}

\hskip 10pt Exam 3: {\it includes RC discharge circuit performance assessment}
 
\hskip 10pt {\bf Troubleshooting Assessment due:} {\it Loaded voltage divider (question 87)}
 
\hskip 10pt Question 88: Troubleshooting log
 
\hskip 10pt Question 89: Sample troubleshooting assessment grading criteria
 
\vskip 10pt
%%%%%%%%%%%%%%%
\hrule \vskip 5pt
\noindent
\underbar{Practice and challenge problems}

\hskip 10pt Questions: {\it 90 through the end of the worksheet}
 
\vskip 10pt
%%%%%%%%%%%%%%%







\vfil \eject

\centerline{\bf ELTR 105 (DC 2), section 3} \bigskip 
 
\vskip 10pt

\noindent
{\bf Skill standards addressed by this course section}

\vskip 5pt

%%%%%%%%%%%%%%%
\hrule \vskip 10pt
\noindent
\underbar{EIA {\it Raising the Standard; Electronics Technician Skills for Today and Tomorrow}, June 1994}

\vskip 5pt

\medskip
\item{\bf B} {\bf Technical Skills -- DC circuits}
\item{\bf B.03} Demonstrate an understanding of the meaning of and relationships among and between voltage, current, resistance and power in DC circuits.
\item{\bf B.07} Demonstrate an understanding of the physical, electrical characteristics of capacitors and inductors.
\item{\bf B.21} Understand principles and operations of DC RC and RL circuits.
\item{\bf B.22} Fabricate and demonstrate DC RC and RL circuits.
\item{\bf B.23} Troubleshoot and repair DC RC and RL circuits.
\medskip

\vskip 5pt

\medskip
\item{\bf B} {\bf Basic and Practical Skills -- Communicating on the Job}
\item{\bf B.01} Use effective written and other communication skills.  {\it Met by group discussion and completion of labwork.}
\item{\bf B.03} Employ appropriate skills for gathering and retaining information.  {\it Met by research and preparation prior to group discussion.}
\item{\bf B.04} Interpret written, graphic, and oral instructions.  {\it Met by completion of labwork.}
\item{\bf B.06} Use language appropriate to the situation.  {\it Met by group discussion and in explaining completed labwork.}
\item{\bf B.07} Participate in meetings in a positive and constructive manner.  {\it Met by group discussion.}
\item{\bf B.08} Use job-related terminology.  {\it Met by group discussion and in explaining completed labwork.}
\item{\bf B.10} Document work projects, procedures, tests, and equipment failures.  {\it Met by project construction and/or troubleshooting assessments.}
\item{\bf C} {\bf Basic and Practical Skills -- Solving Problems and Critical Thinking}
\item{\bf C.01} Identify the problem.  {\it Met by research and preparation prior to group discussion.}
\item{\bf C.03} Identify available solutions and their impact including evaluating credibility of information, and locating information.  {\it Met by research and preparation prior to group discussion.}
\item{\bf C.07} Organize personal workloads.  {\it Met by daily labwork, preparatory research, and project management.}
\item{\bf C.08} Participate in brainstorming sessions to generate new ideas and solve problems.  {\it Met by group discussion.}
\item{\bf D} {\bf Basic and Practical Skills -- Reading}
\item{\bf D.01} Read and apply various sources of technical information (e.g. manufacturer literature, codes, and regulations).  {\it Met by research and preparation prior to group discussion.}
\item{\bf E} {\bf Basic and Practical Skills -- Proficiency in Mathematics}
\item{\bf E.01} Determine if a solution is reasonable.
\item{\bf E.02} Demonstrate ability to use a simple electronic calculator.
\item{\bf E.05} Solve problems and [sic] make applications involving integers, fractions, decimals, percentages, and ratios using order of operations.
\item{\bf E.06} Translate written and/or verbal statements into mathematical expressions.
\item{\bf E.12} Interpret and use tables, charts, maps, and/or graphs.
\item{\bf E.13} Identify patterns, note trends, and/or draw conclusions from tables, charts, maps, and/or graphs.
\item{\bf E.15} Simplify and solve algebraic expressions and formulas.
\item{\bf E.16} Select and use formulas appropriately.
\item{\bf E.17} Understand and use scientific notation.
\item{\bf E.18} Use properties of exponents and logarithms.
\medskip

%%%%%%%%%%%%%%%




\vfil \eject

\centerline{\bf ELTR 105 (DC 2), section 3} \bigskip 
 
\vskip 10pt

\noindent
{\bf Common areas of confusion for students}

\vskip 5pt

%%%%%%%%%%%%%%%
\hrule \vskip 5pt

\vskip 10pt

\noindent
{\bf Difficult concept: } {\it Rates of change.}

When learning the relationships between voltage and current for inductors and capacitors, one must think in terms of how fast a variable is changing.  The amount of voltage induced across an inductor is proportional to how {\it quickly} the current through it changes, not how strong the current is.  Likewise, the amount of current "through" a capacitor is proportional to how {\it quickly} the voltage across it changes.  This is the first hurdle in calculus: to comprehend what a rate of change is, and it is not obvious.

\vskip 10pt

\noindent
{\bf Common mistake: } {\it Series and parallel relationships for capacitors.}

How inductors add and diminish in series and parallel (respectively) is easy to grasp because it resembles the relationships for resistors.  Capacitors are "backwards" to both resistors and inductors, though, which causes confusion.  

The best way I know how to overcome this confusion is to relate the series or parallel connection of capacitors to changes in physical dimension for a theoretical capacitor, and ask what change in capacitance such a change in dimension will yield.  Connecting capacitors in series may be modeled by increasing the distance between plates of a theoretical capacitor, decreasing capacitance.  Connecting capacitors in parallel is analogous to increasing the plate area of a theoretical capacitor, increasing capacitance.

\vskip 10pt

\noindent
{\bf Difficult concept: } {\it The time-constant equation.}

Many students find the time-constant equation difficult because it involves exponents, particularly exponents of Euler's constant $e$.  This exponent is often expressed as a negative quantity, making it even more difficult to understand.  The single most popular mathematical mistake I see students make with this equation is failing to properly follow algebraic order of operations.  Some students try to overcome this weakness by using calculators which allow parenthetical entries, nesting parentheses in such a way that the calculator performs the proper order of operations.  However, if you don't understand order of operations yourself, you will not know where to properly place the parentheses.  If you have trouble with algebraic order of operations, there is no solution but to invest the necessary time and learn it!

Beyond mathematical errors, though, the most common mistake I see students make with the time constant equation is mis-application.  One version of this equation expresses increasing quantities, while another version expresses decreasing quantities.  You must already know what the variables are going to do in your time-constant circuit before you know which equation to use!  You must also be able to recognize one version of this equation from the other: not by memory, lest you should forget; but by noting what the result of the equation does as time ($t$) increases.  Here again there will be trouble if you are not adept applying algebraic order of operations.

