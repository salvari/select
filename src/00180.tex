
%(BEGIN_QUESTION)
% Copyright 2003, Tony R. Kuphaldt, released under the Creative Commons Attribution License (v 1.0)
% This means you may do almost anything with this work of mine, so long as you give me proper credit

What would happen if a wire having no resistance at all (0 $\Omega$) were connected directly across the terminals of a 6-volt battery?  How much power would be dissipated, according to Joule's Law?  

$$\epsfbox{00180x01.eps}$$

Suppose I short-circuited a 6-volt battery in the manner just described and found that the wire used to make the short-circuit gets warm after just a few seconds of carrying this current.  Does this data agree or disagree with your predictions?

\underbar{file 00180}
%(END_QUESTION)





%(BEGIN_ANSWER)

Calculations based on power equations would suggest either zero watts of power dissipated or infinite power dissipation, depending on which equation you chose to calculate power with.  Yet, the experiment described yields a power dissipation that is neither zero nor infinite.

If you think that the wire used in the experiment is not resistance-less (i.e. it {\it does} have resistance), and that this accounts for the disparity between the predicted and measured amounts of current, you are partially correct.  Realistically, a small piece of wire such as that used in the experiment will have a few tenths of an ohm of resistance.  However, if you re-calculate power with a wire resistance of 0.1 $\Omega$, you will still find a large disparity between your prediction and the actual measured power in this short-circuit.

What is going on here?

%(END_ANSWER)





%(BEGIN_NOTES)

Remind students that short-circuit testing of electrical power sources can be dangerous.  A student of mine once stuffed a 6-volt "lantern" battery in his tool pouch, only to have it discharge smoke an hour later, after the battery terminals had been shorted together by the metal handle of a wrench!

%INDEX% Joule's Law, quantitative

%(END_NOTES)


