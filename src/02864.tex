
%(BEGIN_QUESTION)
% Copyright 2005, Tony R. Kuphaldt, released under the Creative Commons Attribution License (v 1.0)
% This means you may do almost anything with this work of mine, so long as you give me proper credit

Explain why the allowable power supply voltage range for a true TTL (not high-speed CMOS) logic gate is so narrow.  What is the typical range of supply voltages for a true TTL gate, and why can't this type of logic gate operate from a wider range of voltages as CMOS gates can?

\underbar{file 02864}
%(END_QUESTION)





%(BEGIN_ANSWER)

Due to the biasing requirements of its constituent bipolar transistors, TTL circuitry requires a much closer-regulated power supply voltage than CMOS.  I'll let you research what this typical range is!

%(END_ANSWER)





%(BEGIN_NOTES)

Many of the old 74xx and 74LSxx logic circuits are considered obsolete, but may still be found in a lot of operating equipment!  It is not uncommon to have students mistakenly research the datasheets of a newer logic family such as 74HCxx which has different power supply requirements than traditional TTL.  Be prepared to elaborate on the difference(s) between these IC families if and when your students encounter this confusion!

%INDEX% TTL logic, power supply voltage for

%(END_NOTES)


