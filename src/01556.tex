
%(BEGIN_QUESTION)
% Copyright 2003, Tony R. Kuphaldt, released under the Creative Commons Attribution License (v 1.0)
% This means you may do almost anything with this work of mine, so long as you give me proper credit

Suppose a mass is connected to a winch by means of a cable, and a person turns the winch drum to raise the mass off the ground:

$$\epsfbox{01556x01.eps}$$

A physicist would likely look at this scenario as an example of energy exchange: the person turning the drum is expending energy, which in turn is being {\it stored} in the mass in potential form.

Suppose now that the person stops turning the drum and instead engages a brake mechanism on the drum so that it reverses rotation and slowly allows the mass to return to ground level.  Once again, a physicist would view this scenario as an exchange of energy: the mass is now {\it releasing} energy, while the brake mechanism is converting that released energy into heat:

$$\epsfbox{01556x02.eps}$$

In each of the above scenarios, draw arrows depicting directions of two forces: the force that the mass exerts on the drum, and the force that the drum exerts on the mass.  Compare these force directions with the direction of motion in each scenario, and explain how these directions relate to the mass and drum alternately acting as energy {\it source} and energy {\it load}.

\underbar{file 01556}
%(END_QUESTION)





%(BEGIN_ANSWER)

$$\epsfbox{01556x03.eps}$$

Follow-up question: although it may not be obvious, this question closely relates to the exchange of energy between components in electrical circuits!  Explain this analogy.

%(END_ANSWER)





%(BEGIN_NOTES)

Students typically find the concept of energy flow confusing with regard to electrical components.  I try to make this concept clearer by using mechanical analogies, in which force and motion act as analog quantities to voltage and current (or visa-versa).

%INDEX% Source versus load
%INDEX% Load versus source

%(END_NOTES)


