
%(BEGIN_QUESTION)
% Copyright 2005, Tony R. Kuphaldt, released under the Creative Commons Attribution License (v 1.0)
% This means you may do almost anything with this work of mine, so long as you give me proper credit

Each of the following faults will cause this audio amplifier circuit to stop working.  Determine what diagnostic voltage measurement(s) would positively identify each one of the faults.

$$\epsfbox{03740x01.eps}$$

\medskip
\item{$\bullet$} Microphone coil fails open:
\vskip 5pt
\item{$\bullet$} Capacitor $C_1$ fails shorted:
\vskip 5pt
\item{$\bullet$} Resistor $R_1$ fails open:
\vskip 5pt
\item{$\bullet$} Resistor $R_2$ fails open:
\vskip 5pt
\item{$\bullet$} Capacitor $C_3$ fails open:
\vskip 5pt
\item{$\bullet$} Transformer $T_1$ primary winding fails open:
\medskip

\underbar{file 03740}
%(END_QUESTION)





%(BEGIN_ANSWER)

\medskip
\item{$\bullet$} Microphone coil fails open: {\it No AC voltage at all across microphone terminals when sound is present.}
\vskip 5pt
\item{$\bullet$} Capacitor $C_1$ fails shorted: {\it DC voltage present across microphone terminals.}
\vskip 5pt
\item{$\bullet$} Resistor $R_1$ fails open: {\it Full DC supply voltage dropped across $R_1$, no DC voltage dropped across $R_2$ (could indicate a shorted $R_2$ as well -- no way to tell unless a resistance measurement is taken).}
\vskip 5pt
\item{$\bullet$} Resistor $R_2$ fails open: {\it Increased DC voltage drop across $R_2$, decreased DC voltage drop across $R_1$, reasonable transistor DC voltages ($V_E$ 0.7 volts less than $V_B$, $V_C$ as expected based on value of $V_E$ and $R_3$, $R_4$ values) indicate that $Q_1$ is probably not the source of the trouble.}
\vskip 5pt
\item{$\bullet$} Capacitor $C_3$ fails open: {\it Larger-than-normal AC voltage at collector terminal, with no AC voltage present across transformer primary winding.}
\vskip 5pt
\item{$\bullet$} Transformer $T_1$ primary winding fails open: {\it Larger-than-normal AC voltage across transformer primary winding, with no AC voltage across transformer secondary winding.}
\medskip

%(END_ANSWER)





%(BEGIN_NOTES)

The purpose of this question is to approach the domain of circuit troubleshooting from a perspective of knowing what the fault is, rather than only knowing what the symptoms are.  Although this is not necessarily a realistic perspective, it helps students build the foundational knowledge necessary to diagnose a faulted circuit from empirical data.  Questions such as this should be followed (eventually) by other questions asking students to identify likely faults based on measurements.

%INDEX% Troubleshooting, predicting effects of fault in common-base amplifier circuit

%(END_NOTES)


