
%(BEGIN_QUESTION)
% Copyright 2003, Tony R. Kuphaldt, released under the Creative Commons Attribution License (v 1.0)
% This means you may do almost anything with this work of mine, so long as you give me proper credit

Although not a popular design, some power supply circuits are transformerless.  Direct rectification of AC line power is a viable option in some applications:

$$\epsfbox{02016x01.eps}$$

However, this form of AC-to-DC power conversion has some significant limits.  Explain why most power supply circuits utilize a transformer instead of directly rectifying the line power as this circuit does.

\underbar{file 02016}
%(END_QUESTION)





%(BEGIN_ANSWER)

Transformers provide voltage/current ratio transformation, and also electrical isolation between the AC line circuit and the DC circuit.  The issue of isolation is a safety concern, as neither of the output conductors in a non-isolated (direct) rectifier circuit is at the same potential as either of the line conductors.  

\vskip 10pt

Follow-up question: explain in detail how the issue of non-isolation could create a safety hazard if this rectifier circuit were energized by an earth-grounded AC line circuit.

%(END_ANSWER)





%(BEGIN_NOTES)

Many old television sets used such transformerless rectifier circuits to save money, but this meant the metal circuit chassis inside the plastic cover was energized rather than being at ground potential!  Very dangerous for technicians to work on.

%INDEX% Rectifier, transformerless

%(END_NOTES)


