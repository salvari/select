
%(BEGIN_QUESTION)
% Copyright 2003, Tony R. Kuphaldt, released under the Creative Commons Attribution License (v 1.0)
% This means you may do almost anything with this work of mine, so long as you give me proper credit

If a neon bulb is used as a visual indicator of static electricity, it will be noticed that only one of the two metal electrodes inside the glass envelope glows upon discharge.  Why is this?  Why don't both electrodes glow?

\underbar{file 00145}
%(END_QUESTION)





%(BEGIN_ANSWER)

Remember that static electricity is an imbalance of electric charge between two objects.  This imbalance has a definite {\it polarity}: one object is positive while the other is negative.  This means that electrons rush in {\it one direction} when the two objects discharge through the path created by the neon gas inside the lamp.  This unidirectional rush of electrons makes only one of the electrodes glow.

%(END_ANSWER)





%(BEGIN_NOTES)

Of course, students will want to know: {\it which} type of charge makes the electrode glow, positive or negative?  Don't just give them the answer, but challenge them to devise an experiment whereby they could tell for certain which pole of the charge creates a glow.  The answer to this question may have to wait until they learn about DC voltage sources like batteries, where the polarity of the voltage is known.  However, even if they are yet unaware of DC voltage sources, it is a good exercise to have them imagine how they might test the polarity indication of a neon bulb (perhaps by postulating a source of static charge where the polarity is already known).

%INDEX% Neon lamp
%INDEX% Polarity

%(END_NOTES)


