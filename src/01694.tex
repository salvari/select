
%(BEGIN_QUESTION)
% Copyright 2003, Tony R. Kuphaldt, released under the Creative Commons Attribution License (v 1.0)
% This means you may do almost anything with this work of mine, so long as you give me proper credit

$$\epsfbox{01694x01.eps}$$

\underbar{file 01694}
\vfil \eject
%(END_QUESTION)





%(BEGIN_ANSWER)

Use circuit simulation software to verify your predicted and measured parameter values.

%(END_ANSWER)





%(BEGIN_NOTES)

Be sure to specify resistor values for the voltage divider that will show a marked impact when measured with the type of voltmeter you expect your students to use.  If you size the resistors for a modest impact measured with an analog voltmeter (20,000 $\Omega$/Volt), your students may not see much of an impact when using a modern digital voltmeter ($Z_{in} > 10 \hbox{ M}\Omega$).

New students often have a difficult time grasping the main idea of this activity, due to the assumption of the voltmeter's indication always being taken as true.  The purpose of this activity is to shatter that assumption: to teach students that electrical measurements are never truly passive -- rather, they invariably impact the circuit being measured in some way.  Usually, the impact is so small it may be safely ignored.  Here, due to the large resistor values used in the divider circuit, the impact of voltmeter usage on the circuit is non-trivial.

Another aspect of this activity that escapes some students' attention is that the circuit must be analyzed twice: once with the meter connected and once without.  The point here is that the meter {\it becomes a component of the circuit when it is connected across $R_2$, and thus changes all the voltages and currents}.

%INDEX% Assessment, performance-based (Voltmeter loading of a high-resistance DC circuit)

%(END_NOTES)


