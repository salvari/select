
%(BEGIN_QUESTION)
% Copyright 2003, Tony R. Kuphaldt, released under the Creative Commons Attribution License (v 1.0)
% This means you may do almost anything with this work of mine, so long as you give me proper credit

\centerline{{\bf Wire-by-Number project:} (PROJECT NAME HERE)} \bigskip 

\noindent Description:

(DESCRIPTION HERE)

\vskip 10pt

\goodbreak

\noindent Schematic diagram: 

% This is the regular schematic diagram for the circuit
%$$\epsfbox{.eps}$$

\vskip 10pt

\goodbreak

\noindent Components:

\medskip
\item{$\bullet$} Battery (6 V): A1=+ , A2=- (ground)
\item{$\bullet$} Resistor R1, ??? k$\Omega$: A3, A4
\item{$\bullet$}
\item{$\bullet$}
\medskip

% This graphic shows all components in schematic format, 
% but with no wires connecting them together.  Terminal
% points denoted by small (0.7 size), italicized labels.
%$$\epsfbox{wbn0x02.eps}$$

\vskip 10pt

\goodbreak

\noindent Pictorial diagram:

% This graphic shows the terminal strip view of all
% components, with no connecting wires.
$$\epsfbox{wbn0x01.eps}$$

\vskip 10pt

\goodbreak

\noindent Wiring sequence:

\medskip
\item{$\bullet$} A1-A6-B5
\item{$\bullet$} 
\item{$\bullet$} 
\item{$\bullet$} 
\medskip

\vskip 10pt

\goodbreak

\noindent Tasks:

% Special assembly or start-up instructions
(DESCRIPTIVE TEXT GOES HERE)

% Verify correct operation and take measurements
(DESCRIPTIVE TEXT GOES HERE)

Predict the correct values for the following variables, then verify by measuring with your test equipment.  Don't forget to include the proper unit symbols (V, mA, $\Omega$) with your data!  Calculate the percentage error of the measured values using the following formula: Error = ${\hbox{Measured} - \hbox{Predicted} \over \hbox{Predicted}} \times 100 \%$

% No blank lines allowed between lines of an \halign structure!
% I use comments (%) instead, so that TeX doesn't choke.

$$\vbox{\offinterlineskip
\halign{\strut
\vrule \quad\hfil # \ \hfil & 
\vrule \quad\hfil # \ \hfil & 
\vrule \quad\hfil # \ \hfil & 
\vrule \quad\hfil # \ \hfil & 
\vrule \quad\hfil # \ \hfil \vrule \cr 
\noalign{\hrule}
%
{\bf Variable} & {\bf Formula} & {\bf Predicted} & {\bf Measured} & {\bf Error} \cr
\noalign{\hrule}
%
$V_{R1}$  &  $V_{supply}\left( R_1 \over {R_1 + R_2}\right)$  &   &   &   \cr
\noalign{\hrule}
} % End of \halign 
}$$ % End of \vbox

% Introduce specific faults and take measurements
(DESCRIPTIVE TEXT GOES HERE)

% Modify circuit to perform differently
(DESCRIPTIVE TEXT GOES HERE)

Explain why these effects were observed:

$$\epsfbox{wbn_box.eps}$$

\vskip 10pt

\goodbreak

\noindent Questions remaining:

{\it Here is where you write any questions or comments you have about this experiment.}

\vskip 50pt


\underbar{file wbn0}
%(END_QUESTION)





%(BEGIN_ANSWER)

(PROVIDE ONE OR TWO APPROXIMATE FIGURES, OR QUALITATIVE RESULTS, THAT STUDENTS CAN USE TO VERIFY PROPER CIRCUIT OPERATION.)

%(END_ANSWER)





%(BEGIN_NOTES)

(NOTES FOR INSTRUCTOR)

%(END_NOTES)


