
%(BEGIN_QUESTION)
% Copyright 2003, Tony R. Kuphaldt, released under the Creative Commons Attribution License (v 1.0)
% This means you may do almost anything with this work of mine, so long as you give me proper credit

% September 30, 2004: error correction by Sam Cheung -- wrong total resistance value given.  Should be 25 ohms and not 50 ohms.

There is a simple equation that gives the equivalent resistance of two resistances connected in parallel.  Write this equation.

Secondly, apply this two-resistance equation to the solution for total resistance in this three-resistor network:

$$\epsfbox{00298x01.eps}$$

No, this is not a "trick" question!  There {\it is} a way to apply a two-resistance equation to solve for three resistances connected in parallel.

\underbar{file 00298}
%(END_QUESTION)





%(BEGIN_ANSWER)

$R_{total}$ = 25 $\Omega$

\vskip 10pt

In case you are still unsure of how to apply the "two-resistance" parallel equation to this network, I'll give you a hint: this equation gives the {\it equivalent resistance} of two parallel-connected resistors.  Examine this modified version of the original schematic diagram:

$$\epsfbox{00298x02.eps}$$

%(END_ANSWER)





%(BEGIN_NOTES)

And who said technological work never involves creativity?  This question challenges students to apply an equation to a problem that it is not ideally suited for.  The basic principle used in the solution of the problem is very practical.  It involves the {\it substitution} of an equivalent component value in place of multiple components, which is a problem-solving technique widely applied in electrical network analysis, as well as other forms of mathematical analysis.

%INDEX% Parallel resistances

%(END_NOTES)


