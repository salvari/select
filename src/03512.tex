
%(BEGIN_QUESTION)
% Copyright 2005, Tony R. Kuphaldt, released under the Creative Commons Attribution License (v 1.0)
% This means you may do almost anything with this work of mine, so long as you give me proper credit

\centerline{\bf Animation: Electromagnetic induction}

\vskip 10pt

{\it This question consists of a series of images (one per page) that form an animation.  Flip the pages with your fingers to view this animation (or click on the "next" button on your viewer) frame-by-frame.}

\vskip 10pt

The following animation shows how voltage is induced in a coil of wire as a magnet is brought near to it and then drawn away.  Carefully note the direction of the current through the resistor and the polarity of the induced magnetic field (Lenz's Law).

\vfil \eject
$$\epsfbox{03512x01.eps}$$

\vfil \eject
$$\epsfbox{03512x02.eps}$$

\vfil \eject
$$\epsfbox{03512x03.eps}$$

\vfil \eject
$$\epsfbox{03512x04.eps}$$

\vfil \eject
$$\epsfbox{03512x05.eps}$$

\vfil \eject
$$\epsfbox{03512x06.eps}$$

\vfil \eject
$$\epsfbox{03512x07.eps}$$

\vfil \eject
$$\epsfbox{03512x08.eps}$$

\vfil \eject
$$\epsfbox{03512x09.eps}$$

\vfil \eject
$$\epsfbox{03512x10.eps}$$

\vfil \eject
$$\epsfbox{03512x11.eps}$$

\vfil \eject
$$\epsfbox{03512x12.eps}$$

\vfil \eject
$$\epsfbox{03512x13.eps}$$

\vfil \eject
$$\epsfbox{03512x14.eps}$$

\underbar{file 03512}

\vfil \eject

%(END_QUESTION)





%(BEGIN_ANSWER)

The relative strengths of the induced current and induced magnetic field (shown by the arrow lengths and pole letter size, respectively) correspond to the velocity of the permanent magnet's motion.

%(END_ANSWER)





%(BEGIN_NOTES)

The main purpose of this animation is to show Lenz's Law at work in the polarity of the coil's induced magnetic field.  The relative magnitudes of the load voltage and current are shown with the assumption that the magnetic flux change ($d\Phi \over dt$) is directly related to magnet velocity and not magnet position.  In real life, the magnetic flux change would grow larger as the magnet drew nearer to the coil, given the concentration of magnetic flux lines at the magnet ends.  I have elected not to show that in order that the viewer may simply observe Lenz's Law with no other distractions.

%INDEX% Animation, electromagnetic induction

%(END_NOTES)


