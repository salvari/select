
%(BEGIN_QUESTION)
% Copyright 2006, Tony R. Kuphaldt, released under the Creative Commons Attribution License (v 1.0)
% This means you may do almost anything with this work of mine, so long as you give me proper credit

The basic circuit for an unamplified {\it capacitor microphone} looks like this:

$$\epsfbox{03814x01.eps}$$

Vibrations in the air cause one metal plate to vibrate next to a stationary metal plate, rapidly changing the capacitance between the two plates.  Explain how this simple circuit generates an AC voltage from this rapidly-varying capacitance, employing the mathematical relationship between charge ($Q$), capacitance ($C$) and voltage ($V$).

\underbar{file 03814}
%(END_QUESTION)





%(BEGIN_ANSWER)

Given the equation $Q = CV$, we can see that changes in capacitance will either result in changes of stored charge for a given voltage ($Q$ changes with $V$ constant), changes in voltage for a given charge ($V$ changes with $Q$ constant), or changes in both $Q$ and $V$.  Remember that changes in electric charge over time ($dQ \over dt$) is the mathematical definition of electric current ($I$).

\vskip 10pt

Follow-up question: capacitor microphones are almost never used without some form of amplification.  The voltage signal generated by the circuit shown in the question is simply too weak to use on its own.  Explain why the signal is so weak.

%(END_ANSWER)





%(BEGIN_NOTES)

Ask your students where they located this information.  it is important for students to learn where and how to research for needed information.

%INDEX% Capacitor microphone
%INDEX% Microphone, capacitor

%(END_NOTES)


