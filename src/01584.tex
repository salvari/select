
%(BEGIN_QUESTION)
% Copyright 2003, Tony R. Kuphaldt, released under the Creative Commons Attribution License (v 1.0)
% This means you may do almost anything with this work of mine, so long as you give me proper credit

Here are a few good steps to take prior to applying any specific troubleshooting strategies to a malfunctioning amplifier circuit:

\medskip
\item{$\bullet$} Measure the output signal with an oscilloscope.
\item{$\bullet$} Determine if the amplifier is receiving a good input signal.
\item{$\bullet$} Check to see that the amplifier is receiving good-quality power.
\medskip

Explain why taking these simple steps may save a lot of time in the troubleshooting process.  For example, why bother checking the amplifier's output signal if you already know it isn't outputting what it's supposed to?  What, exactly, constitutes "good-quality" power for an amplifier circuit?

\underbar{file 01584}
%(END_QUESTION)





%(BEGIN_ANSWER)

It is usually a good idea to verify the exact nature of the malfunction before proceeding with troubleshooting strategies, even if someone has already informed you of the problem.  Seeing the malfunction with your own eyes may illuminate the problem better than if you simply acted on someone else's description, or worse yet your own assumptions.

The rationale for checking the input signal should be easy to understand.  I'll let you answer this one!

"Good-quality" power consists of DC within the proper voltage range of the amplifier circuit, with negligible ripple voltage.

\vskip 10pt

Follow-up question \#1: suppose you discover that the "faulty" amplifier is in fact {\it not} receiving any input signal at all?  Does this test exonerate the amplifier itself?  How would might you simulate a proper input signal for the amplifier, for the purposes of testing it?

\vskip 10pt

Follow-up question \#2: explain how to measure power supply ripple voltage, using only a digital multimeter.  How would you measure ripple using an oscilloscope?

%(END_ANSWER)





%(BEGIN_NOTES)

In my own experience I have found these steps to be valuable time-savers prior to beginning any formal troubleshooting process.  In general terms, {\it check for output}, {\it check for input}, and {\it check for power}.  

New technicians are often surprised at how often complex problems may be caused by something as simple as "dirty" power.  Since it only takes a few moments to check, and can lead to a wide range of problems, it is not wasted effort.

%INDEX% Troubleshooting, recommended first steps for amplifier circuits

%(END_NOTES)


