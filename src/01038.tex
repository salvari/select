
%(BEGIN_QUESTION)
% Copyright 2003, Tony R. Kuphaldt, released under the Creative Commons Attribution License (v 1.0)
% This means you may do almost anything with this work of mine, so long as you give me proper credit

Suppose you were handed a black box with two metal terminals on one side, for attaching electrical (wire) connections.  Inside this box, you were told, was a current source (an ideal current source connected in parallel with a resistance):

$$\epsfbox{01038x01.eps}$$

How would you experimentally determine the current of the ideal current source inside this box, and how would you experimentally determine the resistance of the parallel resistor?  By "experimentally," I mean determine current and resistance using actual test equipment rather than assuming certain component values (remember, this "black box" is sealed, so you cannot look inside!).

\underbar{file 01038}
%(END_QUESTION)





%(BEGIN_ANSWER)

Measure the open-circuit voltage between the two terminals, and then measure the short-circuit current.  The current source's value is measured, while the resistor's value is calculated using Ohm's Law.

%(END_ANSWER)





%(BEGIN_NOTES)

Ask your students how they would apply this technique to an abstract circuit problem, to reduce a complex network of sources and resistances to a single current source and single parallel resistance (Norton equivalent).

%INDEX% Norton's theorem, experimentally determining Norton current and Norton resistance of network

%(END_NOTES)


