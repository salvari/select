
%(BEGIN_QUESTION)
% Copyright 2003, Tony R. Kuphaldt, released under the Creative Commons Attribution License (v 1.0)
% This means you may do almost anything with this work of mine, so long as you give me proper credit

In this graph you will see three different load lines plotted, representing three different values of load resistance in the amplifier circuit:

$$\epsfbox{00955x01.eps}$$

Which one of the three load lines represents the largest value of load resistance ($R_{load}$)?  Which of the three load lines will result in the greatest amount of change in voltage drop across the transistor ($\Delta V_{CE}$) for any given amount of base current change ($\Delta I_B$)?  What do these relationships indicate about the load resistor's effect on the amplifier circuit's {\it voltage gain}?

\underbar{file 00955}
%(END_QUESTION)





%(BEGIN_ANSWER)

The load line closest to horizontal represents the largest value of load resistance, and it also represents the condition in which $V_{CE}$ will vary the most for any given amount of base current (input signal) change.

%(END_ANSWER)





%(BEGIN_NOTES)

This question challenges students to relate load resistor values to load lines, and both to the practical measure of voltage gain in a simple amplifier circuit.  As an illustration, ask the students to analyze changes in the circuit for an input signal that varies between 5 $\mu$A and 10 $\mu$A for the three different load resistor values.  The difference in $\Delta V_{CE}$ should be very evident!

%(END_NOTES)


