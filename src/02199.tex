
%(BEGIN_QUESTION)
% Copyright 2004, Tony R. Kuphaldt, released under the Creative Commons Attribution License (v 1.0)
% This means you may do almost anything with this work of mine, so long as you give me proper credit

This is a schematic diagram of a {\it Y-connected} three-phase generator (with the rotor winding shown):

$$\epsfbox{02199x01.eps}$$

How much AC voltage will appear between any two of the lines ($V_{AB}$, $V_{BC}$, or $V_{AC}$) if each stator coil inside the alternator outputs 277 volts?  Draw a phasor diagram showing how the phase (winding) and line voltages relate.

\underbar{file 02199}
%(END_QUESTION)





%(BEGIN_ANSWER)

Phase voltage = 277 volts AC (given)

Line voltage = $V_{AB} = V_{BC} = V_{AC} =$ 480 volts AC

$$\epsfbox{02199x02.eps}$$

\vskip 10pt

Follow-up question \#1: what is the ratio between the line and phase voltage magnitudes in a Y-connected three-phase system?

\vskip 10pt

Follow-up question \#2: what would happen to the output of this alternator if the rotor winding were to fail open?  Bear in mind that the rotor winding is typically energized with DC through a pair of brushes and slip rings from an external source, the current through this winding being used to control voltage output of the alternator's three-phase "stator" windings.

%(END_ANSWER)





%(BEGIN_NOTES)

Students will quickly discover that $\sqrt{3}$ is the "magic number" for practically all balanced three-phase circuit calculations!

It should be noted that although the {\it magnitudes} of $V_{AB}$, $V_{BC}$, and $V_{AC}$ are equal, their phasor angles are most definitely not.  Therefore,

$$V_{AB} = V_{BC} = V_{AC} \hbox{\hskip 20pt Scalar values equal}$$

$${\bf V_{AB}} \neq {\bf V_{BC}} \neq {\bf V_{AC}} \hbox{\hskip 20pt Phasors unequal}$$

%INDEX% Alternator, three-phase
%INDEX% Line voltage vs. phase voltage
%INDEX% Phase voltage vs. line voltage
%INDEX% Y connected source, three phase

%(END_NOTES)


