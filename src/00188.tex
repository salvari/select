
%(BEGIN_QUESTION)
% Copyright 2003, Tony R. Kuphaldt, released under the Creative Commons Attribution License (v 1.0)
% This means you may do almost anything with this work of mine, so long as you give me proper credit

The storage of electric charge in a capacitor is often likened to the storage of water in a vessel:

$$\epsfbox{00188x01.eps}$$

Complete this analogy, relating the electrical quantities of charge ($Q$), voltage ($E$ or $V$), and capacitance ($C$) to the quantities of water height, water volume, and vessel dimensions.

\underbar{file 00188}
%(END_QUESTION)





%(BEGIN_ANSWER)

Electrical charge $\equiv$ Water volume

Voltage $\equiv$ Height of water column in vessel

Capacitance $\equiv$ Area of vessel, measured on a cross-section with a horizontal plane

%(END_ANSWER)





%(BEGIN_NOTES)

Many students find this a helpful analogy of capacitor action.  But it helps even more if students work together to {\it build} the analogy, and to truly understand it.

Perform some "thought experiments" with vessels of different size, relating the outcomes to charge storage in capacitors of different size.

%INDEX% Capacitance, analogous to water storage

%(END_NOTES)


