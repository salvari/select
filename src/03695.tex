
%(BEGIN_QUESTION)
% Copyright 2005, Tony R. Kuphaldt, released under the Creative Commons Attribution License (v 1.0)
% This means you may do almost anything with this work of mine, so long as you give me proper credit

An electronics technician connects the input of a spectrum analyzer to the secondary winding of an AC power transformer, plugged into a power receptacle.  He sets the spectrum analyzer to show 60 Hz as the fundamental frequency, expecting to see the following display:

$$\epsfbox{03695x01.eps}$$

Instead, however, the spectrum analyzer shows more than just a single peak at the fundamental:

$$\epsfbox{03695x02.eps}$$

Explain what this pattern means, in practical terms.  Why is this power system's harmonic signature different from what the technician expected to see?

\underbar{file 03695}
%(END_QUESTION)





%(BEGIN_ANSWER)

What this pattern means is the power-line voltage waveform is distorted from what should be a perfect sine-wave shape.

%(END_ANSWER)





%(BEGIN_NOTES)

Note to your students that this is quite typical for modern power systems, due to the prevalence of switching power supply circuits and other "non-linear" electrical loads.  The presence of harmonic frequencies in significant quantity can cause severe problems for power systems, including transformer overheating, motor overheating, overloaded neutral conductors (especially in three-phase, four-wire "Wye" systems), and excessive currents through power-factor correction capacitors.

%INDEX% Spectrum analysis, of power-line voltage

%(END_NOTES)


