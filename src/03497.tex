
%(BEGIN_QUESTION)
% Copyright 2005, Tony R. Kuphaldt, released under the Creative Commons Attribution License (v 1.0)
% This means you may do almost anything with this work of mine, so long as you give me proper credit

The formula for calculating the reluctance ($\Re$) of an air-core wire coil ("solenoid") is as follows:

$$\Re = {l \over {\mu_0 A}}$$

\noindent
Where,

$l$ = linear length of coil in meters (m)

$A$ = cross-sectional area of coil "throat" in square meters (m$^{2}$)

$\mu_0$ = permeability of free space = $4 \pi$ $\times$ $10^{-7}$ (T$\cdot$m/A)

\vskip 10pt

Using this formula and the Rowland's Law formula, calculate the amount of magnetic flux ($\Phi$) produced in the throat of an air-core solenoid with 250 turns of wire, a length of 0.2 meters, a cross-sectional area of 6.5 $\times$ $10^{-4}$ square meters, and a coil current of 5 amps:

$$\epsfbox{03497x01.eps}$$

\underbar{file 03497}
%(END_QUESTION)





%(BEGIN_ANSWER)

$\Phi$ = 5.105 $\mu$Wb

%(END_ANSWER)





%(BEGIN_NOTES)

With all the information given, this is nothing more than an exercise in calculation.  However, it is nice for students to have the air-core solenoid reluctance formula handy for other calculations, which is the main point of this question.

%INDEX% Rowland's Law (magnetism) calculation
%INDEX% Solenoid coil (air-core), calculating the reluctance of 

%(END_NOTES)


