
%(BEGIN_QUESTION)
% Copyright 2006, Tony R. Kuphaldt, released under the Creative Commons Attribution License (v 1.0)
% This means you may do almost anything with this work of mine, so long as you give me proper credit

Analog-to-digital converter circuits (ADC) are usually equipped with analog low-pass filters to pre-condition the signal prior to digitization.  This prevents signals with frequencies greater than the sampling rate from being seen by the ADC, causing a detrimental effect called {\it aliasing}.  These analog pre-filters are thus known as {\it anti-aliasing filters}.

Determine which of the following Sallen-Key active filters is of the correct type to be used as an anti-aliasing filter:

$$\epsfbox{04039x01.eps}$$

$$\epsfbox{04039x02.eps}$$

\underbar{file 04039}
%(END_QUESTION)





%(BEGIN_ANSWER)

The low-pass Sallen-Key filter, of course!  What's the matter?  You're not laughing at my answer.  What I'm doing here is asking you to some research on Sallen-Key filters to confirm your qualitative analysis.  And yes, I do expect you to be able to {\it figure out} which of the two filters is low-pass based on your knowledge of capacitors and op-amps, not just look up the answer in an op-amp reference book!

%(END_ANSWER)





%(BEGIN_NOTES)

Discuss with your students various ways of identifying active filter types.  What clues are present in these two circuits to reveal their filtering characteristics?

%INDEX% Anti-aliasing filter, ADC
%INDEX% Filter, anti-aliasing (for ADC)

%(END_NOTES)


