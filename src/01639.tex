
%(BEGIN_QUESTION)
% Copyright 2003, Tony R. Kuphaldt, released under the Creative Commons Attribution License (v 1.0)
% This means you may do almost anything with this work of mine, so long as you give me proper credit

$$\epsfbox{01639x01.eps}$$

\underbar{file 01639}
\vfil \eject
%(END_QUESTION)





%(BEGIN_ANSWER)

The ohmmeter's indication is the "final word" on resistance.

%(END_ANSWER)





%(BEGIN_NOTES)

The purpose of this exercise is to make absolutely sure students can accurately measure resistance with a multimeter, and also that they can interpret resistor color codes.  Select resistors that span a wide range, from less than 10 ohms to millions of ohms.

I recommend the following resistor color codes for students to try (all 5\% tolerance):

\medskip
\item{$\bullet$} Blk, Brn, Grn, Gld
\item{$\bullet$} Brn, Red, Brn, Gld
\item{$\bullet$} Blu, Gry, Blk, Gld
\item{$\bullet$} Red, Red, Org, Gld
\item{$\bullet$} Brn, Grn, Yel, Gld
\item{$\bullet$} Org, Org, Red, Gld
\medskip

A good extension of this assessment is to have students demonstrate competency using both digital and analog multimeters!

%INDEX% Assessment, performance-based (Ohmmeter usage)

%(END_NOTES)


