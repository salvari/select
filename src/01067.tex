
%(BEGIN_QUESTION)
% Copyright 2003, Tony R. Kuphaldt, released under the Creative Commons Attribution License (v 1.0)
% This means you may do almost anything with this work of mine, so long as you give me proper credit

An important consideration when working around circuits containing MOSFETs is {\it electrostatic discharge}, or {\it ESD}.  Describe what this phenomenon is, and why it is an important consideration for MOSFET circuits.

\underbar{file 01067}
%(END_QUESTION)





%(BEGIN_ANSWER)

"Electrostatic discharge" is the application of very high voltages to circuit components as a result of contact or proximity with an electrically charged body, such as a human being.  The high voltages exhibited by static electricity are very damaging to MOSFETs.  I'll let you research why!

%(END_ANSWER)





%(BEGIN_NOTES)

Be sure to ask students to explain the mechanism of transistor damage resulting from ESD, and to discuss the sheer magnitude of static voltages typically generated in dry-air conditions.  If you have any microphotographs of IC damage from ESD, present a few of them during discussion time for your students' viewing pleasure.

%INDEX% Electrostatic discharge
%INDEX% ESD (Electro-Static Discharge)

%(END_NOTES)


