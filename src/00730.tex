
%(BEGIN_QUESTION)
% Copyright 2003, Tony R. Kuphaldt, released under the Creative Commons Attribution License (v 1.0)
% This means you may do almost anything with this work of mine, so long as you give me proper credit

Shunt resistors, being very low in resistance, are usually made from relatively large masses of metal.  Their precise resistance is calibrated through a process known as {\it trimming}, where a technician takes a metal file and "trims" metal from the shunt conductor until the resistance reaches its correct value.  This, of course, only works if the shunt resistor is intentionally manufactured with a resistance that is too low.  Like the old carpenter's joke goes, "I cut the board twice and it's still too short!"

Being that shunt resistors have such incredibly low resistance values, how do we measure the resistance of a shunt with high accuracy during the "trimming" process?  The resistance of a shunt is far too low for an average handheld or even benchtop ohmmeter to measure with precision, and specialized low-resistance ohmmeters such as the {\it Kelvin Double Bridge} are quite expensive.  If you were given the task of trimming a shunt resistor for use in an ammeter, and you only possessed average pieces of test equipment, how could you do it?

\underbar{file 00730}
%(END_QUESTION)





%(BEGIN_ANSWER)

Build the ammeter and trim the shunt resistor in-place, with a calibrated amount of current through it.

%(END_ANSWER)





%(BEGIN_NOTES)

The answer to this question is deceptively simple, yet extremely practical.  Sure, it would be nice to have the best possible test and calibration equipment available to us at any time in our own laboratory, but we must be realistic.  It is extremely important for your students that they engage in discussion on problems like this from a practical perspective.  It is your task and your privilege as their instructor to bring your own experience into such discussions and challenge students with realistic obstacles to their (often) idealistic expectations.

%(END_NOTES)


