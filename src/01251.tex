
%(BEGIN_QUESTION)
% Copyright 2003, Tony R. Kuphaldt, released under the Creative Commons Attribution License (v 1.0)
% This means you may do almost anything with this work of mine, so long as you give me proper credit

In digital electronic circuitry, binary bit values of 0 or 1 are represented in the form of voltages: {\it low} and {\it high} logic states, respectively.  Suppose you need to manually "input" a logic state to one of the pins of a logic circuit.  In the following illustration, the logic circuit (shown as an indistinct, shaded rectangle) is already supplied with DC power (+V and ground), and its output is indicated by an LED.  All it requires is an input from you:

$$\epsfbox{01251x01.eps}$$

Complete this schematic diagram by including a switch in the drawing, such that in each of its two positions, a definite "low" or "high" logic state will be sensed by the circuit's input terminal.

\underbar{file 01251}
%(END_QUESTION)





%(BEGIN_ANSWER)

$$\epsfbox{01251x02.eps}$$

%(END_ANSWER)





%(BEGIN_NOTES)

While this may seem to be a very elementary question, it is important to get students to realize just what logic states are, in their physical representations.  Too often I read textbooks and other digital logic tutorials that leap the student immediately into a boolean analysis of gate circuits, with everything operating off of abstract 0's and 1's (or "low's" and "high's"), without properly introducing the electrical nature of these states to students.  Remember, your students should be quite familiar with electrical circuits, including analog transistor and op-amp circuits, by now, so beginning their study of gates from an electrical perspective should be natural for them.  Only after they realize how logic states are represented by voltages do I recommend discussing gates and truth tables.

%INDEX% Logic states (digital)

%(END_NOTES)


