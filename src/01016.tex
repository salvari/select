
%(BEGIN_QUESTION)
% Copyright 2003, Tony R. Kuphaldt, released under the Creative Commons Attribution License (v 1.0)
% This means you may do almost anything with this work of mine, so long as you give me proper credit

Plot the transfer function ($V_{out}$ versus $V_{in}$) for this opamp circuit, and explain how the circuit works:
 
$$\epsfbox{01016x01.eps}$$

What type of mathematical function is represented by this circuit?

\underbar{file 01016}
%(END_QUESTION)





%(BEGIN_ANSWER)

This circuit represents a {\it logarithmic} function ($y \propto \ln x$):

$$\epsfbox{01016x02.eps}$$

%(END_ANSWER)





%(BEGIN_NOTES)

The direction of the transfer function curve may surprise some students.  Ask them why the curve goes down (negative) for increasingly positive input voltages.

Ask your students how they obtained this transfer function curve.  There are conceptual methods for obtaining it, as well as algebraic methods.  It would be interesting to compare more than one of these methods in a class discussion, and have students gain insight from each others' methods.

%INDEX% Logarithm extractor, nonlinear opamp circuit

%(END_NOTES)


