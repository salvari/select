
%(BEGIN_QUESTION)
% Copyright 2003, Tony R. Kuphaldt, released under the Creative Commons Attribution License (v 1.0)
% This means you may do almost anything with this work of mine, so long as you give me proper credit

Draw the direction of current in this circuit, and also identify the polarity of the voltage across the battery and across the resistor.  Then, compare the battery's polarity with the direction of current through it, and the resistor's polarity with the direction of current through it.  

$$\epsfbox{01555x01.eps}$$

What do you notice about the relationship between voltage polarity and current direction for these two different types of components?  Identify the fundamental distinction between these two components that causes them to behave differently.

\underbar{file 01555}
%(END_QUESTION)





%(BEGIN_ANSWER)

Here I show the answer in two different forms: current shown as {\it electron flow} (left) and current shown as {\it conventional flow} (right).

$$\epsfbox{01555x02.eps}$$

Whichever notation you choose to follow in your analysis of circuits, the understanding should be the same: the reason voltage polarities across the resistor and battery differ despite the same direction of current through both is the flow of power.  The battery acts as a {\it source}, while the resistor acts as a {\it load}.

%(END_ANSWER)





%(BEGIN_NOTES)

This type of distinction is very important in the study of physics as well, where one must determine whether a mechanical system is {\it doing work} or whether {\it work is being done on it}.  A clear understanding of the relationship between voltage polarity and current direction for sources and loads is very important for students to have before they study reactive devices such as inductors and capacitors!

%INDEX% Polarity of voltage drop, with reference to current direction
%INDEX% Source versus load
%INDEX% Load versus source

%(END_NOTES)


