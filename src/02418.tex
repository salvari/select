
%(BEGIN_QUESTION)
% Copyright 2005, Tony R. Kuphaldt, released under the Creative Commons Attribution License (v 1.0)
% This means you may do almost anything with this work of mine, so long as you give me proper credit

Shown here is the schematic diagram for a simple automotive ignition system, to produce pulses of high voltage sufficient to energize spark plugs in an engine:

$$\epsfbox{02418x01.eps}$$

An engineer decides to replace the BJT with a MOSFET, and arrives at the following circuit design:

$$\epsfbox{02418x02.eps}$$

Explain how this revised circuit works.  When does the MOSFET conduct current, when the point contacts are open or closed?  How does this compare to the working of the previous (BJT) circuit?  What purpose does the 10 k$\Omega$ resistor serve?

\underbar{file 02418}
%(END_QUESTION)





%(BEGIN_ANSWER)

The MOSFET conducts current when the point contacts are open, which is opposite that of the BJT.  I'll let you figure out what the purpose of the resistor is!

%(END_ANSWER)





%(BEGIN_NOTES)

If this were a real ignition system, the timing would have to be adjusted, as the spark will now be produced every time the points {\it close} rather than every time the points {\it open} as it did before (with the BJT).  Discuss the operation of this circuit with your students, asking them to explain how they know the MOSFET's status (and the BJT's status, for that matter).

%INDEX% BJT versus MOSFET
%INDEX% MOSFET versus BJT

%(END_NOTES)


