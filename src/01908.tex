
%(BEGIN_QUESTION)
% Copyright 2003, Tony R. Kuphaldt, released under the Creative Commons Attribution License (v 1.0)
% This means you may do almost anything with this work of mine, so long as you give me proper credit

Suppose an oscilloscope has been set up to display a triangle wave, with the horizontal position control turned clockwise until the left-hand edge of the waveform is visible:

$$\epsfbox{01908x01.eps}$$

Then, the technician changes the slope control, changing it from "increasing" to "decreasing":

$$\epsfbox{01908x02.eps}$$

Draw the waveform's new appearance on the oscilloscope screen, with the slope control reversed.

\underbar{file 01908}
%(END_QUESTION)





%(BEGIN_ANSWER)

The waveform will begin at the same voltage level, only on the "down" side instead of on the "up" side:

$$\epsfbox{01908x03.eps}$$

%(END_ANSWER)





%(BEGIN_NOTES)

There is nothing special about a triangle wave here.  To be perfectly honest, it was the easiest waveform for me to draw which had a sloping edge to trigger on!

By the way, for students to really understand how triggering works, it is important for them to spend time "playing" with an oscilloscope and a signal generator trying things like this.  There is only so much one can learn about the operation of a machine by reading!

%INDEX% Oscilloscope, trigger slope control function

%(END_NOTES)


