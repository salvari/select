
%(BEGIN_QUESTION)
% Copyright 2003, Tony R. Kuphaldt, released under the Creative Commons Attribution License (v 1.0)
% This means you may do almost anything with this work of mine, so long as you give me proper credit

NAND and NOR gates both have the interesting property of {\it universality}.  That is, it is possible to create any logic function at all, using nothing but multiple gates of either type.  The key to doing this is DeMorgan's Theorem, because it shows us how properly applied inversion is able to convert between the two fundamental logic gate types (from AND to OR, and visa-versa).

Using this principle, convert the following gate circuit diagram into one built exclusively of NAND gates (no Boolean simplification, please).  Then, do the same using nothing but NOR gates:

\vskip 10pt

$$\epsfbox{01322x01.eps}$$

\vskip 60pt

\underbar{file 01322}
%(END_QUESTION)





%(BEGIN_ANSWER)

Using nothing but NAND gates:

$$\epsfbox{01322x02.eps}$$

\vskip 10pt

Using nothing but NOR gates:

$$\epsfbox{01322x03.eps}$$

%(END_ANSWER)





%(BEGIN_NOTES)

Gate universality is not just an esoteric property of logic gates.  There are (or at least were) entire logic systems made up of nothing but one of these gate types!  I once worked with a fellow who maintained gas turbine control systems for crude oil pumping stations.  He told me that he has seen one manufacturer's turbine control system where the discrete logic was nothing but NAND gates, and another manufacturer's system where the logic was nothing but NOR gates.  Needless to say, it was a bit of a challenge for him to transition between the two manufacturers' systems, since it was natural for him to "get used to" one of the gate types after doing troubleshooting work on either type of system.

%INDEX% NAND gate universality
%INDEX% NOR gate universality

%(END_NOTES)


