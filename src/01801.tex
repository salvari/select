
%(BEGIN_QUESTION)
% Copyright 2003, Tony R. Kuphaldt, released under the Creative Commons Attribution License (v 1.0)
% This means you may do almost anything with this work of mine, so long as you give me proper credit

Electromechanical relays are extremely useful devices, but they have their idiosyncrasies.  One of them is a consequence of the fact that the relay's coil acts as an inductor, storing energy in its magnetic field.

In the following circuit, a relay is used to switch power to a large electric motor, while being controlled by a light-duty pushbutton switch:

$$\epsfbox{01801x01.eps}$$

The problem here is every time the pushbutton switch is released, the contacts arc significantly.  This happens because the inductor releases all of its stored energy as a high-voltage spark across the opening contacts.  {\it Inductive kickback} is the phrase commonly used to describe this effect, and over time it will prematurely destroy the switch.

An electronics technician understands the nature of the problem and proposes a solution.  By connecting a light bulb in parallel with the relay coil, the coil's energy now has a safer place to dissipate whenever the pushbutton switch contacts open:

$$\epsfbox{01801x02.eps}$$

Instead of that stored energy manifesting itself as a high-voltage arc at the switch, it powers the light bulb for a brief time after the switch opens, dissipating in a non-destructive manner.

However, the addition of the light bulb introduces a new, unexpected problem to the circuit.  Now, when the pushbutton switch is released, the relay delays for a fraction of a second before disengaging.  This causes the motor to "overshoot" its position instead of stopping when it is supposed to.

Explain why this happens, with reference to the LR time constant of this circuit before and after the addition of the lamp.

\underbar{file 01801}
%(END_QUESTION)





%(BEGIN_ANSWER)

Adding the light bulb to this circuit {\it increased} the circuit's time constant, causing the coil's inductance to discharge at a slower rate.

%(END_ANSWER)





%(BEGIN_NOTES)

This answer is rather minimal, if not obvious.  Of course, if the relay takes longer to de-energize, and we were told this has something to do with time constants, then the time constant of the circuit must have increased.  What is not so obvious is {\it why} $\tau$ increased.  Discuss this with your students, and see what conclusions they reached.

%INDEX% Inductive "kickback"

%(END_NOTES)


