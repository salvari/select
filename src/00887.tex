
%(BEGIN_QUESTION)
% Copyright 2003, Tony R. Kuphaldt, released under the Creative Commons Attribution License (v 1.0)
% This means you may do almost anything with this work of mine, so long as you give me proper credit

Suppose this differential-pair circuit was perfectly balanced.  In this condition, how much voltage would be expected between the two transistors' collector terminals?

$$\epsfbox{00887x01.eps}$$

What would happen to this differential voltage ($V_{diff}$) if transistor Q2 were to increase in temperature, while transistor Q1 remained at the same temperature?  Explain your answer.

\underbar{file 00887}
%(END_QUESTION)





%(BEGIN_ANSWER)

In a balanced condition, $V_{diff} =$ 0 volts.  If Q2 heats up and Q1 does not, a differential voltage will develop between the two collector terminals, with Q1's collector being the positive and Q2's collector being the negative:

$$\epsfbox{00887x02.eps}$$

\vskip 10pt

Follow-up question: what does this phenomenon mean with regard to the stability of differential-pair transistor circuits under different operating conditions?  What might be a good way to maximize circuit stability over a wide range of operating temperatures?

%(END_ANSWER)





%(BEGIN_NOTES)

Fundamentally, the issue in this question is what happens to a transistor when it heats up, but the electrical supply (power and input signal) parameters do not change.  Ask your students to relate this phenomenon to the behavior of other PN junction devices, such as diodes.

%INDEX% Differential pair circuit
%INDEX% Temperature, effects on BJT amplifiers

%(END_NOTES)


