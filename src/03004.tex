
%(BEGIN_QUESTION)
% Copyright 2005, Tony R. Kuphaldt, released under the Creative Commons Attribution License (v 1.0)
% This means you may do almost anything with this work of mine, so long as you give me proper credit

Here is the block symbol for the 74HC147 decimal-to-BCD encoder:

$$\epsfbox{03004x01.eps}$$

Describe what sort of input conditions would be required to make it generate the code for the number 7, and how that numerical quantity would be represented on the output ($Y$) lines.

\underbar{file 03004}
%(END_QUESTION)





%(BEGIN_ANSWER)

To encode the number 7, you would have to make the $I_7$ line low (connect it to ground).  This would make the outputs assume the following states:

\medskip
\item{$\bullet$} $Y_0$ = low
\item{$\bullet$} $Y_1$ = low
\item{$\bullet$} $Y_2$ = low
\item{$\bullet$} $Y_3$ = high
\medskip

%(END_ANSWER)





%(BEGIN_NOTES)

This is a good review of active-low inputs, how they are drawn on schematics, and what they mean in practical digital circuits.  A potentially confusing aspect of this question is the presence of active-low inputs {\it and} outputs, but it is well worth your time to review with students, because like it or not there are many ICs with active-low I/O lines.

%INDEX% 74HC147 encoder IC
%INDEX% Encoder, 74HC147 IC

%(END_NOTES)


