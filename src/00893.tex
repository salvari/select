
%(BEGIN_QUESTION)
% Copyright 2003, Tony R. Kuphaldt, released under the Creative Commons Attribution License (v 1.0)
% This means you may do almost anything with this work of mine, so long as you give me proper credit

Install a potentiometer in this circuit so that the regulated output voltage of this power supply becomes adjustable:

$$\epsfbox{00893x01.eps}$$

Challenge: leave the potentiometer symbol in its place, and make the necessary wire connections between it and the rest of the circuit!

\underbar{file 00893}
%(END_QUESTION)





%(BEGIN_ANSWER)

$$\epsfbox{00893x02.eps}$$

\vskip 10pt

Challenge question: for any given amount of load current, what voltage setting will cause the transistor to dissipate the most heat energy, low, medium, or high?

%(END_ANSWER)





%(BEGIN_NOTES)

Some students may choose to place the potentiometer on the output of the power supply (connecting to the transistor's emitter terminal).  While this will work, technically, it is not a good solution because the load current will be severely limited by the potentiometer's resistance.  Discuss with your students why the circuit drawn in the answer is more practical.

The answer to the challenge question is nonintuitive, but it makes sense once you determine what variables affect transistor power dissipation (emitter current, and $V_{CE}$).

%INDEX% Common-collector amplifier, as unity-gain "follower" for voltage regulator circuit

%(END_NOTES)


