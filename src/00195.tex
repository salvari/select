
%(BEGIN_QUESTION)
% Copyright 2003, Tony R. Kuphaldt, released under the Creative Commons Attribution License (v 1.0)
% This means you may do almost anything with this work of mine, so long as you give me proper credit

Suppose this circuit were constructed, using a fully discharged capacitor (0 volts) and a voltmeter connected in parallel with it to measure its voltage:

$$\epsfbox{00195x01.eps}$$

What will happen to the capacitor's voltage after the switch is closed?  Be as precise as you can with your answer, and explain why it does what it does.

\vskip 30pt

\underbar{file 00195}
%(END_QUESTION)





%(BEGIN_ANSWER)

The capacitor's voltage will increase asymptotically over time when the switch is closed, with an ultimate voltage equal to the voltage of the source (battery).

%(END_ANSWER)





%(BEGIN_NOTES)

A qualitative analysis of this circuit's behavior may be performed without using any calculus, or even algebra.  Ask students to explain what happens to current in the circuit as the capacitor's voltage begins to increase, and then what effect that has on the rate of voltage rise, and so on, graphing the results for all too see.

Hint: the rate of voltage rise across a capacitor is in direct proportion to the quantity of current going "through" the capacitor.

%INDEX% RC time constant circuit

%(END_NOTES)


