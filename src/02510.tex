
%(BEGIN_QUESTION)
% Copyright 2005, Tony R. Kuphaldt, released under the Creative Commons Attribution License (v 1.0)
% This means you may do almost anything with this work of mine, so long as you give me proper credit

In this automotive fuel level sensing circuit, a current mirror is supposed to maintain a constant current through the fuel level sensor, which is nothing more than a variable resistance (rheostat) that changes with fuel level.  The voltage dropped across this sensor resistance is then sent to a fuel gauge: a voltmeter with the scale calibrated in gallons of fuel level:

$$\epsfbox{02510x01.eps}$$

There is a problem in this circuit, though.  The fuel gauge reads empty even when you know the fuel tank is completely full.  You take two DC voltage measurements to begin your troubleshooting: +11.3 volts between TP2 and ground, and 6.7 volts between TP3 and ground.

From this information, identify two possible faults that could account for the problem and all measured values in this circuit, and also identify two circuit elements that could not possibly be to blame (i.e. two things that you know {\it must} be functioning properly, no matter what else may be faulted).  The circuit elements you identify as either possibly faulted or properly functioning can be wires, traces, and connections as well as components.  Be as specific as you can in your answers, identifying both the circuit element and the type of fault.

\medskip
\goodbreak
\item{$\bullet$} Circuit elements that are possibly faulted
\item{1.}
\item{2.} 
\medskip

\medskip
\goodbreak
\item{$\bullet$} Circuit elements that must be functioning properly
\item{1.} 
\item{2.} 
\medskip

\underbar{file 02510}
%(END_QUESTION)





%(BEGIN_ANSWER)

Note: the following answers are not exhaustive.  There may be more circuit elements possibly at fault and more circuit elements known to be functioning properly!

\medskip
\goodbreak
\item{$\bullet$} Circuit elements that are possibly faulted
\item{1.} Fuel gauge failed
\item{2.} Wire from sensor to fuel gauge failed open
\medskip

\medskip
\goodbreak
\item{$\bullet$} Circuit elements that must be functioning properly
\item{1.} Transistor $Q_1$ (functioning as diode, dropping 0.7 volts)
\item{2.} Battery
\medskip

%(END_ANSWER)





%(BEGIN_NOTES)

Ask your students to identify means by which they could confirm suspected circuit elements, by measuring something other than what has already been measured.

Troubleshooting scenarios are always good for stimulating class discussion.  Be sure to spend plenty of time in class with your students developing efficient and logical diagnostic procedures, as this will assist them greatly in their careers.

%INDEX% Electronic ignition system, engine
%INDEX% Ignition system (electronic), engine
%INDEX% Transistor switch circuit (BJT)
%INDEX% Troubleshooting, electronic engine ignition system (BJT switch)

%(END_NOTES)


