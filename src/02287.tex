
%(BEGIN_QUESTION)
% Copyright 2005, Tony R. Kuphaldt, released under the Creative Commons Attribution License (v 1.0)
% This means you may do almost anything with this work of mine, so long as you give me proper credit

The {\it voltage gain} of a single-ended amplifier is defined as the ratio of output voltage to input voltage:

$$\epsfbox{02287x01.eps}$$

$$A_V = {V_{out} \over V_{in}}$$

Often voltage gain is defined more specifically as the ratio of output voltage change to input voltage change.  This is generally known as the {\it AC} voltage gain of an amplifier:

$$A_{V(AC)} = {{\Delta V_{out}} \over {\Delta V_{in}}}$$

In either case, though, gain is a ratio of a single output voltage to a single input voltage.

\vskip 10pt

How then do we generally define the voltage gain of a {\it differential} amplifier, where there are two inputs, not just one?

$$\epsfbox{02287x02.eps}$$

\underbar{file 02287}
%(END_QUESTION)





%(BEGIN_ANSWER)

Voltage gain for a differential amplifier is defined as the ratio of output voltage to the {\it difference} in voltage between the two inputs.

%(END_ANSWER)





%(BEGIN_NOTES)

This is a very important concept for students to grasp, especially before they proceed to study operational amplifiers, which are nothing more than differential amplifiers with extremely high voltage gains.

%INDEX% Differential versus single-ended amplifier
%INDEX% Single-ended versus differential amplifier
%INDEX% Voltage gain, defined for a differential amplifier

%(END_NOTES)


