 
%(BEGIN_QUESTION)
% Copyright 2003, Tony R. Kuphaldt, released under the Creative Commons Attribution License (v 1.0)
% This means you may do almost anything with this work of mine, so long as you give me proper credit

Draw an energy diagram for a PN semiconductor junction showing the motion of electrons and holes conducting an electric current.

\underbar{file 02006}
%(END_QUESTION)





%(BEGIN_ANSWER)

For the sake of clearly seeing the actions of charge carriers (mobile electrons and holes), non-moving electrons in the valence bands are not shown:

$$\epsfbox{02006x01.eps}$$

The "+" and "-" signs show the locations of ionized acceptor and donor atoms, having taken on electric charges to create valence-band holes and conduction-band electrons, respectively.

\vskip 10pt

{\it Note: $E_f$ represents the Fermi energy level, and not a voltage.  In physics, $E$ always stands for energy and $V$ for potential (voltage).} 

%(END_ANSWER)





%(BEGIN_NOTES)

Students will probably ask why there are a few holes shown in the N-type valence band, and why there are a few electrons in the P-type conduction band.  Let them know that just because N-type materials are specifically designed to have conduction-band electrons does not mean they are completely devoid of valence-band holes, and visa-versa!  What your students see here are {\it minority carriers}.

%INDEX% Energy diagram, forward-biased PN junction
%INDEX% PN junction, electron and hole motion

%(END_NOTES)


