
%(BEGIN_QUESTION)
% Copyright 2003, Tony R. Kuphaldt, released under the Creative Commons Attribution License (v 1.0)
% This means you may do almost anything with this work of mine, so long as you give me proper credit

Draw an equivalent schematic diagram for this circuit, then calculate the voltage dropped by each of these resistors, given a battery voltage of 9 volts.  The resistor color codes are as follows (assume 0\% error on all resistor values):

\medskip
\item {}$R_1 = $ Brn, Grn, Red, Gld
\item {}$R_2 = $ Yel, Vio, Org, Gld
\item {}$R_3 = $ Red, Grn, Red, Gld
\item {}$R_4 = $ Wht, Blk, Red, Gld
\item {}$R_5 = $ Brn, Blk, Org, Gld
\medskip

$$\epsfbox{00356x01.eps}$$

Compare the voltage dropped across R1, R2, R3, and R4, with and without R5 in the circuit.  What general conclusions may be drawn from these voltage figures?

\underbar{file 00356}
%(END_QUESTION)





%(BEGIN_ANSWER)

$$\epsfbox{00356x02.eps}$$

\vskip 5pt

\settabs 2 \columns
\+ With R5 in the circuit: & Without R5 in the circuit: \cr
\+ $E_{R1} = 0.226$ volts & $E_{R1} = 0.225$ volts \cr
\+ $E_{R2} = 7.109$ volts & $E_{R2} = 7.05$ volts \cr
\+ $E_{R3} = 0.303$ volts & $E_{R3} = 0.375$ volts \cr
\+ $E_{R4} = 1.36$ volts & $E_{R4} = 1.35$ volts \cr

%(END_ANSWER)





%(BEGIN_NOTES)

Ask your students to describe the "with R5 / without R5" voltage values in terms of either {\it increase} or {\it decrease}.  A general pattern should be immediately evident when this is done.

%INDEX% Voltage divider

%(END_NOTES)


