
%(BEGIN_QUESTION)
% Copyright 2005, Tony R. Kuphaldt, released under the Creative Commons Attribution License (v 1.0)
% This means you may do almost anything with this work of mine, so long as you give me proper credit

With a trough of water and three pieces of wire, you could make a {\it liquid potentiometer}:

$$\epsfbox{03300x01.eps}$$

Which way would you have to move the middle wire (the one touching the voltmeter's red test lead) in order to increase the voltmeter's reading?

\underbar{file 03300}
%(END_QUESTION)





%(BEGIN_ANSWER)

Move the wire to the left in order to make the voltmeter register a greater voltage.

\vskip 10pt

Follow-up question: identify any advantages and disadvantages of such a "liquid pot" over standard potentiometers using solid pieces.  Are there any potential safety hazards that occur to you as you look at the illustration of this liquid potentiometer?

%(END_ANSWER)





%(BEGIN_NOTES)

Believe it or not, I have actually seen an application in industry where a "liquid rheostat" (not a potentiometer, but close) was used rather than a device made from solid parts.  Very interesting.  Very dangerous as well, since it was being used as part of a large motor speed control circuit, handling many amps of current at potentially deadly voltages!  I do not know what maniac thought of building this contraption, but it was built and had been working for a number of years.

%INDEX% Potentiometer, as variable voltage divider
%INDEX% Potentiometer, made from water bath

%(END_NOTES)


