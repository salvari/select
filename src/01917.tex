
%(BEGIN_QUESTION)
% Copyright 2003, Tony R. Kuphaldt, released under the Creative Commons Attribution License (v 1.0)
% This means you may do almost anything with this work of mine, so long as you give me proper credit

A technician is measuring two waveforms of differing frequency at the same time on a dual-trace oscilloscope.  The waveform measured by channel "A" seems to be triggered just fine, but the other waveform (channel "B") appears to be untriggered: the waveshape slowly scrolls horizontally across the screen as though the trace were free-running.

This presents a problem for the technician, because channel B's waveform is the more important one to have "locked" in place for viewing.  What should the technician do to make channel B's display stable?

\underbar{file 01917}
%(END_QUESTION)





%(BEGIN_ANSWER)

Switch the trigger source control from "A" to "B".

\vskip 10pt

Follow-up question: if the above advice is taken, channel B's waveform will become "locked" in place, but channel A's waveform will now begin to scroll across the screen.  Is there any way to lock {\it both} waveforms in place so neither one appears to scroll across the screen?

%(END_ANSWER)





%(BEGIN_NOTES)

Like so many oscilloscope principles, this is perhaps best understood through actually using an oscilloscope.  Try to set up two signal generators and an oscilloscope in your classroom so that you may demonstrate these controls while discussing them with your students.

%INDEX% Oscilloscope, triggering on channel A versus channel B

%(END_NOTES)


