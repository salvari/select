
%(BEGIN_QUESTION)
% Copyright 2005, Tony R. Kuphaldt, released under the Creative Commons Attribution License (v 1.0)
% This means you may do almost anything with this work of mine, so long as you give me proper credit

An engineer needs to calculate the values of two resistors to set the minimum and maximum resistance ratios for the following potentiometer circuit:

$$\epsfbox{02769x01.eps}$$

First, write an equation for each circuit, showing how resistances $R_1$, $R_2$, and the 10 k$\Omega$ of the potentiometer combine to form the ratio $a \over b$.  Then, use techniques for solving simultaneous equations to calculate actual resistance values for $R_1$ and $R_2$.

\underbar{file 02769}
%(END_QUESTION)





%(BEGIN_ANSWER)

$${a \over b} \hbox{ (minimum)} = {R_1 \over {R_2 + 10000}} \hbox{\hskip 50pt} {a \over b} \hbox{ (maximum)} = {{R_1 + 10000 }\over R_2}$$

\vskip 10pt

$R_1$ = 15.77 k$\Omega$

\vskip 10pt

$R_2$ = 515.5 $\Omega$

%(END_ANSWER)





%(BEGIN_NOTES)

This very practical application of simultaneous equations was actually used by one of my students in establishing the lower and upper bounds for the voltage gain adjustment of an inverting opamp circuit!

%INDEX% Simultaneous equations
%INDEX% Systems of linear equations

%(END_NOTES)


