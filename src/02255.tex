
%(BEGIN_QUESTION)
% Copyright 2003, Tony R. Kuphaldt, released under the Creative Commons Attribution License (v 1.0)
% This means you may do almost anything with this work of mine, so long as you give me proper credit

What is a {\it harmonic} frequency?  If an oscillator circuit outputs a fundamental frequency of 12 kHz, calculate the frequencies of the following harmonics:

\medskip
\item{$\bullet$} 1st harmonic = 
\item{$\bullet$} 2nd harmonic = 
\item{$\bullet$} 3rd harmonic = 
\item{$\bullet$} 4th harmonic = 
\item{$\bullet$} 5th harmonic = 
\item{$\bullet$} 6th harmonic = 
\medskip

\underbar{file 02255}
%(END_QUESTION)





%(BEGIN_ANSWER)

\medskip
\item{$\bullet$} 1st harmonic = 12 kHz
\item{$\bullet$} 2nd harmonic = 24 kHz
\item{$\bullet$} 3rd harmonic = 36 kHz
\item{$\bullet$} 4th harmonic = 48 kHz
\item{$\bullet$} 5th harmonic = 60 kHz
\item{$\bullet$} 6th harmonic = 72 kHz
\medskip

%(END_ANSWER)





%(BEGIN_NOTES)

Ask your students to determine the mathematical relationship between harmonic number, harmonic frequency, and fundamental frequency.  It isn't difficult to figure out!

%INDEX% Harmonic, defined

%(END_NOTES)


