
%(BEGIN_QUESTION)
% Copyright 2003, Tony R. Kuphaldt, released under the Creative Commons Attribution License (v 1.0)
% This means you may do almost anything with this work of mine, so long as you give me proper credit

Define the following terms: {\it energy}, {\it work}, and {\it power}.

\underbar{file 00107}
%(END_QUESTION)





%(BEGIN_ANSWER)

{\it Work} is the exertion of a force over a distance.  {\it Energy} is the capacity to perform work.  {\it Power} is the rate of work performed per unit time.

%(END_ANSWER)





%(BEGIN_NOTES)

Students may find a basic physics text helpful in obtaining these definitions.  "Work" is a difficult concept to precisely define, especially for students unfamiliar with basic physics.  Technically, it is the vector dot-product of force and displacement, meaning that work equals force times distance {\it only} if the force and distance vectors are precisely parallel to each other.  In other words, if I carry a 10 kg mass (lifting {\it up} against the tug of gravity) while walking parallel to the ground (not going up or down), the force and displacement vectors are perpendicular to each other, and the work I do in carrying the mass is {\it zero}.  It is only if my force is directed precisely the same direction as my motion that all of my effort is translated into work.

%INDEX% Energy, defined
%INDEX% Work, defined
%INDEX% Power, defined

%(END_NOTES)


