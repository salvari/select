
%(BEGIN_QUESTION)
% Copyright 2003, Tony R. Kuphaldt, released under the Creative Commons Attribution License (v 1.0)
% This means you may do almost anything with this work of mine, so long as you give me proper credit

The circuit shown here is called an {\it S-R latch}:

$$\epsfbox{01351x01.eps}$$

Identify which of the two input lines is the {\it Set}, and which is the {\it Reset}, and then write a truth table describing the function of this circuit.

\underbar{file 01351}
%(END_QUESTION)





%(BEGIN_ANSWER)

$$\epsfbox{01351x02.eps}$$

\vskip 10pt

Follow-up question: why are the inputs referred to as $\overline{\hbox{Set}}$ and $\overline{\hbox{Reset}}$, rather than just Set and Reset?

%(END_ANSWER)





%(BEGIN_NOTES)

The "latch" state is the most interesting in this circuit.  Discuss what this means with your students, especially since it is impossible to describe the "latch" state in terms of fixed 1's and 0's.  Also ask your students to identify the "invalid" state of this latch circuit, and to explain why it is called "invalid".

Discuss the active-low nature of this latch circuit's inputs.  Explain to your students that many digital functions have active-low inputs, and that it is common to denote those inputs by writing a Boolean complementation bar over the input's name.

%(END_NOTES)


