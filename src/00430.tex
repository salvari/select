
%(BEGIN_QUESTION)
% Copyright 2003, Tony R. Kuphaldt, released under the Creative Commons Attribution License (v 1.0)
% This means you may do almost anything with this work of mine, so long as you give me proper credit

The circuit shown here is called a {\it relaxation oscillator}.  It works on the principles of capacitor charging over time (an RC circuit), and of the {\it hysteresis} of a gas-discharge bulb: the fact that the voltage required to initiate conduction through the bulb is significantly greater than the voltage below which the bulb ceases to conduct current.

In this circuit, the neon bulb ionizes at a voltage of 70 volts, and stops conducting when the voltage falls below 30 volts:

$$\epsfbox{00430x01.eps}$$

Graph the capacitor's voltage over time as this circuit is energized by the DC source.  Note on your graph at what times the neon bulb is lit:

\underbar{file 00430}
%(END_QUESTION)





%(BEGIN_ANSWER)

$$\epsfbox{00430x02.eps}$$

\vskip 10pt

Follow-up question: assuming a source voltage of 100 volts, a resistor value of 27 k$\Omega$, and a capacitor value of 22 $\mu$F, calculate the amount of time it takes for the capacitor to charge from 30 volts to 70 volts (assuming the neon bulb draws negligible current during the charging phase).

%(END_ANSWER)





%(BEGIN_NOTES)

What we have here is a very simple strobe light circuit.  This circuit may be constructed in the classroom with minimal safety hazard if the DC voltage source is a hand-crank generator instead of a battery bank or line-powered supply.  I've demonstrated this in my own classroom before, using a hand-crank "Megger" (high-range, high-voltage ohmmeter) as the power source.

%INDEX% Relaxation oscillator, RC with neon lamp

%(END_NOTES)


