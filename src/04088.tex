
%(BEGIN_QUESTION)
% Copyright 2007, Tony R. Kuphaldt, released under the Creative Commons Attribution License (v 1.0)
% This means you may do almost anything with this work of mine, so long as you give me proper credit

Examine this vertical (``bird's eye'') view of a boat resisting a river's current:

$$\epsfbox{04088x01.eps}$$

Suppose the driver of this boat does not own an anchor, and furthermore that the only form of propulsion is an electric ``trolling'' motor with an on/off switch (no variable speed control).  With the right combination of switch actuations (on, off, on, off), it should be possible for the boat to maintain its position relative to the riverbanks, against the flow of water.

Now, if we know the boat is actually holding position in the middle of the river, by trolling motor power alone, the pattern of on/off switch actuations should tell us something about the speed of the river.  Perform a couple of ``thought experiments'' where you imagine what the driver of the boat would have to do with the motor's on/off switch to maintain position against a fast current, versus against a slow current.  What relationship do you see between the switch actuations and the speed of the current?

\vskip 10pt

Note: once you understand this question, you will be better prepared to grasp the operation of a {\it Delta-Sigma} analog-to-digital converter!

\underbar{file 04088}
%(END_QUESTION)





%(BEGIN_ANSWER)

The {\it duty cycle} of the switch actuations is in direct proportion to the river's speed.

%(END_ANSWER)





%(BEGIN_NOTES)

The purpose of this question is to present an analogy which students may use to grasp the operation of a Delta-Sigma ADC: the idea that a bitstream (PDM) may represent an analog value.

%INDEX% ADC, Delta-Sigma
%INDEX% Delta-Sigma converter, ADC
%INDEX% Pulse-density modulation (PDM), Delta-Sigma converter
%INDEX% Sigma-Delta converter, ADC

%(END_NOTES)


