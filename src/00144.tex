
%(BEGIN_QUESTION)
% Copyright 2003, Tony R. Kuphaldt, released under the Creative Commons Attribution License (v 1.0)
% This means you may do almost anything with this work of mine, so long as you give me proper credit

A special kind of electric light known as a {\it neon bulb} is an excellent tool for demonstrating the presence of static electricity, and is easily obtained at most electronics supply shops:

$$\epsfbox{00144x01.eps}$$

When a large enough static electric charge is applied between the two wires of the neon bulb, the neon gas inside will "ionize" and produce a colored flash of light.  Experiment with generating your own static electricity and then making the bulb flash.  Hint: it may be easier to see the bulb's flash if the room is dark.

What conditions produce the brightest flash from the bulb?  What materials and techniques work best for producing strong static electric charges?

\underbar{file 00144}
%(END_QUESTION)





%(BEGIN_ANSWER)

This is definitely a question best answered by direct experimentation!

%(END_ANSWER)





%(BEGIN_NOTES)

If it is difficult for students to obtain neon bulbs in your area, it might be a good idea to provide a few bulbs to experiment with when you give your students this worksheet.  They are quite inexpensive and durable little devices.

%INDEX% Neon lamp
%INDEX% Gas discharge lamp
%INDEX% Ionization
%INDEX% Static electricity, generating

%(END_NOTES)


