
%(BEGIN_QUESTION)
% Copyright 2003, Tony R. Kuphaldt, released under the Creative Commons Attribution License (v 1.0)
% This means you may do almost anything with this work of mine, so long as you give me proper credit

Calculate the total impedance offered by these three resistors to a sinusoidal signal with a frequency of 10 kHz:

\medskip
\item{$\bullet$} $R_1 = 3.3 \hbox{ k}\Omega$
\item{$\bullet$} $R_2 = 10 \hbox{ k}\Omega$
\item{$\bullet$} $R_3 = 5 \hbox{ k}\Omega$
\medskip

$$\epsfbox{01836x01.eps}$$

State your answer in the form of a scalar number (not complex), but calculate it using two different strategies:

\medskip
\item{$\bullet$} Calculate total resistance ($R_{total}$) first, then total impedance ($Z_{total}$).
\item{$\bullet$} Calculate individual admittances first ($Y_{R1}$, $Y_{R2}$, and $Y_{R3}$), then total impedance ($Z_{total}$).
\medskip

\underbar{file 01836}
%(END_QUESTION)





%(BEGIN_ANSWER)

\noindent
{\bf First strategy:}

$R_{total} = 1.658 \hbox{ k}\Omega$

$Z_{total} = 1.658 \hbox{ k}\Omega$

\vskip 10pt

\goodbreak

\noindent
{\bf Second strategy:}

$Y_{R1} = 303 \> \mu \hbox{S}$

$Y_{R2} = 100 \> \mu \hbox{S}$

$Y_{R3} = 200 \> \mu \hbox{S}$

$Y_{total} = 603 \> \mu \hbox{S}$

$Z_{total} = 1.658 \hbox{ k}\Omega$

%(END_ANSWER)





%(BEGIN_NOTES)

This question is set up to be more complex than it has to be.  Its purpose is to get students thinking in terms of parallel admittances, in a manner similar to parallel conductances.

%INDEX% Impedance in parallel R circuit

%(END_NOTES)


