
%(BEGIN_QUESTION)
% Copyright 2005, Tony R. Kuphaldt, released under the Creative Commons Attribution License (v 1.0)
% This means you may do almost anything with this work of mine, so long as you give me proper credit

Examine this circuit, consisting of an ideal voltage source and several resistors:

$$\epsfbox{03228x01.eps}$$

First, calculate the voltage seen at the load terminals with a voltmeter directly connected across them (an {\it open-circuit} condition):

$$\epsfbox{03228x02.eps}$$

Next, calculate the current seen at the load terminals with an ammeter directly connected across them (a {\it short-circuit} condition):

$$\epsfbox{03228x03.eps}$$

Based on these open- and short-circuit calculations, draw a new circuit consisting of a single voltage source and a single (series) resistor that will respond in the exact same manner.  In other words, design an {\it equivalent circuit} for the circuit shown here, using the minimum number of possible components.

\underbar{file 03228}
%(END_QUESTION)





%(BEGIN_ANSWER)

$$\epsfbox{03228x04.eps}$$

\vskip 10pt

Follow-up question: is this circuit truly equivalent to the original shown in the question?  Sure, it responds the same under extreme conditions (open-circuit and short-circuit), but will it respond the same as the original circuit under modest load conditions (say, with a 5 k$\Omega$ resistor connected across the load terminals)?

%(END_ANSWER)





%(BEGIN_NOTES)

The purpose of this question is to get students thinking about Th\'evenin equivalent circuits from the perspective of how the original circuit responds to extreme variations in load resistance.

This question is also a good review of voltmeter and ammeter behavior: that ideal voltmeters act as open circuits (infinite input resistance) while ideal ammeters act as short circuits (zero input impedance).

%INDEX% Thevenin's theorem, determining equivalent circuit by considering open- and short-circuit load conditions

%(END_NOTES)


