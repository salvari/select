
%(BEGIN_QUESTION)
% Copyright 2003, Tony R. Kuphaldt, released under the Creative Commons Attribution License (v 1.0)
% This means you may do almost anything with this work of mine, so long as you give me proper credit

Suppose a person decides to attach an electrical generator to their exercise bicycle, so as to do something useful with their "pedal power" while they exercise.  The first time this person uses their bicycle generator, the electricity is used to power a single 60-watt light bulb.  However, the next time this person uses their bicycle generator, a second 60-watt light bulb is connected to the generator, for a total load of 120 watts.

When pedaling with the additional load, the person notices the bicycle is much more difficult to pedal than before.  It takes greater force on the pedals to maintain the same speed as before, when there was only a single 60-watt light bulb to power.  What would you say to this person if they asked you, the electricity expert, to explain why the bicycle is more difficult to pedal with the additional light bulb connected?

\underbar{file 00230}
%(END_QUESTION)





%(BEGIN_ANSWER)

I won't give you the answer directly, but here is a hint: the Law of Energy Conservation.

%(END_ANSWER)





%(BEGIN_NOTES)

This phenomenon is more easily understood when experienced directly.  If you happen to have a hand-powered generator available for a classroom demonstration, help your students set it up to demonstrate this principle.

An excellent topic of discussion related to this question is the effect that using more electrical power has on the generators at power plants (hydroelectric, nuclear, coal-fired, etc.).  What would happen at the power plants supplying electricity to the nation's electrical "grid" if everyone simultaneously turned on all their electrical loads at home?

%INDEX% Sources of electricity

%(END_NOTES)


