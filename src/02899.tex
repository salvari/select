
%(BEGIN_QUESTION)
% Copyright 2005, Tony R. Kuphaldt, released under the Creative Commons Attribution License (v 1.0)
% This means you may do almost anything with this work of mine, so long as you give me proper credit

Complete the timing diagram, showing the state of the $Q$ output over time as the Set and Reset switches are actuated.  Assume that $Q$ begins in the low state on power-up:

$$\epsfbox{02899x01.eps}$$

\underbar{file 02899}
%(END_QUESTION)





%(BEGIN_ANSWER)

$$\epsfbox{02899x02.eps}$$

\vskip 10pt

Follow-up question: complete a schematic diagram showing how the $\overline{Q}$ output of the latch could turn on an electric motor through a bipolar junction transistor.  Also, determine whether the latch circuit would be {\it sourcing} or {\it sinking} current to the transistor when the motor is running:

$$\epsfbox{02899x03.eps}$$

%(END_ANSWER)





%(BEGIN_NOTES)

Nothing special here in this question.  Perhaps the main point is to familiarize students with the concept of a timing diagram, and how to transfer the truth table function of a specific logic circuit to a time-domain plot.

%INDEX% Timing diagram, S-R latch circuit

%(END_NOTES)


