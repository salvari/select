
%(BEGIN_QUESTION)
% Copyright 2003, Tony R. Kuphaldt, released under the Creative Commons Attribution License (v 1.0)
% This means you may do almost anything with this work of mine, so long as you give me proper credit

A beginning electronics student is just learning about transistors, and reads in the textbook that a bipolar transistor (either NPN or PNP) can be thought of as two diodes connected back-to-back as such:

$$\epsfbox{02033x01.eps}$$

Acting on this idea, the student proceeds to connect two 1N4001 rectifying diodes back-to-back and try to use it as a transistor.  This idea does not work: although the diode pair reads the same patterns of continuity as a transistor would, it does not amplify.  Explain why.

Note: this is a fairly deep question, and may not be answered without an understanding of charge carrier energy levels and semiconductor junction behavior.

\underbar{file 02033}
%(END_QUESTION)





%(BEGIN_ANSWER)

This makeshift transistor will not work because the metal connection between the two "P"-type materials (the diode anodes) precludes the injection of minority carrier (conduction band level) electrons into the "P" material of the collector-side diode.

\vskip 10pt

Follow-up question: what do you suppose is {\it really} meant by the textbook's statement of bipolar transistors being equivalent to back-to-back diodes, if two diodes connected back-to-back do not exhibit amplifying behavior?  Is this a completely wrong statement, or is there some truth to it?

%(END_ANSWER)





%(BEGIN_NOTES)

The idea for this question came from personal experience.  I actually tried to build my own transistor out of two back-to-back diodes, and failed miserably.  It took {\it many} years before I finally understood enough about semiconductor physics to realize why it would not work!

%INDEX% Transistor, diode equivalent

%(END_NOTES)


