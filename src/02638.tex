
%(BEGIN_QUESTION)
% Copyright 2005, Tony R. Kuphaldt, released under the Creative Commons Attribution License (v 1.0)
% This means you may do almost anything with this work of mine, so long as you give me proper credit

$$\epsfbox{02638x01.eps}$$

\underbar{file 02638}
\vfil \eject
%(END_QUESTION)





%(BEGIN_ANSWER)

Use circuit simulation software to verify your predicted and measured parameter values.

%(END_ANSWER)





%(BEGIN_NOTES)

Use a dual-voltage, regulated power supply to supply power to the opamp.  Specify standard resistor values, all between 1 k$\Omega$ and 100 k$\Omega$ (1k5, 2k2, 2k7, 3k3, 4k7, 5k1, 6k8, 10k, 22k, 33k, 39k 47k, 68k, etc.).

I have had good success using the following values:

\medskip
\item{$\bullet$} +V = +12 volts
\item{$\bullet$} -V = -12 volts
\item{$\bullet$} $R_1$ = 10 k$\Omega$
\item{$\bullet$} $R_{pot}$ = 10 k$\Omega$ multi-turn
\item{$\bullet$} $C_1$ = 0.001 $\mu$F or 0.47 $\mu$F
\item{$\bullet$} $L_1$ = 100 mH
\item{$\bullet$} $U_1$ = one-half of LM1458 dual operational amplifier
\medskip

Note that due to the lack of automatic gain control in this circuit, the potentiometer adjustment is {\it very} sensitive!  Students will have to finely adjust the multi-turn potentiometer to achieve a good sine wave (meeting the Barkhausen criterion).

An extension of this exercise is to incorporate troubleshooting questions.  Whether using this exercise as a performance assessment or simply as a concept-building lab, you might want to follow up your students' results by asking them to predict the consequences of certain circuit faults.

%INDEX% Assessment, performance-based (Opamp LC parallel resonant oscillator)

%(END_NOTES)


