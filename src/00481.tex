
%(BEGIN_QUESTION)
% Copyright 2003, Tony R. Kuphaldt, released under the Creative Commons Attribution License (v 1.0)
% This means you may do almost anything with this work of mine, so long as you give me proper credit

What would a digital voltmeter register, if connected to the circuit as shown below?

$$\epsfbox{00481x01.eps}$$

\underbar{file 00481}
%(END_QUESTION)





%(BEGIN_ANSWER)

If you calculated 3.797 volts, you made a mistake!  In actuality, the voltmeter would register +1.235 volts.

%(END_ANSWER)





%(BEGIN_NOTES)

A very common mistake I've seen students make is to disregard polarities when using Millman's theorem.  If this appears to be a common problem in your class, ask your students if they think reversing one of the voltage source polarities would have any effect on the "bus" voltage.  Of course, it should.  Once students understand that polarity is significant, they can arrive at their own consistent approach to accounting for polarity in the Millman's theorem equation.

Another strategy for getting students to understand the significance of polarity when using Millman's theorem is to go back to the foundations of Millman's theorem: the principle of converting Th\'evenin sources into Norton sources.  If a Th\'evenin source with a "backwards" battery is converted into a Norton source, that current source will {\it subtract} current from the rest of the current sources, leaving less to go through the total Norton resistance.  Students should be able to readily understand the principle of Norton current sources adding versus subtracting, and this should then carry over into their use of the Millman's theorem equation.

%INDEX% Millman's theorem

%(END_NOTES)


