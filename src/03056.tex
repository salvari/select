
%(BEGIN_QUESTION)
% Copyright 2005, Tony R. Kuphaldt, released under the Creative Commons Attribution License (v 1.0)
% This means you may do almost anything with this work of mine, so long as you give me proper credit

Observe the following equivalence:

$${4^2 \over 4^3} = {{4 \times 4} \over {4 \times 4 \times 4}}$$

It should be readily apparent that we may cancel out two quantities from both top and bottom of the fraction, so in the end we are left with this:

$$1 \over 4$$

Following the rule of ${a^m \over a^n} = a^{m-n}$, the reduction of ${4^2 \over 4^3}$ should be $4^{-1}$.  Many students find this confusing, as the intuitive concept of exponents (how many times a number is to be multiplied by itself) fails here.  How in the world do we multiply 4 by itself -1 times?!

\vskip 10pt

Expand each of these expressions so that there are no exponents either:

\vskip 10pt
\goodbreak
\item{$\bullet$} ${3^2 \over 3^5} = $
\vskip 10pt
\item{$\bullet$} ${10^4 \over 10^6} = $
\vskip 10pt
\item{$\bullet$} ${8^3 \over 8^7} = $
\vskip 10pt
\item{$\bullet$} ${20^4 \over 20^5} = $
\vskip 10pt

After expanding each of these expressions, re-write each one in simplest form: one number to a power, just like the final form of the example given ($4^{-1}$), following the rule ${a^m \over a^n} = a^{m-n}$.  From these examples, what easy-to-understand definition can you think of to describe negative exponents?

Also, expand the following expression so there are no exponents, then re-write the result in exponent form following the rule ${a^m \over a^n} = a^{m-n}$:

$$5^3 \over 5^3$$

What does this tell you about exponents of zero?

\underbar{file 03056}
%(END_QUESTION)





%(BEGIN_ANSWER)

A negative exponent is simply the reciprocal ($1/x$) of its positive counterpart.  A zero exponent is always equal to 1.

%(END_ANSWER)





%(BEGIN_NOTES)

I have found that students who cannot fathom the meaning of negative or zero exponents often understand immediately when they construct their own definition based on the general rule (${a^m \over a^n} = a^{m-n}$).

%INDEX% Algebra, exponents
%INDEX% Exponents, algebra

%(END_NOTES)


