
%(BEGIN_QUESTION)
% Copyright 2003, Tony R. Kuphaldt, released under the Creative Commons Attribution License (v 1.0)
% This means you may do almost anything with this work of mine, so long as you give me proper credit

Given the output voltages of the two alternators, it is {\it not} safe to close the breaker.  Explain why.

$$\epsfbox{01056x01.eps}$$

\underbar{file 01056}
%(END_QUESTION)





%(BEGIN_ANSWER)

The greatest problem with closing the breaker is the 37$^{o}$ phase shift between the two alternators' output voltages.

\vskip 10pt

Follow-up question: what must be done to bring the two alternator voltages into phase with each other?

\vskip 10pt

Challenge question: once the breaker is closed, can the two alternators ever fall out of phase with each other again?

%(END_ANSWER)





%(BEGIN_NOTES)

Discuss the consequences of closing the breaker when there is such a large phase shift between the two alternator output voltages.  What will likely happen in the circuit if the breaker is closed under these conditions?

Ask your students whether or not the discrepancy in output voltage (218 VAC versus 216.5 VAC) is of any consequence in closing the breaker.  Why is phase shift the only factor mentioned in the answer as a reason not to close the breaker?

This question serves to illustrate alternator theory as well as AC network analysis principles.  The "phase-locking" phenomenon of two paralleled alternators is very important for students to understand if they are to do any work related to AC power generation systems.

%(END_NOTES)


