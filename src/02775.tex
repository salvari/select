
%(BEGIN_QUESTION)
% Copyright 2005, Tony R. Kuphaldt, released under the Creative Commons Attribution License (v 1.0)
% This means you may do almost anything with this work of mine, so long as you give me proper credit

In ladder logic diagrams, a normally-open relay contact is drawn as a set of parallel lines, almost like a non-polarized capacitor in an electronic schematic diagram.  Normally-closed relay contacts differ in symbolism by having a diagonal line drawn through them.

Analyze the following relay logic circuit, completing the truth table accordingly:

$$\epsfbox{02775x01.eps}$$

\underbar{file 02775}
%(END_QUESTION)





%(BEGIN_ANSWER)

$$\epsfbox{02775x02.eps}$$

%(END_ANSWER)





%(BEGIN_NOTES)

Many students find the "line-through-the-contact" a very intuitive way to represent normally-closed relay contacts.  Be sure to emphasize that the diagonal line, as well as the name {\it normally}-closed, does not refer to any given state of the contact, but rather to the contact's {\it resting} state when the relay coil is de-energized.  I have seen teachers put a diagonal line through a relay contact symbol on a ladder logic diagram to indicate the state of the contact being closed by energization of the coil, during the process of explaining how a circuit functioned.  This is wrong, as it confuses the concept of contacts being normally-closed with the concept of contacts simply being (energized) closed.

%INDEX% Ladder logic diagram (with relays)
%INDEX% Normally closed contact, relay (ladder logic diagram symbol)

%(END_NOTES)


