
%(BEGIN_QUESTION)
% Copyright 2005, Tony R. Kuphaldt, released under the Creative Commons Attribution License (v 1.0)
% This means you may do almost anything with this work of mine, so long as you give me proper credit

Find the datasheet for a real instrumentation amplifier (packaged as a single integrated circuit) and bring it to class for discussion with your classmates.  Analyze and discuss the inner workings of the circuit, and some of its performance parameters.  If you do not know where to begin looking, try researching the Analog Devices model AD623, either in a reference book or on the internet.

\underbar{file 02527}
%(END_QUESTION)





%(BEGIN_ANSWER)

I'll leave the discussion up to you and your classmates.  With any luck, you should have found some example circuits showing how the instrumentation amplifier may be used, or possibly some application notes to complement the datasheet.

%(END_ANSWER)





%(BEGIN_NOTES)

The idea of this question is to get students researching real integrated circuit applications, to teach them how to do this research and also how to interpret what they find.  Since there are so many high-quality instrumentation amplifiers already built and packaged as monolithic units, it is usually not worth the technician's time to fabricate one from individual opamps.  However, when specifying a pre-built instrumentation amplifier, it is essential to know what you need and how to use it once it arrives!

%INDEX% Instrumentation amplifier

%(END_NOTES)


