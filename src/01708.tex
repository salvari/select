
%(BEGIN_QUESTION)
% Copyright 2003, Tony R. Kuphaldt, released under the Creative Commons Attribution License (v 1.0)
% This means you may do almost anything with this work of mine, so long as you give me proper credit

Express the following numbers in scientific notation:

$$0.00045 = $$

$$23,000,000 = $$

$$700,000,000,000 = $$

$$0.00000000000000000001 = $$

$$0.000098 = $$

\underbar{file 01708}
%(END_QUESTION)





%(BEGIN_ANSWER)

$$0.00045 = 4.5 \times 10^{-4}$$

$$23,000,000 = 2.3 \times 10^{7}$$

$$700,000,000,000 = 7 \times 10^{11}$$

$$0.00000000000000000001 = 1 \times 10^{-20}$$

$$0.000098 = 9.8 \times 10^{-5}$$

\vskip 10pt

Follow-up question: explain why scientific notation is used so frequently by scientists and engineers.

%(END_ANSWER)





%(BEGIN_NOTES)

The convenience in which scientific notation allows us to express and manipulate extremely large and small quantities is perhaps the most important reason why it is used by scientists and engineers.

Your students will be pleased to discover that their calculators can be used to easily convert between "expanded" (sometimes called {\it fixed-point}) and scientific notations.

%INDEX% Notation, scientific

%(END_NOTES)


