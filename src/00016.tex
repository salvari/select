
%(BEGIN_QUESTION)
% Copyright 2003, Tony R. Kuphaldt, released under the Creative Commons Attribution License (v 1.0)
% This means you may do almost anything with this work of mine, so long as you give me proper credit

Determine what these four voltmeters (A, B, C, D) will register when connected to this circuit in the following positions (assume a battery voltage of 6 volts):

$$\epsfbox {00016x01.eps}$$

\medskip
\item{$\bullet$} Voltmeter A = 
\item{$\bullet$} Voltmeter B = 
\item{$\bullet$} Voltmeter C = 
\item{$\bullet$} Voltmeter D = 
\medskip

\underbar{file 00016}
%(END_QUESTION)





%(BEGIN_ANSWER)

\medskip
\item{$\bullet$} Voltmeter A = 6 volts
\item{$\bullet$} Voltmeter B = 0 volts
\item{$\bullet$} Voltmeter C = 6 volts
\item{$\bullet$} Voltmeter D = 0 volts
\medskip

%(END_ANSWER)





%(BEGIN_NOTES)

Students often find the terms "open" and "closed" to be confusing with reference to electrical switches, because they sound opposite to the function of a door (i.e. you can only go through an open door, but electricity can only go through a closed switch!).  The words actually make sense, though, if you look at the schematic symbol for an electrical switch as a door mounted "sideways" in the circuit.  At least visually, then, "open" and "closed" will have common references.

One analogy to use for the switch's function that makes sense with the schematic is a drawbridge: when the bridge is down (closed), cars may cross; when the bridge is up (open), cars cannot.

I have found that the concept of {\it electrically common points} is most helpful when students first learn to relate voltage drop with continuity (breaks or non-breaks) in a circuit.

To be able to immediately relate the expected voltage drop between two points with the electrical continuity between those points is a very important foundational skill in electrical troubleshooting.  Without mastery of this skill, students will have great difficulty detecting and correcting faults in circuits caused by poor connections and broken wires, which constitute a fair portion of realistic circuit failures.

%INDEX% Troubleshooting, simple circuit
%INDEX% Voltmeter usage

%(END_NOTES)


