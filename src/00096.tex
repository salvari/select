
%(BEGIN_QUESTION)
% Copyright 2003, Tony R. Kuphaldt, released under the Creative Commons Attribution License (v 1.0)
% This means you may do almost anything with this work of mine, so long as you give me proper credit

Shown here is a circuit constructed on a PCB (a "Printed Circuit Board"), with copper "traces" serving as wires to connect the components together:

$$\epsfbox{00096x01.eps}$$

How would the multimeter be used to measure the voltage across the component labeled "R1" when energized?  Include these important points in your answer:

\item {$\bullet$} The configuration of the multimeter (selector switch position, test lead jacks)
\item {$\bullet$} The connections of the meter test leads to the circuit
\item {$\bullet$} The state of the switch on the PCB (open or closed)

\underbar{file 00096}
%(END_QUESTION)





%(BEGIN_ANSWER)

$$\epsfbox{00096x02.eps}$$

The test lead connections (to the circuit) shown are not the only correct answer.  It is possible to touch the test leads to different points on the PCB and still measure the voltage across the resistor (component labeled R1).  What are some alternative points on the PCB where the voltage across R1 could be measured?

%(END_ANSWER)





%(BEGIN_NOTES)

Many multimeters use "international" symbols to label DC and AC selector switch positions.  It is important for students to understand what these symbols mean.

The test lead connections (to the circuit) shown are not the only correct answer.  It is possible to touch the test leads to different points on the PCB and still measure the voltage across the resistor (R1).  However, if there are poor connections on the circuit board (between component leads and copper traces), measuring voltage at points on the circuit board other than directly across the component in question may give misleading measurements.  Discuss this with your students.

%INDEX% Printed circuit board
%INDEX% Voltmeter usage

%(END_NOTES)


