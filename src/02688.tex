
%(BEGIN_QUESTION)
% Copyright 2005, Tony R. Kuphaldt, released under the Creative Commons Attribution License (v 1.0)
% This means you may do almost anything with this work of mine, so long as you give me proper credit

Examine this progression of mathematical statements:

$$\sqrt{1000} = 10^{1.5}$$

$$\log \sqrt{1000} = \log \left( 10^{1.5} \right)$$

$$\log \left( 1000^{1 \over 2} \right) = \log \left( 10^{1.5} \right)$$

$${1 \over 2}(\log 1000) = \log \left( 10^{1.5} \right)$$

$${1 \over 2}(\log 10^3) = \log \left( 10^{1.5} \right)$$

$${3 \over 2}(\log 10) = \log \left( 10^{1.5} \right)$$

$${3 \over 2}(1) = \log \left( 10^{1.5} \right)$$

$${3 \over 2} = \log \left( 10^{1.5} \right)$$

$${3 \over 2} = 1.5$$

What began as a fractional exponent problem ended up as a simple fraction, through the application of logarithms.  What does this tell you about the utility of logarithms as an arithmetic tool?

\underbar{file 02688}
%(END_QUESTION)





%(BEGIN_ANSWER)

That logarithms can reduce the complexity of an equation from fractional exponentiation, down to simple fractions, indicates its usefulness as a tool to {\it simplify} arithmetic problems.  Specifically, the logarithm of a root of a number is equal to the logarithm of that number divided by the root index.

%(END_ANSWER)





%(BEGIN_NOTES)

In mathematics, any procedure that reduces a complex type of problem into a simpler type of problem is called a {\it transform function}, and logarithms are one of the simplest types of transform functions in existence.

%INDEX% Logarithms, used to transform a root problem into a division problem

%(END_NOTES)


