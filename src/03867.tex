
%(BEGIN_QUESTION)
% Copyright 2006, Tony R. Kuphaldt, released under the Creative Commons Attribution License (v 1.0)
% This means you may do almost anything with this work of mine, so long as you give me proper credit

Suppose you needed to take 24 volts DC and "step" it down to half that voltage to power a load rated for 12 volts.  You could, of course, simply use a dropping resistor to drop the extra 12 volts, or even a potentiometer to divide the 24 volts down to 12 volts.  However, a "buck" DC-DC converter circuit might be a better way of reducing 24 volts down to 12 volts, especially if the 24 volt source was a battery with limited life.  Explain why this is.

\underbar{file 03867}
%(END_QUESTION)





%(BEGIN_ANSWER)

DC-DC converter circuits are able to work kind of like transformers do for AC: stepping voltage one way while stepping current the other.  The result is much greater energy efficiency than any resistive "dropping" or "dividing" circuit.

%(END_ANSWER)





%(BEGIN_NOTES)

{\bf This question is intended for exams only and not worksheets!}.

%(END_NOTES)


