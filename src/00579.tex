
%(BEGIN_QUESTION)
% Copyright 2003, Tony R. Kuphaldt, released under the Creative Commons Attribution License (v 1.0)
% This means you may do almost anything with this work of mine, so long as you give me proper credit

As a general rule, capacitors oppose change in ({\bf choose:} \underbar{voltage} or \underbar{current}), and they do so by . . . (complete the sentence).

\vskip 10pt

Based on this rule, determine how a capacitor would react to a constant AC voltage that increases in frequency.  Would an capacitor pass more or less current, given a greater frequency?  Explain your answer.

\underbar{file 00579}
%(END_QUESTION)





%(BEGIN_ANSWER)

As a general rule, capacitors oppose change in \underbar{voltage}, and they do so by producing a current.

\vskip 10pt

A capacitor will pass a greater amount of AC current, given the same AC voltage, at a greater frequency.

%(END_ANSWER)





%(BEGIN_NOTES)

This question is an exercise in qualitative thinking: relating rates of change to other variables, without the use of numerical quantities.  The general rule stated here is very, very important for students to master, and be able to apply to a variety of circumstances.  If they learn nothing about capacitors except for this rule, they will be able to grasp the function of a great many capacitor circuits.

%(END_NOTES)


