
%(BEGIN_QUESTION)
% Copyright 2004, Tony R. Kuphaldt, released under the Creative Commons Attribution License (v 1.0)
% This means you may do almost anything with this work of mine, so long as you give me proper credit

$$\epsfbox{01935x01.eps}$$

\underbar{file 01935}
\vfil \eject
%(END_QUESTION)





%(BEGIN_ANSWER)

Use circuit simulation software to verify your predicted and measured parameter values.

%(END_ANSWER)





%(BEGIN_NOTES)

Students are allowed to adjust the bias potentiometer to achieve class-A operation after calculating and inserting the resistance values $R_C$ and $R_E$.  However, they are not allowed to change either $R_C$ or $R_E$ once the circuit is powered and tested, lest they achieve the specified gain through trial-and-error!

A good percentage tolerance for gain is +/- 10\%.  The lower you set the target gain, the more accuracy you may expect out of your students' circuits.  I usually select random values of voltage gain between 2 and 10, and I strongly recommend that students choose resistor values between 1 k$\Omega$ and 100 k$\Omega$.  Resistor values much lower than 1 k$\Omega$ lead to excessive quiescent currents, which may cause accuracy problems ($r'_e$ drifting due to temperature effects).

An extension of this exercise is to incorporate troubleshooting questions.  Whether using this exercise as a performance assessment or simply as a concept-building lab, you might want to follow up your students' results by asking them to predict the consequences of certain circuit faults.

%INDEX% Assessment, performance-based (Class-A BJT amplifier circuit with specified voltage gain)

%(END_NOTES)


