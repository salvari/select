
%(BEGIN_QUESTION)
% Copyright 2003, Tony R. Kuphaldt, released under the Creative Commons Attribution License (v 1.0)
% This means you may do almost anything with this work of mine, so long as you give me proper credit

A student builds the following circuit and connects an oscilloscope to its output:

$$\epsfbox{00748x01.eps}$$

The waveform shown on the oscilloscope display looks like this:

$$\epsfbox{00748x02.eps}$$

Definitely not Class-A operation!  Suspecting a problem with the input waveform, the student disconnects the oscilloscope probe from the amplifier output and moves it over to the amplifier input terminal.  There, the following waveform is seen:

$$\epsfbox{00748x03.eps}$$

How can this amplifier circuit be producing such a distorted output waveform with such a clean input waveform?  Explain your answer.

\underbar{file 00748}
%(END_QUESTION)





%(BEGIN_ANSWER)

The DC bias voltage ($V_{bias}$) is excessive.

%(END_ANSWER)





%(BEGIN_NOTES)

Ask your students how they can tell the difference between excessive biasing and insufficient biasing, by inspection of the output waveform.  There is a difference to be seen, but it requires a good understanding of how the circuit works!  Students may be tempted to simply memorize waveforms ("when I see this kind of waveform, I know the problem is excessive biasing . . ."), so prepare to challenge their understanding with questions such as:

\medskip
\item{$\bullet$} What polarity of input signal drives the transistor toward cutoff?
\item{$\bullet$} What polarity of input signal drives the transistor toward saturation?
\item{$\bullet$} Where on the output waveform is the transistor in cutoff (if at all)?
\item{$\bullet$} Where on the output waveform is the transistor in saturation (if at all)?
\item{$\bullet$} Where on the output waveform is the transistor in its active mode?
\medskip

Another point worth mentioning: some students may be confused by the phasing of the input and output waveforms, comparing the two different oscilloscope displays.  For a common-emitter (inverting) amplifier  such as this, they expect to see the output voltage peak positive whenever the input voltage peaks negative, and visa-versa, but here the two oscilloscope displays show positive peaks occurring right next to the left-hand side of the screen.  Why is this?  Because the oscilloscope does not represent phase unless it is in dual-trace mode!  When you disconnect the input probe and move it to another point in the circuit, any time reference is lost, the oscilloscope's triggering function placing the first waveform peak right where you tell it to, usually near the left-hand side of the display.

%INDEX% Common-emitter circuit, improperly biased (clipping signal)

%(END_NOTES)


