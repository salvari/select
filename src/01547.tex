
%(BEGIN_QUESTION)
% Copyright 2003, Tony R. Kuphaldt, released under the Creative Commons Attribution License (v 1.0)
% This means you may do almost anything with this work of mine, so long as you give me proper credit

Determine the polarity of voltage across the resistor in this simple circuit, and be prepared to explain {\it how} you did so:

$$\epsfbox{01547x01.eps}$$

\underbar{file 01547}
%(END_QUESTION)





%(BEGIN_ANSWER)

$$\epsfbox{01547x02.eps}$$

Follow-up question: are the battery voltage values important to the answer?  Explain why or why not.

%(END_ANSWER)





%(BEGIN_NOTES)

The answer to this question is fairly simple, but the real point of it is to get students thinking about how and why it is the way it is.  One thing I've noticed as an instructor of electronics is that most students tend to follow the rule of proximity: the resistor's voltage drop polarity is determined by proximity to poles of the battery.  The resistor terminal closest to the battery's negative terminal must be negative as well, or so the thinking goes.  

In this particular circuit, though, the rule of proximity does not hold very well, and a different rule is necessary.

%INDEX% Voltage drops in circuit

%(END_NOTES)


