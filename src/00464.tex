
%(BEGIN_QUESTION)
% Copyright 2003, Tony R. Kuphaldt, released under the Creative Commons Attribution License (v 1.0)
% This means you may do almost anything with this work of mine, so long as you give me proper credit

There are two wire windings wrapped around a common iron bar in this illustration, such that whatever magnetic flux may be produced by one winding is fully shared by the other winding:

$$\epsfbox{00464x01.eps}$$

Write two equations describing the induced voltage at each winding ($e_p = \dots$ and $e_s = \dots$), in each case expressing the induced voltage in terms of the instantaneous magnetic flux ($\phi$) and the number of turns of wire in that winding ($N_p$ and $N_s$, respectively).

Then, combine these two equations, based on the fact that the magnetic flux is equal for each winding.

\underbar{file 00464}
%(END_QUESTION)





%(BEGIN_ANSWER)

$$e_p = N_p{d\phi \over dt}$$

$$e_s = N_s{d\phi \over dt}$$

\vskip 10pt

Then, combining the two equations:

$${e_p \over N_p} = {e_s \over N_s}$$

%(END_ANSWER)





%(BEGIN_NOTES)

Obtaining the last equation is an application of the mathematical truth that quantities equal to the same thing are equal to each other (if $a = c$ and $b = c$, then $a = b$).  

%INDEX% Algebra, manipulating equations
%INDEX% Mutual inductance

%(END_NOTES)


