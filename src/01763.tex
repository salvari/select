
%(BEGIN_QUESTION)
% Copyright 2003, Tony R. Kuphaldt, released under the Creative Commons Attribution License (v 1.0)
% This means you may do almost anything with this work of mine, so long as you give me proper credit

Show how to build a simple circuit consisting of a battery, a lamp, and a switch, mounting the lamp and switch on a {\it solderless breadboard} (also known as a {\it proto-board}):

$$\epsfbox{01763x01.eps}$$

\underbar{file 01763}
%(END_QUESTION)





%(BEGIN_ANSWER)

This is just one possibility -- there are many others!

$$\epsfbox{01763x02.eps}$$

%(END_ANSWER)





%(BEGIN_NOTES)

Solderless breadboards are extremely useful tools for prototyping circuits in a classroom/lab environment.  Familiarity with their use is a virtual necessity for any modern electronics curriculum.  That being said, it is important not to over-emphasize breadboards, though.  I have seen some electronics courses where breadboards are the only form of circuit construction students ever use!  This does not prepare them for challenges of the job, where breadboards are (rightfully) used only for prototyping.

In summary, use breadboards in your students' labwork, {\it but not all the time}, or even most of the time!

%INDEX% Solderless breadboard, simple circuit built on

%(END_NOTES)


