
%(BEGIN_QUESTION)
% Copyright 2005, Tony R. Kuphaldt, released under the Creative Commons Attribution License (v 1.0)
% This means you may do almost anything with this work of mine, so long as you give me proper credit

\centerline{\bf Wire gauge table}

% No blank lines allowed between lines of an \halign structure!
% I use comments (%) instead, so that TeX doesn't choke.

$$\vbox{\offinterlineskip
\halign{\strut
\vrule \quad\hfil # \ \hfil & 
\vrule \quad\hfil # \ \hfil & 
\vrule \quad\hfil # \ \hfil \vrule \cr
\noalign{\hrule}
%
Gauge \# & Diameter (inches) & Area (circular mils) \cr
%
\noalign{\hrule}
%
4/0 & 0.4600 & 211,600 \cr
%
\noalign{\hrule}
%
3/0 & 0.4100 & 168,100 \cr
%
\noalign{\hrule}
%
2/0 & 0.3650 & 133,225 \cr
%
\noalign{\hrule}
%
1/0 & 0.3250 & 105,625 \cr
%
\noalign{\hrule}
%
1 & 0.2890 & 83,521 \cr
%
\noalign{\hrule}
%
2 & 0.2580 & 66,564 \cr
%
\noalign{\hrule}
%
4 & 0.2040 & 41,616 \cr
%
\noalign{\hrule}
%
6 & 0.1620 & 26,244 \cr
%
\noalign{\hrule}
%
8 & 0.1280 & 16,384 \cr
%
\noalign{\hrule}
%
10 & 0.1020 & 10,404 \cr
%
\noalign{\hrule}
%
12 & 0.0810 & 6,561 \cr
%
\noalign{\hrule}
%
14 & 0.0640 & 4,096 \cr
%
\noalign{\hrule}
%
16 & 0.0510 & 2,601 \cr
%
\noalign{\hrule}
%
18 & 0.0400 & 1,600 \cr
%
\noalign{\hrule}
%
20 & 0.0320 & 1,024 \cr
%
\noalign{\hrule}
%
22 & 0.0253 & 640.1 \cr
%
\noalign{\hrule}
} % End of \halign 
}$$ % End of \vbox


\vskip 50pt



\centerline{\bf Specific resistance table}

% No blank lines allowed between lines of an \halign structure!
% I use comments (%) instead, so that TeX doesn't choke.

$$\vbox{\offinterlineskip
\halign{\strut
\vrule \quad\hfil # \ \hfil & 
\vrule \quad\hfil # \ \hfil & 
\vrule \quad\hfil # \ \hfil \vrule \cr
\noalign{\hrule}
%
Metal type & $\rho$ in $\Omega$ $\cdot$ cmil / ft @ 32$^{o}$F & $\rho$ in $\Omega$ $\cdot$ cmil / ft @ 75$^{o}$F \cr
%
\noalign{\hrule}
%
Silver (pure annealed) & 8.831 & 9.674 \cr
%
\noalign{\hrule}
%
Copper (pure annealed) & 9.390 & 10.351 \cr
%
\noalign{\hrule}
%
Copper (annealed) & 9.590 & 10.505 \cr
%
\noalign{\hrule}
%
Copper (hard-drawn) & 9.810 & 10.745 \cr
%
\noalign{\hrule}
%
Gold (99.9 \% pure) & 13.216 & 14.404 \cr
%
\noalign{\hrule}
%
Aluminum (99.5 \% pure) & 15.219 & 16.758 \cr
%
\noalign{\hrule}
%
Zinc (very pure) & 34.595 & 37.957 \cr
%
\noalign{\hrule}
%
Iron (approx. pure) & 54.529 & 62.643 \cr
%
\noalign{\hrule}
%
Platinum (pure) & 65.670 & 71.418 \cr
%
\noalign{\hrule}
%
Nickel & 74.128 & 85.138 \cr
%
\noalign{\hrule}
%
Tin (pure) & 78.489 & 86.748 \cr
%
\noalign{\hrule}
%
Steel (wire) & 81.179 & 90.150 \cr
%
\noalign{\hrule}
} % End of \halign 
}$$ % End of \vbox

\underbar{file 03387}
%(END_QUESTION)





%(BEGIN_ANSWER)

(There is no answer to give, for this is not even a question!)

%(END_ANSWER)





%(BEGIN_NOTES)

This table "question" exists solely for incorporation into exams or worksheets where students need wire gauge and specific resistance tables for reference.

%INDEX% Table, specific resistance of different metals
%INDEX% Table, wire gauge

%(END_NOTES)


