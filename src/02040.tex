
%(BEGIN_QUESTION)
% Copyright 2003, Tony R. Kuphaldt, released under the Creative Commons Attribution License (v 1.0)
% This means you may do almost anything with this work of mine, so long as you give me proper credit

From an examination of the energy diagram for a BJT in its conducting mode (current existing through each of the three terminals: emitter, base, and collector), determine the biasing of the two PN junctions:

\medskip
\item{$\bullet$} The emitter-base junction ({\it forward} or {\it reverse} biased?)
\item{$\bullet$} The base-collector junction ({\it forward} or {\it reverse} biased?)
\medskip

One of these two junctions actually operates in the reverse-bias mode while the transistor is conducting.  Explain how this is possible, as a simple PN junction (a diode) operating in reverse-bias mode conducts negligible current.

\underbar{file 02040}
%(END_QUESTION)





%(BEGIN_ANSWER)

The emitter-base junction is forward-biased, while the base-collector junction is reverse-biased.  Collector current is made possible across the base-collector junction by the presence of {\it injected charge carriers} from the emitter.

%(END_ANSWER)





%(BEGIN_NOTES)

This question is possible to answer only if one understands the energy levels inside a BJT.  The most common explanations of BJT function I find in introductory (non-engineering) textbooks completely omit discussions of energy levels, making the subject very confusing to new students.

%INDEX% Junction biasing, for BJT during conduction

%(END_NOTES)


