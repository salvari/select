
%(BEGIN_QUESTION)
% Copyright 2005, Tony R. Kuphaldt, released under the Creative Commons Attribution License (v 1.0)
% This means you may do almost anything with this work of mine, so long as you give me proper credit

Suppose an amplifier circuit is connected to a sine-wave signal generator, and a spectrum analyzer used to measure both the input and the output signals of the amplifier:

$$\epsfbox{03307x01.eps}$$

Interpret the two graphical displays and explain why the output signal has more "peaks" than the input.  What is this difference telling us about the amplifier's performance?

\underbar{file 03307}
%(END_QUESTION)





%(BEGIN_ANSWER)

The input signal is clean: a single peak at the 1 kHz mark.  The amplifier's output, on the other hand, is a bit distorted (i.e. no longer a perfect sine-wave shape as the input is).

%(END_ANSWER)





%(BEGIN_NOTES)

The purpose of this question is to get students to realize the presence of harmonics means a departure from a once-perfect sinusoidal wave-shape.  What used to be free of harmonics now contains harmonics, and this indicates distortion of the sine wave somewhere within the amplifier.

By the way, the perfectly flat "noise floor" at -120 dB is highly unusual.  There will always be a "rough" floor shown on the display of a spectrum analyzer, but this is not pertinent to the question at hand so I omitted it for simplicity's sake.

%INDEX% Spectrum analysis, of oscillator circuits

%(END_NOTES)


