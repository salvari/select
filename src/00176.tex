
%(BEGIN_QUESTION)
% Copyright 2003, Tony R. Kuphaldt, released under the Creative Commons Attribution License (v 1.0)
% This means you may do almost anything with this work of mine, so long as you give me proper credit

In this electrical circuit, trace the direction of current through the wires:

$$\epsfbox{00176x01.eps}$$

\underbar{file 00176}
%(END_QUESTION)





%(BEGIN_ANSWER)

This is a "trick" question, as there are two accepted ways of denoting the direction of electric current: {\it conventional flow} (sometimes called {\it hole flow}), and {\it electron flow}.

%(END_ANSWER)





%(BEGIN_NOTES)

This question breaches one of the more contentious subjects in electricity/electronics: which way do we denote the direction of current?  While there is no debate as to which direction {\it electrons} move through a metal conductor carrying current, there are two different conventions for denoting current travel, one of which goes in the direction of electrons and the other which goes against the direction of electrons.  The reason for having these two disparate conventions is embedded in the history of electrical science, and what your students find in their research will likely fuel an interesting conversation.

%INDEX% Conventional flow versus electron flow
%INDEX% Electron flow versus conventional flow
%INDEX% Polarity

%(END_NOTES)


