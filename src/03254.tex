
%(BEGIN_QUESTION)
% Copyright 2005, Tony R. Kuphaldt, released under the Creative Commons Attribution License (v 1.0)
% This means you may do almost anything with this work of mine, so long as you give me proper credit

The formula for calculating voltage across a resistor in a series circuit is as follows:

$$V_R = V_{total}\left(R \over R_{total}\right)$$

In a simple-series circuit with one voltage source and three resistors, we may re-write this formula to be more specific:

$$V_{R1} = V_{source}\left(R_1 \over {R_1 + R_2 + R_3}\right)$$

Suppose we have such a series circuit with a source voltage of 15 volts, and resistor values of $R_1$ = 1 k$\Omega$ and $R_2$ = 8.1 k$\Omega$.  Algebraically manipulate this formula to solve for $R_3$ in terms of all the other variables, then determine the necessary resistance value of $R_3$ to obtain a 0.2 volt drop across resistor $R_1$.

\underbar{file 03254}
%(END_QUESTION)





%(BEGIN_ANSWER)

$$R_3 = V_{source}\left({R_1 \over V_{R1}}\right) - (R_1 + R_2)$$

$R_3$ = 65.9 k$\Omega$

%(END_ANSWER)





%(BEGIN_NOTES)

This question provides students with another practical application of algebraic manipulation.  Ask individual students to show their steps in manipulating the voltage divider formula to solve for $R_3$, so that all may see and learn.

%INDEX% Algebra, manipulating equations
%INDEX% Voltage divider

%(END_NOTES)


