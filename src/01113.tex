
%(BEGIN_QUESTION)
% Copyright 2005, Tony R. Kuphaldt, released under the Creative Commons Attribution License (v 1.0)
% This means you may do almost anything with this work of mine, so long as you give me proper credit

Design a clipper circuit that clips any portion of the input AC waveform below +4 volts:

$$\epsfbox{01113x01.eps}$$

\underbar{file 01113}
%(END_QUESTION)





%(BEGIN_ANSWER)

$$\epsfbox{01113x02.eps}$$

\vskip 10pt

Follow-up question: explain why a {\it Schottky} diode is shown in this circuit rather than a regular silicon PN-junction diode.  What characteristic(s) of Schottky diodes make them well suited for many clipper applications?

%(END_ANSWER)





%(BEGIN_NOTES)

Ask your students whether they would classify this circuit as a {\it series} or a {\it shunt} clipper.

\vskip 10pt

If your students are unfamiliar with Schottky diodes, this is an excellent opportunity to discuss them!  Their low forward voltage drop and fast switching characteristics make them superior for most signal clipper and clamper circuits.

%INDEX% Clipper circuit

%(END_NOTES)


