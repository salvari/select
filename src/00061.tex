
%(BEGIN_QUESTION)
% Copyright 2003, Tony R. Kuphaldt, released under the Creative Commons Attribution License (v 1.0)
% This means you may do almost anything with this work of mine, so long as you give me proper credit

When an electric current passes through a conductor offering some electrical resistance, the temperature of that conductor increases above ambient.  Why is this?  Of what practical importance is this effect?

\underbar{file 00061}
%(END_QUESTION)





%(BEGIN_ANSWER)

Electrical resistance is analogous to mechanical {\it friction}: electrons cannot freely flow through a resistance, and the "friction" they encounter translates some of their energy into heat, just as the friction in a worn mechanical bearing translates some of the kinetic energy of it's rotation into heat, or the friction between a person's hands while rubbing them together on a cold day translates some of the motion into heat.

%(END_ANSWER)





%(BEGIN_NOTES)

This is a good starting point for a discussion on work, energy, and power.  Power, of course, may be directly calculated by multiplying voltage by current, and is measured in {\it watts}.  It also provides an opportunity to discuss some of the practical limitations of electrical conductors.

%INDEX% Conductivity
%INDEX% Resistance, effects of
%INDEX% Joule's Law, conceptual

%(END_NOTES)


