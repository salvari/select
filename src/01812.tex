
%(BEGIN_QUESTION)
% Copyright 2003, Tony R. Kuphaldt, released under the Creative Commons Attribution License (v 1.0)
% This means you may do almost anything with this work of mine, so long as you give me proper credit

Qualitatively determine the voltages across all components as well as the current through all components in this simple LR circuit at three different times: (1) just before the switch closes, (2) at the instant the switch contacts touch, and (3) after the switch has been closed for a long time.

$$\epsfbox{01812x01.eps}$$

Express your answers qualitatively: "maximum," "minimum," or perhaps "zero" if you know that to be the case.

\vskip 10pt
\goodbreak

\noindent
{\bf Before the switch closes:}

$V_{L}$ = 

$V_{R}$ = 

$V_{switch}$ = 

$I$ = 

\vskip 10pt
\goodbreak

\noindent
{\bf At the instant of switch closure:}

$V_{L}$ = 

$V_{R}$ = 

$V_{switch}$ = 

$I$ = 

\vskip 10pt
\goodbreak

\noindent
{\bf Long after the switch has closed:}

$V_{L}$ = 

$V_{R}$ = 

$V_{switch}$ = 

$I$ = 

\vskip 10pt

Hint: a graph may be a helpful tool for determining the answers!

\underbar{file 01812}
%(END_QUESTION)





%(BEGIN_ANSWER)

\noindent
{\bf Before the switch closes:}

$V_{L}$ = zero

$V_{R}$ = zero

$V_{switch}$ = maximum

$I$ = zero

\vskip 10pt
\goodbreak

\noindent
{\bf At the instant of switch closure:}

$V_{L}$ = maximum

$V_{R}$ = zero

$V_{switch}$ = zero

$I$ = zero

\vskip 10pt
\goodbreak

\noindent
{\bf Long after the switch has closed:}

$V_{L}$ = zero

$V_{R}$ = maximum

$V_{switch}$ = zero

$I$ = maximum

\vskip 10pt

Follow-up question: which of these variables remained the same immediately before and immediately after switch closure?  Explain why.

%(END_ANSWER)





%(BEGIN_NOTES)

The purpose of this question is to preview the concept of "initial" and "final" values in RC circuits, before they learn to use the "universal time constant formula."

%INDEX% Time constant calculation, LR circuit, "initial" and "final" values

%(END_NOTES)


