
%(BEGIN_QUESTION)
% Copyright 2005, Tony R. Kuphaldt, released under the Creative Commons Attribution License (v 1.0)
% This means you may do almost anything with this work of mine, so long as you give me proper credit

Shown here is a simple {\it dipole} antenna, comprised of two equal-length wires projecting from the terminals of an RF voltage source:

$$\epsfbox{03458x01.eps}$$

Re-draw this illustration, showing the equivalent inductance and capacitance exhibited by this antenna.  Show these properties using actual inductor and capacitor symbols.

\underbar{file 03458}
%(END_QUESTION)





%(BEGIN_ANSWER)

$$\epsfbox{03458x02.eps}$$

\vskip 10pt

Follow-up question: how would you expect the inductance and capacitance of this antenna to relate to its physical length?  In other words, as you increase the length of an antenna, would its inductance increase or decrease?  As the length increases, would its capacitance increase or decrease?  Explain your reasoning.

%(END_ANSWER)





%(BEGIN_NOTES)

Do not be surprised if some of your students ask whether or not an antenna is capable of resonating, since it possesses both inductance and capacitance.  In fact, this is the subversive point of this question: to get students to realize that even a simple pair of wires may be considered a resonant system, and then to beg the question of what happens at resonance!  The follow-up question suggests a relationship between physical size and resonant frequency, by asking what happens to both L and C as length changes.  Explore these ideas with your students, and watch them gain a surprisingly deep understanding of how an antenna works based on their knowledge of LC resonant circuits.

%INDEX% Dipole antenna, modeled in terms of L and C

%(END_NOTES)


