
%(BEGIN_QUESTION)
% Copyright 2005, Tony R. Kuphaldt, released under the Creative Commons Attribution License (v 1.0)
% This means you may do almost anything with this work of mine, so long as you give me proper credit

Explain how this bridge circuit is capable of being "balanced" for {\it any} values of $R_1$ and $R_2$:

$$\epsfbox{03474x01.eps}$$

\underbar{file 03474}
%(END_QUESTION)





%(BEGIN_ANSWER)

The potentiometer acts as a complementary pair of resistors: moving the wiper one direction increases the value of one while decrease the value of the other.  Thus, it forms a voltage divider with an infinitely adjustable division ratio of 0\% to 100\%, inclusive.

%(END_ANSWER)





%(BEGIN_NOTES)

This question showcases one more use of the potentiometer: as a voltage divider used specifically to balance a bridge circuit for any arbitrary values of fixed resistances.  If students have difficulty seeing how this is possible, you might want to try representing the pot as a pair of fixed resistors ($R_3$ and $R_4$), the wiper position determining the balance of those two resistance values ($R_{pot} = R_1 + R_2$).

%INDEX% Bridge circuit, with potentiometer

%(END_NOTES)


