
%(BEGIN_QUESTION)
% Copyright 2005, Tony R. Kuphaldt, released under the Creative Commons Attribution License (v 1.0)
% This means you may do almost anything with this work of mine, so long as you give me proper credit

$$\epsfbox{02900x01.eps}$$

\underbar{file 02900}
\vfil \eject
%(END_QUESTION)





%(BEGIN_ANSWER)

Use circuit simulation software to verify your predicted and actual truth tables.

%(END_ANSWER)





%(BEGIN_NOTES)

In this activity, students are asked to figure out how to wire the inputs of the J-K flip-flop circuit, and also how to demonstrate the three modes (Set, Reset, and Toggle).  Students will have to properly set up their square-wave signal generators to create a workable clock pulse.  This not only means a clock pulse at the correct voltage levels, but also one that is slow enough to allow them to clearly see the toggling of the flip-flop.

A great thing to do here is have students use a logic probe to sense the clock pulse and compare that frequency with the blinking of the $Q$ and $\overline{Q}$ LEDs.

%INDEX% Assessment, performance-based (J-K flip-flop IC)

%(END_NOTES)


