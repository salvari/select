
%(BEGIN_QUESTION)
% Copyright 2003, Tony R. Kuphaldt, released under the Creative Commons Attribution License (v 1.0)
% This means you may do almost anything with this work of mine, so long as you give me proper credit

Some SCRs and TRIACs are advertised as {\it sensitive-gate} devices.  What does this mean?  What is the difference between a "sensitive gate" SCR and an SCR with a "non-sensitive gate"?

\underbar{file 01093}
%(END_QUESTION)





%(BEGIN_ANSWER)

SCRs and TRIACs with "sensitive gates" resemble the idealized devices illustrated in textbooks.  SCRs and TRIACs with "non-sensitive" gates are intentionally "desensitized" by the addition of an internal loading resistor connected to the gate terminal.

\vskip 10pt

Follow-up question: where would this loading resistor be connected, in the following equivalent diagram for an SCR?

$$\epsfbox{01093x01.eps}$$

%(END_ANSWER)





%(BEGIN_NOTES)

Ask your students why a thyristor such as an SCR or a TRIAC would need to be "de-sensitized" by the addition of a loading resistor?  What is wrong with having a "sensitive" thyristor in a circuit?

%INDEX% SCR, sensitive gate
%INDEX% TRIAC, sensitive gate

%(END_NOTES)


