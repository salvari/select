
%(BEGIN_QUESTION)
% Copyright 2003, Tony R. Kuphaldt, released under the Creative Commons Attribution License (v 1.0)
% This means you may do almost anything with this work of mine, so long as you give me proper credit

Field-effect transistors (FETs) exhibit depletion regions between the oppositely-doped gate and channel sections, just as diodes have depletion regions between the P and N semiconductor halves.  In this illustration, the depletion region appears as a dark, shaded area:

$$\epsfbox{02055x01.eps}$$

Re-draw the depletion regions for the following scenarios, where an external voltage ($V_{GS}$) is applied between the gate and channel:

$$\epsfbox{02055x02.eps}$$

$$\epsfbox{02055x04.eps}$$

Note how the different depletion region sizes affect the conductivity of the transistor's channel.

\underbar{file 02055}
%(END_QUESTION)





%(BEGIN_ANSWER)

$$\epsfbox{02055x03.eps}$$

$$\epsfbox{02055x05.eps}$$

\vskip 10pt

Follow-up question: why do you suppose this type of transistor is called a {\it field-effect} transistor?  What "field" is being referred to in the operation of this device?

%(END_ANSWER)





%(BEGIN_NOTES)

The effect that this external gate voltage has on the effective width of the channel should be obvious, leading students to understand how a JFET allows one signal to exert control over another (the basic principle of {\it any} transistor, field-effect or bipolar).

%INDEX% JFET, effect of gate voltage on channel width

%(END_NOTES)


