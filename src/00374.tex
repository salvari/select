
%(BEGIN_QUESTION)
% Copyright 2003, Tony R. Kuphaldt, released under the Creative Commons Attribution License (v 1.0)
% This means you may do almost anything with this work of mine, so long as you give me proper credit

Components soldered into printed circuit boards often possess "stray" inductance, also known as {\it parasitic} inductance.  Observe this resistor, soldered in to a circuit board:

$$\epsfbox{00374x01.eps}$$

Where does the parasitic inductance come from?  What is it about a resistor, mounted to a circuit board, that creates a (very) small amount of inductance?  How is it possible to minimize this inductance, in case it is detrimental to the circuit's operation?

\underbar{file 00374}
%(END_QUESTION)





%(BEGIN_ANSWER)

Inductance naturally exists along any conductor.  The longer the conductor, the more inductance, all other factors being equal.

%(END_ANSWER)





%(BEGIN_NOTES)

In high-frequency AC circuits, such as computer circuits where pulses of voltage oscillate at millions of cycles per second, even short lengths of wire or traces on a circuit board may present substantial trouble by virtue of their stray inductance.  Some of this parasitic inductance may be reduced by proper circuit board assembly, some of it by a re-design of component layout on the circuit board.

According to an article in \underbar{IEEE Spectrum} magazine ("{\it Putting Passives In Their Place}", July 2003, Volume 40, Number 7, page 29), the transient currents created by fast-switching logic circuits can be as high as 500 amps/ns, which is a $di \over dt$ rate of 500 {\it billion} amps per second!!  At these levels, even a few picohenrys of parasitic inductance along component leads and circuit board traces will result in significant voltage drops.

%INDEX% Inductance, parasitic 

%(END_NOTES)


