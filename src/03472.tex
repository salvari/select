
%(BEGIN_QUESTION)
% Copyright 2005, Tony R. Kuphaldt, released under the Creative Commons Attribution License (v 1.0)
% This means you may do almost anything with this work of mine, so long as you give me proper credit

Suppose that a hand-crank is attached to the shaft of a generator, so that you can turn it to make electricity.  With no load connected to the generator, it is very easy to turn.  However, the generator becomes noticeably more difficult to turn when a low-resistance electrical load is connected to it.  Explain why the physical effort required to turn the generator varies inversely with the load resistance.  Be sure to include all the following principles in your explanation:

\medskip
\item{$\bullet$} Electromagnetic induction (Faraday's Law)
\item{$\bullet$} Ohm's Law
\item{$\bullet$} Electromagnetism
\item{$\bullet$} Lenz's Law
\medskip

\vskip 50pt

\underbar{file 03472}
%(END_QUESTION)





%(BEGIN_ANSWER)

I recommend assigning two points for each of the four principles included in the student's explanation.

\vskip 10pt

As the coils are rotated past the magnetic field of the generator, {\it electromagnetic induction} occurs according to the equation $v = N{d\phi \over dt}$.  Current through the coils is a function of {\it Ohm's Law}: the less resistance, the more current, for any given amount of induced voltage.  This load current in turn generates its own magnetic field according to the principle of {\it electromagnetism}.  However, the polarity of this field must be opposed to the stationary field of the generator in accordance with {\it Lenz's Law}, otherwise the load torque would be in the direction of rotation rather than against it, and the generator would speed up to infinite speed (a clear violation of the Law of Energy Conservation).

%(END_ANSWER)





%(BEGIN_NOTES)

{\bf This question is intended for exams only and not worksheets!}.

%(END_NOTES)


