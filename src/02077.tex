
%(BEGIN_QUESTION)
% Copyright 2004, Tony R. Kuphaldt, released under the Creative Commons Attribution License (v 1.0)
% This means you may do almost anything with this work of mine, so long as you give me proper credit

Students studying AC electrical theory become familiar with the {\it impedance triangle} very soon in their studies:

$$\epsfbox{02077x01.eps}$$

What these students might not ordinarily discover is that this triangle is also useful for calculating electrical quantities other than impedance.  The purpose of this question is to get you to discover some of the triangle's other uses.

Fundamentally, this right triangle represents {\it phasor addition}, where two electrical quantities at right angles to each other (resistive versus reactive) are added together.  In series AC circuits, it makes sense to use the impedance triangle to represent how resistance ($R$) and reactance ($X$) combine to form a total impedance ($Z$), since resistance and reactance are special forms of impedance themselves, and we know that impedances {\it add} in series.

List all of the electrical quantities you can think of that add (in series or in parallel) and then show how similar triangles may be drawn to relate those quantities together in AC circuits.

\underbar{file 02077}
%(END_QUESTION)





%(BEGIN_ANSWER)

\goodbreak

\noindent
{\bf Electrical quantities that add:}

\medskip
\item{$\bullet$} Series impedances
\item{$\bullet$} Series voltages
\item{$\bullet$} Parallel admittances
\item{$\bullet$} Parallel currents
\item{$\bullet$} Power dissipations
\medskip

I will show you one graphical example of how a triangle may relate to electrical quantities other than series impedances:

$$\epsfbox{02077x02.eps}$$

%(END_ANSWER)





%(BEGIN_NOTES)

It is very important for students to understand that the triangle only works as an analysis tool when applied to quantities that {\it add}.  Many times I have seen students try to apply the $Z$-$R$-$X$ impedance triangle to parallel circuits and fail because {\it parallel impedances do not add}.  The purpose of this question is to force students to think about where the triangle is applicable to AC circuit analysis, and not just to use it blindly.

The power triangle is an interesting application of trigonometry applied to electric circuits.  You may not want to discuss power with your students in great detail if they are just beginning to study voltage and current in AC circuits, because power is a sufficiently confusing subject on its own.

%INDEX% Impedance triangle

%(END_NOTES)


