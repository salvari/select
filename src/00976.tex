
%(BEGIN_QUESTION)
% Copyright 2003, Tony R. Kuphaldt, released under the Creative Commons Attribution License (v 1.0)
% This means you may do almost anything with this work of mine, so long as you give me proper credit

High-power PNP transistors tend to be scarcer and more expensive than high-power NPN transistors, a fact which complicates the construction of a high-power complementary push-pull amplifier circuits.  An ingenious solution to this problem is to modify the basic Darlington push-pull circuit, replacing the final PNP transistor with an NPN transistor, like this:

$$\epsfbox{00976x01.eps}$$

The cascaded combination of an NPN and PNP transistor is called a {\it Sziklai pair}, or {\it complementary Darlington pair}.  In this case, the small PNP transistor controls the larger NPN power transistor in the Sziklai pair, performing the same basic function as a PNP Darlington pair.

Modify the circuit shown here to use diodes in the biasing network instead of just resistors.  The solution is not quite the same for this circuit as it is for a conventional Darlington push-pull circuit!

\underbar{file 00976}
%(END_QUESTION)





%(BEGIN_ANSWER)

$$\epsfbox{00976x02.eps}$$

%(END_ANSWER)





%(BEGIN_NOTES)

Discuss the difference between the two halves of this amplifier circuit (the upper Darlington half, and the lower Sziklai half), paying special attention to the number of PN junctions between base and (final) emitter terminals.

%INDEX% Sziklai pair
%INDEX% Push-pull amplifier, using Darlington and Sziklai pairs

%(END_NOTES)


