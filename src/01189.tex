
%(BEGIN_QUESTION)
% Copyright 2003, Tony R. Kuphaldt, released under the Creative Commons Attribution License (v 1.0)
% This means you may do almost anything with this work of mine, so long as you give me proper credit

This electric fence-charging circuit, which is designed to produce short, high-voltage pulses on its output, has failed.  Now, it produces no output voltage at all:

$$\epsfbox{01189x01.eps}$$

A technician does some troubleshooting and determines that the transistor is defective.  She replaces the transistor, and the circuit begins to work again, its rhythmic output pulses indicated by the neon lamp.

But after producing only a few pulses, the circuit stops working.  Puzzled, the technician troubleshoots it again and finds that the transistor has failed (again).  Both the original and the replacement transistor were of the correct part number for this circuit, so the failure is not due to an incorrect component being used.  Something is causing the transistor to fail prematurely.  What do you suppose it is?

\underbar{file 01189}
%(END_QUESTION)





%(BEGIN_ANSWER)

I strongly suspect a bad diode.  Explain why a defective diode would cause the transistor to fail prematurely, and specifically what type of diode failure (open or shorted) would be necessary to cause the transistor to fail in this manner.

%(END_ANSWER)





%(BEGIN_NOTES)

There are many things in this circuit that could prevent it from generating output voltage pulses, but a failed diode (subsequently causing the transistor to fail) is the only problem I can think of which would allow the circuit to briefly function properly after replacing the transistor, and yet fail once more after only a few pulses.  Students will likely suggest other possibilities, so be prepared to explore the consequences of each, determining whether or not the suggested failure(s) would account for {\it all} observed effects.

While your students are giving their reasoning for the diode as a cause of the problem, take some time and analyze the operation of the circuit with them.  How does this circuit use positive feedback to support oscillations?  How could the output pulse rate be altered?  What is the function of each and every component in the circuit?

This circuit provides not only an opportunity to analyze a particular type of amplifier, but it also provides a good review of capacitor, transformer, diode, and transistor theory.

%(END_NOTES)


