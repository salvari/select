
%(BEGIN_QUESTION)
% Copyright 2003, Tony R. Kuphaldt, released under the Creative Commons Attribution License (v 1.0)
% This means you may do almost anything with this work of mine, so long as you give me proper credit

Rank these three light bulb assemblies according to their {\it total} electrical resistance (in order of least to greatest), assuming that each of the bulbs is the same type and rating:

$$\epsfbox{00030x01.eps}$$

Explain how you determined the relative resistances of these light bulb networks.

\underbar{file 00030}
%(END_QUESTION)





%(BEGIN_ANSWER)

\medskip
\item{$\bullet$} C (least total resistance)
\item{$\bullet$} A
\item{$\bullet$} B (greatest total resistance)
\medskip

%(END_ANSWER)





%(BEGIN_NOTES)

I prefer to enter discussion on series and parallel circuits prior to introducing Ohm's Law.  Conceptual analysis tends to be more difficult than numerical analysis in electric circuits, but is a skill worthwhile to build, especially for the sake of effective troubleshooting.

It is effective {\it after} conceptual (qualitative) analysis, though, to go through a numerical (quantitative) analysis of a circuit like this to prove that the concepts are correct, if the students are advanced enough at this point to do series-parallel resistance calculations.

%INDEX% Series versus parallel
%INDEX% Parallel versus series
%INDEX% Resistance, conceptual
%INDEX% Resistances in series
%INDEX% Resistances in parallel

%(END_NOTES)


