
%(BEGIN_QUESTION)
% Copyright 2003, Tony R. Kuphaldt, released under the Creative Commons Attribution License (v 1.0)
% This means you may do almost anything with this work of mine, so long as you give me proper credit

{\it Shunt} resistors used for precision current measurement always have four terminals for the electrical connections, even though normal resistors only have two:

$$\epsfbox{00726x01.eps}$$

Explain what would be wrong with connecting the voltmeter movement directly to the same two terminals conducting high current through the shunt resistor, like this:

$$\epsfbox{00726x02.eps}$$

\underbar{file 00726}
%(END_QUESTION)





%(BEGIN_ANSWER)

A two-wire shunt resistor connection will not be as accurate as a four-wire shunt resistor, due to stray resistance within the bolted connection between the wires and the body of the shunt resistor.

\vskip 10pt

Challenge question: draw a schematic diagram showing all stray resistances within the two-wire shunt connection circuit, in order to clarify the concept.

%(END_ANSWER)





%(BEGIN_NOTES)

Though a few fractions of an ohm of "stray" resistance may not seem like much, they are significant when contrasted against the already (very) low resistance of the shunt resistor's body.

One of the conceptual difficulties I've encountered with students on numerous occasions is confusion over {\it how much} resistance, voltage, current, etc., constitutes a "significant" amount.  For example, I've had students tell me that the difference between 296,342.5 ohms and 296,370.9 ohms is "really big," when in fact it is less than ten thousandths of a percent of the base resistance values.  Students simply subtract the two resistances and obtain 28.4 ohms, then think that "28.4" is a significant quantity because it is comparable to some of the other values they're used to dealing with (100 ohms, 500 ohms, 1000 ohms, etc.).

Conversely, students may fail to see the significance of a few hundredths of an ohm of stray resistance in a shunt resistor circuit, when the entire resistance of the shunt resistor itself is only a few hundredths of an ohm.  What matters most in problems of accuracy is the {\it percentage} or error, not the absolute value of the error itself.  This is another practical application of {\it estimating skills}, which you should reinforce at every opportunity.

%(END_NOTES)


