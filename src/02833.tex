
%(BEGIN_QUESTION)
% Copyright 2005, Tony R. Kuphaldt, released under the Creative Commons Attribution License (v 1.0)
% This means you may do almost anything with this work of mine, so long as you give me proper credit

Draw a schematic for a logic gate circuit using nothing but two-input NOR gates that mimics the operation of this relay circuit:

$$\epsfbox{02833x01.eps}$$

\underbar{file 02833}
%(END_QUESTION)





%(BEGIN_ANSWER)

$$\epsfbox{02833x02.eps}$$

\vskip 10pt

Follow-up question: note the manner in which NOR gates are used as inverters in this circuit.  Compare this against the following (alternative) method:

$$\epsfbox{02833x03.eps}$$

Are there any distinct advantages you see to either method?

%(END_ANSWER)





%(BEGIN_NOTES)

In my very first technical job, I worked as a CNC maintenance technician in a small machine shop, maintaining computer-controlled machine tools such as mills and lathes.  A really neat project I got to work on at that job was the conversion of a 1970's era American-made machine tool to modern Japanese computer control.  A lot of logic in that old machine tool was implemented using relays, and we replaced the cabinets full of relays with solid-state logic in the Japanese control computer.  Actually, the solid state logic was a {\it programmable logic controller} or {\it PLC} function inside the Japanese control computer rather than discrete semiconductor logic gates.  However, we very well could have replaced relays with hard-wired gates.  The purpose of this question, if you haven't guessed by now, is to familiarize students with the concept of replacing electromechanical relays with semiconductor logic gates, especially identical logic gates such as NOR gates which are "universal."

%INDEX% Boolean algebra, conversion of expression into gate logic
%INDEX% Boolean algebra, DeMorgan's Theorem

%(END_NOTES)


