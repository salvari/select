
%(BEGIN_QUESTION)
% Copyright 2003, Tony R. Kuphaldt, released under the Creative Commons Attribution License (v 1.0)
% This means you may do almost anything with this work of mine, so long as you give me proper credit

When a steel manufacturer publishes the magnetic characteristics of their latest alloy, they do so in the form of a "B/H" graph, where flux density ($B$) is plotted as a function of magnetizing force ($H$):

$$\epsfbox{00683x01.eps}$$

Rarely will you see a plot of flux ($\Phi$) shown as as a function of MMF (${\cal F}$), even though such a plot would appear very similar to the "B/H curve" for that same material.  Why is this?  Why would a "B/H" curve be preferable to an engineer rather than a "Flux/MMF" curve? 

\vskip 10pt

Hint: if a copper manufacturer were to publish the electrical characteristics of their latest alloy, they would specify its resistivity in terms of specific resistance ($\rho$), rather than plain resistance ($R$), for the exact same reason!

\underbar{file 00683}
%(END_QUESTION)





%(BEGIN_ANSWER)

A "B/H curve" is independent of the specimen's physical dimensions, communicating the magnetic characteristics of the {\it substance} itself, rather than the characteristics of any one particular {\it piece} of that substance.  

%(END_ANSWER)





%(BEGIN_NOTES)

This concept may confuse some students, so discussion on it is helpful.  Ask your students what "flux density" and "magnetizing force" really mean: they are expressions of flux and MMF {\it per unit dimension}.  So, if a manufacturer states that their new steel alloy will permit a flux density of 0.6 Tesla for an applied magnetizing force of 100 amp-turns/meter, this figure holds true for {\it any} size chunk of that alloy.

To prove this concept via the rhetorical technique of {\it reductio ad absurdum}, ask your students what it would be like if copper manufacturers specified the resistivity of their copper alloys in ohms: "Alloy 123XYZ has a resistivity of 17 ohms."  Of what usefulness is this statement?  What does it mean to us?  How is the statement, "Alloy 123XYZ has a resistivity of 10.5 ohm-cmil per foot," superior?

%INDEX% (Phi) (magnetic) versus B
%INDEX% B versus (Phi)
%INDEX% B-H curve, for ferrous material
%INDEX% Field flux (magnetic) versus flux density
%INDEX% Field force (magnetic) versus magnetizing force
%INDEX% Flux density (magnetic) versus flux 
%INDEX% H versus MMF
%INDEX% Magnetizing force versus field force
%INDEX% MMF versus H

%(END_NOTES)


