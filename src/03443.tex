
%(BEGIN_QUESTION)
% Copyright 2005, Tony R. Kuphaldt, released under the Creative Commons Attribution License (v 1.0)
% This means you may do almost anything with this work of mine, so long as you give me proper credit

Although the voltage divider and current divider equations are very useful in circuit analysis, they are easily confused for one another because they look so similar:

$$V_R = V_{total}\left( {R \over R_{total}} \right) \hbox{\hskip 100pt} I_R = I_{total}\left( {R_{total} \over R} \right)$$

Specifically, it is easy to forget which way the resistance fraction goes for each one.  Is it ${R \over R_{total}}$ or is it ${R_{total} \over R}$ ?  Simply trying to memorize which fraction form goes with which equation is a bad policy, since memorization of arbitrary forms tends to be unreliable.  What is needed is recognition of some sort of {\it principle} that makes the form of each equation sensible.  In other words, each equation needs to make sense {\it why} it is the way it is.

Explain how one would be able to tell that the following equations are wrong, without referring to a book:

$$V_R = V_{total}\left( {R_{total} \over R} \right) \hbox{\hskip 20pt {\bf Wrong equations!} \hskip 20pt} I_R = I_{total}\left( {R \over R_{total}} \right)$$

\underbar{file 03443}
%(END_QUESTION)





%(BEGIN_ANSWER)

The resistance fraction must always be less than 1.

%(END_ANSWER)





%(BEGIN_NOTES)

Note that I do not explain why the fraction is less than 1 for each equation.  I leave it to the students to figure out that $R < R_{total}$ for series circuits and $R_{total} < R$ for parallel circuits.

Many students have the unfortunate tendency to {\it memorize} in favor of {\it understand}, especially when it comes to equations.  Comprehension is, of course, far superior, and exercises like this one help students build comprehension of the formulae they use in electronics.

%INDEX% Current divider formula versus voltage divider formula
%INDEX% Voltage divider formula versus current divider formula

%(END_NOTES)


