
%(BEGIN_QUESTION)
% Copyright 2003, Tony R. Kuphaldt, released under the Creative Commons Attribution License (v 1.0)
% This means you may do almost anything with this work of mine, so long as you give me proper credit

The only way to consistently guarantee a repetitive waveform will appear "still" on an analog oscilloscope screen is for each left-to-right sweep of the CRT's electron beam to begin at the same point on the waveform.  Explain how the "trigger" system on an oscilloscope works to accomplish this. 

\underbar{file 01905}
%(END_QUESTION)





%(BEGIN_ANSWER)

The "trigger" circuitry on an oscilloscope initiates each left-to-right sweep of the electron beam only when certain conditions are met.  Usually, these conditions are that the input signal being measured attains a specified voltage level (set by the technician), in a specified direction (either increasing or decreasing).  Other conditions for triggering are possible, though.

%(END_ANSWER)





%(BEGIN_NOTES)

Triggering is a complex feature for students to grasp in even simple analog oscilloscopes.  Spend as much time with students as you must to give them understanding in this area, as it will be very useful in their labwork and eventually in their careers.

%INDEX% Oscilloscope triggering, basic concepts

%(END_NOTES)


