
%(BEGIN_QUESTION)
% Copyright 2003, Tony R. Kuphaldt, released under the Creative Commons Attribution License (v 1.0)
% This means you may do almost anything with this work of mine, so long as you give me proper credit

Manipulate this equation to solve for resistor value $R_1$, given the values of $R_2$ and $R_{parallel}$:

$$R_{parallel} = {{R_1 R_2} \over {R_1 + R_2}}$$

Then, give an example of a practical situation where you might use this new equation.

\underbar{file 01988}
%(END_QUESTION)





%(BEGIN_ANSWER)

$$R_1 = {{R_2 R_{parallel}} \over {R_2 - R_{parallel}}}$$

I'll let you figure out a situation where this equation would be useful!

%(END_ANSWER)





%(BEGIN_NOTES)

This question is really nothing more than an exercise in algebraic manipulation.

%INDEX% Algebra, manipulating equations

%(END_NOTES)


