
%(BEGIN_QUESTION)
% Copyright 2003, Tony R. Kuphaldt, released under the Creative Commons Attribution License (v 1.0)
% This means you may do almost anything with this work of mine, so long as you give me proper credit

A new residential neighborhood is being built, and you are working as a member of the construction crew.  One day, when the heating technicians are on-site doing checks of the gas furnaces in each new house, they report that the last house in the neighborhood does not have natural gas.  The piping is installed, of course, but when they turn the gas valve on nothing comes out.

The gas pipeline servicing these houses is laid out in the following manner:

$$\epsfbox{01595x01.eps}$$

Each black dot on the diagram is a shutoff valve, used for isolating different sections of the service pipeline.  Based on the heating technician's report, you conclude that the service pipeline going up to that house must not be "live," and that one of the numbered valves was probably left in the {\it off} position.  But which one could it be?

You know that the main utility connection at the street is "live," because the gas heater in the contractor mobile building is working just fine.  You decide to go to valve \#8 and check for gas pressure at that point in the pipeline with a portable pressure gauge, then checking the pressure at each valve location down the pipeline until you find where there is good gas pressure.  However, before you step out of the room to go do this, one of your co-workers suggests you start your search at the middle point of the pipeline instead: at the location of valve \#4.

Explain why your co-worker's idea is better, and also what your next step would be if: (a) you did find pressure at that point, and (b) if you did {\it not} find pressure at that point.

\underbar{file 01595}
%(END_QUESTION)





%(BEGIN_ANSWER)

Your co-worker's strategy is based on the principle of dividing the gas pipeline into halves, and checking for pressure at the half-way point.  This troubleshooting strategy is sometimes referred to as the "divide-and-conquer" method, because it divides the system into small sections to optimize troubleshooting time and effort.

%(END_ANSWER)





%(BEGIN_NOTES)

This problem gives students a chance to explore the "divide and conquer" strategy of troubleshooting in a context that is very simple and does not require knowledge of electricity.

%INDEX% Troubleshooting, gas pipeline example
%INDEX% Troubleshooting strategy, "divide and conquer"

%(END_NOTES)


