
%(BEGIN_QUESTION)
% Copyright 2006, Tony R. Kuphaldt, released under the Creative Commons Attribution License (v 1.0)
% This means you may do almost anything with this work of mine, so long as you give me proper credit

Common-emitter, common-collector, and common-base amplifier circuits are sometimes referred to as {\it grounded-emitter}, {\it grounded-collector}, and {\it grounded-base}, respectively, because these configurations may actually be built with those respective terminals connected straight to ground.

Although this may not be very practical for ease of biasing, it can be done.  Draw the rest of the circuit necessary to provide class-A operation for each of these (partial) transistor circuits.  Be sure to show where the DC power source, signal input, and signal output connect:

\vskip 50pt

$$\epsfbox{03871x01.eps}$$

\underbar{file 03871}
%(END_QUESTION)





%(BEGIN_ANSWER)

$$\epsfbox{03871x02.eps}$$

%(END_ANSWER)





%(BEGIN_NOTES)

Although it is more common in modern times to refer to the three BJT amplifier configurations as common-(e, c, b) rather than as {\it grounded}-(e, c, b), it may help some students grasp why the word "common" came to be used.  "Grounded" makes more literal sense, and seeing these three circuit configurations with directly grounded terminals may serve as a starting point for identifying configurations where the "common" terminals are not directly grounded.

%INDEX% Amplifier configurations, CE versus CB versus CC
%INDEX% Amplifier configurations, grounded-E versus grounded-C versus grounded-B

%(END_NOTES)


