
%(BEGIN_QUESTION)
% Copyright 2003, Tony R. Kuphaldt, released under the Creative Commons Attribution License (v 1.0)
% This means you may do almost anything with this work of mine, so long as you give me proper credit

In this series-parallel circuit, resistors R1 and R2 are in series with each other, but resistor R3 is neither in series nor in parallel with either R1 or R2:

$$\epsfbox{01755x01.eps}$$

Normally, the first step in mathematically analyzing a circuit such as this is to determine the total circuit resistance.  In other words, we need to calculate how much resistance the voltage source "sees" in the network formed by R1, R2, and R3.  If the circuit were a simple series configuration, our task would be easy:

$$\epsfbox{01755x02.eps}$$

Likewise, if the circuit were a simple parallel configuration, we would have no difficulty at all calculating total resistance:

$$\epsfbox{01755x03.eps}$$

Due to the fact that our given circuit is neither purely series nor purely parallel, though, calculation of total resistance is not a simple one-step operation.  However, there is a way we could simplify the circuit to something that {\it is} either simple series or simple parallel.  Describe how that might be done, and demonstrate using numerical values for resistors R1, R2, and R3.

\underbar{file 01755}
%(END_QUESTION)





%(BEGIN_ANSWER)

Suppose we had these resistor values:

\medskip
\item{$\bullet$} R1 = 3000 $\Omega$
\item{$\bullet$} R2 = 2000 $\Omega$
\item{$\bullet$} R3 = 5000 $\Omega$
\medskip

The total resistance in this case would be 2500 $\Omega$.  I'll let you figure out how to do this!  

\vskip 10pt

{\it Hint: 2.5k is exactly one-half of 5k}

%(END_ANSWER)





%(BEGIN_NOTES)

Figuring out how to calculate total resistance in a series-parallel network is an exercise in problem-solving.  Students must determine how to convert a complex problem into multiple, simpler problems which they can then solve with the tools they have.

This sort of exercise is also helpful in getting students to think in terms of {\it incremental} problem-solving.  Being able to take sections of a circuit and reduce them to equivalent component values so that the circuit becomes simpler and simpler to analyze is a very important skill in electronics.

%INDEX% Resistance (total) of series-parallel network
%INDEX% Series-parallel network, calculating total resistance

%(END_NOTES)


