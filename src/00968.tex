
%(BEGIN_QUESTION)
% Copyright 2003, Tony R. Kuphaldt, released under the Creative Commons Attribution License (v 1.0)
% This means you may do almost anything with this work of mine, so long as you give me proper credit

The circuit shown here is a standard {\it push-pull} amplifier, comprised of a complementary pair of bipolar junction transistors:

$$\epsfbox{00968x01.eps}$$

Trace current in this circuit during periods of time when the instantaneous voltage of the signal source ($V_{in}$) is positive, and for those periods when it is negative.  Determine at which times each of the transistors is "on" (conducting current).

\underbar{file 00968}
%(END_QUESTION)





%(BEGIN_ANSWER)

$$\epsfbox{00968x02.eps}$$

\vskip 10pt

Follow-up question: would you classify this amplifier circuit as common-emitter, common-collector, or common-base?  What kind of voltage gain would you expect this amplifier circuit to have?

%(END_ANSWER)





%(BEGIN_NOTES)

Have your students trace current in the circuit on a diagram drawn on the whiteboard, so all can see the analysis.  After analyzing its operation, ask them why they think this amplifier is called "push-pull."  

Another way to approach this circuit is from the perspective of current {\it sourcing} and current {\it sinking}.  Which transistor sources current to the load resistor, and which transistor sinks current from the load resistor?

%INDEX% Push-pull amplifier, complementary transistor pair (NPN and PNP)

%(END_NOTES)


