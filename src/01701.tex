
%(BEGIN_QUESTION)
% Copyright 2003, Tony R. Kuphaldt, released under the Creative Commons Attribution License (v 1.0)
% This means you may do almost anything with this work of mine, so long as you give me proper credit

A co-worker of mine was once installing sheet-styrofoam insulation in the walls of his new garage.  The sheets of insulation came wrapped together in bundles for easy transport and storage.

One side of each sheet was covered in aluminum foil.  This acted as a reflective surface for infra-red (heat) radiation, and improved the effectiveness of the insulating panel.  It also happened to function as a conductor of electricity, much to the dismay of my friend.

My friend found that the sheets were physically stuck together by the force of static electric charges when they were delivered to his house.  Upon trying to pry two of them apart from each other, he received a surprisingly large electric shock from touching the aluminum foil surfaces of the respective panels!  Explain how the physical work he did in separating the panels from each other became manifest as voltage, based on your knowledge of voltage, work, and electric charges.

\underbar{file 01701}
%(END_QUESTION)





%(BEGIN_ANSWER)

Voltage is defined as work per unit charge, which tells us that this fellow's physical work of separating the attractive charges became translated into electrical voltage, which shocked him when the magnitude became great enough.

%(END_ANSWER)





%(BEGIN_NOTES)

By the way, this is an absolutely true story!  This scenario could be easily replicated on a dry (low-humidity) day, with two large sheets of styrofoam insulation and a brave person pulling them apart!

%INDEX% Voltage, related to the forced separation of attracting charges

%(END_NOTES)


