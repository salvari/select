
%(BEGIN_QUESTION)
% Copyright 2003, Tony R. Kuphaldt, released under the Creative Commons Attribution License (v 1.0)
% This means you may do almost anything with this work of mine, so long as you give me proper credit

As an electronics instructor, I have the opportunity to see a lot of creative mistakes made by students as they learn to build circuits.  One very common mistake made in CMOS circuit construction manifests itself in erratic behavior: the circuit may function correctly for a time, but suddenly and randomly it stops.  Then, just by waving your hand next to the circuit, it begins to work again!

This problem is especially prevalent on days where the atmospheric humidity is low, and static electric charges easily accumulate on objects and people.  Explain what sort of CMOS wiring mistake would cause a powered logic gate to behave erratically due to nearby static electric fields, and what the proper solution is to this problem.

\underbar{file 01278}
%(END_QUESTION)





%(BEGIN_ANSWER)

This classic problem is caused by a lack of pullup or pulldown resistors on CMOS gate inputs.

%(END_ANSWER)





%(BEGIN_NOTES)

Students think I'm a wizard by being able to troubleshoot their CMOS circuits just by waving my hand next to them.  No, I'm just wise in the ways of common student error!

%INDEX% Floating input, CMOS

%(END_NOTES)


