
%(BEGIN_QUESTION)
% Copyright 2003, Tony R. Kuphaldt, released under the Creative Commons Attribution License (v 1.0)
% This means you may do almost anything with this work of mine, so long as you give me proper credit

Arc flash may occur when using a voltmeter to measure voltage in a 480 volt electrical system, if there happens to be a power line transient (from a lightning strike, perhaps) at that time.  Some industrial voltmeters carry a "CAT" rating (CAT I, CAT II, CAT III, CAT IV, etc.) specifying their tolerance for overvoltage transients, and subsequent safety provisions.

Describe in detail the meaning of these "CAT" ratings, and the nature of the hazard posed by a meter that fails under transient conditions.

\underbar{file 00567}
%(END_QUESTION)





%(BEGIN_ANSWER)

There is a substantial amount of information available on this subject on the internet, and through meter manufacturer publications.  I will leave the research to you!

%(END_ANSWER)





%(BEGIN_NOTES)

This question makes the point that personal protective equipment (PPE) is not the only line of defense against the hazards of arc flash.  Test instruments also play an important role in electrical safety (or may contribute to the danger!).  It is very important that electrical workers use quality tools for their work.  A CAT-rated multimeter will cost more than one without a similar safety rating, but tell your students that the medical expenses incurred by an arc flash accident are far greater!

%INDEX% Arc flash
%INDEX% CAT ratings of meters
%INDEX% Meter, "CAT" rating of 

%(END_NOTES)


