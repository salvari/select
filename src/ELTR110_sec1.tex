
\centerline{\bf ELTR 110 (AC 1), section 1} \bigskip 
 
\vskip 10pt

\noindent
{\bf Recommended schedule}

\vskip 5pt

%%%%%%%%%%%%%%%
\hrule \vskip 5pt
\noindent
\underbar{Day 1}

\hskip 10pt Topics: {\it Basic concepts of AC and oscilloscope usage}
 
\hskip 10pt Questions: {\it 1 through 20}
 
\hskip 10pt Lab Exercise: {\it Analog oscilloscope set-up (question 81)}
 
%INSTRUCTOR \hskip 10pt {\bf Demo: function generator and speaker to show what AC "sounds like"}

%INSTRUCTOR \hskip 10pt {\bf Demo: function generator and oscilloscope}

\vskip 10pt
%%%%%%%%%%%%%%%
\hrule \vskip 5pt
\noindent
\underbar{Day 2}

\hskip 10pt Topics: {\it RMS quantities, phase shift, and phasor addition}
 
\hskip 10pt Questions: {\it 21 through 40}
 
\hskip 10pt Lab Exercise: {\it RMS versus peak measurements (question 82) and measuring frequency (question 83)}
 
%INSTRUCTOR \hskip 10pt {\bf Socratic Electronics animation: Lissajous figures on an oscilloscope}

%INSTRUCTOR \hskip 10pt {\bf Demo: two function generators and an oscilloscope to show Lissajous figures}

\vskip 10pt
%%%%%%%%%%%%%%%
\hrule \vskip 5pt
\noindent
\underbar{Day 3}

\hskip 10pt Topics: {\it Inductive reactance and impedance, trigonometry for AC circuits}
 
\hskip 10pt Questions: {\it 41 through 60}
 
\hskip 10pt Lab Exercise: {\it Inductive reactance and Ohm's Law for AC (question 84)}
 
%INSTRUCTOR \hskip 10pt {\bf Demo: function generator, inductor, and multimeter to show inductive reactance}

\vskip 10pt
%%%%%%%%%%%%%%%
\hrule \vskip 5pt
\noindent
\underbar{Day 4}

\hskip 10pt Topics: {\it Series and parallel LR circuits}
 
\hskip 10pt Questions: {\it 61 through 80}
 
\hskip 10pt Lab Exercise: {\it Series LR circuit (question 85)}
 
\vskip 10pt
%%%%%%%%%%%%%%%
\hrule \vskip 5pt
\noindent
\underbar{Day 5}

\hskip 10pt Exam 1: {\it includes Inductive reactance performance assessment}
 
\hskip 10pt Lab Exercise: {\it Oscilloscope probe ($\times$ 10) compensation (question 86)}

\vskip 10pt
%%%%%%%%%%%%%%%
\hrule \vskip 5pt
\noindent
\underbar{Practice and challenge problems}

\hskip 10pt Questions: {\it 89 through the end of the worksheet}
 
\vskip 10pt
%%%%%%%%%%%%%%%
\hrule \vskip 5pt
\noindent
\underbar{Impending deadlines}

\hskip 10pt {\bf Troubleshooting assessment (AC bridge circuit) due at end of ELTR110, Section 3}
 
\hskip 10pt Question 87: Troubleshooting log
 
\hskip 10pt Question 88: Sample troubleshooting assessment grading criteria
 
\vskip 10pt
%%%%%%%%%%%%%%%







\vfil \eject

\centerline{\bf ELTR 110 (AC 1), section 1} \bigskip 
 
\vskip 10pt

\noindent
{\bf Skill standards addressed by this course section}

\vskip 5pt

%%%%%%%%%%%%%%%
\hrule \vskip 10pt
\noindent
\underbar{EIA {\it Raising the Standard; Electronics Technician Skills for Today and Tomorrow}, June 1994}

\vskip 5pt

\medskip
\item{\bf C} {\bf Technical Skills -- AC circuits}
\item{\bf C.01} Demonstrate an understanding of sources of electricity in AC circuits.
\item{\bf C.02} Demonstrate an understanding of the properties of an AC signal.
\item{\bf C.03} Demonstrate an understanding of the principles of operation and characteristics of sinusoidal and non-sinusoidal wave forms.
\item{\bf C.05} Demonstrate an understanding of measurement of power in AC circuits.
\item{\bf C.11} Understand principles and operations of AC inductive circuits.
\item{\bf C.12} Fabricate and demonstrate AC inductive circuits.
\item{\bf C.13} Troubleshoot and repair AC inductive circuits.
\medskip

\vskip 5pt

\medskip
\item{\bf B} {\bf Basic and Practical Skills -- Communicating on the Job}
\item{\bf B.01} Use effective written and other communication skills.  {\it Met by group discussion and completion of labwork.}
\item{\bf B.03} Employ appropriate skills for gathering and retaining information.  {\it Met by research and preparation prior to group discussion.}
\item{\bf B.04} Interpret written, graphic, and oral instructions.  {\it Met by completion of labwork.}
\item{\bf B.06} Use language appropriate to the situation.  {\it Met by group discussion and in explaining completed labwork.}
\item{\bf B.07} Participate in meetings in a positive and constructive manner.  {\it Met by group discussion.}
\item{\bf B.08} Use job-related terminology.  {\it Met by group discussion and in explaining completed labwork.}
\item{\bf B.10} Document work projects, procedures, tests, and equipment failures.  {\it Met by project construction and/or troubleshooting assessments.}
\item{\bf C} {\bf Basic and Practical Skills -- Solving Problems and Critical Thinking}
\item{\bf C.01} Identify the problem.  {\it Met by research and preparation prior to group discussion.}
\item{\bf C.03} Identify available solutions and their impact including evaluating credibility of information, and locating information.  {\it Met by research and preparation prior to group discussion.}
\item{\bf C.07} Organize personal workloads.  {\it Met by daily labwork, preparatory research, and project management.}
\item{\bf C.08} Participate in brainstorming sessions to generate new ideas and solve problems.  {\it Met by group discussion.}
\item{\bf D} {\bf Basic and Practical Skills -- Reading}
\item{\bf D.01} Read and apply various sources of technical information (e.g. manufacturer literature, codes, and regulations).  {\it Met by research and preparation prior to group discussion.}
\item{\bf E} {\bf Basic and Practical Skills -- Proficiency in Mathematics}
\item{\bf E.01} Determine if a solution is reasonable.
\item{\bf E.02} Demonstrate ability to use a simple electronic calculator.
\item{\bf E.05} Solve problems and [sic] make applications involving integers, fractions, decimals, percentages, and ratios using order of operations.
\item{\bf E.06} Translate written and/or verbal statements into mathematical expressions.
\item{\bf E.09} Read scale on measurement device(s) and make interpolations where appropriate.  {\it Met by oscilloscope usage.}
\item{\bf E.12} Interpret and use tables, charts, maps, and/or graphs.
\item{\bf E.13} Identify patterns, note trends, and/or draw conclusions from tables, charts, maps, and/or graphs.
\item{\bf E.15} Simplify and solve algebraic expressions and formulas.
\item{\bf E.16} Select and use formulas appropriately.
\item{\bf E.17} Understand and use scientific notation.
\item{\bf E.20} Graph functions.
\item{\bf E.26} Apply Pythagorean theorem.
\item{\bf E.27} Identify basic functions of sine, cosine, and tangent.
\item{\bf E.28} Compute and solve problems using basic trigonometric functions.
\medskip

%%%%%%%%%%%%%%%




\vfil \eject

\centerline{\bf ELTR 110 (AC 1), section 1} \bigskip 
 
\vskip 10pt

\noindent
{\bf Common areas of confusion for students}

\vskip 5pt

%%%%%%%%%%%%%%%
\hrule \vskip 5pt

\vskip 10pt

\noindent
{\bf Difficult concept: } {\it RMS versus peak and average measurements.}

The very idea of assigning a fixed number for AC voltage or current that (by definition) constantly changes magnitude and direction seems strange.  Consequently, there is more than one way to do it.  We may assign that value according to the {\it highest} magnitude reached in a cycle, in which case we call it the {\it peak} measurement.  We may mathematically integrate the waveform over time to figure the mean magnitude, in which case we call it the {\it average} measurement.  Or we may figure out what level of DC (voltage or current) causes the exact same amount of average power to be dissipated by a standard resistive load, in which case we call it the {\it RMS} measurement.  One common mistake here is to think that the relationship between RMS, average, and peak measurements is a matter of fixed ratios.  The number "0.707" is memorized by every beginning electronics student as the ratio between RMS and peak, but what is commonly overlooked is that this particular ratio holds true {\it for perfect sine-waves only!}  A wave with a different shape will have a different mathematical relationship between peak and RMS values.

\vskip 10pt

\noindent
{\bf Difficult concept: } {\it Resistance versus Reactance versus Impedance.}

These three terms represent different forms of opposition to electric current.  Despite the fact that they are measured in the same unit (ohms: $\Omega$), they are not the same.  Resistance is best thought of as electrical {\it friction}, whereas reactance is best thought of as electrical {\it inertia}.  Whereas resistance creates a voltage drop by dissipating energy, reactance creates a voltage drop by {\it storing} and {\it releasing} energy.  Impedance is a term encompassing both resistance and reactance, usually a combination of both.

\vskip 10pt

\noindent
{\bf Difficult concept: } {\it Phasors, used to represent AC amplitude and phase relations.}

A powerful tool used for understanding the operation of AC circuits is the {\it phasor diagram}, consisting of arrows pointing in different directions: the length of each arrow representing the amplitude of some AC quantity (voltage, current, or impedance), and the angle of each arrow representing the shift in phase relative to the other arrows.  By representing each AC quantity thusly, we may more easily calculate their relationships to one another, with the phasors showing us how to apply trigonometry (Pythagorean Theorem, sine, cosine, and tangent functions) to the various calculations.  An analytical parallel to the graphic tool of phasor diagrams is {\it complex numbers}, where we represent each phasor (arrow) by a pair of numbers: either a magnitude and angle (polar notation), or by "real" and "imaginary" magnitudes (rectangular notation).  Where phasor diagrams are helpful is in applications where their respective AC quantities {\it add}: the resultant of two or more phasors stacked tip-to-tail being the mathematical sum of the phasors.  Complex numbers, on the other hand, may be added, subtracted, multiplied, and divided; the last two operations being difficult to graphically represent with arrows.

\vskip 10pt

\noindent
{\bf Difficult concept: } {\it Conductance, susceptance, and admittance.}

Conductance, symbolized by the letter $G$, is the mathematical reciprocal of resistance ($1 \over R$).  Students typically encounter this quantity in their DC studies and quickly ignore it.  In AC calculations, however, conductance and its AC counterparts ({\it susceptance}, the reciprocal of reactance $B = {1 \over X}$ and {\it admittance}, the reciprocal of impedance $Y = {1 \over Z}$) are very necessary in order to draw phasor diagrams for parallel networks.

\vskip 10pt

\noindent
{\bf Common mistake: } {\it Common ground connections on oscilloscope inputs.}

Oscilloscopes having more than one input "channel" share common ground connections between these channels.  That is to say, with two or more input cables plugged into an oscilloscope, the "ground" clip of each input cable is electrically common with the ground clip of every other input cable.  This can easily cause problems, as points in a circuit connected by multiple input cable ground clips will be made common with each other (as well as common with the oscilloscope case, which itself is connected to earth ground).  One way to avoid unintentional short-circuits through these ground connections is to only connect {\it one} ground clip of the oscilloscope to the circuit ground, removing or tying back all the other inputs' ground clips since they are redundant.


