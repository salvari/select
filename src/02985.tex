
%(BEGIN_QUESTION)
% Copyright 2005, Tony R. Kuphaldt, released under the Creative Commons Attribution License (v 1.0)
% This means you may do almost anything with this work of mine, so long as you give me proper credit

A helpful analogy for a shift register is a {\it conveyor belt}.  Examine this illustration showing a single conveyor belt at four different times, and determine which of the following shift register operations the sequence represents:

\medskip
\item{$\bullet$} Parallel-in, serial-out
\item{$\bullet$} Parallel-in, parallel-out
\item{$\bullet$} Serial-in, serial-out
\item{$\bullet$} Serial-in, parallel-out
\medskip

$$\epsfbox{02985x01.eps}$$

\underbar{file 02985}
%(END_QUESTION)





%(BEGIN_ANSWER)

This is a {\it parallel-in, serial-out} shift register analogy, with each box arriving on the conveyor belt one at a time, but leaving together as a group.

%(END_ANSWER)





%(BEGIN_NOTES)

Some analogies can be very helpful to students as they learn new concepts.  I have found that conveyor belts work very well to illustrate the different types of shift register behaviors.

%INDEX% Shift register, analogous to conveyor belt

%(END_NOTES)


