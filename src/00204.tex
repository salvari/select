
%(BEGIN_QUESTION)
% Copyright 2003, Tony R. Kuphaldt, released under the Creative Commons Attribution License (v 1.0)
% This means you may do almost anything with this work of mine, so long as you give me proper credit

Electric fields may be described as "invisible webs" of interaction across space between electrically charged objects.  Most people should be familiar with {\it magnetic} fields from playing with magnets as children: the forces of attraction or repulsion that act across open space between two or more magnetic objects.  But electric fields are not the same as magnetic fields.  The two different kinds of fields exert forces on entirely different objects.

Give an example of where an electric field manifests a tangible, physical force, like the magnetic fields we are all familiar with.  Under what conditions are electric fields strong enough for human beings to detect without instruments?

\underbar{file 00204}
%(END_QUESTION)





%(BEGIN_ANSWER)

"Static cling," where articles of clothing are attracted to one another after being dried in a machine, is an example of an electric field strong enough to produce tangible, physical attraction over a distance.  Another, similar effect is that of peoples' hair standing on end prior to a lightning strike.

In both cases, what condition causes such a strong electric field to develop?

%(END_ANSWER)





%(BEGIN_NOTES)

Electric field force is also used in some precision voltage measuring instruments ("electrostatic" meter movements), as well as the more common {\it electroscope}.  If you happen to have either an electrostatic meter movement or an electroscope available in your classroom, use it to demonstrate the physical effects of electric fields.

%INDEX% Electric field
%INDEX% Field, electric

%(END_NOTES)


