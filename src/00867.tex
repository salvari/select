
%(BEGIN_QUESTION)
% Copyright 2003, Tony R. Kuphaldt, released under the Creative Commons Attribution License (v 1.0)
% This means you may do almost anything with this work of mine, so long as you give me proper credit

A {\it Class-A} transistor amplifier uses a single transistor to generate an output signal to a load.  The amplifier shown here happens to be of the "common collector" topology, one of three configurations common to single-transistor circuits:

$$\epsfbox{00867x01.eps}$$

An analogue for this electronic circuit is this water-pressure control, consisting of a variable valve passing water through an orifice (a restriction), then on to a drain:

$$\epsfbox{00867x02.eps}$$

The "input" to this amplifier is the positioning of the valve control handle.  The "output" of this amplifier is water pressure measured at the end of the horizontal "output" pipe.

\vskip 10pt

Explain how either of these "circuits" meets the criteria of being an amplifier.  In other words, explain how {\it power} is boosted from input to output in both these systems.  Also, describe how efficient each of these amplifiers is, "efficiency" being a measure of how much current (or water) goes to the load device, as compared to how much just goes straight through the controlling element and back to ground (the drain).

\underbar{file 00867}
%(END_QUESTION)





%(BEGIN_ANSWER)

In both systems, a small amount of energy (current through the "base" terminal of the transistor, mechanical motion of the valve handle) exerts control over a larger amount of energy (current to the load, water to the load).  The systems shown here are rather wasteful, especially at high output voltage (pressure).

%(END_ANSWER)





%(BEGIN_NOTES)

Wasteful they may be, but "Class-A" transistor circuits find very common use in modern electronics.  Explain to your students that its inefficiency restricts its practical use to low-power applications.

%INDEX% Single-ended amplifier, water valve analogy for

%(END_NOTES)


