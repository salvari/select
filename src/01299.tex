
%(BEGIN_QUESTION)
% Copyright 2003, Tony R. Kuphaldt, released under the Creative Commons Attribution License (v 1.0)
% This means you may do almost anything with this work of mine, so long as you give me proper credit

Surveying the rules for Boolean multiplication, the 0 and 1 values seem to resemble the truth table of a very common logic gate.  Which type of gate is this, and what does this suggest about the relationship between Boolean multiplication and logic circuits?

$$\hbox{\bf{Rules for Boolean multiplication:}}$$

$$0 \times 0 = 0$$

$$0 \times 1 = 0$$

$$1 \times 0 = 0$$

$$1 \times 1 = 1$$

\underbar{file 01299}
%(END_QUESTION)





%(BEGIN_ANSWER)

This set of Boolean expressions resembles the truth table for an AND logic gate circuit, suggesting that Boolean multiplication may symbolize the logical AND function.

%(END_ANSWER)





%(BEGIN_NOTES)

Students need to be able to readily associate fundamental Boolean operations with logic circuits.  If they can see the relationship between the "strange" rules of Boolean arithmetic and something they are already familiar with (i.e. truth tables), the association is made much easier.

%INDEX% Boolean multiplication, equivalence to the AND logic function

%(END_NOTES)


