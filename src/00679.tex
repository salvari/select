
%(BEGIN_QUESTION)
% Copyright 2003, Tony R. Kuphaldt, released under the Creative Commons Attribution License (v 1.0)
% This means you may do almost anything with this work of mine, so long as you give me proper credit

Suppose that a length of electrical wire is wrapped around a section of an iron torus, and electric current conducted through the wire:

$$\epsfbox{00679x01.eps}$$

What factors influence the amount of MMF, flux, and reluctance in this magnetic "circuit"?

\underbar{file 00679}
%(END_QUESTION)





%(BEGIN_ANSWER)

MMF is determined by the amount of current through the wire coil, and the number of turns in the coil (${\cal F} = I N$).  Reluctance is determined by the cross-sectional area of the magnetic flux path, the length of that path, the type of material the torus is made of, {\it and the amount of flux present in the torus}.  Magnetic flux is determined by the MMF and reluctance.

\vskip 10pt

Follow-up question: how similar are these relationships to voltage, resistance, and current in an electrical circuit?  Note any similarities as well as any differences.

%(END_ANSWER)





%(BEGIN_NOTES)

Perhaps the most interesting part of the answer to this question is that magnetic reluctance ($\Re$) changes with the amount of flux ($\Phi$) in the "circuit".  At first, this may seem quite different from electrical circuits, where resistance ($R$) is constant regardless of current ($I$).

However, the constancy of electrical resistance is something easily taken for granted.  Ask your students to think of electrical devices (or phenomena) where resistance is not stable over a wide range of currents.  After some discussion, you should find that the phenomenon of constant resistance is not as common as one might think!

After students have grasped this concept, ask them what it means with regard to magnetic flux ($\Phi$) versus MMF (${\cal F}$).  In other words, what happens to flux in a magnetic circuit as MMF is increased?

%INDEX% Flux (magnetic), factors influencing
%INDEX% MMF (magnetic), factors influencing
%INDEX% Reluctance (magnetic), factors influencing

%(END_NOTES)


