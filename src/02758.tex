
%(BEGIN_QUESTION)
% Copyright 2005, Tony R. Kuphaldt, released under the Creative Commons Attribution License (v 1.0)
% This means you may do almost anything with this work of mine, so long as you give me proper credit

A {\it logic probe} is a very useful tool for working with digital logic circuits.  It indicates "high" and "low" logic states by means of LED's, giving visual indication only if the voltage levels are appropriate for each state.

Here is a schematic diagram for a logic probe built using comparators.  Each comparator has a threshold adjustment potentiometer, so that it may be set to indicate its respective logic state only if the signal voltage is well within the range stated by the logic manufacturer:

$$\epsfbox{02758x01.eps}$$

Explain how this circuit functions.

\underbar{file 02758}
%(END_QUESTION)





%(BEGIN_ANSWER)

I'll let you and your classmates figure out how this circuit functions!

\vskip 10pt

Follow-up question \#1: explain how you could use a voltmeter as a logic probe to do troubleshooting in a digital circuit.

\vskip 10pt

Follow-up question \#2: write a formula for calculating appropriate current-limiting resistor sizes for the two LEDs in this circuit, given the value of $+V$ and the LED forward voltage and current values.

%(END_ANSWER)





%(BEGIN_NOTES)

It is important for students to understand that there is a certain range of voltage between a guaranteed "high" state and a guaranteed "low" state that is indeterminate, and that this logic probe circuit is designed to indicate this range of voltage by turning neither LED on.

If time permits, discuss some of the benefits and drawbacks to using a voltmeter as a logic probe (especially a {\it digital} voltmeter where the display update time may be relatively long).

%INDEX% Logic probe circuit

%(END_NOTES)


