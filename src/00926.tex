
%(BEGIN_QUESTION)
% Copyright 2003, Tony R. Kuphaldt, released under the Creative Commons Attribution License (v 1.0)
% This means you may do almost anything with this work of mine, so long as you give me proper credit

An op-amp has +3 volts applied to the inverting input and +3.002 volts applied to the noninverting input.  Its open-loop voltage gain is 220,000.  Calculate the output voltage as predicted by the following formula:

$$V_{out} = A_V \left( V_{in(+)} - V_{in(-)} \right)$$

\vskip 10pt

How much differential voltage (input) is necessary to drive the output of the op-amp to a voltage of -4.5 volts?

\underbar{file 00926}
%(END_QUESTION)





%(BEGIN_ANSWER)

$V_{out} =$ 440 volts

\vskip 10pt

Follow-up question: is this voltage figure realistic?  Is it possible for an op-amp such as the model 741 to output 440 volts?  Why or why not?

\vskip 20pt

The differential input voltage necessary to drive the output of this op-amp to -4.5 volts is -20.455 $\mu$V.

\vskip 10pt

Follow-up question: what does it mean for the input voltage differential to be {\it negative} 20.455 microvolts?  Provide an example of two input voltages ($V_{in(+)}$ and $V_{in(-)}$) that would generate this much differential voltage.

%(END_ANSWER)





%(BEGIN_NOTES)

Obviously, there are limitations to the op-amp formula for calculating output voltage, given input voltages and open-loop voltage gain.  Students need to realize the practical limits of an op-amp's output voltage range, and what sets those limits.

%INDEX% Opamp, as high-gain differential amplifier

%(END_NOTES)


