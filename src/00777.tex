
%(BEGIN_QUESTION)
% Copyright 2003, Tony R. Kuphaldt, released under the Creative Commons Attribution License (v 1.0)
% This means you may do almost anything with this work of mine, so long as you give me proper credit

{\it Bipolar transistors} are extremely useful devices, allowing a small electric current to control the flow of a much larger electric current:

$$\epsfbox{00777x01.eps}$$

These devices would be even more useful to us if they were able to control alternating current (AC), but they cannot.  Bipolar transistors are polarized (one-way-only) devices.

This fact does not prevent us from using bipolar transistors to control AC.  We just have to be clever about how we do it:

$$\epsfbox{00777x02.eps}$$

Explain how this circuit functions.  How is the transistor (a DC-only device) able to control alternating current (AC) through the load?

\underbar{file 00777}
%(END_QUESTION)





%(BEGIN_ANSWER)

The four-diode bridge rectifies the AC load's current into DC for the transistor to control.

%(END_ANSWER)





%(BEGIN_NOTES)

Usually, rectifier circuits are thought of exclusively as intermediary steps between AC and DC in the context of an AC-DC power supply.  They have other uses, though, as demonstrated by this interesting circuit!

It should be of no consequence if students have not yet studied transistors.  In fact, it is a good thing to give them a very brief introduction to the function of a previously unknown component and then have them examine a circuit (the rest of which they should understand well) to ascertain an overall function.  This is sometimes called the "Black Box" approach in engineering, and it is necessary when working around state-of-the-art electronic equipment, where you are practically guaranteed not to understand the inner workings of every subsystem and component.

%INDEX% BJT, controlling AC with

%(END_NOTES)


