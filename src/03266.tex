
%(BEGIN_QUESTION)
% Copyright 2005, Tony R. Kuphaldt, released under the Creative Commons Attribution License (v 1.0)
% This means you may do almost anything with this work of mine, so long as you give me proper credit

\centerline{\bf Animation: Lissajous figures on an oscilloscope with 2:1 frequency ratio}

\vskip 10pt

{\it This question consists of a series of images (one per page) that form an animation.  Flip the pages with your fingers to view this animation (or click on the "next" button on your viewer) frame-by-frame.}

\vskip 10pt

The following animation shows how two sinusoidal voltages with a frequency ratio of 2:1 draw a Lissajous figure on the screen of an oscilloscope.

\vfil \eject
$$\epsfbox{03266x01.eps}$$

\vfil \eject
$$\epsfbox{03266x02.eps}$$

\vfil \eject
$$\epsfbox{03266x03.eps}$$

\vfil \eject
$$\epsfbox{03266x04.eps}$$

\vfil \eject
$$\epsfbox{03266x05.eps}$$

\vfil \eject
$$\epsfbox{03266x06.eps}$$

\vfil \eject
$$\epsfbox{03266x07.eps}$$

\vfil \eject
$$\epsfbox{03266x08.eps}$$

\vfil \eject
$$\epsfbox{03266x09.eps}$$

\underbar{file 03266}

\vfil \eject

%(END_QUESTION)





%(BEGIN_ANSWER)

Nothing to note here.

%(END_ANSWER)





%(BEGIN_NOTES)

The purpose of this animation is to let students study the evolution of Lissajous figures and see how they are created from the interrelationship between two sinusoidal waveforms.  Similar to experimentation in the lab, except that here all the data collection is done visually rather than through the use of test equipment, and the students are able to "see" things that are invisible in real life.

%INDEX% Animation, Lissajous figures (2:1 frequency ratio)

%(END_NOTES)


