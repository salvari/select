
%(BEGIN_QUESTION)
% Copyright 2003, Tony R. Kuphaldt, released under the Creative Commons Attribution License (v 1.0)
% This means you may do almost anything with this work of mine, so long as you give me proper credit

Suppose a generator is mechanically coupled to an internal combustion engine in an automobile, for the purpose of charging the starting battery.  In order that the battery not be over-charged by the generator, there must be some way of controlling the generator's output voltage over a wide range of engine speeds.

How is this {\it regulation} of generator output voltage typically achieved?  What variable within the generator may be most easily adjusted to maintain a nearly constant output voltage?  Express your answer in relation to Faraday's Law of electromagnetic induction.

\underbar{file 00811}
%(END_QUESTION)





%(BEGIN_ANSWER)

The most common method of generator voltage control is adjustment of field winding excitation.

%(END_ANSWER)





%(BEGIN_NOTES)

Although adjustable field winding excitation is the most popular form of generator output voltage control, it is not the only means.  Challenge your students with inventing other means of charge control for the battery in this automotive electrical system, besides field winding excitation control.  What else can we do to the generator, or to the circuit it is within, to achieve charge control for the battery?

%INDEX% Voltage regulation, generator

%(END_NOTES)


