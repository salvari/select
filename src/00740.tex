
%(BEGIN_QUESTION)
% Copyright 2003, Tony R. Kuphaldt, released under the Creative Commons Attribution License (v 1.0)
% This means you may do almost anything with this work of mine, so long as you give me proper credit

These two electric motor designs are quite similar in appearance, but differ in the specific principle that makes the rotor move:

$$\epsfbox{00740x01.eps}$$

{\it Synchronous} AC motors use a permanent magnet rotor, while {\it induction} motors use an electromagnet rotor.  Explain what practical difference this makes in each motor's operation, and also explain the meaning of the motors' names.  Why is one called "synchronous" and the other called "induction"?

\underbar{file 00740}
%(END_QUESTION)





%(BEGIN_ANSWER)

Synchronous motors rotate in "sync" to the power line frequency.  Induction motors rotate a bit slower, their rotors always "slipping" slightly slower than the speed of the rotating magnetic field.

\vskip 10pt

Challenge question: what would happen if an induction motor were mechanically brought up to speed with its rotating magnetic field?  Imagine using an engine or some other prime-mover mechanism to force the induction motor's rotor to rotate at synchronous speed, rather than "slipping" behind synchronous speed as it usually does.  What effect(s) would this have?

%(END_ANSWER)





%(BEGIN_NOTES)

It is very important that students realize Lenz's Law is an {\it induced} effect, which only manifests when a {\it changing} magnetic field cuts through perpendicular conductors.  Ask your students to explain how the word "induction" applies to Lenz's Law, and to the induction motor design.  Ask them what conditions are necessary for electromagnetic induction to occur, and how those conditions are met in the normal operation of an induction motor.

The challenge question is really a test of whether or not students have grasped the concept.  If they truly understand how electromagnetic induction takes place in an induction motor, they will realize that there will be no induction when the rotor rotates in "sync" with the rotating magnetic field, and they will be able to relate this loss of induction to rotor torque.

%INDEX% Lenz's Law
%INDEX% Synchronous versus induction motor
%INDEX% Induction versus synchronous motor
%INDEX% AC motor, induction versus synchronous

%(END_NOTES)


