
%(BEGIN_QUESTION)
% Copyright 2006, Tony R. Kuphaldt, released under the Creative Commons Attribution License (v 1.0)
% This means you may do almost anything with this work of mine, so long as you give me proper credit

% Uncomment the following line if the question involves calculus at all:
\vbox{\hrule \hbox{\strut \vrule{} $\int f(x) \> dx$ \hskip 5pt {\sl Calculus alert!} \vrule} \hrule}

{\it Impedance} is defined as the complex ratio of voltage and current:

$${\bf Z} = {{\bf V}\over {\bf I}}$$

We may determine the complex definition of impedance provided by an inductor if we consider the equation relating inductor voltage and inductor current:

$$v = L {d \over dt} [ i(t) ]$$

First, substitute a phasor expression of current into the above equation, then differentiate with respect to time to obtain an expression for voltage:

$$i(t) = e^{j \omega t}$$

$$v = L {d \over dt} \left[ e^{j \omega t} \right]$$

Then, divide the two phasor expressions to obtain an expression for inductive impedance.

\underbar{file 04063}
%(END_QUESTION)





%(BEGIN_ANSWER)

$${\bf Z_L} = j \omega L$$

\vskip 10pt

Follow-up question: explain why the following expression for inductive impedance is equivalent to the one shown above:

$${\bf Z_L} = \omega L \> e^{j \pi / 2}$$

%(END_ANSWER)





%(BEGIN_NOTES)

In order to solve this problem, your students must remember the basic rule for differentiating exponential functions:

$${d \over dx} \left[ e^{ax} \right] = a e^{ax}$$

This is one of the beauties of representing sinusoidal voltages and currents in complex exponential (phasor) form: it makes differentiation and integration relatively easy!  In this sense, the Euler relation of $e^{jx} = \cos x + j \sin x$ is a {\it transform function}, transforming one type of mathematical problem into a different (easier) type.

%INDEX% Impedance, inductive (defined by complex exponential phasor notation)

%(END_NOTES)


