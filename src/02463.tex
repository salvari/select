
%(BEGIN_QUESTION)
% Copyright 2005, Tony R. Kuphaldt, released under the Creative Commons Attribution License (v 1.0)
% This means you may do almost anything with this work of mine, so long as you give me proper credit

The same problem of input bias current affecting the precision of opamp voltage buffer circuits also affects noninverting opamp voltage amplifier circuits:

$$\epsfbox{02463x01.eps}$$

$$\epsfbox{02463x02.eps}$$

To fix this problem in the voltage buffer circuit, we added a "compensating" resistor:

$$\epsfbox{02463x03.eps}$$

To fix the same problem in the noninverting voltage amplifier circuit, we must carefully choose resistors $R_1$ and $R_2$ so that their parallel equivalent equals the source resistance:

$$R_1 || R_2 = R_{source}$$

Of course, we must also be sure the values of $R_1$ and $R_2$ are such that the voltage gain of the circuit is what we want it to be.

Determine values for $R_1$ and $R_2$ to give a voltage gain of 7 while compensating for a source resistance of 1.45 k$\Omega$.

\underbar{file 02463}
%(END_QUESTION)





%(BEGIN_ANSWER)

$R_1$ = 1.692 k$\Omega$ \hskip 50pt $R_2$ = 10.15 k$\Omega$

%(END_ANSWER)





%(BEGIN_NOTES)

Students must apply algebra to solve for the values of these two resistances.  The solution is an application of algebraic {\it substitution}, and it is worthwhile to examine and discuss together in class.

Discuss how this solution to the bias current problem is a practical application of Th\'evenin's theorem: looking at the two voltage divider resistors as a network that may be Th\'evenized to serve as a compensating resistance as well as a voltage divider for the necessary circuit gain.

%INDEX% Bias current compensation, opamp input (in noninverting amplifier circuit)

%(END_NOTES)


