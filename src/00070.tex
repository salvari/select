
%(BEGIN_QUESTION)
% Copyright 2003, Tony R. Kuphaldt, released under the Creative Commons Attribution License (v 1.0)
% This means you may do almost anything with this work of mine, so long as you give me proper credit

Why is it a very bad idea to connect an ammeter directly across a voltage source, like this?

$$\epsfbox{00070x01.eps}$$

\underbar{file 00070}
%(END_QUESTION)





%(BEGIN_ANSWER)

Due to the ammeter's very low resistance, it will "draw" a lot of current from the voltage source.  In effect, the ammeter will form a short circuit with the voltage source, potentially damaging the meter and/or the source.

In applications where the voltage source possesses very little internal resistance of its own, the current surge resulting from such a short-circuit may be huge.  Very large surges of electric current are capable of heating wires to the point where their insulation bursts into flames, as well as causing super-heated blasts of {\it plasma} (electrically ionized gas) to form at any point of electrical contact where there is a spark.  Either of these high-temperature conditions are hazardous to the person holding the meter and test leads!

%(END_ANSWER)





%(BEGIN_NOTES)

An important point to discuss is how electrical safety encompasses more than just shock hazard.  In particular, {\it arc blasts} caused by high-current "faults" such as this may be just as dangerous as electric shock.  At the very least, placing an ammeter directly across the terminals of a voltage source will likely result in the ammeter's fuse being blown.

In some cases, ammeter fuses are more expensive than one might think.  Safety-rated ammeters often use expensive fast-action fuses with significant current interruption ratings.  In the case of the Fluke 187 and 189 multimeters, these fuses cost around \$8 each (American dollars, 2004)!

%INDEX% Ammeter usage

%(END_NOTES)


