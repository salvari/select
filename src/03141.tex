
%(BEGIN_QUESTION)
% Copyright 2005, Tony R. Kuphaldt, released under the Creative Commons Attribution License (v 1.0)
% This means you may do almost anything with this work of mine, so long as you give me proper credit

A special class of electromechanical relays called {\it time-delay} relays provide delayed action, either upon power-up or power-down, and are commonly denoted in ladder logic diagrams by "TD" or "TR" designations near the coil symbols and arrows on the contact symbols.  Here is an example of a time-delay relay contact used in a motor control circuit:

$$\epsfbox{03141x01.eps}$$

In this circuit, the motor delays start-up until three seconds {\it after} the switch is thrown to the "Run" position, but will stop immediately when the switch is returned to the "Stop" position.  The relay contact is referred to as {\it normally-open, timed-closed}, or NOTC.  It is alternatively referred to as a {\it normally-open, on-delay} contact.

Explain how the arrow symbol indicates the nature of this contact's delay, that delay occurs during closure but not during opening.

\underbar{file 03141}
%(END_QUESTION)





%(BEGIN_ANSWER)

Note that the "arrow" is pointing in the up direction, toward the direction of contact closure.

%(END_ANSWER)





%(BEGIN_NOTES)

The arrow symbol is not difficult to figure out, but it is essential to know when working with time-delay relay circuits.  Ask your students to describe their understanding of the arrow symbol as they answer this question.

%INDEX% Relay, time delay 
%INDEX% Time delay relay
%INDEX% Time delay relay circuit, electric motor control

%(END_NOTES)


