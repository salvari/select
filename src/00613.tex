
%(BEGIN_QUESTION)
% Copyright 2003, Tony R. Kuphaldt, released under the Creative Commons Attribution License (v 1.0)
% This means you may do almost anything with this work of mine, so long as you give me proper credit

Suppose you were installing a high-power stereo system in your car, and you wanted to build a simple filter for the "tweeter" (high-frequency) speakers so that no bass (low-frequency) power is wasted in these speakers.  Modify the schematic diagram below with a filter circuit of your choice:

$$\epsfbox{00613x01.eps}$$

Hint: this only requires a single component per tweeter!

\underbar{file 00613}
%(END_QUESTION)





%(BEGIN_ANSWER)

$$\epsfbox{00613x02.eps}$$

\vskip 10pt

Follow-up question: what type of capacitor would you recommend using in this application (electrolytic, mylar, ceramic, etc.)?  Why?

%(END_ANSWER)





%(BEGIN_NOTES)

Ask your students to describe what type of filter circuit a series-connected capacitor forms: low-pass, high-pass, band-pass, or band-stop?  Discuss how the name of this filter should describe its intended function in the sound system.

Regarding the follow-up question, it is important for students to recognize the practical limitations of certain capacitor types.  One thing is for sure, ordinary (polarized) electrolytic capacitors will not function properly in an application like this!

%INDEX% Filter, highpass (audio speaker application)

%(END_NOTES)


