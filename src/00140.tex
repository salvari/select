
%(BEGIN_QUESTION)
% Copyright 2003, Tony R. Kuphaldt, released under the Creative Commons Attribution License (v 1.0)
% This means you may do almost anything with this work of mine, so long as you give me proper credit

Atoms are very, very small pieces of matter.  This fact should go without saying, but it behooves us to be reminded of just how small atoms are.  Chemists and physicists use a unit of measurement to represent quantities of different materials based on how many atoms (or molecules) there are in a particular sample.  This unit of measurement is called the {\it mole}.

1 mole of pure iron metal weighs about 56 grams.  Based on the definition of a mole, how many atoms of iron are in this 56 gram sample?

\underbar{file 00140}
%(END_QUESTION)





%(BEGIN_ANSWER)

There are approximately $6.022 \times 10^{23}$ atoms in this 56 gram sample of iron.  What does this quantity look like when written in non-scientific notation?

%(END_ANSWER)





%(BEGIN_NOTES)

The purpose of this question is twofold: to reinforce the fact that atoms are really, really tiny, and to introduce students to the use of scientific notation.

%INDEX% Mole, chemistry

%(END_NOTES)


