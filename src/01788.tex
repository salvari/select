
%(BEGIN_QUESTION)
% Copyright 2003, Tony R. Kuphaldt, released under the Creative Commons Attribution License (v 1.0)
% This means you may do almost anything with this work of mine, so long as you give me proper credit

Determine the polarity of the coil's induced voltage for each of the following examples.  Be careful to note the direction each coil is wrapped around its core -- the coils are not all identical!

$$\epsfbox{01788x01.eps}$$

\underbar{file 01788}
%(END_QUESTION)





%(BEGIN_ANSWER)

$$\epsfbox{01788x02.eps}$$

%(END_ANSWER)





%(BEGIN_NOTES)

It might help students to visualize the polarity if they imagine a resistive load connected between the two output terminals, and then figured out which direction induced {\it current} would go through that load.  Once that determination is made, voltage polarity (considering the coil as an energy source) should be easier to visualize.  A mistake many beginning students make when doing this, though, is to fail to recognize the coil as the {\it source} of electrical energy and the resistor as the {\it load}, so be prepared to address this misunderstanding.

If this does not help, suggest they first identify the magnetic polarity of the coil's induced field: determine which end of the coil is "trying" to be North and which is "trying" to be South.  Of course, no induced field will form unless the coil has a complete circuit to sustain the induced current, but it is still helpful to imagine a load resistor or even a short completing the circuit so that induced current and thus induced magnetic polarity may be visualized.

%INDEX% Lenz's Law, polarity of induced voltage

%(END_NOTES)


