
%(BEGIN_QUESTION)
% Copyright 2003, Tony R. Kuphaldt, released under the Creative Commons Attribution License (v 1.0)
% This means you may do almost anything with this work of mine, so long as you give me proper credit

This simple electric circuit is capable of {\it resonance}, whereby voltage and current oscillate at a frequency characteristic to the circuit:

$$\epsfbox{00601x01.eps}$$

In a mechanical resonant system -- such as a tuning fork, a bell, or a guitar string -- resonance occurs because the complementary properties of {\it mass} and {\it elasticity} exchange energy back and forth between each other in {\it kinetic} and {\it potential} forms, respectively.  Explain how energy is stored and transferred back and forth between the capacitor and inductor in the resonant circuit shown in the illustration, and identify which of these components stores energy in kinetic form, and which stores energy in potential form.

\underbar{file 00601}
%(END_QUESTION)





%(BEGIN_ANSWER)

Capacitors store energy in potential form, while inductors store energy in kinetic form.

%(END_ANSWER)





%(BEGIN_NOTES)

Ask your students to define "potential" and "kinetic" energy.  These terms, of course, are central to the question, and I have not bothered to define them.  This omission is purposeful, and it is the students' responsibility to research the definitions of these words in the process of answering the question.  If a substantial number of your students stopped trying to answer the question when they encountered new words (instead of taking initiative to find out what the words mean), then it indicates a need to focus on independent learning skills (and attitudes!).

Discuss a typical "cycle" of energy exchange between kinetic and potential forms in a vibrating object, and then relate this exchange process to the oscillations of a tank circuit (capacitor and inductor).

%INDEX% Resonant circuit, analysis in terms of energy storage

%(END_NOTES)


