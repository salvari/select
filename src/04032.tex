
%(BEGIN_QUESTION)
% Copyright 2006, Tony R. Kuphaldt, released under the Creative Commons Attribution License (v 1.0)
% This means you may do almost anything with this work of mine, so long as you give me proper credit

Suppose a student builds this circuit, using 10 M$\Omega$ resistors for $R_1$ and $R_2$, and a 22 $\mu$F capacitor for $C_1$, and finds that he must frequently adjust the potentiometer to keep the circuit at a reasonable level of sensitivity.  A fellow student has encountered and overcome this same problem, and offers to fix the first student's circuit by replacing the 22 $\mu$F capacitor with something much smaller: a 0.001 $\mu$F (1 nF) capacitor.

Now, the modified circuit responds predictably.  No longer must the student frequently adjust the potentiometer to achieve the desired sensitivity.  Explain why a change in capacitance made such a difference, and what the problem was in the original circuit (with the much larger capacitor).

\underbar{file 04032}
%(END_QUESTION)





%(BEGIN_ANSWER)

Here's a hint: calculate the time constant ($\tau$) for the RC network formed by $R_1$, $R_2$, and $C_1$.

%(END_ANSWER)





%(BEGIN_NOTES)

In order to understand the problem, students must recognize the function of this RC network: to filter out the DC bias voltage created by ambient light falling on the LED array, and to replace that changing bias with a fixed bias voltage equal to $1 \over 2$ $V_{\hbox{supply}}$.  Once this purpose is grasped, the problem of a slow time constant should be much more apparent.

%(END_NOTES)


