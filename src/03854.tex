
%(BEGIN_QUESTION)
% Copyright 2006, Tony R. Kuphaldt, released under the Creative Commons Attribution License (v 1.0)
% This means you may do almost anything with this work of mine, so long as you give me proper credit

Identify the problem(s) with the following project construction and wiring practices, explaining how these practices could be improved upon, and why:

\vskip 10pt

\noindent
{\bf Wires cut as short as possible, stretched point-to-point}:

\vskip 30pt

\noindent
{\bf Compression lugs crimped over solid wire}:

\vskip 30pt

\noindent
{\bf Compression lugs crimped using ordinary pliers}:

\vskip 30pt

\noindent
{\bf Bare wire ends clamped beneath nuts (or nuts and washers) on threaded studs}:

\vskip 30pt

\noindent
{\bf Solid wires used in places where bending regularly occurs}:

\vskip 30pt

\noindent
{\bf Signal and power wires bundled together}:

\vskip 30pt

\noindent
{\bf Components anchored in place by glue rather than by removable fasteners}:

\vskip 30pt

\underbar{file 03854}
%(END_QUESTION)





%(BEGIN_ANSWER)

\noindent
{\bf Wires cut as short as possible, stretched point-to-point}:

When wires are strung in such a point-to-point fashion, several problems arise.  First, they are more easily pulled loose from their connection points.  Second, they tend to impede access to other components by occupying open space inside the enclosure rather than "hugging" flat surfaces.  Third, short wire lengths place more stress on the wires and the connections when vibration occurs.

\vskip 10pt

\noindent
{\bf Compression lugs crimped over solid wire}:

Solid, electrical-grade copper does not have enough elasticity to maintain proper tension against the barrel of a small compression-style lug.  Over time, a solid wire will work itself loose from a such a lug.  Stranded wire is the proper type of wire to use in this application.

\vskip 10pt

\noindent
{\bf Compression lugs crimped using ordinary pliers}:

Special crimping pliers are designed to compress the barrel of the lug unevenly, so that the wire is securely held between ridges formed under the pressure of crimping.  Regular pliers with their flat jaws are unable to produce these ridges in the barrel, leaving the wire much less secure.

\vskip 10pt

\noindent
{\bf Bare wire ends clamped beneath nuts (or nuts and washers) on threaded studs}:

When any tension is placed on the wire, it will try to turn the nut.  This is why lugs should always be crimped on to the end of a wire to attach that wire to a stud: the lug will not exert a torque on the holding nut.

\vskip 10pt

\noindent
{\bf Solid wires used in places where bending occurs}:

Copper will harden if repeatedly stressed, leading to brittleness and fatigue.  Solid wire does not bend easily, and will eventually break where it is forced to bend.  Stranded wire is much more supple, and takes bending much better than solid wire.

\vskip 10pt

\noindent
{\bf Signal and power wires bundled together}:

Close proximity between wires leads to inductive and capacitive coupling.  When power and signal wires are placed together, the larger currents and voltages in the power conductors will likely couple unwanted noise into the signal wiring.  As a rule, always separate power and signal wiring.  If these wires must cross paths, do so at right angles to minimize coupling.

\vskip 10pt

\noindent
{\bf Components anchored in place by glue rather than by removable fasteners}:

Glued components are much more difficult to replace than fastened components.  Always build your projects with future maintenance in mind!

%(END_ANSWER)





%(BEGIN_NOTES)

The purpose of this question is to introduce students to good wiring practices.  By asking them to identify what is wrong with a set of improper practices, they are more likely to pay attention than if you simply tell them the right way to do things.

%INDEX% Guidelines for good wiring practice
%INDEX% Wiring practice, guidelines for

%(END_NOTES)


