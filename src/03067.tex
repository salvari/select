
%(BEGIN_QUESTION)
% Copyright 2005, Tony R. Kuphaldt, released under the Creative Commons Attribution License (v 1.0)
% This means you may do almost anything with this work of mine, so long as you give me proper credit

The formula for calculating total resistance of three parallel-connected resistors is as follows:

$$R = {1 \over {{1 \over R_1} + {1 \over R_2} + {1 \over R_3}}}$$

Algebraically manipulate this equation to solve for one of the parallel resistances ($R_1$) in terms of the other two parallel resistances ($R_2$ and $R_3$) and the total resistance ($R$).  In other words, write a formula that solves for $R_1$ in terms of all the other variables.

\underbar{file 03067}
%(END_QUESTION)





%(BEGIN_ANSWER)

$$R_1 = {1 \over {{1 \over R} - ({1 \over R_2} + {1 \over R_3})}} \hbox{\hskip 30pt or \hskip 30pt} R_1 = {1 \over {{1 \over R} - {1 \over R_2} - {1 \over R_3}}}$$

%(END_ANSWER)





%(BEGIN_NOTES)

This question is nothing more than practice algebraically manipulating equations.  Ask your students to show you how they solved it, and how the two given answers are equivalent.

%INDEX% Algebra, manipulating equations
%INDEX% Resistances in parallel
%INDEX% Parallel resistances

%(END_NOTES)


