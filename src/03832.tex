
%(BEGIN_QUESTION)
% Copyright 2006, Tony R. Kuphaldt, released under the Creative Commons Attribution License (v 1.0)
% This means you may do almost anything with this work of mine, so long as you give me proper credit

Predict how the operation of this logic gate circuit will be affected as a result of the following faults.  Consider each fault independently (i.e. one at a time, no multiple faults):

$$\epsfbox{03832x01.eps}$$

\medskip
\item{$\bullet$} Output of AND gate $U_2$ fails low:
\vskip 5pt
\item{$\bullet$} Output of AND gate $U_2$ fails high:
\vskip 5pt
\item{$\bullet$} Output of inverter gate $U_1$ fails low:
\medskip

For each of these conditions, explain {\it why} the resulting effects will occur.

\underbar{file 03832}
%(END_QUESTION)





%(BEGIN_ANSWER)

\medskip
\item{$\bullet$} Output of AND gate $U_2$ fails low: {\it Gate $U_3$ output stuck in the high state.}
\vskip 5pt
\item{$\bullet$} Output of AND gate $U_2$ fails high: {\it Gate $U_3$ output simply equal to $\overline{C}$, no other inputs have any effect on $U_3$'s output.}
\vskip 5pt
\item{$\bullet$} Output of inverter gate $U_1$ fails low: {\it Gate $U_3$ output stuck in the high state.}
\medskip

%(END_ANSWER)





%(BEGIN_NOTES)

The purpose of this question is to approach the domain of circuit troubleshooting from a perspective of knowing what the fault is, rather than only knowing what the symptoms are.  Although this is not necessarily a realistic perspective, it helps students build the foundational knowledge necessary to diagnose a faulted circuit from empirical data.  Questions such as this should be followed (eventually) by other questions asking students to identify likely faults based on measurements.

%INDEX% Troubleshooting, predicting effects of fault in logic gate circuit

%(END_NOTES)


