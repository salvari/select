
%(BEGIN_QUESTION)
% Copyright 2005, Tony R. Kuphaldt, released under the Creative Commons Attribution License (v 1.0)
% This means you may do almost anything with this work of mine, so long as you give me proper credit

{\it Verilog} and {\it VHDL} are two popular examples of a {\it hardware description language}, used when working with programmable logic.  Explain the purpose of such a "language."  What does it mean for a technician or engineer to "speak" this language, and how is it "spoken" to an actual programmable chip?

\underbar{file 03047}
%(END_QUESTION)





%(BEGIN_ANSWER)

A hardware description language (HDL) is a textual convention for specifying the interconnections of a programmable logic device.  Text files are written by a human programmer, then "compiled" into a form that the programmable logic device can directly accept and use.

%(END_ANSWER)





%(BEGIN_NOTES)

If time permits, you may want to compare and contrast fully-featured languages such as Verilog and VHDL with more primitive hardware description languages such as ABEL.  In either case, though, files written in an HDL are intended to describe the interconnections of available logic elements inside a programmable logic device.

%INDEX% HDL, programmable logic
%INDEX% VHDL, programmable logic
%INDEX% Verilog, programmable logic

%(END_NOTES)


