
%(BEGIN_QUESTION)
% Copyright 2003, Tony R. Kuphaldt, released under the Creative Commons Attribution License (v 1.0)
% This means you may do almost anything with this work of mine, so long as you give me proper credit

What is {\it DeMorgan's Theorem}?

\underbar{file 01323}
%(END_QUESTION)





%(BEGIN_ANSWER)

DeMorgan's Theorem is a rule for Boolean expressions, declaring how long complementation "bars" are to be broken into shorter bars.  I'll let you research the terms of this rule, and explain how to apply it to Boolean expressions.

%(END_ANSWER)





%(BEGIN_NOTES)

There are many suitable references for students to be able to learn DeMorgan's Theorem from.  Let them do the research on their own!  Your task is to clarify any misunderstandings after they've done their jobs.

%INDEX% Boolean algebra, DeMorgan's Theorem (defined)
%INDEX% DeMorgan's Theorem, Boolean algebra (defined)

%(END_NOTES)


