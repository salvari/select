
%(BEGIN_QUESTION)
% Copyright 2003, Tony R. Kuphaldt, released under the Creative Commons Attribution License (v 1.0)
% This means you may do almost anything with this work of mine, so long as you give me proper credit

What will happen to each resistor's voltage and current in this circuit if resistor R1 fails open?  Provide individual answers for each resistor, please.

$$\epsfbox{01775x01.eps}$$

\underbar{file 01775}
%(END_QUESTION)





%(BEGIN_ANSWER)

If resistor R1 fails open . . .

\medskip
\item{$\bullet$} $V_{R1}$ will increase to full supply voltage, $I_{R1}$ will decrease to zero
\item{$\bullet$} $V_{R2}$ will decrease to zero, $I_{R2}$ will decrease to zero
\item{$\bullet$} $V_{R3}$ will decrease to zero, $I_{R3}$ will decrease to zero
\item{$\bullet$} $V_{R4}$ will decrease to zero, $I_{R4}$ will decrease to zero
\medskip

\vskip 10pt

Follow-up question: note the order in which I list the qualitative effects of R2's shorted failure.  Reading from the top of the list to the bottom reveals the sequence of my reasoning.  Explain why I would come to the conclusions I did, in the order I did.

%(END_ANSWER)





%(BEGIN_NOTES)

I have found in teaching that many students loathe qualitative analysis, because they cannot let their calculators do the thinking for them.  However, being able to judge whether a circuit parameter will increase, decrease, or remain the same after a component fault is an {\it essential} skill for proficient troubleshooting.

%INDEX% Series-parallel circuit; effect of open fault

%(END_NOTES)


