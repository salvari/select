
%(BEGIN_QUESTION)
% Copyright 2005, Tony R. Kuphaldt, released under the Creative Commons Attribution License (v 1.0)
% This means you may do almost anything with this work of mine, so long as you give me proper credit

Many single-phase "squirrel-cage" induction motors use a special {\it start winding} which is energized only at low (or no) speed.  When the rotor reaches full operating speed, the starting switch opens to de-energize the start winding:

$$\epsfbox{03622x01.eps}$$

Explain why this special winding is necessary for the motor to start, and also why there is a capacitor connected in series with this start winding.  What would happen if the start switch, capacitor, or start winding were to fail open?

\underbar{file 03622}
%(END_QUESTION)





%(BEGIN_ANSWER)

Single-phase AC has no definite direction of "rotation" like polyphase AC does.  Consequently, a second, phase-shifted magnetic field must be generated in order to give the rotor a starting torque.

\vskip 10pt

Challenge question: explain what you would have to do to reverse the direction of this "capacitor-start" motor.

%(END_ANSWER)





%(BEGIN_NOTES)

Capacitor-start squirrel-cage induction motors are very popular in applications where there is a need for high starting torque.  Many induction motor shop tools (drill presses, lathes, radial-arm saws, air compressors) use capacitor-start motors.

%INDEX% Induction motor, capacitor-start (split phase)

%(END_NOTES)


