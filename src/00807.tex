
%(BEGIN_QUESTION)
% Copyright 2003, Tony R. Kuphaldt, released under the Creative Commons Attribution License (v 1.0)
% This means you may do almost anything with this work of mine, so long as you give me proper credit

If this wire (between the magnet poles) is moved in an upward direction, and the wire ends are connected to a resistive load, which way will current go through the wire?

$$\epsfbox{00807x01.eps}$$

We know that current moving through a wire will create a magnetic field, and that this magnetic field will produce a reaction force against the static magnetic fields coming from the two permanent magnets.  Which direction will this reaction force push the current-carrying wire?  How does the direction of this force relate to the direction of the wire's motion?  Does this phenomenon relate to any principle of electromagnetism you've learned so far?

\underbar{file 00807}
%(END_QUESTION)





%(BEGIN_ANSWER)

The reaction force will be directly opposed to the direction of motion, as described by Lenz's Law.

\vskip 10pt

Follow-up question: What does this phenomenon indicate to us about the ease of moving a generator mechanism under load, versus unloaded?  What effect does placing an electrical load on the output terminals of a generator have on the mechanical effort needed to turn the generator?

%(END_ANSWER)





%(BEGIN_NOTES)

If you happen to have a large, permanent magnet DC motor available in your classroom, you may easily demonstrate this principle for your students.  Just have them spin the shaft of the motor (generator) with their hands, with the power terminals open versus shorted together.  Your students will notice a huge difference in the ease of turning between these two states.

After your students have had the opportunity to discuss this phenomenon and/or experience it themselves, ask them why electromechanical meter movement manufacturers usually ship meters with a shorting wire connecting the two meter terminals together.  In what way does a PMMC meter movement resemble an electric generator?  How does shorting the terminals together help to protect against damage from physical vibration during shipping?

Ask your students to describe what factors influence the magnitude of this reaction force.

%INDEX% Electromagnetic induction, conceptual
%INDEX% Electromagnetic induction, moving wire
%INDEX% Lenz's Law

%(END_NOTES)

