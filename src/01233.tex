
%(BEGIN_QUESTION)
% Copyright 2003, Tony R. Kuphaldt, released under the Creative Commons Attribution License (v 1.0)
% This means you may do almost anything with this work of mine, so long as you give me proper credit

A nearly universal standard for representing text data in digital form is the {\it ASCII} code.  What does the acronym "ASCII" stand for, and what is the format of this code?

\underbar{file 01233}
%(END_QUESTION)





%(BEGIN_ANSWER)

"ASCII" = American Standard Code for Information Interchange.  Basic ASCII is a seven-bit binary code capable of representing all alphabetical characters used in the English language (upper-case as well as lower), as well as Arabic numerals, English punctuation marks, and some miscellaneous control codes for teletype machines.

\vskip 10pt

Challenge question: although ASCII technically requires only 7 bits, a full 8 bits (1 byte) is usually reserved for each ASCII character in computer systems.  Explain why.

%(END_ANSWER)





%(BEGIN_NOTES)

ASCII is arguably the {\it lingua franca} of the digital world.  Despite its humble beginnings and Anglo-centric format, it is used worldwide in digital computer and telecommunication systems.  Let your students know that every plain-text computer file is nothing more than a collection of ASCII codes, one code for each text character (including spaces).

%INDEX% ASCII, defined

%(END_NOTES)


