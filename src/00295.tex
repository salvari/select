
%(BEGIN_QUESTION)
% Copyright 2003, Tony R. Kuphaldt, released under the Creative Commons Attribution License (v 1.0)
% This means you may do almost anything with this work of mine, so long as you give me proper credit

What will happen in this circuit as the switches are sequentially turned on, starting with switch number 1 and ending with switch number 3?  

$$\epsfbox{00295x01.eps}$$

Describe how the successive closure of these three switches will impact:

\medskip
\item{$\bullet$} The total amount of circuit resistance "seen" by the battery
\item{$\bullet$} The total amount of current drawn from the battery
\item{$\bullet$} The current through each resistor
\item{$\bullet$} The voltage drop across each resistor 
\medskip

Also, provide a safety-related reason for the existence of the fourth resistor in this circuit, on the left-hand side of the circuit (not bypassed by any switch).

\underbar{file 00295}
%(END_QUESTION)





%(BEGIN_ANSWER)

I won't explain what happens when each of the switches is closed, but I will describe the effects of the first switch closing:

As the first switch (SW1) is closed, the voltage across resistor R1 will decrease to zero, while the voltages across the remaining resistors will increase.  The current through resistor R1 will also decrease to zero, and the current through the remaining resistors will also increase.  Each of the resistors will experience the same amount of current as the others, and this amount of current will also be experienced by the battery.  Overall, the battery "sees" less total resistance than before.

The fourth resistor is there to prevent a short-circuit from developing if all switches are simultaneously closed.

%(END_ANSWER)





%(BEGIN_NOTES)

One problem I've encountered while teaching the "laws" of series circuits is that some students mistakenly think the rule of "all currents in a series circuit being the same" means that the amount of current in a series circuit is fixed over time and cannot change.  The root of this misunderstanding is memorization rather than comprehension: students memorize the rule "{\it all currents are the same}" and think this means the currents must remain the same before and after any change is made to the circuit.  I've actually had students complain to me, saying, "But you told us all currents are {\it the same} in a series circuit!", as though it were my job to decree perfect and universal Laws which would require no critical thinking on the part of the student.  But I digress . . .

This question challenges students' comprehension of series circuit behavior by asking what happens after a change is made to the circuit.  The purpose of the switches is to "remove" resistors from the circuit, one at a time, without actually having to remove components.

%INDEX% Series circuit; voltage, current, resistance, and power in
%INDEX% Short circuit, prevented by resistor
%INDEX% Troubleshooting, simple circuit

%(END_NOTES)


