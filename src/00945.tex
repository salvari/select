
%(BEGIN_QUESTION)
% Copyright 2003, Tony R. Kuphaldt, released under the Creative Commons Attribution License (v 1.0)
% This means you may do almost anything with this work of mine, so long as you give me proper credit

We know that graphs are nothing more than collections of individual points representing correlated data in a system.  Here is a plot of a transistor's characteristic curve (for a single value of base current):

$$\epsfbox{00945x01.eps}$$

And here is a plot of the "load line" for a transistor amplifier circuit:

$$\epsfbox{00945x02.eps}$$

For each of these graphs, pick a single point along the curve (or line) and describe what that single point represents, in real-life terms.  What does any single point of data along either of these graphs {\it mean} in a transistor circuit?

If a transistor's characteristic curve is superimposed with a load line on the same graph, what is the significance of those two plots' intersection?

\underbar{file 00945}
%(END_QUESTION)





%(BEGIN_ANSWER)

For a transistor's characteristic curve, one point of data represents the amount of current that will go through the collector terminal for a given amount of base current, and a given amount of collector-emitter voltage drop.

\vskip 10pt

For a load line, one point of data represents the amount of collector-emitter voltage available to the transistor for a given amount of collector current.

\vskip 10pt

The intersection of a characteristic curve and a load line represents the one collector current (and corresponding $V_{CE}$ voltage drop) that will "satisfy" all components' conditions.

%(END_ANSWER)





%(BEGIN_NOTES)

Discuss this question thoroughly with your students.  So many students of electronics learn to plot load lines for amplifier circuits without ever really understanding {\it why} they must do so.  Load line plots are very useful tools in amplifier circuit analysis, but the meaning of each curve/line must be well understood before it becomes useful as an instrument of understanding.

Ask your students which of the two types of graphs (characteristic curves, or load lines) represents a component's natural, or "free", behavior, and which one represents the bounded conditions within a particular circuit.  

%INDEX% Load line, concept illustrated with transistor amplifier circuit

%(END_NOTES)


