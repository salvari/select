
%(BEGIN_QUESTION)
% Copyright 2006, Tony R. Kuphaldt, released under the Creative Commons Attribution License (v 1.0)
% This means you may do almost anything with this work of mine, so long as you give me proper credit

Two 150-amp circuit breakers are connected in parallel to obtain a total ampacity of 300 amperes for an electric motor service.  The system works just fine for several years, but then both breakers begin to spuriously trip:

$$\epsfbox{03938x01.eps}$$

An electrician measures motor current using a clamp-on ammeter, and discovers the motor's current is no more than 228 amperes at full mechanical load.  Describe what might possibly be wrong that is causing both circuit breakers to trip.

\underbar{file 03938}
%(END_QUESTION)





%(BEGIN_ANSWER)

One of the circuit breakers has probably developed more resistance in its parallel branch than the other (contact resistance, resistance at the wire connection points, etc.).  I will let you explain why this causes false trips.

%(END_ANSWER)





%(BEGIN_NOTES)

This is a very practical problem, directly dealing with parallel resistances.  Unfortunately, I have seen residential circuit breaker panels new from the manufacturer, equipped with paralleled breakers!  Bad idea . . . baaaaaad idea.

Something worthwhile to note as a possible cause of the tripping is a spurious motor problem.  Perhaps the circuit breakers are sharing current equally after all, but the motor is occasionally drawing more than 300 amps of current!  Just because an electrician measured less than 300 amps at full load does not mean the motor {\it never} draws more than 300 amps.  There may be another problem after all.  Discuss this with your students, asking them how they would identify such a problem after having determined the two circuit breakers were doing their job correctly.

%INDEX% Troubleshooting, parallel circuit breakers tripping

%(END_NOTES)


