
%(BEGIN_QUESTION)
% Copyright 2003, Tony R. Kuphaldt, released under the Creative Commons Attribution License (v 1.0)
% This means you may do almost anything with this work of mine, so long as you give me proper credit

In the simplest terms you can think of, define what an electrical {\it circuit} is.

\underbar{file 00017}
%(END_QUESTION)





%(BEGIN_ANSWER)

An electrical {\it circuit} is any continuous path for electrons to flow away from a source of electrical potential (voltage) and back again.

%(END_ANSWER)





%(BEGIN_NOTES)

Although definitions are easy enough to research and repeat, it is important that students learn to cast these concepts into their own words.  Asking students to give practical examples of "circuits" and "non-circuits" is one way to ensure deeper investigation of the concepts than mere term memorization.

The word "circuit," in vernacular usage, often refers to {\it anything} electrical.  Of course, this is not true in the technical sense of the term.  Students will come to realize that many terms they learn and use in an electricity or electronics course are actually mis-used in common speech.  The word "short" is another example: technically it refers to a specific type of circuit fault.  Commonly, though, people use it to refer to {\it any} type of electrical problem.

%INDEX% Circuit, simple

%(END_NOTES)


