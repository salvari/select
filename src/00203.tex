
%(BEGIN_QUESTION)
% Copyright 2003, Tony R. Kuphaldt, released under the Creative Commons Attribution License (v 1.0)
% This means you may do almost anything with this work of mine, so long as you give me proper credit

Capacitance exists between any two conductors separated by an insulating medium.  Given this fact, it makes sense that a length of two-conductor electrical cable will have capacitance distributed naturally along its length:

$$\epsfbox{00203x01.eps}$$

There should be a way to prove the existence of such "stray" capacitance in a substantial length of two-conductor cable.  Devise an experiment to do this.

\underbar{file 00203}
%(END_QUESTION)





%(BEGIN_ANSWER)

It is the nature of capacitance to store electrical charges, manifested in the form of static voltage.  Testing for the presence of a stored charge between the two conductors of a cable would be one way to prove the existence of capacitance within the cable.  I'll leave the details of testing for a stored electrical charge to you!

%(END_ANSWER)





%(BEGIN_NOTES)

The purpose of this question is to make students think critically and creatively about capacitance.  There is more than one way to test for capacitance in a cable, so do not limit students to one method only!

%INDEX% Capacitance, parasitic

%(END_NOTES)


