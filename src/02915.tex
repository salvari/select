
%(BEGIN_QUESTION)
% Copyright 2005, Tony R. Kuphaldt, released under the Creative Commons Attribution License (v 1.0)
% This means you may do almost anything with this work of mine, so long as you give me proper credit

Complete the truth table for a three-input AND gate:

$$\epsfbox{02915x01.eps}$$

% No blank lines allowed between lines of an \halign structure!
% I use comments (%) instead, so that TeX doesn't choke.

$$\vbox{\offinterlineskip
\halign{\strut
\vrule \quad\hfil # \ \hfil & 
\vrule \quad\hfil # \ \hfil & 
\vrule \quad\hfil # \ \hfil & 
\vrule \quad\hfil # \ \hfil \vrule \cr
\noalign{\hrule}
%
% First row
A & B & C & Output \cr
%
\noalign{\hrule}
%
% Second row
0 & 0 & 0 & \cr
%
\noalign{\hrule}
%
% Third row
0 & 0 & 1 &  \cr
%
\noalign{\hrule}
%
% Fourth row
0 & 1 & 0 &  \cr
%
\noalign{\hrule}
%
% Fifth row
0 & 1 & 1 &  \cr
%
\noalign{\hrule}
%
% Sixth row
1 & 0 & 0 &  \cr
%
\noalign{\hrule}
%
% Seventh row
1 & 0 & 1 &  \cr
%
\noalign{\hrule}
%
% Eighth row
1 & 1 & 0 &  \cr
%
\noalign{\hrule}
%
% Ninth row
1 & 1 & 1 &  \cr
%
\noalign{\hrule}
} % End of \halign 
}$$ % End of \vbox

\underbar{file 02915}
%(END_QUESTION)





%(BEGIN_ANSWER)

% No blank lines allowed between lines of an \halign structure!
% I use comments (%) instead, so that TeX doesn't choke.

$$\vbox{\offinterlineskip
\halign{\strut
\vrule \quad\hfil # \ \hfil & 
\vrule \quad\hfil # \ \hfil & 
\vrule \quad\hfil # \ \hfil & 
\vrule \quad\hfil # \ \hfil \vrule \cr
\noalign{\hrule}
%
% First row
A & B & C & Output \cr
%
\noalign{\hrule}
%
% Second row
0 & 0 & 0 & 0 \cr
%
\noalign{\hrule}
%
% Third row
0 & 0 & 1 & 0 \cr
%
\noalign{\hrule}
%
% Fourth row
0 & 1 & 0 & 0 \cr
%
\noalign{\hrule}
%
% Fifth row
0 & 1 & 1 & 0 \cr
%
\noalign{\hrule}
%
% Sixth row
1 & 0 & 0 & 0 \cr
%
\noalign{\hrule}
%
% Seventh row
1 & 0 & 1 & 0 \cr
%
\noalign{\hrule}
%
% Eighth row
1 & 1 & 0 & 0 \cr
%
\noalign{\hrule}
%
% Ninth row
1 & 1 & 1 & 1 \cr
%
\noalign{\hrule}
} % End of \halign 
}$$ % End of \vbox

\vskip 10pt

Follow-up question: how do you suppose the truth tables would look like for three-input OR, NOR, and NAND gates?  Explain how one may transition from the regular two-input gates to gate circuits with more than two inputs.  What remains the same despite additional input lines?

%(END_ANSWER)





%(BEGIN_NOTES)

There isn't much to comment on here, but this is a concept some students do not immediately see (how gates work with more than two inputs).

%INDEX% Truth table, for 3-input AND gate

%(END_NOTES)


