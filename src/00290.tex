
%(BEGIN_QUESTION)
% Copyright 2003, Tony R. Kuphaldt, released under the Creative Commons Attribution License (v 1.0)
% This means you may do almost anything with this work of mine, so long as you give me proper credit

Complete the schematic diagram for a SPDT relay circuit that energizes the green light bulb (only) when the pushbutton switch is pressed, and energizes the red light bulb (only) when the pushbutton switch is released:

$$\epsfbox{00290x01.eps}$$

\underbar{file 00290}
%(END_QUESTION)





%(BEGIN_ANSWER)

In order for this circuit to function as specified, the green light bulb must receive power through the relay's normally-open contact, and the red light bulb through the relay's normally-closed contact.

%(END_ANSWER)





%(BEGIN_NOTES)

If any students ask what "SPDT" means, refer them to a text or other information source on switch contacts in general (SPST, SPDT, DPST, DPDT, etc.).

Ground symbols were used intentionally in this question, to eliminate clutter from the diagram, and also to make students more familiar with their use as a notation for a common (reference) point in a circuit.

This question also reveals another useful feature of relays, and that is logic inversion.  The green light operates in the same mode as the pushbutton switch, but the red light is opposite of the pushbutton switch.  With just a single pushbutton operator, two complementary functions may be performed through the use of a SPDT relay.

%INDEX% Relay circuit, simple

%(END_NOTES)


