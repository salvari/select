
%(BEGIN_QUESTION)
% Copyright 2003, Tony R. Kuphaldt, released under the Creative Commons Attribution License (v 1.0)
% This means you may do almost anything with this work of mine, so long as you give me proper credit

The characteristically colored glow from a gas-discharge electric light is the result of energy emitted by electrons in the gas atoms as they fall from high-level "excited" states back to their natural ("ground") states.  As a general rule of electron behavior, they must absorb energy from an external source to leap into a higher level, and they release that energy upon returning to their original level.

Given the existence of this phenomenon, what do you suspect might be occurring inside a PN junction as it conducts an electric current?

\underbar{file 00911}
%(END_QUESTION)





%(BEGIN_ANSWER)

PN junctions emit energy of a characteristic wavelength when conducting current.  For some types of PN junctions, the wavelengths are within the visible range of light.

\vskip 10pt

Follow-up question: what practical application can you think of for this phenomenon?

%(END_ANSWER)





%(BEGIN_NOTES)

The practical application of this phenomenon should be obvious, and it is very commonplace in modern electronic equipment.  Discuss with your students the energy-efficiency of this light emission as compared to an incandescent lamp.

%INDEX% Electron, excited

%(END_NOTES)


