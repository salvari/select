
% This line effectively turns off "Underfull \vbox" error messages.
\vbadness=10000

\centerline{\bf First year Core Electronics course sequence} \bigskip 

\vskip 5pt
\hrule
\vskip 5pt

\noindent
Quarter 1 (12 weeks = 306 clock hours)

\vskip 10pt

{\bf ELTR100 -- DC 1} \hskip 10pt 75 hours = 4 credits 
 
\vskip 10pt

{\bf ELTR105 -- DC 2} \hskip 10pt 75 hours = 4 credits 
 
\vskip 10pt

{\bf ELTR110 -- AC 1} \hskip 10pt 78 hours = 4 credits 
 
\vskip 10pt

{\bf ELTR115 -- AC 2} \hskip 10pt 78 hours = 4 credits 
 
\vskip 10pt


\vskip 5pt
\hrule
\vskip 5pt

\noindent
Quarter 2 (12 weeks = 306 clock hours)

\vskip 10pt

{\bf ELTR120 -- Semiconductors 1} \hskip 10pt 90 hours = 5 credits 
 
\vskip 10pt

{\bf ELTR125 -- Semiconductors 2} \hskip 10pt 90 hours = 5 credits 
 
\vskip 10pt

{\bf ELTR130 -- Opamps 1} \hskip 10pt 66 hours = 3 credits 
 
\vskip 10pt

{\bf ELTR135 -- Opamps 2} \hskip 10pt 60 hours = 3 credits 
 
\vskip 10pt


\vskip 5pt
\hrule
\vskip 5pt

\noindent
Quarter 3 (12 weeks = 306 clock hours)

\vskip 10pt

{\bf ELTR140 -- Digital 1} \hskip 10pt 90 hours = 5 credits 
 
\vskip 10pt

{\bf ELTR145 -- Digital 2} \hskip 10pt 90 hours = 5 credits 
 
\vskip 10pt

{\bf ELTR152 -- Microprocessors} \hskip 10pt 126 hours
 

\vfil \eject

\centerline{\bf ELTR 100 (DC 1), section 1} \bigskip 
 
\vskip 10pt

\noindent
{\bf Recommended schedule}

\vskip 5pt

%%%%%%%%%%%%%%%
\hrule \vskip 5pt
\noindent
\underbar{Day 1}

\hskip 10pt Topics: {\it Introduction to trades and programs}
 
\hskip 10pt Questions: {\it (none)}
 
\hskip 10pt Lab Exercises: {\it Gather books, tools, and parts}

\vskip 10pt
%%%%%%%%%%%%%%%
\hrule \vskip 5pt
\noindent
\underbar{Day 2}

\hskip 10pt Topics: {\it Basic concepts of electricity, simple circuits, and voltmeter/ammeter usage}
 
\hskip 10pt Questions: {\it 1 through 20}
 
\hskip 10pt Lab Exercises: {\it Voltmeter usage (question 101) and Ammeter usage (question 102)}
 
%INSTRUCTOR \hskip 10pt {\bf MIT 8.02 video clip: Disk 1, Lecture 1; Van De Graff 25:45 to 29:30}

%INSTRUCTOR \hskip 10pt {\bf MIT 8.02 video clip: Disk 1, Lecture 1; Electroscope 42:08 to end}

%INSTRUCTOR \hskip 10pt {\bf Socratic Electronics animation: Simple switch circuit}

\vskip 10pt
%%%%%%%%%%%%%%%
\hrule \vskip 5pt
\noindent
\underbar{Day 3}

\hskip 10pt Topics: {\it Ohm's Law and electrical safety}
 
\hskip 10pt Questions: {\it 21 through 40}
 
\hskip 10pt Lab Exercise: {\it Circuit with switch (question 103)}
 
%INSTRUCTOR \hskip 10pt {\bf MIT 8.02 video clip: Disk 1, Lecture 5; Faraday cage 45:40 to end}

%INSTRUCTOR \hskip 10pt {\bf MIT 6.002 video clip: Disk 1, Lecture 1; V/I plots 29:20 to 35:12}

%INSTRUCTOR \hskip 10pt {\bf MIT 6.002 video clip: Disk 1, Lecture 1; Burning pickle 35:35 to 37:28}

\vskip 10pt
%%%%%%%%%%%%%%%
\hrule \vskip 5pt
\noindent
\underbar{Day 4}

\hskip 10pt Topics: {\it Ohm's Law, Joule's Law, scientific notation, and metric prefixes}
 
\hskip 10pt Questions: {\it 41 through 60}
 
\hskip 10pt Lab Exercise: {\it Ohm's Law (question 104)}
 
\vskip 10pt
%%%%%%%%%%%%%%%
\hrule \vskip 5pt
\noindent
\underbar{Day 5}

\hskip 10pt Topics: {\it Resistors, precision, the standard color code, and ohmmeter usage}
 
\hskip 10pt Questions: {\it 61 through 80}
 
\hskip 10pt Lab Exercise: {\it Ohmmeter usage (question 105)}
 
%INSTRUCTOR \hskip 10pt {\bf Demo: Show dissected potentiometer to students}

\vskip 10pt
%%%%%%%%%%%%%%%
\hrule \vskip 5pt
\noindent
\underbar{Day 6}

\hskip 10pt Topics: {\it Circuit connections, soldering technique, and solderless breadboards}
 
\hskip 10pt Questions: {\it 81 through 100}
 
\hskip 10pt Lab Exercise: {\it Ohm's Law (question 106)}
 
%INSTRUCTOR \hskip 10pt {\bf Demo: Show a breadboard next to a terminal block and a printed circuit board}

%INSTRUCTOR \hskip 10pt {\bf Socratic Electronics animation: Soldering a wire to a lug}

\vskip 10pt
%%%%%%%%%%%%%%%
\hrule \vskip 5pt
\noindent
\underbar{Day 7}

\hskip 10pt Exam 1: {\it includes Ohm's Law performance assessment}
 
\hskip 10pt Lab Exercises: {\it PCB soldering (question 107) and exploring solder-together kit}
 
\vskip 10pt
%%%%%%%%%%%%%%%
\hrule \vskip 5pt
\noindent
\underbar{Practice and challenge problems}

\hskip 10pt Questions: {\it 110 through the end of the worksheet}
 
\vskip 10pt
%%%%%%%%%%%%%%%
\hrule \vskip 5pt
\noindent
\underbar{Impending deadlines}

\hskip 10pt {\bf Troubleshooting assessment (simple lamp circuit) due at end of ELTR100, Section 3}
 
\hskip 10pt Question 108: Troubleshooting log
 
\hskip 10pt Question 109: Sample troubleshooting assessment grading criteria
 
\hskip 10pt {\bf Solder-together kit due at end of ELTR100, Section 3}

\vskip 10pt
%%%%%%%%%%%%%%%




\vfil \eject

\centerline{\bf ELTR 100 (DC 1), section 2} \bigskip 
 
\vskip 10pt

\noindent
{\bf Recommended schedule}

\vskip 5pt

%%%%%%%%%%%%%%%
\hrule \vskip 5pt
\noindent
\underbar{Day 1}

\hskip 10pt Topics: {\it Series circuits and troubleshooting}
 
\hskip 10pt Questions: {\it 1 through 20}
 
\hskip 10pt Lab Exercises: {\it Series resistances (question 61)}
 
%INSTRUCTOR \hskip 10pt {\bf Demo: Set up and use battery/lamp troubleshooting board in front of class}

\vskip 10pt
%%%%%%%%%%%%%%%
\hrule \vskip 5pt
\noindent
\underbar{Day 2}

\hskip 10pt Topics: {\it Series circuits, wire resistance, and overcurrent protection}
 
\hskip 10pt Questions: {\it 21 through 40}
 
\hskip 10pt Lab Exercise: {\it Series DC resistor circuit (question 62)}
 
%INSTRUCTOR \hskip 10pt {\bf Demo: Show wire pieces of several different gauge}

\vskip 10pt
%%%%%%%%%%%%%%%
\hrule \vskip 5pt
\noindent
\underbar{Day 3}

\hskip 10pt Topics: {\it Series circuits, voltage divider circuits, and Kirchhoff's Voltage Law}
 
\hskip 10pt Questions: {\it 41 through 60}
 
\hskip 10pt Lab Exercise: {\it Series DC resistor circuit (question 63)}
 
%INSTRUCTOR \hskip 10pt {\bf MIT 6.002 video clip: Disk 1, Lecture 2; Kirchhoff's Voltage Law 8:02 to 10:36}

%INSTRUCTOR \hskip 10pt {\bf Demo: Set up battery/resistors circuit to validate KVL around any arbitrary path}

%INSTRUCTOR \hskip 10pt {\bf MIT 8.02 video clip: Disk 3, Lecture 19; Electrocardiogram demo 15:55 to 20:25}

\vskip 10pt
%%%%%%%%%%%%%%%
\hrule \vskip 5pt
\noindent
\underbar{Day 4}

\hskip 10pt Exam 2: {\it includes Series DC resistor circuit performance assessment}
 
\hskip 10pt Lab Exercise: {\it Troubleshooting practice (simple light bulb circuit)}
 
\vskip 10pt
%%%%%%%%%%%%%%%
\hrule \vskip 5pt
\noindent
\underbar{Practice and challenge problems}

\hskip 10pt Questions: {\it 66 through the end of the worksheet}
 
\vskip 10pt
%%%%%%%%%%%%%%%
\hrule \vskip 5pt
\noindent
\underbar{Impending deadlines}

\hskip 10pt {\bf Troubleshooting assessment (simple lamp circuit) due at end of ELTR100, Section 3}
 
\hskip 10pt Question 64: Troubleshooting log
 
\hskip 10pt Question 65: Sample troubleshooting assessment grading criteria
 
\hskip 10pt {\bf Solder-together kit due at end of ELTR100, Section 3}

\vskip 10pt
%%%%%%%%%%%%%%%




\vfil \eject

\centerline{\bf ELTR 100 (DC 1), section 3} \bigskip 
 
\vskip 10pt

\noindent
{\bf Recommended schedule}

\vskip 5pt

%%%%%%%%%%%%%%%
\hrule \vskip 5pt
\noindent
\underbar{Day 1}

\hskip 10pt Topics: {\it Parallel circuits, current sources, and troubleshooting}
 
\hskip 10pt Questions: {\it 1 through 20}
 
\hskip 10pt Lab Exercises: {\it Parallel resistances (question 61)}
 
%INSTRUCTOR \hskip 10pt {\bf Demo: Show how to set up regulated power supply for sourcing fixed amount of current}

\vskip 10pt
%%%%%%%%%%%%%%%
\hrule \vskip 5pt
\noindent
\underbar{Day 2}

\hskip 10pt Topics: {\it Parallel circuits, Kirchhoff's Current Law, and chemical batteries}
 
\hskip 10pt Questions: {\it 21 through 40}
 
\hskip 10pt Lab Exercise: {\it Parallel DC resistor circuit (question 62)}
 
%INSTRUCTOR \hskip 10pt {\bf MIT 6.002 video clip: Disk 1, Lecture 2; Kirchhoff's Current Law 10:36 to 12:51}

%INSTRUCTOR \hskip 10pt {\bf MIT 8.02 video clip: Disk 2, Lecture 10; Battery demonstration 11:38 to 15:00}

%INSTRUCTOR \hskip 10pt {\bf MIT 8.02 video clip: Disk 2, Lecture 10; Shorting a 9-V battery 26:47 to 28:29}

%INSTRUCTOR \hskip 10pt {\bf MIT 8.02 video clip: Disk 2, Lecture 10; Shorting a 12-V car battery 29:22 to 31:18}

\vskip 10pt
%%%%%%%%%%%%%%%
\hrule \vskip 5pt
\noindent
\underbar{Day 3}

\hskip 10pt Topics: {\it Parallel circuits, current divider circuits, and temperature coefficient of resistance}
 
\hskip 10pt Questions: {\it 41 through 60}
 
\hskip 10pt Lab Exercise: {\it Parallel DC resistor circuit (question 63)}
 
%INSTRUCTOR \hskip 10pt {\bf Demo: Measure resistance of a magnet wire spool, both cold and room temperature}

\vskip 10pt
%%%%%%%%%%%%%%%
\hrule \vskip 5pt
\noindent
\underbar{Day 4}

\hskip 10pt Exam 3: {\it includes Parallel DC resistor circuit performance assessment}
 
\hskip 10pt {\bf Troubleshooting assessment due:} {\it Simple lamp circuit}
 
\hskip 10pt Question 64: Troubleshooting log
 
\hskip 10pt Question 65: Sample troubleshooting assessment grading criteria
 
\hskip 10pt {\bf Project due:} {\it Solder-together electronic kit}
 
\vskip 10pt
%%%%%%%%%%%%%%%
\hrule \vskip 5pt
\noindent
\underbar{Troubleshooting practice problems}

\hskip 10pt Questions: {\it 66 through 75}
 
\vskip 10pt
%%%%%%%%%%%%%%%
\hrule \vskip 5pt
\noindent
\underbar{General concept practice and challenge problems}

\hskip 10pt Questions: {\it 76 through the end of the worksheet}
 
\vskip 10pt
%%%%%%%%%%%%%%%





\vfil \eject

\centerline{\bf ELTR 105 (DC 2), section 1} \bigskip 
 
\vskip 10pt

\noindent
{\bf Recommended schedule}

\vskip 5pt

%%%%%%%%%%%%%%%
\hrule \vskip 5pt
\noindent
\underbar{Day 1}

\hskip 10pt Topics: {\it Series-parallel circuit analysis}
 
\hskip 10pt Questions: {\it 1 through 15}
 
\hskip 10pt Lab Exercise: {\it Kirchhoff's Voltage Law (question 61)}
 
\vskip 10pt
%%%%%%%%%%%%%%%
\hrule \vskip 5pt
\noindent
\underbar{Day 2}

\hskip 10pt Topics: {\it Series-parallel circuits and Wheatstone bridges}
 
\hskip 10pt Questions: {\it 16 through 30}
 
\hskip 10pt Lab Exercise: {\it Wheatstone bridge (question 62)}
 
\vskip 10pt
%%%%%%%%%%%%%%%
\hrule \vskip 5pt
\noindent
\underbar{Day 3}

\hskip 10pt Topics: {\it Series-parallel circuits, safety grounding, and troubleshooting}
 
\hskip 10pt Questions: {\it 31 through 45}
 
\hskip 10pt Lab Exercise: {\it Series-parallel DC resistor circuit (question 63)}
 
\vskip 10pt
%%%%%%%%%%%%%%%
\hrule \vskip 5pt
\noindent
\underbar{Day 4}

\hskip 10pt Topics: {\it Loaded voltage dividers}
 
\hskip 10pt Questions: {\it 46 through 60}
 
\hskip 10pt Lab Exercise: {\it Loaded voltage divider (question 64)}
 
\vskip 10pt
%%%%%%%%%%%%%%%
\hrule \vskip 5pt
\noindent
\underbar{Day 5}

\hskip 10pt Exam 1: {\it includes Series-parallel DC resistor circuit performance assessment}
 
\hskip 10pt Lab Exercise: {\it Troubleshooting practice (loaded voltage divider circuit -- question 64)}
 
\vskip 10pt
%%%%%%%%%%%%%%%
\hrule \vskip 5pt
\noindent
\underbar{Practice and challenge problems}

\hskip 10pt Questions: {\it 67 through the end of the worksheet}
 
\vskip 10pt
%%%%%%%%%%%%%%%
\hrule \vskip 5pt
\noindent
\underbar{Impending deadlines}

\hskip 10pt {\bf Troubleshooting assessment (voltage divider) due at end of ELTR105, Section 3}
 
\hskip 10pt Question 65: Troubleshooting log
 
\hskip 10pt Question 66: Sample troubleshooting assessment grading criteria
 
\vskip 10pt
%%%%%%%%%%%%%%%








\vfil \eject

\centerline{\bf ELTR 105 (DC 2), section 2} \bigskip 
 
\vskip 10pt

\noindent
{\bf Recommended schedule}

\vskip 5pt

%%%%%%%%%%%%%%%
\hrule \vskip 5pt
\noindent
\underbar{Day 1}

\hskip 10pt Topics: {\it Magnetism, electromagnetism, and electromagnetic induction}
 
\hskip 10pt Questions: {\it 1 through 20}
 
\hskip 10pt Lab Exercises: {\it Electromagnetism (question 71)}
 
%INSTRUCTOR \hskip 10pt {\bf MIT 8.02 video clip: Disk 2, Lecture 14; Single-wire field demo 17:32 to 18:35}

%INSTRUCTOR \hskip 10pt {\bf MIT 8.02 video clip: Disk 2, Lecture 14; Twin-wire field demo 19:08 to 20:14}

%INSTRUCTOR \hskip 10pt {\bf MIT 8.02 video clip: Disk 2, Lecture 11; Electromagnetism demos 6:50 to 12:45}

%INSTRUCTOR \hskip 10pt {\bf MIT 8.02 video clip: Disk 2, Lecture 11; Forces between two wires 12:56 to 17:20}

%INSTRUCTOR \hskip 10pt {\bf MIT 8.02 video clip: Disk 3, Lecture 15; Solenoid coil demos 23:00 to 27:15}

%INSTRUCTOR \hskip 10pt {\bf MIT 8.02 video clip: Disk 3, Lecture 16; Electromagnetic induction 10:30 to 12:30}

\vskip 10pt
%%%%%%%%%%%%%%%
\hrule \vskip 5pt
\noindent
\underbar{Day 2}

\hskip 10pt Topics: {\it Applications of electromagnetism and induction, Lenz's Law}
 
\hskip 10pt Questions: {\it 21 through 40}
 
\hskip 10pt Lab Exercise: {\it Electromagnetic induction (question 72)}
 
%INSTRUCTOR \hskip 10pt {\bf MIT 8.02 video clip: Disk 3, Lecture 21; Solenoid and armature demo 20:30 to 22:55}

%INSTRUCTOR \hskip 10pt {\bf MIT 8.02 video clip: Disk 2, Lecture 11; Torque on wire loop demo 46:34 to end}

%INSTRUCTOR \hskip 10pt {\bf MIT 8.02 video clip: Disk 3, Lecture 21; Homemade motor demo 4:20 to 5:55}

%INSTRUCTOR \hskip 10pt {\bf Demo: Use powerful magnet(s) and aluminum coin to show Lenz's Law}

%INSTRUCTOR \hskip 10pt {\bf MIT 8.02 video clip: Disk 3, Lecture 17; Damped pendulum demo 38:00 to 41:50}

%INSTRUCTOR \hskip 10pt {\bf MIT 8.02 video clip: Disk 3, Lecture 17; Ring drop demo 45:10 to 48:51}

%INSTRUCTOR \hskip 10pt {\bf Demo: Ammeter showing current drawn by electric motor under varying load}

\vskip 10pt
%%%%%%%%%%%%%%%
\hrule \vskip 5pt
\noindent
\underbar{Day 3}

\hskip 10pt Topics: {\it Introduction to Th\'evenin's and Norton's theorems}
 
\hskip 10pt Questions: {\it 41 through 55}
 
\hskip 10pt Lab Exercise: {\it Th\'evenin's theorem (question 73)}
 
%INSTRUCTOR \hskip 10pt {\bf Socratic Electronics animation: Th\'evenin's theorem demonstrated}

%INSTRUCTOR \hskip 10pt {\bf Demo: Build a Th\'evenin equivalent circuit for a more complex circuit}

\vskip 10pt
%%%%%%%%%%%%%%%
\hrule \vskip 5pt
\noindent
\underbar{Day 4}

\hskip 10pt Topics: {\it Th\'evenin's, Norton's, and Maximum Power Transfer theorems}
 
\hskip 10pt Questions: {\it 56 through 70} 
 
\hskip 10pt Lab Exercise: {\it Th\'evenin's theorem (question 73, continued)}
 
\vskip 10pt
%%%%%%%%%%%%%%%
\hrule \vskip 5pt
\noindent
\underbar{Day 5}

\hskip 10pt Exam 2: {\it includes Th\'evenin equivalent circuit performance assessment}
 
\hskip 10pt Lab Exercise: {\it Troubleshooting practice (loaded voltage divider circuit -- question 74)}
 
\vskip 10pt
%%%%%%%%%%%%%%%
\hrule \vskip 5pt
\noindent
\underbar{Practice and challenge problems}

\hskip 10pt Questions: {\it 77 through the end of the worksheet}
 
\vskip 10pt
%%%%%%%%%%%%%%%
\hrule \vskip 5pt
\noindent
\underbar{Impending deadlines}

\hskip 10pt {\bf Troubleshooting assessment (voltage divider) due at end of ELTR105, Section 3}
 
\hskip 10pt Question 75: Troubleshooting log
 
\hskip 10pt Question 76: Sample troubleshooting assessment grading criteria
 
\vskip 10pt
%%%%%%%%%%%%%%%









\vfil \eject

\centerline{\bf ELTR 105 (DC 2), section 3} \bigskip 
 
\vskip 10pt

\noindent
{\bf Recommended schedule}

\vskip 5pt

%%%%%%%%%%%%%%%
\hrule \vskip 5pt
\noindent
\underbar{Day 1}

\hskip 10pt Topics: {\it Inductance and inductors}
 
\hskip 10pt Questions: {\it 1 through 20}
 
\hskip 10pt Lab Exercises: {\it Series inductances (question 81) and parallel inductances (question 82)}
 
%INSTRUCTOR \hskip 10pt {\bf Demo: Light up a neon bulb using an inductor and a low-voltage battery}

%INSTRUCTOR \hskip 10pt {\bf Demo: Show picture of a substation inductor (fault current limiter)}

\vskip 10pt
%%%%%%%%%%%%%%%
\hrule \vskip 5pt
\noindent
\underbar{Day 2}

\hskip 10pt Topics: {\it Capacitance and capacitors}
 
\hskip 10pt Questions: {\it 21 through 40}
 
\hskip 10pt Lab Exercises: {\it Series capacitances (question 83) and parallel capacitances (question 84)}
 
%INSTRUCTOR \hskip 10pt {\bf MIT 8.02 video clip: Disk 1, Lecture 2; Electric field lines shown 41:00 to 45:45}

%INSTRUCTOR \hskip 10pt {\bf MIT 8.02 video clip: Disk 1, Lecture 3; Electric field between plates 47:52 to end}

%INSTRUCTOR \hskip 10pt {\bf MIT 6.002 video clip: Disk 2, Lecture 13; Capacitor discharge demo 24:55 to 27:20}

%INSTRUCTOR \hskip 10pt {\bf Demo: Charge a long two-wire cable, then measure stored voltage}

\vskip 10pt
%%%%%%%%%%%%%%%
\hrule \vskip 5pt
\noindent
\underbar{Day 3}

\hskip 10pt Topics: {\it Time constants}
 
\hskip 10pt Questions: {\it 41 through 60}
 
\hskip 10pt Lab Exercise: {\it RC discharge circuit (question 85)}
 
%INSTRUCTOR \hskip 10pt {\bf MIT 6.002 video clip: Disk 2, Lecture 13; RC time constant demo 47:03 to end}

%INSTRUCTOR \hskip 10pt {\bf MIT 8.02 video clip: Disk 3, Lecture 20; L/R time constant demo 26:27 to 30:30}

\vskip 10pt
%%%%%%%%%%%%%%%
\hrule \vskip 5pt
\noindent
\underbar{Day 4}

\hskip 10pt Topics: {\it Time constant circuits}
 
\hskip 10pt Questions: {\it 61 through 80}
 
\hskip 10pt Lab Exercise: {\it Time-delay relay (question 86)}
 
\vskip 10pt
%%%%%%%%%%%%%%%
\hrule \vskip 5pt
\noindent
\underbar{Day 5}

\hskip 10pt Exam 3: {\it includes RC discharge circuit performance assessment}
 
\hskip 10pt {\bf Troubleshooting Assessment due:} {\it Loaded voltage divider (question 87)}
 
\hskip 10pt Question 88: Troubleshooting log
 
\hskip 10pt Question 89: Sample troubleshooting assessment grading criteria
 
\vskip 10pt
%%%%%%%%%%%%%%%
\hrule \vskip 5pt
\noindent
\underbar{Practice and challenge problems}

\hskip 10pt Questions: {\it 90 through the end of the worksheet}
 
\vskip 10pt
%%%%%%%%%%%%%%%







\vfil \eject

\centerline{\bf ELTR 110 (AC 1), section 1} \bigskip 
 
\vskip 10pt

\noindent
{\bf Recommended schedule}

\vskip 5pt

%%%%%%%%%%%%%%%
\hrule \vskip 5pt
\noindent
\underbar{Day 1}

\hskip 10pt Topics: {\it Basic concepts of AC and oscilloscope usage}
 
\hskip 10pt Questions: {\it 1 through 20}
 
\hskip 10pt Lab Exercise: {\it Analog oscilloscope set-up (question 81)}
 
%INSTRUCTOR \hskip 10pt {\bf Demo: function generator and speaker to show what AC "sounds like"}

%INSTRUCTOR \hskip 10pt {\bf Demo: function generator and oscilloscope}

\vskip 10pt
%%%%%%%%%%%%%%%
\hrule \vskip 5pt
\noindent
\underbar{Day 2}

\hskip 10pt Topics: {\it RMS quantities, phase shift, and phasor addition}
 
\hskip 10pt Questions: {\it 21 through 40}
 
\hskip 10pt Lab Exercise: {\it RMS versus peak measurements (question 82) and measuring frequency (question 83)}
 
%INSTRUCTOR \hskip 10pt {\bf Socratic Electronics animation: Lissajous figures on an oscilloscope}

%INSTRUCTOR \hskip 10pt {\bf Demo: two function generators and an oscilloscope to show Lissajous figures}

\vskip 10pt
%%%%%%%%%%%%%%%
\hrule \vskip 5pt
\noindent
\underbar{Day 3}

\hskip 10pt Topics: {\it Inductive reactance and impedance, trigonometry for AC circuits}
 
\hskip 10pt Questions: {\it 41 through 60}
 
\hskip 10pt Lab Exercise: {\it Inductive reactance and Ohm's Law for AC (question 84)}
 
%INSTRUCTOR \hskip 10pt {\bf Demo: function generator, inductor, and multimeter to show inductive reactance}

\vskip 10pt
%%%%%%%%%%%%%%%
\hrule \vskip 5pt
\noindent
\underbar{Day 4}

\hskip 10pt Topics: {\it Series and parallel LR circuits}
 
\hskip 10pt Questions: {\it 61 through 80}
 
\hskip 10pt Lab Exercise: {\it Series LR circuit (question 85)}
 
\vskip 10pt
%%%%%%%%%%%%%%%
\hrule \vskip 5pt
\noindent
\underbar{Day 5}

\hskip 10pt Exam 1: {\it includes Inductive reactance performance assessment}
 
\hskip 10pt Lab Exercise: {\it Oscilloscope probe ($\times$ 10) compensation (question 86)}

\vskip 10pt
%%%%%%%%%%%%%%%
\hrule \vskip 5pt
\noindent
\underbar{Practice and challenge problems}

\hskip 10pt Questions: {\it 89 through the end of the worksheet}
 
\vskip 10pt
%%%%%%%%%%%%%%%
\hrule \vskip 5pt
\noindent
\underbar{Impending deadlines}

\hskip 10pt {\bf Troubleshooting assessment (AC bridge circuit) due at end of ELTR110, Section 3}
 
\hskip 10pt Question 87: Troubleshooting log
 
\hskip 10pt Question 88: Sample troubleshooting assessment grading criteria
 
\vskip 10pt
%%%%%%%%%%%%%%%







\vfil \eject

\centerline{\bf ELTR 110 (AC 1), section 2} \bigskip 
 
\vskip 10pt

\noindent
{\bf Recommended schedule}

\vskip 5pt

%%%%%%%%%%%%%%%
\hrule \vskip 5pt
\noindent
\underbar{Day 1}

\hskip 10pt Topics: {\it Capacitive reactance and impedance, trigonometry for AC circuits}
 
\hskip 10pt Questions: {\it 1 through 20}
 
\hskip 10pt Lab Exercise: {\it Capacitive reactance and Ohm's Law for AC (question 71)}
 
\vskip 10pt
%%%%%%%%%%%%%%%
\hrule \vskip 5pt
\noindent
\underbar{Day 2}

\hskip 10pt Topics: {\it Series and parallel RC circuits}
 
\hskip 10pt Questions: {\it 21 through 40}
 
\hskip 10pt Lab Exercise: {\it Series RC circuit (question 72)}
 
\vskip 10pt
%%%%%%%%%%%%%%%
\hrule \vskip 5pt
\noindent
\underbar{Day 3}

\hskip 10pt Topics: {\it Superposition principle, AC+DC oscilloscope coupling}
 
\hskip 10pt Questions: {\it 41 through 55}
 
\hskip 10pt Lab Exercise: {\it Parallel RC circuit (question 73)}
 
%INSTRUCTOR \hskip 10pt {\bf MIT 6.002 video clip: Disk 1, Lecture 3; Superposition theorem 29:18 to 34:14}

\vskip 10pt
%%%%%%%%%%%%%%%
\hrule \vskip 5pt
\noindent
\underbar{Day 4}

\hskip 10pt Topics: {\it Passive RC and LR filter circuits}
 
\hskip 10pt Questions: {\it 56 through 70}
 
\hskip 10pt Lab Exercise: {\it Time-domain phase shift measurement (question 74)}
 
%INSTRUCTOR \hskip 10pt {\bf MIT 8.02 video clip: Disk 3, Lecture 20; Lowpass filter using inductor 37:46 to 39:18}

\vskip 10pt
%%%%%%%%%%%%%%%
\hrule \vskip 5pt
\noindent
\underbar{Day 5}

\hskip 10pt Exam 2: {\it includes Series \underbar{or} Parallel RC circuit performance assessment}
 
\hskip 10pt Lab Exercise: {\it Troubleshooting practice (variable phase shift bridge circuit -- question 75)}
  
\vskip 10pt
%%%%%%%%%%%%%%%
\hrule \vskip 5pt
\noindent
\underbar{Practice and challenge problems}

\hskip 10pt Questions: {\it 78 through the end of the worksheet}
 
\vskip 10pt
%%%%%%%%%%%%%%%
\hrule \vskip 5pt
\noindent
\underbar{Impending deadlines}

\hskip 10pt {\bf Troubleshooting assessment (AC bridge circuit) due at end of ELTR110, Section 3}
 
\hskip 10pt Question 76: Troubleshooting log
 
\hskip 10pt Question 77: Sample troubleshooting assessment grading criteria
 
\vskip 10pt
%%%%%%%%%%%%%%%








\vfil \eject

\centerline{\bf ELTR 110 (AC 1), section 3} \bigskip 
 
\vskip 10pt

\noindent
{\bf Recommended schedule}

\vskip 5pt

%%%%%%%%%%%%%%%
\hrule \vskip 5pt
\noindent
\underbar{Day 1}

\hskip 10pt Topics: {\it RLC circuits}
 
\hskip 10pt Questions: {\it 1 through 15}
 
\hskip 10pt Lab Exercise: {\it Passive RC filter circuit design (question 61)}
 
\vskip 10pt
%%%%%%%%%%%%%%%
\hrule \vskip 5pt
\noindent
\underbar{Day 2}

\hskip 10pt Topics: {\it RLC circuits and AC bridge circuits}
 
\hskip 10pt Questions: {\it 16 through 30}
 
\hskip 10pt Lab Exercise: {\it Passive RC filter circuit design (question 61, continued)}
 
\vskip 10pt
%%%%%%%%%%%%%%%
\hrule \vskip 5pt
\noindent
\underbar{Day 3}

\hskip 10pt Topics: {\it Series and parallel resonance}
 
\hskip 10pt Questions: {\it 31 through 45}
 
\hskip 10pt Lab Exercise: {\it Measuring inductance by series resonance (question 62)}
 
\vskip 10pt
%%%%%%%%%%%%%%%
\hrule \vskip 5pt
\noindent
\underbar{Day 4}

\hskip 10pt Topics: {\it Resonant filter circuits, bandwidth, and Q}
 
\hskip 10pt Questions: {\it 46 through 60}
 
\hskip 10pt Lab Exercise: {\it Passive resonant filter circuit (question 63)}
 
%INSTRUCTOR \hskip 10pt {\bf MIT 6.002 video clip: Disk 3, Lecture 20; AM radio tuning 41:44 to end}

\vskip 10pt
%%%%%%%%%%%%%%%
\hrule \vskip 5pt
\noindent
\underbar{Day 5}

\hskip 10pt Exam 3: {\it includes Passive RC filter circuit design performance assessment}
 
\hskip 10pt {\bf Troubleshooting Assessment due:} {\it Variable phase shift bridge circuit (question 64)}
 
\hskip 10pt Question 65: Troubleshooting log
 
\hskip 10pt Question 66: Sample troubleshooting assessment grading criteria
 
\vskip 10pt
%%%%%%%%%%%%%%%
\hrule \vskip 5pt
\noindent
\underbar{Practice and challenge problems}

\hskip 10pt Questions: {\it 67 through the end of the worksheet}
 
\vskip 10pt
%%%%%%%%%%%%%%%








\vfil \eject

\centerline{\bf ELTR 115 (AC 2), section 1} \bigskip 
 
\vskip 10pt

\noindent
{\bf Recommended schedule}

\vskip 5pt

%%%%%%%%%%%%%%%
\hrule \vskip 5pt
\noindent
\underbar{Day 1}

\hskip 10pt Topics: {\it Mutual inductance and transformer theory}
 
\hskip 10pt Questions: {\it 1 through 15}
 
\hskip 10pt Lab Exercise: {\it Transformer voltage/current ratios (question 61)}
 
%INSTRUCTOR \hskip 10pt {\bf Explain suggested project ideas to students}

%INSTRUCTOR \hskip 10pt {\bf Give project grading rubric to students, complete with deadlines}

\vskip 10pt
%%%%%%%%%%%%%%%
\hrule \vskip 5pt
\noindent
\underbar{Day 2}

\hskip 10pt Topics: {\it Transformer step ratio}
 
\hskip 10pt Questions: {\it 16 through 30}
 
\hskip 10pt Lab Exercise: {\it Auto-transformers (question 62)}
 
%INSTRUCTOR \hskip 10pt {\bf MIT 8.02 video clip: Disk 4, Lecture 24; Transformer w/ loose wire 34:13 to 36:40}

%INSTRUCTOR \hskip 10pt {\bf MIT 8.02 video clip: Disk 4, Lecture 24; Shorted transformer 40:03 to 41:50}

%INSTRUCTOR \hskip 10pt {\bf MIT 8.02 video clip: Disk 4, Lecture 24; Ruhmkorff coil demo 47:45 to end}

\vskip 10pt
%%%%%%%%%%%%%%%
\hrule \vskip 5pt
\noindent
\underbar{Day 3}

\hskip 10pt Topics: {\it Maximum power transfer theorem and impedance matching with transformers}
 
\hskip 10pt Questions: {\it 31 through 45}
 
\hskip 10pt Lab Exercise: {\it Auto-transformers (question 63)}
 
\vskip 10pt
%%%%%%%%%%%%%%%
\hrule \vskip 5pt
\noindent
\underbar{Day 4}

\hskip 10pt Topics: {\it Transformer applications, power ratings, and core effects}
 
\hskip 10pt Questions: {\it 46 through 60}
 
\hskip 10pt Lab Exercise: {\it Differential voltage measurement using the oscilloscope (question 64)}
 
%INSTRUCTOR \hskip 10pt {\bf Demo: show a current transformer, and a clamp-on AC ammeter}

\vskip 10pt
%%%%%%%%%%%%%%%
\hrule \vskip 5pt
\noindent
\underbar{Day 5}

\hskip 10pt Exam 1: {\it includes Transformer voltage ratio performance assessment}
 
\hskip 10pt Lab Exercise: {\it work on project}
 
\hskip 10pt Project: {\it Initial project design checked by instructor and components selected (sensitive audio detector circuit recommended)}
  
\vskip 10pt
%%%%%%%%%%%%%%%
\hrule \vskip 5pt
\noindent
\underbar{Practice and challenge problems}

\hskip 10pt Questions: {\it 66 through the end of the worksheet}
 
\vskip 10pt
%%%%%%%%%%%%%%%
\hrule \vskip 5pt
\noindent
\underbar{Impending deadlines}

\hskip 10pt {\bf Project due at end of ELTR115, Section 3}
 
\hskip 10pt Question 65: Sample project grading criteria
 
\vskip 10pt
%%%%%%%%%%%%%%%










\vfil \eject

\centerline{\bf ELTR 115 (AC 2), section 2} \bigskip 
 
\vskip 10pt

\noindent
{\bf Recommended schedule}

\vskip 5pt

%%%%%%%%%%%%%%%
\hrule \vskip 5pt
\noindent
\underbar{Day 1}

\hskip 10pt Topics: {\it Power in AC circuits}
 
\hskip 10pt Questions: {\it 1 through 20}
 
\hskip 10pt Lab Exercise: {\it Lissajous figures for phase shift measurement (question 71)}
 
\vskip 10pt
%%%%%%%%%%%%%%%
\hrule \vskip 5pt
\noindent
\underbar{Day 2}

\hskip 10pt Topics: {\it Power factor correction}
 
\hskip 10pt Questions: {\it 21 through 40}
 
\hskip 10pt Lab Exercise: {\it Power factor correction for AC motor (question 72)}
 
%INSTRUCTOR \hskip 10pt {\bf Demo: show photos of substation power factor correction equipment}

\vskip 10pt
%%%%%%%%%%%%%%%
\hrule \vskip 5pt
\noindent
\underbar{Day 3}

\hskip 10pt Topics: {\it Alternator construction and introduction to polyphase AC}
 
\hskip 10pt Questions: {\it 41 through 55}
 
\hskip 10pt Lab Exercise: {\it Power factor correction for AC motor (question 72, continued)}
 
%INSTRUCTOR \hskip 10pt {\bf Demo: show disassembled automotive alternator}

%INSTRUCTOR \hskip 10pt {\bf Demo: operate 3-phase motor/generator unit}

\vskip 10pt
%%%%%%%%%%%%%%%
\hrule \vskip 5pt
\noindent
\underbar{Day 4}

\hskip 10pt Topics: {\it AC motor construction and polyphase AC circuits}
 
\hskip 10pt Questions: {\it 56 through 70}
 
\hskip 10pt Lab Exercise: {\it work on project}
 
%INSTRUCTOR \hskip 10pt {\bf MIT 8.02 video clip: Disk 3, Lecture 18; 3-phase motor demo 33:35 to 37:13}

%INSTRUCTOR \hskip 10pt {\bf Socratic Electronics animation: Three-phase motor}

%INSTRUCTOR \hskip 10pt {\bf Demo: operate 3-phase motor/generator unit}

%INSTRUCTOR \hskip 10pt {\bf MIT 8.02 video clip: Disk 3, Lecture 18; 1-phase motor demo 49:07 to end}

\vskip 10pt
%%%%%%%%%%%%%%%
\hrule \vskip 5pt
\noindent
\underbar{Day 5}

\hskip 10pt Exam 2: {\it includes Lissajous figure phase shift measurement performance assessment}
 
\vskip 10pt
%%%%%%%%%%%%%%%
\hrule \vskip 5pt
\noindent
\underbar{Practice and challenge problems}

\hskip 10pt Questions: {\it 74 through the end of the worksheet}
 
\vskip 10pt
%%%%%%%%%%%%%%%
\hrule \vskip 5pt
\noindent
\underbar{Impending deadlines}

\hskip 10pt {\bf Project due at end of ELTR115, Section 3}
 
\hskip 10pt Question 73: Sample project grading criteria
 
\vskip 10pt
%%%%%%%%%%%%%%%











\vfil \eject

\centerline{\bf ELTR 115 (AC 2), section 3} \bigskip 
 
\vskip 10pt

\noindent
{\bf Recommended schedule}

\vskip 5pt

%%%%%%%%%%%%%%%
\hrule \vskip 5pt
\noindent
\underbar{Day 1}

\hskip 10pt Topics: {\it Mixed-frequency signals and harmonic analysis}
 
\hskip 10pt Questions: {\it 1 through 15}
 
\hskip 10pt Lab Exercise: {\it Digital oscilloscope set-up (question 61)}
 
%INSTRUCTOR \hskip 10pt {\bf Demo: Use graphing calculator to synthesize square wave from sinusoidal harmonics}

%INSTRUCTOR \hskip 10pt {\bf Demo: Show harmonics using a spectrum analyzer and function generator}

%INSTRUCTOR \hskip 10pt {\bf Demo: Show harmonics in power-line signal using a spectrum analyzer and transformer}

%INSTRUCTOR \hskip 10pt {\bf Demo: Show example of spectrum plot from an amplifier datasheet}

\vskip 10pt
%%%%%%%%%%%%%%%
\hrule \vskip 5pt
\noindent
\underbar{Day 2}

\hskip 10pt Topics: {\it Intro to calculus: differentiation and integration (optional)}
 
\hskip 10pt Questions: {\it 16 through 30}
 
\hskip 10pt Lab Exercise: {\it Passive integrator circuit (question 62)}
 
\vskip 10pt
%%%%%%%%%%%%%%%
\hrule \vskip 5pt
\noindent
\underbar{Day 3}

\hskip 10pt Topics: {\it Passive integrator and differentiator circuits}
 
\hskip 10pt Questions: {\it 31 through 45}
 
\hskip 10pt Lab Exercise: {\it Passive differentiator circuit (question 63)}
 
\vskip 10pt
%%%%%%%%%%%%%%%
\hrule \vskip 5pt
\noindent
\underbar{Day 4}

\hskip 10pt Topics: {\it Using oscilloscope trigger controls}
 
\hskip 10pt Questions: {\it 46 through 60}
 
\hskip 10pt Lab Exercise: {\it work on project}

%INSTRUCTOR \hskip 10pt {\bf Demo: Use oscilloscope to show how to trigger a complex, repetitive signal}

\vskip 10pt
%%%%%%%%%%%%%%%
\hrule \vskip 5pt
\noindent
\underbar{Day 5}

\hskip 10pt Exam 3: {\it includes oscilloscope set-up performance assessment}

\hskip 10pt {\bf Project due}

\hskip 10pt Question 64: Sample project grading criteria
 
\vskip 10pt
%%%%%%%%%%%%%%%
\hrule \vskip 5pt
\noindent
\underbar{Practice and challenge problems}

\hskip 10pt Questions: {\it 65 through the end of the worksheet}
 
%%%%%%%%%%%%%%%





\vfil \eject

\centerline{\bf ELTR 120 (Semiconductors 1), section 1} \bigskip 
 
\vskip 10pt

\noindent
{\bf Recommended schedule}

\vskip 5pt

%%%%%%%%%%%%%%%
\hrule \vskip 5pt
\noindent
\underbar{Day 1}

\hskip 10pt Topics: {\it Semiconductor theory and PN junctions}
 
\hskip 10pt Questions: {\it 1 through 20}
 
\hskip 10pt Lab Exercise: {\it Rectifier diode characteristics (question 91)}
 
%INSTRUCTOR \hskip 10pt {\bf Explain suggested project ideas to students}

%INSTRUCTOR \hskip 10pt {\bf Give project grading rubric to students, complete with deadlines}

%INSTRUCTOR \hskip 10pt {\bf Socratic Electronics animation: Semiconductor diode junction}

%INSTRUCTOR \hskip 10pt {\bf Demo: Set up simple curve tracer circuit and show diode curve}

\vskip 10pt
%%%%%%%%%%%%%%%
\hrule \vskip 5pt
\noindent
\underbar{Day 2}

\hskip 10pt Topics: {\it Diodes and rectifier circuits}
 
\hskip 10pt Questions: {\it 21 through 40}
 
\hskip 10pt Lab Exercise: {\it Full-wave, center-tap rectifier circuit (question 92)}
 
%INSTRUCTOR \hskip 10pt {\bf MIT 6.002 video clip: Disk 1, Lecture 1; V/I component plots 29:20 to 35:12}

%INSTRUCTOR \hskip 10pt {\bf MIT 6.002 video clip: Disk 1, Lecture 7; animated diode V/I plot 10:03 to 12:12}

%INSTRUCTOR \hskip 10pt {\bf MIT 6.002 video clip: Disk 1, Lecture 7; real diode V/I plot 12:27 to 13:53}

\vskip 10pt
%%%%%%%%%%%%%%%
\hrule \vskip 5pt
\noindent
\underbar{Day 3}

\hskip 10pt Topics: {\it AC-DC power supply circuits and troubleshooting}
 
\hskip 10pt Questions: {\it 41 through 60}
 
\hskip 10pt Lab Exercise: {\it Full-wave bridge rectifier circuit (question 93)}
 
%INSTRUCTOR \hskip 10pt {\bf Socratic Electronics animation: Bridge rectifier circuit}

\vskip 10pt
%%%%%%%%%%%%%%%
\hrule \vskip 5pt
\noindent
\underbar{Day 4}

\hskip 10pt Topics: {\it Special diodes and zener voltage regulators}
 
\hskip 10pt Questions: {\it 61 through 80}
 
\hskip 10pt Lab Exercise: {\it Zener diode voltage regulator circuit (question 94)}
 
\vskip 10pt
%%%%%%%%%%%%%%%
\hrule \vskip 5pt
\noindent
\underbar{Day 5}

\hskip 10pt Topics: {\it Electron versus Conventional flow notation}
 
\hskip 10pt Questions: {\it 81 through 90}
 
\hskip 10pt Lab Exercise: {\it LED current limiting (question 95)}
 
\vskip 10pt
%%%%%%%%%%%%%%%
\hrule \vskip 5pt
\noindent
\underbar{Day 6}

\hskip 10pt Exam 1: {\it includes rectifier circuit performance assessment}
 
\hskip 10pt Project selection: {\it Initial project design checked by instructor and components selected (Dual output AC-DC power supply \underbar{strongly} recommended)}
  
\hskip 10pt Lab Exercise: {\it Work on project}

\vskip 10pt
%%%%%%%%%%%%%%%%
\hrule \vskip 5pt
\noindent
\underbar{Troubleshooting practice problems}

\hskip 10pt Questions: {\it 97 through 106}
 
\vskip 10pt
%%%%%%%%%%%%%%%
\hrule \vskip 5pt
\noindent
\underbar{General concept practice and challenge problems}

\hskip 10pt Questions: {\it 107 through the end of the worksheet}
 
\vskip 10pt
%%%%%%%%%%%%%%%
\hrule \vskip 5pt
\noindent
\underbar{Impending deadlines}

\hskip 10pt {\bf Project due at end of ELTR120, Section 3}
 
\hskip 10pt Question 96: Sample project grading criteria
 
\vskip 10pt
%%%%%%%%%%%%%%%











\vfil \eject

\centerline{\bf ELTR 120 (Semiconductors 1), section 2} \bigskip 
 
\vskip 10pt

\noindent
{\bf Recommended schedule}

\vskip 5pt

%%%%%%%%%%%%%%%
\hrule \vskip 5pt
\noindent
\underbar{Day 1}

\hskip 10pt Topics: {\it Bipolar junction transistor theory}
 
\hskip 10pt Questions: {\it 1 through 15}
 
\hskip 10pt Lab Exercise: {\it BJT terminal identification (question 76)}
 
%INSTRUCTOR \hskip 10pt {\bf Demo: show that $I_C$ is nearly independent of $V_{CE}$ for a BJT}

\vskip 10pt
%%%%%%%%%%%%%%%
\hrule \vskip 5pt
\noindent
\underbar{Day 2}

\hskip 10pt Topics: {\it Bipolar junction transistor switching circuits}
 
\hskip 10pt Questions: {\it 16 through 30}
 
\hskip 10pt Lab Exercise: {\it BJT switch circuit (question 77)}
 
\vskip 10pt
%%%%%%%%%%%%%%%
\hrule \vskip 5pt
\noindent
\underbar{Day 3}

\hskip 10pt Topics: {\it Junction field-effect transistor (JFET) theory}
 
\hskip 10pt Questions: {\it 31 through 45}
 
\hskip 10pt Lab Exercise: {\it JFET switch circuit (question 78)}
 
\vskip 10pt
%%%%%%%%%%%%%%%
\hrule \vskip 5pt
\noindent
\underbar{Day 4}

\hskip 10pt Topics: {\it Insulated gate field-effect transistor (MOSFET) theory}
 
\hskip 10pt Questions: {\it 46 through 60}
 
\hskip 10pt Lab Exercise: {\it Work on project}
 
%INSTRUCTOR \hskip 10pt {\bf MIT 6.002 video clip: Disk 1, Lecture 5; MOSFET V/I characteristic 44:33 to 45:41}

\vskip 10pt
%%%%%%%%%%%%%%%
\hrule \vskip 5pt
\noindent
\underbar{Day 5}

\hskip 10pt Topics: {\it Review}
 
\hskip 10pt Questions: {\it 61 through 75}
 
\hskip 10pt Lab Exercise: {\it Work on project}
 
\vskip 10pt
%%%%%%%%%%%%%%%
\hrule \vskip 5pt
\noindent
\underbar{Day 6}

\hskip 10pt Exam 2: {\it includes transistor switch circuit performance assessment}
 
\hskip 10pt Lab Exercise: {\it Work on project}

\vskip 10pt
%%%%%%%%%%%%%%%
\hrule \vskip 5pt
\noindent
\underbar{Troubleshooting practice problems}

\hskip 10pt Questions: {\it 80 through 89}
 
\vskip 10pt
%%%%%%%%%%%%%%%
\hrule \vskip 5pt
\noindent
\underbar{General concept practice and challenge problems}

\hskip 10pt Questions: {\it 90 through the end of the worksheet}
 
\vskip 10pt
%%%%%%%%%%%%%%%
\hrule \vskip 5pt
\noindent
\underbar{Impending deadlines}

\hskip 10pt {\bf Project due at end of ELTR120, Section 3}
 
\hskip 10pt Question 79: Sample project grading criteria
 
\vskip 10pt
%%%%%%%%%%%%%%%












\vfil \eject

\centerline{\bf ELTR 120 (Semiconductors 1), section 3} \bigskip 
 
\vskip 10pt

\noindent
{\bf Recommended schedule}

\vskip 5pt

%%%%%%%%%%%%%%%
\hrule \vskip 5pt
\noindent
\underbar{Day 1}

\hskip 10pt Topics: {\it Clipper, clamper, and voltage multiplier circuits}
 
\hskip 10pt Questions: {\it 1 through 10}
 
\hskip 10pt Lab Exercise: {\it Diode clipper circuit (question 51)}
 
\vskip 10pt
%%%%%%%%%%%%%%%
\hrule \vskip 5pt
\noindent
\underbar{Day 2}

\hskip 10pt Topics: {\it Thyristor devices}
 
\hskip 10pt Questions: {\it 11 through 20}
 
\hskip 10pt Lab Exercise: {\it Work on project}
 
\vskip 10pt
%%%%%%%%%%%%%%%
\hrule \vskip 5pt
\noindent
\underbar{Day 3}

\hskip 10pt Topics: {\it Thyristor power control circuits}
 
\hskip 10pt Questions: {\it 21 through 30}
 
\hskip 10pt Lab Exercise: {\it SCR latch circuit (question 52)}
 
%INSTRUCTOR \hskip 10pt {\bf Demo: show students what a solid-state relay looks like}

\vskip 10pt
%%%%%%%%%%%%%%%
\hrule \vskip 5pt
\noindent
\underbar{Day 4}

\hskip 10pt Topics: {\it Pulse-width modulation power control}
 
\hskip 10pt Questions: {\it 31 through 40}
 
\hskip 10pt Lab Exercise: {\it PWM power controller, discrete (question 53)}
 
\vskip 10pt
%%%%%%%%%%%%%%%
\hrule \vskip 5pt
\noindent
\underbar{Day 5}

\hskip 10pt Topics: {\it Switching power supply circuits}
 
\hskip 10pt Questions: {\it 41 through 50}
 
\hskip 10pt Lab Exercise: {\it Work on project}
 
%INSTRUCTOR \hskip 10pt {\bf Demo: show students a switching regulator IC datasheet}

%INSTRUCTOR \hskip 10pt {\bf MIT 8.02 video clip: Disk 4, Lecture 24; Ruhmkorff coil as a DC converter 47:45 to end}

\vskip 10pt
%%%%%%%%%%%%%%%
\hrule \vskip 5pt
\noindent
\underbar{Day 6}

\hskip 10pt Exam 3: {\it includes thyristor latch circuit performance assessment}
 
\hskip 10pt {\bf Project due}

\hskip 10pt Question 54: Sample project grading criteria
 
\vskip 10pt

%%%%%%%%%%%%%%%
\hrule \vskip 5pt
\noindent
\underbar{Troubleshooting practice problems}

\hskip 10pt Questions: {\it 55 through 64}
 
\vskip 10pt
%%%%%%%%%%%%%%%
\hrule \vskip 5pt
\noindent
\underbar{General concept practice and challenge problems}

\hskip 10pt Questions: {\it 65 through the end of the worksheet}
 
\vskip 10pt
%%%%%%%%%%%%%%%












\vfil \eject

\centerline{\bf ELTR 125 (Semiconductors 2), section 1} \bigskip 
 
\vskip 10pt

\noindent
{\bf Recommended schedule}

\vskip 5pt

%%%%%%%%%%%%%%%
\hrule \vskip 5pt
\noindent
\underbar{Day 1}

\hskip 10pt Topics: {\it The BJT as a linear amplifier, current mirrors}
 
\hskip 10pt Questions: {\it 1 through 15}
 
\hskip 10pt Lab Exercise: {\it Current mirror (question 76)}
 
%INSTRUCTOR \hskip 10pt {\bf Socratic Electronics animation: BJT characteristic curve sketching}

%INSTRUCTOR \hskip 10pt {\bf Demo: Show current mirror circuit regulating current through a rheostat}

\vskip 10pt
%%%%%%%%%%%%%%%
\hrule \vskip 5pt
\noindent
\underbar{Day 2}

\hskip 10pt Topics: {\it Common-collector BJT amplifiers, transistor amplifier biasing}
 
\hskip 10pt Questions: {\it 16 through 30}
 
\hskip 10pt Lab Exercise: {\it Signal biasing/unbiasing network (question 77)}
 
%INSTRUCTOR \hskip 10pt {\bf MIT 6.002 video clip: Disk 1, Lecture 7; Diode V/I curve w/ bias 21:59 to 25:40}
 
%INSTRUCTOR \hskip 10pt {\bf Demo: Show CC amplifier operation w/ and w/o biasing}

\vskip 10pt
%%%%%%%%%%%%%%%
\hrule \vskip 5pt
\noindent
\underbar{Day 3}

\hskip 10pt Topics: {\it Common-emitter BJT amplifiers}
 
\hskip 10pt Questions: {\it 31 through 45}
 
\hskip 10pt Lab Exercise: {\it Common-collector amplifier circuit (question 78)}
 
%INSTRUCTOR \hskip 10pt {\bf Demo: Show CE amplifier operation w/ and w/o biasing}

\vskip 10pt
%%%%%%%%%%%%%%%
\hrule \vskip 5pt
\noindent
\underbar{Day 4}

\hskip 10pt Topics: {\it Common-base BJT amplifiers, gain expressed in decibels}
 
\hskip 10pt Questions: {\it 46 through 60}
 
\hskip 10pt Lab Exercise: {\it Common-emitter amplifier circuit (question 79)}
 
%INSTRUCTOR \hskip 10pt {\bf Demo: Show Fluke 189 multimeter's decibel measurement function}

\vskip 10pt
%%%%%%%%%%%%%%%
\hrule \vskip 5pt
\noindent
\underbar{Day 5}

\hskip 10pt Topics: {\it Input and output impedances of amplifier circuits}
 
\hskip 10pt Questions: {\it 61 through 75}
 
\hskip 10pt Lab Exercise: {\it Common-base amplifier circuit (question 80)}
 
\vskip 10pt
%%%%%%%%%%%%%%%
\hrule \vskip 5pt
\noindent
\underbar{Day 6}

\hskip 10pt Exam 1: {\it includes Amplifier with specified voltage gain performance assessment}
 
\hskip 10pt Lab Exercise: {\it Troubleshooting practice (oscillator/amplifier circuit -- question 81)}
  
\vskip 10pt
%%%%%%%%%%%%%%%
\hrule \vskip 5pt
\noindent
\underbar{Troubleshooting practice problems}

\hskip 10pt Questions: {\it 84 through 93}
 
\vskip 10pt
%%%%%%%%%%%%%%%
\hrule \vskip 5pt
\noindent
\underbar{General concept practice and challenge problems}

\hskip 10pt Questions: {\it 94 through the end of the worksheet}
 
\vskip 10pt
%%%%%%%%%%%%%%%
\hrule \vskip 5pt
\noindent
\underbar{Impending deadlines}

\hskip 10pt {\bf Troubleshooting assessment (oscillator/amplifier) due at end of ELTR125, Section 3}

\hskip 10pt Question 82: Troubleshooting log
 
\hskip 10pt Question 83: Sample troubleshooting assessment grading criteria
 
\vskip 10pt
%%%%%%%%%%%%%%%










\vfil \eject

\centerline{\bf ELTR 125 (Semiconductors 2), section 2} \bigskip 
 
\vskip 10pt

\noindent
{\bf Recommended schedule}

\vskip 5pt

%%%%%%%%%%%%%%%
\hrule \vskip 5pt
\noindent
\underbar{Day 1}

\hskip 10pt Topics: {\it Load lines and amplifier bias calculations}
 
\hskip 10pt Questions: {\it 1 through 15}
 
\hskip 10pt Lab Exercise: {\it Common-drain amplifier circuit (question 61)}
 
\vskip 10pt
%%%%%%%%%%%%%%%
\hrule \vskip 5pt
\noindent
\underbar{Day 2}

\hskip 10pt Topics: {\it FET amplifier configurations}
 
\hskip 10pt Questions: {\it 16 through 25}
 
\hskip 10pt Lab Exercise: {\it Common-source amplifier circuit (question 62)}
 
 
\vskip 10pt
%%%%%%%%%%%%%%%
\hrule \vskip 5pt
\noindent
\underbar{Day 3}

\hskip 10pt Topics: {\it Push-pull amplifier circuits}
 
\hskip 10pt Questions: {\it 26 through 35}
 
\hskip 10pt Lab Exercise: {\it Audio intercom circuit, push-pull output (question 63)}
 
%INSTRUCTOR \hskip 10pt {\bf Socratic Electronics animation: Push-pull transistor amplifier with crossover distortion}

\vskip 10pt
%%%%%%%%%%%%%%%
\hrule \vskip 5pt
\noindent
\underbar{Day 4}

\hskip 10pt Topics: {\it Multi-stage and high-frequency amplifier designs}
 
\hskip 10pt Questions: {\it 36 through 50}
 
\hskip 10pt Lab Exercise: {\it Audio intercom circuit, push-pull output (question 63, continued)}
 
\vskip 10pt
%%%%%%%%%%%%%%%
\hrule \vskip 5pt
\noindent
\underbar{Day 5}

\hskip 10pt Topics: {\it Amplifier troubleshooting}
 
\hskip 10pt Questions: {\it 51 through 60}
 
\hskip 10pt Lab Exercise: {\it Troubleshooting practice (oscillator/amplifier circuit -- question 64)}

\vskip 10pt
%%%%%%%%%%%%%%%
\hrule \vskip 5pt
\noindent
\underbar{Day 6}

\hskip 10pt Exam 2: {\it includes Amplifier circuit performance assessment}
 
\hskip 10pt Lab Exercise: {\it Troubleshooting practice (oscillator/amplifier circuit -- question 64)}
  
\vskip 10pt
%%%%%%%%%%%%%%%
\hrule \vskip 5pt
\noindent
\underbar{General concept practice and challenge problems}

\hskip 10pt Questions: {\it 67 through the end of the worksheet}
 
\vskip 10pt

%%%%%%%%%%%%%%%
\hrule \vskip 5pt
\noindent
\underbar{Impending deadlines}

\hskip 10pt {\bf Troubleshooting assessment (oscillator/amplifier) due at end of ELTR125, Section 3}

\hskip 10pt Question 65: Troubleshooting log
 
\hskip 10pt Question 66: Sample troubleshooting assessment grading criteria
 
\vskip 10pt
%%%%%%%%%%%%%%%










\vfil \eject

\centerline{\bf ELTR 125 (Semiconductors 2), section 3} \bigskip 
 
\vskip 10pt

\noindent
{\bf Recommended schedule}

\vskip 5pt

%%%%%%%%%%%%%%%
\hrule \vskip 5pt
\noindent
\underbar{Day 1}

\hskip 10pt Topics: {\it Basic oscillator theory and relaxation oscillator circuits}
 
\hskip 10pt Questions: {\it 1 through 10}
 
\hskip 10pt Lab Exercise: {\it BJT multivibrator circuit, astable (question 51)}
 
\vskip 10pt
%%%%%%%%%%%%%%%
\hrule \vskip 5pt
\noindent
\underbar{Day 2}

\hskip 10pt Topics: {\it Phase-shift and resonant oscillator circuits}
 
\hskip 10pt Questions: {\it 11 through 20}
 
\hskip 10pt Lab Exercise: {\it Wien bridge oscillator, BJT (question 52)}
 
 
\vskip 10pt
%%%%%%%%%%%%%%%
\hrule \vskip 5pt
\noindent
\underbar{Day 3}

\hskip 10pt Topics: {\it Harmonics}
 
\hskip 10pt Questions: {\it 21 through 30}
 
\hskip 10pt Lab Exercise: {\it Colpitts oscillator, BJT (question 53)}
 
%INSTRUCTOR \hskip 10pt {\bf Demo: Use graphing calculator to synthesize square wave from sinusoidal harmonics}

%INSTRUCTOR \hskip 10pt {\bf Demo: Show harmonics using a spectrum analyzer and function generator}

%INSTRUCTOR \hskip 10pt {\bf Demo: Show harmonics in power-line signal using a spectrum analyzer and transformer}

%INSTRUCTOR \hskip 10pt {\bf Demo: Show example of spectrum plot from an amplifier datasheet}

\vskip 10pt
%%%%%%%%%%%%%%%
\hrule \vskip 5pt
\noindent
\underbar{Day 4}

\hskip 10pt Topics: {\it Fundamentals of radio, amplitude modulation, and frequency modulation} (optional)
 
\hskip 10pt Questions: {\it 31 through 50}
 
\hskip 10pt Lab Exercise: {\it Troubleshooting practice (oscillator/amplifier circuit -- question 55)}

\hskip 10pt Just for fun (not required): {\it AM radio transmitter (question 54)}
 
%INSTRUCTOR \hskip 10pt {\bf Demo: Use signal generator with AM function to broadcast audio tone to radio}

\vskip 10pt
%%%%%%%%%%%%%%%
\hrule \vskip 5pt
\noindent
\underbar{Day 5}

\hskip 10pt Exam 3: {\it includes Oscillator Circuit performance assessment}
 
\hskip 10pt {\bf Troubleshooting Assessment due:} {\it oscillator/amplifier circuit (question 55)}
 
\hskip 10pt Question 56: Troubleshooting log
 
\hskip 10pt Question 57: Sample troubleshooting assessment grading criteria
 
\vskip 10pt
%%%%%%%%%%%%%%%%
\hrule \vskip 5pt
\noindent
\underbar{Troubleshooting practice problems}

\hskip 10pt Questions: {\it 58 through 67}
 
\vskip 10pt
%%%%%%%%%%%%%%%
\hrule \vskip 5pt
\noindent
\underbar{General concept practice and challenge problems}

\hskip 10pt Questions: {\it 68 through the end of the worksheet}
 
\vskip 10pt
%%%%%%%%%%%%%%%





\vfil \eject

\centerline{\bf ELTR 130 (Operational Amplifiers 1), section 1} \bigskip 
 
\vskip 10pt

\noindent
{\bf Recommended schedule}

\vskip 5pt

%%%%%%%%%%%%%%%
\hrule \vskip 5pt
\noindent
\underbar{Day 1}

\hskip 10pt Topics: {\it Differential pair circuits}
 
\hskip 10pt Questions: {\it 1 through 15}
 
\hskip 10pt Lab Exercise: {\it Discrete differential amplifier (question 56)}
 
%INSTRUCTOR \hskip 10pt {\bf Explain suggested project ideas to students}

%INSTRUCTOR \hskip 10pt {\bf Give project grading rubric to students, complete with deadlines}

\vskip 10pt
%%%%%%%%%%%%%%%
\hrule \vskip 5pt
\noindent
\underbar{Day 2}

\hskip 10pt Topics: {\it The basic operational amplifier}
 
\hskip 10pt Questions: {\it 16 through 25}
 
\hskip 10pt Lab Exercise: {\it Discrete differential amplifier (question 56, continued)}
 
\vskip 10pt
%%%%%%%%%%%%%%%
\hrule \vskip 5pt
\noindent
\underbar{Day 3}

\hskip 10pt Topics: {\it Using the operational amplifier as a comparator}
 
\hskip 10pt Questions: {\it 26 through 35}
 
\hskip 10pt Lab Exercise: {\it Comparator circuit (question 57)}
 
%INSTRUCTOR \hskip 10pt {\bf MIT 6.002 video clip: Disk 4, Lecture 21; Open-loop opamp 27:30 to 29:32}

\vskip 10pt
%%%%%%%%%%%%%%%
\hrule \vskip 5pt
\noindent
\underbar{Day 4}

\hskip 10pt Topics: {\it Using the operational amplifier as a voltage buffer}
 
\hskip 10pt Questions: {\it 36 through 45}
 
\hskip 10pt Lab Exercise: {\it Opamp voltage follower (question 58)}
 
%INSTRUCTOR \hskip 10pt {\bf MIT 6.002 video clip: Disk 4, Lecture 21; Open-loop opamp 41:30 to 43:10}

\vskip 10pt
%%%%%%%%%%%%%%%
\hrule \vskip 5pt
\noindent
\underbar{Day 5}

\hskip 10pt Topics: {\it Additional applications of feedback} (optional)
 
\hskip 10pt Questions: {\it 46 through 55}
 
\hskip 10pt Lab Exercise: {\it Linear voltage regulator circuit (question 59)}
 
%INSTRUCTOR \hskip 10pt {\bf Socratic Electronics animation: Push-pull transistor amplifier with crossover distortion}

\vskip 10pt
%%%%%%%%%%%%%%%
\hrule \vskip 5pt
\noindent
\underbar{Day 6}

\hskip 10pt Exam 1: {\it includes Comparator circuit performance assessment}
 
\hskip 10pt Lab Exercise: {\it Select an opamp project to prototype and troubleshoot by the end of the next course section (ELTR130, Section 2)}
  
\vskip 10pt
%%%%%%%%%%%%%%%
\hrule \vskip 5pt
\noindent
\underbar{Troubleshooting practice problems}

\hskip 10pt Questions: {\it 62 through 71}
 
\vskip 10pt
%%%%%%%%%%%%%%%
\hrule \vskip 5pt
\noindent
\underbar{General concept practice and challenge problems}

\hskip 10pt Questions: {\it 72 through the end of the worksheet}
 
\vskip 10pt
%%%%%%%%%%%%%%%
\hrule \vskip 5pt
\noindent
\underbar{Impending deadlines}

\hskip 10pt {\bf Troubleshooting assessment (project prototype) due at end of ELTR130, Section 2}

\hskip 10pt Question 60: Troubleshooting log
 
\hskip 10pt Question 61: Sample troubleshooting assessment grading criteria
 
\vskip 10pt
%%%%%%%%%%%%%%%




\vfil \eject

\centerline{\bf ELTR 130 (Operational Amplifiers 1), section 2} \bigskip 
 
\vskip 10pt

\noindent
{\bf Recommended schedule}

\vskip 5pt

%%%%%%%%%%%%%%%
\hrule \vskip 5pt
\noindent
\underbar{Day 1}

\hskip 10pt Topics: {\it Using the operational amplifier as a noninverting voltage amplifier}
 
\hskip 10pt Questions: {\it 1 through 15}
 
\hskip 10pt Lab Exercise: {\it Opamp as noninverting amplifier (question 61)}
 
%INSTRUCTOR \hskip 10pt {\bf MIT 6.002 video clip: Disk 4, Lecture 21; Noninverting opamp 41:30 to 43:10}

\vskip 10pt
%%%%%%%%%%%%%%%
\hrule \vskip 5pt
\noindent
\underbar{Day 2}

\hskip 10pt Topics: {\it Using the operational amplifier as an inverting voltage amplifier}
 
\hskip 10pt Questions: {\it 16 through 30}
 
\hskip 10pt Lab Exercise: {\it Opamp as inverting amplifier (question 62)}
 
 
\vskip 10pt
%%%%%%%%%%%%%%%
\hrule \vskip 5pt
\noindent
\underbar{Day 3}

\hskip 10pt Topics: {\it Voltage/current converter and summer circuits}
 
\hskip 10pt Questions: {\it 31 through 40}
 
\hskip 10pt Lab Exercise: {\it Troubleshooting practice on prototyped project}
 
\vskip 10pt
%%%%%%%%%%%%%%%
\hrule \vskip 5pt
\noindent
\underbar{Day 4}

\hskip 10pt Topics: {\it Differential and instrumentation amplifier circuits}
 
\hskip 10pt Questions: {\it 41 through 50}
 
\hskip 10pt Lab Exercise: {\it Op-amp as difference amplifier (question 63)}
 
\vskip 10pt
%%%%%%%%%%%%%%%
\hrule \vskip 5pt
\noindent
\underbar{Day 5}

\hskip 10pt Topics: {\it Precision rectifier circuits}
 
\hskip 10pt Questions: {\it 51 through 60}
 
\hskip 10pt Lab Exercise: {\it Precision half-wave rectifier (question 64)}
 
\vskip 10pt
%%%%%%%%%%%%%%%
\hrule \vskip 5pt
\noindent
\underbar{Day 6}

\hskip 10pt Topics: {\it Review}
 
\hskip 10pt Lab Exercise: {\it Troubleshooting practice on prototyped project}
 
\vskip 10pt
%%%%%%%%%%%%%%%
\hrule \vskip 5pt
\noindent
\underbar{Day 7}

\hskip 10pt Exam 2: {\it includes Inverting or Noninverting amplifier circuit performance assessment}
 
\hskip 10pt {\bf Troubleshooting Assessment due:} {\it Opamp project prototype}
 
\hskip 10pt Question 65: Troubleshooting log
 
\hskip 10pt Question 66: Sample troubleshooting assessment grading criteria
 
\vskip 10pt
%%%%%%%%%%%%%%%%
\hrule \vskip 5pt
\noindent
\underbar{Troubleshooting practice problems}

\hskip 10pt Questions: {\it 67 through 76}
 
\vskip 10pt
%%%%%%%%%%%%%%%
\hrule \vskip 5pt
\noindent
\underbar{General concept practice and challenge problems}

\hskip 10pt Questions: {\it 77 through the end of the worksheet}
 
\vskip 10pt
%%%%%%%%%%%%%%%











\vfil \eject

\centerline{\bf ELTR 135 (Operational Amplifiers 2), section 1} \bigskip 
 
\vskip 10pt

\noindent
{\bf Recommended schedule}

\vskip 5pt

%%%%%%%%%%%%%%%
\hrule \vskip 5pt
\noindent
\underbar{Day 1}

\hskip 10pt Topics: {\it Operational amplifier AC performance}
 
\hskip 10pt Questions: {\it 1 through 10}
 
\hskip 10pt Lab Exercise: {\it Opamp slew rate (question 56)}
 
%INSTRUCTOR \hskip 10pt {\bf Demo: Show slew rate, opamp turning high-frequency square wave into triangle wave}

%INSTRUCTOR \hskip 10pt {\bf Demo: Show GBW limiting, high-gain opamp circuit cutting off at high-frequencies}

\vskip 10pt
%%%%%%%%%%%%%%%
\hrule \vskip 5pt
\noindent
\underbar{Day 2}

\hskip 10pt Topics: {\it AC calculations and filter circuit review}
 
\hskip 10pt Questions: {\it 11 through 25}
 
\hskip 10pt Lab Exercise: {\it Opamp gain-bandwidth product (question 57)}
 
\vskip 10pt
%%%%%%%%%%%%%%%
\hrule \vskip 5pt
\noindent
\underbar{Day 3}

\hskip 10pt Topics: {\it Active filter circuits}
 
\hskip 10pt Questions: {\it 26 through 35}
 
\hskip 10pt Lab Exercise: {\it Sallen-Key active lowpass filter (question 58)}
 
\vskip 10pt
%%%%%%%%%%%%%%%
\hrule \vskip 5pt
\noindent
\underbar{Day 4}

\hskip 10pt Topics: {\it Active filter circuits (continued)}
 
\hskip 10pt Questions: {\it 36 through 45}
 
\hskip 10pt Lab Exercise: {\it Sallen-Key active highpass filter (question 59)}
 
\vskip 10pt
%%%%%%%%%%%%%%%
\hrule \vskip 5pt
\noindent
\underbar{Day 5}

\hskip 10pt Topics: {\it Switched-capacitor circuits (optional)}
 
\hskip 10pt Questions: {\it 46 through 55}
 
\hskip 10pt Lab Exercise: {\it Bandpass active filter (question 60)}
 
\vskip 10pt
%%%%%%%%%%%%%%%
\hrule \vskip 5pt
\noindent
\underbar{Day 6}

\hskip 10pt Exam 1: {\it includes Active filter circuit performance assessment}
  
\hskip 10pt Lab Exercise: {\it Work on project}
 
\vskip 10pt
%%%%%%%%%%%%%%%
\hrule \vskip 5pt
\noindent
\underbar{Troubleshooting practice problems}

\hskip 10pt Questions: {\it 62 through 71}
 
\vskip 10pt
%%%%%%%%%%%%%%%
\hrule \vskip 5pt
\noindent
\underbar{General concept practice and challenge problems}

\hskip 10pt Questions: {\it 72 through the end of the worksheet}
 
\vskip 10pt
%%%%%%%%%%%%%%%
\hrule \vskip 5pt
\noindent
\underbar{Impending deadlines}

\hskip 10pt {\bf Project due at end of ELTR135, Section 2}
 
\hskip 10pt Question 61: Sample project grading criteria
 
\vskip 10pt
%%%%%%%%%%%%%%%







\vfil \eject

\centerline{\bf ELTR 135 (Operational Amplifiers 2), section 2} \bigskip 
 
\vskip 10pt

\noindent
{\bf Recommended schedule}

\vskip 5pt

%%%%%%%%%%%%%%%
\hrule \vskip 5pt
\noindent
\underbar{Day 1}

\hskip 10pt Topics: {\it Operational amplifier oscillators}
 
\hskip 10pt Questions: {\it 1 through 10}
 
\hskip 10pt Lab Exercise: {\it Opamp relaxation oscillator (question 46)}
 
\vskip 10pt
%%%%%%%%%%%%%%%
\hrule \vskip 5pt
\noindent
\underbar{Day 2}

\hskip 10pt Topics: {\it Calculus explained through active integrator and differentiator circuits}
 
\hskip 10pt Questions: {\it 11 through 20}
 
\hskip 10pt Lab Exercise: {\it Opamp triangle wave generator (question 47)}
 
\vskip 10pt
%%%%%%%%%%%%%%%
\hrule \vskip 5pt
\noindent
\underbar{Day 3}

\hskip 10pt Topics: {\it Logarithm review}
 
\hskip 10pt Questions: {\it 21 through 35}
 
\hskip 10pt Lab Exercise: {\it Opamp LC resonant oscillator (question 48)}
 
\vskip 10pt
%%%%%%%%%%%%%%%
\hrule \vskip 5pt
\noindent
\underbar{Day 4}

\hskip 10pt Topics: {\it Log/antilog circuits} (optional)
 
\hskip 10pt Questions: {\it 36 through 45}
 
\hskip 10pt Lab Exercise: {\it Work on project}
 
\vskip 10pt
%%%%%%%%%%%%%%%
\hrule \vskip 5pt
\noindent
\underbar{Day 5}

\hskip 10pt Topics: {\it Review}
 
\hskip 10pt Lab Exercise: {\it Work on project}
 
%INSTRUCTOR \hskip 10pt {\bf Show picture(s) of analog computers}

\vskip 10pt
%%%%%%%%%%%%%%%
\hrule \vskip 5pt
\noindent
\underbar{Day 6}

\hskip 10pt Exam 2: {\it includes Oscillator circuit performance assessment}
 
\hskip 10pt {\bf Project due}

\hskip 10pt Question 49: Sample project grading criteria
 
\vskip 10pt
%%%%%%%%%%%%%%%
\hrule \vskip 5pt
\noindent
\underbar{Troubleshooting practice problems}

\hskip 10pt Questions: {\it 50 through 59}
 
\vskip 10pt
%%%%%%%%%%%%%%%
\hrule \vskip 5pt
\noindent
\underbar{General concept practice and challenge problems}

\hskip 10pt Questions: {\it 60 through the end of the worksheet}
 
\vskip 10pt
%%%%%%%%%%%%%%%








\vfil \eject

\centerline{\bf ELTR 140 (Digital 1), section 1} \bigskip 
 
\vskip 10pt

\noindent
{\bf Recommended schedule}

\vskip 5pt

%%%%%%%%%%%%%%%
\hrule \vskip 5pt
\noindent
\underbar{Day 1}

\hskip 10pt Topics: {\it Logic states and simple gate circuits}
 
\hskip 10pt Questions: {\it 1 through 10}
 
\hskip 10pt Lab Exercise: {\it OR gate, diode-resistor logic (question 51)}
 
%INSTRUCTOR \hskip 10pt {\bf Explain suggested project ideas to students}

%INSTRUCTOR \hskip 10pt {\bf Give project grading rubric to students, complete with deadlines}

%INSTRUCTOR \hskip 10pt {\bf MIT 6.002 video clip: Disk 1, Lecture 4; AND gate noise immunity 48:15 to end}

\vskip 10pt
%%%%%%%%%%%%%%%
\hrule \vskip 5pt
\noindent
\underbar{Day 2}

\hskip 10pt Topics: {\it TTL logic gates and truth tables}
 
\hskip 10pt Questions: {\it 11 through 20}
 
\hskip 10pt Lab Exercise: {\it AND gate, simple BJT logic (question 52)}
 
\vskip 10pt
%%%%%%%%%%%%%%%
\hrule \vskip 5pt
\noindent
\underbar{Day 3}

\hskip 10pt Topics: {\it CMOS logic gates and truth tables}
 
\hskip 10pt Questions: {\it 21 through 30}
 
\hskip 10pt Lab Exercise: {\it IC logic gate usage (question 53)}
 
\vskip 10pt
%%%%%%%%%%%%%%%
\hrule \vskip 5pt
\noindent
\underbar{Day 4}

\hskip 10pt Topics: {\it Relay circuits and truth tables}
 
\hskip 10pt Questions: {\it 31 through 40}
 
\hskip 10pt Lab Exercise: {\it AND gate, relay logic (question 54)}
 
%INSTRUCTOR \hskip 10pt {\bf Demo: show photos of old Mark I electromechanical relay computer}

\vskip 10pt
%%%%%%%%%%%%%%%
\hrule \vskip 5pt
\noindent
\underbar{Day 5}

\hskip 10pt Topics: {\it Logic circuit performance parameters}
 
\hskip 10pt Questions: {\it 41 through 50}
 
\hskip 10pt Lab Exercise: {\it Gate-relay interposing (question 55)}
 
%INSTRUCTOR \hskip 10pt {\bf MIT 6.002 video clip: Disk 4, Lecture 24; CMOS power consumption 48:00 to 50:05}

%INSTRUCTOR \hskip 10pt {\bf MIT 6.002 video clip: Disk 4, Lecture 25; CMOS temp. vs freq. 39:03 to 40:01}

\vskip 10pt
%%%%%%%%%%%%%%%
\hrule \vskip 5pt
\noindent
\underbar{Day 6}

\hskip 10pt Exam 1: {\it includes IC logic gate performance assessment}
 
\vskip 10pt
%%%%%%%%%%%%%%%
\hrule \vskip 5pt
\noindent
\underbar{Troubleshooting practice problems}

\hskip 10pt Questions: {\it 57 through 66}
 
\vskip 10pt
%%%%%%%%%%%%%%%
\hrule \vskip 5pt
\noindent
\underbar{DC/AC review problems}

\hskip 10pt Questions: {\it 67 through 86}
 
\vskip 10pt
%%%%%%%%%%%%%%%
\hrule \vskip 5pt
\noindent
\underbar{Basic principles of microcontrollers}

\hskip 10pt Questions: {\it 87 through 96}
 
\vskip 10pt
%%%%%%%%%%%%%%%
\hrule \vskip 5pt
\noindent
\underbar{General concept practice and challenge problems}

\hskip 10pt Questions: {\it 97 through the end of the worksheet}
 
\vskip 10pt
%%%%%%%%%%%%%%%
\hrule \vskip 5pt
\noindent
\underbar{Impending deadlines}

\hskip 10pt {\bf Project due at end of ELTR140, Section 3}
 
\hskip 10pt Question 56: Sample project grading criteria
 
\vskip 10pt
%%%%%%%%%%%%%%%









\vfil \eject

\centerline{\bf ELTR 140 (Digital 1), section 2} \bigskip 
 
\vskip 10pt

\noindent
{\bf Recommended schedule}

\vskip 5pt

%%%%%%%%%%%%%%%
\hrule \vskip 5pt
\noindent
\underbar{Day 1}

\hskip 10pt Topics: {\it Boolean algebra, basic concepts and identities}
 
\hskip 10pt Questions: {\it 1 through 20}
 
\hskip 10pt Lab Exercise: {\it work on project}
 
\vskip 10pt
%%%%%%%%%%%%%%%
\hrule \vskip 5pt
\noindent
\underbar{Day 2}

\hskip 10pt Topics: {\it Boolean algebra, simplification laws}
 
\hskip 10pt Questions: {\it 21 through 40}
 
\hskip 10pt Lab Exercise: {\it Gate circuit from Boolean expression (question 96)}
 
\vskip 10pt
%%%%%%%%%%%%%%%
\hrule \vskip 5pt
\noindent
\underbar{Day 3}

\hskip 10pt Topics: {\it SOP and POS expressions}
 
\hskip 10pt Questions: {\it 41 through 60}
 
\hskip 10pt Lab Exercise: {\it Gate circuit from truth table (question 97)}
 
\vskip 10pt
%%%%%%%%%%%%%%%
\hrule \vskip 5pt
\noindent
\underbar{Day 4}

\hskip 10pt Topics: {\it Karnaugh mapping}
 
\hskip 10pt Questions: {\it 61 through 75}
 
\hskip 10pt Lab Exercise: {\it work on project}
 
\vskip 10pt
%%%%%%%%%%%%%%%
\hrule \vskip 5pt
\noindent
\underbar{Day 5}

\hskip 10pt Topics: {\it DeMorgan's Theorem and gate universality}
 
\hskip 10pt Questions: {\it 76 through 95}
 
\hskip 10pt Lab Exercise: {\it NAND gate universality (question 98)}
 
\vskip 10pt
%%%%%%%%%%%%%%%
\hrule \vskip 5pt
\noindent
\underbar{Day 6}

\hskip 10pt Exam 2: {\it includes Boolean-to-gate performance assessment}
 
\vskip 10pt
%%%%%%%%%%%%%%%
\hrule \vskip 5pt
\noindent
\underbar{Troubleshooting practice problems}

\hskip 10pt Questions: {\it 100 through 109}
 
\vskip 10pt
%%%%%%%%%%%%%%%
\hrule \vskip 5pt
\noindent
\underbar{General concept practice and challenge problems}

\hskip 10pt Questions: {\it 110 through the end of the worksheet}
 
\vskip 10pt
%%%%%%%%%%%%%%%
\hrule \vskip 5pt
\noindent
\underbar{Impending deadlines}

\hskip 10pt {\bf Project due at end of ELTR140, Section 3}
 
\hskip 10pt Question 99: Sample project grading criteria
 
\vskip 10pt
%%%%%%%%%%%%%%%









\vfil \eject

\centerline{\bf ELTR 140 (Digital 1), section 3} \bigskip 
 
\vskip 10pt

\noindent
{\bf Recommended schedule}

\vskip 5pt

%%%%%%%%%%%%%%%
\hrule \vskip 5pt
\noindent
\underbar{Day 1}

\hskip 10pt Topics: {\it Numeration systems}
 
\hskip 10pt Questions: {\it 1 through 10}
 
\hskip 10pt Lab Exercise: {\it work on project}
 
\vskip 10pt
%%%%%%%%%%%%%%%
\hrule \vskip 5pt
\noindent
\underbar{Day 2}

\hskip 10pt Topics: {\it Digital codes}
 
\hskip 10pt Questions: {\it 11 through 20}
 
\hskip 10pt Lab Exercise: {\it Gray code to binary converter (question 56)}
 
\vskip 10pt
%%%%%%%%%%%%%%%
\hrule \vskip 5pt
\noindent
\underbar{Day 3}

\hskip 10pt Topics: {\it Binary arithmetic}
 
\hskip 10pt Questions: {\it 21 through 30}
 
\hskip 10pt Lab Exercise: {\it Half adder circuit (question 57)}
 
\vskip 10pt
%%%%%%%%%%%%%%%
\hrule \vskip 5pt
\noindent
\underbar{Day 4}

\hskip 10pt Topics: {\it Binary arithmetic circuits}
 
\hskip 10pt Questions: {\it 31 through 40}
 
\hskip 10pt Lab Exercise: {\it Full adder circuit (question 58)}
 
\vskip 10pt
%%%%%%%%%%%%%%%
\hrule \vskip 5pt
\noindent
\underbar{Day 5}

\hskip 10pt Topics: {\it Digital circuit troubleshooting}
 
\hskip 10pt Questions: {\it 41 through 55}
 
\hskip 10pt Lab Exercise: {\it Analog-digital converter IC (question 59)}
 
\vskip 10pt
%%%%%%%%%%%%%%%
\hrule \vskip 5pt
\noindent
\underbar{Day 6}

\hskip 10pt Exam 3: {\it includes binary adder circuit performance assessment}
 
\hskip 10pt {\bf Project due}

\hskip 10pt Question 60: Sample project grading criteria
 
\vskip 10pt
%%%%%%%%%%%%%%%
\hrule \vskip 5pt
\noindent
\underbar{DC/AC review problems}

\hskip 10pt Questions: {\it 61 through 80}
 
\vskip 10pt
%%%%%%%%%%%%%%%
\hrule \vskip 5pt
\noindent
\underbar{General concept practice and challenge problems}

\hskip 10pt Questions: {\it 81 through the end of the worksheet}
 
%%%%%%%%%%%%%%%






\vfil \eject

\centerline{\bf ELTR 145 (Digital 2), section 1} \bigskip 
 
\vskip 10pt

\noindent
{\bf Recommended schedule}

\vskip 5pt

%%%%%%%%%%%%%%%
\hrule \vskip 5pt
\noindent
\underbar{Day 1}

\hskip 10pt Topics: {\it Latch circuits}
 
\hskip 10pt Questions: {\it 1 through 10}
 
\hskip 10pt Lab Exercise: {\it S-R latch from individual gates (question 51)}
 
%INSTRUCTOR \hskip 10pt {\bf Demo: show switch bounce using a digital oscilloscope}

\vskip 10pt
%%%%%%%%%%%%%%%
\hrule \vskip 5pt
\noindent
\underbar{Day 2}

\hskip 10pt Topics: {\it 555 timer circuit}
 
\hskip 10pt Questions: {\it 11 through 20}
 
\hskip 10pt Lab Exercise: {\it 555 timer in astable mode (question 52)}
 
\vskip 10pt
%%%%%%%%%%%%%%%
\hrule \vskip 5pt
\noindent
\underbar{Day 3}

\hskip 10pt Topics: {\it Gated latch circuits}
 
\hskip 10pt Questions: {\it 21 through 30}
 
\hskip 10pt Lab Exercise: {\it Troubleshooting practice (decade counter circuit -- question 54)}
 
\vskip 10pt
%%%%%%%%%%%%%%%
\hrule \vskip 5pt
\noindent
\underbar{Day 4}

\hskip 10pt Topics: {\it Flip-flops}
 
\hskip 10pt Questions: {\it 31 through 40}
 
\hskip 10pt Lab Exercise: {\it Troubleshooting practice (decade counter circuit -- question 54)}
 
\vskip 10pt
%%%%%%%%%%%%%%%
\hrule \vskip 5pt
\noindent
\underbar{Day 5}

\hskip 10pt Topics: {\it Flip-flops (continued)}
 
\hskip 10pt Questions: {\it 41 through 50}
 
\hskip 10pt Lab Exercise: {\it J-K flip-flop IC (question 53)}
 
\vskip 10pt
%%%%%%%%%%%%%%%
\hrule \vskip 5pt
\noindent
\underbar{Day 6}

\hskip 10pt Exam 1: {\it includes S-R latch circuit performance assessment}
 
\hskip 10pt Lab Exercise: {\it Troubleshooting practice (decade counter circuit -- question 54)}
 
\vskip 10pt
%%%%%%%%%%%%%%%
\hrule \vskip 5pt
\noindent
\underbar{Troubleshooting practice problems}

\hskip 10pt Questions: {\it 57 through 66}
 
\vskip 10pt
%%%%%%%%%%%%%%%
\hrule \vskip 5pt
\noindent
\underbar{DC/AC/Semiconductor/Opamp review problems}

\hskip 10pt Questions: {\it 67 through 86}
 
\vskip 10pt
%%%%%%%%%%%%%%%
\hrule \vskip 5pt
\noindent
\underbar{General concept practice and challenge problems}

\hskip 10pt Questions: {\it 87 through the end of the worksheet}
 
\vskip 10pt
%%%%%%%%%%%%%%%
\hrule \vskip 5pt
\noindent
\underbar{Impending deadlines}

\hskip 10pt {\bf Troubleshooting assessment (counter circuit) due at end of ELTR145, Section 3}

\hskip 10pt Question 55: Troubleshooting log
 
\hskip 10pt Question 56: Sample troubleshooting assessment grading criteria
 
\vskip 10pt
%%%%%%%%%%%%%%%








\vfil \eject

\centerline{\bf ELTR 145 (Digital 2), section 2} \bigskip 
 
\vskip 10pt

\noindent
{\bf Recommended schedule}

\vskip 5pt

%%%%%%%%%%%%%%%
\hrule \vskip 5pt
\noindent
\underbar{Day 1}

\hskip 10pt Topics: {\it Counter circuits}
 
\hskip 10pt Questions: {\it 1 through 10}
 
\hskip 10pt Lab Exercise: {\it 2-bit counter from flip-flops (question 56)}
 
\vskip 10pt
%%%%%%%%%%%%%%%
\hrule \vskip 5pt
\noindent
\underbar{Day 2}

\hskip 10pt Topics: {\it Counter circuits (continued)}
 
\hskip 10pt Questions: {\it 11 through 20}
 
\hskip 10pt Lab Exercise: {\it 4-bit up/down counter IC (question 57)}
 
\vskip 10pt
%%%%%%%%%%%%%%%
\hrule \vskip 5pt
\noindent
\underbar{Day 3}

\hskip 10pt Topics: {\it Shift registers}
 
\hskip 10pt Questions: {\it 21 through 30}
 
\hskip 10pt Lab Exercise: {\it Troubleshooting practice (decade counter circuit -- question 60)}
 
\vskip 10pt
%%%%%%%%%%%%%%%
\hrule \vskip 5pt
\noindent
\underbar{Day 4}

\hskip 10pt Topics: {\it Shift registers and serial data communication}
 
\hskip 10pt Questions: {\it 31 through 40}
 
\hskip 10pt Lab Exercise: {\it Frequency divider circuit (question 58)}
 
\vskip 10pt
%%%%%%%%%%%%%%%
\hrule \vskip 5pt
\noindent
\underbar{Day 5}

\hskip 10pt Topics: {\it Memory technologies}
 
\hskip 10pt Questions: {\it 41 through 55}
 
\hskip 10pt Lab Exercise: {\it 4-bit universal shift register IC (question 59)}
 
%INSTRUCTOR \hskip 10pt {\bf Socratic Electronics animation: ROM memory addressing}

\vskip 10pt
%%%%%%%%%%%%%%%
\hrule \vskip 5pt
\noindent
\underbar{Day 6}

\hskip 10pt Exam 2: {\it includes Counter circuit performance assessment}
 
\hskip 10pt Lab Exercise: {\it Troubleshooting practice (decade counter circuit -- question 60)}
  
\vskip 10pt
%%%%%%%%%%%%%%%
\hrule \vskip 5pt
\noindent
\underbar{Troubleshooting practice problems}

\hskip 10pt Questions: {\it 63 through 72}
 
\vskip 10pt
%%%%%%%%%%%%%%%
\hrule \vskip 5pt
\noindent
\underbar{DC/AC/Semiconductor/Opamp review problems}

\hskip 10pt Questions: {\it 73 through 92}
 
\vskip 10pt

%%%%%%%%%%%%%%%
\hrule \vskip 5pt
\noindent
\underbar{General concept practice and challenge problems}

\hskip 10pt Questions: {\it 93 through the end of the worksheet}
 
\vskip 10pt
%%%%%%%%%%%%%%%
\hrule \vskip 5pt
\noindent
\underbar{Impending deadlines}

\hskip 10pt {\bf Troubleshooting assessment (counter circuit) due at end of ELTR145, Section 3}

\hskip 10pt Question 61: Troubleshooting log
 
\hskip 10pt Question 62: Sample troubleshooting assessment grading criteria
 
\vskip 10pt
%%%%%%%%%%%%%%%







\vfil \eject

\centerline{\bf ELTR 145 (Digital 2), section 3} \bigskip 
 
\vskip 10pt

\noindent
{\bf Recommended schedule}

\vskip 5pt

%%%%%%%%%%%%%%%
\hrule \vskip 5pt
\noindent
\underbar{Day 1}

\hskip 10pt Topics: {\it Encoders and decoders}
 
\hskip 10pt Questions: {\it 1 through 10}
 
\hskip 10pt Lab Exercise: {\it 4-line to 16-line decoder (question 41)}
 
\vskip 10pt
%%%%%%%%%%%%%%%
\hrule \vskip 5pt
\noindent
\underbar{Day 2}

\hskip 10pt Topics: {\it Multiplexers and demultiplexers}
 
\hskip 10pt Questions: {\it 11 through 20}
 
\hskip 10pt Lab Exercise: {\it Arbitrary logic function with multiplexer (question 42)}
 
%INSTRUCTOR \hskip 10pt {\bf Demo: oscilloscope in "chop" and "alternate" modes}

\vskip 10pt
%%%%%%%%%%%%%%%
\hrule \vskip 5pt
\noindent
\underbar{Day 3}

\hskip 10pt Topics: {\it Display decoder/driver circuits}
 
\hskip 10pt Questions: {\it 21 through 30}
 
\hskip 10pt Lab Exercise: {\it 7-segment display circuit (question 43)}
 
\vskip 10pt
%%%%%%%%%%%%%%%
\hrule \vskip 5pt
\noindent
\underbar{Day 4}

\hskip 10pt Topics: {\it Programmable logic technology}
 
\hskip 10pt Questions: {\it 31 through 40}
 
\hskip 10pt Lab Exercise: {\it Troubleshooting practice (decade counter circuit -- question 44)}
 
\vskip 10pt
%%%%%%%%%%%%%%%
\hrule \vskip 5pt
\noindent
\underbar{Day 5}

\hskip 10pt Exam 3: {\it includes Arbitrary logic function performance assessment}
 
\hskip 10pt {\bf Troubleshooting Assessment due:} {\it Decade counter circuit}
 
\hskip 10pt Question 45: Troubleshooting log
 
\hskip 10pt Question 46: Sample troubleshooting assessment grading criteria
 
\vskip 10pt
%%%%%%%%%%%%%%%
\hrule \vskip 5pt
\noindent
\underbar{Troubleshooting practice problems}

\hskip 10pt Questions: {\it 47 through 56}
 
\vskip 10pt
%%%%%%%%%%%%%%%
\hrule \vskip 5pt
\noindent
\underbar{DC/AC/Semiconductor/Opamp review problems}

\hskip 10pt Questions: {\it 57 through 76}
 
\vskip 10pt
%%%%%%%%%%%%%%%
\hrule \vskip 5pt
\noindent
\underbar{General concept practice and challenge problems}

\hskip 10pt Questions: {\it 77 through the end of the worksheet}
 
%%%%%%%%%%%%%%%











\vfil \eject
\bye
