
%(BEGIN_QUESTION)
% Copyright 2006, Tony R. Kuphaldt, released under the Creative Commons Attribution License (v 1.0)
% This means you may do almost anything with this work of mine, so long as you give me proper credit

Many microcontrollers come equipped with a built-in PWM function, so that you do not have to code a custom PWM algorithm yourself.  This fact points to the popularity of pulse-width modulation as a control scheme.  Explain why PWM is so popular, and give a few practical examples of how it may be used.

\underbar{file 03991}
%(END_QUESTION)





%(BEGIN_ANSWER)

I'll let you do your own research for this question!  The answer(s) is/are not hard to find.

%(END_ANSWER)





%(BEGIN_NOTES)

Pulse-width modulation (PWM) is a very common and useful way of generating an analog output from a microcontroller (or other digital electronic circuit) capable only of "high" and "low" voltage level output.  With PWM, time (or more specifically, {\it duty cycle}) is the analog domain, while amplitude is the digital domain.  This allows us to "sneak" an analog signal through a digital (on-off) data channel.

%INDEX% Microcontroller, pulse-width modulation (PWM) output

%(END_NOTES)


