
%(BEGIN_QUESTION)
% Copyright 2003, Tony R. Kuphaldt, released under the Creative Commons Attribution License (v 1.0)
% This means you may do almost anything with this work of mine, so long as you give me proper credit

One major different between a common-emitter amplifier configuration and a common-collector amplifier configuration is a principle called {\it negative feedback}, where changes in output voltage "feed back" to influence the amplifier's input signal, which in turn influences the output voltage again.  Common-collector amplifier circuits have large amounts of negative feedback inherent to their design.

The absence or presence of negative feedback in an amplifier circuit has profound effects on voltage gain ($A_V$).  Compare the relative voltage gains of the following amplifiers:

$$\epsfbox{00963x01.eps}$$

At first, the low voltage gain of the common-collector amplifier may appear to be a disadvantage of that circuit design.  However, there is one major benefit relevant to the common-collector amplifier's voltage gain, being a direct result of negative feedback.  What is this advantage?

\underbar{file 00963}
%(END_QUESTION)





%(BEGIN_ANSWER)

Although the common-collector amplifier has a very low voltage gain ($A_V = 1$), that gain is absolutely {\it stable} over a wide range of operating conditions and component selection.

%(END_ANSWER)





%(BEGIN_NOTES)

Discuss with your students the many factors that can influence voltage gain in a common-emitter amplifier circuit, and then compare that relative instability with the rock-solid stability of the common-collector amplifier's voltage gain.  What benefit is it to a circuit designer to have a stable voltage gain from a particular amplifier design?

%INDEX% Negative feedback inherent in common-collector amplifiers

%(END_NOTES)


