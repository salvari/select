
%(BEGIN_QUESTION)
% Copyright 2003, Tony R. Kuphaldt, released under the Creative Commons Attribution License (v 1.0)
% This means you may do almost anything with this work of mine, so long as you give me proper credit

Explain the simplest way to perform a single-point calibration on a highly sensitive, precision voltmeter.  How do you ensure the voltmeter is receiving a fixed input of known quantity, especially without having expensive calibration equipment available?

\underbar{file 00527}
%(END_QUESTION)





%(BEGIN_ANSWER)

Short the voltmeter's test leads together.  This creates an input condition of zero volts.

%(END_ANSWER)





%(BEGIN_NOTES)

Challenge your students with this question: "If zero is an appropriate signal to use for a single-point calibration, then why not just leave the two test leads disconnected?  Why should you {\it short} them together?"

If you happen to have a sensitive voltmeter available in the classroom, the answer may be demonstrated with ease.  This is especially true if you place the voltmeter in the "AC millivolt" range so it picks up stray power-line voltages from nearby electrical devices and utilities.

%(END_NOTES)


