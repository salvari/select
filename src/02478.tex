
%(BEGIN_QUESTION)
% Copyright 2005, Tony R. Kuphaldt, released under the Creative Commons Attribution License (v 1.0)
% This means you may do almost anything with this work of mine, so long as you give me proper credit

The schematic diagram shown here is for a {\it "Cuk" converter circuit}, a type of DC-DC "switching" power conversion circuit:

$$\epsfbox{02478x01.eps}$$

In this circuit, the transistor is either fully on or fully off; that is, driven between the extremes of saturation or cutoff.  By avoiding the transistor's "active" mode (where it would drop substantial voltage while conducting current), very low transistor power dissipations can be achieved.  With little power wasted in the form of heat, "switching" power conversion circuits are typically very efficient.

Trace all current directions during both states of the transistor.  Also, mark the both inductors' voltage polarities during both states of the transistor.

\underbar{file 02478}
%(END_QUESTION)





%(BEGIN_ANSWER)

$$\epsfbox{02478x02.eps}$$

\vskip 10pt

Follow-up question: how does the load voltage of this converter relate to the supply (battery) voltage?  Does the load receive more or less voltage than that provided by the battery?

%(END_ANSWER)





%(BEGIN_NOTES)

The "strange" name of this circuit comes from the last name of the engineer who invented it!  For more information, consult the writings of Rudy Severns on the general topic of switch-mode power conversion circuits.

%INDEX% Cuk converter circuit

%(END_NOTES)


