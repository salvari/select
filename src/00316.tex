
%(BEGIN_QUESTION)
% Copyright 2003, Tony R. Kuphaldt, released under the Creative Commons Attribution License (v 1.0)
% This means you may do almost anything with this work of mine, so long as you give me proper credit

In the United States of America, the National Electrical Code (NEC) specifies certain color codes to be used for designation of grounded, grounding, and ungrounded wires in AC power systems.  First, relate the terms "hot," "neutral," and "ground" to the terms "grounded," "grounding," and "ungrounded".  Then, describe the acceptable color codes for each conductor.  Also, note the article and section of the National Electrical Code under which these specifications are found.

\underbar{file 00316}
%(END_QUESTION)





%(BEGIN_ANSWER)

Grounded = White or natural grey.

Grounding = Bare (uninsulated), green, or green with a yellow stripe.

Ungrounded = Colors other than white, natural grey, or green.

%(END_ANSWER)





%(BEGIN_NOTES)

It is important to note that even though these color codes have been standardized for many years, it is unwise to rely on color coding as an indication of hazard.  In other words, just because a wire has green insulation does not mean it is necessarily safe to touch!  There will always be mistakes made from time to time in wire installation, and in older systems where someone might have re-connected wires in incorrect ways.  When in doubt, always use a voltmeter to check for the presence of hazardous voltage!

%INDEX% Hot conductor, color code
%INDEX% Neutral conductor, color code
%INDEX% Ground conductor, color code
%INDEX% NEC
%INDEX% National Electrical Code

%(END_NOTES)


