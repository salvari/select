
%(BEGIN_QUESTION)
% Copyright 2003, Tony R. Kuphaldt, released under the Creative Commons Attribution License (v 1.0)
% This means you may do almost anything with this work of mine, so long as you give me proper credit

Calculate the voltage gain of this circuit, if R1 has a resistance of 8.1 k$\Omega$ and R2 has a resistance of 1.75 k$\Omega$:

$$\epsfbox{00673x01.eps}$$

\underbar{file 00673}
%(END_QUESTION)





%(BEGIN_ANSWER)

$A_V =$ 0.178

\vskip 10pt

Follow-up question: how does this gain figure ($A_V$) relate to the "voltage divider formula"?

$$E_R = E_{total} \Bigl({R \over R_{total}}\Bigr)$$


%(END_ANSWER)





%(BEGIN_NOTES)

Students should readily recognize this circuit as a voltage divider, from their education in basic DC circuits.  Though it may seem strange to calculate the "gain" of a completely passive and indeed {\it dissipative} circuit, it is entirely valid.

Discuss with your students the maximum and minimum possible power gain values for a circuit of this type.

%INDEX% Gain, of voltage divider circuit

%(END_NOTES)


