
%(BEGIN_QUESTION)
% Copyright 2003, Tony R. Kuphaldt, released under the Creative Commons Attribution License (v 1.0)
% This means you may do almost anything with this work of mine, so long as you give me proper credit

Suppose a capacitor is connected directly to an adjustable-voltage source, and the voltage of that source is steadily {\it increased} over time.  We know that an increasing voltage across a capacitor will produce an electric field of increasing strength.  Does this increase in electric field constitute an {\it accumulation} of energy in the capacitor, or a {\it release} of energy from the capacitor?  In this scenario, does the capacitor act as a {\it load} or as a {\it source} of electrical energy?

$$\epsfbox{00190x01.eps}$$

Now, suppose the adjustable voltage source is steadily {\it decreased} over time.  We know this will result in an electric field of decreasing strength in the capacitor.  Does this decrease in electric field constitute an {\it accumulation} of energy in the capacitor, or a {\it release} of energy from the capacitor?  In this scenario, does the capacitor act as a {\it load} or as a {\it source} of electrical energy?

$$\epsfbox{00190x02.eps}$$

For each of these scenarios, label the direction of current in the circuit.

\underbar{file 00190}
%(END_QUESTION)





%(BEGIN_ANSWER)

As the applied voltage increases, the capacitor acts as a load, accumulating additional energy from the voltage source.  Acting as a load, the current going "through" the capacitor will be in the same direction as through a resistor.

$$\epsfbox{00190x03.eps}$$

As the applied voltage decreases, the capacitor acts as a source, releasing accumulated energy to the rest of the circuit, as though it were a voltage source itself of superior voltage.  Acting as a source, the current going "through" the capacitor will be in the same direction as through a battery, powering a load.

$$\epsfbox{00190x04.eps}$$

%(END_ANSWER)





%(BEGIN_NOTES)

Relating the direction of current in a capacitor to a change of applied voltage over time is a complex concept for many students.  Since it involves rates of change over time, it is an excellent opportunity to introduce calculus concepts (${d \over dt}$).

Vitally important to students' conceptual understanding of a capacitor exposed to increasing or decreasing voltages is the distinction between an electrical energy {\it source} versus a {\it load}.  Students need to think "battery" and "resistor," respectively when determining the relationship between direction of current and voltage drop.  The complicated aspect of capacitors (and inductors!) is that they may switch character in an instant, from being a source of energy to being a load, and visa-versa.  The relationship is not fixed as it is for resistors, which are always energy {\it loads}.

%INDEX% Capacitance, voltage versus current in

%(END_NOTES)


