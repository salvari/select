
%(BEGIN_QUESTION)
% Copyright 2003, Tony R. Kuphaldt, released under the Creative Commons Attribution License (v 1.0)
% This means you may do almost anything with this work of mine, so long as you give me proper credit

$$\epsfbox{01657x01.eps}$$

\underbar{file 01657}
\vfil \eject
%(END_QUESTION)





%(BEGIN_ANSWER)

Use circuit simulation software to verify your predicted and measured parameter values.

%(END_ANSWER)





%(BEGIN_NOTES)

I recommend choosing resistor and capacitor values that yield time constants in the range that may be accurately tracked with a stopwatch.  I also recommend using resistor values significantly less than the voltmeter's input impedance, so that voltmeter loading does not significantly contribute to the decay rate.

Good time values to use ($t_1$, $t_2$, $t_3$) would be in the range of 5, 10, and 15 seconds, respectively.

An extension of this exercise is to incorporate troubleshooting questions.  Whether using this exercise as a performance assessment or simply as a concept-building lab, you might want to follow up your students' results by asking them to predict the consequences of certain circuit faults.

%INDEX% Assessment, performance-based (RC time constant)

%(END_NOTES)


