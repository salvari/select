
%(BEGIN_QUESTION)
% Copyright 2003, Tony R. Kuphaldt, released under the Creative Commons Attribution License (v 1.0)
% This means you may do almost anything with this work of mine, so long as you give me proper credit

A {\it logic probe} is a very useful tool for working with digital logic circuits.  It indicates "high" and "low" logic states by means of LED's, giving visual indication only if the voltage levels are appropriate for each state.

Here is a schematic diagram for a logic probe built using comparators.  Each comparator has a threshold adjustment potentiometer, so that it may be set to indicate its respective logic state only if the signal voltage is well within the range stated by the logic manufacturer:

$$\epsfbox{01262x01.eps}$$

When this logic probe circuit is connected to the $V_{CC}$ and $V_{EE}$ power supply terminals of a powered TTL circuit, what voltage levels should test points TP1 and TP2 be adjusted to, in order for the probe to properly indicate "high" and "low" TTL logic states?  Consult a datasheet for the quad NAND gate numbered either 74LS00 or 54LS00.  Both are legacy TTL integrated circuits.

\underbar{file 01262}
%(END_QUESTION)





%(BEGIN_ANSWER)

I'll let you do your own research on this question.  DO NOT obtain your answer from a textbook, but consult a manufacturer's datasheet instead!

\vskip 10pt

Follow-up question: given the standard $V_{CC}$ voltage level of 5.0 volts for TTL circuits, and assuming the use of LEDs that drop 1.7 volts at 20 mA, calculate an appropriate resistance value for the two LED current-limiting resistors.

\vskip 10pt

Challenge question: the logic probe circuit shown is minimal in component count.  To make a more practical and reliable probe, one would probably want to have reverse-polarity protection (in case someone were to accidently connect the probe backward across the power supply) as well as decoupling for immunity against electrical noise.  Add whatever necessary components you think there should be in this circuit to provide these features.

%(END_ANSWER)





%(BEGIN_NOTES)

The most obvious lesson of this question is to introduce (or review as the case may be) the purpose and operation of a logic probe.  However, this question is also a veiled introduction (or review) of TTL logic levels.

%INDEX% Logic gate voltage levels, "high" and "low" (TTL)
%INDEX% Logic probe circuit

%(END_NOTES)


