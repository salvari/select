
%(BEGIN_QUESTION)
% Copyright 2003, Tony R. Kuphaldt, released under the Creative Commons Attribution License (v 1.0)
% This means you may do almost anything with this work of mine, so long as you give me proper credit

Identify the mathematical function of this circuit (if you look closely, you'll notice that the transistors are connected in such a way that they act very similar to diodes):

$$\epsfbox{01019x01.eps}$$

Note: the two resistors labeled "R" are equal in value.

\underbar{file 01019}
%(END_QUESTION)





%(BEGIN_ANSWER)

This circuit {\it squares} the input signal ($y = x^2$).

\vskip 10pt

Challenge question: why are transistors used instead of diodes, since they have been effectively "disabled" to act as such?

%(END_ANSWER)





%(BEGIN_NOTES)

This circuit is not nearly as complex as it may appear at first, if students take the time to isolate it section-by-section and identify the mathematical function each section performs.

%INDEX% Exponentiator, nonlinear opamp circuit
%INDEX% Exponents and logarithms, as inverse functions (in a real nonlinear circuit)
%INDEX% Logarithm extractor, nonlinear opamp circuit
%INDEX% Logarithms and exponents, as inverse functions (in a real nonlinear circuit)

%(END_NOTES)


