
\centerline{\bf ELTR 100 (DC 1), section 2} \bigskip 
 
\vskip 10pt

\noindent
{\bf Recommended schedule}

\vskip 5pt

%%%%%%%%%%%%%%%
\hrule \vskip 5pt
\noindent
\underbar{Day 1}

\hskip 10pt Topics: {\it Series circuits and troubleshooting}
 
\hskip 10pt Questions: {\it 1 through 20}
 
\hskip 10pt Lab Exercises: {\it Series resistances (question 61)}
 
%INSTRUCTOR \hskip 10pt {\bf Demo: Set up and use battery/lamp troubleshooting board in front of class}

\vskip 10pt
%%%%%%%%%%%%%%%
\hrule \vskip 5pt
\noindent
\underbar{Day 2}

\hskip 10pt Topics: {\it Series circuits, wire resistance, and overcurrent protection}
 
\hskip 10pt Questions: {\it 21 through 40}
 
\hskip 10pt Lab Exercise: {\it Series DC resistor circuit (question 62)}
 
%INSTRUCTOR \hskip 10pt {\bf Demo: Show wire pieces of several different gauge}

\vskip 10pt
%%%%%%%%%%%%%%%
\hrule \vskip 5pt
\noindent
\underbar{Day 3}

\hskip 10pt Topics: {\it Series circuits, voltage divider circuits, and Kirchhoff's Voltage Law}
 
\hskip 10pt Questions: {\it 41 through 60}
 
\hskip 10pt Lab Exercise: {\it Series DC resistor circuit (question 63)}
 
%INSTRUCTOR \hskip 10pt {\bf MIT 6.002 video clip: Disk 1, Lecture 2; Kirchhoff's Voltage Law 8:02 to 10:36}

%INSTRUCTOR \hskip 10pt {\bf Demo: Set up battery/resistors circuit to validate KVL around any arbitrary path}

%INSTRUCTOR \hskip 10pt {\bf MIT 8.02 video clip: Disk 3, Lecture 19; Electrocardiogram demo 15:55 to 20:25}

\vskip 10pt
%%%%%%%%%%%%%%%
\hrule \vskip 5pt
\noindent
\underbar{Day 4}

\hskip 10pt Exam 2: {\it includes Series DC resistor circuit performance assessment}
 
\hskip 10pt Lab Exercise: {\it Troubleshooting practice (simple light bulb circuit)}
 
\vskip 10pt
%%%%%%%%%%%%%%%
\hrule \vskip 5pt
\noindent
\underbar{Practice and challenge problems}

\hskip 10pt Questions: {\it 66 through the end of the worksheet}
 
\vskip 10pt
%%%%%%%%%%%%%%%
\hrule \vskip 5pt
\noindent
\underbar{Impending deadlines}

\hskip 10pt {\bf Troubleshooting assessment (simple lamp circuit) due at end of ELTR100, Section 3}
 
\hskip 10pt Question 64: Troubleshooting log
 
\hskip 10pt Question 65: Sample troubleshooting assessment grading criteria
 
\hskip 10pt {\bf Solder-together kit due at end of ELTR100, Section 3}

\vskip 10pt
%%%%%%%%%%%%%%%




\vfil \eject

\centerline{\bf ELTR 100 (DC 1), section 2} \bigskip 
 
\vskip 10pt

\noindent
{\bf Skill standards addressed by this course section}

\vskip 5pt

%%%%%%%%%%%%%%%
\hrule \vskip 10pt
\noindent
\underbar{EIA {\it Raising the Standard; Electronics Technician Skills for Today and Tomorrow}, June 1994}

\vskip 5pt

\medskip
\item{\bf B} {\bf Technical Skills -- DC circuits}
\item{\bf B.03} Demonstrate an understanding of the meaning of and relationships among and between voltage, current, resistance and power in DC circuits.
\item{\bf B.05} Demonstrate an understanding of application of Ohm's Law to series, parallel, and series-parallel circuits.  {\it Partially met -- series circuits only.}
\item{\bf B.08} Understand principles and operations of DC series circuits.
\item{\bf B.09} Fabricate and demonstrate DC series circuits.
\item{\bf B.10} Troubleshoot and repair DC series circuits.
\medskip

\vskip 5pt

\medskip
\item{\bf B} {\bf Basic and Practical Skills -- Communicating on the Job}
\item{\bf B.01} Use effective written and other communication skills.  {\it Met by group discussion and completion of labwork.}
\item{\bf B.03} Employ appropriate skills for gathering and retaining information.  {\it Met by research and preparation prior to group discussion.}
\item{\bf B.04} Interpret written, graphic, and oral instructions.  {\it Met by completion of labwork.}
\item{\bf B.06} Use language appropriate to the situation.  {\it Met by group discussion and in explaining completed labwork.}
\item{\bf B.07} Participate in meetings in a positive and constructive manner.  {\it Met by group discussion.}
\item{\bf B.08} Use job-related terminology.  {\it Met by group discussion and in explaining completed labwork.}
\item{\bf B.10} Document work projects, procedures, tests, and equipment failures.  {\it Met by project construction and/or troubleshooting assessments.}
\item{\bf C} {\bf Basic and Practical Skills -- Solving Problems and Critical Thinking}
\item{\bf C.01} Identify the problem.  {\it Met by research and preparation prior to group discussion.}
\item{\bf C.03} Identify available solutions and their impact including evaluating credibility of information, and locating information.  {\it Met by research and preparation prior to group discussion.}
\item{\bf C.07} Organize personal workloads.  {\it Met by daily labwork, preparatory research, and project management.}
\item{\bf C.08} Participate in brainstorming sessions to generate new ideas and solve problems.  {\it Met by group discussion.}
\item{\bf D} {\bf Basic and Practical Skills -- Reading}
\item{\bf D.01} Read and apply various sources of technical information (e.g. manufacturer literature, codes, and regulations).  {\it Met by research and preparation prior to group discussion.}
\item{\bf E} {\bf Basic and Practical Skills -- Proficiency in Mathematics}
\item{\bf E.01} Determine if a solution is reasonable.
\item{\bf E.02} Demonstrate ability to use a simple electronic calculator.
\item{\bf E.05} Solve problems and [sic] make applications involving integers, fractions, decimals, percentages, and ratios using order of operations.
\item{\bf E.06} Translate written and/or verbal statements into mathematical expressions.
\item{\bf E.12} Interpret and use tables, charts, maps, and/or graphs.
\item{\bf E.13} Identify patterns, note trends, and/or draw conclusions from tables, charts, maps, and/or graphs.
\item{\bf E.15} Simplify and solve algebraic expressions and formulas.
\item{\bf E.16} Select and use formulas appropriately.
\item{\bf E.17} Understand and use scientific notation.
\medskip

%%%%%%%%%%%%%%%





\vfil \eject

\centerline{\bf ELTR 100 (DC 1), section 2} \bigskip 
 
\vskip 10pt

\noindent
{\bf Common areas of confusion for students}

\vskip 5pt

%%%%%%%%%%%%%%%
\hrule \vskip 5pt

\vskip 10pt

\noindent
{\bf Difficult concept: } {\it Using Ohm's Law in context.}

When applying Ohm's Law ($E = IR$ ; $I = {E \over R}$ ; $R = {E \over I}$) to circuits containing multiple resistances, students often mix contexts of voltage, current, and resistance.  Whenever you use any equation describing a physical phenomenon, be sure that each variable of that equation relates to the proper real-life value in the problem you're working on solving.  For example, when calculating the voltage drop across resistor $R_2$, you must be sure that the values for current and resistance are appropriate for that resistor and not some other resistor in the circuit.  If you are calculating $E_{R_2}$ using the Ohm's Law equation $E = IR$, then you must use the value of {\it that resistor's} current ($I_{R_2}$) and {\it that resistor's} resistance ($R_2$), not some other current and/or resistance value(s).  Some students have an unfortunate tendency to overlook context when seeking values to substitute in place of variables in Ohm's Law problems, and this leads to incorrect results.

\vskip 10pt

\noindent
{\bf \underbar{Very} difficult concept: } {\it Voltage is a relative quantity -- it only exists \underbar{between} two points.}

Unlike current, which may be measured at a single point in a circuit, voltage is fundamentally relative: it only exists as a {\it difference} between two points.  In other words, there is no such thing as voltage existing at a single location.  Therefore, while we speak of current going {\it through} a component in a circuit, we speak of voltage being {\it across} a component, measured between two different points on that component.  So confusing is this concept that a significant number of students continue to harbor conceptual errors about the nature of voltage for several months after having first learned about it.  A good way to understand voltage is to experiment with a voltmeter, measuring voltage between different pairs of points in safe, low-voltage circuits.  Another good way to gain proficiency is to practice on conceptual problems relating to the measurement of voltage in circuits.

\vskip 10pt

\noindent
{\bf Difficult concept: } {\it Kirchhoff's Voltage Law.}

The reason this concept is difficult to grasp is because it directly builds on the concept of voltage as a quantity relative between two points, which itself is a difficult concept to grasp.  Many students find the "altitude change" analogy helpful in understanding how voltage exists between pairs of points, and the practice problems using that analogy to be helpful in developing confidence with this concept.


