
%(BEGIN_QUESTION)
% Copyright 2005, Tony R. Kuphaldt, released under the Creative Commons Attribution License (v 1.0)
% This means you may do almost anything with this work of mine, so long as you give me proper credit

A technician is troubleshooting a power supply circuit with no DC output voltage.  The output voltage is supposed to be 15 volts DC, but instead it is actually outputting nothing at all (zero volts):

$$\epsfbox{02306x01.eps}$$

The technician measures 120 volts AC between test points TP1 and TP3.  Based on this voltage measurement and the knowledge that there is zero DC output voltage, identify two possible faults that could account for the problem and all measured values in this circuit, and also identify two circuit elements that could not possibly be to blame (i.e. two things that you know {\it must} be functioning properly, no matter what else may be faulted).  The circuit elements you identify as either possibly faulted or properly functioning can be wires, traces, and connections as well as components.  Be as specific as you can in your answers, identifying both the circuit element and the type of fault.

\medskip
\goodbreak
\item{$\bullet$} Circuit elements that are possibly faulted
\item{1.} 
\item{2.} 
\medskip

\medskip
\goodbreak
\item{$\bullet$} Circuit elements that must be functioning properly
\item{1.} 
\item{2.} 
\medskip


\underbar{file 02306}
%(END_QUESTION)





%(BEGIN_ANSWER)

I'll let you and your classmates figure out some possibilities here!

%(END_ANSWER)





%(BEGIN_NOTES)

Troubleshooting scenarios are always good for stimulating class discussion.  Be sure to spend plenty of time in class with your students developing efficient and logical diagnostic procedures, as this will assist them greatly in their careers.

%INDEX% Troubleshooting, power supply (AC-DC)

%(END_NOTES)


