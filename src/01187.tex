
%(BEGIN_QUESTION)
% Copyright 2003, Tony R. Kuphaldt, released under the Creative Commons Attribution License (v 1.0)
% This means you may do almost anything with this work of mine, so long as you give me proper credit

This is a schematic for a simple VCO:

$$\epsfbox{01187x01.eps}$$

The oscillator is of the RC "phase shift" design.  Explain how this circuit works.  Why does the output frequency vary as the control voltage varies?  Does the output frequency increase or decrease as the control voltage input receives a more positive voltage?

Hint: the JFETs in this circuit are not functioning as amplifiers!

\underbar{file 01187}
%(END_QUESTION)





%(BEGIN_ANSWER)

To understand how the JFETs are functioning in this VCO design, closely examine the "saturation" regions of a JFET's characteristic curves.  Note that these regions appear as nearly straight-line sections.  This indicates something about the behavior of a saturated JFET that is exploited in this VCO circuit.

The output frequency {\it decreases} as the control voltage becomes more positive.

%(END_ANSWER)





%(BEGIN_NOTES)

Not only does this question allow students to examine the workings of a VCO, but it also provides a good review of JFET theory, as well as a practical example of a special application of junction field-effect transistors.

Note: the schematic diagram for this circuit was derived from one found on page 997 of John Markus' \underbar{Guidebook of Electronic Circuits}, first edition.  Apparently, the design originated from a Motorola publication on using field effect transistors ("Low Frequency Applications of Field-Effect Transistors," AN-511, 1971).

%INDEX% VCO, discrete transistor example circuit

%(END_NOTES)


