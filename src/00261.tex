
%(BEGIN_QUESTION)
% Copyright 2003, Tony R. Kuphaldt, released under the Creative Commons Attribution License (v 1.0)
% This means you may do almost anything with this work of mine, so long as you give me proper credit

If an electric current is passed through this wire loop, in which position will it try to orient itself?

$$\epsfbox{00261x01.eps}$$

If this experiment is carried out, it may be found that the torque generated is quite small without resorting to high currents and/or strong magnetic fields.  Devise a way to modify this apparatus so as to generate stronger torques using modest current levels and ordinary magnets.

\underbar{file 00261}
%(END_QUESTION)





%(BEGIN_ANSWER)

The loop will try to orient itself in a vertical plane, perpendicular to the axis of magnetic flux between the magnet poles:

$$\epsfbox{00261x02.eps}$$

To increase the torque generated by the wire loop, you could use a loop with more than 1 "turn" of wire.  This is not the only solution, though.

%(END_ANSWER)





%(BEGIN_NOTES)

This question presents an excellent opportunity for discussing the "right-hand rule" (or "left-hand rule" for those using electron flow notation rather than conventional flow notation).

%(END_NOTES)


