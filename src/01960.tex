
%(BEGIN_QUESTION)
% Copyright 2003, Tony R. Kuphaldt, released under the Creative Commons Attribution License (v 1.0)
% This means you may do almost anything with this work of mine, so long as you give me proper credit

A {\it barometric altimeter} is a device used to measure altitude (height) by means of atmospheric pressure.  The higher you go up from sea level, the less air pressure you encounter.  This decrease in air pressure is closely correlated with height, and thus can be used to infer altitude.

This type of altimeter usually comes equipped with a "zero" adjustment, so that the instrument's indication may be offset to compensate for changes in air pressure resulting from different weather conditions.  This same "zero" adjustment may also be used to establish the altimeter's zero indication at any arbitrary height.

For example, if a mountain climber sets her barometric altimeter to zero meters at the base of a mountain, then climbs to the summit of that mountain (3400 meters higher than the base), the altimeter should register 3400 meters at the summit:

$$\epsfbox{01960x01.eps}$$

While at the summit, the climber may re-set the altimeter's "zero" adjustment to register 0 meters once again.  If the climber then descends to the base of the mountain, the altimeter will register -3400 meters:

$$\epsfbox{01960x02.eps}$$

Explain how this scenario of mountain climbing and altimeter calibration relates to the measurement of voltage between points {\bf A} and {\bf B} in the following circuit:

$$\epsfbox{01960x03.eps}$$

\underbar{file 01960}
%(END_QUESTION)





%(BEGIN_ANSWER)

The voltmeter's black lead is analogous to the "zero reference" level in the mountain-climbing altimeter scenario: the point at which the altimeter is calibrated to register 0 meters height.

%(END_ANSWER)





%(BEGIN_NOTES)

Physical height (and depth) is a very useful analogy for electrical potential, helping students relate this abstract thing called "voltage" to more common differential measurements.

%INDEX% Voltmeter usage

%(END_NOTES)


