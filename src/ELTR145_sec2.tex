
\centerline{\bf ELTR 145 (Digital 2), section 2} \bigskip 
 
\vskip 10pt

\noindent
{\bf Recommended schedule}

\vskip 5pt

%%%%%%%%%%%%%%%
\hrule \vskip 5pt
\noindent
\underbar{Day 1}

\hskip 10pt Topics: {\it Counter circuits}
 
\hskip 10pt Questions: {\it 1 through 10}
 
\hskip 10pt Lab Exercise: {\it 2-bit counter from flip-flops (question 56)}
 
\vskip 10pt
%%%%%%%%%%%%%%%
\hrule \vskip 5pt
\noindent
\underbar{Day 2}

\hskip 10pt Topics: {\it Counter circuits (continued)}
 
\hskip 10pt Questions: {\it 11 through 20}
 
\hskip 10pt Lab Exercise: {\it 4-bit up/down counter IC (question 57)}
 
\vskip 10pt
%%%%%%%%%%%%%%%
\hrule \vskip 5pt
\noindent
\underbar{Day 3}

\hskip 10pt Topics: {\it Shift registers}
 
\hskip 10pt Questions: {\it 21 through 30}
 
\hskip 10pt Lab Exercise: {\it Troubleshooting practice (decade counter circuit -- question 60)}
 
\vskip 10pt
%%%%%%%%%%%%%%%
\hrule \vskip 5pt
\noindent
\underbar{Day 4}

\hskip 10pt Topics: {\it Shift registers and serial data communication}
 
\hskip 10pt Questions: {\it 31 through 40}
 
\hskip 10pt Lab Exercise: {\it Frequency divider circuit (question 58)}
 
\vskip 10pt
%%%%%%%%%%%%%%%
\hrule \vskip 5pt
\noindent
\underbar{Day 5}

\hskip 10pt Topics: {\it Memory technologies}
 
\hskip 10pt Questions: {\it 41 through 55}
 
\hskip 10pt Lab Exercise: {\it 4-bit universal shift register IC (question 59)}
 
%INSTRUCTOR \hskip 10pt {\bf Socratic Electronics animation: ROM memory addressing}

\vskip 10pt
%%%%%%%%%%%%%%%
\hrule \vskip 5pt
\noindent
\underbar{Day 6}

\hskip 10pt Exam 2: {\it includes Counter circuit performance assessment}
 
\hskip 10pt Lab Exercise: {\it Troubleshooting practice (decade counter circuit -- question 60)}
  
\vskip 10pt
%%%%%%%%%%%%%%%
\hrule \vskip 5pt
\noindent
\underbar{Troubleshooting practice problems}

\hskip 10pt Questions: {\it 63 through 72}
 
\vskip 10pt
%%%%%%%%%%%%%%%
\hrule \vskip 5pt
\noindent
\underbar{DC/AC/Semiconductor/Opamp review problems}

\hskip 10pt Questions: {\it 73 through 92}
 
\vskip 10pt

%%%%%%%%%%%%%%%
\hrule \vskip 5pt
\noindent
\underbar{General concept practice and challenge problems}

\hskip 10pt Questions: {\it 93 through the end of the worksheet}
 
\vskip 10pt
%%%%%%%%%%%%%%%
\hrule \vskip 5pt
\noindent
\underbar{Impending deadlines}

\hskip 10pt {\bf Troubleshooting assessment (counter circuit) due at end of ELTR145, Section 3}

\hskip 10pt Question 61: Troubleshooting log
 
\hskip 10pt Question 62: Sample troubleshooting assessment grading criteria
 
\vskip 10pt
%%%%%%%%%%%%%%%







\vfil \eject

\centerline{\bf ELTR 145 (Digital 2), section 2} \bigskip 
 
\vskip 10pt

\noindent
{\bf Skill standards addressed by this course section}

\vskip 5pt

%%%%%%%%%%%%%%%
\hrule \vskip 10pt
\noindent
\underbar{EIA {\it Raising the Standard; Electronics Technician Skills for Today and Tomorrow}, June 1994}

\vskip 5pt

\medskip
\item{\bf F} {\bf Technical Skills -- Digital Circuits}
\item{\bf F.14} Understand principles and operations of types of registers and counters. 
\item{\bf F.15} Fabricate and demonstrate types of registers and counters. 
\item{\bf F.16} Troubleshoot and repair types of registers and counters. 
\medskip

\vskip 5pt

\medskip
\item{\bf B} {\bf Basic and Practical Skills -- Communicating on the Job}
\item{\bf B.01} Use effective written and other communication skills.  {\it Met by group discussion and completion of labwork.}
\item{\bf B.03} Employ appropriate skills for gathering and retaining information.  {\it Met by research and preparation prior to group discussion.}
\item{\bf B.04} Interpret written, graphic, and oral instructions.  {\it Met by completion of labwork.}
\item{\bf B.06} Use language appropriate to the situation.  {\it Met by group discussion and in explaining completed labwork.}
\item{\bf B.07} Participate in meetings in a positive and constructive manner.  {\it Met by group discussion.}
\item{\bf B.08} Use job-related terminology.  {\it Met by group discussion and in explaining completed labwork.}
\item{\bf B.10} Document work projects, procedures, tests, and equipment failures.  {\it Met by project construction and/or troubleshooting assessments.}
\item{\bf C} {\bf Basic and Practical Skills -- Solving Problems and Critical Thinking}
\item{\bf C.01} Identify the problem.  {\it Met by research and preparation prior to group discussion.}
\item{\bf C.03} Identify available solutions and their impact including evaluating credibility of information, and locating information.  {\it Met by research and preparation prior to group discussion.}
\item{\bf C.07} Organize personal workloads.  {\it Met by daily labwork, preparatory research, and project management.}
\item{\bf C.08} Participate in brainstorming sessions to generate new ideas and solve problems.  {\it Met by group discussion.}
\item{\bf D} {\bf Basic and Practical Skills -- Reading}
\item{\bf D.01} Read and apply various sources of technical information (e.g. manufacturer literature, codes, and regulations).  {\it Met by research and preparation prior to group discussion.}
\item{\bf E} {\bf Basic and Practical Skills -- Proficiency in Mathematics}
\item{\bf E.01} Determine if a solution is reasonable.
\item{\bf E.02} Demonstrate ability to use a simple electronic calculator.
\item{\bf E.06} Translate written and/or verbal statements into mathematical expressions.
\item{\bf E.07} Compare, compute, and solve problems involving binary, octal, decimal, and hexadecimal numbering systems.
\item{\bf E.12} Interpret and use tables, charts, maps, and/or graphs.
\item{\bf E.13} Identify patterns, note trends, and/or draw conclusions from tables, charts, maps, and/or graphs.
\item{\bf E.15} Simplify and solve algebraic expressions and formulas.
\item{\bf E.16} Select and use formulas appropriately.
\medskip

%%%%%%%%%%%%%%%




\vfil \eject

\centerline{\bf ELTR 145 (Digital 2), section 2} \bigskip 
 
\vskip 10pt

\noindent
{\bf Common areas of confusion for students}

\vskip 5pt

%%%%%%%%%%%%%%%
\hrule \vskip 5pt

\vskip 10pt

\noindent
{\bf Common mistake: } {\it How set-up time for flip-flops influences stage-to-stage propagation.}

When many students first examine cascaded, synchronous flip-flop circuits (where the $Q$ output of one enters the input of the next), they wonder why pulses don't just ripple through the whole chain of flip-flops with one clock pulse.  The reason this does not happen is that each flip-flop has a finite amount of {\it set-up time} required for the input state to be stable before it is recognized at the active edge of a new clock pulse.  Even with instantaneous flip-flop output state changes, the cascaded signal still cannot reach the input of the next flip-flop {\it before} the clock pulse arrives at that next flip-flop.  Therefore, the fastest a logic state can progress from one flip-flop to another in a synchronous counter circuit is one flip-flop per clock pulse.



