
%(BEGIN_QUESTION)
% Copyright 2005, Tony R. Kuphaldt, released under the Creative Commons Attribution License (v 1.0)
% This means you may do almost anything with this work of mine, so long as you give me proper credit

A famous illustrative story for understanding exponents goes something like this:

\vskip 10pt {\narrower \noindent \baselineskip5pt

{\it A pauper saves the life of a king.  In return, the king offers the pauper anything he desires as a reward.  The pauper, being a shrewd man, tells the king he does not want much, only a grain of rice today, then double that (two grains of rice) the next day, then double that (four grains of rice) the next day, and so on.  The king asks how long he is to give the pauper rice, and the pauper responds by saying one day for every square on a chess board (64 days).  This does not sound like much to the king, who never took a math course, and so he agrees.}

\par} \vskip 10pt

In just a short amount of time, though, the king finds himself bankrupted to the pauper because the quantity of rice is so enormously large.  Such is the nature of exponential functions: they grow incredibly large with modest gains in $x$.

\vskip 10pt

Graph the pauper's rice function ($y = 2^x$), with each division on the horizontal axis representing 1 unit and each division on the vertical axis representing 100 units.

$$\epsfbox{03058x01.eps}$$

\underbar{file 03058}
%(END_QUESTION)





%(BEGIN_ANSWER)

$$\epsfbox{03058x02.eps}$$

\vskip 10pt

Follow-up question: what do you think this graph will look like for negative values of $x$?

%(END_ANSWER)





%(BEGIN_NOTES)

From the graph shown, it may appear that the function approaches 0 as $x$ approaches zero.  This is not the case, as a simple calculation ($y = 2^0$) shows.  In order for students to adequately see what is going on near the origin, they will have to re-scale the graph.

%INDEX% Algebra, graphing simple functions
%INDEX% Graphing simple functions, algebra

%(END_NOTES)


