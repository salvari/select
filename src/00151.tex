
%(BEGIN_QUESTION)
% Copyright 2003, Tony R. Kuphaldt, released under the Creative Commons Attribution License (v 1.0)
% This means you may do almost anything with this work of mine, so long as you give me proper credit

Is it possible to have a condition where an electrical voltage exists, but no electric current exists?  Conversely, is it possible to have a condition where an electric current exists without an accompanying voltage?  Explain your answers, and give practical examples where the stated conditions are indeed possible.

\underbar{file 00151}
%(END_QUESTION)





%(BEGIN_ANSWER)

It is not only possible, but quite common in fact, to have a condition of voltage with no current.  However, the existence of an electric current must normally be accompanied by a voltage.  Only in very unique conditions (in "superconducting" circuits) may an electric current exist in the absence of a voltage.

%(END_ANSWER)





%(BEGIN_NOTES)

This question challenges students' comprehension of voltage and current by asking them to explain the relationship between the two quantities in practical contexts.  Do not allow students to simply give a "yes" or a "no" answer to either of the stated conditions.  Encourage them to think of examples illustrating a possible condition.

The term "superconducting" may spur some additional questions.  As usual, do not simply tell students what superconductivity is, but let them research this on their own.  Your more inquisitive students will probably have already researched this topic in response to the answer!

%INDEX% Voltage with no current
%INDEX% Current with no voltage
%INDEX% Superconducting wire

%(END_NOTES)


