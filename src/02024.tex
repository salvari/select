
%(BEGIN_QUESTION)
% Copyright 2003, Tony R. Kuphaldt, released under the Creative Commons Attribution License (v 1.0)
% This means you may do almost anything with this work of mine, so long as you give me proper credit

Suppose you were handed a black box with two metal terminals on one side, for attaching electrical (wire) connections.  Inside this box, you were told, was a current source connected in parallel with a resistance.

$$\epsfbox{02024x01.eps}$$

Your task was to experimentally determine the values of the current source and the resistor inside the box, and you did just that.  From your experimental data you then sketched a circuit with the following component values:

$$\epsfbox{02024x02.eps}$$

However, you later discovered that you had been tricked.  Instead of containing a current source and a resistor, the circuit inside the box was actually a {\it voltage source} connected in {\it series} with a resistor:

$$\epsfbox{02024x03.eps}$$

Demonstrate that these two different circuits are indistinguishable from the perspective of the two metal terminals, and explain what general principle this equivalence represents.

\underbar{file 02024}
%(END_QUESTION)





%(BEGIN_ANSWER)

A good way to demonstrate the electrical equivalence of these circuits is to calculate their responses to identical load resistor values.  The equivalence you see here proves that Th\'evenin and Norton equivalent circuits are interchangeable.

\vskip 10pt

Follow-up question: give a step-by-step procedure for converting a Th\'evenin equivalent circuit into a Norton equivalent circuit, and visa-versa.

%(END_ANSWER)





%(BEGIN_NOTES)

Ask your students to clearly state both Th\'evenin's and Norton's Theorems, and also discuss why both these theorems are important electrical analysis tools.

%INDEX% Thevenin and Norton circuit equivalence

%(END_NOTES)


