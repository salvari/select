
%(BEGIN_QUESTION)
% Copyright 2003, Tony R. Kuphaldt, released under the Creative Commons Attribution License (v 1.0)
% This means you may do almost anything with this work of mine, so long as you give me proper credit

An important parameter of every amplifier is {\it gain}.  Explain what "gain" is, and write a simple equation defining gain in terms of signal voltage.

\underbar{file 00864}
%(END_QUESTION)





%(BEGIN_ANSWER)

"Gain" is the degree of "magnification" that an amplifier provides its input signal.  Voltage gain may be defined in two different ways:

$$A_{V(dc)} = {V_{out} \over V_{in}}$$

$$A_{V(ac)} = {\Delta V_{out} \over \Delta V_{in}}$$

%(END_ANSWER)





%(BEGIN_NOTES)

As your students should be able to discern through context, the symbol used to represent gain in equations is the capital letter "A".  One potential point of confusion is the difference between the two gain equations shown in the answer.  Why would we have two different equations saying pretty much the same thing?  If this issue comes up in discussion, you can give your students the example of an amplifier with a {\it DC bias}, where $V_{out} = (4)(V_{in}) + 3$ volts.  Here, the (AC) gain is always 4, but the DC gain varies according to how much voltage we apply to the input!

Based on this example, which gain calculation do your students think is the more practical?

%(END_NOTES)


