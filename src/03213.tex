
%(BEGIN_QUESTION)
% Copyright 2005, Tony R. Kuphaldt, released under the Creative Commons Attribution License (v 1.0)
% This means you may do almost anything with this work of mine, so long as you give me proper credit

The following motor control circuit has a problem.  When the "Start" button is pressed, the motor refuses to start, although an audible "clunk" may be heard from the contactor when the "Start" switch is pressed, and again when the "Stop" switch is pressed:

$$\epsfbox{03213x01.eps}$$

Using your digital voltmeter, you measure 480 volts AC between TP2 and TP3 with the "Start" switch pressed.  From this information, identify two possible faults that could account for the problem and all measured values in this circuit, and also identify two circuit elements that could not possibly be to blame (i.e. two things that you know {\it must} be functioning properly, no matter what else may be faulted).  The circuit elements you identify as either possibly faulted or properly functioning can be wires, traces, and connections as well as components.  Be as specific as you can in your answers, identifying both the circuit element and the type of fault.

\medskip
\goodbreak
\item{$\bullet$} Circuit elements that are possibly faulted
\item{1.}
\item{2.} 
\medskip

\medskip
\goodbreak
\item{$\bullet$} Circuit elements that must be functioning properly
\item{1.} 
\item{2.} 
\medskip

\underbar{file 03213}
%(END_QUESTION)





%(BEGIN_ANSWER)

Note: the following answers are not exhaustive.  There may be more circuit elements possibly at fault and more circuit elements known to be functioning properly!

\medskip
\goodbreak
\item{$\bullet$} Circuit elements that are possibly faulted
\item{1.} Motor winding(s) failed open
\item{2.} One or more power conductors to motor failed open
\item{3.} One or more overload heater elements failed open
\item{4.} Power contact (one of the three M1 contacts to the motor) failed open
\medskip

\medskip
\goodbreak
\item{$\bullet$} Circuit elements that must be functioning properly
\item{1.} 3-phase power source good
\item{2.} Control power transformer (480/120 volt unit supplying L1 and L2)
\item{3.} Both fuses
\item{4.} "Start" switch contacts
\medskip

%(END_ANSWER)





%(BEGIN_NOTES)

{\bf This question is intended for exams only and not worksheets!}.

%INDEX% Troubleshooting, motor control circuit

%(END_NOTES)


