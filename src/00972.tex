
%(BEGIN_QUESTION)
% Copyright 2003, Tony R. Kuphaldt, released under the Creative Commons Attribution License (v 1.0)
% This means you may do almost anything with this work of mine, so long as you give me proper credit

A simple yet impractical way to eliminate crossover distortion in a Class B amplifier is to add two small voltage sources to the circuit like this:

$$\epsfbox{00972x01.eps}$$

Explain why this solution works to eliminate crossover distortion.

\vskip 10pt

Also, explain what practical purpose this push-pull amplifier circuit might serve, since its voltage gain is only 1 (0 dB).

\underbar{file 00972}
%(END_QUESTION)





%(BEGIN_ANSWER)

Each voltage source biases its respective transistor to be at the brink of turning on when the instantaneous voltage of the input ($V_{in}$) is 0 volts.

\vskip 10pt

Such amplifier circuits are typically used as {\it voltage buffers}: effectively diminishing the output impedance of the source (boosting its current sourcing/sinking capability) so that it may supply more current to a load.

\vskip 10pt

Challenge question: how would you estimate the output impedance of such an amplifier circuit?

%(END_ANSWER)





%(BEGIN_NOTES)

Ask your students to relate these bias voltage sources to the DC bias voltages previously seen in Class A amplifier designs.  How much voltage do they think would be necessary to properly bias each transistor?

%INDEX% Crossover distortion, push-pull amplifier
%INDEX% Push-pull amplifier, biasing
%INDEX% Push-pull amplifier, crossover distortion

%(END_NOTES)


