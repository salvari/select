
%(BEGIN_QUESTION)
% Copyright 2004, Tony R. Kuphaldt, released under the Creative Commons Attribution License (v 1.0)
% This means you may do almost anything with this work of mine, so long as you give me proper credit

$$\epsfbox{02097x01.eps}$$

\underbar{file 02097}
\vfil \eject
%(END_QUESTION)





%(BEGIN_ANSWER)

Use circuit simulation software to verify your predicted and measured parameter values.

%(END_ANSWER)





%(BEGIN_NOTES)

Use a sine-wave function generator for the AC voltage source.  Specify a resonant frequency within the audio range.

This exercise assumes access to a variety of capacitor and/or inductor sizes, so the student may make series-parallel networks to achieve the necessary values of $L$ and/or $C$.  It would be wise for you (the instructor) to check your students' component kits for $L$ and $C$ values prior to choosing resonant frequencies for them to try to achieve.

I also recommend having students use an oscilloscope to measure AC voltage in a circuit such as this, because some digital multimeters have difficulty accurately measuring AC voltage much beyond line frequency range.  I find it particularly helpful to set the oscilloscope to the "X-Y" mode so that it draws a thin line on the screen rather than sweeps across the screen to show an actual waveform.  This makes it easier to measure peak-to-peak voltage.

%INDEX% Assessment, performance-based (Passive LC filter circuit design, given resonant frequency)

%(END_NOTES)


