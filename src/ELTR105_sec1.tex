
\centerline{\bf ELTR 105 (DC 2), section 1} \bigskip 
 
\vskip 10pt

\noindent
{\bf Recommended schedule}

\vskip 5pt

%%%%%%%%%%%%%%%
\hrule \vskip 5pt
\noindent
\underbar{Day 1}

\hskip 10pt Topics: {\it Series-parallel circuit analysis}
 
\hskip 10pt Questions: {\it 1 through 15}
 
\hskip 10pt Lab Exercise: {\it Kirchhoff's Voltage Law (question 61)}
 
\vskip 10pt
%%%%%%%%%%%%%%%
\hrule \vskip 5pt
\noindent
\underbar{Day 2}

\hskip 10pt Topics: {\it Series-parallel circuits and Wheatstone bridges}
 
\hskip 10pt Questions: {\it 16 through 30}
 
\hskip 10pt Lab Exercise: {\it Wheatstone bridge (question 62)}
 
\vskip 10pt
%%%%%%%%%%%%%%%
\hrule \vskip 5pt
\noindent
\underbar{Day 3}

\hskip 10pt Topics: {\it Series-parallel circuits, safety grounding, and troubleshooting}
 
\hskip 10pt Questions: {\it 31 through 45}
 
\hskip 10pt Lab Exercise: {\it Series-parallel DC resistor circuit (question 63)}
 
\vskip 10pt
%%%%%%%%%%%%%%%
\hrule \vskip 5pt
\noindent
\underbar{Day 4}

\hskip 10pt Topics: {\it Loaded voltage dividers}
 
\hskip 10pt Questions: {\it 46 through 60}
 
\hskip 10pt Lab Exercise: {\it Loaded voltage divider (question 64)}
 
\vskip 10pt
%%%%%%%%%%%%%%%
\hrule \vskip 5pt
\noindent
\underbar{Day 5}

\hskip 10pt Exam 1: {\it includes Series-parallel DC resistor circuit performance assessment}
 
\hskip 10pt Lab Exercise: {\it Troubleshooting practice (loaded voltage divider circuit -- question 64)}
 
\vskip 10pt
%%%%%%%%%%%%%%%
\hrule \vskip 5pt
\noindent
\underbar{Practice and challenge problems}

\hskip 10pt Questions: {\it 67 through the end of the worksheet}
 
\vskip 10pt
%%%%%%%%%%%%%%%
\hrule \vskip 5pt
\noindent
\underbar{Impending deadlines}

\hskip 10pt {\bf Troubleshooting assessment (voltage divider) due at end of ELTR105, Section 3}
 
\hskip 10pt Question 65: Troubleshooting log
 
\hskip 10pt Question 66: Sample troubleshooting assessment grading criteria
 
\vskip 10pt
%%%%%%%%%%%%%%%








\vfil \eject

\centerline{\bf ELTR 105 (DC 2), section 1} \bigskip 
 
\vskip 10pt

\noindent
{\bf Skill standards addressed by this course section}

\vskip 5pt

%%%%%%%%%%%%%%%
\hrule \vskip 10pt
\noindent
\underbar{EIA {\it Raising the Standard; Electronics Technician Skills for Today and Tomorrow}, June 1994}

\vskip 5pt

\medskip
\item{\bf B} {\bf Technical Skills -- DC circuits}
\item{\bf B.03} Demonstrate an understanding of the meaning of and relationships among and between voltage, current, resistance and power in DC circuits.
\item{\bf B.05} Demonstrate an understanding of application of Ohm's Law to series, parallel, and series-parallel circuits.
\item{\bf B.14} Understand the principles and operations of DC series-parallel and bridge circuits.
\item{\bf B.15} Fabricate and demonstrate DC series-parallel and bridge circuits.
\item{\bf B.16} Troubleshoot and repair series-parallel and bridge circuits.
\item{\bf B.17} Understand the principles and operations of the Wheatstone bridge.
\item{\bf B.18} Understand principles and operations of DC voltage divider circuits (loaded and unloaded).
\item{\bf B.19} Fabricate and demonstrate DC voltage divider circuits (loaded and unloaded).
\item{\bf B.20} Troubleshoot and repair DC voltage divider circuits (loaded and unloaded).
\item{\bf C} {\bf Technical Skills -- AC circuits}
\item{\bf C.07} Demonstrate an understanding of the principle and operation of safety grounding systems: (lightning arrestors, ground fault interrupters, etc.).
\medskip

\vskip 5pt

\medskip
\item{\bf B} {\bf Basic and Practical Skills -- Communicating on the Job}
\item{\bf B.01} Use effective written and other communication skills.  {\it Met by group discussion and completion of labwork.}
\item{\bf B.03} Employ appropriate skills for gathering and retaining information.  {\it Met by research and preparation prior to group discussion.}
\item{\bf B.04} Interpret written, graphic, and oral instructions.  {\it Met by completion of labwork.}
\item{\bf B.06} Use language appropriate to the situation.  {\it Met by group discussion and in explaining completed labwork.}
\item{\bf B.07} Participate in meetings in a positive and constructive manner.  {\it Met by group discussion.}
\item{\bf B.08} Use job-related terminology.  {\it Met by group discussion and in explaining completed labwork.}
\item{\bf B.10} Document work projects, procedures, tests, and equipment failures.  {\it Met by project construction and/or troubleshooting assessments.}
\item{\bf C} {\bf Basic and Practical Skills -- Solving Problems and Critical Thinking}
\item{\bf C.01} Identify the problem.  {\it Met by research and preparation prior to group discussion.}
\item{\bf C.03} Identify available solutions and their impact including evaluating credibility of information, and locating information.  {\it Met by research and preparation prior to group discussion.}
\item{\bf C.07} Organize personal workloads.  {\it Met by daily labwork, preparatory research, and project management.}
\item{\bf C.08} Participate in brainstorming sessions to generate new ideas and solve problems.  {\it Met by group discussion.}
\item{\bf D} {\bf Basic and Practical Skills -- Reading}
\item{\bf D.01} Read and apply various sources of technical information (e.g. manufacturer literature, codes, and regulations).  {\it Met by research and preparation prior to group discussion.}
\item{\bf E} {\bf Basic and Practical Skills -- Proficiency in Mathematics}
\item{\bf E.01} Determine if a solution is reasonable.
\item{\bf E.02} Demonstrate ability to use a simple electronic calculator.
\item{\bf E.05} Solve problems and [sic] make applications involving integers, fractions, decimals, percentages, and ratios using order of operations.
\item{\bf E.06} Translate written and/or verbal statements into mathematical expressions.
\item{\bf E.12} Interpret and use tables, charts, maps, and/or graphs.
\item{\bf E.13} Identify patterns, note trends, and/or draw conclusions from tables, charts, maps, and/or graphs.
\item{\bf E.15} Simplify and solve algebraic expressions and formulas.
\item{\bf E.16} Select and use formulas appropriately.
\item{\bf E.17} Understand and use scientific notation.
\medskip

%%%%%%%%%%%%%%%




\vfil \eject

\centerline{\bf ELTR 105 (DC 2), section 1} \bigskip 
 
\vskip 10pt

\noindent
{\bf Common areas of confusion for students}

\vskip 5pt

%%%%%%%%%%%%%%%
\hrule \vskip 5pt

\vskip 10pt

\noindent
{\bf Difficult concept: } {\it Respective rules for series versus parallel circuits.}

By themselves, series circuits should be fairly easy for you to analyze at this point.  The same may be said for parallel circuits by themselves.  The respective rules for each (voltages same across parallel components; currents same through series components; parallel currents add; series voltages add; etc.) are not that difficult to remember, especially if you understand {\it why} each one is true.  Things become much trickier, though, when we begin to mix series networks of resistors with parallel networks of resistors, as we jump from one rule set to the other in our analysis.  My best advice here is to avoid trying to memorize the rules by rote, and rather work on comprehending why each rule is as it is for each type of circuit.  And of course, practice, practice, practice!

\vskip 10pt

\noindent
{\bf Difficult concept: } {\it Using Ohm's Law in context.}

When applying Ohm's Law ($E = IR$ ; $I = {E \over R}$ ; $R = {E \over I}$) to circuits containing multiple resistances, students often mix contexts of voltage, current, and resistance.  Whenever you use any equation describing a physical phenomenon, be sure that each variable of that equation relates to the proper real-life value in the problem you're working on solving.  For example, when calculating the current through resistor $R_2$, you must be sure that the values for voltage and resistance are appropriate for that resistor and not some other resistor in the circuit.  If you are calculating $I_{R_2}$ using the Ohm's Law equation $I = {E \over R}$, then you must use the value of {\it that resistor's} voltage ($E_{R_2}$) and {\it that resistor's} resistance ($R_2$), not some other voltage and/or resistance value(s).  Some students have an unfortunate tendency to overlook context when seeking values to substitute in place of variables in Ohm's Law problems, and this leads to incorrect results.

\vskip 10pt

\noindent
{\bf Common mistake: } {\it Carelessness when analyzing series-parallel networks.}

When analyzing a series-parallel circuit to determine component voltages and currents, one must be very careful to reduce the circuit step-by-step, section-by-section, into equivalent resistances.  Students new to this process typically see all the work that is involved and try to save effort by taking shortcuts, which paradoxically causes {\it more} work and {\it more} confusion for them later on.  Use lots of paper to document your work when you reduce series-parallel resistor networks to equivalent resistances, re-drawing the circuit for each reduction.  This helps reduce the number of mistakes, and also makes it easier to transfer calculated values of voltage and current to the correct components (see the difficult concept shown above: using Ohm's Law in context).

\vskip 10pt

\noindent
{\bf Difficult concept: } {\it Qualitative analysis of circuits.}

Most students find that qualitative analysis of electric circuits is much more difficult than quantitative analysis.  In other words, it is easier to use a calculator to compute how much current goes through a particular resistor in a circuit than it is to figure out if that current will increase, decrease, or remain the same given a certain change in the circuit.  What this requires is a "feel" for how variables in Ohm's Law and Kirchhoff's Laws relate to each other, which is a very different skill than numerical calculation.  A good way to build this skill is to practice qualitative analysis on every quantitative circuit question you encounter.  Even though the question only asks for a numerical answer, you can challenge yourself by asking whether or not a particular variable will change (and which direction that change will be in) given any particular change in the circuit (more or less supply voltage, a wire breaking open, a resistor failing open, or a resistor failing shorted).  Then, you may re-calculate the numerical answers to see if your qualitative predictions are correct.  Computer-based circuit simulation programs are excellent for this, as they allow one to skip the steps of numerical calculation to rapidly see the effects of certain circuit faults.

