
%(BEGIN_QUESTION)
% Copyright 2003, Tony R. Kuphaldt, released under the Creative Commons Attribution License (v 1.0)
% This means you may do almost anything with this work of mine, so long as you give me proper credit

Shown here is a simple two-bit binary counter circuit:

$$\epsfbox{01374x01.eps}$$

The $Q$ output of the first flip-flop constitutes the least significant bit (LSB), while the second flip-flop's $Q$ output constitutes the most significant bit (MSB).

Based on a timing diagram analysis of this circuit, determine whether it counts in an {\it up} sequence (00, 01, 10, 11) or a {\it down} sequence (00, 11, 10, 01).  Then, determine what would have to be altered to make it count in the other direction.

\underbar{file 01374}
%(END_QUESTION)





%(BEGIN_ANSWER)

This counter circuit counts in the {\it down} direction.  I'll let you figure out how to alter its direction of count!

%(END_ANSWER)





%(BEGIN_NOTES)

Actually, the counting sequence may be determined simply by analyzing the flip-flops' actions after the first clock pulse.  Writing a whole timing diagram for the count sequence may help some students to understand how the circuit works, but the more insightful students will be able to determine its counting direction without having to draw any timing diagram at all.

%INDEX% Counter circuit, up versus down sequence

%(END_NOTES)


