
%(BEGIN_QUESTION)
% Copyright 2003, Tony R. Kuphaldt, released under the Creative Commons Attribution License (v 1.0)
% This means you may do almost anything with this work of mine, so long as you give me proper credit

Suppose two wires, separated by an air gap, are connected to opposite terminals on a voltage source (such as a battery).  An electric field will develop in the space between the two wires: an invisible web of interaction, similar in some ways to a magnetic field.  In this diagram, draw the invisible "lines of flux" for this electric field, showing their physical range:

$$\epsfbox{00189x01.eps}$$

\underbar{file 00189}
%(END_QUESTION)





%(BEGIN_ANSWER)

$$\epsfbox{00189x02.eps}$$

\vskip 10pt

Follow-up question: explain how electric flux lines differ in geometry from magnetic flux lines.

%(END_ANSWER)





%(BEGIN_NOTES)

Students may note that electric lines of flux do not follow the same paths that magnetic lines of flux would.  Whereas magnetic lines of flux are always circular, electric lines of flux always terminate between points.

Note to your students the relevance of this fact in shielding: unlike magnetic shields which must {\it divert} the inevitable paths of magnetic flux lines, electric shields are able to {\it terminate} electric flux lines.

%INDEX% Electric field
%INDEX% Field, electric

%(END_NOTES)


