
%(BEGIN_QUESTION)
% Copyright 2003, Tony R. Kuphaldt, released under the Creative Commons Attribution License (v 1.0)
% This means you may do almost anything with this work of mine, so long as you give me proper credit

Trace the directions of all currents in this circuit, and determine which current is larger: the current through resistor R1 or the current through resistor R2, assuming equal resistor values.

$$\epsfbox{00522x01.eps}$$

If switch SW2 were opened (and switch SW1 remained closed), what would happen to the currents through R1 and R2?  

\vskip 10pt

If switch SW1 were opened (and switch SW2 remained closed), what would happen to the currents through R1 and R2?
 
\underbar{file 00522}
%(END_QUESTION)





%(BEGIN_ANSWER)

I'll let you determine the directions of all currents in this circuit!  Although it is impossible to tell with absolute certainty, the current through R1 is likely to be much greater than the current through R2.

\vskip 10pt

If SW2 opens while SW1 remains closed, both currents will cease.  If SW1 opens while SW2 remains closed, there will be no current through R1, but the current through R2 will actually increase.

\vskip 10pt

Follow-up question: what does this indicate about the nature of the two currents?  Which current exerts {\it control} over the other through the transistor?

%(END_ANSWER)





%(BEGIN_NOTES)

The most important principle in this question is that of {\it dependency}: one of the transistor's currents needs the other in order to exist, but not visa-versa.  I like to emphasize this relationship with the words {\it controlling} and {\it controlled}.

%INDEX% Transistor switch circuit (BJT)

%(END_NOTES)


