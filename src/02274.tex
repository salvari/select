
%(BEGIN_QUESTION)
% Copyright 2004, Tony R. Kuphaldt, released under the Creative Commons Attribution License (v 1.0)
% This means you may do almost anything with this work of mine, so long as you give me proper credit

A common modulation technique employed in radio broadcasting is {\it frequency modulation}, or {\it FM}.  Explain how a high-frequency {\it carrier} signal would be modulated by a lower-frequency signal such as in the case of the two signals shown here in the time domain:

$$\epsfbox{02274x01.eps} \hbox{\hskip 20pt} \epsfbox{02274x02.eps}$$

$$\epsfbox{02274x03.eps}$$

\underbar{file 02274}
%(END_QUESTION)





%(BEGIN_ANSWER)

$$\epsfbox{02274x04.eps}$$

%(END_ANSWER)





%(BEGIN_NOTES)

I do not expect that students will be able to precisely sketch the modulated waveform, especially when the period of the carrier is so short.  However, they should be able to express the general idea of frequency modulation in some form of drawing or sketch, and that's all I'm interested in seeing from them in response to this question.

%INDEX% Frequency modulation (FM), defined

%(END_NOTES)


