
\centerline{\bf ELTR 105 (DC 2), section 2} \bigskip 
 
\vskip 10pt

\noindent
{\bf Recommended schedule}

\vskip 5pt

%%%%%%%%%%%%%%%
\hrule \vskip 5pt
\noindent
\underbar{Day 1}

\hskip 10pt Topics: {\it Magnetism, electromagnetism, and electromagnetic induction}
 
\hskip 10pt Questions: {\it 1 through 20}
 
\hskip 10pt Lab Exercises: {\it Electromagnetism (question 71)}
 
%INSTRUCTOR \hskip 10pt {\bf MIT 8.02 video clip: Disk 2, Lecture 14; Single-wire field demo 17:32 to 18:35}

%INSTRUCTOR \hskip 10pt {\bf MIT 8.02 video clip: Disk 2, Lecture 14; Twin-wire field demo 19:08 to 20:14}

%INSTRUCTOR \hskip 10pt {\bf MIT 8.02 video clip: Disk 2, Lecture 11; Electromagnetism demos 6:50 to 12:45}

%INSTRUCTOR \hskip 10pt {\bf MIT 8.02 video clip: Disk 2, Lecture 11; Forces between two wires 12:56 to 17:20}

%INSTRUCTOR \hskip 10pt {\bf MIT 8.02 video clip: Disk 3, Lecture 15; Solenoid coil demos 23:00 to 27:15}

%INSTRUCTOR \hskip 10pt {\bf MIT 8.02 video clip: Disk 3, Lecture 16; Electromagnetic induction 10:30 to 12:30}

\vskip 10pt
%%%%%%%%%%%%%%%
\hrule \vskip 5pt
\noindent
\underbar{Day 2}

\hskip 10pt Topics: {\it Applications of electromagnetism and induction, Lenz's Law}
 
\hskip 10pt Questions: {\it 21 through 40}
 
\hskip 10pt Lab Exercise: {\it Electromagnetic induction (question 72)}
 
%INSTRUCTOR \hskip 10pt {\bf MIT 8.02 video clip: Disk 3, Lecture 21; Solenoid and armature demo 20:30 to 22:55}

%INSTRUCTOR \hskip 10pt {\bf MIT 8.02 video clip: Disk 2, Lecture 11; Torque on wire loop demo 46:34 to end}

%INSTRUCTOR \hskip 10pt {\bf MIT 8.02 video clip: Disk 3, Lecture 21; Homemade motor demo 4:20 to 5:55}

%INSTRUCTOR \hskip 10pt {\bf Demo: Use powerful magnet(s) and aluminum coin to show Lenz's Law}

%INSTRUCTOR \hskip 10pt {\bf MIT 8.02 video clip: Disk 3, Lecture 17; Damped pendulum demo 38:00 to 41:50}

%INSTRUCTOR \hskip 10pt {\bf MIT 8.02 video clip: Disk 3, Lecture 17; Ring drop demo 45:10 to 48:51}

%INSTRUCTOR \hskip 10pt {\bf Demo: Ammeter showing current drawn by electric motor under varying load}

\vskip 10pt
%%%%%%%%%%%%%%%
\hrule \vskip 5pt
\noindent
\underbar{Day 3}

\hskip 10pt Topics: {\it Introduction to Th\'evenin's and Norton's theorems}
 
\hskip 10pt Questions: {\it 41 through 55}
 
\hskip 10pt Lab Exercise: {\it Th\'evenin's theorem (question 73)}
 
%INSTRUCTOR \hskip 10pt {\bf Socratic Electronics animation: Th\'evenin's theorem demonstrated}

%INSTRUCTOR \hskip 10pt {\bf Demo: Build a Th\'evenin equivalent circuit for a more complex circuit}

\vskip 10pt
%%%%%%%%%%%%%%%
\hrule \vskip 5pt
\noindent
\underbar{Day 4}

\hskip 10pt Topics: {\it Th\'evenin's, Norton's, and Maximum Power Transfer theorems}
 
\hskip 10pt Questions: {\it 56 through 70} 
 
\hskip 10pt Lab Exercise: {\it Th\'evenin's theorem (question 73, continued)}
 
\vskip 10pt
%%%%%%%%%%%%%%%
\hrule \vskip 5pt
\noindent
\underbar{Day 5}

\hskip 10pt Exam 2: {\it includes Th\'evenin equivalent circuit performance assessment}
 
\hskip 10pt Lab Exercise: {\it Troubleshooting practice (loaded voltage divider circuit -- question 74)}
 
\vskip 10pt
%%%%%%%%%%%%%%%
\hrule \vskip 5pt
\noindent
\underbar{Practice and challenge problems}

\hskip 10pt Questions: {\it 77 through the end of the worksheet}
 
\vskip 10pt
%%%%%%%%%%%%%%%
\hrule \vskip 5pt
\noindent
\underbar{Impending deadlines}

\hskip 10pt {\bf Troubleshooting assessment (voltage divider) due at end of ELTR105, Section 3}
 
\hskip 10pt Question 75: Troubleshooting log
 
\hskip 10pt Question 76: Sample troubleshooting assessment grading criteria
 
\vskip 10pt
%%%%%%%%%%%%%%%









\vfil \eject

\centerline{\bf ELTR 105 (DC 2), section 2} \bigskip 
 
\vskip 10pt

\noindent
{\bf Skill standards addressed by this course section}

\vskip 5pt

%%%%%%%%%%%%%%%
\hrule \vskip 10pt
\noindent
\underbar{EIA {\it Raising the Standard; Electronics Technician Skills for Today and Tomorrow}, June 1994}

\vskip 5pt

\medskip
\item{\bf B} {\bf Technical Skills -- DC circuits}
\item{\bf B.01} Demonstrate an understanding of sources of electricity in DC circuits.
\item{\bf B.03} Demonstrate an understanding of the meaning of and relationships among and between voltage, current, resistance and power in DC circuits.
\item{\bf B.06} Demonstrate an understanding of magnetic properties of circuits and devices.
\medskip

\vskip 5pt

\medskip
\item{\bf B} {\bf Basic and Practical Skills -- Communicating on the Job}
\item{\bf B.01} Use effective written and other communication skills.  {\it Met by group discussion and completion of labwork.}
\item{\bf B.03} Employ appropriate skills for gathering and retaining information.  {\it Met by research and preparation prior to group discussion.}
\item{\bf B.04} Interpret written, graphic, and oral instructions.  {\it Met by completion of labwork.}
\item{\bf B.06} Use language appropriate to the situation.  {\it Met by group discussion and in explaining completed labwork.}
\item{\bf B.07} Participate in meetings in a positive and constructive manner.  {\it Met by group discussion.}
\item{\bf B.08} Use job-related terminology.  {\it Met by group discussion and in explaining completed labwork.}
\item{\bf B.10} Document work projects, procedures, tests, and equipment failures.  {\it Met by project construction and/or troubleshooting assessments.}
\item{\bf C} {\bf Basic and Practical Skills -- Solving Problems and Critical Thinking}
\item{\bf C.01} Identify the problem.  {\it Met by research and preparation prior to group discussion.}
\item{\bf C.03} Identify available solutions and their impact including evaluating credibility of information, and locating information.  {\it Met by research and preparation prior to group discussion.}
\item{\bf C.07} Organize personal workloads.  {\it Met by daily labwork, preparatory research, and project management.}
\item{\bf C.08} Participate in brainstorming sessions to generate new ideas and solve problems.  {\it Met by group discussion.}
\item{\bf D} {\bf Basic and Practical Skills -- Reading}
\item{\bf D.01} Read and apply various sources of technical information (e.g. manufacturer literature, codes, and regulations).  {\it Met by research and preparation prior to group discussion.}
\item{\bf E} {\bf Basic and Practical Skills -- Proficiency in Mathematics}
\item{\bf E.01} Determine if a solution is reasonable.
\item{\bf E.02} Demonstrate ability to use a simple electronic calculator.
\item{\bf E.05} Solve problems and [sic] make applications involving integers, fractions, decimals, percentages, and ratios using order of operations.
\item{\bf E.06} Translate written and/or verbal statements into mathematical expressions.
\item{\bf E.09} Read scale on measurement device(s) and make interpolations where appropriate.  {\it Met by DC voltmeter circuit.}
\item{\bf E.12} Interpret and use tables, charts, maps, and/or graphs.
\item{\bf E.13} Identify patterns, note trends, and/or draw conclusions from tables, charts, maps, and/or graphs.
\item{\bf E.15} Simplify and solve algebraic expressions and formulas.
\item{\bf E.16} Select and use formulas appropriately.
\item{\bf E.17} Understand and use scientific notation.
\medskip

\vskip 5pt

\medskip
\item{\bf F} {\bf Additional Skills -- Electromechanics}
\item{\bf B.01e} Types of motors.
\medskip

%%%%%%%%%%%%%%%




\vfil \eject

\centerline{\bf ELTR 105 (DC 2), section 2} \bigskip 
 
\vskip 10pt

\noindent
{\bf Common areas of confusion for students}

\vskip 5pt

%%%%%%%%%%%%%%%
\hrule \vskip 5pt

\vskip 10pt

\noindent
{\bf Difficult concept: } {\it Right-hand rule versus left-hand rule.}

This is an unnecessary difficulty brought on by the disagreement in convention describing electric current.  Some textbooks use "conventional flow" notation where current is shown moving from the positive terminal of a source to the negative terminal of a source.  With conventional flow notation, the relationship between current and magnetic field direction is described by the right-hand rule.  Other textbooks use "electron flow" notation, where the actual motion of electrons in a conductor is the assumed direction for current.  In this convention, current moves from the negative terminal of a source to the positive terminal, with the left-hand rule describing how a magnetic field relates to the current.

Which one is correct?  With metallic conductor circuits (which is nearly every practical circuit), electron flow notation is actually the most accurate description of charge carrier motion.  However, there are reasons why conventional flow notation (and the right-hand rule) is considered more "mathematically" correct, which is why engineers tend to use it.  This is especially the case when studying electromagnetism in depth, where the right-hand rule is revealed to be a specific application of the general mathematical rule for {\it vector cross-products}, and where a "left-hand rule" would go against all mathematical convention with regard to vectors.

In short, either rule will yield the correct results, so long as you are consistent in your conventions!  In other words, if you consistently stick with conventional flow and the right-hand rule, you will not go wrong.  Or, if you consistently stick with electron flow and the left-hand rule, you will not go wrong.  You may get yourself into trouble, though, if you try to switch back and forth between the two different notations for current.  My personal recommendation is to go with conventional flow notation and the right-hand rule, because most technical literature is written by and for engineers who predominantly use this.  Later on in your study of electronics you will see that all semiconductor device symbols were invented with conventional flow notation in mind, their arrows always pointing in that direction.

Another factor complicating this subject is the presence of similar rules invented to help simplify motor and generator operation, which in fact do not follow the conventions used in physics to relate electric current and magnetism at all.  Popularized in U.S. Navy publications, and called the "left-hand rule for generators" and the "right hand rule for motors," these rules switch the ordering of magnetic flux and current direction for the index and middle fingers, and use the thumb to represent direction of motion rather than direction of force as the real vector cross-product relationship goes.  The choice to have the thumb point in the direction of motion rather than the direction of force is why they have to use two different rules, one for generators and one for motors.  If you maintain the conventions used in physics where the index finger represents current, the middle finger represents magnetic flux, and the thumb points in the direction of {\it force} (not motion!), one rule will describe all scenarios, generators and motors alike.  The existence of two different rules (one for generators and one for motors) is a classic example of how over-simplification actually leads to confusion.

\vskip 10pt

\noindent
{\bf Difficult concept: } {\it Rates of change.}

When studying electromagnetic induction and Lenz's Law, one must think in terms of how fast a variable is changing.  The amount of voltage induced in a conductor is proportional to how {\it quickly} the magnetic field changes, not how strong the field is.  This is the first hurdle in calculus: to comprehend what a rate of change is, and it is not obvious.

The best examples I know of to describe rates of change are {\it velocity} and {\it acceleration}.  Velocity is nothing more than a rate of change of position: how quickly one's position is changing over time.  Therefore, if the variable $x$ describes position, then the derivative ${dx \over dt}$ (rate of change of $x$ over time $t$) must describe velocity.  Likewise, acceleration is nothing more than the rate of change of velocity: how quickly velocity changes over time.  If the variable $v$ describes velocity, then the derivative ${dv \over dt}$ must describe velocity.  Or, since we know that velocity is itself the derivative of position, we could describe acceleration as the {\it second derivative} of position: ${d^2 x \over dt^2}$

\vskip 10pt

\vfil \eject

\noindent
{\bf Difficult concept: } {\it Th\'evenin's and Norton's Theorems.}

These two theorems are used to simplify complex circuit networks so that you may mathematically analyze them easier.  Two major concepts are difficult here.  First is the concept of {\it equivalent networks}.  This is where we take a complex bunch of interconnected components and model them as a much simpler circuit that behaves in the same manner.  Understanding that it is possible for a simple circuit to behave identically to a larger, more complex circuit is not intuitive.  Naturally, we expect more complex things to have more complex behaviors.  However, both Th\'evenin's and Norton's theorems exploit the fact that certain types of circuits may indeed be greatly simplified and still maintain their original behaviors.

The second major hurdle for students is seeing where these Theorems are actually useful.  Here is where many textbooks fail in their presentation of these theorems: by showing the steps of Th\'evenin's and Norton's theorems only as a prelude to future applications.  Indeed, while these theorems are extremely useful in analyzing the transistor circuits you will encounter later on in your studies, it is a mistake to assume there are no immediate, useful applications which you may understand right away.  Hopefully the questions contained in this worksheet will allow you to comprehend some of these immediate applications.

\vskip 10pt

\noindent
{\bf Common mistake: } {\it Misunderstanding the Maximum Power Transfer Theorem.}

At first, this theorem appears to be very simple: maximum power will be dissipated at the load when the load resistance is equal to the source circuit's equivalent Th\'evenin or Norton resistance.  However, many students misunderstand the scope of this theorem.  What it is really saying is that power will be maximized at the load when $R_{load} = R_{Thevenin}$, {\it if $R_{load}$ is the only resistance you have the freedom to change}.  If $R_{load}$ happens to be fixed, but you have the freedom to change the source circuit's internal resistance, then maximum load power will be achieved when $R_{Thevenin}$ is equal to zero!

Another misconception is that this theorem tells you when optimum power {\it efficiency} is reached.  In fact, when $R_{load} = R_{Thevenin}$, the system efficiency is only 50\%.  Exactly half of the total source power will be wasted, with only half getting to the load!  Maximum {\it efficiency} is actually reached when the load resistance is very large compared to the source resistance, but of course the actual amount of load power (measured in watts) will be much less than if the load resistance were closer in value to $R_{Thevenin}$.

The best way to wrap your mind around all these concepts is to draw a two-resistor, one-battery circuit and do lots of sample calculations for load resistor power, seeing for yourself what happens as the different resistor values change.  Approach this scientifically: do a lot of calculations changing only one resistor at a time.  Make sure you do enough calculations where you can see the trend of how that one resistor's value affects load power before you decide to change the other resistor's value.  Changing only one variable at a time in a scientific test is very important.  Otherwise, it will be difficult (or impossible) to tell {\it what} what is causing what to happen.

