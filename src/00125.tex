
%(BEGIN_QUESTION)
% Copyright 2003, Tony R. Kuphaldt, released under the Creative Commons Attribution License (v 1.0)
% This means you may do almost anything with this work of mine, so long as you give me proper credit

Suppose this battery and light bulb circuit failed to work:

$$\epsfbox{00125x01.eps}$$

Using a voltmeter, a technician measures 0 volts between the points C and H.  The result of this single measurement indicates which half of the circuit there is a definite problem in.  What would you recommend as the {\it next} voltmeter measurement to take in troubleshooting the circuit, following the same "divide in half" strategy?

\underbar{file 00125}
%(END_QUESTION)





%(BEGIN_ANSWER)

To "divide the circuit in half" again, measure voltage between points B and G.

%(END_ANSWER)





%(BEGIN_NOTES)

Some troubleshooters refer to this strategy as "divide and conquer," because it divides the possibilities of fault location by a factor of 2 with each step.  

It is important to realize in situations such as this that no determination of faultlessness in the circuit has been made yet.  By measuring 0 volts between points C and H, we know there is a definite problem in the left half of the circuit, but we have by no means "cleared" the right half of the circuit of any fault.  For all we know, there may be faults in {\it both} halves of the circuit!  Only further investigation will reveal the truth.

%INDEX% Troubleshooting, simple circuit

%(END_NOTES)


