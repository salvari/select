
%(BEGIN_QUESTION)
% Copyright 2003, Tony R. Kuphaldt, released under the Creative Commons Attribution License (v 1.0)
% This means you may do almost anything with this work of mine, so long as you give me proper credit

There are two wire windings wrapped around a common iron bar in this illustration, such that whatever magnetic flux may be produced by one winding is fully shared by the other winding:

$$\epsfbox{00465x01.eps}$$

Write two equations describing the induced voltage at each winding ($e_p = \dots$ and $e_s = \dots$), in each case expressing the induced voltage in terms of the instantaneous current through that winding ($i_p$ and $i_s$, respectively) and the inductance of each winding ($L_p$ and $L_s$, respectively).

We know that the induced voltages in the two windings are related to each other by this equation, if there is perfect "coupling" of magnetic flux between the two windings:

$${e_p \over N_p} = {e_s \over N_s}$$

Knowing this, write two more equations describing induced voltage, this time expressing the induced voltage in each winding in terms of the instantaneous current in the {\it other} winding.  In other words,

$$e_p = \dots i_s$$

$$e_s = \dots i_p$$

\underbar{file 00465}
%(END_QUESTION)





%(BEGIN_ANSWER)

Equations describing self-inductance:

$$e_p = L_p{di_p \over dt}$$

$$e_s = L_s{di_s \over dt}$$

\vskip 15pt

Equations describing inductance from one winding to the other:

$$e_p = L_s {N_p \over N_s} {di_s \over dt}$$

$$e_s = L_p {N_s \over N_p} {di_p \over dt}$$

%(END_ANSWER)





%(BEGIN_NOTES)

The first two equations are mere review.  The second two equations require algebraic manipulation and substitution between equations.

%INDEX% Algebra, manipulating equations
%INDEX% Mutual inductance

%(END_NOTES)


