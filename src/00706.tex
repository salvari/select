
%(BEGIN_QUESTION)
% Copyright 2005, Tony R. Kuphaldt, released under the Creative Commons Attribution License (v 1.0)
% This means you may do almost anything with this work of mine, so long as you give me proper credit

Approximate the voltage gains of this {\it active filter} circuit at $f = 0$ and $f = \infty$ (assume ideal op-amp behavior):

$$\epsfbox{00706x01.eps}$$

\vskip 10pt

Approximate the voltage gains of this other "active filter" circuit at $f = 0$ and $f = \infty$ (assume ideal op-amp behavior):

$$\epsfbox{00706x02.eps}$$

\vskip 10pt

What type of filtering function (low pass, high pass, band pass, band stop) is provided by both these filter circuits?  Comparing these two circuit designs, which one do you think is more practical?  Explain your answer.

\underbar{file 00706}
%(END_QUESTION)





%(BEGIN_ANSWER)

These are both low pass filters.  The circuit with the shunt capacitor is the more practical one, because its voltage gain remains finite for all possible input signal frequencies:

$$\epsfbox{00706x01.eps}$$

%(END_ANSWER)





%(BEGIN_NOTES)

Discuss with your students their methods of determining filter type.  How did they approach this problem, to see what type of filter both these circuits were?

Also, discuss with your students the problem of having an amplifier circuit with an unchecked gain (approaching infinity).  Ask them what is wrong with obtaining such high voltage gains from any amplifier.  Have them describe to you the effect of a huge voltage gain on the integrity of the amplified signal.

%INDEX% Active filter circuit, conceptual

%(END_NOTES)


