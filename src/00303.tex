
%(BEGIN_QUESTION)
% Copyright 2003, Tony R. Kuphaldt, released under the Creative Commons Attribution License (v 1.0)
% This means you may do almost anything with this work of mine, so long as you give me proper credit

Determine whether or not a shock hazard exists for a person standing on the ground, by touching any one of the points labeled in this circuit:

$$\epsfbox{00303x01.eps}$$

\medskip
\item{$\bullet$} Point "A"
\item{$\bullet$} Point "B"
\item{$\bullet$} Point "C"
\item{$\bullet$} Point "D"
\item{$\bullet$} Point "E"
\medskip

\underbar{file 00303}
%(END_QUESTION)





%(BEGIN_ANSWER)

\medskip
\item{$\bullet$} Point "A" {\it dangerous to touch}
\item{$\bullet$} Point "B" {\it dangerous to touch}
\item{$\bullet$} Point "C" {\it dangerous to touch when motor is turned on}
\item{$\bullet$} Point "D" {\it safe to touch}
\item{$\bullet$} Point "E" {\it safe to touch}
\medskip

%(END_ANSWER)





%(BEGIN_NOTES)

One suggestion for approaching this question is to ask your students to identify which of the power source conductors is "hot" and which one is "neutral," then identify which points in the circuit are electrically common to either one or the other source conductors.

%INDEX% Safety, electrical

%(END_NOTES)


