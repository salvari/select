
%(BEGIN_QUESTION)
% Copyright 2005, Tony R. Kuphaldt, released under the Creative Commons Attribution License (v 1.0)
% This means you may do almost anything with this work of mine, so long as you give me proper credit

When performing tests on a radio transmitter, it is often necessary to do so without actually broadcasting a signal through an antenna.  In such scenarios, an equivalent {\it resistor} is connected to the output of the transmitter instead of an actual antenna.  If chosen properly, the resistor "looks" the same as an antenna from the perspective of the transmitter.

Explain how this is possible, since real antennae are built to have as little resistance as possible.  How can a {\it resistor} adequately substitute for an antenna, which is nothing like a resistor in either construction or purpose?

\underbar{file 03457}
%(END_QUESTION)





%(BEGIN_ANSWER)

Although an antenna has little actual resistance, it does radiate energy into space, just like a resistor dissipates energy in the form of heat.  The only significant difference is that an antenna's radiation is in the form of electromagnetic waves at the same frequency as the transmitter output.

%(END_ANSWER)





%(BEGIN_NOTES)

Ask your students what criteria they think a resistor needs to meet in order to properly serve as a "dummy" antenna.  Discuss impedance, Q factor, power rating, etc.

%INDEX% Antenna, modeled as a resistor
%INDEX% Resistor, serving as a "dummy" antenna

%(END_NOTES)


