
%(BEGIN_QUESTION)
% Copyright 2003, Tony R. Kuphaldt, released under the Creative Commons Attribution License (v 1.0)
% This means you may do almost anything with this work of mine, so long as you give me proper credit

Explain how this electrical system functions in a similar manner to a laboratory "balance" scale:

$$\epsfbox{00524x01.eps}$$

The circle with the letter "G" in it is a symbol for a {\it galvanometer}.  What practical purpose might a system like this serve?

\underbar{file 00524}
%(END_QUESTION)





%(BEGIN_ANSWER)

The galvanometer registers zero if and only if the two voltage sources are precisely equal.

%(END_ANSWER)





%(BEGIN_NOTES)

The first question your students are likely to ask is, "What is a {\it galvanometer}?"  Let them do the research on this question!  The answer is easy to find.

Next, your students will have to explain how this system may be used to do something practical.  Just as a laboratory balance-beam serves the purpose of measuring mass, this system also measures something.  Challenge your students to draw analogies between the components of this system and the components of a laboratory balance.  What essential feature must the $E_{standard}$ voltage source have in order for this to be a useful measurement system?

%INDEX% Galvanometer
%INDEX% Potentiometric (null-balance) voltage measurement

%(END_NOTES)


