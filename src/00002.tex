
%(BEGIN_QUESTION)
% Copyright 2003, Tony R. Kuphaldt, released under the Creative Commons Attribution License (v 1.0)
% This means you may do almost anything with this work of mine, so long as you give me proper credit

Determine if the light bulb will de-energize for each of the following breaks in the circuit.  Consider just one break at a time:

$$\epsfbox{00002x01.eps}$$

\medskip 
\item{$\bullet$} {\bf Choose one option for each point:}
\item{$\bullet$} A: de-energize / no effect
\item{$\bullet$} B: de-energize / no effect
\item{$\bullet$} C: de-energize / no effect
\item{$\bullet$} D: de-energize / no effect
\item{$\bullet$} E: de-energize / no effect
\item{$\bullet$} F: de-energize / no effect
\medskip 

\underbar{file 00002}
%(END_QUESTION)





%(BEGIN_ANSWER)

\medskip 
\item{$\bullet$} A: de-energize
\item{$\bullet$} B: no effect
\item{$\bullet$} C: no effect
\item{$\bullet$} D: no effect
\item{$\bullet$} E: de-energize
\item{$\bullet$} F: no effect
\medskip 

%(END_ANSWER)





%(BEGIN_NOTES)

This question is an important one in the students' process of learning troubleshooting.  Emphasize the importance of inductive thinking: deriving general principles from specific instances.  What does the behavior of this circuit tell us about {\it electrical continuity}?

%INDEX% Troubleshooting, simple circuit

%(END_NOTES)


