
%(BEGIN_QUESTION)
% Copyright 2003, Tony R. Kuphaldt, released under the Creative Commons Attribution License (v 1.0)
% This means you may do almost anything with this work of mine, so long as you give me proper credit

A common conceptual model of electrons within atoms is the "planetary" model, with electrons depicted as orbiting satellites whirling around the "planet" of the nucleus.  The physicist Ernest Rutherford is known as the inventor of this atomic model.  

A major improvement over this conceptual model of the atom came from Niels Bohr, who introduced the idea that electrons inhabited "stationary states" around the nucleus of an atom, and could only assume a new state by way of a {\it quantum leap}: a sudden "jump" from one energy level to another.

What led Bohr to his radical proposal of "quantum leaps" as an alternative to Rutherford's model?  What experimental evidence led scientists to abandon the old planetary model of the atom, and how does this evidence relate to modern electronics?

\underbar{file 00900}
%(END_QUESTION)





%(BEGIN_ANSWER)

The fact that atomic electrons inhabit "quantized" energy states is evidenced by the characteristic wavelengths of light emitted by certain atoms when they are "excited" by external energy sources.  Rutherford's planetary model could not account for this behavior, thus the necessity for a new model of the atom.  

Semiconductor electronics is made possible by the "quantum revolution" in physics.  Electrical current travel through semiconductors is impossible to adequately explain apart from quantum theory.

\vskip 10pt

Challenge question: think of an experiment that could be performed in the classroom to demonstrate the characteristic wavelengths emitted by "excited" atoms.

%(END_ANSWER)





%(BEGIN_NOTES)

It is no understatement to say that the advent of quantum theory changed the world, for it made possible modern solid-state electronics.  While the subject of quantum theory can be arcane, certain aspects of it are nevertheless essential to understanding electrical conduction in semiconductors.  

I cringe every time I read an introductory electronics textbook discuss electrons orbiting atomic nuclei like tiny satellites "held in orbit by electrostatic attraction and centrifugal force".  Then, a few pages later, these books start talking about valence bands, conduction bands, forbidden zones, and a host of other phenomenon that make absolutely no sense within the planetary model of the atom, but only make sense in a quantum view (where electrons are only "allowed" to inhabit certain, discrete energy states around the nucleus).

In case none of your students are able to answer the challenge question, you may give them this hint: {\it gas-discharge lamps} (neon, hydrogen, mercury vapor, sodium, etc.)!

%INDEX% Atomic models
%INDEX% Quantum physics

%(END_NOTES)


