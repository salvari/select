
%(BEGIN_QUESTION)
% Copyright 2005, Tony R. Kuphaldt, released under the Creative Commons Attribution License (v 1.0)
% This means you may do almost anything with this work of mine, so long as you give me proper credit

Trace the paths of {\it injection}, {\it diffusion}, and {\it collection} currents in this energy diagram for an NPN transistor as it is conducting:

$$\epsfbox{02034x01.eps}$$

\underbar{file 02034}
%(END_QUESTION)





%(BEGIN_ANSWER)

$$\epsfbox{02034x02.eps}$$

%(END_ANSWER)





%(BEGIN_NOTES)

A picture is worth a thousand words, they say.  For me, this illustration is the one that finally made transistors make sense to me.  By forward-biasing the emitter-base junction, minority carriers are injected into the base (electrons in the "P" type material, in the case of an NPN transistor), which then fall easily into the collector region.  This energy diagram is invaluable for explaining why collector current can flow even when the base-collector junction is reverse biased.

%INDEX% Collection current, BJT
%INDEX% Diffusion current, BJT
%INDEX% Energy diagram, conducting BJT
%INDEX% Injection current, BJT

%(END_NOTES)


