
%(BEGIN_QUESTION)
% Copyright 2003, Tony R. Kuphaldt, released under the Creative Commons Attribution License (v 1.0)
% This means you may do almost anything with this work of mine, so long as you give me proper credit

You will notice that a very important factor in your motor's performance is the {\it brush angle}: the points of contact that the brushes make with the commutator bars, as compared to the axis of the field magnets.  What brush angle results in maximum unloaded motor speed?  Why can't we just have the brushes contact the commutator bars at any arbitrary angle?

\underbar{file 01504}
%(END_QUESTION)





%(BEGIN_ANSWER)

There are certain brush angles for which the motor simply will not run at all.  This relates back to basic DC motor theory, and the interaction between the armature's magnetic field and the permanent magnets.

I cannot provide an "optimum" brush angle as an answer here, because that will vary with the specific design of your motor.

%(END_ANSWER)





%(BEGIN_NOTES)

Students are often fascinated with the effect that different brush contact angles have on motor performance, often spending hours of time experimenting to find the "sweet spot" for their particular motors.

%(END_NOTES)


