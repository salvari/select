
%(BEGIN_QUESTION)
% Copyright 2003, Tony R. Kuphaldt, released under the Creative Commons Attribution License (v 1.0)
% This means you may do almost anything with this work of mine, so long as you give me proper credit

Calculate and superimpose the {\it load line} for this circuit on top of the transistor's characteristic curves:

$$\epsfbox{00944x01.eps}$$

Then, determine the amount of collector current in the circuit at the following base current values:

\medskip
\item{$\bullet$} $I_B =$ 10 $\mu$A
\item{$\bullet$} $I_B =$ 20 $\mu$A
\item{$\bullet$} $I_B =$ 30 $\mu$A
\item{$\bullet$} $I_B =$ 40 $\mu$A
\medskip

\underbar{file 00944}
%(END_QUESTION)





%(BEGIN_ANSWER)

$$\epsfbox{00944x02.eps}$$

\medskip
\item{$\bullet$} $I_B =$ 10 $\mu$A ; $I_C =$ 3.75 mA 
\item{$\bullet$} $I_B =$ 20 $\mu$A ; $I_C =$ 6.25 mA
\item{$\bullet$} $I_B =$ 30 $\mu$A ; $I_C =$ 8.5 mA
\item{$\bullet$} $I_B =$ 40 $\mu$A ; $I_C =$ 9.5 mA
\medskip

%(END_ANSWER)





%(BEGIN_NOTES)

It would be good to point out something here: superimposing a linear function on a set of nonlinear functions and looking for the intersection points allows us to solve for multiple variables in a nonlinear mathematical system.  Normally, only {\it linear} systems of equations are considered "solvable" without resorting to very time-consuming arithmetic computations, but here we have a powerful (graphical) tool for approximating the values of variables in a nonlinear system.  Since approximations are the best we can hope for in transistor circuits anyway, this is good enough!

%INDEX% Load line calculation, BJT amplifier circuit

%(END_NOTES)


