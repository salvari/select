
%(BEGIN_QUESTION)
% Copyright 2003, Tony R. Kuphaldt, released under the Creative Commons Attribution License (v 1.0)
% This means you may do almost anything with this work of mine, so long as you give me proper credit

Voltage and current gains, expressed in units of decibels, may be calculated as such:

$$A_{V(dB)} = 10 \log \left( {A_{V(ratio)}} \right) ^2$$

$$A_{I(dB)} = 10 \log \left( {A_{I(ratio)}} \right) ^2$$

Another way of writing this equation is like this:

$$A_{V(dB)} = 20 \log A_{V(ratio)}$$

$$A_{I(dB)} = 20 \log A_{I(ratio)}$$

What law of algebra allows us to simplify a logarithmic equation in this manner?

\underbar{file 00830}
%(END_QUESTION)





%(BEGIN_ANSWER)

$$\log a^b = b \log a$$

\vskip 10pt

Challenge question: knowing this algebraic law, solve for $x$ in the following equation:

$$520 = 8^x$$

%(END_ANSWER)





%(BEGIN_NOTES)

Logarithms are a confusing, but powerful, algebraic tool.  In this example, we see how the logarithm of a power function is converted into a simple multiplication function.

The challenge question asks students to apply this relationship to an equation not containing logarithms at all.  However, the fundamental rule of algebra is that you may perform any operation (including logarithms) to any equation so long as you apply it equally to {\it both sides} of the equation.  Logarithms allow us to take an algebra problem such as this and simplify it significantly.

%INDEX% Algebra, solving for an exponent using logarithms
%INDEX% Logarithm of a power

%(END_NOTES)


