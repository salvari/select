
%(BEGIN_QUESTION)
% Copyright 2003, Tony R. Kuphaldt, released under the Creative Commons Attribution License (v 1.0)
% This means you may do almost anything with this work of mine, so long as you give me proper credit

Suppose this battery and light bulb circuit failed to work.  Using nothing but a voltmeter, how would you check the circuit to determine where the problem is located?  Note: the letters indicate "test points" along the wiring where you may probe with the circuit with your voltmeter.

$$\epsfbox{00118x01.eps}$$

\underbar{file 00118}
%(END_QUESTION)





%(BEGIN_ANSWER)

There are several strategies which may be employed to find the location of the problem in this circuit.  One popular technique is to "divide the circuit in half" by testing for voltage between points C and H first.  The presence of absence of voltage between these two points will indicate whether the problem lies between those points and the battery, or between those points and the light bulb (assuming there is but a {\it single} problem in the circuit -- a large assumption!).

%(END_ANSWER)





%(BEGIN_NOTES)

A circuit like this is very easy to construct, and makes for an excellent classroom demonstration piece.  I've used such a circuit, constructed on a piece of pegboard 2 feet by 4 feet, with metal screws acting as test points, for students to develop their troubleshooting skills in front of the class where everyone may observe and learn together.

It has been my experience that students who experience difficulty troubleshooting circuits in general usually experience difficulty troubleshooting this simple circuit in particular.  Although the circuit itself couldn't be simpler, the fundamental concept of {\it voltage} as a quantity measurable only between 2 points is confusing for many.  Spending lots of time learning to troubleshoot a circuit such as this will be greatly beneficial in the future!

%INDEX% Troubleshooting, simple circuit
%INDEX% Voltmeter usage

%(END_NOTES)


