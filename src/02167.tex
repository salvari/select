
%(BEGIN_QUESTION)
% Copyright 2004, Tony R. Kuphaldt, released under the Creative Commons Attribution License (v 1.0)
% This means you may do almost anything with this work of mine, so long as you give me proper credit

A {\it soldering gun} is a tool used to quickly heat electrical connections for soldering.  Too bulky for printed-circuit board (PCB) applications, it is better suited for point-to-point wiring applications where large wires are to be joined to metal lugs and to other wires.

In addition to being a useful soldering tool, this device is also an excellent example of a {\it step-down transformer}.  Explain how the construction of a soldering gun employs a step-down transformer (with a very large step ratio!) to generate high temperatures at the soldering tip.

\underbar{file 02167}
%(END_QUESTION)





%(BEGIN_ANSWER)

This question is best answered through disassembly and inspection of a real soldering gun.  These tools are fairly easy to take apart and reassemble, so there should be little concern of damage from such exploration.  Although it should go without saying, {\it never disassemble an electrical device that is still connected to line power!}

%(END_ANSWER)





%(BEGIN_NOTES)

For students lacking soldering guns to take apart, and for those who do not want to take any chances ruining a tool through improper disassembly/reassembly, it is not difficult to find photographs of soldering gun internals.  The step-down transformer assembly should be obvious when inspected.

%INDEX% Soldering gun, as step-down transformer

%(END_NOTES)


