
%(BEGIN_QUESTION)
% Copyright 2003, Tony R. Kuphaldt, released under the Creative Commons Attribution License (v 1.0)
% This means you may do almost anything with this work of mine, so long as you give me proper credit

There are two general classes of MOSFETs: MOSFETs that conduct with no applied gate voltage, and MOSFETs that require a gate voltage to be applied for conduction.  What are each of these MOSFET types called, and what are their respective schematic symbols?

Each of the symbols for these different types of MOSFETs hold clues to the transistor types they represent.  Explain how the symbols hint at the characteristics of their respective transistor types.

\underbar{file 01069}
%(END_QUESTION)





%(BEGIN_ANSWER)

{\it Depletion-type} (D-type) MOSFETs conduct current with no applied gate voltage.  {\it Enhancement-type} (E-type) MOSFETs require a gate voltage to be applied for conduction.

$$\epsfbox{01069x01.eps}$$

%(END_ANSWER)





%(BEGIN_NOTES)

The part of this question asking about clues within the transistor symbols is very important.  It will be far easier for your students to remember the function of each transistor type if they are able to recognize clues in the symbolism.

%INDEX% MOSFET, schematic symbols for

%(END_NOTES)


