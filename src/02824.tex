
%(BEGIN_QUESTION)
% Copyright 2005, Tony R. Kuphaldt, released under the Creative Commons Attribution License (v 1.0)
% This means you may do almost anything with this work of mine, so long as you give me proper credit

A {\it seven segment decoder} is a digital circuit designed to drive a very common type of digital display device: a set of LED (or LCD) segments that render numerals 0 through 9 at the command of a four-bit code:

$$\epsfbox{02824x01.eps}$$

The behavior of the display driver IC may be represented by a truth table with seven outputs: one for each segment of the seven-segment display ($a$ through $g$).  In the following table, a "1" output represents an active display segment, while a "0" output represents an inactive segment:

% No blank lines allowed between lines of an \halign structure!
% I use comments (%) instead, so that TeX doesn't choke.

$$\vbox{\offinterlineskip
\halign{\strut
\vrule \quad\hfil # \ \hfil & 
\vrule \quad\hfil # \ \hfil & 
\vrule \quad\hfil # \ \hfil & 
\vrule \quad\hfil # \ \hfil & 
\vrule \quad\hfil # \ \hfil & 
\vrule \quad\hfil # \ \hfil & 
\vrule \quad\hfil # \ \hfil & 
\vrule \quad\hfil # \ \hfil & 
\vrule \quad\hfil # \ \hfil & 
\vrule \quad\hfil # \ \hfil & 
\vrule \quad\hfil # \ \hfil & 
\vrule \quad\hfil # \ \hfil \vrule \cr
\noalign{\hrule}
%
% First row
D & C & B & A & a & b & c & d & e & f & g & Display\cr
%
\noalign{\hrule}
%
% Second row
0 & 0 & 0 & 0 & 1 & 1 & 1 & 1 & 1 & 1 & 0 & "0" \cr
%
\noalign{\hrule}
%
% Third row
0 & 0 & 0 & 1 & 0 & 1 & 1 & 0 & 0 & 0 & 0 & "1" \cr
%
\noalign{\hrule}
%
% Fourth row
0 & 0 & 1 & 0 & 1 & 1 & 0 & 1 & 1 & 0 & 1 & "2" \cr
%
\noalign{\hrule}
%
% Fifth row
0 & 0 & 1 & 1 & 1 & 1 & 1 & 1 & 0 & 0 & 1 & "3" \cr
%
\noalign{\hrule}
%
% Sixth row
0 & 1 & 0 & 0 & 0 & 1 & 1 & 0 & 0 & 1 & 1 & "4" \cr
%
\noalign{\hrule}
%
% Seventh row
0 & 1 & 0 & 1 & 1 & 0 & 1 & 1 & 0 & 1 & 1 & "5" \cr
%
\noalign{\hrule}
%
% Eighth row
0 & 1 & 1 & 0 & 1 & 0 & 1 & 1 & 1 & 1 & 1 & "6" \cr
%
\noalign{\hrule}
%
% Ninth row
0 & 1 & 1 & 1 & 1 & 1 & 1 & 0 & 0 & 0 & 0 & "7" \cr
%
\noalign{\hrule}
%
% Tenth row
1 & 0 & 0 & 0 & 1 & 1 & 1 & 1 & 1 & 1 & 1 & "8" \cr
%
\noalign{\hrule}
%
% Eleventh row
1 & 0 & 0 & 1 & 1 & 1 & 1 & 1 & 0 & 1 & 1 & "9" \cr
%
\noalign{\hrule}
} % End of \halign 
}$$ % End of \vbox

Write the unsimplified SOP or POS expressions (choose the most appropriate form) for outputs $a$, $b$, $c$, and $e$.

\underbar{file 02824}
%(END_QUESTION)





%(BEGIN_ANSWER)

Raw (unsimplified) expressions:

$$a = (D + C + B + \overline{A})(D + \overline{C} + B + A)$$

$$b = (D + \overline{C} + B + \overline{A})(D + \overline{C} + \overline{B} + A)$$

$$c = D + C + \overline{B} + A$$

$$e = \overline{D} \> \overline{C} \> \overline{B} \> \overline{A} + \overline{D} \> \overline{C} B \overline{A} + \overline{D} C B \overline{A} + D \overline{C} \> \overline{B} \> \overline{A}$$

\vskip 10pt

Challenge question: use the laws of Boolean algebra to simplify each of the above expressions into their simplest forms.

%(END_ANSWER)





%(BEGIN_NOTES)

This shows a very practical example of SOP and POS Boolean forms, and why simplification is necessary to reduce the number of required gates to a practical minimum.

%INDEX% 7-segment decoder circuit
%INDEX% POS expression, Boolean algebra (from analysis of a truth table)
%INDEX% Product-of-Sums expression, Boolean algebra (from analysis of a truth table)
%INDEX% SOP expression, Boolean algebra (from analysis of a truth table)
%INDEX% Seven-segment decoder circuit
%INDEX% Sum-of-Products expression, Boolean algebra (from analysis of a truth table)

%(END_NOTES)


