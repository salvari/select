
%(BEGIN_QUESTION)
% Copyright 2003, Tony R. Kuphaldt, released under the Creative Commons Attribution License (v 1.0)
% This means you may do almost anything with this work of mine, so long as you give me proper credit

How many degrees of phase shift must the feedback circuit (the box in this schematic) introduce to the signal in order for this two-stage common-emitter amplifier circuit to oscillate?

$$\epsfbox{01212x01.eps}$$

Why is this amount of phase shift different from that of a single-transistor oscillator?

\underbar{file 01212}
%(END_QUESTION)





%(BEGIN_ANSWER)

The feedback network in this circuit must provide 0 degrees of phase shift, in order to sustain oscillations.  

%(END_ANSWER)





%(BEGIN_NOTES)

Ask your students to explain why the feedback network must provide 180 degrees of phase shift to the signal.  Ask them to explain how this requirement relates to the need for {\it regenerative} feedback in an oscillator circuit.

The question and answer concerning feedback component selection is a large conceptual leap for some students.  It may baffle some that the phase shift of a reactive circuit will always be the proper amount to ensure regenerative feedback, for any arbitrary combination of component values, because they should know the phase shift of a reactive circuit depends on the values of its constituent components.  However, once they realize that the phase shift of a reactive circuit is {\it also} dependent on the signal frequency, the resolution to this paradox is much easier to understand.

%INDEX% Oscillator circuit, feedback network phase shift necessary for oscillation

%(END_NOTES)


