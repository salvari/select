
%(BEGIN_QUESTION)
% Copyright 2003, Tony R. Kuphaldt, released under the Creative Commons Attribution License (v 1.0)
% This means you may do almost anything with this work of mine, so long as you give me proper credit

If we apply Ohm's Law to the calculation of armature current, based on the applied voltage and the known armature winding resistance, we obtain a value for current that {\it far} exceeds the value of current measured when the motor is running at speed.  Explain why this is.

\underbar{file 01503}
%(END_QUESTION)





%(BEGIN_ANSWER)

I won't directly answer the question here, but I will tell you that the calculated armature current will the same as the actual motor running current {\it if} the armature is forcibly stalled (held so that it cannot move).  It is only when the armature gains rotational speed that the operating current dramatically decreases: this is your hint.

%(END_ANSWER)





%(BEGIN_NOTES)

Once students understand this phenomenon, they will comprehend why electric motors of all kinds draw massive amounts of current when first started ("{\it inrush} current"), but do not when running at full speed.  Being able to see this first-hand is a great educational experience for most students.

%(END_NOTES)


