
%(BEGIN_QUESTION)
% Copyright 2004, Tony R. Kuphaldt, released under the Creative Commons Attribution License (v 1.0)
% This means you may do almost anything with this work of mine, so long as you give me proper credit

Explain why it is important for the final power transistor(s) in a PWM power control circuit to operate at full cutoff and full saturation, and not in the linear (active) mode in between those two extremes.  What might happen if the power transistor(s) were to be less than cut-off or less than saturated when carrying load current?

\underbar{file 02155}
%(END_QUESTION)





%(BEGIN_ANSWER)

Transistor power dissipation will increase if operating in its "linear" range of operation rather than being completely cut off or saturated.  This decreases its service life as well as the energy efficiency of the circuit.

%(END_ANSWER)





%(BEGIN_NOTES)

Review with your students what it means for a transistor to be in "cutoff" or in "saturation," if they are not familiar with these terms or if it has been a while since they have studied this.  A clear understanding of this concept is crucial to their being able to understand the efficiency of PWM power control.

%INDEX% Pulse-width modulation (PWM)

%(END_NOTES)


