
%(BEGIN_QUESTION)
% Copyright 2003, Tony R. Kuphaldt, released under the Creative Commons Attribution License (v 1.0)
% This means you may do almost anything with this work of mine, so long as you give me proper credit

What will happen to the counter-EMF of this DC motor if the "field control" resistor value is suddenly decreased (while it is running)?

$$\epsfbox{00551x01.eps}$$

What effect will this change in field excitation do to the operating speed of the motor?

\underbar{file 00551}
%(END_QUESTION)





%(BEGIN_ANSWER)

The counter-EMF will increase, and the motor will slow down.

%(END_ANSWER)





%(BEGIN_NOTES)

In case your students have never heard the word "excitation" used in this context, it would be a good idea to explain it now.  The electrical power used to energize a circuit in which a particular output is expected is sometimes referred to as "excitation."  Bridge circuit power supplies are another example of an "excitation" source.

This is a very important, but often misunderstood, aspect of DC motor control.  While it seems paradoxical that an increase in power applied to the field winding will cause the motor to slow down, it is indeed true.  Ask your students to explain what will happen to the motor speed if the field excitation is {\it weakened}.

%INDEX% Counter-EMF, electric motor
%INDEX% DC electric motor, speed and torque

%(END_NOTES)


