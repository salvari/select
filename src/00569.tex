
%(BEGIN_QUESTION)
% Copyright 2003, Tony R. Kuphaldt, released under the Creative Commons Attribution License (v 1.0)
% This means you may do almost anything with this work of mine, so long as you give me proper credit

In addition to possessing a primary current rating, fuses and circuit breakers also possess an {\it interruption} current rating, usually far in excess of their primary ratings.  For example, a typical 15 amp circuit breaker for 120 volt residential use may have an interruption rating of {\it 10,000 amps} (10 kA)!  Under what conditions could such a circuit ever bear so much current, and why is this rating different than the breaker's primary current rating of 15 amps?

\underbar{file 00569}
%(END_QUESTION)





%(BEGIN_ANSWER)

Short-circuit conditions in electrical power systems may create "transient" currents of extraordinary magnitude.  Overcurrent protection devices will, of course, either "blow" or "trip" under such conditions, but their interruption ratings have nothing to do with the current at which they open.  Rather, a device with an interruption rating of 10,000 amps is certified to be able to {\it stop} a fault current of that magnitude once having opened.

%(END_ANSWER)





%(BEGIN_NOTES)

Students will naturally wonder how a blown fuse or a tripped circuit breaker could {\it not} stop a fault current, since there is no longer a condition of electrical continuity between its terminals once it has opened.  Or is there?  Discuss with your students how an electric current of enormous magnitude could possibly continue to conduct through a blown fuse or tripped circuit breaker.

%INDEX% Interruption current rating, fuse or circuit breaker

%(END_NOTES)


