
%(BEGIN_QUESTION)
% Copyright 2003, Tony R. Kuphaldt, released under the Creative Commons Attribution License (v 1.0)
% This means you may do almost anything with this work of mine, so long as you give me proper credit

Engineers often write the capacitive and inductive reactance formulae in a different way from what you may have seen:

$$X_L = \omega L$$

$$X_C = {1 \over {\omega C}}$$

These equations should look familiar to you, from having seen similar equations containing a term for frequency ($f$).  Given these equations' forms, what is the mathematical definition of $\omega$?  In other words, what combination of variables and constants comprise "$\omega$", and what unit is it properly expressed in?

\underbar{file 00635}
%(END_QUESTION)





%(BEGIN_ANSWER)

$\omega = 2 \pi f$, and it is expressed in units of {\it radians per second}.

%(END_ANSWER)





%(BEGIN_NOTES)

Students who have taken trigonometry should recognize the {\it radian} as a unit for measuring angles.  Discuss with your students why multiplying frequency ($f$, cycles per second) by the constant $2 \pi$ results in the unit changing to "radians per second".

Engineers often refer to $\omega$ as the {\it angular velocity} of an AC system.  Discuss why the term "velocity" is appropriate for $\omega$.

%INDEX% Angular velocity, (lower-case omega)

%(END_NOTES)


