
%(BEGIN_QUESTION)
% Copyright 2003, Tony R. Kuphaldt, released under the Creative Commons Attribution License (v 1.0)
% This means you may do almost anything with this work of mine, so long as you give me proper credit

Imagine you are using a digital voltmeter to measure voltages between pairs of points in a circuit, following the sequence of steps shown in these diagrams:

$$\epsfbox{00348x01.eps}$$

How much voltage would be registered by the voltmeter in each of the steps?  Be sure to include the sign of the DC voltage measured (note the coloring of the voltmeter leads, with the red lead always on the first point denoted in the subscript: $V_{BA}$ = red lead on "B" and black lead on "A"):

\medskip
\item{$\bullet$} $V_{BA} = $
\item{$\bullet$} $V_{CA} = $
\item{$\bullet$} $V_{DA} = $
\item{$\bullet$} $V_{AA} = $
\medskip

\vskip 10pt

Challenge question: how do these voltage measurements {\it prove} Kirchhoff's Voltage Law, where the algebraic sum of all voltages in a loop equals 0 volts?

\underbar{file 00348}
%(END_QUESTION)





%(BEGIN_ANSWER)

\medskip
\item{$\bullet$} $V_{BA} = +10.8$ volts
\item{$\bullet$} $V_{CA} = +18.0$ volts
\item{$\bullet$} $V_{DA} = +36.0$ volts
\item{$\bullet$} $V_{AA} = 0$ volts
\medskip

With each successive step, one more voltage is being added to the previously measured voltage(s), as the meter measures across a larger series chain of resistors.  Ultimately, 

$$V_{AA} = V_{AD} + V_{DC} + V_{CB} + V_{BA} = 0 \hbox{ volts }$$

%(END_ANSWER)





%(BEGIN_NOTES)

Building on the knowledge of series voltages being additive, this example serves as a simple proof of Kirchhoff's Voltage Law.  At the final step of the sequence, the voltage indicated by the meter {\it must} be zero volts because the two test leads are now electrically common to each other.

%INDEX% Kirchhoff's Voltage Law

%(END_NOTES)


