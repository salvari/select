
%(BEGIN_QUESTION)
% Copyright 2003, Tony R. Kuphaldt, released under the Creative Commons Attribution License (v 1.0)
% This means you may do almost anything with this work of mine, so long as you give me proper credit

Amplifier {\it distortion} occurs when its gain varies as a function of the instantaneous signal amplitude.  That is, some parts of the signal waveform become amplified more than others, and this results in the waveform taking on a slightly different shape.

All active devices, bipolar junction transistors included, are {\it nonlinear} to some extent.  This term means that their gain varies throughout their operating ranges.  During the 1920's, an electrical engineer named Harold Black was pondering this problem in the design of telephone system amplifiers.  His solution came to him in a flash of insight one day, as he was commuting from work on a ferry boat.  Explain what his solution to this problem was.

\underbar{file 01244}
%(END_QUESTION)





%(BEGIN_ANSWER)

Harold Black is credited as the first to apply {\it negative feedback} as a solution to the problem of nonlinearity in electronic amplifiers.

\vskip 10pt

Challenge question: since negative feedback has the undesirable effect of diminishing overall amplifier gain, it would seem at first that low distortion and high gain are mutually exclusive design goals for an amplifier.  Is this true, or is there a way to obtain both low distortion and high gain from an amplifier?  If so, how?

%(END_ANSWER)





%(BEGIN_NOTES)

Although Black's solution has been wildly successful in amplifier design, it also finds application in a wide range of processes.  Control theory, for example, where machines are automated in such a way as to stabilize physical variables such as pressure, flow, temperature, etc., depends heavily on negative feedback as an operating principle.

An interesting historical side-note is that Black's 1928 patent application was initially rejected on the grounds that he was trying to submit a perpetual motion device!  The concept of negative feedback in an amplifier circuit was so contrary to established engineering thought at the time, that Black experienced significant resistance to the idea within the engineering community.  At that time the United States patent office was inundated with fraudulent "perpetual motion" claims, and so dismissed Black's invention at first sight.

%INDEX% Distortion, amplifier
%INDEX% Distortion and negative feedback

%(END_NOTES)


