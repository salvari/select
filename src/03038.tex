
%(BEGIN_QUESTION)
% Copyright 2005, Tony R. Kuphaldt, released under the Creative Commons Attribution License (v 1.0)
% This means you may do almost anything with this work of mine, so long as you give me proper credit

Liquid crystal display (LCD) technology used to have very narrow viewing angles.  Anyone who remembers the first LCD displays on portable personal computers will recall how you could only see the display if you viewed it perpendicular to the display surface, or at a very slight angle from perpendicular.

Modern LCD technology is much better, is still not as good as viewing printed paper, the "gold standard" for non-emissive display.  One term frequently used to describe the quality of viewing with regard to angle is {\it Lambertian}.  Define what "Lambertian" means with regard to display surfaces.

\underbar{file 03038}
%(END_QUESTION)





%(BEGIN_ANSWER)

A "Lambertian" surface emits (or reflects) light with an intensity proportional to the cosine of the viewing angle (relative to perpendicular).  Paper is Lambertian in its reflective characteristics, which is one of the reasons it is so easy to read compared to contemporary digital display technologies.

%(END_ANSWER)





%(BEGIN_NOTES)

This question is destined for obsolescence, as Lambertian displays will likely become a reality in the next several years.  But for now (May 2005), it is a term worth defining in the introductory study of display technologies.

An example of an early attempt at full-Lambertian display is the {\it Gyricon} technology developed by Xerox.  Research this and be prepared to discuss it with your students as an example of a novel approach for non-emissive electronic displays.

%INDEX% Lambertian surface

%(END_NOTES)


