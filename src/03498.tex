
%(BEGIN_QUESTION)
% Copyright 2005, Tony R. Kuphaldt, released under the Creative Commons Attribution License (v 1.0)
% This means you may do almost anything with this work of mine, so long as you give me proper credit

When engineers and physicists draw pictures illustrating the magnetic field produced by a straight current-carrying wire, they usually do so using this notation:

$$\epsfbox{03498x01.eps}$$

Explain what the circle-and-dot and circle-and-cross symbols mean, with reference to the {\it right-hand rule}.

\underbar{file 03498}
%(END_QUESTION)





%(BEGIN_ANSWER)

The circles with dots show the magnetic flux vectors coming at you from out of the paper.  The circles with crosses show the magnetic flux vectors going away from you into the paper.  Think of these as images of arrows with points (dots) and fletchings (crosses).

%(END_ANSWER)





%(BEGIN_NOTES)

As a follow-up to this question, you might wish to draw current-carrying wires at different angles, and with current moving in different directions, as practice problems for your students to draw the corresponding arrow points and tails.

%INDEX% Arrow notation, for showing direction of magnetic flux lines
%INDEX% Electromagnetism
%INDEX% Flux, magnetic lines
%INDEX% Magnetic lines of flux
%INDEX% Right-hand rule, electromagnetism

%(END_NOTES)


