
%(BEGIN_QUESTION)
% Copyright 2003, Tony R. Kuphaldt, released under the Creative Commons Attribution License (v 1.0)
% This means you may do almost anything with this work of mine, so long as you give me proper credit

% Uncomment the following line if the question involves calculus at all:
\vbox{\hrule \hbox{\strut \vrule{} $\int f(x) \> dx$ \hskip 5pt {\sl Calculus alert!} \vrule} \hrule}

We know that the formula relating instantaneous voltage and current in an inductor is this:

$$e = L{di \over dt}$$

Knowing this, determine at what points on this sine wave plot for inductor current is the inductor voltage equal to zero, and where the voltage is at its positive and negative peaks.  Then, connect these points to draw the waveform for inductor voltage:

$$\epsfbox{00576x01.eps}$$

How much phase shift (in degrees) is there between the voltage and current waveforms?  Which waveform is leading and which waveform is lagging?

\underbar{file 00576}
%(END_QUESTION)





%(BEGIN_ANSWER)

$$\epsfbox{00576x02.eps}$$

For an inductor, voltage is leading and current is lagging, by a phase shift of 90$^{o}$.

%(END_ANSWER)





%(BEGIN_NOTES)

This question is an excellent application of the calculus concept of the {\it derivative}: relating one function (instantaneous voltage, $e$) with the instantaneous rate-of-change of another function (current, $di \over dt$).

%(END_NOTES)


