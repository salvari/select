
%(BEGIN_QUESTION)
% Copyright 2003, Tony R. Kuphaldt, released under the Creative Commons Attribution License (v 1.0)
% This means you may do almost anything with this work of mine, so long as you give me proper credit

Explain what the {\it ohms-per-volt sensitivity} rating of an analog voltmeter means.  Many analog voltmeters exhibit a sensitivity of 20 k$\Omega$ per volt.  Is it better for a voltmeter to have a high ohms-per-volt rating, or a low ohms-per-volt rating?  Why?

\underbar{file 00721}
%(END_QUESTION)





%(BEGIN_ANSWER)

The "ohms-per-volt" sensitivity rating of a voltmeter is an expression of how many ohms of input resistance the meter has, per range of volt measurement.  The higher this figure is, the better the voltmeter.

%(END_ANSWER)





%(BEGIN_NOTES)

If students have analog voltmeters in their possession (which I greatly encourage them to have), the ohms-per-volt sensitivity rating is often found in a corner of the meter scale, in fine print.  If not, the rating should be found in the user's guide that came with the meter.

%(END_NOTES)


