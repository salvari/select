
%(BEGIN_QUESTION)
% Copyright 2005, Tony R. Kuphaldt, released under the Creative Commons Attribution License (v 1.0)
% This means you may do almost anything with this work of mine, so long as you give me proper credit

Suppose two voltmeters are connected to source of "mains" AC power in a residence, one meter is analog (D'Arsonval PMMC meter movement) while the other is true-RMS digital.  They both register 117 volts while connected to this AC source.  

Suddenly, a large electrical load is turned on somewhere in the system.  This load both reduces the mains voltage and slightly distorts the shape of the waveform.  The overall effect of this is average AC voltage has decreased by 4.5\% from where it was, while RMS AC voltage has decreased by 6\% from where it was.  How much voltage does each voltmeter register now?

\underbar{file 02790}
%(END_QUESTION)





%(BEGIN_ANSWER)

Analog voltmeter now registers: 111.7 volts

\vskip 10pt

True-RMS digital voltmeter now registers: 110 volts

%(END_ANSWER)





%(BEGIN_NOTES)

Students sometimes have difficulty grasping the significance of PMMC meter movements being "average-responding" rather than RMS-responding.  Hopefully, the answer to this question will help illuminate this subject more.

%INDEX% Meter response, peak vs. RMS vs. average

%(END_NOTES)


