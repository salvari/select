
%(BEGIN_QUESTION)
% Copyright 2003, Tony R. Kuphaldt, released under the Creative Commons Attribution License (v 1.0)
% This means you may do almost anything with this work of mine, so long as you give me proper credit

Most modern analog oscilloscopes have the ability to display multiple traces on their screens (dual-trace is the standard), even though the CRT itself used by the 'scope may only have one electron gun, and thus only be able to "paint" one flying dot on the screen at a time.

Oscilloscopes with single-gun display tubes achieve dual-trace capability by way of {\it multiplexing} the two input channels to the same CRT.  There are usually two different modes for this multiplexing, though: {\bf alternate} and {\bf chop}.

Explain how these multiplexing techniques work, and what conditions would prompt you to use the two different multiplexing modes.  I {\it strongly} encourage you to experiment with displaying two different signals on one of these oscilloscopes as your research.  You will likely learn far more from a hands-on exercise than if you were to read about it in a book!

\underbar{file 01344}
%(END_QUESTION)





%(BEGIN_ANSWER)

{\it Chop} is used to display two waveforms when the timebase is set to a slow (low-frequency) setting.  {\it Alternate} is used to display two waveforms when the timebase is set to a fast (high-frequency) setting.

%(END_ANSWER)





%(BEGIN_NOTES)

Don't simply tell your students how the {\it alternate} and {\it chop} facilities of their oscilloscopes work.  Let them experience these two modes of multiplexing directly, with hands-on investigation.  If nothing else, this will provide them with additional practice using oscilloscopes.

%INDEX% Multiplexing, oscilloscope display
%INDEX% Alternate dual-channel mode, oscilloscope display
%INDEX% Chop dual-channel mode, oscilloscope display

%(END_NOTES)


