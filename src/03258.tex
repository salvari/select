
%(BEGIN_QUESTION)
% Copyright 2005, Tony R. Kuphaldt, released under the Creative Commons Attribution License (v 1.0)
% This means you may do almost anything with this work of mine, so long as you give me proper credit

In the early days of electrical metrology, the best way to measure the value of an unknown resistance was to use a {\it bridge} circuit.  Explain how a four-resistor bridge (a "Wheatstone" bridge) could be used to accurately measure an unknown resistance.  What components would this bridge circuit have to be constructed from?  Did the power source have to be precision as well?  Did the voltmeter in the middle of the bridge have to be accurately calibrated?

\underbar{file 03258}
%(END_QUESTION)





%(BEGIN_ANSWER)

Such a bridge circuit needed to be built with three "standard" resistors, having precisely known resistances.  At least one of these resistors needed to be adjustable, with a precision scale attached to it for indication of its resistance at any given position.  The source ("excitation") voltage did not not have to be precise, and the null meter only had to be sensitive and accurate at zero volts.

%(END_ANSWER)





%(BEGIN_NOTES)

In the past I've lectured on Wheatstone bridges only to find a fair number of students completely misunderstanding the concept.  The fact that a bridge circuit balances when the four arms' resistances are in proportion is the easy part.  What these students didn't grasp is {\it how} such a bridge might be used to actually measure an unknown resistance, or why it was not possible for them to build a laboratory-usable Wheatstone bridge circuit with the cheap resistors found in their parts kits.

For example, when asked how such a bridge circuit might be used, it was not unusual to hear a student respond that they would make one of the arms of the bridge adjustable, {\it then measure that arm of the bridge with their digital ohmmeter after having achieved balance} in order to calculate the unknown resistance by ratio.  Though it may seem humorous to an instructor that someone might not realize the sheer existence of a precise ohmmeter would render the bridge circuit obsolete, it nevertheless revealed to me how foreign the concept of a Wheatstone bridge as a resistance {\it measuring} circuit is to students working with modern test equipment.  Such a technological "generation gap" is not to be underestimated!

In order for students to understand the practicality of a Wheatstone bridge, they need to realize that the only affordable calibration artifacts of the time were standard resistors and standard cells (mercury batteries).

%INDEX% Bridge circuit

%(END_NOTES)


