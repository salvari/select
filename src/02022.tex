
%(BEGIN_QUESTION)
% Copyright 2003, Tony R. Kuphaldt, released under the Creative Commons Attribution License (v 1.0)
% This means you may do almost anything with this work of mine, so long as you give me proper credit

$$\epsfbox{02022x01.eps}$$

Note: when testing the frequency response of the tone control circuit, you may need to replace the headphones with a non-inductive resistor of equivalent impedance, and measure $V_{out}$ across it. 

\underbar{file 02022}
\vfil \eject
%(END_QUESTION)





%(BEGIN_ANSWER)

Use circuit simulation software to verify your predicted and measured parameter values.

%(END_ANSWER)





%(BEGIN_NOTES)

A good source of audio signal is the headphone output jack of almost any radio, media player, or other portable audio device.  Students like being able to do a lab exercise that directly relates to technology they're already familiar with.

The higher-impedance the headphones are, the better this circuit works, since the combination of potentiometers and mixing resistors tends to result in a relatively high output impedance.  I have used cheap headphones (32 ohm) with some success, given the following component values:

\medskip
\item{$\bullet$} $C_1$ = 0.1 $\mu$F 
\item{$\bullet$} $L_1$ = 200 mH (actually two 100 mH inductors in series)
\item{$\bullet$} $R_1$ = $R_2$ = 1 k$\Omega$
\item{$\bullet$} $R_{pot1}$ = $R_{pot2}$ = 10 k$\Omega$
\medskip

Some students with limited hearing range have difficulty detecting the changes in tone using 10 k$\Omega$ potentiometers.  You may wish to use 100 k$\Omega$ potentiometers instead for added attenuation.  Operating such a circuit is akin to operating a water faucet with "hot" and "cold" water valves: the two settings together determine temperature {\it and} flow (tone and volume, respectively, for the metaphorically challenged).

An extension of this exercise is to incorporate troubleshooting questions.  Whether using this exercise as a performance assessment or simply as a concept-building lab, you might want to follow up your students' results by asking them to predict the consequences of certain circuit faults.

\vskip 10pt

If you plan to use this exercise as a troubleshooting assessment, I recommend {\it against} inducing the following component failures, as they are difficult to detect when the signal source is music rather than a constant tone of known frequency and amplitude:

\medskip
\goodbreak
\item{$\bullet$} Shorted capacitor ($C_1$)
\item{$\bullet$} Shorted inductor ($L_1$)
\item{$\bullet$} Shorted fixed-value resistors ($R_1$ or $R_2$)
\medskip

%INDEX% Assessment, performance-based (Tone balance control circuit)

%(END_NOTES)


