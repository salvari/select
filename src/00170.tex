
%(BEGIN_QUESTION)
% Copyright 2005, Tony R. Kuphaldt, released under the Creative Commons Attribution License (v 1.0)
% This means you may do almost anything with this work of mine, so long as you give me proper credit

In 1820, the French physicist Andr\'e Marie Amp\`ere discovered that two parallel wires carrying electrical current would either be attracted to one another, or repelled by one another, depending on what directions the two currents were going.  Devise an experiment to reproduce Amp\`ere's results, and determine which directions current must go to produce an attractive versus a repulsive force.

\underbar{file 00170}
%(END_QUESTION)





%(BEGIN_ANSWER)

I won't indicate the answer here, as the whole point of the question is to stimulate you to design and operate an experiment.  Let the facts themselves give you the answer!

%(END_ANSWER)





%(BEGIN_NOTES)

This experiment is well worth performing during discussion time with your students.  There are several ways to demonstrate the effect of electromagnetism in the way that Amp\`ere did back in 1820.  It will be interesting to compare your students' different approaches to this experiment.

One of the habits you should encourage in your students is experimentation to discover or confirm principles.  While researching other peoples' findings is a valid mode to obtaining knowledge, the rewards of primary research (i.e. direct experimentation) are greater and the results more authoritative.

Another point you might want to mention here is the problem-solving technique of {\it altering the problem}.  Instead of envisioning two {\it straight} parallel wires, imagine those wires being bent so they form two parallel {\it coils}.  Now the right-hand rule applies for determining magnetic polarity, and the question of attraction versus repulsion is more easily answered.

%INDEX% Electromagnetism

%(END_NOTES)


