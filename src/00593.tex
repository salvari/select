
%(BEGIN_QUESTION)
% Copyright 2003, Tony R. Kuphaldt, released under the Creative Commons Attribution License (v 1.0)
% This means you may do almost anything with this work of mine, so long as you give me proper credit

Not only do reactive components unavoidably possess some parasitic ("stray") resistance, but they also exhibit parasitic reactance of the {\it opposite} kind.  For instance, inductors are bound to have a small amount of capacitance built-in, and capacitors are bound to have a small amount of inductance built-in.  These effects are not intentional, but they exist anyway.

Describe how a small amount of capacitance comes to exist within an inductor, and how a small amount of inductance comes to exist within a capacitor.  Explain what it is about the construction of these two reactive components that allows the existence of "opposite" characteristics.

\underbar{file 00593}
%(END_QUESTION)





%(BEGIN_ANSWER)

Capacitance exists any time there are two conductors separated by an insulating medium.  Inductance exists any time a magnetic field is permitted to exist around a current-carrying conductor.  Look for each of these conditions within the respective structures of inductors and capacitors to determine where the parasitic effects originate.

%(END_ANSWER)





%(BEGIN_NOTES)

Once students have identified the {\it mechanisms} of parasitic reactances, challenge them with inventing means of minimizing these effects.  This is an especially practical exercise for understanding parasitic inductance in capacitors, which is very undesirable in decoupling capacitors used to stabilize power supply voltages near integrated circuit "chips" on printed circuit boards.  Fortunately, most of the stray inductance in a decoupling capacitor is due to how it's mounted to the board, rather than anything within the structure of the capacitor itself.

%(END_NOTES)


