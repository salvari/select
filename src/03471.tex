
%(BEGIN_QUESTION)
% Copyright 2005, Tony R. Kuphaldt, released under the Creative Commons Attribution License (v 1.0)
% This means you may do almost anything with this work of mine, so long as you give me proper credit

$$\epsfbox{03471x01.eps}$$

\underbar{file 03471}
\vfil \eject
%(END_QUESTION)





%(BEGIN_ANSWER)

Use circuit simulation software to verify your predicted and measured parameter values.

%(END_ANSWER)





%(BEGIN_NOTES)

Use a variable-voltage, regulated power supply for $V_{thevenin}$, and a fixed-voltage supply for $V_{source}$.  Specify standard resistor values, all between 1 k$\Omega$ and 100 k$\Omega$ (1k5, 2k2, 2k7, 3k3, 4k7, 5k1, 6k8, 10k, 22k, 33k, 39k 47k, 68k, etc.) for resistors in the original circuit.  A decade box or potentiometer will suffice for $R_{thevenin}$.

In case it is not already crystal-clear, I want students to build {\it two different circuits} for this exercise: the "original" circuit and also a "Thevenin equivalent" circuit, then plug the exact same load resistor into both circuits (one at a time) to see that the voltage across it is the same in both cases.  Many students seem to struggle with the basic concept of equivalent circuits, and I have found this exercise (once successfully completed) to be excellent for "making it real" to these students.

An extension of this exercise is to incorporate troubleshooting questions.  Whether using this exercise as a performance assessment or simply as a concept-building lab, you might want to follow up your students' results by asking them to predict the consequences of certain circuit faults.

%INDEX% Assessment, performance-based (Thevenin's theorem)

%(END_NOTES)


