
%(BEGIN_QUESTION)
% Copyright 2003, Tony R. Kuphaldt, released under the Creative Commons Attribution License (v 1.0)
% This means you may do almost anything with this work of mine, so long as you give me proper credit

$$\epsfbox{01817x01.eps}$$

\underbar{file 01817}
\vfil \eject
%(END_QUESTION)





%(BEGIN_ANSWER)

Use circuit simulation software to verify your predicted and measured parameter values.

%(END_ANSWER)





%(BEGIN_NOTES)

Use a sine-wave function generator for the AC voltage source.  I recommend against using line-power AC because of strong harmonic frequencies which may be present (due to nonlinear loads operating on the same power circuit).  Specify standard resistor and capacitor values.

If students are to use a multimeter to make their current and voltage measurements, be sure it is capable of accurate measurement at the circuit frequency!  Inexpensive digital multimeters often experience difficulty measuring AC voltage and current toward the high end of the audio-frequency range.

An extension of this exercise is to incorporate troubleshooting questions.  Whether using this exercise as a performance assessment or simply as a concept-building lab, you might want to follow up your students' results by asking them to predict the consequences of certain circuit faults.

%INDEX% Assessment, performance-based (Parallel AC RC circuit)

%(END_NOTES)


