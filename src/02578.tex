
%(BEGIN_QUESTION)
% Copyright 2005, Tony R. Kuphaldt, released under the Creative Commons Attribution License (v 1.0)
% This means you may do almost anything with this work of mine, so long as you give me proper credit

$$\epsfbox{02578x01.eps}$$

\underbar{file 02578}
\vfil \eject
%(END_QUESTION)





%(BEGIN_ANSWER)

Use circuit simulation software to verify your predicted and measured parameter values.

%(END_ANSWER)





%(BEGIN_NOTES)

I recommend setting the function generator output for 1 volt, to make it easier for students to measure the point of "cutoff".  You may set it at some other value, though, if you so choose (or let students set the value themselves when they test the circuit!).

For resistors, I recommend students choose three (3) resistors of equal value if they wish to build the Sallen-Key circuit with a Butterworth response (where $R_2 = {1 \over 2} R_1$).  Resistor $R_1$ will be a single resistor, while resistor $R_2$ will be two resistors connected in parallel.  This generally ensures a more precise 1:2 ratio than choosing individual components.

I also recommend having students use an oscilloscope to measure AC voltage in a circuit such as this, because some digital multimeters have difficulty accurately measuring AC voltage much beyond line frequency range.  I find it particularly helpful to set the oscilloscope to the "X-Y" mode so that it draws a thin line on the screen rather than sweeps across the screen to show an actual waveform.  This makes it easier to measure peak-to-peak voltage.

Values that have proven to work well for this exercise are given here, although of course many other values are possible:

\medskip
\goodbreak
\item{$\bullet$} +V = +12 volts
\item{$\bullet$} -V = -12 volts
\item{$\bullet$} $R_1$ = 10 k$\Omega$
\item{$\bullet$} $R_2$ = 5 k$\Omega$ (actually, two 10 k$\Omega$ resistors in parallel)
\item{$\bullet$} $R_{comp}$ = 10 k$\Omega$
\item{$\bullet$} $C_1$ = 0.002 $\mu$F (actually, two 0.001 $\mu$F capacitors in parallel)
\item{$\bullet$} $C_2$ = 0.002 $\mu$F (actually, two 0.001 $\mu$F capacitors in parallel)
\item{$\bullet$} $U_1$ = one-half of LM1458 dual operational amplifier
\medskip

This combination of components gave a predicted cutoff frequency of 11.25 kHz, with an actual cutoff frequency (not factoring in component tolerances) of 11.11 kHz.

%INDEX% Assessment, performance-based (Sallen-Key active highpass filter)

%(END_NOTES)


