
%(BEGIN_QUESTION)
% Copyright 2003, Tony R. Kuphaldt, released under the Creative Commons Attribution License (v 1.0)
% This means you may do almost anything with this work of mine, so long as you give me proper credit

One thing you will want to do with your completed power supply is subject it to a full-current (1 amp) load, and test the output voltage.  To do this, you will need a load that draws close to 1 amp without overheating or causing other problems.  

A resistor will work fine for this task, but which resistor should you use?  Identify the {\it two} parameters you must be concerned about when selecting a load resistor for the task, and explain exactly how those parameters will be calculated.

\underbar{file 01508}
%(END_QUESTION)





%(BEGIN_ANSWER)

The two parameters are: {\it resistance} (how many ohms), and {\it power rating} (how many watts).  I will leave it to you to show how to calculate each parameter for your particular power supply.

%(END_ANSWER)





%(BEGIN_NOTES)

Even though this is nothing more than an application of Ohm's Law, do not be surprised if students approach you dumbfounded by this question.  There is a large cognitive difference between calculating current and power for a resistor of known value and a voltage source of known voltage, and selecting a resistor based on known current and voltage for a practical test of a power supply.  While studying Ohm's Law in a theoretical context, students become comfortable making calculations on paper, but may not realize just how to apply that same math to a real-world situation.  Or, they may express apprehension when faced with having to make calculations that carry real consequences (such as damaging their power supply!).

%(END_NOTES)


