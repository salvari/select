
%(BEGIN_QUESTION)
% Copyright 2005, Tony R. Kuphaldt, released under the Creative Commons Attribution License (v 1.0)
% This means you may do almost anything with this work of mine, so long as you give me proper credit

A crude measurement circuit for harmonic content of a signal uses a notch filter tuned to the fundamental frequency of the signal being measured.  Examine the following circuit and then explain how you think it would work:

$$\epsfbox{03455x01.eps}$$

\underbar{file 03455}
%(END_QUESTION)





%(BEGIN_ANSWER)

If the signal source is pure (no harmonics), the voltmeter will register nothing (negative infinite decibels) when the switch is flipped to the "test" position.

%(END_ANSWER)





%(BEGIN_NOTES)

This test circuit relies on the assumption that the notch filter is perfect (i.e. that its attenuation in the stop-band is complete).  Since no filter is perfect, it would be a good idea to ask your students what effect they think an imperfect notch filter would have on the validity of the test.  In other words, what will a notch filter that lets a little bit of the fundamental frequency through do to the "test" measurement?

%INDEX% Harmonic distortion, simple measurement circuit

%(END_NOTES)


