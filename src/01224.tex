
%(BEGIN_QUESTION)
% Copyright 2005, Tony R. Kuphaldt, released under the Creative Commons Attribution License (v 1.0)
% This means you may do almost anything with this work of mine, so long as you give me proper credit

Determine the {\it two's complement} of the binary number $01100101_2$.  Explain how you did the conversion, step by step.

Next, determine the two's complement representation of the quantity {\it five} for a digital system where all numbers are represented by four bits, and also for a digital system where all numbers are represented by eight bits (one {\it byte}).  Identify the difference that "word length" (the number of bits allocated to represent quantities in a particular digital system) makes in determining the two's complement of any number.

\underbar{file 01224}
%(END_QUESTION)





%(BEGIN_ANSWER)

The two's complement of $01100101$ is $10011011$.

\vskip 10pt

The two's complement of five is $1011$ in the four-bit system.  It is $11111011$ in the eight-bit system. 

%(END_ANSWER)





%(BEGIN_NOTES)

The point about word-length is extremely important.  One cannot arrive at a definite two's complement for any number unless the word length is first known!

%INDEX% Two's complement, importance of word length

%(END_NOTES)


