
%(BEGIN_QUESTION)
% Copyright 2003, Tony R. Kuphaldt, released under the Creative Commons Attribution License (v 1.0)
% This means you may do almost anything with this work of mine, so long as you give me proper credit

Shown here is the time-current curve for a {\it dual-element} fuse.  {\it Thermal-magnetic} circuit breakers exhibit similar time-current curves:

$$\epsfbox{00333x01.eps}$$

Based on this curve, what do you think the purpose of a "dual-element" fuse or "thermal-magnetic" circuit breaker is?  Why would this style of overcurrent protection device be chosen over a "normal" fuse or circuit breaker?

\underbar{file 00333}
%(END_QUESTION)





%(BEGIN_ANSWER)

Such overcurrent protection devices combine the properties of "time-overcurrent" and "instantaneous overcurrent" protection in a single device, for protection against different types of damage resulting from different types of overcurrent conditions.

%(END_ANSWER)





%(BEGIN_NOTES)

Obviously, not all overcurrent conditions are the same.  Ask your students to describe circuit faults that would cause slight overcurrent, versus faults that would cause extreme overcurrent in a circuit, and discuss the destructive consequences of each type of condition.

%INDEX% Fuse curve
%INDEX% Time-current curve, circuit breaker

%(END_NOTES)


