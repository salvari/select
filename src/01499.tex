
%(BEGIN_QUESTION)
% Copyright 2003, Tony R. Kuphaldt, released under the Creative Commons Attribution License (v 1.0)
% This means you may do almost anything with this work of mine, so long as you give me proper credit

The most important component of your voltmeter will be the galvanometer movement itself.  Identify some of the ideal characteristics of this device.  For example, what should its full-scale current rating be (very high or very low), and why?

\underbar{file 01499}
%(END_QUESTION)





%(BEGIN_ANSWER)

Look for a galvanometer with the lowest full-scale current rating possible (you should be able to obtain meter movements with $I_{FS}$ ratings as low as 50 $\mu$A with little difficulty).  Ruggedness is another ideal characteristic, as is a mirrored scale to avoid parallax errors.

%(END_ANSWER)





%(BEGIN_NOTES)

I have found a good source of meter movements for students' own voltmeter projects to be inexpensive analog multimeters.  These may be obtained from most tool stores in the United States for \$30 or less (2004 prices), and come with multiple scales on the faces, plus mirror bands for anti-parallax reading.  General-purpose galvanometer movements may be obtained through wholesale electronics suppliers, but generally not at the same (low) cost as these consumer-grade meters.

It may seem like a shame to purchase a multimeter only to tear it apart and re-build it as a simple voltmeter, but the purpose here is to learn how to design range resistor circuitry and calibrate a meter once it's built.

%(END_NOTES)


