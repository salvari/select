
%(BEGIN_QUESTION)
% Copyright 2003, Tony R. Kuphaldt, released under the Creative Commons Attribution License (v 1.0)
% This means you may do almost anything with this work of mine, so long as you give me proper credit

{\it Rotary encoders} are electromechanical devices used to convert an angular position (shaft rotation) into a digital signal.  The simplest form of rotary encoder uses a slotted wheel with a single LED/photodetector pair to generate pulses as the wheel turns:

$$\epsfbox{01236x01.eps}$$

Some rotary encoder designs have multiple-bit outputs, with each LED/photodetector pair reading a different "track" of slots in the disk:

$$\epsfbox{01236x02.eps}$$

In the illustration shown above, identify which LED/photodetector pairs represent the MSB (Most Significant Bit) and LSB (Least Significant Bit) of the binary output.  Also, identify which direction the wheel must turn in order to produce an increasing count.

Note: assume that the darkest areas on the illustration represent slots cut through the disk, while the grey areas represent parts of the disk that are opaque.

\underbar{file 01236}
%(END_QUESTION)





%(BEGIN_ANSWER)

I'll let you figure out the MSB, LSB, and up-count direction on your own!  It isn't difficult to do if you have mastered counting in binary.

%(END_ANSWER)





%(BEGIN_NOTES)

Ask your students to brainstorm possible applications for rotary encoders.  Where might we use such a device?  Also, ask them to contrast the two encoder types (1 bit versus 3-bit) shown in the question.  What applications might demand the 3-bit, versus only require a 1-bit encoder?

%INDEX% Encoder, rotary
%INDEX% Rotary encoder

%(END_NOTES)


