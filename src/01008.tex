
%(BEGIN_QUESTION)
% Copyright 2003, Tony R. Kuphaldt, released under the Creative Commons Attribution License (v 1.0)
% This means you may do almost anything with this work of mine, so long as you give me proper credit

% Uncomment the following line if the question involves calculus at all:
\vbox{\hrule \hbox{\strut \vrule{} $\int f(x) \> dx$ \hskip 5pt {\sl Calculus alert!} \vrule} \hrule}

How much current will go "through" the capacitor in this op-amp circuit, and what effect does this have on the output voltage?

$$\epsfbox{01008x01.eps}$$

\underbar{file 01008}
%(END_QUESTION)





%(BEGIN_ANSWER)

$I_C =$ 1.818 mA

\vskip 10pt

This circuit is an {\it integrator}: its output voltage changes over time at a rate proportional to the input voltage magnitude.

\vskip 10pt

Follow-up question: what is the output voltage rate-of-change over time (${dv \over dt}$) for the circuit shown in the question?

%(END_ANSWER)





%(BEGIN_NOTES)

This question is a good review of capacitor theory (relating voltage and current with regard to a capacitor), as well as an introduction to how op-amp circuits can perform calculus functions.

Challenge your students to calculate the output ${dv \over dt}$ without using a calculator!

%INDEX% Integrator circuit, op-amp

%(END_NOTES)


