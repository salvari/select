
%(BEGIN_QUESTION)
% Copyright 2005, Tony R. Kuphaldt, released under the Creative Commons Attribution License (v 1.0)
% This means you may do almost anything with this work of mine, so long as you give me proper credit

Examine the following American Wire Gauge table.  Please note that most of the odd-numbered gauges have been omitted, because the even-numbered gauges tend to be more common:

% No blank lines allowed between lines of an \halign structure!
% I use comments (%) instead, so that TeX doesn't choke.

$$\vbox{\offinterlineskip
\halign{\strut
\vrule \quad\hfil # \ \hfil & 
\vrule \quad\hfil # \ \hfil & 
\vrule \quad\hfil # \ \hfil \vrule \cr
\noalign{\hrule}
%
Gauge \# & Diameter (inches) & Area (circular mils) \cr
%
\noalign{\hrule}
%
4/0 & 0.4600 & 211,600 \cr
%
\noalign{\hrule}
%
3/0 & 0.4100 & 168,100 \cr
%
\noalign{\hrule}
%
2/0 & 0.3650 & 133,225 \cr
%
\noalign{\hrule}
%
1/0 & 0.3250 & 105,625 \cr
%
\noalign{\hrule}
%
1 & 0.2890 & 83,521 \cr
%
\noalign{\hrule}
%
2 & 0.2580 & 66,564 \cr
%
\noalign{\hrule}
%
4 & 0.2040 & 41,616 \cr
%
\noalign{\hrule}
%
6 & 0.1620 & 26,244 \cr
%
\noalign{\hrule}
%
8 & 0.1280 & 16,384 \cr
%
\noalign{\hrule}
%
10 & 0.1020 & 10,404 \cr
%
\noalign{\hrule}
%
12 & 0.0810 & 6,561 \cr
%
\noalign{\hrule}
%
14 & 0.0640 & 4,096 \cr
%
\noalign{\hrule}
%
16 & 0.0510 & 2,601 \cr
%
\noalign{\hrule}
%
18 & 0.0400 & 1,600 \cr
%
\noalign{\hrule}
%
20 & 0.0320 & 1,024 \cr
%
\noalign{\hrule}
%
22 & 0.0253 & 640.1 \cr
%
\noalign{\hrule}
} % End of \halign 
}$$ % End of \vbox

How many gauge numbers must you increase to (approximately) double the diameter of any given wire gauge?  What effect does the doubling of diameter have on the cross-sectional {\it area} of the wire?

\underbar{file 03380}
%(END_QUESTION)





%(BEGIN_ANSWER)

Wire diameter approximately doubles once for every six wire gauge sizes.  Cross-sectional area {\it quadruples} for the same wire gauge interval.

%(END_ANSWER)





%(BEGIN_NOTES)

Wire gauge numbers and diameters for this table were taken from table 2-85 of the {\it American Electrician's Handbook} (eleventh edition) by Terrell Croft and Wilford Summers.  Area in circular mils for each AWG size was calculated from the given diameter.

%INDEX% Circular mils, equivalence to wire gauge
%INDEX% Table, wire gauge
%INDEX% Wire gauge, equivalence to circular mils
%INDEX% Wire gauge table

%(END_NOTES)


