
%(BEGIN_QUESTION)
% Copyright 2003, Tony R. Kuphaldt, released under the Creative Commons Attribution License (v 1.0)
% This means you may do almost anything with this work of mine, so long as you give me proper credit

In the very early days of radio communication, a popular style of transmitter was the {\it spark gap} circuit.  Explain how this circuit functioned, and why it is no longer used as a practical transmitter design.

\underbar{file 01532}
%(END_QUESTION)





%(BEGIN_ANSWER)

"Spark gap" transmitter circuits were built very much like you would expect, from their names: an air gap through which a high-voltage electric spark jumped.  Because the pulse durations of the sparks were so short, the equivalent output frequencies spanned a very wide range, ultimately rendering this technology impractical due to interference between multiple transmitters.

%(END_ANSWER)





%(BEGIN_NOTES)

Anyone who has ever heard "popping" noises on an AM radio produced by a (pulsed) electric fence of the type used around farms to keep animals from wandering off will understand how spark-gap transmitters broadcast across a large range of frequencies.

This question could very well lead into a fascinating discussion on Fourier transforms, if your students are so inclined.  According to Fourier theory, the shorter the duration of a pulse, the broader its frequency range.  The product of uncertainties for the pulse's location in time and its frequency is equal to or greater than a certain constant.  Theoretically, a pulse of infinitesimal width would encompass an infinitely wide (infinitely uncertain) range of frequencies.

Incidentally, the math behind this is precisely the same as for Heisenberg's Uncertainty Principle: that quantum physics theory which states the certainty of a particle's position is inversely proportional to the certainty of its momentum, and visa-versa.  Contrary to popular belief, this phenomenon is not an artifact induced by the act of measuring either position or momentum.  It is not as though one could obtain perfectly precise measurements of position {\it and} momentum if only one had access to the perfect measuring device(s).  Rather, this Principle is a fundamental limit on the certainty {\it possessed} by a particle with regard to its position and momentum.  Likewise, an infinitesimal pulse {\it has no definite frequency}.

%INDEX% Spark gap radio transmitter
%INDEX% Radio transmitter, spark gap
%INDEX% Uncertainty Principle, time versus frequency

%(END_NOTES)


