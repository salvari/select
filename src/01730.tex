
%(BEGIN_QUESTION)
% Copyright 2003, Tony R. Kuphaldt, released under the Creative Commons Attribution License (v 1.0)
% This means you may do almost anything with this work of mine, so long as you give me proper credit

A very old but highly accurate technique for measuring voltage is to use a sensitive meter (a {\it galvanometer}) as a "balance" indicator between two voltage sources: the unknown source and a calibrated, adjustable source called the {\it standard}:

$$\epsfbox{01730x01.eps}$$

The technique is to adjust the standard voltage source until the galvanometer registers exactly zero.  When this condition is met, the system is said to be "balanced," indicating that the two voltage sources are equal in magnitude to each other.  This voltage magnitude is then read from the calibrated adjustment dial on the standard source.

Explain how this voltage measurement technique exploits Kirchhoff's Voltage Law, which states that the algebraic sum of all voltages in a loop must equal zero:

$$E_{\Sigma (loop)} = 0$$

\underbar{file 01730}
%(END_QUESTION)





%(BEGIN_ANSWER)

I'll leave the answer for you to demonstrate!

%(END_ANSWER)





%(BEGIN_NOTES)

Potentiometric voltage measurement is not only practical, but for quite a while in the early field of electrical measurements it was the {\it only} way to accurately measure voltage from "weak" (high-impedance) sources.  It is still used today in electrical metrology, and serves as a simple application of KVL for students to examine.

%INDEX% Galvanometer
%INDEX% Potentiometric (null-balance) voltage measurement

%(END_NOTES)


