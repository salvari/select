
%(BEGIN_QUESTION)
% Copyright 2006, Tony R. Kuphaldt, released under the Creative Commons Attribution License (v 1.0)
% This means you may do almost anything with this work of mine, so long as you give me proper credit

Suppose a person is more familiar with conventional flow notation than electron flow notation.  If this person find themselves in a situation where they {\it must} draw the direction of current according to electron flow notation, what advice would you give them for making the transition.

\underbar{file 04079}
%(END_QUESTION)





%(BEGIN_ANSWER)

Begin by drawing all currents in the more familiar notation of conventional flow, then reverse each and every arrow!

%(END_ANSWER)





%(BEGIN_NOTES)

A good strategy might be to use a pencil and {\it lightly} draw the arrows in the direction of conventional flow, then over-draw those arrows in the reverse direction using more hand pressure (making a darker line).

It should go without saying that this technique works just as well for the person who is more comfortable with electron flow notation, but must switch to conventional flow for some reason.

%INDEX% Conventional flow versus electron flow
%INDEX% Electron flow versus conventional flow

%(END_NOTES)


