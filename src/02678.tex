
%(BEGIN_QUESTION)
% Copyright 2005, Tony R. Kuphaldt, released under the Creative Commons Attribution License (v 1.0)
% This means you may do almost anything with this work of mine, so long as you give me proper credit

Electronic calculators with logarithm capability have at least two different types of logarithms: {\it common} logarithm and {\it natural} logarithm, symbolized as "$\log$" and "$\ln$", respectively.  Explain what the difference is between these two types of logarithms.

\underbar{file 02678}
%(END_QUESTION)





%(BEGIN_ANSWER)

The common logarithm function assumes a "base" value of ten, whereas the natural logarithm assumes a base value of $e$ (Euler's constant).

\vskip 10pt

Follow-up question: what is the approximate value of $e$?  How can you get your calculator to give you the answer (rather than looking it up in a math book?

%(END_ANSWER)





%(BEGIN_NOTES)

Some calculators, of course, allow you to extract the logarithm of any number to any base.  Here, I simply want students to become familiar with the two logarithm functions available on the most basic scientific calculators.

Note that some calculators will show just enough digits of $e$ to give the false impression that they repeat (ten digits: $e$ = 2.718281828).  If anyone suggests that $e$ is a (rational) repeating decimal number, correct this misunderstanding by telling them it is irrational just like $\pi$.

%INDEX% Logarithm, common versus natural

%(END_NOTES)


