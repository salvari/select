
%(BEGIN_QUESTION)
% Copyright 2003, Tony R. Kuphaldt, released under the Creative Commons Attribution License (v 1.0)
% This means you may do almost anything with this work of mine, so long as you give me proper credit

The numeration system we use in our daily lives is called {\it base ten}, also called {\it decimal} or {\it denary}.  What, exactly, does "base ten" mean?

Given the following base-ten number, identify which digits occupy the "one's place," "ten's place," "hundred's place," and "thousand's place," respectively:

$$5,183$$

\underbar{file 01195}
%(END_QUESTION)





%(BEGIN_ANSWER)

"Base ten" means that numbers are represented by combinations of symbols (ciphers), of which there are only ten (0, 1, 2, 3, 4, 5, 6, 7, 8, and 9).

\vskip 10pt

Analyzing the number 5,183:

\medskip
\item{$\bullet$} One's place: {\bf 3}
\item{$\bullet$} Ten's place: {\bf 8}
\item{$\bullet$} Hundred's place: {\bf 1}
\item{$\bullet$} Thousand's place: {\bf 5}
\medskip

%(END_ANSWER)





%(BEGIN_NOTES)

The "base" of our numeration system is something people usually don't think much about -- it is simply taken for granted.  The purpose of this question is to help students realize what numerical symbols actually mean, in preparation for understanding other systems of numeration.

%INDEX% Base-10 numeration system
%INDEX% Place weighting, base-10 numeration system

%(END_NOTES)


