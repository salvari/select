
%(BEGIN_QUESTION)
% Copyright 2004, Tony R. Kuphaldt, released under the Creative Commons Attribution License (v 1.0)
% This means you may do almost anything with this work of mine, so long as you give me proper credit

Based on what you know about bipolar junction transistors, what will the collector current do (increase, decrease, or remain the same) if the variable resistor's resistance is decreased?  The small voltage source (0.7 volts) is just enough to make the transistor conduct, but not enough to fully saturate it.

$$\epsfbox{02165x01.eps}$$

From the variable resistor's perspective, what does the rest of the transistor circuit "look" like?

\underbar{file 02165}
%(END_QUESTION)





%(BEGIN_ANSWER)

The collector current will remain (approximately) the same as the variable resistance is changed.  In this manner, the transistor circuit "looks" like a {\it current source} to the variable resistor.

%(END_ANSWER)





%(BEGIN_NOTES)

This question is really nothing more than review of a transistor's characteristic curves.  You might want to ask your students to relate this circuit's behavior to the common characteristic curves shown in textbooks for bipolar junction transistors.  What portion of the characteristic curve is this transistor operating in while it regulates current?

%INDEX% BJT, as current regulator

%(END_NOTES)


