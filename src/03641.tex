
%(BEGIN_QUESTION)
% Copyright 2005, Tony R. Kuphaldt, released under the Creative Commons Attribution License (v 1.0)
% This means you may do almost anything with this work of mine, so long as you give me proper credit

A technician uses a stroboscope to "freeze" the rotation of a motorized wheel with a mark on it.  With the mark appearing to stand still, the technician measures the strobe frequency at 10 Hz.  Based on this measurement the technician determines the wheel's speed to be 600 RPM:

$$\epsfbox{03641x01.eps}$$

A second technician comes along and disagrees with the first.  According to the second, more testing is required before it can be sure that the wheel's speed is actually 600 RPM.  The first technician is perplexed: how can the speed be anything {\it but} 600 RPM, if 10 Hz is the strobe frequency where the mark is seen to "stand still?"

\underbar{file 03641}
%(END_QUESTION)





%(BEGIN_ANSWER)

The wheel may be turning at 600 RPM, or 1200 RPM, or 1800 RPM, or 2400 RPM or . . .

\vskip 10pt

Follow-up question: what must the technician do with the stroboscope to be absolutely sure of the wheel's speed?

%(END_ANSWER)





%(BEGIN_NOTES)

Ask your students to explain why the mark would appear to stand still at so many different frequencies.  Shouldn't there be only one strobe frequency where the mark appears to stand still?

%INDEX% Stroboscope, operation of

%(END_NOTES)


