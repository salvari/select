
%(BEGIN_QUESTION)
% Copyright 2004, Tony R. Kuphaldt, released under the Creative Commons Attribution License (v 1.0)
% This means you may do almost anything with this work of mine, so long as you give me proper credit

Describe what happens to the collector current of the transistor as the variable resistor's value is changed:

$$\epsfbox{02166x01.eps}$$

Hint: it is helpful to remember that the voltage drop across a PN junction is not exactly constant as the current through it varies.  There is a nonlinear relationship between diode voltage drop ($V_D$) and diode current ($I_D$) as described by the {\it diode equation}:

$$I_D = I_S (e^{\left({qV_D \over NkT}\right)} - 1)$$

\underbar{file 02166}
%(END_QUESTION)





%(BEGIN_ANSWER)

The transistor's collector current rises and falls with the diode's current, as dictated by the variable resistor.  Ideally, the transistor collector current {\it precisely matches} the diode's current.

%(END_ANSWER)





%(BEGIN_NOTES)

This circuit is really the beginning of a {\it current mirror}.  I have found this to be an excellent starting point for student learning on linear transistor operation, as well as a good practical introduction of current regulation circuits.  Once students recognize that bipolar transistors are essentially voltage-controlled current regulators (albeit very nonlinear!), they are ready to comprehend their application as signal amplifiers.

%INDEX% BJT, as current regulator
%INDEX% Current mirror circuit, BJT
%INDEX% Diode equation

%(END_NOTES)


