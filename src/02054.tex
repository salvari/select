
%(BEGIN_QUESTION)
% Copyright 2003, Tony R. Kuphaldt, released under the Creative Commons Attribution License (v 1.0)
% This means you may do almost anything with this work of mine, so long as you give me proper credit

A {\it field-effect transistor} is made from a continuous "channel" of doped semiconductor material, either N or P type.  In the illustration shown below, the channel is N-type:

$$\epsfbox{02054x01.eps}$$

Trace the direction of current through the channel if a voltage is applied across the length as shown in the next illustration.  Determine what type of charge carriers (electrons or holes) constitute the majority of the channel current:

$$\epsfbox{02054x02.eps}$$

The next step in the fabrication of a field-effect transistor is to implant regions of oppositely-doped semiconductor on either side of the channel as shown in the next illustration.  These two regions are connected together by wire, and called the "gate" of the transistor:

$$\epsfbox{02054x03.eps}$$

Show how the presence of these "gate" regions in the channel influence the flow of charge carriers.  Use small arrows if necessary to show how the charge carriers move through the channel and past the gate regions of the transistor.  Finally, label which terminal of the transistor is the {\it source} and which terminal is the {\it drain}, based on the type of majority charge carrier present in the channel and the direction of those charge carriers' motion.

\underbar{file 02054}
%(END_QUESTION)





%(BEGIN_ANSWER)

The majority charge carriers in this transistor's channel are electrons, not holes.  Thus, the arrows drawn in the following diagrams point in the direction of electron flow:

$$\epsfbox{02054x04.eps}$$

This makes the right-hand terminal the {\it source} and the left-hand terminal the {\it drain}.

\vskip 10pt

Follow-up question: explain why the charge carriers avoid traversing the PN junctions formed by the gate-channel interfaces.  In other words, explain why we do {\it not} see this happening:

$$\epsfbox{02054x05.eps}$$

%(END_ANSWER)





%(BEGIN_NOTES)

Students typically find junction field-effect transistors much easier to understand than bipolar junction transistors, because there is less understanding of energy levels required to grasp the operation of JFETs than what is required to comprehend the operation of BJTs.  Still, students need to understand how different charge carriers move through N- and P-type semiconductors, and what the significance of a depletion region is.

%INDEX% FET, distinction between drain and source terminals
%INDEX% JFET, construction of 

%(END_NOTES)


