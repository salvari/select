
%(BEGIN_QUESTION)
% Copyright 2004, Tony R. Kuphaldt, released under the Creative Commons Attribution License (v 1.0)
% This means you may do almost anything with this work of mine, so long as you give me proper credit

If the source voltage in this circuit is assumed to be the phase reference (that is, the voltage is defined to be at an angle of 0 degrees), determine the relative phase angles of each current in this parallel circuit:

$$\epsfbox{02112x01.eps}$$

\medskip
\item{$\bullet$} $\Theta_{I(R)}$ =
\item{$\bullet$} $\Theta_{I(C)}$ =
\item{$\bullet$} $\Theta_{I(total)}$ =
\medskip

\underbar{file 02112}
%(END_QUESTION)





%(BEGIN_ANSWER)

\medskip
\goodbreak
\item{$\bullet$} $\Theta_{I(R)}$ = 0$^{o}$
\item{$\bullet$} $\Theta_{I(C)}$ = 90$^{o}$
\item{$\bullet$} $\Theta_{I(total)}$ = some positive angle between 0$^{o}$ and 90$^{o}$, exclusive
\medskip

%(END_ANSWER)





%(BEGIN_NOTES)

Some students will be confused about the positive phase angles, since this is a capacitive circuit and they have learned to associate negative angles with capacitors.  It is important for these students to realize, though, that the negative angles they immediately associate with capacitors are in reference to {\it impedance} and not necessarily to other variables in the circuit!

%INDEX% Phase angle, current in an RC circuit

%(END_NOTES)


