
%(BEGIN_QUESTION)
% Copyright 2003, Tony R. Kuphaldt, released under the Creative Commons Attribution License (v 1.0)
% This means you may do almost anything with this work of mine, so long as you give me proper credit

$$\epsfbox{01989x01.eps}$$

\underbar{file 01989}
\vfil \eject
%(END_QUESTION)





%(BEGIN_ANSWER)

Use circuit simulation software to verify your predicted and measured parameter values.

\vskip 10pt

You might be surprised to find that $L_{total} \neq L_1 + L_2$.  This is due to the {\it mutual inductance} between inductors $L_1$ and $L_2$.

%(END_ANSWER)





%(BEGIN_NOTES)

In case students don't have access to a pair of inductors on a common core, they may either make their own by winding wire around a long ferromagnetic core, or use a center-tapped inductor (or transformer winding).  The latter solution is probably the easiest:

$$\epsfbox{01989x02.eps}$$

Inexpensive audio output transformers (with center-tapped 1000 $\Omega$ primary windings) work very well for this.  Your students' parts kits should contain at least one of these transformers anyway if they are to do audio coupling experiments later.

\vskip 10pt

You will need an inductance meter in your lab to do this exercise.  If you don't have one, you should get one right away!

%INDEX% Assessment, performance-based (Series coupled inductors)

%(END_NOTES)


