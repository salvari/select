
%(BEGIN_QUESTION)
% Copyright 2005, Tony R. Kuphaldt, released under the Creative Commons Attribution License (v 1.0)
% This means you may do almost anything with this work of mine, so long as you give me proper credit

The following DC-DC converter circuit is called a {\it forward converter}.  It is called this because the energy transfer from input to output occurs while the transistor is conducting, not while it is off.  Verify this feature of the circuit by tracing current through all portions of it while the transistor is on:

$$\epsfbox{03724x01.eps}$$

Now, trace current through the circuit while the transistor is off, and explain the purpose of the {\it reset} winding in the transformer:

$$\epsfbox{03724x02.eps}$$

\underbar{file 03724}
%(END_QUESTION)





%(BEGIN_ANSWER)

In the following schematics, conventional flow notation has been used to denote direction of currents:

$$\epsfbox{03724x03.eps}$$

$$\epsfbox{03724x04.eps}$$

The purpose of the reset winding is to rid the transformer core of stored energy during the off cycle.  If this were not done, the transformer core's magnetic flux levels would reach saturation after just a few on/off cycles of the transistor.

%(END_ANSWER)





%(BEGIN_NOTES)

This question is a great review of the "dot convention" used in transformer schematic symbols.

%INDEX% Forward converter circuit
%INDEX% Reset winding, forward converter circuit

%(END_NOTES)


