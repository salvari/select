
%(BEGIN_QUESTION)
% Copyright 2005, Tony R. Kuphaldt, released under the Creative Commons Attribution License (v 1.0)
% This means you may do almost anything with this work of mine, so long as you give me proper credit

Students new to the study of transistors often have difficulty remembering the proper directions of currents through bipolar junction transistors, since there are three different currents ($I_B$, $I_C$, $I_E$) and they must "mesh" through the transistor in a particular way.

Draw the proper current directions for each of these transistors, and explain how you are able to remember the correct directions they go:

$$\epsfbox{02406x01.eps}$$

\underbar{file 02406}
%(END_QUESTION)





%(BEGIN_ANSWER)

$$\epsfbox{02406x02.eps}$$

%(END_ANSWER)





%(BEGIN_NOTES)

Rather than present a "rule of thumb" to use in remembering the proper current directions, I opt to let the students figure this out on their own.  An important element of this should be the mathematics of BJT currents, primarily this equation:

$$I_E = I_C + I_B$$

This relationship, combined with Kirchhoff's Current Law, should provide all the help necessary to formulate a rule.

%INDEX% BJT, current directions
%INDEX% Current through a BJT, directions of each

%(END_NOTES)


