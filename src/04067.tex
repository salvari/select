
%(BEGIN_QUESTION)
% Copyright 2006, Tony R. Kuphaldt, released under the Creative Commons Attribution License (v 1.0)
% This means you may do almost anything with this work of mine, so long as you give me proper credit

A windmill-powered generator and a battery work together to supply DC power to a light bulb.  Calculate the amount of current through each of these three components, given the values shown in the schematic diagram.  Assume internal resistances of the generator and battery to be negligible:

$$\epsfbox{04067x01.eps}$$

$I_{batt}$ = \hskip 100pt $I_{bulb}$ = \hskip 100pt $I_{gen}$ =

\vskip 10pt

\underbar{file 04067}
%(END_QUESTION)





%(BEGIN_ANSWER)

$I_{batt}$ = 3.992 A (charging) \hskip 75pt $I_{bulb}$ = 3.013 A \hskip 75pt $I_{gen}$ = 7.006 A

%(END_ANSWER)





%(BEGIN_NOTES)

Discuss your students' procedures with them in class, so that all may see how a problem such as this may be solved.  Superposition theorem is probably the most direct way of solving for all currents, although students could apply Kirchhoff's Laws if they are familiar with solving linear systems of equations.

%INDEX% Superposition theorem, quantitative

%(END_NOTES)


