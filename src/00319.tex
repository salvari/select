
%(BEGIN_QUESTION)
% Copyright 2003, Tony R. Kuphaldt, released under the Creative Commons Attribution License (v 1.0)
% This means you may do almost anything with this work of mine, so long as you give me proper credit

Special safety devices called {\it Ground Fault Current Interrupters}, or {\it GFCI's}, reduce the risk of shock hazard in electrical systems even where there is no ground conductor.  Explain the operating principle of a GFCI.  How are they able to sense a "ground fault" condition, so as to automatically turn off power to a receptacle?

\underbar{file 00319}
%(END_QUESTION)





%(BEGIN_ANSWER)

"GFCI" devices work on the principle of {\it differential current measurement} as a means to sense ground faults.

%(END_ANSWER)





%(BEGIN_NOTES)

Be sure to discuss with your students what "differential current measurement" means, rather than assume they all researched the answer in greater detail than what is provided here.  This really is a clever way to detect the presence of a ground fault!

You should point out to your students that GFCI power receptacles are commonly installed in "wet" areas of residences, such as in the bathroom and outside, where the hazard of ground-fault electric shock is maximized by the presence of water.

This question also provides an opportunity to discuss why the presence of water at a point of bodily contact increases the severity of electric shock.

%INDEX% GFCI
%INDEX% Ground Fault Current Interrupter

%(END_NOTES)


