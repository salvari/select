
%(BEGIN_QUESTION)
% Copyright 2005, Tony R. Kuphaldt, released under the Creative Commons Attribution License (v 1.0)
% This means you may do almost anything with this work of mine, so long as you give me proper credit

A rectangular building foundation with an area of 18,500 square feet measures 100 feet along one side.  You need to lay in a diagonal run of conduit from one corner of the foundation to the other.  Calculate how much conduit you will need to make the run:

$$\epsfbox{03275x01.eps}$$

Also, write an equation for calculating this conduit run length ($L$) given the rectangular area ($A$) and the length of one side ($x$).

\underbar{file 03275}
%(END_QUESTION)





%(BEGIN_ANSWER)

Conduit run = 210 feet, 3.6 inches from corner to corner.

\vskip 10pt

Note: the following equation is not the only form possible for calculating the diagonal length.  Do not be worried if your equation does not look exactly like this!

$$L = {{\sqrt{x^4 + A^2}} \over x}$$

%(END_ANSWER)





%(BEGIN_NOTES)

Determining the necessary length of conduit for this question involves both the Pythagorean theorem and simple geometry.

Most students will probably arrive at this form for their diagonal length equation:

$$L = \sqrt{x^2 + \left({A \over x}\right)^2}$$

While this is perfectly correct, it is an interesting exercise to have students convert the equation from this (simple) form to that given in the answer.  It is also a very practical question, as equations given in reference books do not always follow the most direct form, but rather are often written in such a way as to look more esthetically pleasing.  The simple and direct form of the equation shown here (in the Notes section) looks "ugly" due to the fraction inside the radicand.

%INDEX% Trigonometry, Pythagorean theorem

%(END_NOTES)


