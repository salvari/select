
%(BEGIN_QUESTION)
% Copyright 2003, Tony R. Kuphaldt, released under the Creative Commons Attribution License (v 1.0)
% This means you may do almost anything with this work of mine, so long as you give me proper credit

Usually, propagation delay is considered an undesirable characteristic of logic gates, which we simply have to live with.  Other times, it is a useful, even necessary, trait.  Take for example this circuit:

$$\epsfbox{01362x01.eps}$$

If the gates constituting this circuit had zero propagation delay, it would perform no useful function at all.  To verify this sad fact, analyze its steady-state response to a "low" input signal, then to a "high" input signal.  What state is the AND gate's output always in?

Now, consider propagation delay in your analysis by completing a timing diagram for each gate's output, as the input signal transitions from low to high, then from high to low:

$$\epsfbox{01362x02.eps}$$

What do you notice about the state of the AND gate's output now?

\underbar{file 01362}
%(END_QUESTION)





%(BEGIN_ANSWER)

$$\epsfbox{01362x03.eps}$$

\vskip 10pt

Follow-up question: describe exactly what conditions are necessary to obtain a "high" signal from the output of this circuit, and what determines the duration of this "high" pulse.

%(END_ANSWER)





%(BEGIN_NOTES)

Tell your students that this circuit is a special type of {\it one-shot}, outputting a single pulse of limited duration for each leading-edge transition of the input signal.

Ask your students what we might do if we wanted to make the output pulse of this one-shot circuit longer (or shorter).

%INDEX% One-shot circuit, exploiting gate propagation delays
%INDEX% Propagation delay, gate circuits
%INDEX% Timing diagram showing propagation delays

%(END_NOTES)


