
%(BEGIN_QUESTION)
% Copyright 2003, Tony R. Kuphaldt, released under the Creative Commons Attribution License (v 1.0)
% This means you may do almost anything with this work of mine, so long as you give me proper credit

Describe how proper biasing is accomplished in this headphone amplifier circuit (suitable for amplifying the audio output of a small radio):

$$\epsfbox{00750x01.eps}$$

Also, describe the functions of the 10 k$\Omega$ potentiometer and the 22 $\mu$F capacitor.

\underbar{file 00750}
%(END_QUESTION)





%(BEGIN_ANSWER)

Biasing is accomplished through the 100 k$\Omega$ resistor.  The 10 k$\Omega$ potentiometer is the volume control, and the 22 $\mu$F capacitor serves to "couple" the input signal to the transistor's base, while blocking any DC bias voltage from being "fed back" to the audio signal source.

\vskip 10pt

Challenge question: there is a name used to describe the dual-transistor configuration used in this circuit, where a pair of PNP or NPN transistors is cascaded, with the emitter of one going to the base of the other.  What is this name, and what advantage does this configuration provide over a single transistor?

%(END_ANSWER)





%(BEGIN_NOTES)

This circuit is simple enough to assemble and test in an hour or two, on a solderless breadboard.  It would make a great lab experiment, and can be used by the students outside of class!

%INDEX% Biasing, BJT amplifier circuit

%(END_NOTES)


