
%(BEGIN_QUESTION)
% Copyright 2005, Tony R. Kuphaldt, released under the Creative Commons Attribution License (v 1.0)
% This means you may do almost anything with this work of mine, so long as you give me proper credit

Research and describe what the {\it Hall effect} is, and list some practical uses of it.

\underbar{file 03671}
%(END_QUESTION)





%(BEGIN_ANSWER)

There are many good references describing the nature and discovery of the Hall effect.  I'll let you find this on your own!

Practical applications of the Hall effect include magnetic field measurement, non-contact current measurement, position sensors, and analog computation.

%(END_ANSWER)





%(BEGIN_NOTES)

One of the most interesting applications of the Hall effect I've ever seen was an analog {\it multiplier}, which output a voltage signal proportional to the product of two input signals.  In the specific example I saw, the two input signals were voltage and current for an AC load, the output of the Hall element representing instantaneous load {\it power}.  A precise fraction of the AC load current passed through the Hall element, while a precise fraction of the AC load voltage energized a coil to generate a magnetic field flux perpendicular to the current (through the Hall element).

%INDEX% Hall effect, defined

%(END_NOTES)


