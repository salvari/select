
%(BEGIN_QUESTION)
% Copyright 2003, Tony R. Kuphaldt, released under the Creative Commons Attribution License (v 1.0)
% This means you may do almost anything with this work of mine, so long as you give me proper credit

Suppose you are an electrician, and you need to disconnect the power conductors from a large electric motor.  What steps should you take to ensure no shock hazard exists prior to touching the bare conductors?

\underbar{file 00301}
%(END_QUESTION)





%(BEGIN_ANSWER)

\medskip
\item{$\bullet$} Turn the motor off (using the start/stop switch).
\item{$\bullet$} Shut off the circuit breaker (or disconnect switch) providing electrical power to the motor.
\item{$\bullet$} Lock and tag the circuit breaker (or disconnect switch) so no one else will turn it on in your absence.
\item{$\bullet$} Check the operation of a voltmeter by connecting it to a known source of voltage.
\item{$\bullet$} Check for the presence of voltage at the conductor terminations with a voltmeter.
\item{$\bullet$} Re-check the operation of a voltmeter by connecting it to a known source of voltage.
\item{$\bullet$} As a final step, touch the bare conductor with the {\it back} of your right hand, before touching it in any other way.
\medskip

Follow-up questions:

\medskip
\item{$\bullet$} Why check the voltmeter both before and after testing for voltage at the motor connections?
\item{$\bullet$} Why touch the conductor with the back of your hand?  What does it matter which side of your hand should touch the bare conductor?
\item{$\bullet$} Why use your right hand?  Why not your left?
\medskip

%(END_ANSWER)





%(BEGIN_NOTES)

A lot of electrical safety principles are covered in this one question.  Be sure to spend adequate time with the follow-up questions, to be sure students understand exactly why the given steps are necessary to ensure safety.

%INDEX% Safety, electrical
%INDEX% Voltmeter usage

%(END_NOTES)


