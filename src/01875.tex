
%(BEGIN_QUESTION)
% Copyright 2003, Tony R. Kuphaldt, released under the Creative Commons Attribution License (v 1.0)
% This means you may do almost anything with this work of mine, so long as you give me proper credit

If a coil of insulated wire is wrapped around an iron core, an inductance will be formed.  Even if the wire has negligible resistance, the current through the coil from an AC source will be limited by the inductive reactance ($X_L$) of the coil, as the magnetic flux in the iron core oscillates back and forth to induce a counter-EMF:

$$\epsfbox{01875x01.eps}$$

Plot the instantaneous magnetic flux ($\phi$) waveform in the iron core corresponding to the instantaneous applied voltage ($v$) shown in this graph:

$$\epsfbox{01875x02.eps}$$

\underbar{file 01875}
%(END_QUESTION)





%(BEGIN_ANSWER)

$$\epsfbox{01875x03.eps}$$

%(END_ANSWER)





%(BEGIN_NOTES)

There is a simple formula (albeit containing a derivative term) describing the relationship between instantaneous flux ($\phi$) and instantaneous induced voltage ($v$).  Your students ought to know what it is, and that it should be applied to this question!

%INDEX% Faraday's Law
%INDEX% Magnetic flux, relationship to counter-EMF

%(END_NOTES)


