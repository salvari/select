
%(BEGIN_QUESTION)
% Copyright 2005, Tony R. Kuphaldt, released under the Creative Commons Attribution License (v 1.0)
% This means you may do almost anything with this work of mine, so long as you give me proper credit

Shown here are eight rules of Boolean algebra (these are not the only rules, of course).  

\medskip
\item{$\bullet$} $A + \overline{A} = 1$
\item{$\bullet$} $A + A = A$
\item{$\bullet$} $A + 1 = 1$
\item{$\bullet$} $AA = A$
\item{$\bullet$} $A \> \overline{A} = 0$
\item{$\bullet$} $A(B + C) = AB + AC$
\item{$\bullet$} $A + AB = A$
\item{$\bullet$} $A + \overline{A}B = A + B$
\medskip

Determine which rule is being used in each step of the following Boolean simplification:

$$AB + B(B + \overline{C}) + \overline{B}C$$

$$AB + BB + B\overline{C} + \overline{B}C$$

$$AB + B + B\overline{C} + \overline{B}C$$

$$AB + B + \overline{B}C$$

$$AB + B + C$$

$$B + C$$

\underbar{file 02805}
%(END_QUESTION)





%(BEGIN_ANSWER)

$$AB + B(B + \overline{C}) + \overline{B}C$$

$$\hbox{{\bf Rule: }} A(B + C) = AB + AC$$

$$AB + BB + B\overline{C} + \overline{B}C$$

$$\hbox{{\bf Rule: }} AA = A$$

$$AB + B + B\overline{C} + \overline{B}C$$

$$\hbox{{\bf Rule: }} A + AB = A$$

$$AB + B + \overline{B}C$$

$$\hbox{{\bf Rule: }} A + \overline{A}B = A + B$$

$$AB + B + C$$

$$\hbox{{\bf Rule: }} A + AB = A$$

$$B + C$$

%(END_ANSWER)





%(BEGIN_NOTES)

Quite frequently (and quite distressingly), I meet students who seem to have the most difficult time relating algebraic rules in their general form to specific instances of reduction.  For example, a student who cannot tell that the rule $A + AB = A$ applies to the expression $QR + R$, or worse yet $B + AB$.  This skill requires time and hard work to master, because it is fundamentally a matter of abstraction: leaping from literal expressions to {\it similar} expressions, applying patterns from general rules to specific instances.

Questions such as this help students develop this abstraction ability.  Let students explain how they "made the connection" between Boolean rules and the given reductions.  Often, it helps to have a student explain the process to another student, because they are better able than you to put it into terms the struggling students can understand.

%INDEX% Boolean algebra, practice matching reductions to rules

%(END_NOTES)


