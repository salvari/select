
%(BEGIN_QUESTION)
% Copyright 2003, Tony R. Kuphaldt, released under the Creative Commons Attribution License (v 1.0)
% This means you may do almost anything with this work of mine, so long as you give me proper credit

A student built this resistor circuit on a solderless breadboard, but made a mistake positioning resistor R3.  It should be located one hole to the left instead of where it is right now:

$$\epsfbox{01770x01.eps}$$

Determine what the voltage drop will be across each resistor, in this faulty configuration, assuming that the battery outputs 9 volts.

\medskip
\item{$\bullet$} $R_1 = 2 \hbox{ k} \Omega$ \hskip 10pt $V_{R1} = $
\item{$\bullet$} $R_2 = 1 \hbox{ k} \Omega$ \hskip 10pt $V_{R2} = $
\item{$\bullet$} $R_3 = 3.3 \hbox{ k} \Omega$ \hskip 10pt $V_{R3} = $
\item{$\bullet$} $R_4 = 4.7 \hbox{ k} \Omega$ \hskip 10pt $V_{R4} = $
\item{$\bullet$} $R_5 = 4.7 \hbox{ k} \Omega$ \hskip 10pt $V_{R5} = $
\medskip

\underbar{file 01770}
%(END_QUESTION)





%(BEGIN_ANSWER)

Rather than tell you each voltage drop, I'll give you this one hint: there is only {\it one} resistor in this breadboard circuit that has voltage across it!  All the other resistors in this circuit are de-energized, thanks to the misplacement of resistor R3.

%(END_ANSWER)





%(BEGIN_NOTES)

Tell your students that the fault shown in this question is quite typical.  The hole spacings on solderless breadboards are small enough that it is surprisingly easy to mis-locate a component in the manner shown.

Point out to your students (if they haven't already noticed) that no calculations are necessary to answer this question!  It may be answered through simple, qualitative analysis alone.

%INDEX% Series-parallel circuit; effect of open fault

%(END_NOTES)


