
%(BEGIN_QUESTION)
% Copyright 2003, Tony R. Kuphaldt, released under the Creative Commons Attribution License (v 1.0)
% This means you may do almost anything with this work of mine, so long as you give me proper credit

Electromechanical watt-hour meters use an aluminum disk that is spun by an electric motor.  To generate a constant "drag" on the disk necessary to limit its rotational speed, a strong magnet is placed in such a way that its lines of magnetic flux pass perpendicularly through the disk's thickness:

$$\epsfbox{00259x01.eps}$$

Explain the phenomenon behind this magnetic "drag" mechanism, and also explain how the permanent magnet assembly should be re-positioned so that it provides {\it less} drag on the disk for the same rotational speed.

\underbar{file 00259}
%(END_QUESTION)





%(BEGIN_ANSWER)

This is an example of {\it Lenz' Law}.  For decreased drag on the disk, the magnet needs to be moved over a spot on the disk that has less surface velocity (I'll let you figure out where that might be).

\vskip 10pt

Follow-up question: suppose you {\it move} a strong magnet past the surface of an aluminum disk.  What will happen to the disk, if anything?

%(END_ANSWER)





%(BEGIN_NOTES)

An important calibration adjustment on electromechanical wattmeter assemblies is the positioning of the "drag" magnet, making this question a very practical one.  An interesting challenge for students is to ask them to sketch the flow of induced electric current in the disk as it rotates past the magnet!

The follow-up question is actually a preview of induction motor theory, and may be illustrated with a powerful (rare-earth) magnet and a metal coin (Japanese Yen, made of aluminum, work very well for this, being good electrical conductors and lightweight!).

%INDEX% Lenz's Law

%(END_NOTES)


