
%(BEGIN_QUESTION)
% Copyright 2005, Tony R. Kuphaldt, released under the Creative Commons Attribution License (v 1.0)
% This means you may do almost anything with this work of mine, so long as you give me proper credit

Predict how the motor function in this circuit will be affected as a result of the following faults.  Consider each fault independently (i.e. one at a time, no multiple faults):

$$\epsfbox{03714x01.eps}$$

\medskip
\item{$\bullet$} Transistor $Q_1$ fails open (collector-to-emitter):
\vskip 5pt
\item{$\bullet$} Transistor $Q_2$ fails open (collector-to-emitter):
\vskip 5pt
\item{$\bullet$} Transistor $Q_3$ fails open (collector-to-emitter):
\vskip 5pt
\item{$\bullet$} Transistor $Q_4$ fails open (collector-to-emitter):
\vskip 5pt
\item{$\bullet$} Resistor $R_1$ fails open:
\vskip 5pt
\item{$\bullet$} Resistor $R_2$ fails open:
\vskip 5pt
\item{$\bullet$} Resistor $R_3$ fails open:
\vskip 5pt
\item{$\bullet$} Transistor $Q_3$ fails shorted (collector-to-emitter):
\vskip 5pt
\item{$\bullet$} Transistor $Q_4$ fails shorted (collector-to-emitter):
\medskip

For each of these conditions, explain {\it why} the resulting effects will occur.

\underbar{file 03714}
%(END_QUESTION)





%(BEGIN_ANSWER)

\medskip
\item{$\bullet$} Transistor $Q_1$ fails open (collector-to-emitter): {\it Motor fails to turn in "reverse" direction, can still turn in "forward" direction.}
\vskip 5pt
\item{$\bullet$} Transistor $Q_2$ fails open (collector-to-emitter): {\it Motor fails to turn in "forward" direction, can still turn in "reverse" direction.}
\vskip 5pt
\item{$\bullet$} Transistor $Q_3$ fails open (collector-to-emitter): {\it Motor fails to turn in "forward" direction, can still turn in "reverse" direction.}
\vskip 5pt
\item{$\bullet$} Transistor $Q_4$ fails open (collector-to-emitter): {\it Motor fails to turn in "reverse" direction, can still turn in "forward" direction.}
\vskip 5pt
\item{$\bullet$} Resistor $R_1$ fails open: {\it Motor fails to turn in "forward" direction, can still turn in "reverse" direction.}
\vskip 5pt
\item{$\bullet$} Resistor $R_2$ fails open: {\it Motor fails to turn in "reverse" direction, can still turn in "forward" direction.}
\vskip 5pt
\item{$\bullet$} Resistor $R_3$ fails open: {\it Motor cannot turn in either direction.}
\vskip 5pt
\item{$\bullet$} Transistor $Q_3$ fails shorted (collector-to-emitter): {\it Motor turns in "forward" direction even when the switch is in the center (off) position.}
\vskip 5pt
\item{$\bullet$} Transistor $Q_4$ fails shorted (collector-to-emitter): {\it Motor turns in "reverse" direction even when the switch is in the center (off) position.}
\medskip

%(END_ANSWER)





%(BEGIN_NOTES)

The purpose of this question is to approach the domain of circuit troubleshooting from a perspective of knowing what the fault is, rather than only knowing what the symptoms are.  Although this is not necessarily a realistic perspective, it helps students build the foundational knowledge necessary to diagnose a faulted circuit from empirical data.  Questions such as this should be followed (eventually) by other questions asking students to identify likely faults based on measurements.

%INDEX% Troubleshooting, predicting effects of fault in H-bridge motor control circuit

%(END_NOTES)


