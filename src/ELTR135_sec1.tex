
\centerline{\bf ELTR 135 (Operational Amplifiers 2), section 1} \bigskip 
 
\vskip 10pt

\noindent
{\bf Recommended schedule}

\vskip 5pt

%%%%%%%%%%%%%%%
\hrule \vskip 5pt
\noindent
\underbar{Day 1}

\hskip 10pt Topics: {\it Operational amplifier AC performance}
 
\hskip 10pt Questions: {\it 1 through 10}
 
\hskip 10pt Lab Exercise: {\it Opamp slew rate (question 56)}
 
%INSTRUCTOR \hskip 10pt {\bf Demo: Show slew rate, opamp turning high-frequency square wave into triangle wave}

%INSTRUCTOR \hskip 10pt {\bf Demo: Show GBW limiting, high-gain opamp circuit cutting off at high-frequencies}

\vskip 10pt
%%%%%%%%%%%%%%%
\hrule \vskip 5pt
\noindent
\underbar{Day 2}

\hskip 10pt Topics: {\it AC calculations and filter circuit review}
 
\hskip 10pt Questions: {\it 11 through 25}
 
\hskip 10pt Lab Exercise: {\it Opamp gain-bandwidth product (question 57)}
 
\vskip 10pt
%%%%%%%%%%%%%%%
\hrule \vskip 5pt
\noindent
\underbar{Day 3}

\hskip 10pt Topics: {\it Active filter circuits}
 
\hskip 10pt Questions: {\it 26 through 35}
 
\hskip 10pt Lab Exercise: {\it Sallen-Key active lowpass filter (question 58)}
 
\vskip 10pt
%%%%%%%%%%%%%%%
\hrule \vskip 5pt
\noindent
\underbar{Day 4}

\hskip 10pt Topics: {\it Active filter circuits (continued)}
 
\hskip 10pt Questions: {\it 36 through 45}
 
\hskip 10pt Lab Exercise: {\it Sallen-Key active highpass filter (question 59)}
 
\vskip 10pt
%%%%%%%%%%%%%%%
\hrule \vskip 5pt
\noindent
\underbar{Day 5}

\hskip 10pt Topics: {\it Switched-capacitor circuits (optional)}
 
\hskip 10pt Questions: {\it 46 through 55}
 
\hskip 10pt Lab Exercise: {\it Bandpass active filter (question 60)}
 
\vskip 10pt
%%%%%%%%%%%%%%%
\hrule \vskip 5pt
\noindent
\underbar{Day 6}

\hskip 10pt Exam 1: {\it includes Active filter circuit performance assessment}
  
\hskip 10pt Lab Exercise: {\it Work on project}
 
\vskip 10pt
%%%%%%%%%%%%%%%
\hrule \vskip 5pt
\noindent
\underbar{Troubleshooting practice problems}

\hskip 10pt Questions: {\it 62 through 71}
 
\vskip 10pt
%%%%%%%%%%%%%%%
\hrule \vskip 5pt
\noindent
\underbar{General concept practice and challenge problems}

\hskip 10pt Questions: {\it 72 through the end of the worksheet}
 
\vskip 10pt
%%%%%%%%%%%%%%%
\hrule \vskip 5pt
\noindent
\underbar{Impending deadlines}

\hskip 10pt {\bf Project due at end of ELTR135, Section 2}
 
\hskip 10pt Question 61: Sample project grading criteria
 
\vskip 10pt
%%%%%%%%%%%%%%%







\vfil \eject

\centerline{\bf ELTR 135 (Operational Amplifiers 2), section 1} \bigskip 
 
\vskip 10pt

\noindent
{\bf Skill standards addressed by this course section}

\vskip 5pt

%%%%%%%%%%%%%%%
\hrule \vskip 10pt
\noindent
\underbar{EIA {\it Raising the Standard; Electronics Technician Skills for Today and Tomorrow}, June 1994}

\vskip 5pt

\medskip
\item{\bf E} {\bf Technical Skills -- Analog Circuits}
\item{\bf E.10} Understand principles and operations of operational amplifier circuits.
\item{\bf E.11} Fabricate and demonstrate operational amplifier circuits.
\item{\bf E.12} Troubleshoot and repair operational amplifier circuits.
\item{\bf E.18} Understand principles and operations of active filter circuits.
\item{\bf E.19} Troubleshoot and repair active filter circuits.
\medskip

\vskip 5pt

\medskip
\item{\bf B} {\bf Basic and Practical Skills -- Communicating on the Job}
\item{\bf B.01} Use effective written and other communication skills.  {\it Met by group discussion and completion of labwork.}
\item{\bf B.03} Employ appropriate skills for gathering and retaining information.  {\it Met by research and preparation prior to group discussion.}
\item{\bf B.04} Interpret written, graphic, and oral instructions.  {\it Met by completion of labwork.}
\item{\bf B.06} Use language appropriate to the situation.  {\it Met by group discussion and in explaining completed labwork.}
\item{\bf B.07} Participate in meetings in a positive and constructive manner.  {\it Met by group discussion.}
\item{\bf B.08} Use job-related terminology.  {\it Met by group discussion and in explaining completed labwork.}
\item{\bf B.10} Document work projects, procedures, tests, and equipment failures.  {\it Met by project construction and/or troubleshooting assessments.}
\item{\bf C} {\bf Basic and Practical Skills -- Solving Problems and Critical Thinking}
\item{\bf C.01} Identify the problem.  {\it Met by research and preparation prior to group discussion.}
\item{\bf C.03} Identify available solutions and their impact including evaluating credibility of information, and locating information.  {\it Met by research and preparation prior to group discussion.}
\item{\bf C.07} Organize personal workloads.  {\it Met by daily labwork, preparatory research, and project management.}
\item{\bf C.08} Participate in brainstorming sessions to generate new ideas and solve problems.  {\it Met by group discussion.}
\item{\bf D} {\bf Basic and Practical Skills -- Reading}
\item{\bf D.01} Read and apply various sources of technical information (e.g. manufacturer literature, codes, and regulations).  {\it Met by research and preparation prior to group discussion.}
\item{\bf E} {\bf Basic and Practical Skills -- Proficiency in Mathematics}
\item{\bf E.01} Determine if a solution is reasonable.
\item{\bf E.02} Demonstrate ability to use a simple electronic calculator.
\item{\bf E.05} Solve problems and [sic] make applications involving integers, fractions, decimals, percentages, and ratios using order of operations.
\item{\bf E.06} Translate written and/or verbal statements into mathematical expressions.
\item{\bf E.09} Read scale on measurement device(s) and make interpolations where appropriate.  {\it Met by oscilloscope usage.}
\item{\bf E.12} Interpret and use tables, charts, maps, and/or graphs.
\item{\bf E.13} Identify patterns, note trends, and/or draw conclusions from tables, charts, maps, and/or graphs.
\item{\bf E.15} Simplify and solve algebraic expressions and formulas.
\item{\bf E.16} Select and use formulas appropriately.
\item{\bf E.17} Understand and use scientific notation.
\item{\bf E.18} Use properties of exponents and logarithms.
\medskip

%%%%%%%%%%%%%%%



\vfil \eject

\centerline{\bf ELTR 135 (Operational Amplifiers 2), section 1} \bigskip 
 
\vskip 10pt

\noindent
{\bf Common areas of confusion for students}

\vskip 5pt


\hrule \vskip 5pt

\vskip 10pt

\noindent
{\bf Difficult concept: } {\it Trigonometry and phasor diagrams.}

AC circuit calculations tend to be difficult, if only because of all the math involved.  There is no way around this problem except to strengthen one's math competence, hence the review questions at the end of this worksheet.

\vskip 10pt

\noindent
{\bf Difficult concept: } {\it Identifying filter circuit types.}

Many students have a predisposition to memorization (as opposed to comprehension of concepts), and so when approaching filter circuits they try to identify the various types by memorizing the positions of reactive components.  As I like to tell my students, {\it memory will fail you}, and so a better approach is to develop analytical techniques by which you may determine circuit function based on "first principles" of circuits.  The approach I recommend begins by identifying component impedance (open or short) for very low and very high frequencies, respectively, then qualitatively analyzing voltage drops under those extreme conditions.  If a filter circuit outputs a strong voltage at low frequencies and a weak voltage at high frequencies then it must be a low-pass filter.  If it outputs a weak voltage at both low and high frequencies then it must be a band-pass filter, etc.



