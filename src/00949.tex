
%(BEGIN_QUESTION)
% Copyright 2003, Tony R. Kuphaldt, released under the Creative Commons Attribution License (v 1.0)
% This means you may do almost anything with this work of mine, so long as you give me proper credit

Suppose you were building a simple half-wave rectifier circuit for a 480 volt AC source.  The diode needs to withstand the full (peak) voltage of this AC source every other half-cycle of the waveform, or else it will fail.  The bad news is, the only diodes you have available for building this rectifier circuit are model 1N4002 diodes.

Describe how you could use multiple 1N4002 rectifying diodes to handle this much reverse voltage.

\underbar{file 00949}
%(END_QUESTION)





%(BEGIN_ANSWER)

Use seven 1N4002 diodes connected in series, like this:

$$\epsfbox{00949x01.eps}$$

\vskip 10pt

Follow-up question: while this solution should work (in theory), in practice one or more of the diodes will fail prematurely due to overvoltage.  The fix for this problem is to connect "divider" resistors in parallel with the diodes like this:

$$\epsfbox{00949x02.eps}$$

Explain why these resistors are necessary to ensure long diode life in this application.

%(END_ANSWER)





%(BEGIN_NOTES)

The answer to this question should not be much of a challenge to your students, although the follow-up question is a bit challenging.  Ask your students what purpose the divider resistors serve.  What do we know about voltage dropped across the diodes if one or more of them will fail without the divider resistors in place?

What value of resistor would your students recommend for this application?  What factors influence their decision regarding the resistance value?

%INDEX% Diodes, series connections for extra voltage

%(END_NOTES)


