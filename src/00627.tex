
%(BEGIN_QUESTION)
% Copyright 2003, Tony R. Kuphaldt, released under the Creative Commons Attribution License (v 1.0)
% This means you may do almost anything with this work of mine, so long as you give me proper credit

A coil of wire is formed of many loops.  These loops, though tracing a circular path, may be thought of as being parallel to each other.  We know that whenever two parallel wires carry an electric current, there will be a mechanical force generated between those two wires (as in Andr\'e Marie Amp\`ere's famous experiment).

When electric current is passed through a coil of wire, does the inter-loop force tend to compress the coil or extend it?  Explain your answer.

\underbar{file 00627}
%(END_QUESTION)





%(BEGIN_ANSWER)

The coil will tend to compress as current travels through its loops.

\vskip 10pt

Challenge question: what will happen to a wire coil if {\it alternating} current is passed through it instead of direct current?  Will the coil compress, extend, or do something entirely different?

%(END_ANSWER)





%(BEGIN_NOTES)

Questions such as this require the student to visualize a "bent" version of a phenomenon defined in terms of straight wires (Amp\'ere's experiment).  Some students, of course, will have a much more difficult time visualizing this than others.  For those that struggle with this form of problem-solving, spend some discussion time on problem-solving (visualization) techniques to help those who find this difficult to do.  Is there a particular drawing, sketch, or analogy that other students have found useful in their analysis of the problem?

The challenge question regarding alternating current is meant to be a "trick" question of sorts.  The "unthinking" answer is that with alternating current, there will be a force that alternates direction: repulsion one half-cycle, then attraction for the next half-cycle.  You may find students divided on this assessment, some thinking there will an alternating force, while others think the force will remain in the same direction at all times.  There is one sure way to prove who is correct here: set up an experiment with AC power and see for yourself (straight, parallel wires will work just fine for this)!

%INDEX% Electromagnetism

%(END_NOTES)


