
%(BEGIN_QUESTION)
% Copyright 2003, Tony R. Kuphaldt, released under the Creative Commons Attribution License (v 1.0)
% This means you may do almost anything with this work of mine, so long as you give me proper credit

$$\epsfbox{01926x01.eps}$$

\underbar{file 01926}
\vfil \eject
%(END_QUESTION)





%(BEGIN_ANSWER)

Use circuit simulation software to verify your predicted and measured parameter values.

%(END_ANSWER)





%(BEGIN_NOTES)

Students need not measure potentiometer shaft angles in order to do this exercise.  Rather, all they need to do is measure resistance between the wiper and the two outer terminals to set the potentiometer to a position where it will produce the specified division of voltage.

$R_{pot}$ refers to the potentiometer's nominal full-range value (for example, 1 k$\Omega$ or 5 k$\Omega$), and not to its particular setting.  The setting is what the student must figure out to achieve $V_{out}$.

An extension of this exercise is to incorporate troubleshooting questions.  Whether using this exercise as a performance assessment or simply as a concept-building lab, you might want to follow up your students' results by asking them to predict the consequences of certain circuit faults.

%INDEX% Assessment, performance-based (Potentiometer as loaded voltage divider)

%(END_NOTES)


