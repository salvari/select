
%(BEGIN_QUESTION)
% Copyright 2004, Tony R. Kuphaldt, released under the Creative Commons Attribution License (v 1.0)
% This means you may do almost anything with this work of mine, so long as you give me proper credit

The output voltage of a {\it boost converter} circuit is a function of the input voltage and the duty cycle of the switching signal, represented by the variable $D$ (ranging in value from 0\% to 100\%), where $D = {{t_{on}} \over {t_{on} + t_{off}}}$:


$$\epsfbox{02159x01.eps}$$

Based on this mathematical relationship, calculate the output voltage of this converter circuit at these duty cycles, assuming an input voltage of 40 volts:

\medskip
\goodbreak
\item{$\bullet$} $D$ = 0\% ; $V_{out}$ = 
\item{$\bullet$} $D$ = 25\% ; $V_{out}$ =
\item{$\bullet$} $D$ = 50\% ; $V_{out}$ =
\item{$\bullet$} $D$ = 75\% ; $V_{out}$ =
\item{$\bullet$} $D$ = 100\% ; $V_{out}$ =
\medskip

\underbar{file 02159}
%(END_QUESTION)





%(BEGIN_ANSWER)

\medskip
\goodbreak
\item{$\bullet$} $D$ = 0\% ; $V_{out}$ = 40 volts
\item{$\bullet$} $D$ = 25\% ; $V_{out}$ = 53.3 volts
\item{$\bullet$} $D$ = 50\% ; $V_{out}$ = 80 volts
\item{$\bullet$} $D$ = 75\% ; $V_{out}$ = 160 volts
\item{$\bullet$} $D$ = 100\% ; $V_{out}$ = 0 volts
\medskip

%(END_ANSWER)





%(BEGIN_NOTES)

The calculations for this circuit should be straightforward, except for the last calculation with a duty cycle of $D$ = 100\%.  Here, students must take a close look at the circuit and not just follow the formula blindly.  

Note that the switching element in the schematic diagram is shown in generic form.  It would never be a mechanical switch, but rather a transistor of some kind.

%INDEX% Boost converter circuit, calculating output voltage of
%INDEX% PWM, used to control boost converter output voltage

%(END_NOTES)


