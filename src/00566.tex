
%(BEGIN_QUESTION)
% Copyright 2003, Tony R. Kuphaldt, released under the Creative Commons Attribution License (v 1.0)
% This means you may do almost anything with this work of mine, so long as you give me proper credit

In this motor control circuit, how would you ensure that there is no danger of electric shock prior to touching either of the motor terminals (shown as points {\bf A} and {\bf B} in the schematic diagram)?  Describe both the action required to secure the power, and the means by which you would check for the presence of hazardous voltage at the motor:

$$\epsfbox{00566x01.eps}$$

\underbar{file 00566}
%(END_QUESTION)





%(BEGIN_ANSWER)

Follow these steps:

\medskip
\item{1.} Open the circuit breaker.
\item{2.} Lock the circuit breaker in its "open" position so no one can close it.
\item{3.} Try to start the motor by turning the On/Off switch "on".
\item{4.} Leave the On/Off switch in the "off" position.
\item{5.} Test for hazardous voltage (both AC and DC) between A and B, between A and ground, and between B and ground.
\medskip

%(END_ANSWER)





%(BEGIN_NOTES)

For each of the steps given in the answer, discuss the rationale with your students.  {\it Why} is it important we do each one of those steps, in the order shown?  How many voltage checks must we do with the voltmeter, total?  

Also, be sure to ask your students how they would know whether or not their voltmeter was functioning properly prior to using it to check for the presence of hazardous voltage.  What types of faults in the meter could cause it to not indicate voltage when there really was voltage?

%INDEX% Safety, electrical
%INDEX% Lock-out / tag-out
%INDEX% Voltmeter usage

%(END_NOTES)


