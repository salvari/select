
%(BEGIN_QUESTION)
% Copyright 2003, Tony R. Kuphaldt, released under the Creative Commons Attribution License (v 1.0)
% This means you may do almost anything with this work of mine, so long as you give me proper credit

Shown here is a schematic diagram for a simple battery-powered flashlight:

$$\epsfbox{00084x01.eps}$$

What could be modified about the circuit or its components to make the flashlight produce more light when turned on?

\underbar{file 00084}
%(END_QUESTION)





%(BEGIN_ANSWER)

Somehow, the power dissipated by the light bulb must be increased.  Perhaps the most obvious way to increase power dissipation is to use a battery with a greater voltage output, thus giving greater bulb current and greater power.  However, this is not the only option!  Think of another way the flashlight's output may be increased.

%(END_ANSWER)





%(BEGIN_NOTES)

The "obvious" solution is a direct application of Ohm's Law.  Other solutions may not be so direct, but they will all relate back to Ohm's Law somehow.

%INDEX% Ohm's Law, conceptual
%INDEX% Joule's Law, conceptual

%(END_NOTES)


