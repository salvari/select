
%(BEGIN_QUESTION)
% Copyright 2003, Tony R. Kuphaldt, released under the Creative Commons Attribution License (v 1.0)
% This means you may do almost anything with this work of mine, so long as you give me proper credit

A {\it solenoid valve} uses magnetism from an electromagnet coil to actuate a valve mechanism:

$$\epsfbox{00172x01.eps}$$

Essentially, this is an electrically-controlled on/off water valve.  In the development of this valve, though, the design engineers discover that the magnetic force produced by the electromagnet coil is not strong enough to achieve reliable valve actuation every time.  What can be changed in this solenoid valve design to produce a greater actuating force?

\underbar{file 00172}
%(END_QUESTION)





%(BEGIN_ANSWER)

Here are a few ways in which the strength of the magnetic field may be increased: pass a greater electrical current through the coil, use more turns of wire in the coil, or accentuate the field strength using better or larger magnetic core materials.  These are not the only ways to increase the mechanical force generated by the action of the magnetic field on the iron armature, but they are perhaps the most direct.

\vskip 10pt

Follow-up question: suppose this valve did not open like it was supposed to when the solenoid coil was energized.  Identify some possible reasons for this type of failure.

%(END_ANSWER)





%(BEGIN_NOTES)

The fundamental question here is, "what factors influence the strength of an electromagnetically-generated magnetic field?"  It is easy to research the effects of coil dimensions, core materials, current levels, etc.  What students need to do in this question is {\it apply} those techniques to this real-life scenario.

Be sure to spend time on the follow-up question with your students, considering non-electrical as well as electrical fault possibilities.

%INDEX% Solenoid valve, construction

%(END_NOTES)


