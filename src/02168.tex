
%(BEGIN_QUESTION)
% Copyright 2004, Tony R. Kuphaldt, released under the Creative Commons Attribution License (v 1.0)
% This means you may do almost anything with this work of mine, so long as you give me proper credit

An inductive AC load draws 13.4 amps of current at a voltage of 208 volts.  The phase shift between line voltage and line current is measured with an oscilloscope, and determined to be 23$^{o}$.  Calculate the following:

\medskip
\goodbreak
\item{$\bullet$} Apparent power ($S$) =
\item{$\bullet$} True power ($P$) =
\item{$\bullet$} Reactive power ($Q$) =
\item{$\bullet$} Power factor =
\medskip

An electrician suggests to you that the lagging power factor may be corrected by connecting a capacitor in parallel with this load.  If the capacitor is sized just right, it will exactly offset the reactive power of the inductive load, resulting in zero total reactive power and a power factor of unity (1).  Calculate the size of the necessary capacitor in Farads, assuming a line frequency of 60 Hz.

\underbar{file 02168}
%(END_QUESTION)





%(BEGIN_ANSWER)

\medskip
\goodbreak
\item{$\bullet$} Apparent power ($S$) = 2.787 kVA
\item{$\bullet$} True power ($P$) = 2.567 kW
\item{$\bullet$} Reactive power ($Q$) = 1.089 kVAR
\item{$\bullet$} Power factor = 0.921
\item{$\bullet$} Correction capacitor value = 66.77 $\mu$F
\medskip

\vskip 10pt

Challenge question: write an equation solving for the power factor correction capacitor size (in Farads) given any or all of the variables provided in the question ($S$, $P$, $Q$, $f$, $V$, P.F.).

%(END_ANSWER)





%(BEGIN_NOTES)

There are multiple methods of solution for this problem, so be sure to have your students present their thoughts and strategies during discussion!  The formula they write in answer to the challenge question will be nothing more than a formalized version of the solution strategy.

%INDEX% Power factor correction

%(END_NOTES)


