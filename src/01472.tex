
%(BEGIN_QUESTION)
% Copyright 2003, Tony R. Kuphaldt, released under the Creative Commons Attribution License (v 1.0)
% This means you may do almost anything with this work of mine, so long as you give me proper credit

Explain how a shift register circuit could be built from D-type flip-flops with the ability to shift data either to the right or to the left, on command.  I'm not necessarily asking for a schematic diagram so much as I'm looking for an {\it explanation} of how such a circuit might be built.  Of course, if your best way of presenting your idea is to draw a schematic diagram, go ahead!

\underbar{file 01472}
%(END_QUESTION)





%(BEGIN_ANSWER)

In order to provide {\it bidirectional} shift direction ability to a shift register circuit, you will probably have to use {\it steering gates} to direct the flip-flop outputs to different flip-flop inputs.  I'll leave the details for you to research and explain!

%(END_ANSWER)





%(BEGIN_NOTES)

I purposely avoided asking a question about schematics for a reason: it is too easy to simply look through a textbook or research a datasheet and copy a drawing.  What is most important here is that students comprehend {\it how} bidirectionality is achieved in shift register circuits.  What, exactly, is a {\it steering gate}, why are they used, and what in/out flip-flop connections are needed to achieve a desired shift direction.

%INDEX% Shift register, bidirectional

%(END_NOTES)


