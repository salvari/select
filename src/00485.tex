
%(BEGIN_QUESTION)
% Copyright 2003, Tony R. Kuphaldt, released under the Creative Commons Attribution License (v 1.0)
% This means you may do almost anything with this work of mine, so long as you give me proper credit

A spool holds an unknown length of aluminum wire.  The size of the wire is 4 AWG.  Fortunately, both ends of the wire are available for contact with an ohmmeter, to measure the resistance of the entire spool.  When measured, the wire's total resistance is 0.135 $\Omega$.  How much wire is on the spool (assuming the spool is at room temperature)?

\underbar{file 00485}
%(END_QUESTION)





%(BEGIN_ANSWER)

353.51 feet

%(END_ANSWER)





%(BEGIN_NOTES)

This question illustrates another practical application of specific resistance calculations: how to determine the length of wire on a spool.  The amount of resistance in this example is quite low, being a mere fraction of an ohm.  Ask your students what kinds of problems they might encounter trying to measure such a low resistance with accuracy.  Would the typical errors incurred in such a low-resistance measurement tend to make their calculation of length be excessive, or too low?  Why?

%INDEX% Conductor resistance calculation

%(END_NOTES)


