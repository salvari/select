
%(BEGIN_QUESTION)
% Copyright 2006, Tony R. Kuphaldt, released under the Creative Commons Attribution License (v 1.0)
% This means you may do almost anything with this work of mine, so long as you give me proper credit

Here, a differential pair circuit is driven by an input voltage at the base of $Q_2$, while the output is taken at the collector of $Q_2$.  Meanwhile, the other input ($Q_1$ base) is connected to ground:

$$\epsfbox{03921x01.eps}$$

Identify what types of amplifier circuits the two transistors are functioning as (common-collector, common-emitter, common-base) when the differential pair is used like this, and write an equation describing the circuit's voltage gain.  Here is another schematic, showing the transistors modeled as controlled current sources, to help you with the equation:

$$\epsfbox{03921x02.eps}$$

\underbar{file 03921}
%(END_QUESTION)





%(BEGIN_ANSWER)

$Q_2$ operates as a common-emitter amplifier, while $Q_1$ does not really act as an amplifier at all (given that no input or output connects to it).  The gain equation is as such:

$$A_{V(invert)} = {R_C \over {r'_e + (r'_e || R_E)}}$$

\vskip 10pt

Follow-up question \#1: explain why it is appropriate to simplify the gain equation to this:

$$A_{V(invert)} \approx {R_C \over 2r'_e}$$

Follow-up question \#2: explain why the simplified gain equation is sometimes written with a negative sign in it:

$$A_{V(invert)} \approx - {R_C \over 2r'_e}$$

%(END_ANSWER)





%(BEGIN_NOTES)

The purpose of this question is to have students analyze the resistances in the differential pair circuit to develop their own gain equation, based on their understanding of how simpler transistor amplifier circuit gains are derived.  Ultimately, this question should lead into another one asking students to express the {\it differential} voltage gain of the circuit (a superposition of the gain equations for each input considered separately).

%INDEX% Differential pair circuit, inverting voltage gain

%(END_NOTES)


