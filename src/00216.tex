
%(BEGIN_QUESTION)
% Copyright 2003, Tony R. Kuphaldt, released under the Creative Commons Attribution License (v 1.0)
% This means you may do almost anything with this work of mine, so long as you give me proper credit

Observe the following "4-band" resistors, their color codes, and corresponding resistance values (note that the last color band is omitted, since it deals with precision and not nominal value):

$$\epsfbox{00216x01.eps}$$

What patterns do you notice between the color codes (given as three-letter abbreviations, so as to avoid interpretational errors resulting from variations in print quality), the resistance values, and the physical sizes of the resistors?
 
\underbar{file 00216}
%(END_QUESTION)





%(BEGIN_ANSWER)

The first three color "bands" for all these four-band resistors denote two digits and a "multiplier" value, respectively.  Physical size has no relation to resistance.

\vskip 10pt

Follow-up question: what does the physical size of a resistor represent, if not resistance?

%(END_ANSWER)





%(BEGIN_NOTES)

The normal way to teach students the resistor color code is to show them the code first, then show them some resistors.  Here, the sequence is reversed: show the students some resistors, and have them figure out the code.  An important cognitive skill is the ability to detect and apply patterns in sets of data.  Exercises such as this help build that skill.

%INDEX% Color code, resistor (4-band)

%(END_NOTES)


