
%(BEGIN_QUESTION)
% Copyright 2003, Tony R. Kuphaldt, released under the Creative Commons Attribution License (v 1.0)
% This means you may do almost anything with this work of mine, so long as you give me proper credit

Suppose I wish to listen to the "hum" of ripple voltage from an AC-to-DC power supply using this detector.  "Ripple voltage," in case you don't know, is a small AC voltage superimposed on a large DC voltage.  If I simply connect my detector directly to the power supply's terminals, I hear a LOUD "click."  If I turn the volume control down until the "click" is tolerable, the hum is too faint to hear.  If I turn the volume control up far enough to hear the hum, then the "click" is far too loud for comfort.

How can I set up the detector so that it {\it only} detects the AC portion (the "ripple" voltage) of the power supply's output, and not the DC portion?

\underbar{file 01498}
%(END_QUESTION)





%(BEGIN_ANSWER)

This is actually quite simple to do, and it involves connecting a specific type of component in series with the detector's test leads.  I won't directly tell you what this component is, but I'll give you a hint: you will be doing that same thing that is done inside an oscilloscope when you set the "coupling" switch from the "DC" position to the "AC" position.

\vskip 10pt

Follow-up question: modify the schematic diagram to include an AC/DC coupling switch, so the detector's coupling may be switched from one mode to the other just like an oscilloscope.

%(END_ANSWER)





%(BEGIN_NOTES)

It would be very helpful to have an AC-DC power supply with substantial ripple voltage available for students to do this exercise.

%(END_NOTES)


