
%(BEGIN_QUESTION)
% Copyright 2005, Tony R. Kuphaldt, released under the Creative Commons Attribution License (v 1.0)
% This means you may do almost anything with this work of mine, so long as you give me proper credit

The following two expressions are frequently used to calculate values of changing variables (voltage and current) in RC and LR timing circuits:

$$e^{-{t \over \tau}} \hbox{\hskip 30pt or \hskip 30pt} 1 - e^{-{t \over \tau}}$$

One of these expressions describes the percentage that a changing value in an RC or LR circuit {\it has gone} from the starting time.  The other expression describes how far that same variable {\it has left to go} before it reaches its ultimate value (at $t = \infty$).  

The question is, which expression represents which quantity?  This is often a point of confusion, because students have a tendency to try to correlate these expressions to the quantities by rote memorization.  Does the expression $e^{-{t \over \tau}}$ represent the amount a variable has changed, or how far it has left to go until it stabilizes?  What about the other expression $1 - e^{-{t \over \tau}}$?  More importantly, {\it how can we figure this out so we don't have to rely on memory?}

$$\epsfbox{03117x01.eps}$$

\underbar{file 03117}
%(END_QUESTION)





%(BEGIN_ANSWER)

Here is a hint: set $x$ to zero and evaluate each equation.

%(END_ANSWER)





%(BEGIN_NOTES)

It is very important for students to understand what this expression means and how it works, lest they rely solely on memorization to use it in their calculations.  As I always tell my students, rote memorization {\it will} fail you!  If a student does not comprehend why the expression works as it does, they will be helpless to retain it as an effective "tool" for performing calculations in the future.

%INDEX% Time constant calculation, RC or LR circuit

%(END_NOTES)


