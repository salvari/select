
%(BEGIN_QUESTION)
% Copyright 2005, Tony R. Kuphaldt, released under the Creative Commons Attribution License (v 1.0)
% This means you may do almost anything with this work of mine, so long as you give me proper credit

Most microcomputers can only perform one task (operation) at a time.  They achieve the illusion of "multi-tasking" by alternately devoting time to one of several tasks in a rapid fashion -- a sort of multiplexed computation.  Most programmable logic devices, on the other hand, are easily able to perform multiple logic operations in a truly simultaneous manner.  Explain how this is possible, whereas a microprocessor can only do one thing at a time.

\underbar{file 03046}
%(END_QUESTION)





%(BEGIN_ANSWER)

The secret is in the programming: programmable logic devices are literally "wired" by the programs you write for them, with thousands of logic elements available to be connected in almost any way you desire.  Microprocessors, on the other hand, have fixed wiring that responds to sequences of steps, the program merely specifying those sequence of those steps.

%(END_ANSWER)





%(BEGIN_NOTES)

Understanding the distinction between microcontrollers and programmable logic devices can be difficult, especially if one has limited experience with both (as most students do).  The purpose of this question is to shed some more light on this often misunderstood subject, while simultaneously highlighting an important feature of programmable logic: true simultaneity.

The fundamental principle I want students to see from these analogies is that microcontrollers and microprocessors are re-programmed by changing a {\it sequence} of fixed operations, while programmable logic systems are re-programmed by changing {\it associations} between fixed elements.

%INDEX% Microcontroller versus programmable logic
%INDEX% Programmable logic versus microcontroller

%(END_NOTES)


