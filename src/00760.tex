
%(BEGIN_QUESTION)
% Copyright 2003, Tony R. Kuphaldt, released under the Creative Commons Attribution License (v 1.0)
% This means you may do almost anything with this work of mine, so long as you give me proper credit

Electrostatic meter movements use the physical attraction between metal plates caused by a voltage to deflect a pointer, instead of using electromagnetism as is common with most other meter movement designs.  Although electrostatic meter movements are not as sensitive as PMMC mechanisms, they have the advantage of being able to measure both AC and DC with equal ease.

Suppose you calibrated an electrostatic meter movement from 0 volts to 500 volts DC.  Then, you connected this meter to a sinusoidal AC source and watched it register a voltage of 216 volts.  What is the voltage of this AC source, in volts RMS?

\underbar{file 00760}
%(END_QUESTION)





%(BEGIN_ANSWER)

240 V RMS

\vskip 10pt

Hint: most mechanical meter movements naturally indicate the {\it average} value of an AC waveform!

%(END_ANSWER)





%(BEGIN_NOTES)

Discuss with your students the reason {\it why} most meter movements function on the average value of an AC waveform.  Why don't meter movement naturally indicate RMS value?  Ask your students how "average" and "RMS" AC values are defined, and what physical systems represent these mathematical models.

%(END_NOTES)


