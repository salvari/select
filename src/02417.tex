
%(BEGIN_QUESTION)
% Copyright 2005, Tony R. Kuphaldt, released under the Creative Commons Attribution License (v 1.0)
% This means you may do almost anything with this work of mine, so long as you give me proper credit

D-type MOSFETs are {\it normally-on} devices just like junction field-effect transistors, the natural state of their channels being passable to electric currents.  Thus, a state of cutoff will only occur on command from an external source.

Explain what must be done to an D-type MOSFET, specifically, to drive it into a state of cutoff (where the channel is fully depleted).

\underbar{file 02417}
%(END_QUESTION)





%(BEGIN_ANSWER)

A voltage must be applied between gate and substrate (or gate and source if the substrate is connected to the source terminal) in such a way that the polarity of the gate terminal electrostatically repels the channel's majority charge carriers.

\vskip 10pt

Follow-up question: unlike JFETs, D-type MOSFETs may be safely "enhanced" beyond the conductivity of their natural state.  Describe what is necessary to "command" a D-type MOSFET to conduct better than it naturally does.

%(END_ANSWER)





%(BEGIN_NOTES)

This is perhaps the most important question your students could learn to answer when first studying D-type MOSFETs.  What, exactly, is necessary to turn one off?  Have your students draw diagrams to illustrate their answers as they present in front of the class.

Ask them specifically to identify what polarity of $V_{GS}$ would have to be applied to turn off an N-channel D-type MOSFET, and also a P-channel D-type MOSFET.

%INDEX% MOSFET, conditions necessary for conduction and cutoff (D-type)
%INDEX% Gate voltages, for D-type MOSFET during conduction and cutoff

%(END_NOTES)


