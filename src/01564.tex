
%(BEGIN_QUESTION)
% Copyright 2003, Tony R. Kuphaldt, released under the Creative Commons Attribution License (v 1.0)
% This means you may do almost anything with this work of mine, so long as you give me proper credit

Plot the typical response of a band-pass filter circuit, showing signal output (amplitude) on the vertical axis and frequency on the horizontal axis:

$$\epsfbox{01564x01.eps}$$

Also, identify and label the {\it bandwidth} of the circuit on your filter plot.

\underbar{file 01564}
%(END_QUESTION)





%(BEGIN_ANSWER)

$$\epsfbox{01564x02.eps}$$

The bandwidth of a band-pass filter circuit is that range of frequencies where the output amplitude is at least 70.7\% of maximum:

$$\epsfbox{01564x03.eps}$$

%(END_ANSWER)





%(BEGIN_NOTES)

Bandwidth is an important concept in electronics, for more than just filter circuits.  Your students may discover references to bandwidth of amplifiers, transmission lines, and other circuit elements as they do their research.  Despite the many and varied applications of this term, the principle is fundamentally the same.

%INDEX% Bandwidth, band-pass filter circuit

%(END_NOTES)


