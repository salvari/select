
%(BEGIN_QUESTION)
% Copyright 2005, Tony R. Kuphaldt, released under the Creative Commons Attribution License (v 1.0)
% This means you may do almost anything with this work of mine, so long as you give me proper credit

What will the output voltage of this integrator circuit do when the DPDT ("Double-Pole, Double-Throw") switch is flipped back and forth?

$$\epsfbox{02706x01.eps}$$

Be as specific as you can in your answer, explaining what happens in the switch's "up" position as well as in its "down" position.

\underbar{file 02706}
%(END_QUESTION)





%(BEGIN_ANSWER)

With the switch in the "up" position, the opamp output linearly ramps in a negative-going direction over time.  With the switch in the "down" position, the opamp output linearly ramps in a positive-going direction over time.

\vskip 10pt

Follow-up question: what do you suppose the output of the following circuit would do over time (assuming the square wave input was true AC, positive and negative)?

$$\epsfbox{02706x02.eps}$$

%(END_ANSWER)





%(BEGIN_NOTES)

The DPDT switch arrangement may be a bit confusing, but its only purpose is to provide a reversible input voltage polarity.  Discuss with your students the directions of all currents in this circuit for both switch positions, and how the opamp output integrates over time for different input voltages.

%INDEX% Integrator circuit, op-amp

%(END_NOTES)


