
%(BEGIN_QUESTION)
% Copyright 2003, Tony R. Kuphaldt, released under the Creative Commons Attribution License (v 1.0)
% This means you may do almost anything with this work of mine, so long as you give me proper credit

Draw the necessary wire connections to build the circuit shown in this ladder diagram:

\vskip 10pt

Ladder diagram:

$$\epsfbox{00836x01.eps}$$

\vskip 10pt

Illustration showing components:

$$\epsfbox{00836x02.eps}$$

\vskip 10pt

Yes, the "Run" switch shown in the diagram is a SPST, but the switch shown in the illustration is a SPDT.  This is a realistic scenario, where the only type of switch you have available is a SPDT, but the wiring diagram calls for something different.  It is your job to improvise a solution!

\underbar{file 00836}
%(END_QUESTION)





%(BEGIN_ANSWER)

$$\epsfbox{00836x03.eps}$$

\vskip 10pt

Challenge question: which switch position (handle to the left or handle to the right) turns the motor on?

%(END_ANSWER)





%(BEGIN_NOTES)

This question helps students build their spatial-relations skills, as they relate a neat, clean diagram to a relatively "messy" real-world circuit.  As usual, the circuit shown here is not the only way it could have been built, but it is one solution.

In reference to the challenge question, the particular style of SPDT switch shown is very common, and the terminal connections on the bottom might not be what you would expect from looking at its schematic symbol.

%INDEX% Ladder logic diagram

%(END_NOTES)


