
%(BEGIN_QUESTION)
% Copyright 2005, Tony R. Kuphaldt, released under the Creative Commons Attribution License (v 1.0)
% This means you may do almost anything with this work of mine, so long as you give me proper credit

Large power distribution centers are often equipped with capacitors to correct for lagging (inductive) power factor of many industrial loads.  There is never any one value for capacitance that will precisely correct for the power factor, though, because load conditions constantly change.  At first it may seem that a variable capacitor would be the answer (adjustable to compensate for {\it any} value of lagging power factor), but variable capacitors with the ratings necessary for power line compensation would be prohibitively large and expensive.

One solution to this problem of variable capacitance uses a set of electromechanical relays with fixed-value capacitors:

$$\epsfbox{03625x01.eps}$$

Explain how a circuit such as this provides a step-variable capacitance, and determine the range of capacitance it can provide.

\underbar{file 03625}
%(END_QUESTION)





%(BEGIN_ANSWER)

Capacitors may be selected in combination to provide anywhere from 0 $\mu$F to 15 $\mu$F, in 1 $\mu$F steps.

%(END_ANSWER)





%(BEGIN_NOTES)

Although semiconductor-based static VAR compensator circuits are now the method of choice for modern power systems, this technique is still valid and is easy enough for beginning students to comprehend.  A circuit such as this is a great application of the binary number system, too!

%INDEX% Power factor correction, binary circuit

%(END_NOTES)


