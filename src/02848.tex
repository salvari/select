
%(BEGIN_QUESTION)
% Copyright 2005, Tony R. Kuphaldt, released under the Creative Commons Attribution License (v 1.0)
% This means you may do almost anything with this work of mine, so long as you give me proper credit

Explain the meaning of the digital lines A, B, F, and S in the following schematic diagram:

$$\epsfbox{02848x01.eps}$$

\underbar{file 02848}
%(END_QUESTION)





%(BEGIN_ANSWER)

Lines A, B, and F (with the slash marks and the number "4") represent four actual conductors, carrying four bits of digital information.  The thick line (S) is also a four-bit "bus" but is denoted by a slightly different convention.

In case you were wondering, it is unusual to mix two different bus symbol conventions in the same schematic diagram.  I show this here only for your benefit, to see that there is more than one "standard" way to draw it.

%(END_ANSWER)





%(BEGIN_NOTES)

The answer pretty much says it all.  The fact that the IC is an ALU is quite incidental.  Some students may research the part number to get a better understanding of what is going on.  That is fine, but my emphasis in this question is the schematic diagram convention(s) for multi-conductor busses.

%INDEX% Bus (digital), schematic symbol for

%(END_NOTES)


