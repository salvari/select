
\centerline{\bf ELTR 140 (Digital 1), section 2} \bigskip 
 
\vskip 10pt

\noindent
{\bf Recommended schedule}

\vskip 5pt

%%%%%%%%%%%%%%%
\hrule \vskip 5pt
\noindent
\underbar{Day 1}

\hskip 10pt Topics: {\it Boolean algebra, basic concepts and identities}
 
\hskip 10pt Questions: {\it 1 through 20}
 
\hskip 10pt Lab Exercise: {\it work on project}
 
\vskip 10pt
%%%%%%%%%%%%%%%
\hrule \vskip 5pt
\noindent
\underbar{Day 2}

\hskip 10pt Topics: {\it Boolean algebra, simplification laws}
 
\hskip 10pt Questions: {\it 21 through 40}
 
\hskip 10pt Lab Exercise: {\it Gate circuit from Boolean expression (question 96)}
 
\vskip 10pt
%%%%%%%%%%%%%%%
\hrule \vskip 5pt
\noindent
\underbar{Day 3}

\hskip 10pt Topics: {\it SOP and POS expressions}
 
\hskip 10pt Questions: {\it 41 through 60}
 
\hskip 10pt Lab Exercise: {\it Gate circuit from truth table (question 97)}
 
\vskip 10pt
%%%%%%%%%%%%%%%
\hrule \vskip 5pt
\noindent
\underbar{Day 4}

\hskip 10pt Topics: {\it Karnaugh mapping}
 
\hskip 10pt Questions: {\it 61 through 75}
 
\hskip 10pt Lab Exercise: {\it work on project}
 
\vskip 10pt
%%%%%%%%%%%%%%%
\hrule \vskip 5pt
\noindent
\underbar{Day 5}

\hskip 10pt Topics: {\it DeMorgan's Theorem and gate universality}
 
\hskip 10pt Questions: {\it 76 through 95}
 
\hskip 10pt Lab Exercise: {\it NAND gate universality (question 98)}
 
\vskip 10pt
%%%%%%%%%%%%%%%
\hrule \vskip 5pt
\noindent
\underbar{Day 6}

\hskip 10pt Exam 2: {\it includes Boolean-to-gate performance assessment}
 
\vskip 10pt
%%%%%%%%%%%%%%%
\hrule \vskip 5pt
\noindent
\underbar{Troubleshooting practice problems}

\hskip 10pt Questions: {\it 100 through 109}
 
\vskip 10pt
%%%%%%%%%%%%%%%
\hrule \vskip 5pt
\noindent
\underbar{General concept practice and challenge problems}

\hskip 10pt Questions: {\it 110 through the end of the worksheet}
 
\vskip 10pt
%%%%%%%%%%%%%%%
\hrule \vskip 5pt
\noindent
\underbar{Impending deadlines}

\hskip 10pt {\bf Project due at end of ELTR140, Section 3}
 
\hskip 10pt Question 99: Sample project grading criteria
 
\vskip 10pt
%%%%%%%%%%%%%%%









\vfil \eject

\centerline{\bf ELTR 140 (Digital 1), section 2} \bigskip 
 
\vskip 10pt

\noindent
{\bf Skill standards addressed by this course section}

\vskip 5pt

%%%%%%%%%%%%%%%
\hrule \vskip 10pt
\noindent
\underbar{EIA {\it Raising the Standard; Electronics Technician Skills for Today and Tomorrow}, June 1994}

\vskip 5pt

\medskip
\item{\bf F} {\bf Technical Skills -- Digital Circuits}
\item{\bf F.02} Demonstrate an understanding of minimizing logic circuits using Boolean operations.
\item{\bf F.08} Understand principles and operations of combinational logic circuits.
\item{\bf F.09} Fabricate and demonstrate combinational logic circuits.
\item{\bf F.10} Troubleshoot and repair combinational logic circuits.
\medskip

\vskip 5pt

\medskip
\item{\bf B} {\bf Basic and Practical Skills -- Communicating on the Job}
\item{\bf B.01} Use effective written and other communication skills.  {\it Met by group discussion and completion of labwork.}
\item{\bf B.03} Employ appropriate skills for gathering and retaining information.  {\it Met by research and preparation prior to group discussion.}
\item{\bf B.04} Interpret written, graphic, and oral instructions.  {\it Met by completion of labwork.}
\item{\bf B.06} Use language appropriate to the situation.  {\it Met by group discussion and in explaining completed labwork.}
\item{\bf B.07} Participate in meetings in a positive and constructive manner.  {\it Met by group discussion.}
\item{\bf B.08} Use job-related terminology.  {\it Met by group discussion and in explaining completed labwork.}
\item{\bf B.10} Document work projects, procedures, tests, and equipment failures.  {\it Met by project construction and/or troubleshooting assessments.}
\item{\bf C} {\bf Basic and Practical Skills -- Solving Problems and Critical Thinking}
\item{\bf C.01} Identify the problem.  {\it Met by research and preparation prior to group discussion.}
\item{\bf C.03} Identify available solutions and their impact including evaluating credibility of information, and locating information.  {\it Met by research and preparation prior to group discussion.}
\item{\bf C.07} Organize personal workloads.  {\it Met by daily labwork, preparatory research, and project management.}
\item{\bf C.08} Participate in brainstorming sessions to generate new ideas and solve problems.  {\it Met by group discussion.}
\item{\bf D} {\bf Basic and Practical Skills -- Reading}
\item{\bf D.01} Read and apply various sources of technical information (e.g. manufacturer literature, codes, and regulations).  {\it Met by research and preparation prior to group discussion.}
\item{\bf E} {\bf Basic and Practical Skills -- Proficiency in Mathematics}
\item{\bf E.01} Determine if a solution is reasonable.
\item{\bf E.02} Demonstrate ability to use a simple electronic calculator.
\item{\bf E.06} Translate written and/or verbal statements into mathematical expressions.
\item{\bf E.12} Interpret and use tables, charts, maps, and/or graphs.
\item{\bf E.13} Identify patterns, note trends, and/or draw conclusions from tables, charts, maps, and/or graphs.
\item{\bf E.15} Simplify and solve algebraic expressions and formulas.
\item{\bf E.16} Select and use formulas appropriately.
\item{\bf E.18} Use properties of exponents and logarithms.
\item{\bf E.21} Use Boolean algebra to break down logic circuits.
\medskip

%%%%%%%%%%%%%%%




\vfil \eject

\centerline{\bf ELTR 140 (Digital 1), section 2} \bigskip 
 
\vskip 10pt

\noindent
{\bf Common areas of confusion for students}

\vskip 5pt

%%%%%%%%%%%%%%%
\hrule \vskip 5pt

\vskip 10pt

\noindent
{\bf Difficult concept: } {\it Boolean rules and identities.}

In many ways Boolean algebra is simpler than regular algebra (e.g., there is no such thing as subtraction or division to worry about), but it is still algebra, and because of this fact many students struggle with it.  Working with algebraic expressions means precisely following a specific set of absolute rules.  This requires a reliable knowledge of those rules and an ability to think rigorously.  These are not easy requirements for most human beings, which is why so many people dislike mathematics.  There is but one solution to this problem: practice, practice, and practice again.  If you find yourself making algebraic mistakes, don't give up -- that would be exactly the wrong thing to do.  Learn from your mistakes, pick yourself back up, and try again.  It {\it can} be done!

\vskip 10pt

\noindent
{\bf Difficult concept: } {\it Algebraic substitution.}

Many Boolean rules such as $A + AB = A$ are easy enough to learn in their canonical form, but more difficult to apply when seen in a form such as $CF\overline{G} + C \overline{G} = C \overline{G}$.  Fundamentally, this is a problem with the algebraic principle of {\it substitution}: replacing one variable with a different variable or a whole expression.  The key to substitution is the ability to perform visual pattern-matching, which some people have a much easier time with than others.  Once again, the only solution to this problem is practice, practice, and more practice.  Remember that no one comes out of the womb knowing how to do this stuff!  Everyone who has any knowledge of algebra at all once started knowing nothing about it.  People {\it can} learn and succeed at this material, yourself included.  Your personal journey through algebra may be more difficult than it is for others, for any number of reasons, but it is not an impossible journey.


