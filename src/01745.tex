
%(BEGIN_QUESTION)
% Copyright 2003, Tony R. Kuphaldt, released under the Creative Commons Attribution License (v 1.0)
% This means you may do almost anything with this work of mine, so long as you give me proper credit

Batteries store energy in the form of chemical bonds between atoms.  When a complete circuit is connected to the terminals of a battery, a current begins inside the battery and through the circuit.  Chemical reactions begin inside the battery at the same time the circuit is formed, fueling this current.  When all the chemicals have been depleted, though, the current will cease.  This is what happens when a battery "dies."

Explain what takes place when a battery is {\it recharged}, and what electrical device(s) must be connected to a battery to recharge it.  Be as specific as possible in your answer.

\underbar{file 01745}
%(END_QUESTION)





%(BEGIN_ANSWER)

{\it Recharging} is a reversal of the chemical reactions taking place during the discharge of a battery.  In order for this to occur, some external source of energy must force current {\it backward} through the battery.

%(END_ANSWER)





%(BEGIN_NOTES)

It is important to note to your students that not all battery types lend themselves to recharging, even though many "primary" cells can be rejuvenated a small degree by being put through a charging (reverse-current) cycle.

%INDEX% Battery charging, conceptual

%(END_NOTES)


