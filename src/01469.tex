
%(BEGIN_QUESTION)
% Copyright 2003, Tony R. Kuphaldt, released under the Creative Commons Attribution License (v 1.0)
% This means you may do almost anything with this work of mine, so long as you give me proper credit

Suppose we wished to use a shift register circuit to input several binary bits at once ({\it parallel} data transfer), and then output the bits one at a time over a single line ({\it serial} data transfer).  You should be aware of how shift registers are constructed with D-type flip-flops.  Now, describe how we can get parallel data entered into a shift register circuit.  Note: there is more than one answer to this question!

\underbar{file 01469}
%(END_QUESTION)





%(BEGIN_ANSWER)

Perhaps the most direct way to provide parallel data entry is to make use of the flip-flops' asynchronous inputs.

%(END_ANSWER)





%(BEGIN_NOTES)

During discussion, have your students draw a picture of a parallel-in/serial-out shift register circuit, or at least cite a page number reference in their textbook, so you may be sure they understand what they're talking about (and not just repeating the given answer).

%INDEX% Shift register, parallel in / serial out

%(END_NOTES)


