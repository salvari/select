
\centerline{\bf ELTR 135 (Operational Amplifiers 2), section 2} \bigskip 
 
\vskip 10pt

\noindent
{\bf Recommended schedule}

\vskip 5pt

%%%%%%%%%%%%%%%
\hrule \vskip 5pt
\noindent
\underbar{Day 1}

\hskip 10pt Topics: {\it Operational amplifier oscillators}
 
\hskip 10pt Questions: {\it 1 through 10}
 
\hskip 10pt Lab Exercise: {\it Opamp relaxation oscillator (question 46)}
 
\vskip 10pt
%%%%%%%%%%%%%%%
\hrule \vskip 5pt
\noindent
\underbar{Day 2}

\hskip 10pt Topics: {\it Calculus explained through active integrator and differentiator circuits}
 
\hskip 10pt Questions: {\it 11 through 20}
 
\hskip 10pt Lab Exercise: {\it Opamp triangle wave generator (question 47)}
 
\vskip 10pt
%%%%%%%%%%%%%%%
\hrule \vskip 5pt
\noindent
\underbar{Day 3}

\hskip 10pt Topics: {\it Logarithm review}
 
\hskip 10pt Questions: {\it 21 through 35}
 
\hskip 10pt Lab Exercise: {\it Opamp LC resonant oscillator (question 48)}
 
\vskip 10pt
%%%%%%%%%%%%%%%
\hrule \vskip 5pt
\noindent
\underbar{Day 4}

\hskip 10pt Topics: {\it Log/antilog circuits} (optional)
 
\hskip 10pt Questions: {\it 36 through 45}
 
\hskip 10pt Lab Exercise: {\it Work on project}
 
\vskip 10pt
%%%%%%%%%%%%%%%
\hrule \vskip 5pt
\noindent
\underbar{Day 5}

\hskip 10pt Topics: {\it Review}
 
\hskip 10pt Lab Exercise: {\it Work on project}
 
%INSTRUCTOR \hskip 10pt {\bf Show picture(s) of analog computers}

\vskip 10pt
%%%%%%%%%%%%%%%
\hrule \vskip 5pt
\noindent
\underbar{Day 6}

\hskip 10pt Exam 2: {\it includes Oscillator circuit performance assessment}
 
\hskip 10pt {\bf Project due}

\hskip 10pt Question 49: Sample project grading criteria
 
\vskip 10pt
%%%%%%%%%%%%%%%
\hrule \vskip 5pt
\noindent
\underbar{Troubleshooting practice problems}

\hskip 10pt Questions: {\it 50 through 59}
 
\vskip 10pt
%%%%%%%%%%%%%%%
\hrule \vskip 5pt
\noindent
\underbar{General concept practice and challenge problems}

\hskip 10pt Questions: {\it 60 through the end of the worksheet}
 
\vskip 10pt
%%%%%%%%%%%%%%%








\vfil \eject

\centerline{\bf ELTR 135 (Operational Amplifiers 2), section 2} \bigskip 
 
\vskip 10pt

\noindent
{\bf Skill standards addressed by this course section}

\vskip 5pt

%%%%%%%%%%%%%%%
\hrule \vskip 10pt
\noindent
\underbar{EIA {\it Raising the Standard; Electronics Technician Skills for Today and Tomorrow}, June 1994}

\vskip 5pt

\medskip
\item{\bf E} {\bf Technical Skills -- Analog Circuits}
\item{\bf E.10} Understand principles and operations of operational amplifier circuits.
\item{\bf E.11} Fabricate and demonstrate operational amplifier circuits.
\item{\bf E.12} Troubleshoot and repair operational amplifier circuits.
\item{\bf E.20} Understand principles and operations of sinusoidal and non-sinusoidal oscillator circuits.
\item{\bf E.21} Troubleshoot and repair sinusoidal and non-sinusoidal oscillator circuits.
\medskip

\vskip 5pt

\medskip
\item{\bf B} {\bf Basic and Practical Skills -- Communicating on the Job}
\item{\bf B.01} Use effective written and other communication skills.  {\it Met by group discussion and completion of labwork.}
\item{\bf B.03} Employ appropriate skills for gathering and retaining information.  {\it Met by research and preparation prior to group discussion.}
\item{\bf B.04} Interpret written, graphic, and oral instructions.  {\it Met by completion of labwork.}
\item{\bf B.06} Use language appropriate to the situation.  {\it Met by group discussion and in explaining completed labwork.}
\item{\bf B.07} Participate in meetings in a positive and constructive manner.  {\it Met by group discussion.}
\item{\bf B.08} Use job-related terminology.  {\it Met by group discussion and in explaining completed labwork.}
\item{\bf B.10} Document work projects, procedures, tests, and equipment failures.  {\it Met by project construction and/or troubleshooting assessments.}
\item{\bf C} {\bf Basic and Practical Skills -- Solving Problems and Critical Thinking}
\item{\bf C.01} Identify the problem.  {\it Met by research and preparation prior to group discussion.}
\item{\bf C.03} Identify available solutions and their impact including evaluating credibility of information, and locating information.  {\it Met by research and preparation prior to group discussion.}
\item{\bf C.07} Organize personal workloads.  {\it Met by daily labwork, preparatory research, and project management.}
\item{\bf C.08} Participate in brainstorming sessions to generate new ideas and solve problems.  {\it Met by group discussion.}
\item{\bf D} {\bf Basic and Practical Skills -- Reading}
\item{\bf D.01} Read and apply various sources of technical information (e.g. manufacturer literature, codes, and regulations).  {\it Met by research and preparation prior to group discussion.}
\item{\bf E} {\bf Basic and Practical Skills -- Proficiency in Mathematics}
\item{\bf E.01} Determine if a solution is reasonable.
\item{\bf E.02} Demonstrate ability to use a simple electronic calculator.
\item{\bf E.05} Solve problems and [sic] make applications involving integers, fractions, decimals, percentages, and ratios using order of operations.
\item{\bf E.06} Translate written and/or verbal statements into mathematical expressions.
\item{\bf E.09} Read scale on measurement device(s) and make interpolations where appropriate.  {\it Met by oscilloscope usage.}
\item{\bf E.12} Interpret and use tables, charts, maps, and/or graphs.
\item{\bf E.13} Identify patterns, note trends, and/or draw conclusions from tables, charts, maps, and/or graphs.
\item{\bf E.15} Simplify and solve algebraic expressions and formulas.
\item{\bf E.16} Select and use formulas appropriately.
\item{\bf E.17} Understand and use scientific notation.
\item{\bf E.18} Use properties of exponents and logarithms.
\medskip

%%%%%%%%%%%%%%%



\vfil \eject

\centerline{\bf ELTR 135 (Operational Amplifiers 2), section 2} \bigskip 
 
\vskip 10pt

\noindent
{\bf Common areas of confusion for students}

\vskip 5pt


\hrule \vskip 5pt

\vskip 10pt

\noindent
{\bf Difficult concept: } {\it Opamp relaxation oscillator circuit.}

This circuit can be difficult to grasp, because there is a tendency to immediately apply one of the "canonical rules" of opamp circuits: that there is negligible voltage between the inverting and noninverting inputs when negative feedback is present.  It is wrong to apply this rule here, though, because this circuit's behavior is dominated by {\it positive} feedback.  This positive feedback comes through a plain resistor network, while the negative feedback comes through a "first-order lag" RC time constant network, which means the positive feedback is immediate while the negative feedback is delayed.  Because of this, the opamp output swings back and forth between saturated states, and never settles to an equilibrium position.  Thus, it does not act like a simple amplifier and there will be substantial voltage between the two opamp inputs.  

A better way to consider this circuit is to think of it as a comparator with hysteresis, the inverting input "chasing" the output with a time lag.

\vskip 10pt

\noindent
{\bf Difficult concept: } {\it Rates of change.}

When studying integrator and differentiator circuits, one must think in terms of how fast a variable is changing.  This is the first hurdle in calculus: to comprehend what a rate of change is, and it is not obvious.  One thing I really like about teaching electronics is that capacitor and inductors naturally exhibit the calculus principles of integration and differentiation (with respect to time), and so provide an excellent context in which the electronics student may explore basic principles of calculus.  Integrator and differentiator circuits exploit these properties, so that the output voltage is approximately either the time-integral or time-derivative (respectively) of the input voltage signal.

It is helpful, though, to relate these principles to more ordinary contexts, which is why I often describe rates of change in terms of {\it velocity} and {\it acceleration}.  Velocity is nothing more than a rate of change of position: how quickly one's position is changing over time.  Therefore, if the variable $x$ describes position, then the derivative ${dx \over dt}$ (rate of change of $x$ over time $t$) must describe velocity.  Likewise, acceleration is nothing more than the rate of change of velocity: how quickly velocity changes over time.  If the variable $v$ describes velocity, then the derivative ${dv \over dt}$ must describe velocity.  Or, since we know that velocity is itself the derivative of position, we could describe acceleration as the {\it second derivative} of position: ${d^2 x \over dt^2}$

\vskip 10pt

\noindent
{\bf Difficult concept: } {\it Derivative versus integral.}

The two foundational concepts of calculus are inversely related: {\it differentiation} and {\it integration} are flip-sides of the same coin.  That is to say, one "un-does" the other.  If you can grasp what one of these operations is, then the other is simply the reverse.

One of the better ways to illustrate the inverse nature of these two operations is to consider them in the context of motion analysis, relating {\it position} ($x$), velocity ($v$), and {\it acceleration} ($a$).  Differentiating with respect to time, the derivative of position is velocity ($v = {dx \over dt}$), and the derivative of velocity is acceleration ($a = {dv \over dt}$).  Integrating with respect to time, the integral of acceleration is velocity ($v = \int a \> dt$) and the integral of velocity is position ($x = \int v \> dt$).

Fortunately, electronics provides a ready context in which to understand differentiation and integration.  It is very easy to build {\it differentiator} and {\it integrator} circuits, which take a voltage signal input and differentiate or integrate (respectively) that signal with respect to time.  This means if we have a voltage signal from a velocity sensor measuring the velocity of an object (such as a robotic arm, for example), we may send that signal through a differentiator circuit to obtain a voltage signal representing the robotic arm's acceleration, or we may send the velocity signal through a integrator circuit to obtain a voltage signal representing the robotic arm's position.

