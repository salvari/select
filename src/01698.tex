
%(BEGIN_QUESTION)
% Copyright 2003, Tony R. Kuphaldt, released under the Creative Commons Attribution License (v 1.0)
% This means you may do almost anything with this work of mine, so long as you give me proper credit

Early scientific researchers hypothesized that electricity was an invisible {\it fluid} that could move through certain substances.  Those substances "porous" to this "fluid" were called {\it conductors}, while substances impervious to this "fluid" were called {\it insulators}.

\vskip 10pt

We now know what electricity is composed of: tiny bits of matter, smaller than atoms.  What name do we give these tiny bits of matter?  How do these particles of matter relate to whole atoms?

\vskip 10pt

In terms of these tiny particles, what is the difference between the atoms of conductive substances versus the atoms of insulating substances?

\underbar{file 01698}
%(END_QUESTION)





%(BEGIN_ANSWER)

The tiny bits of matter that move through electrically conductive substances, comprising electricity, are called {\it electrons}.  Electrons are the outermost components of atoms:

$$\epsfbox{01698x01.eps}$$

Although electrons are present in all atoms, and therefore in all normal substances, the outer electrons in conductive substances are freer to leave the parent atoms than the electrons of insulating substances.  Such "free" electrons wander throughout the bulk of the substance randomly.  If directed by a force to drift in a consistent direction, this motion of free electrons becomes what we call {\it electricity}.

%(END_ANSWER)





%(BEGIN_NOTES)

It is worthy to note to your students that metallic substances -- the best naturally-occurring conductors -- are characterized by extremely mobile electrons.  In fact, solid-state physicists often refer to the free electrons in metals as a "gas" or a "sea," ironically paying homage to the "fluid" hypothesis of those early experimenters.

The specific details of why some atoms have freer electrons than others are extremely complex.  Suffice it to say, a knowledge of quantum physics is necessary to really grasp this basic phenomenon we call "electricity."  The subject becomes even more complex when we turn to superconductivity and semiconducting substances.

%INDEX% Electricity, composed of free electrons 
%INDEX% Electricity, "fluid" theory of 

%(END_NOTES)


