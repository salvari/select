
%(BEGIN_QUESTION)
% Copyright 2005, Tony R. Kuphaldt, released under the Creative Commons Attribution License (v 1.0)
% This means you may do almost anything with this work of mine, so long as you give me proper credit

Complete the truth tables for these two Boolean expressions:

% No blank lines allowed between lines of an \halign structure!
% I use comments (%) instead, so that TeX doesn't choke.

$$\hbox{Output } = \overline{A} + B$$

$$\vbox{\offinterlineskip
\halign{\strut
\vrule \quad\hfil # \ \hfil & 
\vrule \quad\hfil # \ \hfil & 
\vrule \quad\hfil # \ \hfil \vrule \cr
\noalign{\hrule}
%
% First row
A & B & Output \cr
%
\noalign{\hrule}
%
% Second row
0 & 0 &  \cr
%
\noalign{\hrule}
%
% Third row
0 & 1 &  \cr
%
\noalign{\hrule}
%
% Fourth row
1 & 0 &  \cr
%
\noalign{\hrule}
%
% Fifth row
1 & 1 &  \cr
%
\noalign{\hrule}
} % End of \halign 
}$$ % End of \vbox


\vskip 20pt


$$\hbox{Output } = A + \overline{A}B$$

$$\vbox{\offinterlineskip
\halign{\strut
\vrule \quad\hfil # \ \hfil & 
\vrule \quad\hfil # \ \hfil & 
\vrule \quad\hfil # \ \hfil \vrule \cr
\noalign{\hrule}
%
% First row
A & B & Output \cr
%
\noalign{\hrule}
%
% Second row
0 & 0 &  \cr
%
\noalign{\hrule}
%
% Third row
0 & 1 &  \cr
%
\noalign{\hrule}
%
% Fourth row
1 & 0 &  \cr
%
\noalign{\hrule}
%
% Fifth row
1 & 1 &  \cr
%
\noalign{\hrule}
} % End of \halign 
}$$ % End of \vbox

\underbar{file 02820}
%(END_QUESTION)





%(BEGIN_ANSWER)

% No blank lines allowed between lines of an \halign structure!
% I use comments (%) instead, so that TeX doesn't choke.

$$\hbox{Output } = \overline{A} + B$$

$$\vbox{\offinterlineskip
\halign{\strut
\vrule \quad\hfil # \ \hfil & 
\vrule \quad\hfil # \ \hfil & 
\vrule \quad\hfil # \ \hfil \vrule \cr
\noalign{\hrule}
%
% First row
A & B & Output \cr
%
\noalign{\hrule}
%
% Second row
0 & 0 & 1 \cr
%
\noalign{\hrule}
%
% Third row
0 & 1 & 1 \cr
%
\noalign{\hrule}
%
% Fourth row
1 & 0 & 0 \cr
%
\noalign{\hrule}
%
% Fifth row
1 & 1 & 1 \cr
%
\noalign{\hrule}
} % End of \halign 
}$$ % End of \vbox


\vskip 20pt


$$\hbox{Output } = A + \overline{A}B$$

$$\vbox{\offinterlineskip
\halign{\strut
\vrule \quad\hfil # \ \hfil & 
\vrule \quad\hfil # \ \hfil & 
\vrule \quad\hfil # \ \hfil \vrule \cr
\noalign{\hrule}
%
% First row
A & B & Output \cr
%
\noalign{\hrule}
%
% Second row
0 & 0 & 0 \cr
%
\noalign{\hrule}
%
% Third row
0 & 1 & 1 \cr
%
\noalign{\hrule}
%
% Fourth row
1 & 0 & 1 \cr
%
\noalign{\hrule}
%
% Fifth row
1 & 1 & 1 \cr
%
\noalign{\hrule}
} % End of \halign 
}$$ % End of \vbox

%(END_ANSWER)





%(BEGIN_NOTES)

Ask your students to explain exactly how they figured out the "Output" states to fill in the blanks in the truth tables, for the different input combinations.  Ask them also to compare and contrast this process with that of figuring out the truth table for a given logic gate circuit.

%INDEX% Boolean algebra, conversion of expression into truth table

%(END_NOTES)


