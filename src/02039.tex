
%(BEGIN_QUESTION)
% Copyright 2003, Tony R. Kuphaldt, released under the Creative Commons Attribution License (v 1.0)
% This means you may do almost anything with this work of mine, so long as you give me proper credit

One of the most important parameters for semiconductor components is the {\it power rating}.  Explain why power rating is such a critical parameter, especially compared with other types of electronic components (resistors, inductors, capacitors, etc.).

\underbar{file 02039}
%(END_QUESTION)





%(BEGIN_ANSWER)

Semiconductor devices tend to be especially sensitive to temperature.  Thus, it is paramount to maintain power dissipation below maximum rated levels.

\vskip 10pt

Challenge question: some semiconductor datasheets specify altitude values (height above sea level) along with the power ratings.  Explain why altitude has anything to do with the power rating of an electronic component.

%(END_ANSWER)





%(BEGIN_NOTES)

High temperature is the bane of most semiconductors.  A classic example of this, though a bit dated, is the temperature sensitivity of the original germanium transistors.  These devices were extremely sensitive to heat, and would fail rather quickly if allowed to overheat.  Solid state design engineers had to be very careful in the techniques they used for transistor circuits to ensure their sensitive germanium transistors would not suffer from "thermal runaway" and destroy themselves.

Silicon is much more forgiving then germanium, but heat is still a problem with these devices.  At the time of this writing (2004), there is promising developmental work on silicon carbide and gallium nitride transistor technology, which is able to function under {\it much} higher temperatures than silicon.

%INDEX% Semiconductor components, temperature effects on
%INDEX% Temperature effects on semiconductor components

%(END_NOTES)


