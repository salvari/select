
%(BEGIN_QUESTION)
% Copyright 2003, Tony R. Kuphaldt, released under the Creative Commons Attribution License (v 1.0)
% This means you may do almost anything with this work of mine, so long as you give me proper credit

Digital computers use a numeration system with a base of {\it two}, rather than a base of {\it ten} as we are accustomed to using.  It is much easier to engineer circuitry that counts in "binary" than it is to design circuits that count in any other base system.  Based on what you know of numeration systems, answer the following questions:

\medskip
\item{$\bullet$} How many different symbols (ciphers) are there in the binary numeration system?
\item{$\bullet$} What are the different place-weight values in the binary system?
\item{$\bullet$} How would you represent the number "seventeen" in binary?
\item{$\bullet$} In our base-ten (denary) numeration system, each character is called a "digit."  What is each character called in the binary numeration system?
\medskip

\underbar{file 01198}
%(END_QUESTION)





%(BEGIN_ANSWER)

There are only two valid ciphers in the binary system: 0 and 1.  Each successive place carries twice the "weight" of the one before it.  Seventeen = 10001 (in binary).  Each character in the binary system is called a "bit" rather than a "digit".

%(END_ANSWER)





%(BEGIN_NOTES)

Ask your students to hypothesize why binary is used in digital circuitry rather than base-ten (denary) numeration.  I've revealed that it's easier to build circuits representing quantities in binary than any other numerical base, but why?

%INDEX% Base-2 numeration system
%INDEX% Place weighting, base-2 numeration system

%(END_NOTES)


