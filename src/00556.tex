
%(BEGIN_QUESTION)
% Copyright 2003, Tony R. Kuphaldt, released under the Creative Commons Attribution License (v 1.0)
% This means you may do almost anything with this work of mine, so long as you give me proper credit

Semiconductor components such as diodes and transistors are easily damaged by the high temperatures of soldering, so care must be taken to protect these components during the soldering process.  One way to do this is to use a {\it heat sink} to draw heat away from the component without cooling the connection point too much.  Heat sinks made out of sheet metal may be temporarily clipped to the component leads, one at a time, to prevent the solder's high temperature from thermally conducting all the way to the component body:

$$\epsfbox{00556x01.eps}$$

In the absence of a formal heat sink, can you think of any ways to fashion your own "impromptu" heat sinks out of commonly available tools and/or objects?

\underbar{file 00556}
%(END_QUESTION)





%(BEGIN_ANSWER)

One trick that works well is to wrap a rubber band around the handles of a pair of needle-nose pliers, so the jaws clamp together on their own, then clip the jaws on to the component lead being soldered.

%(END_ANSWER)





%(BEGIN_NOTES)

This is an exercise in creative thinking, as well as an introduction to one of the practical concerns of soldering.  Even some passive components (electrolytic capacitors, for example) do not handle high temperatures well, and so learning to manage temperatures is an important skill when performing maintenance or prototype soldering.  See if you and your students can invent any other simple "heat sinks" for use when a formal heat sink is not available.

%INDEX% Heat sink, soldering
%INDEX% Soldering, heat sink

%(END_NOTES)


