
%(BEGIN_QUESTION)
% Copyright 2006, Tony R. Kuphaldt, released under the Creative Commons Attribution License (v 1.0)
% This means you may do almost anything with this work of mine, so long as you give me proper credit

A different way to view the functions of two-input logic gates is to think of them in terms of {\it signal controllers}, where the status of one input affects how the other input's signal passes through to the output.  The generic schematic diagram for this format is as such:

$$\epsfbox{03834x01.eps}$$

Identify the types of logic gates which do the following (there is more than one type of gate for each of the following rules!):

\medskip
\item{$\bullet$} $B = A$ when Control is high
\item{$\bullet$} $B = A$ when Control is low
\item{$\bullet$} $B = \overline{A}$ when Control is high
\item{$\bullet$} $B = \overline{A}$ when Control is low
\medskip

Also, explain how an understanding of this can be helpful in troubleshooting faulted logic gates.

\underbar{file 03834}
%(END_QUESTION)





%(BEGIN_ANSWER)

\medskip
\item{$\bullet$} $B = A$ when Control is high: {\bf AND} gate and {\bf XNOR} gate.
\item{$\bullet$} $B = A$ when Control is low: {\bf OR} gate and {\bf XOR} gate.
\item{$\bullet$} $B = \overline{A}$ when Control is high: {\bf NAND} gate and {\bf XOR} gate.
\item{$\bullet$} $B = \overline{A}$ when Control is low: {\bf NOR} gate and {\bf XNOR} gate.
\medskip

\vskip 10pt

Follow-up question: explain why XOR and XNOR gates are so useful as signal controllers.

%(END_ANSWER)





%(BEGIN_NOTES)

This is a very useful way to think of the different logic gate types, as it is often required to use a gate as a controlled buffer or controlled inverter.

%INDEX% Truth tables for different gate types

%(END_NOTES)


