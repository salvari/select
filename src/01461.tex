
%(BEGIN_QUESTION)
% Copyright 2003, Tony R. Kuphaldt, released under the Creative Commons Attribution License (v 1.0)
% This means you may do almost anything with this work of mine, so long as you give me proper credit

The fact that switched capacitor networks can behave equivalently to resistors is exploited in a variety of integrated circuits.  At first, this may seem strange, as switched capacitor networks generally require at least two switching transistors and a two-phase, non-overlapping clock (in addition to the capacitor itself) in order to function, which seems like a lot of peripheral circuitry compared to a single resistor.  What possible advantage is there to using switched capacitor networks in integrated circuits instead of resistors?  Support your answer with research, if possible.

\underbar{file 01461}
%(END_QUESTION)





%(BEGIN_ANSWER)

Switched capacitor networks are not just resistor equivalents, they are {\it variable} resistor equivalents.  Also, these networks actually tend to be smaller than integrated circuit resistors, and are less prone to drift.

There are other advantages to switched capacitor networks, but these are just some of the basic reasons behind their prevalence in modern integrated circuits.

%(END_ANSWER)





%(BEGIN_NOTES)

Ask your students to explain why switched capacitor networks are less prone to drift than resistors constructed on a semiconductor substrate.  Just focus on one source of drift, such as temperature, to simplify the topic.  What effect does temperature have on a semiconducting resistor, and why?  What effect does temperature have on a capacitor built of semiconductor layers separated by an insulating layer, and why?  With discrete components, are capacitors more stable over time than resistors?  Why or why not?

%INDEX% Switched capacitor network, resistor equivalent

%(END_NOTES)


