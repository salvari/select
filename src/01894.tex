
%(BEGIN_QUESTION)
% Copyright 2003, Tony R. Kuphaldt, released under the Creative Commons Attribution License (v 1.0)
% This means you may do almost anything with this work of mine, so long as you give me proper credit

By visual inspection, determine which of the following waveforms contain even-numbered harmonics:

$$\epsfbox{01894x01.eps}$$

Note that only one cycle is shown for each waveform.  Remember that we're dealing with continuous waveforms, endlessly repeating, and not single cycles as you see here.

\underbar{file 01894}
%(END_QUESTION)





%(BEGIN_ANSWER)

The following waveforms contain even-numbered harmonics: {\bf B}, {\bf C}, {\bf D}, {\bf F}, and {\bf I}.  The rest only contain odd harmonics of the fundamental.

%(END_ANSWER)





%(BEGIN_NOTES)

Ask your students how they were able to discern the presence of even-numbered harmonics by visual inspection.  This typically proves difficult for some of my students whose spatial-relations skills are weak.  These students need some sort of algorithmic (step-by-step) procedure to see what other students see immediately, and discussion time is a great opportunity for students to share technique.

Mathematically, this symmetry is defined as such:

$$f(t) = -f \left(t + {T \over 2} \right)$$

\noindent
Where,

$f(t) = $ Function of waveform with time as the independent variable

$t = $ Time

$T = $ Period of waveform, in same units of time as $t$

%INDEX% Harmonics, detecting presence of even-numbered by symmetry

%(END_NOTES)


