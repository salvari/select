
%(BEGIN_QUESTION)
% Copyright 2003, Tony R. Kuphaldt, released under the Creative Commons Attribution License (v 1.0)
% This means you may do almost anything with this work of mine, so long as you give me proper credit

You should know that a capacitor is formed by two conductive plates separated by an electrically insulating material.  As such, there is no "ohmic" path for electrons to flow between the plates.  This may be vindicated by an ohmmeter measurement, which tells us a capacitor has (nearly) infinite resistance once it is charged to the ohmmeter's full output voltage.

Explain then, how a capacitor is able to continuously pass {\it alternating} current, even though it cannot continuously pass DC.

\underbar{file 01553}
%(END_QUESTION)





%(BEGIN_ANSWER)

I'll let you figure this out on your own!

%(END_ANSWER)





%(BEGIN_NOTES)

This feature of capacitors is extremely useful in electronic circuitry.  Your students will find many applications of it later on in their studies!

%INDEX% Capacitors, current "through"
%INDEX% Capacitor check with ohmmeter

%(END_NOTES)


