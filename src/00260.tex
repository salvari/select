
%(BEGIN_QUESTION)
% Copyright 2003, Tony R. Kuphaldt, released under the Creative Commons Attribution License (v 1.0)
% This means you may do almost anything with this work of mine, so long as you give me proper credit

If the motion of a conductor through a magnetic field induces a voltage in that conductor, it stands to reason that a conductive fluid moving through a pipe can also generate a voltage, if properly exposed to a magnetic field.  Draw a picture showing the necessary orientation of the pipe, the magnetic field, and the electrodes intercepting the induced voltage.

\underbar{file 00260}
%(END_QUESTION)





%(BEGIN_ANSWER)

$$\epsfbox{00260x01.eps}$$

%(END_ANSWER)





%(BEGIN_NOTES)

This question really tests students' comprehension of the orthogonal relationships between magnetic flux, conductor motion, and induced voltage.  Additionally, it reveals a novel method of producing electricity: {\it magnetohydrodynamics}.

There are a few interesting applications of magnetohydrodynamics, including power generation and flow measurement.  Discuss these with your students if time permits.

%INDEX% Electromagnetic induction
%INDEX% Induction, electromagnetic
%INDEX% Magnetohydrodynamic effect

%(END_NOTES)


