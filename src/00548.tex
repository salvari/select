
%(BEGIN_QUESTION)
% Copyright 2003, Tony R. Kuphaldt, released under the Creative Commons Attribution License (v 1.0)
% This means you may do almost anything with this work of mine, so long as you give me proper credit

A {\it strain gauge} is a device used to measure the strain (compression or expansion) of a solid object by producing a resistance change proportional to the amount of strain:

$$\epsfbox{00548x01.eps}$$

The bridge circuit is supposed to respond to changes in specimen strain, but explain what will happen to the voltage measured across this bridge circuit ($V_{AB}$) if the specimen's temperature increases (with no stress applied), assuming that the bridge begins in a balanced condition with no strain on the gauge, at room temperature.  Assume a positive $\alpha$ value for the strain gauge conductors.

What does this indicate about the effectiveness of this device as a strain-measuring instrument?

\underbar{file 00548}
%(END_QUESTION)





%(BEGIN_ANSWER)

If the specimen heats up, a voltage will develop between points {\bf A} and {\bf B}, with {\bf A} being positive and {\bf B} being negative.

%(END_ANSWER)





%(BEGIN_NOTES)

Be sure to have your students explain how they arrived at their answers for polarity across the voltmeter terminals.

Ask your students whether or not the fact of the circuit's sensitivity to temperature invalidates its use as a strain-measuring system.  Is it impossible to obtain a reliable measurement of strain, if we know temperature also affects the circuit output voltage?  How could we compensate for the effects of temperature on the system?

%INDEX% Bridge circuit, used to measure strain
%INDEX% Strain gauge

%(END_NOTES)


