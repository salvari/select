
%(BEGIN_QUESTION)
% Copyright 2005, Tony R. Kuphaldt, released under the Creative Commons Attribution License (v 1.0)
% This means you may do almost anything with this work of mine, so long as you give me proper credit

$$\epsfbox{03178x01.eps}$$

\underbar{file 03178}
\vfil \eject
%(END_QUESTION)





%(BEGIN_ANSWER)

Use circuit simulation software to verify your predicted and measured parameter values.

%(END_ANSWER)





%(BEGIN_NOTES)

I recommend a supply voltage of 12 volts, a potentiometer value of 10 k$\Omega$, a capacitor value of 0.1 $\mu$F, and a loading resistor ($R_1$) of 1 M$\Omega$.  Use a DMM so as to not load the circuit any more than necessary.  If you wish to choose different capacitor/resistor values, I strongly suggest choosing them such that the time constant ($\tau$) of the circuit significantly faster than 1 second.

An extension of this exercise is to incorporate troubleshooting questions.  Whether using this exercise as a performance assessment or simply as a concept-building lab, you might want to follow up your students' results by asking them to predict the consequences of certain circuit faults.

%INDEX% Assessment, performance-based (Rate of change indicator circuit)

%(END_NOTES)


