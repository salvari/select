
%(BEGIN_QUESTION)
% Copyright 2003, Tony R. Kuphaldt, released under the Creative Commons Attribution License (v 1.0)
% This means you may do almost anything with this work of mine, so long as you give me proper credit

Ideally, should an ammeter have a very low input resistance, or a very high input resistance (input resistance being the amount of electrical resistance intrinsic to the meter, as measured between its test leads)?  Explain your answer.

\underbar{file 00728}
%(END_QUESTION)





%(BEGIN_ANSWER)

Ideally, an ammeter should have the least amount of input resistance possible.  This is important when using it to measure current in circuits containing little resistance.

%(END_ANSWER)





%(BEGIN_NOTES)

The answer to this question is related to the very important principle of {\it meter loading}.  Technicians, especially, have to be very aware of meter loading, and how erroneous measurements may result from it.  The answer is also related to how ammeters are connected with the circuits under test: {\it always in series!}

%(END_NOTES)


