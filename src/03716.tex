
%(BEGIN_QUESTION)
% Copyright 2005, Tony R. Kuphaldt, released under the Creative Commons Attribution License (v 1.0)
% This means you may do almost anything with this work of mine, so long as you give me proper credit

Predict how this circuit will be affected as a result of the following faults.  Consider each fault independently (i.e. one at a time, no multiple faults):

$$\epsfbox{03716x01.eps}$$

\medskip
\item{$\bullet$} Transistor $Q_1$ fails open (drain-to-source):
\vskip 5pt
\item{$\bullet$} Transistor $Q_1$ fails shorted (drain-to-source):
\vskip 5pt
\item{$\bullet$} Resistor $R_1$ fails open:
\medskip

For each of these conditions, explain {\it why} the resulting effects will occur.

\underbar{file 03716}
%(END_QUESTION)





%(BEGIN_ANSWER)

\medskip
\item{$\bullet$} Transistor $Q_1$ fails open (drain-to-source): {\it Motor refuses to run.}
\vskip 5pt
\item{$\bullet$} Transistor $Q_1$ fails shorted (drain-to-source): {\it Motor runs all the time and will not turn off.}
\vskip 5pt
\item{$\bullet$} Resistor $R_1$ fails open: {\it Motor runs when switch is pressed, takes a long time to turn off when switch is released.}
\medskip

%(END_ANSWER)





%(BEGIN_NOTES)

The purpose of this question is to approach the domain of circuit troubleshooting from a perspective of knowing what the fault is, rather than only knowing what the symptoms are.  Although this is not necessarily a realistic perspective, it helps students build the foundational knowledge necessary to diagnose a faulted circuit from empirical data.  Questions such as this should be followed (eventually) by other questions asking students to identify likely faults based on measurements.

%INDEX% Troubleshooting, predicting effects of fault in simple transistor motor-control circuit

%(END_NOTES)


