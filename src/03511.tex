
%(BEGIN_QUESTION)
% Copyright 2005, Tony R. Kuphaldt, released under the Creative Commons Attribution License (v 1.0)
% This means you may do almost anything with this work of mine, so long as you give me proper credit

A {\it helical resonator} is a special type of band-pass filter commonly used in VHF and UHF radio receiver circuitry.  Such a device is made up of multiple metal cavities, each containing a helix (coil) of wire connected to the cavity at one end and free at the other.  Slots cut between the cavities permits coupling between the coils, with the input at one extreme end and the output at the other:

$$\epsfbox{03511x01.eps}$$

The above illustration shows a three-stage helical resonator, with adjustable metal plates at the top of each helix for tuning.  Draw a schematic representation of this resonator, and explain where the capacitance comes from that allows each of the coils to form a resonant circuit.

\underbar{file 03511}
%(END_QUESTION)





%(BEGIN_ANSWER)

$$\epsfbox{03511x02.eps}$$

\vskip 10pt

Follow-up question: why do you suppose multiple stages of tuned ("tank") circuits would be necessary in a high-quality tuner circuit?  Why not just use a single tank circuit as a filter?  Would that not be simpler and less expensive?

%(END_ANSWER)





%(BEGIN_NOTES)

If students have a difficult time seeing where the capacitance comes from, remind them that we are dealing with {\it very} high frequencies here, and that air between metal parts is a sufficient dielectric to create the needed capacitance.

The coupling between coils may be a bit more difficult to grasp, especially if your students have not yet studied mutual inductance.  Suffice it to say that energy is transferred between coils with little loss at high frequencies, permitting an RF signal to enter at one end of the resonator and exit out the other without any wires physically connecting the stages together.

%INDEX% Helical resonator

%(END_NOTES)


