
%(BEGIN_QUESTION)
% Copyright 2003, Tony R. Kuphaldt, released under the Creative Commons Attribution License (v 1.0)
% This means you may do almost anything with this work of mine, so long as you give me proper credit

A paradoxical property of resonant circuits is that they have the ability to produce quantities of voltage or current (in series and parallel circuits, respectively) exceeding that output by the power source itself.  This is due to the cancellation of inductive and capacitive reactances at resonance.

Not all resonant circuits are equally effective in this regard.  One way to quantify the performance of resonant circuits is to assign them a {\it quality factor}, or $Q$ rating.  This rating is very similar to the one given inductors as a measure of their reactive "purity."

Suppose we have a resonant circuit operating at its resonant frequency.  How may we calculate the $Q$ of this operating circuit, based on empirical measurements of voltage or current?  There are {\it two} answers to this question: one for series circuits and one for parallel circuits.

\underbar{file 01390}
%(END_QUESTION)





%(BEGIN_ANSWER)

$$Q_{series} = {E_C \over E_{source}} = {E_L \over E_{source}}$$

$$Q_{parallel} = {I_C \over I_{source}} = {I_L \over I_{source}}$$

\vskip 10pt

Follow-up question: what unique safety hazards may high-Q resonant circuits pose?

%(END_ANSWER)





%(BEGIN_NOTES)

Ask your students to determine which type of danger(s) are posed by high-Q series and parallel resonant circuits, respectively.  The answer to this question may seem paradoxical at first: that series resonant circuits whose overall impedance is nearly zero can manifest large voltage drops, while parallel resonant circuits whose overall impedance is nearly infinite can manifest large currents.

%INDEX% Quality factor of LC resonant circuit, defined

%(END_NOTES)


