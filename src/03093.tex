
%(BEGIN_QUESTION)
% Copyright 2005, Tony R. Kuphaldt, released under the Creative Commons Attribution License (v 1.0)
% This means you may do almost anything with this work of mine, so long as you give me proper credit

Suppose you were given the following two equations and asked to find solutions for $x$ and $y$ that will satisfy {\it both} at the same time:

$$y + x = 8$$

$$y - x = 3$$

Now, you know that we may do anything we want to either equation as long as we do the same thing to both sides (on either side of the "equal" sign).  This is the basic rule we follow when manipulating an equation to solve for a particular variable.  For example, we may take the equation $y + x = 8$ and subtract $x$ from both sides to yield an equation expressed in terms of $y$:

$$\epsfbox{03093x01.eps}$$

Following the same principle, we may take two equations and combine them either by adding or subtracting both sides.  For example, we may take the equation $y - x = 3$ and add both sides of it to the respective sides of the first equation $y + x = 8$:

$$\epsfbox{03093x02.eps}$$

What beneficial result comes of this action?  In other words, how can I use this new equation $2y = 11$ to solve for values of $x$ and $y$ that satisfy both of the original equations?

\underbar{file 03093}
%(END_QUESTION)





%(BEGIN_ANSWER)

We may use the result ($2y = 11$) to solve for a value of $y$, which when substituted into either of the original equations may be used to solve for a value of $x$ to satisfy {\it both} equations at the same time.

%(END_ANSWER)





%(BEGIN_NOTES)

While not intuitively obvious to most people, the technique of adding two entire equations to each other for the purpose of eliminating a variable is not only possible to do, but very powerful when looking for solutions to satisfy both original equations.  Discuss with your students why it is allowable for us to add $y - x$ to $y + x$ and to add $3$ to $8$. to yield the equation $2y = 11$.

%INDEX% Simultaneous equations
%INDEX% Systems of linear equations

%(END_NOTES)


