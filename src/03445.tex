
%(BEGIN_QUESTION)
% Copyright 2005, Tony R. Kuphaldt, released under the Creative Commons Attribution License (v 1.0)
% This means you may do almost anything with this work of mine, so long as you give me proper credit

A {\it solenoid valve} is a mechanical shutoff device actuated by electricity.  An electromagnet coil produces an attractive force on an iron "armature" which then either opens or closes a valve mechanism to control the flow of some fluid.  Shown here are two different types of illustrations, both showing a solenoid valve:

$$\epsfbox{03445x01.eps}$$

Some solenoid valves are constructed in such a way that the coil assembly may be removed from the valve body, separating these two pieces so that maintenance work may be done on one without interfering with the other.  Of course, this means the valve mechanism will no longer be actuated by the magnetic field, but at least one piece may be worked upon without having to remove the other piece from whatever it may be connected to:

$$\epsfbox{03445x02.eps}$$

This is commonly done when replacement of the valve mechanism is needed.  First, the coil is lifted off the valve mechanism, then the maintenance technician is free to remove the valve body from the pipes and replace it with a new valve body.  Lastly, the coil is re-installed on the new valve body and the solenoid is once more ready for service, all without having to electrically disconnect the coil from its power source.

\goodbreak
However, if this is done while the coil is energized, it will overheat and burn up in just a few minutes.  To prevent this from happening, the maintenance technicians have learned to insert a steel screwdriver through the center hole of the coil while it is removed from the valve body, like this:

$$\epsfbox{03445x03.eps}$$

With the steel screwdriver shank taking the place of the iron armature inside the valve body, the coil will not overheat and burn up even if continually powered.  Explain the nature of the problem (why the coil tends to burn up when separated from the valve body) and also why a screwdriver put in place of the iron armature works to prevent this from happening.

\underbar{file 03445}
%(END_QUESTION)





%(BEGIN_ANSWER)

With the iron armature no longer in the center of the solenoid coil, the coil's inductance -- and therefore its inductive reactance to AC -- dramatically diminishes unless the armature is replaced by something else ferromagnetic.

%(END_ANSWER)





%(BEGIN_NOTES)

When I first saw this practice in action, I almost fell over laughing.  It is both practical and ingenious, as well as being an excellent example of variable inductance (and inductive reactance) arising from varying reluctance.

%INDEX% Solenoid valve, construction

%(END_NOTES)


