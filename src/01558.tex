
%(BEGIN_QUESTION)
% Copyright 2003, Tony R. Kuphaldt, released under the Creative Commons Attribution License (v 1.0)
% This means you may do almost anything with this work of mine, so long as you give me proper credit

Using a computer or graphing calculator, plot the sum of these two sine waves:

$$\epsfbox{01558x01.eps}$$

Hint: you will need to enter equations into your plotting device that look something like this:

\vskip 10pt

\goodbreak

{\tt y1 = sin x}

\vskip 10pt

{\tt y2 = 2 * sin (x + 90)}

\vskip 10pt

{\tt y3 = y1 + y2}

\vskip 10pt

Note: the second equation assumes your calculator has been set up to calculate trigonometric functions in angle units of {\it degrees} rather than {\it radians}.  If you wish to plot these same waveforms (with the same phase shift shown) using radians as the unit of angle measurement, you must enter the second equation as follows:

\vskip 10pt

{\tt y2 = 2 * sin (x + 1.5708)}

\vskip 10pt

\underbar{file 01558}
%(END_QUESTION)





%(BEGIN_ANSWER)

$$\epsfbox{01558x02.eps}$$

\vskip 10pt

Follow-up question: note that the sum of the 1-volt wave and the 2-volt wave does {\it not} equate to a 3-volt wave!  Explain why.

%(END_ANSWER)





%(BEGIN_NOTES)

Graphing calculators are excellent tools to use for learning experiences such as this.  In far less time than it would take to plot a third sine wave by hand, students may see the sinusoidal sum for themselves.

The point of this question is to get students thinking about how it is possible for sinusoidal voltages to not add up as one might expect.  This is very important, because it indicates simple arithmetic processes like addition will not be as simple in AC circuits as it was in DC circuits, due to phase shift.  Be sure to emphasize this point to your students.

%INDEX% Addition of two sine waves, with phase shift

%(END_NOTES)


