
%(BEGIN_QUESTION)
% Copyright 2003, Tony R. Kuphaldt, released under the Creative Commons Attribution License (v 1.0)
% This means you may do almost anything with this work of mine, so long as you give me proper credit

What is required to make a {\it Shockley diode} or {\it DIAC} begin conducting current?  What condition(s) have to be met in order for electrical conduction to occur through one of these devices?

Also, explain what must be done to stop the flow of electric current through a Shockley diode or a DIAC.

\underbar{file 01090}
%(END_QUESTION)





%(BEGIN_ANSWER)

Turn on: voltage drop across device must exceed a certain threshold voltage (the {\it breakover voltage}) before conduction occurs.

\vskip 10pt

Turn off: current through the device must be brought to a minimum level before the device stops conducting ({\it low-current dropout}).

%(END_ANSWER)





%(BEGIN_NOTES)

Although the answer may seem obvious to many, it is worthwhile to ask your students how the behavior of a Shockley diode compares to that of a normal (rectifying) diode.  The fact that the Shockley diode is called a "diode" at all may have fooled some of your students into thinking that it behaves much like a normal diode.

Ask you students to explain how these two devices (Shockley diodes versus rectifying diodes) are similar.  In what ways are they different?

Another good discussion question to bring up is the difference between a {\it Shockley} diode and a {\it Schottky} diode.  Although the names are very similar, the two devices are definitely not!

%INDEX% DIAC
%INDEX% Shockley diode

%(END_NOTES)


