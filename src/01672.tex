
%(BEGIN_QUESTION)
% Copyright 2003, Tony R. Kuphaldt, released under the Creative Commons Attribution License (v 1.0)
% This means you may do almost anything with this work of mine, so long as you give me proper credit

An oscilloscope is connected to a battery of unknown voltage.  The result is a straight line on the display:

$$\epsfbox{01672x01.eps}$$

Assuming the oscilloscope display has been properly "zeroed" and the vertical sensitivity is set to 5 volts per division, determine the voltage of the battery.

\underbar{file 01672}
%(END_QUESTION)





%(BEGIN_ANSWER)

The battery voltage is slightly greater than 6.5 volts.

%(END_ANSWER)





%(BEGIN_NOTES)

Measuring voltage on an oscilloscope display is very similar to measuring voltage on an analog voltmeter.  The mathematical relationship between scale divisions and range is much the same.  This is one reason I encourage students to use analog multimeters occasionally in their labwork, if for no other reason than to preview the principles of oscilloscope scale interpretation.

%INDEX% Oscilloscope, measuring voltage (quantitative)

%(END_NOTES)


