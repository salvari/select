
%(BEGIN_QUESTION)
% Copyright 2003, Tony R. Kuphaldt, released under the Creative Commons Attribution License (v 1.0)
% This means you may do almost anything with this work of mine, so long as you give me proper credit

What does it mean, in general terms, to {\it encode} something?  Conversely, what does it mean to {\it decode} something?  Perhaps the most common context for these terms to be used is cryptography (code-making and code-breaking), but they also find application in common digital circuits.

\underbar{file 01407}
%(END_QUESTION)





%(BEGIN_ANSWER)

To {\it encode} something is to convert an unambiguous piece of information into a form of code that is not so clearly understood.  To {\it decode} is to perform the reverse operation: translating a code back into an unambiguous form.

%(END_ANSWER)





%(BEGIN_NOTES)

This question gets students thinking about encoding and decoding in general terms -- terms which they are probably already familiar with.  This is a good first step in instruction, to identify a well-known context for a new subject, so students have an easier time relating to it.

%INDEX% Encoding, digital (defined)
%INDEX% Decoding, digital (defined)

%(END_NOTES)


