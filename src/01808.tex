
%(BEGIN_QUESTION)
% Copyright 2003, Tony R. Kuphaldt, released under the Creative Commons Attribution License (v 1.0)
% This means you may do almost anything with this work of mine, so long as you give me proper credit

Calculate the final value of current through the inductor with the switch in the left-hand position (assuming that many time constants' worth of time have passed):

$$\epsfbox{01808x01.eps}$$

Now, assume that the switch is {\it instantly} moved to the right-hand position.  How much voltage will the inductor initially drop?

$$\epsfbox{01808x02.eps}$$

Explain why this voltage is so very different from the supply voltage.  What practical uses might a circuit such as this have?

\underbar{file 01808}
%(END_QUESTION)





%(BEGIN_ANSWER)

$I_{switch-left} = 2 \hbox{ mA}$

$V_{switch-right} = 182 \hbox{ V}$

\vskip 10pt

Follow-up question: suppose this circuit were built and tested, and it was found that the voltage developed across the inductor at the moment the switch moved to the right-hand position far exceeded 182 volts.  Identify some possible problems in the circuit which could account for this excessive voltage.

%(END_ANSWER)





%(BEGIN_NOTES)

The main purpose of this question is to get students thinking in terms of "initial" and "final" values for LR circuits, and how one might calculate them.  It is largely a conceptual question, with just a bit of calculation necessary.

One practical application of this circuit is for "stepping up" DC voltage.  The circuit topology shown in the question is that of an {\it inverting} converter circuit.  This form of DC-DC converter circuit has the ability to step voltage up {\it or} down, depending on the duty cycle of the switch's oscillation.

%INDEX% Inductor, used to produce high-voltage transients

%(END_NOTES)


