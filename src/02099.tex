
%(BEGIN_QUESTION)
% Copyright 2004, Tony R. Kuphaldt, released under the Creative Commons Attribution License (v 1.0)
% This means you may do almost anything with this work of mine, so long as you give me proper credit

When drawing phasor diagrams, there is a standardized orientation for all angles used to ensure consistency between diagrams.  This orientation is usually referenced to a set of perpendicular lines, like the $x$ and $y$ axes commonly seen when graphing algebraic functions:

$$\epsfbox{02099x01.eps}$$

The intersection of the two axes is called the {\it origin}, and straight horizontal to the right is the definition of zero degrees (0$^{o}$).  Thus, a phasor with a magnitude of 6 and an angle of 0$^{o}$ would look like this on the diagram:

$$\epsfbox{02099x02.eps}$$

Draw a phasor with a magnitude of 10 and an angle of 100 degrees on the above diagram, as well as a phasor with a magnitude of 2 and an angle of -45 degrees.  Label what directions 90$^{o}$, 180$^{o}$, and 270$^{o}$ would indicate on the same diagram.

\underbar{file 02099}
%(END_QUESTION)





%(BEGIN_ANSWER)

$$\epsfbox{02099x03.eps}$$

%(END_ANSWER)





%(BEGIN_NOTES)

Graph paper, a ruler, and a protractor may be helpful for your students as they begin to draw and interpret phasor diagrams.  Even if they have no prior knowledge of trigonometry or phasors, they should still be able to graphically represent simple phasor systems and even solve for resultant phasors.

%INDEX% Phasor diagram, standard angle orientations

%(END_NOTES)


