
%(BEGIN_QUESTION)
% Copyright 2003, Tony R. Kuphaldt, released under the Creative Commons Attribution License (v 1.0)
% This means you may do almost anything with this work of mine, so long as you give me proper credit

Two important variables in magnetic circuit analysis are $B$ and $H$.  Explain what these two variables represent, in terms of the more fundamental magnetic quantities of MMF (${\cal F}$) and flux ($\Phi$), and relate them, if possible, to electrical quantities.  Also, determine the units of measurement for these two variables, in the CGS, SI, and English measurement systems.

\underbar{file 00681}
%(END_QUESTION)





%(BEGIN_ANSWER)

$$\epsfbox{00681x01.eps}$$

Field intensity ($H$) is also known as "magnetizing force," and is the amount of MMF per unit length of the magnetic flux path.  Flux density ($B$) is the amount of magnetic flux per unit area.

%(END_ANSWER)





%(BEGIN_NOTES)

Although the equivalent electrical variables to field intensity and flux density are not commonly used in electronics, they do exist!  Ask your students if anyone was able to determine what these variables are.  Also, ask them where they were able to obtain the information on magnetic quantities and units of measurement.

You should mention to your students that the SI units are considered to be the most "modern" of those shown here, the SI system being the international standard for metric units in all applications.

%INDEX% Magnetic units of measurement
%INDEX% Units of measurement, magnetism

%(END_NOTES)


