
%(BEGIN_QUESTION)
% Copyright 2006, Tony R. Kuphaldt, released under the Creative Commons Attribution License (v 1.0)
% This means you may do almost anything with this work of mine, so long as you give me proper credit

Euler's constant, that ubiquitous constant in mathematics symbolized by the letter $e$, may be found as the result of the following mathematical series:

$$e = \sum_{n=0}^{\infty} {1 \over {n!}}$$

Approximate the value of $e$ in steps, using the following table:

% No blank lines allowed between lines of an \halign structure!
% I use comments (%) instead, so that TeX doesn't choke.

$$\vbox{\offinterlineskip
\halign{\strut
\vrule \quad\hfil # \ \hfil & 
\vrule \quad\hfil # \ \hfil & 
\vrule \quad\hfil # \ \hfil & 
\vrule \quad\hfil # \ \hfil \vrule \cr
\noalign{\hrule}
%
% First row
$n$ & $n!$ & $1 / n!$ & $\approx e$ \cr
%
\noalign{\hrule}
%
% Another row
0 &  &  &  \cr
%
\noalign{\hrule}
%
% Another row
1 &  &  &  \cr
%
\noalign{\hrule}
%
% Another row
2 &  &  &  \cr
%
\noalign{\hrule}
%
% Another row
3 &  &  &  \cr
%
\noalign{\hrule}
%
% Another row
4 &  &  &  \cr
%
\noalign{\hrule}
%
% Another row
5 &  &  &  \cr
%
\noalign{\hrule}
%
% Another row
6 &  &  &  \cr
%
\noalign{\hrule}
%
% Another row
7 &  &  &  \cr
%
\noalign{\hrule}
} % End of \halign 
}$$ % End of \vbox

Also, express this series as a partial sum up to $n = 7$.

\underbar{file 04036}
%(END_QUESTION)





%(BEGIN_ANSWER)

% No blank lines allowed between lines of an \halign structure!
% I use comments (%) instead, so that TeX doesn't choke.

$$\vbox{\offinterlineskip
\halign{\strut
\vrule \quad\hfil # \ \hfil & 
\vrule \quad\hfil # \ \hfil & 
\vrule \quad\hfil # \ \hfil & 
\vrule \quad\hfil # \ \hfil \vrule \cr
\noalign{\hrule}
%
% First row
$n$ & $n!$ & $1 / n!$ & $\approx e$ \cr
%
\noalign{\hrule}
%
% Another row
0 & 1 & 1 & 1 \cr
%
\noalign{\hrule}
%
% Another row
1 & 1 & 1 & 2 \cr
%
\noalign{\hrule}
%
% Another row
2 & 2 & 0.5 & 2.5 \cr
%
\noalign{\hrule}
%
% Another row
3 & 6 & 0.16667 & 2.66667 \cr
%
\noalign{\hrule}
%
% Another row
4 & 24 & 0.04167 & 2.70833 \cr
%
\noalign{\hrule}
%
% Another row
5 & 120 & 0.00833 & 2.71667 \cr
%
\noalign{\hrule}
%
% Another row
6 & 720 & 0.00139 & 2.71806 \cr
%
\noalign{\hrule}
%
% Another row
7 & 5040 & 0.00020 & 2.71825 \cr
%
\noalign{\hrule}
} % End of \halign 
}$$ % End of \vbox

\vskip 10pt

Shown here is the partial sum up to $n=7$:

$${1 \over 0!} + {1 \over 1!} + {1 \over 2!} + {1 \over 3!} + {1 \over 4!} + {1 \over 5!} + {1 \over 6!} + {1 \over 7!}$$

\vskip 10pt

. . . and again, in slightly different form . . .

$$1 + 1 + {1 \over 2} + {1 \over 6} + {1 \over 24} + {1 \over 120} + {1 \over 720} + {1 \over 5040}$$

%(END_ANSWER)





%(BEGIN_NOTES)

It should go without saying that students should consult their electronic calculators (or a textbook) to see what the actual value of $e$ is, and compare that to their partial sum approximation.

This questions provides students with the opportunity to see where $e$ comes from, and also to explore the behavior of a series by calculating partial sums.  Many students find it fascinating to see the sequence of partial sums converge on the true value of $e$, seeing how this formerly mysterious constant can actually be calculated using nothing more than repeated arithmetic.

%INDEX% Series (mathematical), for Euler's constant (e)

%(END_NOTES)


