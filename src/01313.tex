
%(BEGIN_QUESTION)
% Copyright 2003, Tony R. Kuphaldt, released under the Creative Commons Attribution License (v 1.0)
% This means you may do almost anything with this work of mine, so long as you give me proper credit

{\it Factoring} is a powerful simplification technique in Boolean algebra, just as it is in real-number algebra.  Show how you can use factoring to help simplify the following Boolean expressions:

$$C + CD$$

$$A\overline{B}C +  A\overline{B} \> \overline{C}$$

$$XY\overline{Z} + XYZ + XYW$$

$$\overline{D}EF + AB + \overline{D}E + 0 + ABC$$

\underbar{file 01313}
%(END_QUESTION)





%(BEGIN_ANSWER)

You will be expected to show your work (including all factoring) in your answers!

$$C + CD = C$$

$$A\overline{B}C +  A\overline{B} \> \overline{C} = A\overline{B}$$

$$XY\overline{Z} + XYZ + XYW = XY$$

$$\overline{D}EF + AB + \overline{D}E + 0 + ABC = AB + \overline{D}E$$

%(END_ANSWER)





%(BEGIN_NOTES)

For some reason, many of my students (who enter my course weak in algebra skills) generally seem to have a lot of trouble with factoring, be it Boolean algebra or regular algebra.  This is unfortunate, as factoring is a powerful analytical tool.  The "trick," if there is any such thing, is recognizing common variables in different product terms, and identifying which of them should be factored out to reduce the expression most efficiently.

Like all challenging things, factoring takes time and practice to learn.  There are no shortcuts, really.

%INDEX% Boolean algebra, factoring

%(END_NOTES)


