
%(BEGIN_QUESTION)
% Copyright 2003, Tony R. Kuphaldt, released under the Creative Commons Attribution License (v 1.0)
% This means you may do almost anything with this work of mine, so long as you give me proper credit

When a resistor conducts electric current, its temperature increases.  Explain how this phenomenon is significant to the application of resistors in electric circuits.  In other words, why would we care about a resistor's temperature increasing?

Also, what does this indicate about the technical ratings of resistors?  Aside from having a specific resistance rating (i.e. a certain number of {\it ohms}), what other rating is important for proper selection of resistors in electric circuits?

\underbar{file 00214}
%(END_QUESTION)





%(BEGIN_ANSWER)

The heating effect of electricity through a resistance is significant because that resistance may be damaged by excessive temperature.  To avoid damage, resistors must be selected to be able to withstand a certain amount of heating.

%(END_ANSWER)





%(BEGIN_NOTES)

Students need to understand that resistance alone does not fully dictate the selection of a resistor for electrical service.  Failure to heed the dissipation ratings of a resistor can result in catastrophic failure!

A good follow-up question to this is to ask what the {\it unit of measurement} is for this kind of thermal rating.

%INDEX% Joule's Law, conceptual
%INDEX% Resistor, power rating

%(END_NOTES)


