
%(BEGIN_QUESTION)
% Copyright 2004, Tony R. Kuphaldt, released under the Creative Commons Attribution License (v 1.0)
% This means you may do almost anything with this work of mine, so long as you give me proper credit

A very important concept in electronics is that of {\it negative feedback}.  This is an extremely important concept to grasp, as a great many electronic systems exploit this principle for their operation and cannot be properly understood without a comprehension of it.

However important negative feedback might be, it is not the easiest concept to understand.  In fact, it is quite a conceptual leap for some.  The following is a list of examples -- some electronic, some not -- exhibiting negative feedback:

\medskip
\goodbreak
\item{$\bullet$} A voltage regulating circuit
\item{$\bullet$} An auto-pilot system for an aircraft or boat
\item{$\bullet$} A thermostatic temperature control system ("thermostat")
\item{$\bullet$} Emitter resistor in a BJT amplifier circuit
\item{$\bullet$} Lenz's Law demonstration (magnetic damping of a moving object)
\item{$\bullet$} Body temperature of a mammal
\item{$\bullet$} Natural regulation of prices in a free market economy (Adam Smith's "invisible hand")
\item{$\bullet$} A scientist learning about the behavior of a natural system through experimentation.
\medskip

For each case, answer the following questions:

\medskip
\goodbreak
\item{$\bullet$} What variable is being stabilized by negative feedback?
\item{$\bullet$} How is the feedback taking place (step by step)?
\item{$\bullet$} What would the system's response be like if negative feedback were not present?
\medskip

\underbar{file 02253}
%(END_QUESTION)





%(BEGIN_ANSWER)

I will provide answers for only one of the examples, the voltage regulator:

\medskip
\goodbreak
\item{$\bullet$} What variable is being stabilized by negative feedback? 
\item{ }{\it Output voltage.}
\vskip 5pt
\item{$\bullet$} How is the feedback taking place (step by step)?
\item{ }{\it When output voltage rises, the system takes action to drop more voltage internally, leaving less for the output.}
\vskip 5pt
\item{$\bullet$} What would the system's response be like if negative feedback were not present?
\item{ }{\it Without negative feedback, the output voltage would rise and fall directly with the input voltage, and inversely with the load current.}
\medskip

%(END_ANSWER)





%(BEGIN_NOTES)

It is difficult to overstate the importance of grasping negative feedback in the study of electronics.  So many different types of systems depend on it for their operation that it cannot be omitted from any serious electronics curriculum.  Yet I see many textbooks fail to explore this principle in adequate depth, or discuss it in a mathematical sense only where students are likely to miss the basic concept because they will be too focused on solving the equations.

%INDEX% Negative feedback, examples of

%(END_NOTES)


