
%(BEGIN_QUESTION)
% Copyright 2003, Tony R. Kuphaldt, released under the Creative Commons Attribution License (v 1.0)
% This means you may do almost anything with this work of mine, so long as you give me proper credit

In high-speed digital circuits, a very important logic gate parameter is {\it propagation delay}: the delay time between a change-of-state on a gate's input and the corresponding change-of-state on that gate's output.  Consult a manufacturer's datasheet for any TTL logic gate and report the typical propagation delay times published there.

Also, explain what causes propagation delay in logic gates.  Why isn't the change in output state instantaneous when an input changes states?

\underbar{file 01264}
%(END_QUESTION)





%(BEGIN_ANSWER)

I'll leave the research of specific propagation time delay figures up to you!  The reason propagation delay exists is because transistors cannot turn on and turn off instantaneously.  In bipolar transistors, this is due to the time required to establish minority carrier flow within the base layer of the transistor (to turn it on), and to "sweep out" those minority charge carriers out of the base (to turn it off).

\vskip 10pt

Follow-up questions: What difference is there between high-to-low output transitions versus low-to-high output transitions for the gate you researched?  Which transition is faster?

%(END_ANSWER)





%(BEGIN_NOTES)

I purposely omitted answers for this question, not only because I want students to do the research on their own, but also because it makes it more interesting when students consult different datasheets and derive different answers (for different logic "families")!

%INDEX% Propagation delay, cause of (TTL)

%(END_NOTES)


