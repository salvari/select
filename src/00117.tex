
%(BEGIN_QUESTION)
% Copyright 2003, Tony R. Kuphaldt, released under the Creative Commons Attribution License (v 1.0)
% This means you may do almost anything with this work of mine, so long as you give me proper credit

A 22-gauge metal wire three feet in length contains approximately $28.96 \times 10^{21}$ "free" electrons within its volume.  Suppose this wire is placed in an electric circuit conducting a current equal to $6.25 \times 10^{18}$ electrons per second.  That is, if you were able to choose a spot along the length of this wire and were able to count electrons as they drifted by that spot, you would tally 6,250,000,000,000,000,000 electrons passing by each second.  (This is a reasonable rate for electric current in a wire of this size.)

Calculate the average velocity of electrons through this wire.

\underbar{file 00117}
%(END_QUESTION)





%(BEGIN_ANSWER)

Average electron velocity = 0.000647 feet per second, or $6.47 \times 10^{-4}$ ft/s.  This is very slow: only 0.00777 inches per second, or 0.197 millimeters per second!

%(END_ANSWER)





%(BEGIN_NOTES)

Despite the rapid progression of the {\it effects} of electron motion throughout a circuit (i.e. approximately the speed of light), the actual electron velocity is extremely slow by comparison.

Base figures used in this calculation are as follows:

\vskip 10pt

\item {$\bullet$} Number of free electrons per cubic meter of metal (an example taken from \underbar{Encyclopedia Brittanica} 15th edition, 1983, volume 6, page 551) = $10^{29}$ electrons per $m^3$.  The metal type was not specified.
\item {$\bullet$} 22 gauge wire has a diameter of 0.025 inches.

\vskip 10pt

Questions like this may be challenging to students without a strong math or science background.  One problem-solving strategy I have found very useful is to simplify the terms of a problem until a solution becomes obvious, then use that simplified example to establish a pattern (equation) for obtaining a solution given {\it any} initial parameters.  For instance, what would be the average electron velocity if the current were $28.96 \times 10^{21}$ electrons per second, the same figure as the number of free electrons residing in the wire?  Obviously, the flow velocity would be one wire length per second, or 3 feet per second.  Now, alter the current rate so that it is something closer to the one given in the problem ($6.25 \times 10^{18}$), but yet still simple enough to calculate mentally.  Say, half the first rate: $14.48 \times 10^{21}$ electrons per second.  Obviously, with a flow rate half as much, the velocity will be half as well: 1.5 feet per second instead of 3 feet per second.  A few iterations of this technique should reveal a pattern for solution:

$$v = 3 {I \over Q}$$

\noindent
Where,

$v =$ Average electron velocity (feet per second)

$I =$ Electric current (electrons per second)

$Q =$ Number of electrons contained in wire

\vskip 10pt

It is also very helpful to have knowledgeable students demonstrate their solution techniques in front of the class so that others may learn novel methods of problem-solving.

%INDEX% Electricity, velocity of

%(END_NOTES)


