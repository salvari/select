
%(BEGIN_QUESTION)
% Copyright 2003, Tony R. Kuphaldt, released under the Creative Commons Attribution License (v 1.0)
% This means you may do almost anything with this work of mine, so long as you give me proper credit

In simple AC power systems, one of the two conductors is typically called the {\it hot}, while the other conductor is typically called the {\it neutral}.  What distinguishes the "hot" conductor from the "neutral" conductor in such a system?  In other words, what exactly determines whether a conductor will be called either "hot" or "neutral"?

$$\epsfbox{00312x01.eps}$$

\underbar{file 00312}
%(END_QUESTION)





%(BEGIN_ANSWER)

The "neutral" conductor in an AC power system is always made electrically common with the earth, while the non-grounded conductor in an AC power system is called the "hot" conductor:

$$\epsfbox{00312x02.eps}$$

\vskip 10pt

Follow-up question \#1: in a power circuit such as this, there is typically a single fuse or circuit breaker for overcurrent protection.  Identify the {\it best} location to place this fuse or circuit breaker in the circuit.

\vskip 10pt

Follow-up question \#2: explain how to use a multimeter to identify the "hot" and "neutral" conductors of a power system such as this.

%(END_ANSWER)





%(BEGIN_NOTES)

Ask your students what safety issues surround the "hot" and "neutral" conductors, respectively.  What do the names imply about relative hazard, and why is this?  Ask them to explain how these safety considerations impact the placement of the overcurrent protection device.

%INDEX% Hot conductor
%INDEX% Neutral conductor

%(END_NOTES)


