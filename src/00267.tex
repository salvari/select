
%(BEGIN_QUESTION)
% Copyright 2003, Tony R. Kuphaldt, released under the Creative Commons Attribution License (v 1.0)
% This means you may do almost anything with this work of mine, so long as you give me proper credit

Many electronic circuits use what is called a {\it split} or a {\it dual} power supply:

$$\epsfbox{00267x01.eps}$$

Determine what a digital voltmeter would indicate if connected between the following points:

\medskip
\item{$\bullet$} Red lead on "A", black lead on ground
\item{$\bullet$} Red lead on "B", black lead on ground
\item{$\bullet$} Red lead on "A", black lead on "B"
\item{$\bullet$} Red lead on "B", black lead on "A"
\medskip

NOTE: {\it in electronic systems, "ground" is often not associated with an actual earth-soil contact.  It usually only refers to a common point of reference somewhere in the circuit used to take voltage measurements.  This allows us to specify voltages at single points in the circuit, with the implication that "ground" is the \underbar{other} point for the voltmeter to connect to.}

\vskip 10pt

\underbar{file 00267}
%(END_QUESTION)





%(BEGIN_ANSWER)

\medskip
\item{$\bullet$} Red lead on "A", black lead on ground ({\it Digital voltmeter reads +15 volts})
\item{$\bullet$} Red lead on "B", black lead on ground ({\it Digital voltmeter reads -15 volts})
\item{$\bullet$} Red lead on "A", black lead on "B" ({\it Digital voltmeter reads +30 volts})
\item{$\bullet$} Red lead on "B", black lead on "A" ({\it Digital voltmeter reads -30 volts})
\medskip

%(END_ANSWER)





%(BEGIN_NOTES)

This question may be easily answered with only a voltmeter, two batteries, and a single "jumper" wire to connect the two batteries in series.  It does not matter if the batteries are 15 volts each!  The fundamental principle may still be investigated with batteries of any voltage, so this is a very easy demonstration to set up during discussion time.

%INDEX% Split power supply
%INDEX% Dual power supply
%INDEX% Ground, defined for electronic circuits
%INDEX% Voltmeter usage

%(END_NOTES)


