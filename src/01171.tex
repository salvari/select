
%(BEGIN_QUESTION)
% Copyright 2005, Tony R. Kuphaldt, released under the Creative Commons Attribution License (v 1.0)
% This means you may do almost anything with this work of mine, so long as you give me proper credit

This is a very common opamp oscillator circuit, technically of the {\it relaxation} type:

$$\epsfbox{01171x01.eps}$$

Explain how this circuit works, and what waveforms will be measured at points A and B.  Be sure to make reference to {\it RC time constants} in your explanation.

\underbar{file 01171}
%(END_QUESTION)





%(BEGIN_ANSWER)

You will measure a sawtooth-like waveform at point A, and a square wave at point B.

\vskip 10pt

Challenge question: explain how you might go about calculating the frequency of such a circuit, based on what you know about RC time constant circuits.  Assume that the opamp can swing its output rail-to-rail, for simplicity.

%(END_ANSWER)





%(BEGIN_NOTES)

This circuit is best understood by building and testing.  If you use large capacitor values and/or a large-value resistor in the capacitor's current path, the oscillation will be slow enough to analyze with a voltmeter rather than an oscilloscope.

%INDEX% Relaxation oscillator, opamp

%(END_NOTES)


