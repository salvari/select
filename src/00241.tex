
%(BEGIN_QUESTION)
% Copyright 2003, Tony R. Kuphaldt, released under the Creative Commons Attribution License (v 1.0)
% This means you may do almost anything with this work of mine, so long as you give me proper credit

Devices called {\it step regulators} are sometimes used in electrical power distribution systems to boost or suppress line voltages, typically to a maximum of 10\%:

$$\epsfbox{00241x01.eps}$$

The "Line" side of this autotransformer is the source of power, and the "Load" side connects to those devices consuming electrical power.  The "Reversing switch" alters the transformer's function from boost to buck.

Determine which switch position is "boost" and which is "buck," and then determine which way the moving contact (the arrow symbol) needs to be moved in order to increase the voltage output, and to decrease the voltage output.
 
\underbar{file 00241}
%(END_QUESTION)





%(BEGIN_ANSWER)

When the reversing switch is in the left position, the regulator output voltage is reduced.  When the switch is in the right position, the regulator output voltage is increased.  Moving the contact further to the right increases the amount of boost or buck, depending on which position the reversing switch is in.

Incidentally, the zig-zag pattern of the windings is not standard for electronic schematic diagrams, though it is more common in electrical power system diagrams.  I'll let you research how they draw resistors in these types of diagrams!

%(END_ANSWER)





%(BEGIN_NOTES)

The solution to this question may be easier for some to recognize if the regulator circuit is re-drawn using normal electronic schematic diagrams.  Another tip is to simplify the schematic diagram by removing all the different tap positions, and draw the circuit using the full series winding.

It is important for students to realize that there are different {\it types} of electrical schematic diagrams.  Power diagram symbolism is often confusing to people familiar with electronic schematic symbolism, and visa-versa.

%INDEX% Regulator, AC line
%INDEX% Step regulator

%(END_NOTES)


