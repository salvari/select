
%(BEGIN_QUESTION)
% Copyright 2003, Tony R. Kuphaldt, released under the Creative Commons Attribution License (v 1.0)
% This means you may do almost anything with this work of mine, so long as you give me proper credit

Most ammeters contain {\it fuses} inside to provide protection for the person using the ammeter, as well as for the ammeter mechanism itself.  Voltmeters generally do {\it not} contain fuses inside, because they are unnecessary.

Explain why ammeters use fuses and voltmeters do not?  What is it about the nature of an ammeter and how it is used that makes fuse-protection necessary?

\underbar{file 00274}
%(END_QUESTION)





%(BEGIN_ANSWER)

The purpose of the fuse inside the ammeter is to "blow" (open-circuit) in the event of excessive current through the meter.

%(END_ANSWER)





%(BEGIN_NOTES)

This is a very important question for several reasons.  First, it alerts students that ammeters are generally fuse-protected, and therefore they must be aware that this fuse could "blow" and render the ammeter inoperative.  Secondly, it highlights one of the major differences between voltmeters and ammeters, and that is the amount of current typically handled by the respective meter types.

%INDEX% Ammeter, internal fuse
%INDEX% Ammeter usage

%(END_NOTES)


