
%(BEGIN_QUESTION)
% Copyright 2003, Tony R. Kuphaldt, released under the Creative Commons Attribution License (v 1.0)
% This means you may do almost anything with this work of mine, so long as you give me proper credit

Suppose you were to build this circuit and take measurements of current through the resistor and voltage across the resistor:

$$\epsfbox{00057x01.eps}$$

Recording these numerical values in a table, the results look something like this:

\vskip 5pt

\settabs \+ \hskip 1.5in & \quad XXXXXXX \quad & \quad XXXXXXX \quad & \cr
\+ & \hfill Current \quad & \quad Voltage \quad & \cr
\+ & \hfill 0.22 A    \quad & \quad 0.66 V  \quad & \cr
\+ & \hfill 0.47 A    \quad & \quad 1.42 V  \quad & \cr
\+ & \hfill 0.85 A    \quad & \quad 2.54 V  \quad & \cr
\+ & \hfill 1.05 A    \quad & \quad 3.16 V  \quad & \cr
\+ & \hfill 1.50 A \quad & \quad 4.51 V \quad & \cr
\+ & \hfill 1.80 A \quad & \quad 5.41 V \quad & \cr
\+ & \hfill 2.00 A \quad & \quad 5.99 V \quad & \cr
\+ & \hfill 2.51 A    \quad & \quad 7.49 V \quad & \cr

\vskip 5pt

Plot these figures on the following graph:

$$\epsfbox{00057x02.eps}$$

What mathematical relationship do you see between voltage and current in this simple circuit?

\underbar{file 00057}
%(END_QUESTION)





%(BEGIN_ANSWER)

This is an example of a {\it linear} function: where the plot describing the data set traces a straight line on a graph.  From this line, and also from the numerical figures, you should be able to discern a constant ratio between voltage and current.

%INDEX% Ohm's Law
%INDEX% Graphing

%(END_ANSWER)





%(BEGIN_NOTES)

The raw data figures were made intentionally "noisy" in this problem to simulate the types of measurement errors encountered in real life.  One tool which helps overcome interpretational problems resulting from noise like this is graphing.  Even with noise present, the linearity of the function is quite clearly revealed.

Your students should learn to make graphs as tools for their own understanding of data.  When relationships between numbers are represented in graphical form, it lends another mode of expression to the data, helping people to apprehend patterns easier than by reviewing rows and columns of numbers.

%(END_NOTES)


