
%(BEGIN_QUESTION)
% Copyright 2003, Tony R. Kuphaldt, released under the Creative Commons Attribution License (v 1.0)
% This means you may do almost anything with this work of mine, so long as you give me proper credit

In extrinsic semiconductors, what are {\it majority carriers} and how do they differ from {\it minority carriers}?

\underbar{file 00912}
%(END_QUESTION)





%(BEGIN_ANSWER)

"Majority carriers" are those charge carriers existing by the purposeful addition of doping elements to the material.  "Minority carriers" are the opposite type of charge carrier, inhabiting a semiconductor only because it is impossible to completely eliminate the impurities generating them.

%(END_ANSWER)





%(BEGIN_NOTES)

We speak of pure semiconductor materials, and of "doping" pieces of semiconductor material with just the right quantity and type(s) of dopants, but the reality is it is impossible to assure perfect quality control, and thus there {\it will} be other impurities in any semiconductor sample.  

Ask your students to specifically identify the majority and minority charge carriers for "P" and "N" type extrinsic semiconductors.  In each case, are they electrons, or holes?

%INDEX% Charge carriers
%INDEX% Majority carriers
%INDEX% Minority carriers

%(END_NOTES)


