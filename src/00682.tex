
%(BEGIN_QUESTION)
% Copyright 2003, Tony R. Kuphaldt, released under the Creative Commons Attribution License (v 1.0)
% This means you may do almost anything with this work of mine, so long as you give me proper credit

If we were to graph the "response" of a resistor to various levels of applied voltage, we would obtain a plot that looks like this:

$$\epsfbox{00682x01.eps}$$

If we were to graph the "response" of a ferromagnetic sample to various levels of applied magnetomotive force, we would obtain a plot that looks something like this:

$$\epsfbox{00682x02.eps}$$

What does this graph indicate to you, as compared to the graph for a resistor's characteristics?  What is the significance of this with regard to magnetic "circuit" analysis?

\underbar{file 00682}
%(END_QUESTION)





%(BEGIN_ANSWER)

The MMF/flux plot for a ferromagnetic material is quite {\it nonlinear}, unlike the plot for an electrical resistor.

%(END_ANSWER)





%(BEGIN_NOTES)

Ask your students to identify {\it resistance} on the V/I graph shown in the question.  Where on that graph is resistance represented?  Your more mathematically astute students will recognize (or perhaps recall from earlier discussions) that the slope of the plot indicates the resistance of the circuit.  The less resistance, the steeper the plot (at least in this case, where current is on the vertical axis and voltage on the horizontal).  At any point on the plot, the slope is the same, indicating that resistance does not change over a wide range of voltage and current.

Now, direct their attention to the MMF/flux graph.  Where is {\it reluctance} indicated on the graph?  What conclusion may we form regarding reluctance in a magnetic circuit, from analyzing the shape of the MMF/flux curve shown?  At what point is reluctance the greatest?  At what point is it the least?

%INDEX% Magnetism, nonlinearity in flux/MMF relationship for ferrous materials

%(END_NOTES)


