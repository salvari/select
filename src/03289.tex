
%(BEGIN_QUESTION)
% Copyright 2005, Tony R. Kuphaldt, released under the Creative Commons Attribution License (v 1.0)
% This means you may do almost anything with this work of mine, so long as you give me proper credit

All other factors being equal, which possesses a greater potential for inducing harmful electric shock, DC electricity or AC electricity at a frequency of 60 Hertz?  Be sure to back up your answer with research data!

\underbar{file 03289}
%(END_QUESTION)





%(BEGIN_ANSWER)

From a perspective of inducing electric shock, AC has been experimentally proven to possess greater hazard than DC (all other factors being equal).  See the research of Charles Dalziel for supporting data.

%(END_ANSWER)





%(BEGIN_NOTES)

A common misconception is that DC is more capable of delivering a harmful electric shock than AC, all other factors being equal.  In fact, this is something I used to teach myself (because I had heard it numerous times from others) before I discovered the research of Charles Dalziel.  One of the explanations used to support the myth of DC being more dangerous is that DC has the ability to cause muscle tetanus more readily than AC.  However, at 60 Hertz, the reversals of polarity occur so quickly that no human muscle could relax fast enough to enable a shock victim to release a "hot" wire anyway, so that fact that AC stops multiple times per second is of no benefit to the victim.

Do not be surprised if some students react unfavorably to the answer given here!  The myth that DC is more dangerous than AC is so prevalent, especially among people who have a little background knowledge of the subject, that to counter it is to invite dispute.  This is why I included the condition of supporting any answer by research data in the question.

This just goes to show that there are many misconceptions about electricity that are passed from person to person as "common knowledge" which have little or no grounding in fact (lightning never strikes twice in the same spot, electricity takes the least path of resistance, high current is more dangerous than high voltage, etc., etc.).  The study of electricity and electronics is science, and in science experimental data is our sole authority.  One of the most important lessons to be learned in science is that human beings have a propensity to believe things which are not true, and some will continue to defend false beliefs even in the face of conclusive evidence.

%INDEX% AC versus DC, relative shock hazard of
%INDEX% DC versus AC, relative shock hazard of
%INDEX% Electric shock

%(END_NOTES)

