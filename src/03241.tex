
%(BEGIN_QUESTION)
% Copyright 2005, Tony R. Kuphaldt, released under the Creative Commons Attribution License (v 1.0)
% This means you may do almost anything with this work of mine, so long as you give me proper credit

Use Th\'evenin's Theorem to determine a simple equivalent circuit for the 10 volt source, the 5 k$\Omega$ resistor, and the 2.2 k$\Omega$ resistor; then calculate the voltage across the 1 k$\Omega$ load:

$$\epsfbox{03241x01.eps}$$

This exercise may seem pointless, as it is easy enough to obtain the answer simply by series-parallel analysis of this circuit.  However, there is definite value in determining a Th\'evenin equivalent circuit for the voltage source, 5 k$\Omega$ resistor, and 2.2 k$\Omega$ resistor if the load voltage for {\it several different} values of load resistance needs to be predicted.  Explain why Th\'evenin's Theorem becomes the more efficient way to predict load voltage if multiple load resistor values are considered.

\underbar{file 03241}
%(END_QUESTION)





%(BEGIN_ANSWER)

$V_{load}$ = 1.209 volts

%(END_ANSWER)





%(BEGIN_NOTES)

Ask your students to show how (step-by-step) they arrived at the equivalent circuit, prior to calculating load voltage.

In case students are unfamiliar with the "double-chevron" symbols in the schematic diagram, let them know that these represent male/female connector pairs.

%INDEX% Thevenin's Theorem, applied to resistive network

%(END_NOTES)


