
%(BEGIN_QUESTION)
% Copyright 2005, Tony R. Kuphaldt, released under the Creative Commons Attribution License (v 1.0)
% This means you may do almost anything with this work of mine, so long as you give me proper credit

Transistors act as {\it controlled current sources}.  That is, with a fixed control signal in, they tend to regulate the amount of current going through them.  Design an experimental circuit to prove this tendency of transistors.  In other words, how could you {\it demonstrate} this current-regulating behavior to be a fact?

\underbar{file 03722}
%(END_QUESTION)





%(BEGIN_ANSWER)

$$\epsfbox{03722x01.eps}$$

\vskip 10pt

Procedure: measure the voltage dropped across $R_C$ while varying $V_{CC}$, for several different values of $I_B$ (inferred by measuring voltage drop across $R_B$).

%(END_ANSWER)





%(BEGIN_NOTES)

Here, students must think like an experimental scientist: figuring out how to prove the relative stability of one variable despite variations in another, while holding the controlling variable constant.  Encourage your students to actually build this circuit!

%INDEX% BJT, as current regulator

%(END_NOTES)


