
%(BEGIN_QUESTION)
% Copyright 2005, Tony R. Kuphaldt, released under the Creative Commons Attribution License (v 1.0)
% This means you may do almost anything with this work of mine, so long as you give me proper credit

A passive differentiator is used to "shorten" the pulse width of a square wave by sending the differentiated signal to a "level detector" circuit, which outputs a "high" signal (+5 volts) whenever the input exceeds 3.5 volts and a "low" signal (0 volts) whenever the input drops below 3.5 volts:

$$\epsfbox{03653x01.eps}$$

Each time the differentiator's output voltage signal spikes up to +5 volts and quickly decays to 0 volts, it causes the level detector circuit to output a narrow voltage pulse, which is what we want.

Calculate how wide this final output pulse will be if the input (square wave) frequency is 2.5 kHz.

\underbar{file 03653}
%(END_QUESTION)





%(BEGIN_ANSWER)

$t_{pulse}$ = 11.77 $\mu$s 

%(END_ANSWER)





%(BEGIN_NOTES)

This question requires students to calculate a length of time in an RC circuit, given specific voltage levels and component values.  It is a very practical question, as it may be necessary to build or troubleshoot such a circuit some day!

%INDEX% Passive differentiator circuit, calculating output pulse time

%(END_NOTES)


