
%(BEGIN_QUESTION)
% Copyright 2006, Tony R. Kuphaldt, released under the Creative Commons Attribution License (v 1.0)
% This means you may do almost anything with this work of mine, so long as you give me proper credit

Predict how the operation of this astable 555 timer circuit will be affected as a result of the following faults.  Specifically, identify what will happen to the capacitor voltage ($V_{C1}$) and the output voltage ($V_{out}$) for each fault condition.  Consider each fault independently (i.e. one at a time, no multiple faults):

$$\epsfbox{03890x01.eps}$$

\medskip
\item{$\bullet$} Resistor $R_1$ fails open:
\vskip 5pt
\item{$\bullet$} Solder bridge (short) across resistor $R_1$:
\vskip 5pt
\item{$\bullet$} Resistor $R_2$ fails open:
\vskip 5pt
\item{$\bullet$} Solder bridge (short) across resistor $R_2$:
\vskip 5pt
\item{$\bullet$} Capacitor $C_1$ fails shorted:
\medskip

For each of these conditions, explain {\it why} the resulting effects will occur.

\underbar{file 03890}
%(END_QUESTION)





%(BEGIN_ANSWER)

\medskip
\item{$\bullet$} Resistor $R_1$ fails open: {\it Capacitor voltage holds at last value, output voltage holds at last value.}
\vskip 5pt
\item{$\bullet$} Solder bridge (short) across resistor $R_1$: {\it Timer IC will become damaged at the first discharge cycle.}
\vskip 5pt
\item{$\bullet$} Resistor $R_2$ fails open: {\it Capacitor voltage holds at last value, output voltage holds at last value.}
\vskip 5pt
\item{$\bullet$} Solder bridge (short) across resistor $R_2$: {\it Oscillation frequency nearly doubles, and the duty cycle increases to nearly 100\%.}
\vskip 5pt
\item{$\bullet$} Capacitor $C_1$ fails shorted: {\it Capacitor voltage goes to 0 volts DC, output voltage stays "high".}
\medskip

%(END_ANSWER)





%(BEGIN_NOTES)

The purpose of this question is to approach the domain of circuit troubleshooting from a perspective of knowing what the fault is, rather than only knowing what the symptoms are.  Although this is not necessarily a realistic perspective, it helps students build the foundational knowledge necessary to diagnose a faulted circuit from empirical data.  Questions such as this should be followed (eventually) by other questions asking students to identify likely faults based on measurements.

%INDEX% Troubleshooting, predicting effects of fault in 555 timer circuit

%(END_NOTES)


