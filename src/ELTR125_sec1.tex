
\centerline{\bf ELTR 125 (Semiconductors 2), section 1} \bigskip 
 
\vskip 10pt

\noindent
{\bf Recommended schedule}

\vskip 5pt

%%%%%%%%%%%%%%%
\hrule \vskip 5pt
\noindent
\underbar{Day 1}

\hskip 10pt Topics: {\it The BJT as a linear amplifier, current mirrors}
 
\hskip 10pt Questions: {\it 1 through 15}
 
\hskip 10pt Lab Exercise: {\it Current mirror (question 76)}
 
%INSTRUCTOR \hskip 10pt {\bf Socratic Electronics animation: BJT characteristic curve sketching}

%INSTRUCTOR \hskip 10pt {\bf Demo: Show current mirror circuit regulating current through a rheostat}

\vskip 10pt
%%%%%%%%%%%%%%%
\hrule \vskip 5pt
\noindent
\underbar{Day 2}

\hskip 10pt Topics: {\it Common-collector BJT amplifiers, transistor amplifier biasing}
 
\hskip 10pt Questions: {\it 16 through 30}
 
\hskip 10pt Lab Exercise: {\it Signal biasing/unbiasing network (question 77)}
 
%INSTRUCTOR \hskip 10pt {\bf MIT 6.002 video clip: Disk 1, Lecture 7; Diode V/I curve w/ bias 21:59 to 25:40}
 
%INSTRUCTOR \hskip 10pt {\bf Demo: Show CC amplifier operation w/ and w/o biasing}

\vskip 10pt
%%%%%%%%%%%%%%%
\hrule \vskip 5pt
\noindent
\underbar{Day 3}

\hskip 10pt Topics: {\it Common-emitter BJT amplifiers}
 
\hskip 10pt Questions: {\it 31 through 45}
 
\hskip 10pt Lab Exercise: {\it Common-collector amplifier circuit (question 78)}
 
%INSTRUCTOR \hskip 10pt {\bf Demo: Show CE amplifier operation w/ and w/o biasing}

\vskip 10pt
%%%%%%%%%%%%%%%
\hrule \vskip 5pt
\noindent
\underbar{Day 4}

\hskip 10pt Topics: {\it Common-base BJT amplifiers, gain expressed in decibels}
 
\hskip 10pt Questions: {\it 46 through 60}
 
\hskip 10pt Lab Exercise: {\it Common-emitter amplifier circuit (question 79)}
 
%INSTRUCTOR \hskip 10pt {\bf Demo: Show Fluke 189 multimeter's decibel measurement function}

\vskip 10pt
%%%%%%%%%%%%%%%
\hrule \vskip 5pt
\noindent
\underbar{Day 5}

\hskip 10pt Topics: {\it Input and output impedances of amplifier circuits}
 
\hskip 10pt Questions: {\it 61 through 75}
 
\hskip 10pt Lab Exercise: {\it Common-base amplifier circuit (question 80)}
 
\vskip 10pt
%%%%%%%%%%%%%%%
\hrule \vskip 5pt
\noindent
\underbar{Day 6}

\hskip 10pt Exam 1: {\it includes Amplifier with specified voltage gain performance assessment}
 
\hskip 10pt Lab Exercise: {\it Troubleshooting practice (oscillator/amplifier circuit -- question 81)}
  
\vskip 10pt
%%%%%%%%%%%%%%%
\hrule \vskip 5pt
\noindent
\underbar{Troubleshooting practice problems}

\hskip 10pt Questions: {\it 84 through 93}
 
\vskip 10pt
%%%%%%%%%%%%%%%
\hrule \vskip 5pt
\noindent
\underbar{General concept practice and challenge problems}

\hskip 10pt Questions: {\it 94 through the end of the worksheet}
 
\vskip 10pt
%%%%%%%%%%%%%%%
\hrule \vskip 5pt
\noindent
\underbar{Impending deadlines}

\hskip 10pt {\bf Troubleshooting assessment (oscillator/amplifier) due at end of ELTR125, Section 3}

\hskip 10pt Question 82: Troubleshooting log
 
\hskip 10pt Question 83: Sample troubleshooting assessment grading criteria
 
\vskip 10pt
%%%%%%%%%%%%%%%










\vfil \eject

\centerline{\bf ELTR 125 (Semiconductors 2), section 1} \bigskip 
 
\vskip 10pt

\noindent
{\bf Skill standards addressed by this course section}

\vskip 5pt

%%%%%%%%%%%%%%%
\hrule \vskip 10pt
\noindent
\underbar{EIA {\it Raising the Standard; Electronics Technician Skills for Today and Tomorrow}, June 1994}

\vskip 5pt

\medskip
\item{\bf D} {\bf Technical Skills -- Discrete Solid-State Devices}
\item{\bf D.12} Understand principles and operations of single stage amplifiers.
\item{\bf D.13} Fabricate and demonstrate single stage amplifiers.
\item{\bf D.14} Troubleshoot and repair single stage amplifiers.
\medskip

\vskip 5pt

\medskip
\item{\bf B} {\bf Basic and Practical Skills -- Communicating on the Job}
\item{\bf B.01} Use effective written and other communication skills.  {\it Met by group discussion and completion of labwork.}
\item{\bf B.03} Employ appropriate skills for gathering and retaining information.  {\it Met by research and preparation prior to group discussion.}
\item{\bf B.04} Interpret written, graphic, and oral instructions.  {\it Met by completion of labwork.}
\item{\bf B.06} Use language appropriate to the situation.  {\it Met by group discussion and in explaining completed labwork.}
\item{\bf B.07} Participate in meetings in a positive and constructive manner.  {\it Met by group discussion.}
\item{\bf B.08} Use job-related terminology.  {\it Met by group discussion and in explaining completed labwork.}
\item{\bf B.10} Document work projects, procedures, tests, and equipment failures.  {\it Met by project construction and/or troubleshooting assessments.}
\item{\bf C} {\bf Basic and Practical Skills -- Solving Problems and Critical Thinking}
\item{\bf C.01} Identify the problem.  {\it Met by research and preparation prior to group discussion.}
\item{\bf C.03} Identify available solutions and their impact including evaluating credibility of information, and locating information.  {\it Met by research and preparation prior to group discussion.}
\item{\bf C.07} Organize personal workloads.  {\it Met by daily labwork, preparatory research, and project management.}
\item{\bf C.08} Participate in brainstorming sessions to generate new ideas and solve problems.  {\it Met by group discussion.}
\item{\bf D} {\bf Basic and Practical Skills -- Reading}
\item{\bf D.01} Read and apply various sources of technical information (e.g. manufacturer literature, codes, and regulations).  {\it Met by research and preparation prior to group discussion.}
\item{\bf E} {\bf Basic and Practical Skills -- Proficiency in Mathematics}
\item{\bf E.01} Determine if a solution is reasonable.
\item{\bf E.02} Demonstrate ability to use a simple electronic calculator.
\item{\bf E.05} Solve problems and [sic] make applications involving integers, fractions, decimals, percentages, and ratios using order of operations.
\item{\bf E.06} Translate written and/or verbal statements into mathematical expressions.
\item{\bf E.09} Read scale on measurement device(s) and make interpolations where appropriate.  {\it Met by oscilloscope usage.}
\item{\bf E.12} Interpret and use tables, charts, maps, and/or graphs.
\item{\bf E.13} Identify patterns, note trends, and/or draw conclusions from tables, charts, maps, and/or graphs.
\item{\bf E.15} Simplify and solve algebraic expressions and formulas.
\item{\bf E.16} Select and use formulas appropriately.
\item{\bf E.17} Understand and use scientific notation.
\medskip

%%%%%%%%%%%%%%%




\vfil \eject

\centerline{\bf ELTR 125 (Semiconductors 2), section 1} \bigskip 
 
\vskip 10pt

\noindent
{\bf Common areas of confusion for students}

\vskip 5pt


\hrule \vskip 5pt

\vskip 10pt

\noindent
{\bf Difficult concept: } {\it Inverting nature of common-emitter amplifier.}

Some students find it quite difficult to grasp why the DC output voltage of a common-emitter amplifier {\it decreases} as the DC input voltage level increases.  Step-by-step DC analysis of the circuit is the only remedy I have found to this conceptual block: getting students to carefully analyze what happens as voltages increase and decrease.

\vskip 10pt

\noindent
{\bf Difficult concept: } {\it Transistor biasing.}

Transistors (at least BJTs) are unilateral, DC-only devices, which leads to a problem if we wish to use them to amplify AC signals.  The way we usually get around this is to {\it bias} them with a DC voltage in order to "trick" them into staying in conduction through more of the AC cycle.  Realizing that biasing is nothing more than a clever trick used to make a DC-only device handle AC signals is a major step in understanding how it works.

