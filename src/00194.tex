
%(BEGIN_QUESTION)
% Copyright 2003, Tony R. Kuphaldt, released under the Creative Commons Attribution License (v 1.0)
% This means you may do almost anything with this work of mine, so long as you give me proper credit

A 470 $\mu$F capacitor is subjected to an applied voltage that changes at a rate of 200 volts per second.  How much current will there be "through" this capacitor?

Explain why I placed quotation marks around the word "through" in the previous sentence.  Why can't this word be used in its fullest sense when describing electric current in a capacitor circuit?

\underbar{file 00194}
%(END_QUESTION)





%(BEGIN_ANSWER)

This capacitor will have a constant current of 94 milliamps (mA) going "through" it.  The word "through" is placed in quotation marks because capacitors have no continuity.

%(END_ANSWER)





%(BEGIN_NOTES)

Don't give your students the equation with which to perform this calculation!  Let them find it on their own.  The ${dv \over dt}$ notation may be foreign to students lacking a strong mathematical background, but don't let this be an obstacle to learning!  Rather, use this as a way to introduce those students to the concept of {\it rates of change}, and to the calculus concept of the {\it derivative}.

%INDEX% Capacitance, voltage versus current in
%INDEX% Capacitors, current "through"

%(END_NOTES)


