
%(BEGIN_QUESTION)
% Copyright 2003, Tony R. Kuphaldt, released under the Creative Commons Attribution License (v 1.0)
% This means you may do almost anything with this work of mine, so long as you give me proper credit

$$\epsfbox{01939x01.eps}$$

\underbar{file 01939}
\vfil \eject
%(END_QUESTION)





%(BEGIN_ANSWER)

Use circuit simulation software to verify your predicted and measured parameter values.

%(END_ANSWER)





%(BEGIN_NOTES)

Use a variable-voltage, regulated power supply to supply any amount of DC voltage below 30 volts.  Specify standard resistor values, all between 1 k$\Omega$ and 100 k$\Omega$ (1k5, 2k2, 2k7, 3k3, 4k7, 5k1, 6k8, 10k, 22k, 33k, 39k 47k, 68k, etc.). 

This circuit produces nice, sharp-edged square wave signals at the transistor collector terminals when resistors $R_1$ and $R_4$ are substantially smaller than resistors $R_2$ and $R_3$.  This way, $R_2$ and $R_3$ dominate the capacitors' charging times, making calculation of duty cycle much more accurate.  Component values I've used with success are 1 k$\Omega$ for $R_1$ and $R_4$, 100 k$\Omega$ for $R_2$ and $R_3$, and 0.1 $\mu$F for $C_1$ and $C_2$.

An extension of this exercise is to incorporate troubleshooting questions.  Whether using this exercise as a performance assessment or simply as a concept-building lab, you might want to follow up your students' results by asking them to predict the consequences of certain circuit faults.

%INDEX% Assessment, performance-based (BJT multivibrator circuit, astable)

%(END_NOTES)


