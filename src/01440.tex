
%(BEGIN_QUESTION)
% Copyright 2003, Tony R. Kuphaldt, released under the Creative Commons Attribution License (v 1.0)
% This means you may do almost anything with this work of mine, so long as you give me proper credit

Determine whether the following recording devices are {\it random access} or {\it sequential access}, and discuss the advantage(s) of one type of access over the other:

\medskip
\item{$\bullet$} DVD (disk)
\item{$\bullet$} Audio tape cassette
\item{$\bullet$} CD-ROM (disk)
\item{$\bullet$} ROM memory chip
\item{$\bullet$} Vinyl phonograph record
\item{$\bullet$} Video tape cassette
\item{$\bullet$} Magnetic "hard" drive
\item{$\bullet$} Magnetic bubble memory
\item{$\bullet$} Paper tape (a long strip of tape with holes punched in it)
\item{$\bullet$} RAM memory chip 
\medskip

\underbar{file 01440}
%(END_QUESTION)





%(BEGIN_ANSWER)

\medskip
\item{$\bullet$} DVD (disk) -- {\it random access}
\item{$\bullet$} Audio tape cassette -- {\it sequential access}
\item{$\bullet$} CD-ROM (disk) -- {\it random access}
\item{$\bullet$} ROM memory chip -- {\it random access}
\item{$\bullet$} Vinyl phonograph record -- {\it random access}
\item{$\bullet$} Video tape cassette -- {\it sequential access}
\item{$\bullet$} Magnetic "hard" drive -- {\it random access}
\item{$\bullet$} Magnetic bubble memory -- {\it sequential access}
\item{$\bullet$} Paper tape (a long strip of tape with holes punched in it) -- {\it sequential access}
\item{$\bullet$} RAM memory chip -- {\it random access}
\medskip

Be prepared to discuss how each of these recording technologies works, and {\it why} each one is either random or sequential access.

%(END_ANSWER)





%(BEGIN_NOTES)

One of the purposes of this question is to get students to realize that "RAM" memory (solid-state, volatile memory "chips" in a computer) is not the only type of data storage device capable of randomly accessing its contents, and that the term "RAM" as it is commonly used is something of a misnomer.

%INDEX% Random access memory
%INDEX% Sequential access memory

%(END_NOTES)


