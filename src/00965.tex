
%(BEGIN_QUESTION)
% Copyright 2005, Tony R. Kuphaldt, released under the Creative Commons Attribution License (v 1.0)
% This means you may do almost anything with this work of mine, so long as you give me proper credit

A popular method of "reclaiming" some of the lost voltage gain resulting from the addition of an emitter resistor ($R_E$) to a common-emitter amplifier circuit is to connect a "bypass" capacitor in parallel with that resistor:

$$\epsfbox{00965x01.eps}$$

Explain why this technique works to increase the circuit's AC voltage gain, without leading to the problems associated with directly grounding the emitter.

\underbar{file 00965}
%(END_QUESTION)





%(BEGIN_ANSWER)

To an AC signal, a large capacitor "looks" like a lower impedance than the emitter resistor.  Usually, this capacitor is sized such that $X_C$ is very small.

The addition of a bypass capacitor maintains DC stability, because DC cannot go through the capacitor but must go through the emitter resistor ($R_E$) just as if the bypass capacitor were not there at all.

%(END_ANSWER)





%(BEGIN_NOTES)

This question provides a good opportunity to review capacitive reactance ($X_C$).  The polarized capacitor symbol hints at the capacitor's relatively large value, and your students should realize that a large capacitor's reactance will be relatively low to most AC signals.  An idea to help communicate the "bypass" concept is to have one of your students re-draw the circuit as "seen" from the perspective of an AC signal, not a DC signal.  With the capacitor effectively acting as a short-circuit to AC signals, what does the amplifier circuit look like to those signals?

%INDEX% Bypass capacitor, common-emitter amplifier

%(END_NOTES)


