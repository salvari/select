
%(BEGIN_QUESTION)
% Copyright 2004, Tony R. Kuphaldt, released under the Creative Commons Attribution License (v 1.0)
% This means you may do almost anything with this work of mine, so long as you give me proper credit

Describe what happens to the UJT as the potentiometer is slowly adjusted upward to provide a variable voltage at point {\bf A} in this circuit, starting from 0 volts and ending at the trigger voltage $V_P$:

$$\epsfbox{02144x01.eps}$$

Now describe what must be done to the potentiometer to cause the UJT to turn back off again.

\underbar{file 02144}
%(END_QUESTION)





%(BEGIN_ANSWER)

The UJT will remain in the nonconducting state as the potentiometer voltage increases from 0 volts, until it reaches $V_P$.  At that voltage, the UJT turns on and stays on.  To turn the UJT off, the potentiometer must be adjusted back down in voltage until the current through point {\bf A} decreases to a certain "dropout" value.

%(END_ANSWER)





%(BEGIN_NOTES)

Ask your students to describe how hysteresis is exhibited by the UJT in this scenario.

%INDEX% UJT, trigger characteristics
%INDEX% Unijunction transistor, trigger characteristics

%(END_NOTES)


