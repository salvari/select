
%(BEGIN_QUESTION)
% Copyright 2005, Tony R. Kuphaldt, released under the Creative Commons Attribution License (v 1.0)
% This means you may do almost anything with this work of mine, so long as you give me proper credit

Explain how it is possible for a fault in the biasing circuitry of a transistor amplifier to completely kill the (AC) output of that amplifier.  How and why can a shift in DC bias voltage have an effect on the AC signal being amplified?

\underbar{file 03741}
%(END_QUESTION)





%(BEGIN_ANSWER)

If the DC bias voltage shifts far enough away from the normal (quiescent) levels, the transistor may be forced into saturation or cutoff so it cannot reproduce the AC signal.

%(END_ANSWER)





%(BEGIN_NOTES)

This question asks students to explore the possibility of complete AC signal failure due to a simple shift in DC bias, based on their understanding of how transistor amplifiers function.  It may seem paradoxical that such a "small" fault could have such a large effect on an amplifier circuit, but it should make sense once students grasp how important bias is to class-A amplifier operation.

%INDEX% Troubleshooting, explaining how bias shift can kill output of transistor amplifier circuit

%(END_NOTES)


