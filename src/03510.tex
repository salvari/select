
%(BEGIN_QUESTION)
% Copyright 2005, Tony R. Kuphaldt, released under the Creative Commons Attribution License (v 1.0)
% This means you may do almost anything with this work of mine, so long as you give me proper credit

Controlling electrical "noise" in automotive electrical systems can be problematic, as there are many sources of "noise" voltages throughout a car.  Spark ignitions and alternators can both generate substantial noise voltages, superimposed on the DC voltage in a car's electrical system.  A simple way to electrically model this noise is to draw it as an AC "noise voltage" source in series with the DC source.  If this noise enters a radio or audio amplifier, the result will be an irritating sound produced at the speakers:

$$\epsfbox{03510x01.eps}$$

What would you suggest as a "fix" for this problem if a friend asked you to apply your electronics expertise to their noisy car audio system?  Be sure to provide at least two practical suggestions.

\underbar{file 03510}
%(END_QUESTION)





%(BEGIN_ANSWER)

This is perhaps the easiest solution, to install a very large capacitor ($C_{huge}$) in parallel with the audio load:

$$\epsfbox{03510x02.eps}$$

Other, more sophisticated solutions exist, however!

\vskip 10pt

Follow-up question: use superposition theorem to show {\it why} the capacitor mitigates the electrical noise without interfering with the transfer of DC power to the radio/amplifier.

%(END_ANSWER)





%(BEGIN_NOTES)

The follow-up question is yet another example of how practical the superposition theorem is when analyzing filter circuits.

%INDEX% Noise filter, for car audio system

%(END_NOTES)


