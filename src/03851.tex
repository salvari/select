
%(BEGIN_QUESTION)
% Copyright 2006, Tony R. Kuphaldt, released under the Creative Commons Attribution License (v 1.0)
% This means you may do almost anything with this work of mine, so long as you give me proper credit

$$\epsfbox{03851x01.eps}$$

\underbar{file 03851}
\vfil \eject
%(END_QUESTION)





%(BEGIN_ANSWER)

Use circuit simulation software to verify your predicted and measured parameter values.

%(END_ANSWER)





%(BEGIN_NOTES)

I have purposely left the details of the schematic diagram vague, so that students must do a lot of datasheet research on their own to figure out how to make an event counter circuit.  You may choose to give your students part numbers for the integrated circuits, or choose not to, depending on how capable your students are.  The point is, they must figure out how to make the ICs work based on what they read from the manufacturer.

Something else students will probably have to do is de-bounce the event switch.  Some event switches are inherently bounceless, while others are definitely not.  Switch debouncing is something your students need to learn about and integrate into this circuit.

An extension of this exercise is to incorporate troubleshooting questions.  Whether using this exercise as a performance assessment or simply as a concept-building lab, you might want to follow up your students' results by asking them to predict the consequences of certain circuit faults.

%INDEX% Assessment, performance-based (Event counter circuit)

%(END_NOTES)


