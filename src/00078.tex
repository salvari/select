
%(BEGIN_QUESTION)
% Copyright 2003, Tony R. Kuphaldt, released under the Creative Commons Attribution License (v 1.0)
% This means you may do almost anything with this work of mine, so long as you give me proper credit

Just as electricity may be harnessed to produce magnetism, magnetism may also be harnessed to produce electricity.  The latter process is known as {\it electromagnetic induction}.  Design a simple experiment to explore the phenomenon of electromagnetic induction.

\underbar{file 00078}
%(END_QUESTION)





%(BEGIN_ANSWER)

Perhaps the easiest way to demonstrate electromagnetic induction is to build a simple circuit formed from a coil of wire and a sensitive electrical meter (a digital meter is preferred for this experiment), then move a magnet past the wire coil.  You should notice a direct correlation between the position of the magnet relative to the coil over time, and the amount of voltage or current indicated by the meter.

%(END_ANSWER)





%(BEGIN_NOTES)

Many students improperly assume that electromagnetic induction may take place in the presence of {\it static} magnetic fields.  This is not true.  The simple experimental setup described in the "Answer" section for this question is sufficient to dispel that myth, and to illuminate students' understanding of this principle.  Incidentally, this activity is a great way to get students started thinking in calculus terms: relating one variable to the {\it rate of change over time} of another variable.

%INDEX% Electromagnetic induction
%INDEX% Induction, electromagnetic

%(END_NOTES)


