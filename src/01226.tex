
%(BEGIN_QUESTION)
% Copyright 2003, Tony R. Kuphaldt, released under the Creative Commons Attribution License (v 1.0)
% This means you may do almost anything with this work of mine, so long as you give me proper credit

In an eight-bit digital system, where all numbers are represented in two's complement form, what is the largest (most positive) quantity that may be represented with those eight bits?  What is the smallest (most negative) quantity that may be represented?  Express your answers in both binary (two's complement) and decimal form.

\underbar{file 01226}
%(END_QUESTION)





%(BEGIN_ANSWER)

Largest (most positive): $01111111_2$ = $127_{10}$

\vskip 10pt

Smallest (most negative): $10000000_2$ = $-128_{10}$

%(END_ANSWER)





%(BEGIN_NOTES)

The most important concept in this question is that of {\it range}: what are the limits of the representable quantities, given a certain number of bits.  Two's complement just makes the concept a bit more interesting.

%INDEX% Range, of two's complement number field

%(END_NOTES)


