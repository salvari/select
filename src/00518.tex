
%(BEGIN_QUESTION)
% Copyright 2003, Tony R. Kuphaldt, released under the Creative Commons Attribution License (v 1.0)
% This means you may do almost anything with this work of mine, so long as you give me proper credit

Some brands of dry-cell batteries come equipped with a built-in charge indicator in the form of a thermal strip along one side.  Pressing two white dots closes a circuit, which heats up the strip and indicates battery charge by changing color.

Compare the accuracy of a charge indicator of this general design against using a voltmeter to measure open-circuit battery voltage.  Which method of measurement is a more accurate indication of battery charge, and why?

\underbar{file 00518}
%(END_QUESTION)





%(BEGIN_ANSWER)

The thermal strip charge indicator is actually a more accurate indication of battery charge than an open-circuit voltage test.

\vskip 10pt

Challenge question: in the absence of such a "charge indicator" on the side of a battery, how could you perform an accurate assessment of battery charge?

%(END_ANSWER)





%(BEGIN_NOTES)

This question is practical because I've seen students measure the open-circuit voltage of a battery and declare it "good" when it is really past the end of its useful life.  This is not to say that a plain voltmeter is useless for determining battery charge.  Obviously if an open-circuit voltage test yields abnormally low voltage, we know the battery is dead.  The question is, under what condition(s) is a "good" voltage measurement a reliable indicator of adequate battery charge?

%INDEX% Battery charge, measuring

%(END_NOTES)


