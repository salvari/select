
%(BEGIN_QUESTION)
% Copyright 2005, Tony R. Kuphaldt, released under the Creative Commons Attribution License (v 1.0)
% This means you may do almost anything with this work of mine, so long as you give me proper credit

Power supplies are sometimes equipped with {\it EMI/RFI} filters on their inputs, to prevent high-frequency "noise" voltage created within the power supply circuit from getting back to the power source where it might interfere with other powered equipment.  This is especially useful for "switching" power supply circuits, where transistors are used to switch power on and off very rapidly in the voltage transformation and regulation process:

$$\epsfbox{03699x01.eps}$$

Determine what type of filter circuit this is (LP, HP, BP, or BS), and also determine the inductive and capacitive reactances of its components at 60 Hz, if the inductors are 100 $\mu$H each and the capacitors are 0.022 $\mu$F each.

\underbar{file 03699}
%(END_QUESTION)





%(BEGIN_ANSWER)

$X_L$ = 0.0377 $\Omega$ (each)

\vskip 10pt

$X_C$ = 120.6 k$\Omega$ (each)

%(END_ANSWER)





%(BEGIN_NOTES)

Ask your students how they determined the identity of this filter.  Are they strictly memorizing filter configurations, or do they have a technique for determining what type of filter circuit it is based on basic electrical principles (reactance of components to different frequencies)?  Remind them that rote memorization is a very poor form of learning!

%INDEX% Filter, RFI
%INDEX% Harmonics, generated by AC-DC power supply circuit
%INDEX% RFI filter

%(END_NOTES)


