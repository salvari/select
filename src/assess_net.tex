
\centerline{\bf Performance-based assessments for network analysis competencies} \bigskip 
 
This worksheet and all related files are licensed under the Creative Commons Attribution License, version 1.0.  To view a copy of this license, visit http://creativecommons.org/licenses/by/1.0/, or send a letter to Creative Commons, 559 Nathan Abbott Way, Stanford, California 94305, USA.  The terms and conditions of this license allow for free copying, distribution, and/or modification of all licensed works by the general public.

\bigskip 

\hrule

\vskip 10pt

The purpose of these assessments is for instructors to accurately measure the learning of their electronics students, in a way that melds theoretical knowledge with hands-on application.  In each assessment, students are asked to predict the behavior of a circuit from a schematic diagram and component values, then they build that circuit and measure its real behavior.  If the behavior matches the predictions, the student then simulates the circuit on computer and presents the three sets of values to the instructor.  If not, then the student then must correct the error(s) and once again compare measurements to predictions.  Grades are based on the number of attempts required before all predictions match their respective measurements.

You will notice that no component values are given in this worksheet.  The {\it instructor} chooses component values suitable for the students' parts collections, and ideally chooses different values for each student so that no two students are analyzing and building the exact same circuit.  These component values may be hand-written on the assessment sheet, printed on a separate page, or incorporated into the document by editing the graphic image.

\vskip 10pt

\noindent
This is the procedure I envision for managing such assessments:

\vskip 10pt

\item{1.} The instructor hands out individualized assessment sheets to each student.

\item{2.} Each student predicts their circuit's behavior at their desks using pencil, paper, and calculator (if appropriate).

\item{3.} Each student builds their circuit at their desk, under such conditions that it is impossible for them to verify their predictions using test equipment.  Usually this will mean the use of a multimeter only (for measuring component values), but in some cases even the use of a multimeter would not be appropriate.

\item{4.} When ready, each student brings their predictions and completed circuit up to the instructor's desk, where any necessary test equipment is already set up to operate and test the circuit.  There, the student sets up their circuit and takes measurements to compare with predictions.

\item{5.} If any measurement fails to match its corresponding prediction, the student goes back to their own desk with their circuit and their predictions in hand.  There, the student tries to figure out where the error is and how to correct it.

\item{6.} Students repeat these steps as many times as necessary to achieve correlation between all predictions and measurements.  The instructor's task is to count the number of attempts necessary to achieve this, which will become the basis for a percentage grade.

\item{7.} (OPTIONAL) As a final verification, each student simulates the same circuit on computer, using circuit simulation software (Spice, Multisim, etc.) and presenting the results to the instructor as a final pass/fail check.

\vskip 10pt

\noindent
These assessments more closely mimic real-world work conditions than traditional written exams:

\medskip
\item{$\bullet$} Students cannot pass such assessments only knowing circuit theory or only having hands-on construction and testing skills -- they must be proficient at both.
\item{$\bullet$} Students do not receive the ``authoritative answers'' from the instructor.  Rather, they learn to validate their answers through real circuit measurements.
\item{$\bullet$} Just as on the job, the work isn't complete until {\it all errors} are corrected.
\item{$\bullet$} Students must recognize and correct their own errors, rather than having someone else do it for them.
\item{$\bullet$} Students must be fully prepared on exam days, bringing not only their calculator and notes, but also their tools, breadboard, and circuit components.
\medskip

Instructors may elect to reveal the assessments before test day, and even use them as preparatory labwork and/or discussion questions.  Remember that there is absolutely nothing wrong with ``teaching to the test'' {\it so long as the test is valid}.  Normally, it is bad to reveal test material in detail prior to test day, lest students merely memorize responses in advance.  With performance-based assessments, however, there is no way to pass without truly understanding the subject(s).

