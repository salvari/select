
%(BEGIN_QUESTION)
% Copyright 2003, Tony R. Kuphaldt, released under the Creative Commons Attribution License (v 1.0)
% This means you may do almost anything with this work of mine, so long as you give me proper credit

While simple "brute-force" AC-DC power supply circuits (transformer, rectifier, filter, regulator) are still used in a variety of electronic equipment, another form of power supply is more prevalent in systems where small size and efficiency are design requirements.  This type of power supply is called a {\it switching power supply}.

Explain what a "switching power supply" is, and provide a schematic diagram of one for presentation and discussion.  (Hint: most electronic computers use "switching" power supplies instead of "brute force" power supplies, so schematic diagrams should not be difficult to find.)

\underbar{file 01107}
%(END_QUESTION)





%(BEGIN_ANSWER)

I'll let you do all the research for this question!

%(END_ANSWER)





%(BEGIN_NOTES)

While many "switching" power supply circuits will be too complex for beginning electronics students to fully understand, it will still be a useful exercise to analyze such a schematic and identify the major components (and functions).

Ask your students why "switching" power supplies are smaller and more efficient than "brute force" designs.  Ask your students to note the type of transformer used in switching power supplies, and contrast its construction to that of line-frequency power transformers.

%INDEX% Switching power supply, defined

%(END_NOTES)


