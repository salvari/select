
%(BEGIN_QUESTION)
% Copyright 2003, Tony R. Kuphaldt, released under the Creative Commons Attribution License (v 1.0)
% This means you may do almost anything with this work of mine, so long as you give me proper credit

Calculate the voltage across the switch contacts the exact moment they open, and 15 milliseconds after they have been opened:

$$\epsfbox{00458x01.eps}$$

\underbar{file 00458}
%(END_QUESTION)





%(BEGIN_ANSWER)

$E_{switch} =$ 40.91 V @ $t =$ 0 seconds

$E_{switch} =$ 9.531 V @ $t =$ 15 milliseconds

\vskip 10pt

Follow-up question: predict all voltage drops in this circuit in the event that the inductor fails open (broken wire inside).

%(END_ANSWER)





%(BEGIN_NOTES)

There is quite a lot to calculate in order to reach the solutions in this question.  There is more than one valid way to approach it, as well.  An important fact to note: the voltage across the switch contacts, in both examples, is greater than the battery voltage!  Just as capacitive time-constant circuits can generate currents in excess of what their power sources can supply, inductive time-constant circuits can generate voltages in excess of what their power sources can supply.

The follow-up question is simply an exercise in troubleshooting theory.

%INDEX% Time constant calculation, LR circuit

%(END_NOTES)


