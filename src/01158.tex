
%(BEGIN_QUESTION)
% Copyright 2003, Tony R. Kuphaldt, released under the Creative Commons Attribution License (v 1.0)
% This means you may do almost anything with this work of mine, so long as you give me proper credit

Find one or two real fuses and bring them with you to class for discussion.  Identify as much information as you can about your fuses prior to discussion:

\medskip
\item{$\bullet$} Current rating
\item{$\bullet$} Voltage rating
\item{$\bullet$} Interruption rating
\item{$\bullet$} Fuse curve (opening characteristics: fast-acting, slow-blow, etc.)
\item{$\bullet$} Status of fuse (blown or not)
\medskip

\underbar{file 01158}
%(END_QUESTION)





%(BEGIN_ANSWER)

If possible, find a manufacturer's datasheet for your components (or at least a datasheet for a similar component) to discuss with your classmates.

Be prepared to {\it prove} the status of your fuse in class, using your multimeter!

%(END_ANSWER)





%(BEGIN_NOTES)

The purpose of this question is to get students to kinesthetically interact with the subject matter.  It may seem silly to have students engage in a "show and tell" exercise, but I have found that activities such as this greatly help some students.  For those learners who are kinesthetic in nature, it is a great help to actually {\it touch} real components while they're learning about their function.  Of course, this question also provides an excellent opportunity for them to practice interpreting component markings, use a multimeter, access datasheets, etc.

%INDEX% Fuse ratings

%(END_NOTES)


