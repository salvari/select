
%(BEGIN_QUESTION)
% Copyright 2003, Tony R. Kuphaldt, released under the Creative Commons Attribution License (v 1.0)
% This means you may do almost anything with this work of mine, so long as you give me proper credit

A quantity sometimes used in DC circuits is {\it conductance}, symbolized by the letter $G$.  Conductance is the reciprocal of resistance ($G = {1 \over R}$), and it is measured in the unit of siemens.

Expressing the values of resistors in terms of conductance instead of resistance has certain benefits in parallel circuits.  Whereas resistances ($R$) add in series and "diminish" in parallel (with a somewhat complex equation), conductances ($G$) add in parallel and "diminish" in series.  Thus, doing the math for series circuits is easier using resistance and doing math for parallel circuits is easier using conductance:

$$\epsfbox{01845x01.eps}$$

In AC circuits, we also have reciprocal quantities to reactance ($X$) and impedance ($Z$).  The reciprocal of reactance is called {\it susceptance} ($B = {1 \over X}$), and the reciprocal of impedance is called {\it admittance} ($Y = {1 \over Z}$).  Like conductance, both these reciprocal quantities are measured in units of siemens.

Write an equation that solves for the admittance ($Y$) of this parallel circuit.  The equation need not solve for the phase angle between voltage and current, but merely provide a scalar figure for admittance (in siemens):

$$\epsfbox{01845x02.eps}$$

\underbar{file 01845}
%(END_QUESTION)





%(BEGIN_ANSWER)

$Y_{total} = \sqrt{G^2 + B^2}$

\vskip 10pt

Follow-up question \#1: draw a phasor diagram showing how $Y$, $G$, and $B$ relate.

\vskip 10pt

Follow-up question \#2: re-write this equation using quantities of resistance ($R$), reactance ($X$), and impedance ($Z$), instead of conductance ($G$), susceptance ($B$), and admittance ($Y$).

%(END_ANSWER)





%(BEGIN_NOTES)

Ask your students if this equation looks familiar to them.  It should!

\vskip 10pt

The answer to the challenge question is a matter of algebraic substitution.  Work through this process with your students, and then ask them to compare the resulting equation with other equations they've seen before.  Does its form look familiar to them in any way?

%INDEX% Admittance (Y), defined
%INDEX% Y (admittance), defined
%INDEX% Susceptance (B), defined
%INDEX% B (susceptance), defined

%(END_NOTES)


