
%(BEGIN_QUESTION)
% Copyright 2006, Tony R. Kuphaldt, released under the Creative Commons Attribution License (v 1.0)
% This means you may do almost anything with this work of mine, so long as you give me proper credit

A type of resistor network known as an {\it R-2R ladder} is often used in digital-to-analog conversion circuits:

$$\epsfbox{03999x01.eps}$$

When all switches in the R-2R ladder are in the "ground" position, the network has a very interesting property regardless of its size.  Analyze the Th\'evenin equivalent resistance (as seen from the output terminal) of the following R-2R ladder networks, then comment on the results you obtain:

$$\epsfbox{03999x06.eps}$$

$$\epsfbox{03999x05.eps}$$

$$\epsfbox{03999x04.eps}$$

$$\epsfbox{03999x03.eps}$$

$$\epsfbox{03999x02.eps}$$

\underbar{file 03999}
%(END_QUESTION)





%(BEGIN_ANSWER)

Did you honestly think I'd do all the work for you and just give you the answer?

%(END_ANSWER)





%(BEGIN_NOTES)

The answer is not difficult to obtain if you use each Th\'evenin equivalent resistance to model the left-hand portion of each successive R-2R ladder network as they become more complex!  Those students who do not take this problem-solving step are doomed to perform a {\it lot} of series-parallel calculations!

%INDEX% R-2R ladder network

%(END_NOTES)


