
%(BEGIN_QUESTION)
% Copyright 2004, Tony R. Kuphaldt, released under the Creative Commons Attribution License (v 1.0)
% This means you may do almost anything with this work of mine, so long as you give me proper credit

The Fourier series for a square wave is as follows:

$$v_{square} = {4 \over \pi}V_m \left(\sin \omega t + {1 \over 3}\sin 3 \omega t + {1 \over 5}\sin 5 \omega t + {1 \over 7} \sin 7 \omega t + \cdots + {1 \over n} \sin n \omega t \right)$$

\noindent
Where,

$V_m$ = Peak amplitude of square wave

$\omega$ = Angular velocity of square wave (equal to $2 \pi f$, where $f$ is the fundamental frequency)

$n$ = An odd integer

\vskip 10pt

Electrically, we might represent a square-wave voltage source as a circle with a square-wave symbol inside, like this:

$$\epsfbox{02260x01.eps}$$

Knowing the Fourier series of this voltage, however, allows us to represent the same voltage source as a set of series-connected voltage sources, each with its own (sinusoidal) frequency.  Draw the equivalent schematic for a 10 volt (peak), 200 Hz square-wave source in this manner showing only the first four harmonics, labeling each sinusoidal voltage source with its own RMS voltage value and frequency:

\vskip 60pt

Hint: $\omega = 2 \pi f$

\vskip 10pt

\underbar{file 02260}
%(END_QUESTION)





%(BEGIN_ANSWER)

$$\epsfbox{02260x02.eps}$$

%(END_ANSWER)





%(BEGIN_NOTES)

To be honest, the four-harmonic equivalent circuit is a rather poor approximation for a square wave.  The real purpose of this question, though, is to have students relate the sinusoidal terms of a common Fourier series (for a square wave) to a schematic diagram, translating between angular velocity and frequency, peak values and RMS values.

Please note that the voltage magnitudes shown in the answer are {\it RMS} and not peak!  If you were to calculate peak sinusoid source values, you would obtain these results:

\medskip
\goodbreak
\item{$\bullet$} 1st harmonic: ${40 \over \pi}$ volts peak = 12.73 volts peak
\vskip 5pt
\item{$\bullet$} 3rd harmonic: ${40 \over {3 \pi}}$ volts peak = 4.244 volts peak
\vskip 5pt
\item{$\bullet$} 5th harmonic: ${40 \over {5 \pi}}$ volts peak = 2.546 volts peak
\vskip 5pt
\item{$\bullet$} 7th harmonic: ${40 \over {7 \pi}}$ volts peak = 1.819 volts peak
\medskip

%INDEX% Fourier series, electrically defined

%(END_NOTES)


