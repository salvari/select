
%(BEGIN_QUESTION)
% Copyright 2006, Tony R. Kuphaldt, released under the Creative Commons Attribution License (v 1.0)
% This means you may do almost anything with this work of mine, so long as you give me proper credit

The trigonometric function of {\it cosine} may be found as the result of an infinite series.  Note that this series assumes the angle $x$ to be expressed in units of {\it radians}, not degrees:

$$\cos x = \sum_{n=0}^{\infty} (-1)^n {x^{2n} \over {(2n)!}}$$

Approximate the cosine of 1 radian ($\cos 1$) in steps, using the following table, then write the partial sum expansion up to $n = 5$:

% No blank lines allowed between lines of an \halign structure!
% I use comments (%) instead, so that TeX doesn't choke.

$$\vbox{\offinterlineskip
\halign{\strut
\vrule \quad\hfil # \ \hfil & 
\vrule \quad\hfil # \ \hfil & 
\vrule \quad\hfil # \ \hfil & 
\vrule \quad\hfil # \ \hfil & 
\vrule \quad\hfil # \ \hfil \vrule \cr
\noalign{\hrule}
%
% First row
$n$ & $(-1)^n$ & $x^{2n}$ & $(2n)!$ & $\approx \cos x$ \cr
%
\noalign{\hrule}
%
% Another row
0 &  &  &  &  \cr
%
\noalign{\hrule}
%
% Another row
1 &  &  &  &  \cr
%
\noalign{\hrule}
%
% Another row
2 &  &  &  &  \cr
%
\noalign{\hrule}
%
% Another row
3 &  &  &  &  \cr
%
\noalign{\hrule}
%
% Another row
4 &  &  &  &  \cr
%
\noalign{\hrule}
%
% Another row
5 &  &  &  &  \cr
%
\noalign{\hrule}
} % End of \halign 
}$$ % End of \vbox

\vskip 30pt

\underbar{file 04056}
%(END_QUESTION)





%(BEGIN_ANSWER)

$$\vbox{\offinterlineskip
\halign{\strut
\vrule \quad\hfil # \ \hfil & 
\vrule \quad\hfil # \ \hfil & 
\vrule \quad\hfil # \ \hfil & 
\vrule \quad\hfil # \ \hfil & 
\vrule \quad\hfil # \ \hfil \vrule \cr
\noalign{\hrule}
%
% First row
$n$ & $(-1)^n$ & $x^{2n}$ & $(2n)!$ & $\approx \cos x$ \cr
%
\noalign{\hrule}
%
% Another row
0 & 1 & 1 & 1 & 1 \cr
%
\noalign{\hrule}
%
% Another row
1 & -1 & 1 & 2 & 0.5 \cr
%
\noalign{\hrule}
%
% Another row
2 & 1 & 1 & 24 & 0.5416667 \cr
%
\noalign{\hrule}
%
% Another row
3 & -1 & 1 & 720 & 0.5402778 \cr
%
\noalign{\hrule}
%
% Another row
4 & 1 & 1 & 40320 & 0.5403026 \cr
%
\noalign{\hrule}
%
% Another row
5 & -1 & 1 & 3628800 & 0.5403023 \cr
%
\noalign{\hrule}
} % End of \halign 
}$$ % End of \vbox

\vskip 10pt

Shown here is the partial sum expansion up to $n = 5$:

$$\cos x \approx 1 - {x^{2} \over 2!} + {x^{4} \over 4!} - {x^{6} \over 6!} + {x^{8} \over 8!} - {x^{10} \over 10!}$$

%(END_ANSWER)





%(BEGIN_NOTES)

It should go without saying that students should consult their electronic calculators to see what the actual value of $\cos 1$ is, and compare that to their partial sum approximation.  Students should also feel free to explore the validity of this series by approximating the cosine of angles other than 1 radian.

This questions provides students with the opportunity to see how $\cos x$ may be arithmetically calculated.  This, in fact, is how many digital electronic computers determine trigonometric functions: from partial sum approximations.

%INDEX% Series (mathematical), for cosine function

%(END_NOTES)


