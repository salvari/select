
\centerline{\bf Design Project: Dual-output AC-DC power supply} \bigskip 
 
This worksheet and all related files are licensed under the Creative Commons Attribution License, version 1.0.  To view a copy of this license, visit http://creativecommons.org/licenses/by/1.0/, or send a letter to Creative Commons, 559 Nathan Abbott Way, Stanford, California 94305, USA.  The terms and conditions of this license allow for free copying, distribution, and/or modification of all licensed works by the general public.

\bigskip 

\hrule

\vskip 10pt

Your project is to design and build a dual-output AC-to-DC "brute force" power supply, complete with filtering to minimize ripple.  The power supply will output balanced DC voltages (+5 volts and -5 volts minimum; +15 volts and -15 volts maximum), at a maximum current of 1 amp, and will contain overcurrent protection on both the AC (line) and DC (load) sides.  As a line-powered device, it will also be equipped with an indicator light showing the presence of AC voltage, and the case (if metal) will be safety grounded.  Here is a sample schematic diagram for you to follow when designing your system:

% Sample schematic diagram here
$$\epsfbox{proj_pwr2.eps}$$

Of course, you are not restricted to using this exact design.  One design feature I insist on, though, is that you build your own rectifier circuit from individual diodes, rather than use a packaged rectifier assembly.  I highly recommend the use of "barrier strips" or "terminal strips" to make the electrical connections between components.  These connection devices provide solid, permanent electrical connections while still allowing components to be easily installed and removed.

I also insist on the AC power cord being securely and safely attached to the case.  One of the best ways of doing this is to use a "cord grip bushing" or "CGB" to firmly secure the cord as it passes through a hole in the side of the case.  Most hardware stores carry this common electrical fitting, in enough different sizes to accommodate most any power cord diameter.

All final versions of the power supply must be safety-checked before plugging them into line (AC) power.  The safety check will consist of the following points:

\medskip
\item{$1.$} Ground prong to (metal) case resistance check -- {\it use a Kelvin 4-wire test configuration to measure voltage drop between the prong and case while the ground conductor is carrying at least a 1-amp current (from a benchtop power supply configured as a current source)}.  Total ground path resistance from prong to case shall not exceed 0.1 ohm (100 mV drop at 1 amp).
\item{$2.$} Hot-to-neutral continuity measurement to check for proper operation of power switch and AC fuse.
\item{$3.$} Hot-to-ground and neutral-to-ground continuity measurements to ensure isolation from chassis ground.
\item{$4.$} Transformer secondary-to-positive output continuity measurement to check for proper operation of DC fuse.
\medskip

\vskip 10pt

\goodbreak
\noindent
Deadlines (set by instructor):

\medskip
\item{$\bullet$} Project design completed: 
\item{$\bullet$} Components purchased:
\item{$\bullet$} Working prototype:
\item{$\bullet$} Finished system:
\item{$\bullet$} Full documentation:
\medskip



