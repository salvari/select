
%(BEGIN_QUESTION)
% Copyright 2005, Tony R. Kuphaldt, released under the Creative Commons Attribution License (v 1.0)
% This means you may do almost anything with this work of mine, so long as you give me proper credit

Calculate both the total resistance of this voltage divider circuit (as "seen" from the perspective of the 25 volt source) and its output voltage (as measured from the $V_{out}$ terminal to ground): 

$$\epsfbox{03271x01.eps}$$

Note that the upper 5 k$\Omega$ potentiometer is set to its 20\% position ($m = 0.2$), while the lower 5 k$\Omega$ potentiometer is set to its 90\% position ($m = 0.9$), and the 100 k$\Omega$ potentiometer is set to its 40\% position ($m = 0.4$).

\underbar{file 03271}
%(END_QUESTION)





%(BEGIN_ANSWER)

$R_{total}$ = 9.978 k$\Omega$

\vskip 10pt

$V_{out}$ = 12.756 V

%(END_ANSWER)





%(BEGIN_NOTES)

Ask your students to explain why the output voltage is expressed as a positive quantity.  Is this significant, or could it be properly expressed as a negative quantity as well?  In other words, is this an {\it absolute value} of a voltage which may be negative, or is it definitely a positive voltage?  How may we tell?

Also, it might be good to ask your students to show the equivalent circuit (made up entirely of fixed resistors) that they drew in route to solving for total resistance and output voltage.  Encourage them to take this step if they have not already, for although it does involve "extra" work, it helps greatly in keeping track of the series-parallel relationships and all calculated circuit values.

%INDEX% Potentiometer, as variable voltage divider
%INDEX% Voltage divider

%(END_NOTES)


