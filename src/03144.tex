
%(BEGIN_QUESTION)
% Copyright 2005, Tony R. Kuphaldt, released under the Creative Commons Attribution License (v 1.0)
% This means you may do almost anything with this work of mine, so long as you give me proper credit

Electromechanical relays used to start and stop high-power electric motors (called "contactors" or "starters") must be considered a possible source of {\it arc flash}.  Explain why this is.  What is it about the construction or operation of such a relay that invites this dangerous phenomenon?

\underbar{file 03144}
%(END_QUESTION)





%(BEGIN_ANSWER)

Electromechanical relays interrupt circuit current by drawing pairs of metal contacts apart, separating them with an air gap.  Because this contact motion is not instantaneous, it is possible to generate an arc across the air gaps of such magnitude that it becomes an arc flash.

%(END_ANSWER)





%(BEGIN_NOTES)

Arc flash is just as hazardous to electrical technicians as electric shock, yet I have seen (and worked with) people who pay no attention to the dangers!  It must be understood that motor starters are by their very nature arc-generating devices, and that under certain unusual conditions may generate lethal arc flashes.  You might want to ask your students what sorts of unusual conditions could lead to a contactor producing an actual arc flash (rather than merely a few small sparks).

%INDEX% Arc flash, caused by motor contactor (relay)

%(END_NOTES)


