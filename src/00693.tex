
%(BEGIN_QUESTION)
% Copyright 2003, Tony R. Kuphaldt, released under the Creative Commons Attribution License (v 1.0)
% This means you may do almost anything with this work of mine, so long as you give me proper credit

Suppose we have a single resistor powered by two parallel-connected current sources.  Each of the current sources is "ideal," possessing infinite internal resistance:

$$\epsfbox{00693x01.eps}$$

Calculate the resistor's voltage drop and current in this circuit.

\vskip 10pt

Now, suppose we were to remove one current source from the circuit, replacing it with its internal resistance ($\infty$ $\Omega$).  Re-calculate the resistor's voltage drop and current in the resulting circuit:

$$\epsfbox{00693x02.eps}$$

\vskip 10pt

Now, suppose we were to remove the other current source from the circuit, replacing it with its internal resistance ($\infty$ $\Omega$).  Re-calculate the resistor's voltage drop and current in the resulting circuit:

$$\epsfbox{00693x03.eps}$$

\vskip 10pt

One last exercise: "superimpose" (add) the resistor voltages and superimpose (add) the resistor currents in the last two circuit examples, and compare these voltage and current figures with the calculated values of the original circuit.  What do you notice?

\underbar{file 00693}
%(END_QUESTION)





%(BEGIN_ANSWER)

Original circuit: $E_R =$ 15 volts ; $I_R =$ 3 A

\vskip 10pt

With 7 amp current source only: $E_R =$ 35 volts ; $I_R =$ 7 A

\vskip 10pt

With 4 amp current source only: $E_R =$ 20 volts ; $I_R =$ 4 A

\vskip 10pt

35 volts - 20 volts = 15 volts

7 A - 4 A = 3 A

%(END_ANSWER)





%(BEGIN_NOTES)

This circuit is so simple, students should not even require the use of a calculator to determine the current figures.  If students are not familiar with current sources, this question provides an excellent opportunity to review them.  The main point of the question is, however, to get students to see the concept of {\it superposition} of voltages and currents.

%INDEX% Superposition theorem, conceptual

%(END_NOTES)


