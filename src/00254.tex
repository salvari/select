
%(BEGIN_QUESTION)
% Copyright 2003, Tony R. Kuphaldt, released under the Creative Commons Attribution License (v 1.0)
% This means you may do almost anything with this work of mine, so long as you give me proper credit

If a copper ring is brought closer to the end of a permanent magnet, a repulsive force will develop between the magnet and the ring.  This force will cease, however, when the ring stops moving.  What is this effect called?

$$\epsfbox{00254x01.eps}$$

Also, describe what will happen if the copper ring is moved {\it away} from the end of the permanent magnet.

\underbar{file 00254}
%(END_QUESTION)





%(BEGIN_ANSWER)

The phenomenon is known as {\it Lenz' Law}.  If the copper ring is moved {\it away} from the end of the permanent magnet, the direction of force will reverse and become attractive rather than repulsive.


\vskip 10pt

Follow-up question: trace the direction of rotation for the induced electric current in the ring necessary to produce both the repulsive and the attractive force.

\vskip 10pt

Challenge question: what would happen if the magnet's orientation were reversed (south pole on left and north pole on right)?

%(END_ANSWER)





%(BEGIN_NOTES)

This phenomenon is difficult to demonstrate without a very powerful magnet.  However, if you have such apparatus available in your lab area, it would make a great piece for demonstration!

\vskip 10pt

One practical way I've demonstrated Lenz's Law is to obtain a rare-earth magnet ({\it very} powerful!), set it pole-up on a table, then drop an aluminum coin (such as a Japanese Yen) so it lands on top of the magnet.  If the magnet is strong enough and the coin is light enough, the coin will gently come to rest on the magnet rather than hit hard and bounce off.

A more dramatic illustration of Lenz's Law is to take the same coin and spin it (on edge) on a table surface.  Then, bring the magnet close to the edge of the spinning coin, and watch the coin promptly come to a halt, without contact between the coin and magnet.

Another illustration is to set the aluminum coin on a smooth table surface, then quickly move the magnet over the coin, parallel to the table surface.  If the magnet is close enough, the coin will be "dragged" a short distance as the magnet passes over.

In all these demonstrations, it is significant to show to your students that the coin itself is not magnetic.  It will not stick to the magnet as an iron or steel coin would, thus any force generated between the coin and magnet is strictly due to {\it induced currents} and not ferromagnetism.

%INDEX% Lenz's Law

%(END_NOTES)


