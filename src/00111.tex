
%(BEGIN_QUESTION)
% Copyright 2003, Tony R. Kuphaldt, released under the Creative Commons Attribution License (v 1.0)
% This means you may do almost anything with this work of mine, so long as you give me proper credit

Different types of atoms are distinguished by different numbers of elementary particles within them.  Determine the numbers of elementary particles within each of these types of atoms:

\vskip 10pt

\item {$\bullet$} Carbon
\item {$\bullet$} Hydrogen
\item {$\bullet$} Helium
\item {$\bullet$} Aluminum

\vskip 10pt

Hint: look up each of these elements on a {\it periodic table}.

\underbar{file 00111}
%(END_QUESTION)





%(BEGIN_ANSWER)

Each atom of carbon is guaranteed to contain 6 protons.  Unless the atom is electrically charged, it will contain 6 electrons as well to balance the charge of the protons.  Most carbon atoms contain 6 neutrons, but some may contain more or less than 6. 

\vskip 5pt

Each atom of hydrogen is guaranteed to contain 1 proton.  Unless the atom is electrically charged, it will contain 1 electron as well to balance the charge of the one proton.  Most hydrogen atoms contain no neutrons, but some contain either one or two neutrons.
 
\vskip 5pt

Each atom of helium is guaranteed to contain 2 protons.  Unless the atom is electrically charged, it will contain 2 electrons as well to balance the charge of the protons.  Most helium atoms contain 2 neutrons, but some may contain more or less than 2.

\vskip 5pt

Each atom of aluminum is guaranteed to contain 13 protons.  Unless the atom is electrically charged, it will contain 13 electrons as well to balance the charge of the protons.  Most aluminum atoms contain 14 neutrons, but some may contain more or less than 14.

\vskip 10pt

While you're researching the numbers of particles inside each of these atom types, you may come across these terms: {\it atomic number} and {\it atomic mass} (sometimes called {\it atomic weight}).  Be prepared to discuss what these two terms mean.

%(END_ANSWER)





%(BEGIN_NOTES)

Be sure to ask your students what definitions they found for "atomic number" and "atomic mass".

It is highly recommended that students seek out periodic tables to help them with their research on this question.  The ordering of elements on a periodic table may provoke a few additional questions such as, "Why are the different elements arranged like this?"  This may build to a very interesting discussion on basic chemistry, so be prepared to engage in such an interaction on these subjects if necessary.

%INDEX% Periodic table
%INDEX% Atomic number
%INDEX% Atomic mass
%INDEX% Atomic weight

%(END_NOTES)


