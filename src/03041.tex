
%(BEGIN_QUESTION)
% Copyright 2005, Tony R. Kuphaldt, released under the Creative Commons Attribution License (v 1.0)
% This means you may do almost anything with this work of mine, so long as you give me proper credit

Define the following acronyms as they apply to digital logic circuits:

\medskip
\item{$\bullet$} ASIC
\item{$\bullet$} PAL
\item{$\bullet$} PLA
\item{$\bullet$} PLD
\item{$\bullet$} CPLD
\item{$\bullet$} FPGA
\medskip

\underbar{file 03041}
%(END_QUESTION)





%(BEGIN_ANSWER)

\medskip
\item{$\bullet$} ASIC: {\it Application-Specific Integrated Circuit}
\item{$\bullet$} PAL: {\it Programmable Array Logic}
\item{$\bullet$} PLA: {\it Programmable Logic Array}
\item{$\bullet$} PLD: {\it Programmable Logic Device}
\item{$\bullet$} CPLD: {\it Complex Programmable Logic Device}
\item{$\bullet$} FPGA: {\it Field-Programmable Gate Array}
\medskip

\vskip 10pt

Follow-up question: now, comment on what each of these acronyms actually means, going beyond a mere recitation of the definition.

%(END_ANSWER)





%(BEGIN_NOTES)

There is a veritable "alphabet soup" of acronyms in the world of programmable digital logic, and these are just a few.  Going into the precise meaning of each acronym may not be the best use of time in answering this question, as there is little context in which to understand the meanings.  Please do not attempt to do what so many technical courses do, and that is stuff students' heads with acronym definitions to the neglect of actually {\it understanding} the various technologies.  This question is intended only as an opening to an in-depth discussion of programmable logic, and not an end in itself!

%INDEX% ASIC, defined
%INDEX% FPGA, defined
%INDEX% PAL, defined
%INDEX% PLA, defined
%INDEX% PLD, defined
%INDEX% CPLD, defined

%(END_NOTES)


