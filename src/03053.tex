
%(BEGIN_QUESTION)
% Copyright 2005, Tony R. Kuphaldt, released under the Creative Commons Attribution License (v 1.0)
% This means you may do almost anything with this work of mine, so long as you give me proper credit

When evaluating an expression such as this, it is very important to follow proper order of operations.  Otherwise, the correct result will be impossible to arrive at:

$$3 \log 2^5 + 14$$

To show what the proper order of operations is for this expression, I show it being evaluated {\it step by step} here\footnote{$^{\dag}$}{By the way, this is a highly recommended practice for those struggling with mathematical principles: {\it document each and every step} by re-writing the expression.  Although it takes more paper and more effort, it will save you from needless error and frustration!}:

$$3 \log 2^5 + 14$$

$$3 \log 32 + 14$$

$$3 \times 1.5051 + 14$$

$$4.5154 + 14$$

$$18.5154$$

Do the same for each of the following expressions:

\medskip
\goodbreak
\item{$\bullet$} $10 - 25 \times 2 + 5$
\vskip 5pt
\item{$\bullet$} $-8 + 10^3 \times 51$
\vskip 5pt
\item{$\bullet$} $12^4 \times (3 + 11)$
\vskip 5pt
\item{$\bullet$} $21^{(7 - 4)} \times 40$
\vskip 5pt
\item{$\bullet$} $\log \sqrt{6 + 35^2}$
\vskip 5pt
\item{$\bullet$} $\sqrt{\left({220 \over 16} - 2.75\right) \times 2}$
\medskip

\underbar{file 03053}
%(END_QUESTION)





%(BEGIN_ANSWER)

I'll let you determine and document the proper order of operations, but here are the results of each expression:

\medskip
\goodbreak
\item{$\bullet$} $10 - 25 \times 2 + 5 = -35$
\vskip 5pt
\item{$\bullet$} $-8 + 10^3 \times 51 = 50992$
\vskip 5pt
\item{$\bullet$} $12^4 \times (3 + 11) = 290304$
\vskip 5pt
\item{$\bullet$} $21^{(7 - 4)} \times 40 = 370440$
\vskip 5pt
\item{$\bullet$} $\log \sqrt{6 + 35^2} = 1.5451$
\vskip 5pt
\item{$\bullet$} $\sqrt{\left({220 \over 16} - 2.75\right) \times 2} = 4.6904$
\medskip

%(END_ANSWER)





%(BEGIN_NOTES)

Order of operations is extremely important, as it becomes critical to recognize proper order of evaluation when "stripping" an expression down to isolate a particular variable.  In essence, the normal order of operations is reversed when "undoing" an expression, so students must recognize what the proper order of operations is.

%INDEX% Arithmetic, order of operations
%INDEX% Order of operations, arithmetic

%(END_NOTES)


