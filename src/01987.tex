
%(BEGIN_QUESTION)
% Copyright 2003, Tony R. Kuphaldt, released under the Creative Commons Attribution License (v 1.0)
% This means you may do almost anything with this work of mine, so long as you give me proper credit

$$\epsfbox{01987x01.eps}$$

\underbar{file 01987}
\vfil \eject
%(END_QUESTION)





%(BEGIN_ANSWER)

Use circuit simulation software to verify your predicted and measured parameter values.

%(END_ANSWER)





%(BEGIN_NOTES)

I have had good success using 12 volts DC for the supply voltage, an MCR8SN silicon-controlled rectifier, and a small brushless DC fan motor (80 mA running current) as the load.  The MCR8SN is a "sensitive gate" SCR, which makes it easy to demonstrate static triggering (just {\it touch} the gate terminal with your finger to start the motor!).  Some SCR's may be difficult to keep latched with low-current loads, so be sure to prototype your SCR/load combination before assigning part numbers to your students!

An extension of this exercise is to incorporate troubleshooting questions.  Whether using this exercise as a performance assessment or simply as a concept-building lab, you might want to follow up your students' results by asking them to predict the consequences of certain circuit faults.

%INDEX% Assessment, performance-based (SCR latch circuit)

%(END_NOTES)


