
%(BEGIN_QUESTION)
% Copyright 2003, Tony R. Kuphaldt, released under the Creative Commons Attribution License (v 1.0)
% This means you may do almost anything with this work of mine, so long as you give me proper credit

One practical application of S-R latch circuits is {\it switch debouncing}.  Explain what "bounce" refers to in mechanical switches, and also explain how this circuit eliminates it:

$$\epsfbox{01353x01.eps}$$

Also, show where an oscilloscope could be connected to display any switch "bounce," and explain how the oscilloscope would have to be configured to capture this transient event.

\underbar{file 01353}
%(END_QUESTION)





%(BEGIN_ANSWER)

The "latching" ability of the S-R latch circuit holds the output state steady during the mechanical switch's bouncing action, allowing a "clean" output transition to take place.  

Connecting the input probe of an oscilloscope to either the S or R input of the latch will show bounce, if it occurs.  To capture this event, the 'scope would have to be configured for single-sweep mode, and have the triggering controls properly set.  A digital storage oscilloscope is essential for this type of work!

\vskip 10pt

Follow-up question: how do you suggest choosing appropriate pull-down resistor sizes for this circuit, or any CMOS circuit for that matter?

%(END_ANSWER)





%(BEGIN_NOTES)

Many textbooks use switch debouncing as a practical example of S-R latch function, so I won't bother giving hints as to how this circuit works.  Let the students do their own research, and let them explain it to you during discussion.

If students need practical examples of how switch "bouncing" can be bad, suggest digital counter circuits, where a mechanical switch causes a counter to increment (or decrement) once per actuation.  If the switch bounces, the counter will increment (or decrement) more than once per switch actuation, which is undesirable.

Oscilloscope triggering is one of those features that separates novice 'scope users from competent 'scope users.  Anyone can learn to display a repetitive waveform on an oscilloscope with a minimum of adjustment.  Many modern digital oscilloscopes even have "auto-configure" features to lock in such waveforms for display.  However, to set up triggering on one-time events requires that the user understand not only the oscilloscope's functions, but also the nature of the event to be captured.

Note to your students how the $\overline{Q}$ output of the latch doesn't go anywhere.  Often, we have applications where the second output of a latch is unused.  Ask your students whether or not this constitutes a problem.  (If you get blank stares from asking this question, remind students that unused CMOS inputs have to be grounded or tied to $V_{DD}$, or else damage may occur.  Ask them whether or not the same rule applies to gate outputs.)  This will be a good review of internal gate circuit construction.

%INDEX% Bounce, switch contact
%INDEX% Switch bounce

%(END_NOTES)


