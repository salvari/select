
%(BEGIN_QUESTION)
% Copyright 2003, Tony R. Kuphaldt, released under the Creative Commons Attribution License (v 1.0)
% This means you may do almost anything with this work of mine, so long as you give me proper credit

If a junction field-effect transistor is subjected to several different gate-to-source voltages ($V_{gs}$), and the drain-to-source voltage ($V_{ds}$) "swept" through the full range for each of these gate voltage values, data for an entire "family" of characteristic curves may be obtained and graphed for the transistor:

$$\epsfbox{00993x01.eps}$$

Identify the {\it saturation}, {\it active}, and {\it breakdown} regions on this graph.

\vskip 10pt

What do these characteristic curves indicate about the gate voltage's control over drain current?  How are the two signals related to each other?

\underbar{file 00993}
%(END_QUESTION)





%(BEGIN_ANSWER)

The {\it saturation}, {\it active}, and {\it breakdown} regions on this graph are equivalent to the same regions in bipolar junction transistor characteristic curves.

\vskip 10pt

The drain current regulation point is established by the gate voltage while in the "active" region.

%(END_ANSWER)





%(BEGIN_NOTES)

Ask your students what the characteristic curves would look like for a {\it perfect} transistor: one that was a perfect regulator of drain current over the full range of drain-source voltage.

%INDEX% Characteristic curve, JFET

%(END_NOTES)


