
%(BEGIN_QUESTION)
% Copyright 2003, Tony R. Kuphaldt, released under the Creative Commons Attribution License (v 1.0)
% This means you may do almost anything with this work of mine, so long as you give me proper credit

Under certain conditions, harmonics may be produced in AC power systems by inductors and transformers.  How is this possible, as these devices are normally considered to be linear?

\underbar{file 00655}
%(END_QUESTION)





%(BEGIN_ANSWER)

I'll answer this question with another question: is the "B-H" plot for a ferromagnetic material typically linear or nonlinear?  This is the key to understanding how an electromagnetic device can produce harmonics from a "pure" sinusoidal power source.

%(END_ANSWER)





%(BEGIN_NOTES)

Ask your students what it means for an electrical or electronic device to be "linear."  How many devices qualify as linear?  And of those devices, are they {\it always} linear, or are they capable of nonlinear behavior under special conditions?

Use the discussion time to review B-H curves for ferromagnetic materials with your students, asking them to draw the curves and point out where along those curves inductors and transformers normally operate.  What conditions, specifically, would make an iron-core device act nonlinearly?

On a similar note, the (slightly) nonlinear nature of ferromagnetic core transformers is known to permit signals to {\it modulate} each other in certain audio amplifier designs, to produce a specific kind of audio signal distortion known as {\it intermodulation distortion}.  Normally, modulation is a function possible only in nonlinear systems, so the fact that modulation occurs in a transformer is proof positive of (at least some degree of) nonlinearity.

%INDEX% Harmonics, caused by saturated inductive components

%(END_NOTES)


