
%(BEGIN_QUESTION)
% Copyright 2005, Tony R. Kuphaldt, released under the Creative Commons Attribution License (v 1.0)
% This means you may do almost anything with this work of mine, so long as you give me proper credit

Explain how the operational amplifier maintains a constant current through the load:

$$\epsfbox{02512x01.eps}$$

Write an equation solving for the regulated load current, given any relevant variables shown in the schematic diagram ($R_1$, $V_Z$, $V_{supply}$, $A_{V(OL)}$, etc.).

\underbar{file 02512}
%(END_QUESTION)





%(BEGIN_ANSWER)

$$I_{load} = {V_Z \over R_2}$$

\vskip 10pt

Follow-up question: is the transistor {\it sourcing} current to the load, or {\it sinking} current from it?

\vskip 10pt

Challenge question \#1: modify the given equation to more precisely predict load current, taking the $\beta$ of the transistor into account.

\vskip 10pt

Challenge question \#2: modify the location of the load in this circuit so that the given equation does precisely predict load current, rather than closely approximate load current.

%(END_ANSWER)





%(BEGIN_NOTES)

This is a good example of how operational amplifiers may greatly improve the functions of discrete-component circuits.  In this case, the opamp performs the function of a {\it current mirror} circuit, and does so with greater precision and reliability than a simple current mirror ever could.

It should be noted that the equation provided in the answer does not directly predict the current through the load, rather it predicts current through resistor $R_2$.  This is equal to load current only if the transistor's base current is zero, which of course it cannot be.  The {\it real} equation for predicting load current will be a bit more complex than what is given in the answer, and I leave it for your students to derive.

%INDEX% Constant-current circuit, opamp

%(END_NOTES)


