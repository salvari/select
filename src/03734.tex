
%(BEGIN_QUESTION)
% Copyright 2005, Tony R. Kuphaldt, released under the Creative Commons Attribution License (v 1.0)
% This means you may do almost anything with this work of mine, so long as you give me proper credit

This crowbar circuit has a problem.  It used to work just fine, and then one day it blew the fuse.  Upon replacing the fuse, the new fuse immediately blew:

$$\epsfbox{03734x01.eps}$$

Measuring the supply voltage with a voltmeter, everything checks out well.  There does not appear to be an overvoltage condition causing a legitimate "crowbar" event in the circuit.  Disconnecting the load from the crowbar circuit and powering it up with a standard bench-top laboratory power supply reveals the load to be in perfect condition.  Thus, both the source and the load have been eliminated as possibilities that may have blown the fuse(s).

Moving on to the crowbar circuit itself, identify some component faults that could (each, independently) account for the problem, and explain your reasoning.

\underbar{file 03734}
%(END_QUESTION)





%(BEGIN_ANSWER)

\noindent
{\bf Possible faults} (not an exhaustive list)

\medskip
\item{$\bullet$} SCR failed shorted
\item{$\bullet$} Zener diode failed shorted 
\item{$\bullet$} $R_1$ failed shorted
\item{$\bullet$} $R_2$ failed open
\item{$\bullet$} $R_4$ failed open (especially if SCR is a sensitive-gate type)
\item{$\bullet$} UJT $Q_1$ failed shorted between base terminals
\medskip

%(END_ANSWER)





%(BEGIN_NOTES)

Discuss with your students the initial troubleshooting steps described in the question.  What strategy or strategies is the technician taking to isolate the problem?

%INDEX% Troubleshooting, crowbar circuit

%(END_NOTES)


