
%(BEGIN_QUESTION)
% Copyright 2003, Tony R. Kuphaldt, released under the Creative Commons Attribution License (v 1.0)
% This means you may do almost anything with this work of mine, so long as you give me proper credit

Solid-state switching circuits usually keep their constituent transistors in one of two modes: {\it cutoff} or {\it saturation}.  Explain what each of these terms means.

\underbar{file 02043}
%(END_QUESTION)





%(BEGIN_ANSWER)

"Cutoff" refers to that condition where a transistor is not conducting any collector current (it is fully off).  "Saturation" means that condition where a transistor is conducting maximum collector current (fully on).

\vskip 10pt

Follow-up question: is there such a thing as a state where a transistor operates somewhere between cutoff (fully off) and saturation (fully on)?  Would this state be useful in a switching circuit?

%(END_ANSWER)





%(BEGIN_NOTES)

In all fairness, not all transistor switching circuits operate between these two extreme states.  Some types of switching circuits fall shy of true saturation in the "on" state, which allows transistors to switch back to the cutoff mode faster than if they had to switch back from a state of full saturation.  ECL (Emitter-Coupled Logic) digital circuits are an example of non-saturating switch circuit technology.

%INDEX% Cutoff, defined for a transistor
%INDEX% Saturation, defined for a transistor

%(END_NOTES)


