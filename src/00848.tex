
%(BEGIN_QUESTION)
% Copyright 2003, Tony R. Kuphaldt, released under the Creative Commons Attribution License (v 1.0)
% This means you may do almost anything with this work of mine, so long as you give me proper credit

An {\it operational amplifier} is a particular type of {\it differential amplifier}.  Most op-amps receive two input voltage signals and output one voltage signal:

$$\epsfbox{00848x01.eps}$$

Here is a single op-amp, shown under two different conditions (different input voltages).  Determine the voltage gain of this op-amp, given the conditions shown:

$$\epsfbox{00848x02.eps}$$

$$\epsfbox{00848x03.eps}$$

\vskip 10pt

Also, write a mathematical formula solving for differential voltage gain ($A_V$) in terms of an op-amp's input and output voltages.

\underbar{file 00848}
%(END_QUESTION)





%(BEGIN_ANSWER)

$A_V =$ 530,000

$$A_V = {\Delta V_{out} \over \Delta (V_{in2} - V_{in1})}$$

\vskip 10pt

Follow-up question: convert this voltage gain figure (as a ratio) into a voltage gain figure in decibels.

%(END_ANSWER)





%(BEGIN_NOTES)

The calculations for voltage gain here are not that different from the voltage gain calculations for any other amplifier, except that here we're dealing with a {\it differential} amplifier instead of a single-ended amplifier.

A differential voltage gain of 530,000 is not unreasonable for a modern operational amplifier!  A gain so extreme may come as a surprise to many students, but they will discover later the utility of such a high gain.

%INDEX% Voltage gain, of operational amplifier (open-loop)

%(END_NOTES)


