
%(BEGIN_QUESTION)
% Copyright 2005, Tony R. Kuphaldt, released under the Creative Commons Attribution License (v 1.0)
% This means you may do almost anything with this work of mine, so long as you give me proper credit

Complete the truth tables for these two Boolean expressions:

$$\hbox{Output } = \overline{A} + \overline{B} + C$$

% No blank lines allowed between lines of an \halign structure!
% I use comments (%) instead, so that TeX doesn't choke.

$$\vbox{\offinterlineskip
\halign{\strut
\vrule \quad\hfil # \ \hfil & 
\vrule \quad\hfil # \ \hfil & 
\vrule \quad\hfil # \ \hfil & 
\vrule \quad\hfil # \ \hfil \vrule \cr
\noalign{\hrule}
%
% First row
A & B & C & Output \cr
%
\noalign{\hrule}
%
% Second row
0 & 0 & 0 & \cr
%
\noalign{\hrule}
%
% Third row
0 & 0 & 1 &  \cr
%
\noalign{\hrule}
%
% Fourth row
0 & 1 & 0 &  \cr
%
\noalign{\hrule}
%
% Fifth row
0 & 1 & 1 &  \cr
%
\noalign{\hrule}
%
% Sixth row
1 & 0 & 0 &  \cr
%
\noalign{\hrule}
%
% Seventh row
1 & 0 & 1 &  \cr
%
\noalign{\hrule}
%
% Eighth row
1 & 1 & 0 &  \cr
%
\noalign{\hrule}
%
% Ninth row
1 & 1 & 1 &  \cr
%
\noalign{\hrule}
} % End of \halign 
}$$ % End of \vbox


\vskip 20pt


$$\hbox{Output } = A(B + AC + \overline{A})$$

$$\vbox{\offinterlineskip
\halign{\strut
\vrule \quad\hfil # \ \hfil & 
\vrule \quad\hfil # \ \hfil & 
\vrule \quad\hfil # \ \hfil & 
\vrule \quad\hfil # \ \hfil \vrule \cr
\noalign{\hrule}
%
% First row
A & B & C & Output \cr
%
\noalign{\hrule}
%
% Second row
0 & 0 & 0 & \cr
%
\noalign{\hrule}
%
% Third row
0 & 0 & 1 &  \cr
%
\noalign{\hrule}
%
% Fourth row
0 & 1 & 0 &  \cr
%
\noalign{\hrule}
%
% Fifth row
0 & 1 & 1 &  \cr
%
\noalign{\hrule}
%
% Sixth row
1 & 0 & 0 &  \cr
%
\noalign{\hrule}
%
% Seventh row
1 & 0 & 1 &  \cr
%
\noalign{\hrule}
%
% Eighth row
1 & 1 & 0 &  \cr
%
\noalign{\hrule}
%
% Ninth row
1 & 1 & 1 &  \cr
%
\noalign{\hrule}
} % End of \halign 
}$$ % End of \vbox

\underbar{file 02821}
%(END_QUESTION)





%(BEGIN_ANSWER)

$$\hbox{Output } = \overline{A} + \overline{B} + C$$

% No blank lines allowed between lines of an \halign structure!
% I use comments (%) instead, so that TeX doesn't choke.

$$\vbox{\offinterlineskip
\halign{\strut
\vrule \quad\hfil # \ \hfil & 
\vrule \quad\hfil # \ \hfil & 
\vrule \quad\hfil # \ \hfil & 
\vrule \quad\hfil # \ \hfil \vrule \cr
\noalign{\hrule}
%
% First row
A & B & C & Output \cr
%
\noalign{\hrule}
%
% Second row
0 & 0 & 0 & 1 \cr
%
\noalign{\hrule}
%
% Third row
0 & 0 & 1 & 1 \cr
%
\noalign{\hrule}
%
% Fourth row
0 & 1 & 0 & 1 \cr
%
\noalign{\hrule}
%
% Fifth row
0 & 1 & 1 & 1 \cr
%
\noalign{\hrule}
%
% Sixth row
1 & 0 & 0 & 1 \cr
%
\noalign{\hrule}
%
% Seventh row
1 & 0 & 1 & 1 \cr
%
\noalign{\hrule}
%
% Eighth row
1 & 1 & 0 & 0 \cr
%
\noalign{\hrule}
%
% Ninth row
1 & 1 & 1 & 1 \cr
%
\noalign{\hrule}
} % End of \halign 
}$$ % End of \vbox

\vskip 20pt


$$\hbox{Output } = A(B + AC + \overline{A})$$

$$\vbox{\offinterlineskip
\halign{\strut
\vrule \quad\hfil # \ \hfil & 
\vrule \quad\hfil # \ \hfil & 
\vrule \quad\hfil # \ \hfil & 
\vrule \quad\hfil # \ \hfil \vrule \cr
\noalign{\hrule}
%
% First row
A & B & C & Output \cr
%
\noalign{\hrule}
%
% Second row
0 & 0 & 0 & 0 \cr
%
\noalign{\hrule}
%
% Third row
0 & 0 & 1 & 0 \cr
%
\noalign{\hrule}
%
% Fourth row
0 & 1 & 0 & 0 \cr
%
\noalign{\hrule}
%
% Fifth row
0 & 1 & 1 & 0 \cr
%
\noalign{\hrule}
%
% Sixth row
1 & 0 & 0 & 0 \cr
%
\noalign{\hrule}
%
% Seventh row
1 & 0 & 1 & 1 \cr
%
\noalign{\hrule}
%
% Eighth row
1 & 1 & 0 & 1 \cr
%
\noalign{\hrule}
%
% Ninth row
1 & 1 & 1 & 1 \cr
%
\noalign{\hrule}
} % End of \halign 
}$$ % End of \vbox


%(END_ANSWER)





%(BEGIN_NOTES)

Ask your students to explain exactly how they figured out the "Output" states to fill in the blanks in the truth tables, for the different input combinations.  Ask them also to compare and contrast this process with that of figuring out the truth table for a given logic gate circuit.

It is especially educational if you ask your students to suggest techniques for {\it quickly} determining truth table states, based on certain features of the Boolean expression.  For instance, there is a way we can tell the first four "Output" states in the truth table (reading top to bottom) will be 0 without having to plug values into the expression for B and C.  Discuss with your students how we can look at the expression, seeing A as a multiplier for the sum within the parentheses, and immediately conclude that half of the truth table outputs will be 0.

%INDEX% Boolean algebra, conversion of expression into truth table

%(END_NOTES)


