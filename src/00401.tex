
%(BEGIN_QUESTION)
% Copyright 2003, Tony R. Kuphaldt, released under the Creative Commons Attribution License (v 1.0)
% This means you may do almost anything with this work of mine, so long as you give me proper credit

Suppose a DC power source with a voltage of 50 volts is connected to a 10 $\Omega$ load.  How much power will this load dissipate?

Now suppose the same 10 $\Omega$ load is connected to a sinusoidal AC power source with a {\it peak} voltage of 50 volts.  Will the load dissipate the same amount of power, more power, or less power?  Explain your answer.

\underbar{file 00401}
%(END_QUESTION)





%(BEGIN_ANSWER)

50 volts DC applied to a 10 $\Omega$ load will dissipate 250 watts of power.  50 volts (peak, sinusoidal) AC will deliver less than 250 watts to the same load.

%(END_ANSWER)





%(BEGIN_NOTES)

There are many analogies to explain this discrepancy between the two "50 volt" sources.  One is to compare the physical effort of a person pushing with a constant force of 50 pounds, versus someone who pushes intermittently with only a {\it peak} force of 50 pounds.

%INDEX% Peak (AC) versus DC-equivalent

%(END_NOTES)


