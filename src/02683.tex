
%(BEGIN_QUESTION)
% Copyright 2005, Tony R. Kuphaldt, released under the Creative Commons Attribution License (v 1.0)
% This means you may do almost anything with this work of mine, so long as you give me proper credit

Examine this progression of mathematical statements:

$$(100)(1000) = 100000$$

$$(100)(1000) = 10^5$$

$$\log [(100)(1000)] = \log 10^5$$

$$\log 100 + \log 1000 = \log 10^5$$

$$\log 10^2 + \log 10^3 = \log 10^5$$

$$2 + 3 = 5$$

What began as a multiplication problem ended up as an addition problem, through the application of logarithms.  What does this tell you about the utility of logarithms as an arithmetic tool?

\underbar{file 02683}
%(END_QUESTION)





%(BEGIN_ANSWER)

That logarithms can reduce the complexity of an equation from multiplication, down to addition, indicates its usefulness as a tool to {\it simplify} arithmetic problems.  Specifically, the logarithm of a product is equal to the sum of the logarithms of the two numbers being multiplied.

%(END_ANSWER)





%(BEGIN_NOTES)

In mathematics, any procedure that reduces a complex type of problem into a simpler type of problem is called a {\it transform function}, and logarithms are one of the simplest types of transform functions in existence.

%INDEX% Logarithms, used to transform a multiplication problem into an addition problem

%(END_NOTES)


