
%(BEGIN_QUESTION)
% Copyright 2005, Tony R. Kuphaldt, released under the Creative Commons Attribution License (v 1.0)
% This means you may do almost anything with this work of mine, so long as you give me proper credit

Identify the type of oscillator circuit shown in this schematic diagram, and draw the transformer phasing dots in the right places to ensure regenerative feedback:

$$\epsfbox{02633x01.eps}$$

Also, write the equation describing the operating frequency of this type of oscillator circuit.

\underbar{file 02633}
%(END_QUESTION)





%(BEGIN_ANSWER)

This is an {\it Armstrong} oscillator circuit, and the combination of capacitor $C_3$ and primary transformer winding inductance $L_1$ establishes its frequency of operation.

$$f = {1 \over {2 \pi \sqrt{L_1 C_3}}}$$

$$\epsfbox{02633x02.eps}$$

%(END_ANSWER)





%(BEGIN_NOTES)

Ask your students to describe the amount of phase shift the transformer-based tank circuit provides to the feedback signal.  Having them place phasing dots near the transformer windings is a great review of this topic, and a practical context for winding "polarity".  Also, ask them to explain how the oscillator circuit's natural frequency may be altered.

%INDEX% Armstrong oscillator circuit
%INDEX% Dot convention for transformer winding "polarity" (Armstrong oscillator)
%INDEX% Oscillator circuit, Armstrong
%INDEX% Phase marking, transformer winding (Armstrong oscillator)
%INDEX% Polarity, transformer windings (Armstrong oscillator)
%INDEX% Transformer winding "polarity" (Armstrong oscillator)

%(END_NOTES)


