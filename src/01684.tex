
%(BEGIN_QUESTION)
% Copyright 2003, Tony R. Kuphaldt, released under the Creative Commons Attribution License (v 1.0)
% This means you may do almost anything with this work of mine, so long as you give me proper credit

$$\epsfbox{01684x01.eps}$$

\underbar{file 01684}
\vfil \eject
%(END_QUESTION)





%(BEGIN_ANSWER)

Use circuit simulation software to verify your predicted and measured parameter values.

%(END_ANSWER)





%(BEGIN_NOTES)

By the word {\it overtone}, I imply the musical concept: the successive harmonics appearing in this particular non-sinusoidal waveform.  If I were to say "1st harmonic," "2nd harmonic," and "3rd harmonic," this would specifically refer to $f_{fundamental}$, $2f_{fundamental}$, and $3f_{fundamental}$, respectively.  The first of these would be given, the second nonexistent in a square wave with 50\% duty cycle, leaving only the third as something to actually predict and measure.  By "overtones," I mean the first, second, and third frequencies greater than the fundamental frequency found in a square wave, that happen to be harmonics of the fundamental.

If you lack a spectrum analyzer in your lab, fear not!  There are free software packages in existence allowing you to use the audio input of a personal computer's sound card as a (limited) spectrum analyzer and oscilloscope!  You may find some of these packages by searching on the Internet.  One that I've used (2002) successfully in my own class is called WinScope.

%INDEX% Assessment, performance-based (Fourier analysis of a square-wave signal)

%(END_NOTES)


