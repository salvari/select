
%(BEGIN_QUESTION)
% Copyright 2003, Tony R. Kuphaldt, released under the Creative Commons Attribution License (v 1.0)
% This means you may do almost anything with this work of mine, so long as you give me proper credit

$$\epsfbox{01950x01.eps}$$

\underbar{file 01950}
\vfil \eject
%(END_QUESTION)





%(BEGIN_ANSWER)

Use circuit simulation software to verify your predicted and measured parameter values.

%(END_ANSWER)





%(BEGIN_NOTES)

Use a variable-voltage, regulated power supply to supply any amount of DC voltage below 30 volts.  Specify standard resistor values, all between 1 k$\Omega$ and 100 k$\Omega$ (1k5, 2k2, 2k7, 3k3, 4k7, 5k1, 6k8, 10k, 22k, 33k, 39k 47k, 68k, etc.). 

I have had relatively good success with the following values:

\medskip
\item{$\bullet$} $V_{CC}$ = 9 volts (from battery)
\item{$\bullet$} $C_1$ through $C_3$ = 0.001 $\mu$F
\item{$\bullet$} $C_4$ and $C_5$ = 4.7 $\mu$F
\item{$\bullet$} $R_1$ through $R_3$ = 10 k$\Omega$
\item{$\bullet$} $R_4$ = 270 k$\Omega$
\item{$\bullet$} $R_5$ = 50 k$\Omega$ (two 100 k$\Omega$ resistors in parallel)
\item{$\bullet$} $R_6$ = 12 k$\Omega$ (you might want to make this resistor variable so students can  experiment with $A_V$)
\item{$\bullet$} $R_7$ = 1 k$\Omega$
\item{$\bullet$} $Q_1$ = part number 2N3403
\medskip

One of the problems with the RC phase-shift oscillator circuit design is the loading of the phase-shift network by the transistor's biasing network ($R_4$ and $R_5$), which will offset the predicted oscillation frequency from what you might expect from the RC network alone.  While it is possible to account for all the factors in this circuit, it is not a simple task for students just beginning to understand how the circuit is supposed to work.

I have also noticed that the frequency of this circuit is significantly reduced by the capacitance of any test leads connected to it.  Beware of oscilloscope probe cables -- the capacitance they add to the circuit will offset the oscillation frequency!

An extension of this exercise is to incorporate troubleshooting questions.  Whether using this exercise as a performance assessment or simply as a concept-building lab, you might want to follow up your students' results by asking them to predict the consequences of certain circuit faults.

%INDEX% Assessment, performance-based (RC phase-shift oscillator, BJT)

%(END_NOTES)


