
%(BEGIN_QUESTION)
% Copyright 2003, Tony R. Kuphaldt, released under the Creative Commons Attribution License (v 1.0)
% This means you may do almost anything with this work of mine, so long as you give me proper credit

A {\it very} common application of electromechanical relay logic is motor control circuitry.  Here is a ladder diagram for a simple DC motor control, where a momentary pushbutton switch starts the motor, and another pushbutton switch stops the motor:

$$\epsfbox{01295x01.eps}$$

Translate this ladder diagram into point-to-point connections between the following components (shown in the following illustration):

$$\epsfbox{01295x02.eps}$$

\underbar{file 01295}
%(END_QUESTION)





%(BEGIN_ANSWER)

The wiring sequence shown here is not the only valid solution to this problem!

$$\epsfbox{01295x03.eps}$$

%(END_ANSWER)





%(BEGIN_NOTES)

This circuit provides students with an opportunity to analyze a simple {\it latch}: a system that "remembers" prior switch actuations by holding a "state" (either set or reset; latched or unlatched).  A simple motor start/stop circuit such as this is about as simple as latch circuits get.

Students should be able to immediately comprehend the benefit of using nice, neat, structured ladder diagrams when they see the tangled mess of wires in a real motor control circuit.  And this is not even a complex motor control circuit!  It takes very little imagination to think of something even uglier than this, and what a task it would be to troubleshoot such a circuit without the benefit of a ladder diagram for guidance.

%INDEX% Motor control circuit

%(END_NOTES)


