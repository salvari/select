
%(BEGIN_QUESTION)
% Copyright 2004, Tony R. Kuphaldt, released under the Creative Commons Attribution License (v 1.0)
% This means you may do almost anything with this work of mine, so long as you give me proper credit

Although radio transmitter antennae ideally possesses only inductance and capacitance (no resistance), in practice they are found to be very dissipative.  In other words, they tend to act as large {\it resistors} to the transmitters they are connected to.  Explain why this is.  In what form is the dissipated energy manifest (heat, light, or something else)?

\underbar{file 02280}
%(END_QUESTION)





%(BEGIN_ANSWER)

Ideally, 100\% of the energy input to an antenna leaves in the form of electromagnetic radiation.

%(END_ANSWER)





%(BEGIN_NOTES)

Although students may have some to associate the concept of "dissipation" exclusively with resistors, this is not entirely correct.  All that is meant by "dissipation" is the dispersal of energy; that is, energy leaving an electric circuit and not returning.  With resistors, this occurs in the form of heat, but this is not the only kind of dissipation!  In electric motors, most of the energy is dissipated in the form of mechanical energy, which goes into doing {\it work} (and some heat, of course).  Light bulbs dissipate energy in the form of light, not just heat.

%INDEX% Antenna, as a dissipative load

%(END_NOTES)


