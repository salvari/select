
%(BEGIN_QUESTION)
% Copyright 2003, Tony R. Kuphaldt, released under the Creative Commons Attribution License (v 1.0)
% This means you may do almost anything with this work of mine, so long as you give me proper credit

How might you use a meter (or a conductivity/continuity tester) to determine whether this electrical switch is in the {\it open} or {\it closed} state?

$$\epsfbox{00065x01.eps}$$

\underbar{file 00065}
%(END_QUESTION)





%(BEGIN_ANSWER)

Most multimeters have a "resistance" measurement range ("Ohms scale") that may be used to check continuity.  Either using a meter or a conductivity/continuity tester, measure between the two screw terminals of this switch: if the resistance is low (good conductivity), then the switch is {\it closed}.  If the measured resistance is infinite (no conductivity), then the switch is {\it open}.

%(END_ANSWER)





%(BEGIN_NOTES)

This is another question which lends itself well to experimentation.  A vitally important skill for students to develop is how to use their test equipment to diagnose the states of individual components.

An inexpensive source of simple (SPST) switches is a hardware store: use the same type of switch that is used in household light control.  These switches are very inexpensive, rugged, and come with heavy-duty screw terminals for wire attachment.  When used in small battery-powered projects, they are nearly indestructible!

%INDEX% Ohmmeter usage
%INDEX% Conductivity
%INDEX% Continuity

%(END_NOTES)


