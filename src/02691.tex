
%(BEGIN_QUESTION)
% Copyright 2005, Tony R. Kuphaldt, released under the Creative Commons Attribution License (v 1.0)
% This means you may do almost anything with this work of mine, so long as you give me proper credit

You may be wondering why anyone would bother using logarithms to solve arithmetic problems for which we have perfectly good and effective digital electronic calculator functions at our disposal.  For example, why would anyone do this:

$$10^{\log 7 + \log 5}$$

. . . when they could just do the following on the same calculator?

$$7 \times 5$$

The quick answer to this very good question is, "when it is more difficult to directly multiply two numbers."  The trouble is, most people have a difficult time imagining when it would ever be easier to take two logarithms, add them together, and raise ten to that power than it would be to simply multiply the original two numbers together.

The answer to {\it that} mystery is found in operational amplifier circuitry.  As it turns out, it is much easier to build single opamp circuits that add, subtract, exponentiate, or take logarithms than it is to build one that directly multiplies or divides two quantities (analog voltages) together.

We may think of these opamp functions as "blocks" which may be interconnected to perform composite arithmetic functions:

$$\epsfbox{02691x01.eps}$$

Using this model of specific math-function "blocks," show how the following set of analog math function blocks may be connected together to multiply two analog voltages together:

$$\epsfbox{02691x02.eps}$$

\underbar{file 02691}
%(END_QUESTION)





%(BEGIN_ANSWER)

$$\epsfbox{02691x03.eps}$$

%(END_ANSWER)





%(BEGIN_NOTES)

The purpose of this question is simple: to provide a practical application for logarithms as computational aids in an age of cheap, ubiquitous, digital computing devices.

%INDEX% Logarithms and antilogarithms, used in analog circuitry

%(END_NOTES)


