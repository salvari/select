
%(BEGIN_QUESTION)
% Copyright 2005, Tony R. Kuphaldt, released under the Creative Commons Attribution License (v 1.0)
% This means you may do almost anything with this work of mine, so long as you give me proper credit

Predict how all component voltages and currents in this circuit will be affected as a result of the following faults.  Consider each fault independently (i.e. one at a time, no multiple faults):

$$\epsfbox{03703x01.eps}$$

\medskip
\item{$\bullet$} Any one diode fails open:
\vskip 5pt
\item{$\bullet$} Transformer secondary winding fails open:
\vskip 5pt
\item{$\bullet$} Inductor $L_1$ fails open:
\vskip 5pt
\item{$\bullet$} Capacitor $C_1$ fails shorted:
\medskip

For each of these conditions, explain {\it why} the resulting effects will occur.

\underbar{file 03703}
%(END_QUESTION)





%(BEGIN_ANSWER)

\medskip
\item{$\bullet$} Any one diode fails open: {\it Half-wave rectification rather than full-wave, less DC voltage across load, more ripple (AC) voltage across load}.
\vskip 5pt
\item{$\bullet$} Transformer secondary winding fails open: {\it no voltage or current on secondary side of circuit after $C_1$ discharges through load, little current through primary winding}.
\vskip 5pt
\item{$\bullet$} Inductor $L_1$ fails open: {\it no voltage across load, no current through load, no current through rest of secondary-side components, little current through primary winding}.
\vskip 5pt
\item{$\bullet$} Capacitor $C_1$ fails shorted: {\it increased current through both transformer windings, increased current through diodes, increased current through inductor, little voltage across or current through load, capacitor and all diodes will likely get hot}.
\medskip

%(END_ANSWER)





%(BEGIN_NOTES)

The purpose of this question is to approach the domain of circuit troubleshooting from a perspective of knowing what the fault is, rather than only knowing what the symptoms are.  Although this is not necessarily a realistic perspective, it helps students build the foundational knowledge necessary to diagnose a faulted circuit from empirical data.  Questions such as this should be followed (eventually) by other questions asking students to identify likely faults based on measurements.

%INDEX% Troubleshooting, predicting effects of fault in AC-DC power supply circuit

%(END_NOTES)


