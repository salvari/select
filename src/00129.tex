
%(BEGIN_QUESTION)
% Copyright 2003, Tony R. Kuphaldt, released under the Creative Commons Attribution License (v 1.0)
% This means you may do almost anything with this work of mine, so long as you give me proper credit

What does the "50 ohm" rating of an RG-58/U coaxial cable represent?  Explain how a simple cable, with no continuity between its two conductors, could possibly be rated in {\it ohms}.

Hint: this "50 ohm" rating is commonly referred to as the {\it characteristic impedance} of the cable.  Another term for this parameter is {\it surge impedance}, which I think is more descriptive.

\underbar{file 00129}
%(END_QUESTION)





%(BEGIN_ANSWER)

A cable with a {\it characteristic}, or {\it surge}, impedance of 50 ohms behaves as a 50-ohm resistor to any voltage surges impressed at either end, at least until the surge has had enough time to propagate down the cable's full length and back again.

%(END_ANSWER)





%(BEGIN_NOTES)

This concept will seem very strange to students who are only familiar with resistance in the context of resistors and other simple electrical components, where resistance does not change appreciably over time.  In this example, though, the "resistance" of the cable is extremely time-dependent, and the time spans involved are typically very short -- so short that measurements made with ohmmeters will not reveal it at all!

%INDEX% Characteristic impedance, defined
%INDEX% Surge impedance, defined

%(END_NOTES)


