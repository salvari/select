
%(BEGIN_QUESTION)
% Copyright 2003, Tony R. Kuphaldt, released under the Creative Commons Attribution License (v 1.0)
% This means you may do almost anything with this work of mine, so long as you give me proper credit

Write the Boolean expression for this TTL logic gate circuit, then reduce that expression to its simplest form using any applicable Boolean laws and theorems.  Finally, draw a new gate circuit diagram based on the simplified Boolean expression that performs the exact same logic function.

$$\epsfbox{01318x01.eps}$$

\underbar{file 01318}
%(END_QUESTION)





%(BEGIN_ANSWER)

Original Boolean expression: $\overline{AB + AC}$

\vskip 10pt

Reduced gate circuit:

$$\epsfbox{01318x02.eps}$$

%(END_ANSWER)





%(BEGIN_NOTES)

The Boolean simplification for this particular problem is tricky.  Remind students that complementation bars act as {\it grouping symbols}, and that parentheses should be used when in doubt to maintain grouping after "breaking bars" with DeMorgan's Theorem.

Ask your students to compare the "simplified" circuit with the original circuit.  Are any advantages apparent to the version given in the answer?  Certainly, the Boolean expression for that version of the circuit is simpler compared to that of the original circuit, but is the circuit itself significantly improved?

This question underscores an important lesson about Boolean algebra and logic simplification in general: just because a mathematical expression is simpler does not necessarily mean that the expression's physical realization will be any simpler than the original!

%INDEX% Boolean algebra, gate circuit simplification

%(END_NOTES)


