
%(BEGIN_QUESTION)
% Copyright 2005, Tony R. Kuphaldt, released under the Creative Commons Attribution License (v 1.0)
% This means you may do almost anything with this work of mine, so long as you give me proper credit

An unfortunate tendency that many new students have is to immediately plug numbers into equations when faced with a time-constant circuit problem, before carefully considering the circuit.  Explain why the following steps are very wise to follow {\it before} performing any mathematical calculations:

\medskip
\item{$\bullet$} Step 1: Identify and list all the known ("given") quantities.
\item{$\bullet$} Step 2: Draw a schematic of the circuit, if none is given to you.
\item{$\bullet$} Step 3: Label components in the schematic with all known quantities. 
\item{$\bullet$} Step 4: Sketch a rough plot of how you expect the variable(s) in the circuit to change over time.
\item{$\bullet$} Step 5: Label starting and final values for these graphed variables, wherever possible.
\medskip

\underbar{file 03553}
%(END_QUESTION)





%(BEGIN_ANSWER)

I'll let you discuss this question with your classmates and instructor!

%(END_ANSWER)





%(BEGIN_NOTES)

This is advice I always give my students, after seeing so many students get themselves into trouble by blindly plugging numbers into equations.  {\it Think} before you act, is the motto here!

Actually, this general advice applies to most all physics-type problems: identify what it is you're trying to solve and what you have to work with before jumping into calculations.

%INDEX% Time constant calculation, RC or LR circuit

%(END_NOTES)


