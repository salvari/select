
%(BEGIN_QUESTION)
% Copyright 2003, Tony R. Kuphaldt, released under the Creative Commons Attribution License (v 1.0)
% This means you may do almost anything with this work of mine, so long as you give me proper credit

The {\it Orbiting Astronomical Observatory} was a NASA project during the late 1960's and 1970's to place precision observational instruments in earth orbit for scientific purposes.  Satellites designed for this program had to have "hardened" circuitry to withstand the radiation, extreme temperatures, and other harsh conditions of space.

An example of some of this "fail-safe" circuitry is shown here: a quad-redundant inverter (NOT) gate:

$$\epsfbox{01330x01.eps}$$

Explain why the circuit is referred to as {\it quad}-redundant.  How many individual component failures, minimum, must occur before the gate's functionality is compromised?  Prove your answer through an analysis of the circuit's operation.

\underbar{file 01330}
%(END_QUESTION)





%(BEGIN_ANSWER)

If you analyze this circuit carefully, you will find that it may actually fail with just {\it two} component faults, if they are the right type of fault, in the right locations!

%(END_ANSWER)





%(BEGIN_NOTES)

This circuit is a good one to discuss with your students in class.  Ask them to explain its basic operation: if all components are functioning properly, what happens when it receives a "high" input, versus a "low" input?  Do the input signals need to be current-{\it sourcing} or current-{\it sinking}?  What about the output of this circuit: does it source or sink current?

%(END_NOTES)


