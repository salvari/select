
%(BEGIN_QUESTION)
% Copyright 2003, Tony R. Kuphaldt, released under the Creative Commons Attribution License (v 1.0)
% This means you may do almost anything with this work of mine, so long as you give me proper credit

Analog computers have been all but replaced by digital computers in modern electronic systems.  Yet, analog computational circuits still enjoy certain advantages over digital circuits.  Describe what some of the limitations of analog computers are, and why these limitations have led to their obsolescence.  Also, discuss some of the advantages of analog computers, and why a designer might still choose to use an analog computational circuit in a modern system.

\underbar{file 01005}
%(END_QUESTION)





%(BEGIN_ANSWER)

Analog computational circuits are much less precise than their digital counterparts.  On the other hand, analog circuits tend to be much simpler than digital circuits (for the same functions), and they are generally faster.

%(END_ANSWER)





%(BEGIN_NOTES)

I like to introduce analog computational circuits to beginning electronics students because of their elegant simplicity, and for the fact that they greatly help to "link" the abstract world of mathematics to real mechanisms.  Students are generally excited to realize they can build an actual {\it computer} with just a handful of inexpensive electronic components.

%INDEX% Analog computer
%INDEX% Computer, analog

%(END_NOTES)


