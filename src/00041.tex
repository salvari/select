
%(BEGIN_QUESTION)
% Copyright 2003, Tony R. Kuphaldt, released under the Creative Commons Attribution License (v 1.0)
% This means you may do almost anything with this work of mine, so long as you give me proper credit

Suppose someone mechanically couples an electric motor to an electric generator, then electrically couples the two devices together in an effort to make a perpetual-motion machine:

$$\epsfbox{00041x01.eps}$$

Why won't this assembly spin forever, once started?

\underbar{file 00041}
%(END_QUESTION)





%(BEGIN_ANSWER)

This will not work because neither the motor nor the generator is 100\% efficient.

%(END_ANSWER)





%(BEGIN_NOTES)

The easy answer to this question is "the Law of Conservation of Energy (or the Second Law of Thermodynamics) forbids it," but citing such a "Law" really doesn't explain {\it why} perpetual motion machines are doomed to failure.  It is important for students to realize that reality is not {\it bound} to the physical "Laws" scientists set; rather, what we call "Laws" are actually just {\it descriptions} of regularities seen in nature.  It is important to emphasize critical thinking in a question like this, for it is no more intellectually mature to deny the possibility of an event based on dogmatic adherence to a Law than it is to naively believe that anything is possible.

%INDEX% Perpetual motion
%INDEX% Efficiency

%(END_NOTES)


