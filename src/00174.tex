
%(BEGIN_QUESTION)
% Copyright 2003, Tony R. Kuphaldt, released under the Creative Commons Attribution License (v 1.0)
% This means you may do almost anything with this work of mine, so long as you give me proper credit

Which magnet motion past the wire will produce the greatest voltmeter indication: perpendicular, parallel, or no motion at all?

$$\epsfbox{00174x01.eps}$$

\underbar{file 00174}
%(END_QUESTION)





%(BEGIN_ANSWER)

The answer to this question is easy enough to determine experimentally.  I'll let you discover it for yourself rather than give you the answer here.

Hint: the voltage generated by a magnetic field with a single wire is quite weak, so I recommend using a very sensitive voltmeter and/or a powerful magnet.  Also, if the meter is analog (has a moving pointer and a scale rather than a digital display), you must keep it far away from the magnet, so that the magnet's field does not directly influence the pointer position.

\vskip 10pt

Follow-up question: identify some potential problems which could arise in this experiment to prevent induction from occurring.

%(END_ANSWER)





%(BEGIN_NOTES)

This is another one of those concepts that is better learned through experimentation than by direct pronouncement, especially since the experiment itself is so easy to set up.

%INDEX% Electromagnetic induction
%INDEX% Induction, electromagnetic

%(END_NOTES)


