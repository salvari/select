
%(BEGIN_QUESTION)
% Copyright 2006, Tony R. Kuphaldt, released under the Creative Commons Attribution License (v 1.0)
% This means you may do almost anything with this work of mine, so long as you give me proper credit

$$\epsfbox{03151x01.eps}$$

\underbar{file 03151}
\vfil \eject
%(END_QUESTION)





%(BEGIN_ANSWER)

The real circuit you build will validate your circuit design.

%(END_ANSWER)





%(BEGIN_NOTES)

An extension of this exercise is to incorporate troubleshooting questions.  Whether using this exercise as a performance assessment or simply as a concept-building lab, you might want to follow up your students' results by asking them to predict the consequences of certain circuit faults.

The two diodes in this circuit are a matter of necessity: getting the circuit to work with only two sets of switch contacts per relay.  Ideally, each relay would be 3PDT with separate contact sets for latching, interlocking, and motor power.  To use a DPDT relay requires that one of these contact sets do double-duty.  In this case, one of the contact sets on each relay handling power to the motor must also handle the job of seal-in (latching).  Without the diodes in place, both relays chatter when either motion button is pressed.  This is because both relay coils receive power: one coil directly through the switch; the other through the same switch, back through the motor, and then through the seal-in (latching) connection.  The diodes prevent this "feed-through" to the other relay coil from happening, without interfering with the normal latching function.

%INDEX% Assessment, performance-based (Reversing start/stop motor control circuit)

%(END_NOTES)


