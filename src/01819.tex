
%(BEGIN_QUESTION)
% Copyright 2003, Tony R. Kuphaldt, released under the Creative Commons Attribution License (v 1.0)
% This means you may do almost anything with this work of mine, so long as you give me proper credit

{\it Radio waves} are comprised of oscillating electric and magnetic fields, which radiate away from sources of high-frequency AC at (nearly) the speed of light.  An important measure of a radio wave is its {\it wavelength}, defined as the distance the wave travels in one complete cycle.

Suppose a radio transmitter operates at a fixed frequency of 950 kHz.  Calculate the approximate wavelength ($\lambda$) of the radio waves emanating from the transmitter tower, in the metric distance unit of meters.  Also, write the equation you used to solve for $\lambda$.

\underbar{file 01819}
%(END_QUESTION)





%(BEGIN_ANSWER)

$\lambda \approx$ 316 meters

\vskip 10pt

I'll let you find the equation on your own!

%(END_ANSWER)





%(BEGIN_NOTES)

I purposely omit the velocity of light, as well as the time/distance/velocity equation, so that students will have to do some simple research this calculate this value.  Neither of these concepts is beyond high-school level science students, and should pose no difficulty at all for college-level students to find on their own.

%INDEX% Wavelength, of radio wave (calculated)

%(END_NOTES)


