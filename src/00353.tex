
%(BEGIN_QUESTION)
% Copyright 2003, Tony R. Kuphaldt, released under the Creative Commons Attribution License (v 1.0)
% This means you may do almost anything with this work of mine, so long as you give me proper credit

% Uncomment the following line if the question involves calculus at all:
\vbox{\hrule \hbox{\strut \vrule{} $\int f(x) \> dx$ \hskip 5pt {\sl Calculus alert!} \vrule} \hrule}

Suppose an inductor is connected to a variable current source, where the current is steadily increased at a rate of 1.5 amps per second.  How much voltage will the 4 Henry inductor drop, and what will be the polarity of that drop?  Remember, the direction of the arrow in a current source symbol points in the direction of {\it conventional flow}, not electron flow!

$$\epsfbox{00353x01.eps}$$

In real life, an inductor will not drop the exact same amount of voltage that you will calculate here.  Determine if the real voltage drop across such an inductor would be greater or less than predicted, and explain why.

\underbar{file 00353}
%(END_QUESTION)





%(BEGIN_ANSWER)

$$\epsfbox{00353x02.eps}$$

In real life, through, the inductor would drop more than 6 volts, due to winding resistance.

\vskip 10pt

Follow-up question: research the typical winding resistance of a 4 henry inductor.

%(END_ANSWER)





%(BEGIN_NOTES)

Ahhh, the controversy of conventional versus electron flow.  The existence of two contradicting conventions for denoting direction of electric current irritates me to no end, especially when the one upon which almost all electronic device symbolism is based on is actually incorrect with regard to charge flow through metallic conductors (the majority case in electric circuits)!  Your students will surely encounter both "conventional" and "electron" flow in their careers, so be sure to introduce them to {\it both} conventions.

Discuss with your students the consequences of winding resistance in real inductors.  Is it significant?  Work together with your students to calculate how much extra voltage would be dropped across the inductor, based on their research on the typical winding resistance of a 4 henry inductor (ask them where they obtained the information!), given a ${di \over dt}$ rate of 1.5 amps per second.

Ask your students if they think it might be possible to create an inductor with no "stray" resistance at all to interfere with perfect, theoretical inductor behavior.  What would be required to make the "perfect" inductor?

%INDEX% Inductance, "Ohm's Law" for

%(END_NOTES)


