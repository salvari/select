
%(BEGIN_QUESTION)
% Copyright 2003, Tony R. Kuphaldt, released under the Creative Commons Attribution License (v 1.0)
% This means you may do almost anything with this work of mine, so long as you give me proper credit

Suppose a power system were delivering AC power to a resistive load drawing 150 amps:

$$\epsfbox{01723x01.eps}$$

Calculate the load voltage, load power dissipation, the power dissipated by the wire resistance ($R_{wire}$), and the overall power efficiency, indicated by the Greek letter "eta" ($\eta = {P_{load} \over P_{source}}$).

\medskip
\item {} $E_{load} = $ 
\item {} $P_{load} = $ 
\item {} $P_{lines} = $ 
\item {} $\eta = $ 
\medskip

Now, suppose we were to re-design both the generator and the load to operate at 2400 volts instead of 240 volts.  This ten-fold increase in voltage allows just one-tenth the current to convey the same amount of power.  Rather than replace all the wire with different wire, we decide to use the exact same wire as before, having the exact same resistance (0.1 $\Omega$ per length) as before.  Re-calculate load voltage, load power, wasted power, and overall efficiency of this (higher voltage) system:

$$\epsfbox{01723x02.eps}$$

\medskip
\item {} $E_{load} = $ 
\item {} $P_{load} = $ 
\item {} $P_{lines} = $ 
\item {} $\eta = $ 
\medskip

\underbar{file 01723}
%(END_QUESTION)





%(BEGIN_ANSWER)

240 volt system:

\item {} $E_{load} = 210$ volts
\item {} $P_{load} = 31.5$ kW
\item {} $P_{lines} = 4.5$ kW
\item {} $\eta = 87.5 \%$ 
\bigskip

2400 volt system:

\item {} $E_{load} = 2397$ volts
\item {} $P_{load} = 35.96$ kW
\item {} $P_{lines} = 45$ W
\item {} $\eta = 99.88 \%$ 
\medskip

%(END_ANSWER)





%(BEGIN_NOTES)

An example like this usually does a good job clarifying the benefits of using high voltage over low voltage for transmission of large amounts of electrical power over substantial distances.

%INDEX% Efficiency of power lines
%INDEX% Power line resistance
%INDEX% Resistance of power line conductors

%(END_NOTES)


