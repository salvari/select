
%(BEGIN_QUESTION)
% Copyright 2003, Tony R. Kuphaldt, released under the Creative Commons Attribution License (v 1.0)
% This means you may do almost anything with this work of mine, so long as you give me proper credit

What, exactly, is a {\it short circuit}?  What does it mean if a circuit becomes {\it shorted}?  How does this differ from an {\it open} circuit?

\underbar{file 00026}
%(END_QUESTION)





%(BEGIN_ANSWER)

A {\it short circuit} is a circuit having very little resistance, permitting large amounts of current.  If a circuit becomes {\it shorted}, it means that a path for current formerly possessing substantial resistance has been bypassed by a path having negligible (almost zero) resistance.

\vskip 10pt

Conversely, an {\it open circuit} is one where there is a break preventing any current from going through at all.

%(END_ANSWER)





%(BEGIN_NOTES)

Discuss with your students some of the potential hazards of short circuits.  It will then be apparent why a "short circuit" is a bad thing.  Ask students if they can think of any realistic circumstance that could lead to a short-circuit developing.

I have noticed over several years of teaching electronics that the terms "short" or "short-circuit" are often used by new students as generic labels for {\it any} type of circuit fault, rather than the specific condition just described.  This is a habit that must be corrected, if students are to communicate intelligently with others in the profession.  To say that a component "is shorted" means a very definite thing: it is not a generic term for any type of circuit fault.

%INDEX% Short circuit, defined
%INDEX% Open versus short
%INDEX% Short versus open

%(END_NOTES)


