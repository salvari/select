
%(BEGIN_QUESTION)
% Copyright 2003, Tony R. Kuphaldt, released under the Creative Commons Attribution License (v 1.0)
% This means you may do almost anything with this work of mine, so long as you give me proper credit

Large power distribution circuit breakers look nothing like the small breakers seen in residential and commercial electrical systems.  They are large units, which "plug" into cubicles so as to facilitate removal and replacement for routine maintenance.

When securing an electrical system in a {\it zero energy state} prior to commencement of maintenance work, it is common practice to "rack out" any large circuit breakers feeding power to the system.  What exactly does this term mean, and what is the procedure for "racking out" a circuit breaker?

\underbar{file 00570}
%(END_QUESTION)





%(BEGIN_ANSWER)

To "rack out" a circuit breaker means to unplug it from its cubicle so that it cannot conduct electric power to the circuit where work is being performed, even if someone were to close its contacts.

%(END_ANSWER)





%(BEGIN_NOTES)

Ask your students whether the circuit breaker should be opened (turned off) before or after racking it out of the cubicle.  Does this sequence matter?  Why or why not?  Also, ask your students where they think the standard locking and tagging procedures apply in a breaker that it racked out.  What, exactly, should the lock prevent someone from doing?

%INDEX% Lock-out / tag-out
%INDEX% Breaker, racking out
%INDEX% Zero energy state

%(END_NOTES)


