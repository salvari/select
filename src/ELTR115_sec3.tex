
\centerline{\bf ELTR 115 (AC 2), section 3} \bigskip 
 
\vskip 10pt

\noindent
{\bf Recommended schedule}

\vskip 5pt

%%%%%%%%%%%%%%%
\hrule \vskip 5pt
\noindent
\underbar{Day 1}

\hskip 10pt Topics: {\it Mixed-frequency signals and harmonic analysis}
 
\hskip 10pt Questions: {\it 1 through 15}
 
\hskip 10pt Lab Exercise: {\it Digital oscilloscope set-up (question 61)}
 
%INSTRUCTOR \hskip 10pt {\bf Demo: Use graphing calculator to synthesize square wave from sinusoidal harmonics}

%INSTRUCTOR \hskip 10pt {\bf Demo: Show harmonics using a spectrum analyzer and function generator}

%INSTRUCTOR \hskip 10pt {\bf Demo: Show harmonics in power-line signal using a spectrum analyzer and transformer}

%INSTRUCTOR \hskip 10pt {\bf Demo: Show example of spectrum plot from an amplifier datasheet}

\vskip 10pt
%%%%%%%%%%%%%%%
\hrule \vskip 5pt
\noindent
\underbar{Day 2}

\hskip 10pt Topics: {\it Intro to calculus: differentiation and integration (optional)}
 
\hskip 10pt Questions: {\it 16 through 30}
 
\hskip 10pt Lab Exercise: {\it Passive integrator circuit (question 62)}
 
\vskip 10pt
%%%%%%%%%%%%%%%
\hrule \vskip 5pt
\noindent
\underbar{Day 3}

\hskip 10pt Topics: {\it Passive integrator and differentiator circuits}
 
\hskip 10pt Questions: {\it 31 through 45}
 
\hskip 10pt Lab Exercise: {\it Passive differentiator circuit (question 63)}
 
\vskip 10pt
%%%%%%%%%%%%%%%
\hrule \vskip 5pt
\noindent
\underbar{Day 4}

\hskip 10pt Topics: {\it Using oscilloscope trigger controls}
 
\hskip 10pt Questions: {\it 46 through 60}
 
\hskip 10pt Lab Exercise: {\it work on project}

%INSTRUCTOR \hskip 10pt {\bf Demo: Use oscilloscope to show how to trigger a complex, repetitive signal}

\vskip 10pt
%%%%%%%%%%%%%%%
\hrule \vskip 5pt
\noindent
\underbar{Day 5}

\hskip 10pt Exam 3: {\it includes oscilloscope set-up performance assessment}

\hskip 10pt {\bf Project due}

\hskip 10pt Question 64: Sample project grading criteria
 
\vskip 10pt
%%%%%%%%%%%%%%%
\hrule \vskip 5pt
\noindent
\underbar{Practice and challenge problems}

\hskip 10pt Questions: {\it 65 through the end of the worksheet}
 
%%%%%%%%%%%%%%%





\vfil \eject

\centerline{\bf ELTR 115 (AC 2), section 3} \bigskip 
 
\vskip 10pt

\noindent
{\bf Skill standards addressed by this course section}

\vskip 5pt

%%%%%%%%%%%%%%%
\hrule \vskip 10pt
\noindent
\underbar{EIA {\it Raising the Standard; Electronics Technician Skills for Today and Tomorrow}, June 1994}

\vskip 5pt

\medskip
\item{\bf C} {\bf Technical Skills -- AC circuits}
\item{\bf C.02} Demonstrate an understanding of the properties of an AC signal.
\item{\bf C.03} Demonstrate an understanding of the principles of operation and characteristics of sinusoidal and non-sinusoidal wave forms.
\item{\bf C.18} Understand principles and operations of AC differentiator and integrator circuits (determine RC and RL time constants).
\item{\bf C.19} Fabricate and demonstrate AC differentiator and integrator circuits.
\item{\bf C.20} Troubleshoot and repair AC differentiator and integrator circuits.
\medskip

\vskip 5pt

\medskip
\item{\bf B} {\bf Basic and Practical Skills -- Communicating on the Job}
\item{\bf B.01} Use effective written and other communication skills.  {\it Met by group discussion and completion of labwork.}
\item{\bf B.03} Employ appropriate skills for gathering and retaining information.  {\it Met by research and preparation prior to group discussion.}
\item{\bf B.04} Interpret written, graphic, and oral instructions.  {\it Met by completion of labwork.}
\item{\bf B.06} Use language appropriate to the situation.  {\it Met by group discussion and in explaining completed labwork.}
\item{\bf B.07} Participate in meetings in a positive and constructive manner.  {\it Met by group discussion.}
\item{\bf B.08} Use job-related terminology.  {\it Met by group discussion and in explaining completed labwork.}
\item{\bf B.10} Document work projects, procedures, tests, and equipment failures.  {\it Met by project construction and/or troubleshooting assessments.}
\item{\bf C} {\bf Basic and Practical Skills -- Solving Problems and Critical Thinking}
\item{\bf C.01} Identify the problem.  {\it Met by research and preparation prior to group discussion.}
\item{\bf C.03} Identify available solutions and their impact including evaluating credibility of information, and locating information.  {\it Met by research and preparation prior to group discussion.}
\item{\bf C.07} Organize personal workloads.  {\it Met by daily labwork, preparatory research, and project management.}
\item{\bf C.08} Participate in brainstorming sessions to generate new ideas and solve problems.  {\it Met by group discussion.}
\item{\bf D} {\bf Basic and Practical Skills -- Reading}
\item{\bf D.01} Read and apply various sources of technical information (e.g. manufacturer literature, codes, and regulations).  {\it Met by research and preparation prior to group discussion.}
\item{\bf E} {\bf Basic and Practical Skills -- Proficiency in Mathematics}
\item{\bf E.01} Determine if a solution is reasonable.
\item{\bf E.02} Demonstrate ability to use a simple electronic calculator.
\item{\bf E.05} Solve problems and [sic] make applications involving integers, fractions, decimals, percentages, and ratios using order of operations.
\item{\bf E.06} Translate written and/or verbal statements into mathematical expressions.
\item{\bf E.09} Read scale on measurement device(s) and make interpolations where appropriate.  {\it Met by oscilloscope usage.}
\item{\bf E.12} Interpret and use tables, charts, maps, and/or graphs.
\item{\bf E.13} Identify patterns, note trends, and/or draw conclusions from tables, charts, maps, and/or graphs.
\item{\bf E.15} Simplify and solve algebraic expressions and formulas.
\item{\bf E.16} Select and use formulas appropriately.
\item{\bf E.17} Understand and use scientific notation.
\item{\bf E.26} Apply Pythagorean theorem.
\item{\bf E.27} Identify basic functions of sine, cosine, and tangent.
\item{\bf E.28} Compute and solve problems using basic trigonometric functions.
\medskip

%%%%%%%%%%%%%%%






\vfil \eject

\centerline{\bf ELTR 115 (AC 2), section 3} \bigskip 
 
\vskip 10pt

\noindent
{\bf Common areas of confusion for students}

\vskip 5pt

%%%%%%%%%%%%%%%
\hrule \vskip 5pt

\vskip 10pt

\noindent
{\bf Difficult concept: } {\it Fourier analysis.}

No doubt about it, Fourier analysis is a strange concept to understand.  Strange, but incredibly useful!  While it is relatively easy to grasp the principle that we may create a square-shaped wave (or any other symmetrical waveshape) by mixing together the right combinations of sine waves at different frequencies and amplitudes, it is far from obvious that {\it any} periodic waveform may be decomposed into a series of sinusoidal waves the same way.  The practical upshot of this is that is it possible to consider very complex waveshapes as being nothing more than a bunch of sine waves added together.  Since sine waves are easy to analyze in the context of electric circuits, this means we have a way of simplifying what would otherwise be a dauntingly complex problem: analyzing how circuits respond to non-sinusoidal waveforms.

The actual "nuts and bolts" of Fourier analysis is highly mathematical and well beyond the scope of this course.  Right now all I want you to grasp is the concept and significance of equivalence between arbitrary waveshapes and series of sine waves.

A great way to experience this equivalence is to play with a digital oscilloscope with a built-in spectrum analyzer.  By introducing different wave-shape signals to the input and switching back and forth between the time-domain (scope) and frequency-domain (spectrum) displays, you may begin to see patterns that will enlighten your understanding.

\vskip 10pt

\noindent
{\bf Difficult concept: } {\it Rates of change.}

When studying integrator and differentiator circuits, one must think in terms of how fast a variable is changing.  This is the first hurdle in calculus: to comprehend what a rate of change is, and it is not obvious.  One thing I really like about teaching electronics is that capacitor and inductors naturally exhibit the calculus principles of integration and differentiation (with respect to time), and so provide an excellent context in which the electronics student may explore basic principles of calculus.  Integrator and differentiator circuits exploit these properties, so that the output voltage is approximately either the time-integral or time-derivative (respectively) of the input voltage signal.

It is helpful, though, to relate these principles to more ordinary contexts, which is why I often describe rates of change in terms of {\it velocity} and {\it acceleration}.  Velocity is nothing more than a rate of change of position: how quickly one's position is changing over time.  Therefore, if the variable $x$ describes position, then the derivative ${dx \over dt}$ (rate of change of $x$ over time $t$) must describe velocity.  Likewise, acceleration is nothing more than the rate of change of velocity: how quickly velocity changes over time.  If the variable $v$ describes velocity, then the derivative ${dv \over dt}$ must describe velocity.  Or, since we know that velocity is itself the derivative of position, we could describe acceleration as the {\it second derivative} of position: ${d^2 x \over dt^2}$

\vskip 10pt

\noindent
{\bf Difficult concept: } {\it Derivative versus integral.}

The two foundational concepts of calculus are inversely related: {\it differentiation} and {\it integration} are flip-sides of the same coin.  That is to say, one "un-does" the other.

One of the better ways to illustrate the inverse nature of these two operations is to consider them in the context of motion analysis, relating {\it position} ($x$), velocity ($v$), and {\it acceleration} ($a$).  Differentiating with respect to time, the derivative of position is velocity ($v = {dx \over dt}$), and the derivative of velocity is acceleration ($a = {dv \over dt}$).  Integrating with respect to time, the integral of acceleration is velocity ($v = \int a \> dt$) and the integral of velocity is position ($x = \int v \> dt$).

Fortunately, electronics provides a ready context in which to understand differentiation and integration.  It is very easy to build {\it differentiator} and {\it integrator} circuits, which take a voltage signal input and differentiate or integrate (respectively) that signal with respect to time.  This means if we have a voltage signal from a velocity sensor measuring the velocity of an object (such as a robotic arm, for example), we may send that signal through a differentiator circuit to obtain a voltage signal representing the robotic arm's acceleration, or we may send the velocity signal through a integrator circuit to obtain a voltage signal representing the robotic arm's position.






