
%(BEGIN_QUESTION)
% Copyright 2003, Tony R. Kuphaldt, released under the Creative Commons Attribution License (v 1.0)
% This means you may do almost anything with this work of mine, so long as you give me proper credit

Would you characterize this transistor amplifier as being {\it inverting} or {\it noninverting}, with the base terminal of transistor Q1 being considered the input?  Explain your answer.

$$\epsfbox{00886x01.eps}$$

\underbar{file 00886}
%(END_QUESTION)





%(BEGIN_ANSWER)

This is a {\it noninverting} amplifier.

\vskip 10pt

Follow-up question: what happens to the collector-emitter conductivity of transistor Q2 as transistor Q1 passes more current due to an increasing $V_{in}$ signal?

%(END_ANSWER)





%(BEGIN_NOTES)

Analysis of this circuit is aided by applying the "variable resistor" model of the transistor to it.  Substitute variable resistors for the transistors Q1 and Q2, and then have your students analyze it as a voltage divider circuit.

The follow-up question is important because it implies transistor Q2 is not a static entity with changes in Q1's base signal.  The same may be said for Q1 when changes occur at the base of Q2.  Discuss this effect with your students, making sure they understand {\it why} both transistors' conductivity changes with a change in only one of the base voltages.

%INDEX% Differential pair circuit

%(END_NOTES)


