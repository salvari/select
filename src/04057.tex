
%(BEGIN_QUESTION)
% Copyright 2006, Tony R. Kuphaldt, released under the Creative Commons Attribution License (v 1.0)
% This means you may do almost anything with this work of mine, so long as you give me proper credit

The trigonometric function of {\it sine} may be found as the result of an infinite series.  Note that this series assumes the angle $x$ to be expressed in units of {\it radians}, not degrees:

$$\sin x = \sum_{n=0}^{\infty} (-1)^n {x^{2n+1} \over {(2n+1)!}}$$

Approximate the sine of 1 radian ($\sin 1$) in steps, using the following table, then write the partial sum expansion up to $n = 5$:

% No blank lines allowed between lines of an \halign structure!
% I use comments (%) instead, so that TeX doesn't choke.

$$\vbox{\offinterlineskip
\halign{\strut
\vrule \quad\hfil # \ \hfil & 
\vrule \quad\hfil # \ \hfil & 
\vrule \quad\hfil # \ \hfil & 
\vrule \quad\hfil # \ \hfil & 
\vrule \quad\hfil # \ \hfil \vrule \cr
\noalign{\hrule}
%
% First row
$n$ & $(-1)^n$ & $x^{2n+1}$ & $(2n+1)!$ & $\approx \sin x$ \cr
%
\noalign{\hrule}
%
% Another row
0 &  &  &  &  \cr
%
\noalign{\hrule}
%
% Another row
1 &  &  &  &  \cr
%
\noalign{\hrule}
%
% Another row
2 &  &  &  &  \cr
%
\noalign{\hrule}
%
% Another row
3 &  &  &  &  \cr
%
\noalign{\hrule}
%
% Another row
4 &  &  &  &  \cr
%
\noalign{\hrule}
%
% Another row
5 &  &  &  &  \cr
%
\noalign{\hrule}
} % End of \halign 
}$$ % End of \vbox

\vskip 30pt

\underbar{file 04057}
%(END_QUESTION)





%(BEGIN_ANSWER)

$$\vbox{\offinterlineskip
\halign{\strut
\vrule \quad\hfil # \ \hfil & 
\vrule \quad\hfil # \ \hfil & 
\vrule \quad\hfil # \ \hfil & 
\vrule \quad\hfil # \ \hfil & 
\vrule \quad\hfil # \ \hfil \vrule \cr
\noalign{\hrule}
%
% First row
$n$ & $(-1)^n$ & $x^{2n+1}$ & $(2n+1)!$ & $\approx \sin x$ \cr
%
\noalign{\hrule}
%
% Another row
0 & 1 & 1 & 1 & 1 \cr
%
\noalign{\hrule}
%
% Another row
1 & -1 & 1 & 6 & 0.8333333 \cr
%
\noalign{\hrule}
%
% Another row
2 & 1 & 1 & 120 & 0.8416667 \cr
%
\noalign{\hrule}
%
% Another row
3 & -1 & 1 & 5040 & 0.8414683 \cr
%
\noalign{\hrule}
%
% Another row
4 & 1 & 1 & 362880 & 0.8414710 \cr
%
\noalign{\hrule}
%
% Another row
5 & -1 & 1 & 39916800 & 0.8414710 \cr
%
\noalign{\hrule}
} % End of \halign 
}$$ % End of \vbox

\vskip 10pt

Shown here is the partial sum expansion up to $n = 5$:

$$\sin x \approx x - {x^{3} \over 3!} + {x^{5} \over 5!} - {x^{7} \over 7!} + {x^{9} \over 9!} - {x^{11} \over 11!}$$

\vskip 10pt

Challenge question: which series (sine or cosine) would a computer be able to calculate quickest, given a certain number of digits (precision) past the decimal point?  In other words, which of these two infinite series {\it converges fastest}?

\vskip 10pt

Shown here is a partial sum expansion of the cosine function, for comparison:

$$\cos x \approx 1 - {x^{2} \over 2!} + {x^{4} \over 4!} - {x^{6} \over 6!} + {x^{8} \over 8!} - {x^{10} \over 10!}$$

%(END_ANSWER)





%(BEGIN_NOTES)

It should go without saying that students should consult their electronic calculators to see what the actual value of $\sin 1$ is, and compare that to their partial sum approximation.  Students should also feel free to explore the validity of this series by approximating the sine of angles other than 1 radian.

This questions provides students with the opportunity to see how $\sin x$ may be arithmetically calculated.  This, in fact, is how many digital electronic computers determine trigonometric functions: from partial sum approximations.

%INDEX% Series (mathematical), for sine function

%(END_NOTES)


