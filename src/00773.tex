
%(BEGIN_QUESTION)
% Copyright 2003, Tony R. Kuphaldt, released under the Creative Commons Attribution License (v 1.0)
% This means you may do almost anything with this work of mine, so long as you give me proper credit

In AC power systems, a common way of thinking about reactive power among engineers is in terms of production and consumption.  Inductive loads, it is said, {\it consume} reactive power.  Conversely, capacitive loads {\it produce} reactive power.

Explain how the models of "production" and "consumption" relate to reactive power in capacitors and inductors, respectively.  Being that neither type of component actually dissipates or generates electrical energy, how can these terms be appropriate in describing their behavior?

\underbar{file 00773}
%(END_QUESTION)





%(BEGIN_ANSWER)

It is true that inductors and capacitors alike neither dissipate nor generate electrical energy.  They do, however, {\it store} and {\it release} energy.  And they do so in complementary fashion, inductors storing energy at the same time that capacitors release, and visa-versa.

Part of the answer to this question lies in the fact that most large AC loads are inductive in nature.  From a power plant's perspective, the reactive power of a customer (a "consumer" of power) is inductive in nature, and so that form of reactive power would naturally be considered "consumption."

%(END_ANSWER)





%(BEGIN_NOTES)

Ask your students this question: if customers on an electrical system "consume" reactive power, then who has the job of supplying it?  Carrying this question a bit further, are the alternators used to generate power rated in watts or in volt-amps?  Is it possible for an alternator to supply an infinite amount of purely reactive power, or is there some kind of limit inherent to the device?  To phrase the question another way, does the necessity of "supplying" reactive power to customers limit the amount of true power than a power plant may output?

%INDEX% Power factor, production and consumption of reactive power

%(END_NOTES)


