
%(BEGIN_QUESTION)
% Copyright 2003, Tony R. Kuphaldt, released under the Creative Commons Attribution License (v 1.0)
% This means you may do almost anything with this work of mine, so long as you give me proper credit

Safety is very important when designing and/or building a device powered by potentially lethal electrical sources such as residential AC line power.  Explain how you can {\it prove} that the metal case of your power supply is indeed "grounded" for safety, such that an internal fault from one of the "hot" conductors to the metal case will result in a short-circuit that will trip the power receptacle's fuse or breaker rather than shock the individual touching the case?  

Hint: a visual inspection is not good enough, and we don't want to actually create a ground fault situation to test the grounding.

\underbar{file 01505}
%(END_QUESTION)





%(BEGIN_ANSWER)

Measure resistance from the ground prong on the power plug to the metal case.  The resistance should be no more than a few tenths of an ohm.  A resistance measurement of 0.5 ohm or more indicates a poor connection.

%(END_ANSWER)





%(BEGIN_NOTES)

As a general policy, I inspect each and every one of my students' power supplies before energizing for the first time.  This does not mean, though, that I am the one making the meter measurements!  Each student must prove to me that their grounding is adequate by making the meter measurement under my direct supervision.  This way, they learn how to do it themselves while safety is still personally ensured by me.

I also have the students tug on the wire connections while making this test, to ensure that it is not a {\it random} low-resistance measurement we are seeing with the ohmmeter.  Loose wire connections are best tested by resistance measurements under mechanical stress, not by visual inspection alone.

%(END_NOTES)


