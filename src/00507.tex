
%(BEGIN_QUESTION)
% Copyright 2003, Tony R. Kuphaldt, released under the Creative Commons Attribution License (v 1.0)
% This means you may do almost anything with this work of mine, so long as you give me proper credit

An electronics instructor wants to demonstrate to his students the effect of electrical resistance changing with temperature.  To do this, he selects a carbon resistor about 3 centimeters in length and 5 millimeters in diameter, black in color, with a wire at each end, and connects it to an ohmmeter.  Whenever he grasps the resistor between his fingers, the ohmmeter instantly responds by showing a greatly reduced resistance.

What is wrong with this experiment?

\underbar{file 00507}
%(END_QUESTION)





%(BEGIN_ANSWER)

If the change in resistance is truly due to a change in resistor temperature, it should not be {\it instant}.

%(END_ANSWER)





%(BEGIN_NOTES)

I must confess, the genesis of this question was an experience from my own education.  This really happened!  I still remember staring at the demonstration, perplexed that the resistance would change so quickly and so greatly when the instructor grasped the resistor.  I also recall the mild insult the instructor directed at me as I attempted to communicate my confusion: "What's the matter?  Too complicated for you?"  Please, never treat your students like this.

Some students may believe the experiment is flawed because they expect the resistance to rise with increased temperature, rather than fall.  This, however, makes a fundamental assumption about the nature of temperature-induced resistance changes, which is a bad thing in science.  Let the experimental evidence tell you how the phenomenon works, don't tell it what it should do!

Discuss with your students what they think the {\it real} mechanism of resistance change is in this experiment, and how they might modify the experiment so as to isolate temperature as the only changing variable.

%INDEX% Resistance, relation to conductor temperature
%INDEX% Ohmmeter usage

%(END_NOTES)


