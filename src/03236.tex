
%(BEGIN_QUESTION)
% Copyright 2005, Tony R. Kuphaldt, released under the Creative Commons Attribution License (v 1.0)
% This means you may do almost anything with this work of mine, so long as you give me proper credit

\centerline{\bf Animation: telephony multiplexer}

\vskip 10pt

{\it This question consists of a series of images (one per page) that form an animation.  Flip the pages with your fingers to view this animation (or click on the "next" button on your viewer) frame-by-frame.}

\vskip 10pt

The following animation shows how multiple telephone conversations may be "multiplexed" across a single communication channel.  A high switching speed is necessary to make the conversations seamless, which may be simulated by watching the animation at a high frame rate and seeing that the respective conversation outputs appear to be constant.  Some questions to ponder:

\medskip
\goodbreak
\item{$\bullet$} Why do you suppose anyone would do this?  Why not just have five separate lines, one for each conversation?
\item{$\bullet$} How fast do you suppose the mux/demux pair would have to switch in order for the received conversations to appear seamless?
\medskip

\vfil \eject
$$\epsfbox{03236x01.eps}$$

\vfil \eject
$$\epsfbox{03236x02.eps}$$

\vfil \eject
$$\epsfbox{03236x03.eps}$$

\vfil \eject
$$\epsfbox{03236x04.eps}$$

\vfil \eject
$$\epsfbox{03236x05.eps}$$

\underbar{file 03236}

\vfil \eject

%(END_QUESTION)





%(BEGIN_ANSWER)

This animation must be played with a {\it very} fast frame rate to do the principle justice.  If this is not possible, imagine the mux/demux pair moving at a blinding speed -- so fast that your eyes could not follow the motions of the selector switches.  What do you suppose the words next to the output lines (Conversation A, Conversation B, etc.) would look like?

%(END_ANSWER)





%(BEGIN_NOTES)

The purpose of this animation is to let students study the behavior of this multiplexer circuit and reach their own conclusions.  Similar to experimentation in the lab, except that here all the data collection is done visually rather than through the use of test equipment, and the students are able to "see" things that are invisible in real life.

%INDEX% Animation, multiplexer (telephone system)
%INDEX% Animation, telephony multiplexer

%(END_NOTES)


