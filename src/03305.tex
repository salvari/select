
%(BEGIN_QUESTION)
% Copyright 2005, Tony R. Kuphaldt, released under the Creative Commons Attribution License (v 1.0)
% This means you may do almost anything with this work of mine, so long as you give me proper credit

$$\epsfbox{03305x01.eps}$$

\underbar{file 03305}
\vfil \eject
%(END_QUESTION)





%(BEGIN_ANSWER)

You may use circuit simulation software to set up similar oscilloscope display interpretation scenarios, for practice or for verification of what you see in this exercise.

%(END_ANSWER)





%(BEGIN_NOTES)

Use a sine-wave function generator for the AC voltage source, and be sure set the frequency to some reasonable value (well within the capability of both the oscilloscope and counter to measure).

Some digital oscilloscopes have "auto set" controls which automatically set the vertical, horizontal, and triggering controls to "lock in" a waveform.  Be sure students are learning how to set up these controls on their own rather than just pushing the "auto set" button!

%INDEX% Assessment, performance-based (Setting up digital oscilloscope to display a sine wave)

%(END_NOTES)


