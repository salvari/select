
%(BEGIN_QUESTION)
% Copyright 2003, Tony R. Kuphaldt, released under the Creative Commons Attribution License (v 1.0)
% This means you may do almost anything with this work of mine, so long as you give me proper credit

A mechanic goes to school and takes a course in AC electric circuits.  Upon learning about step-up and step-down transformers, he makes the remark that "Transformers act like electrical versions of gears, with different ratios."  

What does the mechanic mean by this statement?  What exactly is a "gear ratio," and is this an accurate analogy for a transformer?

\underbar{file 00253}
%(END_QUESTION)





%(BEGIN_ANSWER)

Just as meshing gears with different tooth counts transform mechanical power between different levels of speed and torque, electrical transformers transform power between different levels of voltage and current.

%(END_ANSWER)





%(BEGIN_NOTES)

Not only is this a sound analogy, but one that many mechanically-minded people relate with easily!  If you happen to have some mechanics in your classroom, provide them with the opportunity to explain the concept of gear ratios to those students who are unaware of gear system mathematics.

%(END_NOTES)


