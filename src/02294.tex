
%(BEGIN_QUESTION)
% Copyright 2005, Tony R. Kuphaldt, released under the Creative Commons Attribution License (v 1.0)
% This means you may do almost anything with this work of mine, so long as you give me proper credit

Determine the "trip" voltage of this comparator circuit: the value of input voltage at which the opamp's output changes state from fully positive to fully negative or visa-versa:

$$\epsfbox{02294x01.eps}$$

Now, what do you suppose would happen if the output were fed back to the noninverting input through a resistor?  You answer merely has to be qualitative, not quantitative:

$$\epsfbox{02294x02.eps}$$

For your information, this circuit configuration is often referred to as a {\it Schmitt trigger}.

\underbar{file 02294}
%(END_QUESTION)





%(BEGIN_ANSWER)

With no feedback resistor, the "trip" voltage would be 9.21 volts.  With the feedback resistor in place, the "trip" voltage would change depending on the state of the opamp's output!

\vskip 10pt

Follow-up question: describe what effect this changing "trip" voltage value will have on the operation of this comparator circuit.

%(END_ANSWER)





%(BEGIN_NOTES)

Schmitt trigger circuits are very popular for their ability to "cleanly" change states given a noisy input signal.  I have intentionally avoided numerical calculations in this question, so that students may concentrate on the {\it concept} of positive feedback and how it affects this circuit.

%INDEX% Positive feedback, in opamp comparator circuit
%INDEX% Schmitt trigger, made from an opamp (comparator)

%(END_NOTES)


