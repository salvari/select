
%(BEGIN_QUESTION)
% Copyright 2003, Tony R. Kuphaldt, released under the Creative Commons Attribution License (v 1.0)
% This means you may do almost anything with this work of mine, so long as you give me proper credit

Suppose that in the course of building this exponential circuit you encounter severe inaccuracies: the circuit seems to work some of the time, but often its output deviates substantially (as much as +/- 10\%) from what it ought to be:

$$\epsfbox{01020x01.eps}$$

Based on what you know of the components in this circuit, what could be varying so much as to cause these errors?  What do you recommend as a solution to the problem?

\underbar{file 01020}
%(END_QUESTION)





%(BEGIN_ANSWER)

The solution is to make sure both transistors are precisely matched, at held at the exact same temperature:

$$\epsfbox{01020x02.eps}$$

\vskip 10pt

Challenge question: is there a part we could order that contains two matched, heat-stabilized transistors for an application such as this?  Are there any other circuit applications you can think of that could benefit from using a precision-matched pair of transistors?

%(END_ANSWER)





%(BEGIN_NOTES)

Ask your students to explain how they know temperature is an influencing factor in the accuracy of this circuit.  Ask them to show any equations describing transistor behavior that demonstrate temperature dependence.

This question provides an opportunity to review the meaning of fractional exponents with your students.  What, exactly, does $y = x^{0.5}$ mean?  Ask your students to write this expression using more common symbols.  Also, ask them what would have to be modified in this circuit to alter the exponent's value.

As for the challenge question, ask your students to produce a part number for the precision-matched transistor pair they find.  Where did they obtain the information on this component?

%INDEX% Exponentiator, nonlinear opamp circuit
%INDEX% Exponents and logarithms, as inverse functions (in a real nonlinear circuit)
%INDEX% Logarithm extractor, nonlinear opamp circuit
%INDEX% Logarithms and exponents, as inverse functions (in a real nonlinear circuit)
%INDEX% Troubleshooting, nonlinear opamp circuit

%(END_NOTES)


