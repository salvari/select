
%(BEGIN_QUESTION)
% Copyright 2005, Tony R. Kuphaldt, released under the Creative Commons Attribution License (v 1.0)
% This means you may do almost anything with this work of mine, so long as you give me proper credit

In order to make the most practical AC generator (or {\it alternator}, as it is also known), which design makes more sense: a stationary permanent magnet with a rotating wire coil, or a rotating permanent magnet with a stationary wire coil?  Explain your choice.

\underbar{file 00817}
%(END_QUESTION)





%(BEGIN_ANSWER)

It is more practical by far to build an alternator with a stationary wire coil and a rotating magnet than to build one with a stationary magnet and a rotating wire coil, because a machine with a rotating coil would require some form of {\it brushes} and {\it slip rings} to conduct power from the rotating shaft to the load.

\vskip 10pt

Follow-up question: what is so bad about brushes and slip rings that we want to avoid them in alternator design if possible?

%(END_ANSWER)





%(BEGIN_NOTES)

Answering the follow-up question may require a bit of research on the part of your students.  Ask them to describe what "brushes" are and what "slip rings" are, and then the mechanical wearing aspects of these parts should become plain.

%INDEX% Alternator design

%(END_NOTES)


