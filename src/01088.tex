
%(BEGIN_QUESTION)
% Copyright 2003, Tony R. Kuphaldt, released under the Creative Commons Attribution License (v 1.0)
% This means you may do almost anything with this work of mine, so long as you give me proper credit

Silicon-controlled rectifiers (SCRs) may be modeled by the following transistor circuit.  Explain how this circuit functions, in the presence of and absence of a "triggering" voltage pulse at the gate terminal:

$$\epsfbox{01088x01.eps}$$

\underbar{file 01088}
%(END_QUESTION)





%(BEGIN_ANSWER)

The positive feedback intrinsic to this circuit gives it hysteretic properties: once triggered "on," it tends to stay on.  When "off," it tends to stay off (until triggered).

%(END_ANSWER)





%(BEGIN_NOTES)

Have students demonstrate the positive feedback "latching" action of this circuit by drawing directions of current on a diagram for the class to see (on the whiteboard, in view of everyone).  Ask your students why the circuit "waits" until a triggering pulse to turn on, and why it "latches" on once triggered.

%INDEX% SCR, equivalent circuit

%(END_NOTES)


