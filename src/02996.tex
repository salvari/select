
%(BEGIN_QUESTION)
% Copyright 2005, Tony R. Kuphaldt, released under the Creative Commons Attribution License (v 1.0)
% This means you may do almost anything with this work of mine, so long as you give me proper credit

An important performance parameter of digital communications networks is the number of {\it bits per second} (bps) of data it can handle.  Unfortunately, a different term called {\it baud} is often used interchangeably with {\it bps}.  Define what "baud" is, and how it differs from "bits per second."

\underbar{file 02996}
%(END_QUESTION)





%(BEGIN_ANSWER)

"Baud" technically refers to the number of logic level transitions (low-to-high or high-to-low) per second on a network, while "bps" actually refers to the number of data bits transmitted per second.  For a specific application where the two terms significantly differ, research a method of data modulation known as {\it Manchester encoding}.

%(END_ANSWER)





%(BEGIN_NOTES)

While some may argue the difference to be academic, I believe that precision of language and precision of thinking are closely related.  The person who does not recognize the difference between "baud" and "bps" most likely knows little about how digital information is encoded for serial transmission.  Of course, this is the ultimate issue -- understanding how digital data is transmitted.  So while we're at it, we might as well address a common mis-use of language and gain a deeper understanding of how things work, right?

%INDEX% Baud, versus bits per second

%(END_NOTES)


