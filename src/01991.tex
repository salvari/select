
%(BEGIN_QUESTION)
% Copyright 2003, Tony R. Kuphaldt, released under the Creative Commons Attribution License (v 1.0)
% This means you may do almost anything with this work of mine, so long as you give me proper credit

$$\epsfbox{01991x01.eps}$$

\underbar{file 01991}
\vfil \eject
%(END_QUESTION)





%(BEGIN_ANSWER)

Use circuit simulation software to verify your predicted and measured parameter values.

%(END_ANSWER)





%(BEGIN_NOTES)

Use a variable-voltage, regulated power supply to supply any amount of DC voltage below 30 volts.  Specify standard resistor values, all between 1 k$\Omega$ and 100 k$\Omega$ (1k5, 2k2, 2k7, 3k3, 4k7, 5k1, 6k8, 10k, 22k, 33k, 39k 47k, 68k, etc.). 

This circuit produces nice, sharp-edged square wave signals at the transistor collector terminals when resistors $R_1$ and $R_4$ are substantially smaller than the combined resistance of resistors $R_2$ and $R_3$ and the respective potentiometer section resistances.  This way, $R_{pot}$, $R_2$, and $R_3$ dominate the capacitors' charging times, making calculation of duty cycle much more accurate.  Component values I've used with success are 1 k$\Omega$ for $R_1$ and $R_4$, 10 k$\Omega$ for $R_2$ and $R_3$, 100 k$\Omega$ for $R_{pot}$, and 0.001 $\mu$F for $C_1$ and $C_2$.  In my prototype circuit, I used 2N2222 bipolar transistors and an IRF510 power MOSFET.

Although small DC motors work well as demonstrative loads, their counter-EMF may wreak havoc with measurements of average load voltage.  Purely resistive loads work best when comparing measured average load voltage against predicted average load voltage.  Also, motors and other inductive loads may cause the MOSFET to switch incorrectly (or not switch at all!) unless a commutating diode is installed to limit the voltage induced by the collapsing magnetic field every time the transistor turns off.

An extension of this exercise is to incorporate troubleshooting questions.  Whether using this exercise as a performance assessment or simply as a concept-building lab, you might want to follow up your students' results by asking them to predict the consequences of certain circuit faults.

%INDEX% Assessment, performance-based (PWM power controller, discrete)

%(END_NOTES)


