
%(BEGIN_QUESTION)
% Copyright 2005, Tony R. Kuphaldt, released under the Creative Commons Attribution License (v 1.0)
% This means you may do almost anything with this work of mine, so long as you give me proper credit

\vbox{\hrule \hbox{\strut \vrule{} $\int f(x) \> dx$ \hskip 5pt {\sl Calculus alert!} \vrule} \hrule}

Shown here is the graph for the function $y = x^2$:

$$\epsfbox{03647x01.eps}$$

Sketch an approximate plot for the {\it derivative} of this function.

\underbar{file 03647}
%(END_QUESTION)





%(BEGIN_ANSWER)

$$\epsfbox{03647x02.eps}$$

\vskip 10pt

\goodbreak
Challenge question: derivatives of power functions are easy to determine if you know the procedure.  In this case, the derivative of the function $y = x^2$ is ${dy \over dx} = 2x$.  Examine the following functions and their derivatives to see if you can recognize the "rule" we follow:

\medskip
\item{$\bullet$} $y = x^3$ \hskip 10pt ${dy \over dx} = 3x^2$
\vskip 10pt 
\item{$\bullet$} $y = x^4$ \hskip 10pt ${dy \over dx} = 4x^3$
\vskip 10pt 
\item{$\bullet$} $y = 2x^4$ \hskip 10pt ${dy \over dx} = 8x^3$
\vskip 10pt 
\item{$\bullet$} $y = 10x^5$ \hskip 10pt ${dy \over dx} = 50x^4$
\vskip 10pt 
\item{$\bullet$} $y = 2x^3 + 5x^2 - 7x$ \hskip 20pt ${dy \over dx} = 6x^2 + 10x - 7$
\vskip 10pt 
\item{$\bullet$} $y = 5x^3 - 2x - 16$ \hskip 20pt ${dy \over dx} = 15x^2 - 2$
\vskip 10pt 
\item{$\bullet$} $y = 4x^7 - 6x^3 + 9x + 1$ \hskip 20pt ${dy \over dx} = 28x^6 - 18x^2 + 9$
\medskip

%(END_ANSWER)





%(BEGIN_NOTES)

Usually students find the concept of the derivative easiest to understand in graphical form: being the {\it slope} of the graph.  This is true whether or not the independent variable is time (an important point given that most "intuitive" examples of the derivative are time-based!).

Even if your students are not yet familiar with the power rule for calculating derivatives, they should be able to tell that $dy \over dx$ is zero when $x = 0$, positive when $x > 0$, and negative when $x < 0$.

%INDEX% Calculus, derivative (defined in a graphical sense)

%(END_NOTES)


