
%(BEGIN_QUESTION)
% Copyright 2005, Tony R. Kuphaldt, released under the Creative Commons Attribution License (v 1.0)
% This means you may do almost anything with this work of mine, so long as you give me proper credit

A {\it function} is a mathematical relationship with an input (usually $x$) and an output (usually $y$).  Here is an example of a simple function:

$$y = 2x + 1$$ 

One way to show the pattern of any given function is with a table of numbers.  Complete this table for the given values of $x$:

% No blank lines allowed between lines of an \halign structure!
% I use comments (%) instead, so that TeX doesn't choke.

$$\vbox{\offinterlineskip
\halign{\strut
\vrule \quad\hfil # \ \hfil & 
\vrule \quad\hfil # \ \hfil \vrule \cr
\noalign{\hrule}
%
% First row
$x$ & $2x + 1$ \cr
%
\noalign{\hrule}
%
% Second row
0 &  \cr
%
\noalign{\hrule}
%
% Third row
1 & \cr
%
\noalign{\hrule}
%
% Fourth row
2 &  \cr
%
\noalign{\hrule}
%
% Fifth row
3 &  \cr
%
\noalign{\hrule}
%
% Sixth row
4 &  \cr
%
\noalign{\hrule}
%
% Seventh row
5 &  \cr
%
\noalign{\hrule}
} % End of \halign 
}$$ % End of \vbox

A more common (and intuitive) way to show the pattern of any given function is with a {\it graph}.  Complete this graph for the same function $y = 2x + 1$.  Consider each division on the axes to be 1 unit:

$$\epsfbox{03057x01.eps}$$

\underbar{file 03057}
%(END_QUESTION)





%(BEGIN_ANSWER)

% No blank lines allowed between lines of an \halign structure!
% I use comments (%) instead, so that TeX doesn't choke.

$$\vbox{\offinterlineskip
\halign{\strut
\vrule \quad\hfil # \ \hfil & 
\vrule \quad\hfil # \ \hfil \vrule \cr
\noalign{\hrule}
%
% First row
$x$ & $2x + 1$ \cr
%
\noalign{\hrule}
%
% Second row
0 & 1 \cr
%
\noalign{\hrule}
%
% Third row
1 & 3 \cr
%
\noalign{\hrule}
%
% Fourth row
2 & 5 \cr
%
\noalign{\hrule}
%
% Fifth row
3 & 7 \cr
%
\noalign{\hrule}
%
% Sixth row
4 & 9 \cr
%
\noalign{\hrule}
%
% Seventh row
5 & 11 \cr
%
\noalign{\hrule}
} % End of \halign 
}$$ % End of \vbox

$$\epsfbox{03057x02.eps}$$

%(END_ANSWER)





%(BEGIN_NOTES)

It is very important for your students to understand graphs, as they are very frequently used to illustrate the behavior of circuits and mathematical functions alike.  Discuss with them how the line represents a continuous string of points and not just the integer values calculated in the table.

%INDEX% Algebra, graphing simple functions
%INDEX% Graphing simple functions, algebra

%(END_NOTES)


