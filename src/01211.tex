
%(BEGIN_QUESTION)
% Copyright 2003, Tony R. Kuphaldt, released under the Creative Commons Attribution License (v 1.0)
% This means you may do almost anything with this work of mine, so long as you give me proper credit

Explain what the {\it Barkhausen criterion} is for an oscillator circuit.  How will the oscillator circuit's performance be affected if the Barkhausen criterion falls below 1, or goes much above 1?

\underbar{file 01211}
%(END_QUESTION)





%(BEGIN_ANSWER)

I'll let you determine exactly what the "Barkhausen" criterion is.  If its value is less than 1, the oscillator's output will diminish in amplitude over time.  If its value is greater than 1, the oscillator's output will not be sinusoidal!

%(END_ANSWER)





%(BEGIN_NOTES)

The question of "What is the Barkhausen criterion" could be answered with a short sentence, memorized verbatim from a textbook.  But what I'm looking for here is real comprehension of the subject.  Have your students explain to you the reason why oscillation amplitude depends on this factor.

%INDEX% Barkhausen criterion, oscillator circuit
%INDEX% Oscillator circuit, Barkhausen criterion

%(END_NOTES)


