
%(BEGIN_QUESTION)
% Copyright 2005, Tony R. Kuphaldt, released under the Creative Commons Attribution License (v 1.0)
% This means you may do almost anything with this work of mine, so long as you give me proper credit

Identify some of the distinguishing characteristics of inverting and noninverting summer circuits.  How may you identify which is which, and how may you determine the proper resistor values to make each one work as it should?

\underbar{file 02520}
%(END_QUESTION)





%(BEGIN_ANSWER)

I won't directly answer the questions here, but I will give some hints.  A noninverting summer circuit is composed of a {\it passive voltage averager} circuit coupled to a {\it noninverting voltage amplifier} with a voltage gain equal to the number of inputs on the averager.  An inverting summer circuit is composed of a {\it passive current summer} node coupled to a {\it current-to-voltage converter}.

%(END_ANSWER)





%(BEGIN_NOTES)

This question is designed to spur discussion amongst your students, exchanging ideas about each circuit's defining characteristics.  Having students explore each circuit type on their own, reaching their own conclusions about how to differentiate the two, is a far more effective way of making them understand the differences than simply telling them outright.

%INDEX% Summer circuits, inverting versus noninverting

%(END_NOTES)


