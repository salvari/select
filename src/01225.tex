
%(BEGIN_QUESTION)
% Copyright 2003, Tony R. Kuphaldt, released under the Creative Commons Attribution License (v 1.0)
% This means you may do almost anything with this work of mine, so long as you give me proper credit

In a computer system that represents all integer quantities using two's complement form, the most significant bit has a negative place-weight.  For an eight-bit system, the place weights are as follows:

\vskip 20pt

$$\epsfbox{01225x01.eps}$$

Given this place-weighting, convert the following eight-bit two's complement binary numbers into decimal form:

\medskip
\item{$\bullet$} $01000101_2$ = 
\item{$\bullet$} $01110000_2$ = 
\item{$\bullet$} $11000001_2$ = 
\item{$\bullet$} $10010111_2$ = 
\item{$\bullet$} $01010101_2$ = 
\item{$\bullet$} $10101010_2$ = 
\item{$\bullet$} $01100101_2$ = 
\medskip

\underbar{file 01225}
%(END_QUESTION)





%(BEGIN_ANSWER)

\medskip
\item{$\bullet$} $01000101_2 = 69_{10}$
\item{$\bullet$} $01110000_2 = 112_{10}$
\item{$\bullet$} $11000001_2 = -63_{10}$
\item{$\bullet$} $10010111_2 = -105_{10}$
\item{$\bullet$} $01010101_2 = 85_{10}$
\item{$\bullet$} $10101010_2 = -86_{10}$
\item{$\bullet$} $01100101_2 = 101_{10}$
\medskip

%(END_ANSWER)





%(BEGIN_NOTES)

Students accustomed to checking their conversions with calculators may find difficulty with these examples, given the negative place weight!  Two's complement notation may seem unusual at first, but it possesses decided advantages in binary arithmetic.

%INDEX% Two's complement conversions

%(END_NOTES)


