
%(BEGIN_QUESTION)
% Copyright 2003, Tony R. Kuphaldt, released under the Creative Commons Attribution License (v 1.0)
% This means you may do almost anything with this work of mine, so long as you give me proper credit

What type of substance(s) must be added to an intrinsic semiconductor in order to produce "donor" electrons?  When this is done, how do we denote this type of "doped" semiconducting substance?

Likewise, what type of substance(s) must be added to an intrinsic semiconductor in order to produce "acceptor" holes?  When this is done, how to we denote this type of "doped" semiconducting substance?

\underbar{file 00907}
%(END_QUESTION)





%(BEGIN_ANSWER)

To create donor electrons, you must add a substance with a greater number of valence electrons than the base semiconductor material.  When this is done, it is called an {\bf N-type} semiconductor.

To create acceptor holes, you must add a substance with a lesser number of valence electrons than the base semiconductor material.  When this is done, it is called a {\bf P-type} semiconductor.

\vskip 10pt

Follow-up question: identify some common "donor" (N-type) and "acceptor" (P-type) dopants.

%(END_ANSWER)





%(BEGIN_NOTES)

When doping silicon and germanium substrates, the materials used are classified as either {\it pentavalent} or {\it trivalent} substances.  Ask your students which one of these terms refers to the greater valence number, and which refers to the lesser valence number.

%INDEX% Doping, semiconductor

%(END_NOTES)


