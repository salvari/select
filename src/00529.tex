
%(BEGIN_QUESTION)
% Copyright 2003, Tony R. Kuphaldt, released under the Creative Commons Attribution License (v 1.0)
% This means you may do almost anything with this work of mine, so long as you give me proper credit

The following voltmeter circuit avoids problems of "loading" when measuring high-resistance voltage sources.  Describe how to operate this circuit, and how loading error is eliminated by using such a {\it potentiometric} instrument:

$$\epsfbox{00529x01.eps}$$

\underbar{file 00529}
%(END_QUESTION)





%(BEGIN_ANSWER)

Move the switch to the "Bal" position and adjust the potentiometer until the meter movement registers zero volts precisely.  Then, move the switch to the "Meas" position and read the voltage directly from the meter movement.

%(END_ANSWER)





%(BEGIN_NOTES)

"Potentiometric" DC voltage measurements used to be commonplace in industry prior to the advent of precision electronic voltmeters with high-resistance inputs.  The technique, though, is certainly not obsolete, and in fact is still employed in metrological laboratories worldwide to obtain the most accurate (no-load) voltage measurements possible.

It is impressive to have students build a potentiometric voltmeter circuit using a cheap analog VOM (Volt-Ohm-Milliammeter), and have it outperform a direct-reading, laboratory-quality digital voltmeter costing hundreds of dollars!  The greater the resistance inherent to the voltage source being measured, the more severe the loading error of any voltmeter, and the more a potentiometric instrument proves its worth. 

%(END_NOTES)


