
%(BEGIN_QUESTION)
% Copyright 2003, Tony R. Kuphaldt, released under the Creative Commons Attribution License (v 1.0)
% This means you may do almost anything with this work of mine, so long as you give me proper credit

A parallel AC circuit draws 100 mA of current through a purely resistive branch and 85 mA of current through a purely capacitive branch:

$$\epsfbox{02091x01.eps}$$

Calculate the total current and the angle $\Theta$ of the total current, explaining your trigonometric method(s) of solution.

\underbar{file 02091}
%(END_QUESTION)





%(BEGIN_ANSWER)

$I_{total}$ = 131.2 mA

$\Theta$ = 40.36$^{o}$

\vskip 10pt

Follow-up question: in calculating $\Theta$, it is recommended to use the arctangent function instead of either the arcsine or arc-cosine functions.  The reason for doing this is accuracy: less possibility of compounded error, due to either rounding and/or calculator-related (keystroke) errors.  Explain why the use of the arctangent function to calculate $\Theta$ incurs less chance of error than either of the other two arcfunctions.

%(END_ANSWER)





%(BEGIN_NOTES)

The follow-up question illustrates an important principle in many different disciplines: avoidance of unnecessary risk by choosing calculation techniques using given quantities instead of derived quantities.  This is a good topic to discuss with your students, so make sure you do so.

%INDEX% Trigonometry, applied to parallel AC currents

%(END_NOTES)


