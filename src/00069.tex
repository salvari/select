
%(BEGIN_QUESTION)
% Copyright 2003, Tony R. Kuphaldt, released under the Creative Commons Attribution License (v 1.0)
% This means you may do almost anything with this work of mine, so long as you give me proper credit

Examine this schematic diagram:

$$\epsfbox{00069x01.eps}$$

Now, without moving the following components, show how they may be connected together with wires to form the same circuit depicted in the schematic diagram above:

$$\epsfbox{00069x02.eps}$$

\underbar{file 00069}
%(END_QUESTION)





%(BEGIN_ANSWER)

$$\epsfbox{00069x03.eps}$$

\vskip 10pt

Follow-up question: suppose the circuit were built like this but the light bulb did not turn on when the switch was closed.  Identify at least five specific things that could be wrong with the circuit to cause the light not to turn on when it should.

%(END_ANSWER)





%(BEGIN_NOTES)

One of the more difficult skills for students to develop is the ability to translate a nice, neat schematic diagram into a messy, real-world circuit, and visa-versa.  Developing this skill requires lots of practice.

In case students have not learned battery symbol convention yet, please point out to them the "+" and "-" polarity marks, and note which side of the battery is which.

One analogy to use for the switch's function that makes sense with the schematic is a drawbridge: when the bridge is down (closed), cars may cross; when the bridge is up (open), cars cannot.

%INDEX% Schematic diagram

%(END_NOTES)


