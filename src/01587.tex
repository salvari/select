
%(BEGIN_QUESTION)
% Copyright 2003, Tony R. Kuphaldt, released under the Creative Commons Attribution License (v 1.0)
% This means you may do almost anything with this work of mine, so long as you give me proper credit

In order to successfully troubleshoot any electronic circuit to the component level, one must have a good understanding of each component's function within the context of that circuit.  Transistor amplifiers are no exception to this rule.  The following schematic shows a simple, two-stage audio amplifier circuit:

$$\epsfbox{01587x01.eps}$$

Identify the role of the following components in this audio amplifier circuit:

\medskip
\item{$\bullet$} The $0.47 \> \mu \hbox{F}$ capacitor connected to the microphone
\item{$\bullet$} The $220 \hbox{ k} \Omega$ and $27 \hbox{ k} \Omega$ resistor pair
\item{$\bullet$} The $4.7 \> \mu \hbox{F}$ electrolytic capacitor connected across the $1.5 \hbox{ k} \Omega$ resistor
\item{$\bullet$} The $33 \> \mu \hbox{F}$ electrolytic capacitor connected to the speaker
\item{$\bullet$} The $47 \> \mu \hbox{F}$ electrolytic capacitor connected to the power supply rail
\medskip

Additionally, answer the following questions concerning the circuit's design:

\medskip
\item{$\bullet$} What configuration is each stage (common-base, common-collector, common-emitter)?
\item{$\bullet$} Why not just use one transistor stage to drive the speaker?  Why is an additional stage necessary?
\item{$\bullet$} What might happen if the $47 \> \mu \hbox{F}$ "decoupling" capacitor were not in the circuit? 
\item{$\bullet$} Why does the second stage of the amplifier not need its own voltage divider to set bias voltage as the first stage does?
\medskip

\underbar{file 01587}
%(END_QUESTION)





%(BEGIN_ANSWER)

\medskip
\item{$\bullet$} The $0.47 \> \mu \hbox{F}$ capacitor connected to the microphone: {\it passes (AC) audio signal, blocks DC bias voltage from reaching microphone}
\item{$\bullet$} The $220 \hbox{ k} \Omega$ and $27 \hbox{ k} \Omega$ resistor pair: {\it sets DC bias voltage for first transistor stage}
\item{$\bullet$} The $4.7 \> \mu \hbox{F}$ electrolytic capacitor connected across the $1.5 \hbox{ k} \Omega$ resistor: {\it bypasses (AC) audio signal around emitter resistor, for maximum AC voltage gain}
\item{$\bullet$} The $33 \> \mu \hbox{F}$ electrolytic capacitor connected to the speaker: {\it couples (AC) audio signal to speaker while blocking DC bias voltage from speaker}
\item{$\bullet$} The $47 \> \mu \hbox{F}$ electrolytic capacitor connected to the power supply rail: {\it "decouples" any AC signal from the power supply, by providing a low-impedance (short) path to ground}
\medskip

The question regarding the necessity of the 47 $\mu$F decoupling capacitor is tricky to answer, so I'll elaborate a bit here.  Power supply decoupling is a good design practice, because it can ward off a wide range of problems.  AC "ripple" voltage should never be present on the power supply "rail" conductors, as transistor circuits function best with pure DC power.  The purpose of a decoupling capacitor is to subdue any ripple, whatever its source, by acting as a low-impedance "short" to ground for AC while not presenting any loading to the DC power.  

Although it may not seem possible at first inspection, the lack of a decoupling capacitor in this audio amplifier circuit can actually lead to self-oscillation (where the amplifier becomes a tone generator) under certain power supply and load conditions!  If the power supply is poorly regulated and/or poorly filtered, the presence of a decoupling capacitor will greatly diminish line-frequency "hum" noise heard in the speaker.

For the rest of the questions, I'll let you figure out answers on your own!

%(END_ANSWER)





%(BEGIN_NOTES)

Incidentally, this circuit makes a good "intercom" amplifier for a student project.  Using a small dynamic speaker for the microphone, and another speaker (or audio headset) on the receiving end of a long cable connected to the amplifier output, students can easily talk between two rooms in a building, or even between buildings.

%INDEX% Amplifier, multi-stage
%INDEX% Amplifier, function of individual components in

%(END_NOTES)


