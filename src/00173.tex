
%(BEGIN_QUESTION)
% Copyright 2003, Tony R. Kuphaldt, released under the Creative Commons Attribution License (v 1.0)
% This means you may do almost anything with this work of mine, so long as you give me proper credit

A {\it permanent magnet} is a device that retains a magnetic field without need for a power source.  Though many of us have experienced the effects of magnetism from a permanent magnet, very few people can describe what {\it causes} permanent magnetism.  Explain the cause of permanent magnetism, in your own words.

\underbar{file 00173}
%(END_QUESTION)





%(BEGIN_ANSWER)

Magnetism is caused by electric charges in motion.  Since electrons in atoms are known to move in certain ways, they are able to produce their own magnetic fields.  In some types of materials, the motions of atomic electrons are easily aligned with respect to one another, causing an overall magnetic field to be produced by the material.

\vskip 10pt

Follow-up question: what does the term {\it retentivity} mean, in relation to permanent magnetism?

%(END_ANSWER)





%(BEGIN_NOTES)

The answers students find to this question may be philosophically unsatisfying.  It is one thing to discover that magnetism is produced by moving electric charges, but quite another to discover (much less explain) just {\it what} a magnetic field is in an ontological sense.  Sure, it is easy to explain what magnetic fields {\it do}, or even how they relate to other phenomenon.  But what, exactly, {\it is} a magnetic field?  This question is on the same level as, "what is an electric charge?"

%INDEX% Permanent magnetism, defined
%INDEX% Magnetism, permanent

%(END_NOTES)


