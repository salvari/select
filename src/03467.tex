
%(BEGIN_QUESTION)
% Copyright 2005, Tony R. Kuphaldt, released under the Creative Commons Attribution License (v 1.0)
% This means you may do almost anything with this work of mine, so long as you give me proper credit

It is common in audio systems to connect a capacitor in series with each "tweeter" (high-frequency) speaker to act as a simple high-pass filter.  The choice of capacitors for this task is important in a high-power audio system.

A friend of mine once had such an arrangement for the tweeter speakers in his car.  Unfortunately, though, the capacitors kept blowing up when he operated the stereo at full volume!  Tired of replacing these non-polarized electrolytic capacitors, he came to me for advice.  I suggested he use mylar or polystyrene capacitors instead of electrolytics.  These were a bit more expensive than electrolytic capacitors, but they did not blow up.  Explain why.

\underbar{file 03467}
%(END_QUESTION)





%(BEGIN_ANSWER)

The issue here was not polarity (AC versus DC), because these were {\it non-polarized} electrolytic capacitors which were blowing up.  What {\it was} an issue was ESR (Equivalent Series Resistance), which electrolytic capacitors are known to have high values of.

%(END_ANSWER)





%(BEGIN_NOTES)

Your students may have to do a bit of refreshing (or first-time research!) on the meaning of ESR before they can understand why large ESR values could cause a capacitor to explode under extreme operating conditions.

%INDEX% Capacitor, non-polarized electrolytic (audio use)
%INDEX% Capacitor, used as HP filter for audio "tweeter" speaker

%(END_NOTES)


