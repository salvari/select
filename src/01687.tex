
%(BEGIN_QUESTION)
% Copyright 2009, Tony R. Kuphaldt, released under the Creative Commons Attribution License (v 1.0)
% This means you may do almost anything with this work of mine, so long as you give me proper credit

$$\epsfbox{01687x01.eps}$$

\underbar{file 01687}
\vfil \eject
%(END_QUESTION)





%(BEGIN_ANSWER)

Use circuit simulation software to verify your predicted and measured parameter values.

%(END_ANSWER)





%(BEGIN_NOTES)

Here, students must calculate values for $C_1$ and $R_1$ that will produce the $V_{out}$ waveshape specified in the "Given conditions" oscilloscope plot.  The input signal, of course, is a square wave.  Students should be able to show mathematically why the time constant of the integrator ($\tau$) must be 1.443 times the waveform's half-period ($e^{-t \over \tau} = {1 \over 2}$).  Instructors, note: the calculations for this circuit, with $V_{out} = {1 \over 3}V_{in}$, are exactly the same as for a 555 timer circuit, because 555 timers also cycle their capacitors' voltages at peak-to-peak values equal to one-third of the supply voltage.

There are many different combinations of values for $C_1$ and $R_1$ possible for any given square-wave signal frequencies.  The purpose of this exercise is for students to be able to predict and select practical component values from their parts kits.

An extension of this exercise is to incorporate troubleshooting questions.  Whether using this exercise as a performance assessment or simply as a concept-building lab, you might want to follow up your students' results by asking them to predict the consequences of certain circuit faults.

%INDEX% Assessment, performance-based (Passive integrator circuit)

%(END_NOTES)


