
%(BEGIN_QUESTION)
% Copyright 2003, Tony R. Kuphaldt, released under the Creative Commons Attribution License (v 1.0)
% This means you may do almost anything with this work of mine, so long as you give me proper credit

Very interesting things happen to resonant systems when they are "excited" by external sources of oscillation.  For example, a {\it pendulum} is a simple example of a mechanically resonant system, and we all know from experience with swings in elementary school that we can make a pendulum achieve very high amplitudes of oscillation if we "oscillate" our legs at just the right times to match the swing's natural (resonant) frequency.

Identify an example of a mechanically resonant system that is "excited" by an external source of oscillations near its resonant frequency.  Hint: research the word "resonance" in engineering textbooks, and you are sure to read about some dramatic examples of resonance in action.

\underbar{file 00603}
%(END_QUESTION)





%(BEGIN_ANSWER)

Large buildings have (very low) resonant frequencies, that may be matched by the motion of the ground in an earthquake, so that even a relatively small earthquake can cause major damage to the building.

\vskip 10pt

Challenge question: after researching the behavior of mechanical resonant systems when driven by external oscillations of the same frequency, determine what the effects might be of external oscillations on an {\it electrical} resonant system.

%(END_ANSWER)





%(BEGIN_NOTES)

Many, many examples of mechanical resonance exist, some of which are quite dramatic.  A famous example of destructive mechanical resonance (of a well-known bridge in Washington state) has been immortalized in video form, and is easily available on the internet.  If possible, provide the means within your classroom to display a video clip on computer, for any of the students who happen to find this video file and bring it to discussion.

%INDEX% Resonant system, externally driven (mechanical)

%(END_NOTES)


