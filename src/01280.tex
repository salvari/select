
%(BEGIN_QUESTION)
% Copyright 2003, Tony R. Kuphaldt, released under the Creative Commons Attribution License (v 1.0)
% This means you may do almost anything with this work of mine, so long as you give me proper credit

Many modern CMOS gate circuits are {\it buffered} with additional transistor stages on their outputs.  For example, an unbuffered AND gate is shown here, with no more transistors than is necessary to fulfill the "AND" logic function:

$$\epsfbox{01280x01.eps}$$

One type of "buffered" CMOS AND gate looks like this:

$$\epsfbox{01280x02.eps}$$

As far as the basic logic function is concerned, the additional transistors are unnecessary.  However, the "buffering" they provide does serve a useful function.  What is that function?  Are there any disadvantages to buffered logic gates, versus unbuffered?

\underbar{file 01280}
%(END_QUESTION)





%(BEGIN_ANSWER)

Buffered gates exhibit better noise immunity than unbuffered gates.  One disadvantage to buffering, though, is increased propagation delay time.

\vskip 10pt

Follow-up question: identify the on/off states of all transistors in the buffered circuit for both (high and low) input conditions.

%(END_ANSWER)





%(BEGIN_NOTES)

Texas Instruments publishes an excellent application report (SCHA004 -- October 2002) comparing buffered versus unbuffered CMOS logic gates.  I highly recommend it for your reference.

%INDEX% Buffered output CMOS gate circuit
%INDEX% CMOS gate circuit, buffered output
%INDEX% CMOS gate circuit, internal schematic

%(END_NOTES)


