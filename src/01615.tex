
%(BEGIN_QUESTION)
% Copyright 2003, Tony R. Kuphaldt, released under the Creative Commons Attribution License (v 1.0)
% This means you may do almost anything with this work of mine, so long as you give me proper credit

$$\epsfbox{01615x01.eps}$$

\underbar{file 01615}
\vfil \eject
%(END_QUESTION)





%(BEGIN_ANSWER)

Use circuit simulation software to verify your predicted and measured parameter values.

%(END_ANSWER)





%(BEGIN_NOTES)

Use a sine-wave function generator for the AC voltage source.  Specify standard resistor, capacitor, and inductor values.  I have used a small 100 mH inductor, a 0.033 $\mu$F capacitor, and 100 $\Omega$ resistor with good success.  In any case I recommend keeping the resonant frequency within the mid-audio range (1 kHz to 10 kHz) to avoid problems with low inductor $Q$ (frequency too low) and stray capacitance and inductance issues (frequency too high).

I also recommend having students use an oscilloscope to measure AC voltage in a circuit such as this, because some digital multimeters have difficulty accurately measuring AC voltage much beyond line frequency range.  I find it particularly helpful to set the oscilloscope to the "X-Y" mode so that it draws a thin line on the screen rather than sweeps across the screen to show an actual waveform.  This makes it easier to measure peak-to-peak voltage.

An extension of this exercise is to incorporate troubleshooting questions.  Whether using this exercise as a performance assessment or simply as a concept-building lab, you might want to follow up your students' results by asking them to predict the consequences of certain circuit faults.

%INDEX% Assessment, performance-based (Resonant band-pass filter circuit, resonant frequency calculation)

%(END_NOTES)


