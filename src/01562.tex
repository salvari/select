
%(BEGIN_QUESTION)
% Copyright 2003, Tony R. Kuphaldt, released under the Creative Commons Attribution License (v 1.0)
% This means you may do almost anything with this work of mine, so long as you give me proper credit

Suppose you had an audio impedance-matching transformer ($1000 \> \Omega \hbox{ to } 8 \> \Omega$), and you wished to know the phase relationships of the windings.  Unfortunately, the only test equipment you have is an analog DC voltmeter and a 9-volt battery.  Explain how you would use these two devices to test winding "polarity."

$$\epsfbox{01562x01.eps}$$

Note: you may disregard the center-tap of the $1000 \> \Omega$ primary winding.

\underbar{file 01562}
%(END_QUESTION)





%(BEGIN_ANSWER)

I won't give you the answer directly here, but I will give you a hint: although transformers cannot function on continuous DC, they will respond to {\it pulses} of current in a given direction!  In other words, you can use intermittent DC from the 9-volt battery to test the windings.

\vskip 10pt

Challenge question: calculate the turns ratio of this transformer, based on the ratio of impedances.

%(END_ANSWER)





%(BEGIN_NOTES)

Note that the answer given to this question leaves the particular details of how to interpret the voltmeter readings unanswered.  Challenge your students to figure this out on their own.

Audio matching transformers are easy to obtain, so I encourage you have your students try this as a lab exercise.

%INDEX% Polarity, transformer windings
%INDEX% Transformer winding "polarity"

%(END_NOTES)


