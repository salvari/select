
%(BEGIN_QUESTION)
% Copyright 2003, Tony R. Kuphaldt, released under the Creative Commons Attribution License (v 1.0)
% This means you may do almost anything with this work of mine, so long as you give me proper credit

How much voltage is there between the following pairs of points in this circuit?

$$\epsfbox{00668x01.eps}$$

\medskip
\item{$\bullet$} Between {\bf D} and {\bf F}: $V_{DF}$ =
\item{$\bullet$} Between {\bf A} and {\bf B}: $V_{AB}$ =
\item{$\bullet$} Between {\bf D} and {\bf E}: $V_{DE}$ =
\item{$\bullet$} Between {\bf C} and {\bf F}: $V_{CF}$ =
\medskip

\underbar{file 00668}
%(END_QUESTION)





%(BEGIN_ANSWER)

\medskip
\item{$\bullet$} Between {\bf D} and {\bf F}: $V_{DF}$ = 16 V
\item{$\bullet$} Between {\bf A} and {\bf B}: $V_{AB}$ = 9.14 V
\item{$\bullet$} Between {\bf D} and {\bf E}: $V_{DE}$ = 4.57 V
\item{$\bullet$} Between {\bf C} and {\bf F}: $V_{CF}$ = 20.57 V
\medskip

%(END_ANSWER)





%(BEGIN_NOTES)

This question checks students' ability to relate the winding ratios of a transformer to voltages in the circuit.  Note that one of the voltages actually exceeds the source voltage of 16 volts, even though this is technically a "step-down" transformer.  The fact that the source voltage is not impressed upon the entire primary winding is key here.

The symbolism for this transformer is common in Europe, but not so common in the United States.

%INDEX% Autotransformer, voltage calculations

%(END_NOTES)


