
%(BEGIN_QUESTION)
% Copyright 2005, Tony R. Kuphaldt, released under the Creative Commons Attribution License (v 1.0)
% This means you may do almost anything with this work of mine, so long as you give me proper credit

A rheostat (variable resistor) and a switch are both examples of electric components exhibiting different degrees of conductivity:

$$\epsfbox{02753x01.eps}$$

Which of these devices would be considered {\it discrete} and which would be considered {\it continuous} in terms of their electrical conductivity?  What do each of these words mean, and how might they apply to variables in electric circuits other than conductivity?

\underbar{file 02753}
%(END_QUESTION)





%(BEGIN_ANSWER)

A {\it continuous} quantity is one that may be smoothly varied from one extreme value to another, while a {\it discrete} quantity is one that can only assume a finite (limited) number of distinct states.  Here, the rheostat exhibits a {\it continuously} adjustable electrical continuity while the switch is {\it discrete} because it can only be conducting or non-conducting.

\vskip 10pt

Follow-up question: what is the difference between a continuous voltage versus a discrete voltage?

%(END_ANSWER)





%(BEGIN_NOTES)

The purpose of this question is to get students thinking in terms of "digital" quantities, which by their very nature are non-continuous.  Since most electronics curricula focus on continuous quantities before discrete, it is good to have students reflect on the inherent simplicity of discrete circuitry and components after having studied continuous (analog) circuitry.

%INDEX% Continuous versus discrete
%INDEX% Discrete versus continuous

%(END_NOTES)


