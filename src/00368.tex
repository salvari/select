
%(BEGIN_QUESTION)
% Copyright 2003, Tony R. Kuphaldt, released under the Creative Commons Attribution License (v 1.0)
% This means you may do almost anything with this work of mine, so long as you give me proper credit

We know that the voltage in a parallel circuit may be calculated with this formula:

$$E = {I_{total} R_{total}}$$

We also know that the current through any single resistor in a parallel circuit may be calculated with this formula:

$$I_R = {E \over R}$$

Combine these two formulae into one, in such a way that the $E$ variable is eliminated, leaving only $I_R$ expressed in terms of $I_{total}$, $R_{total}$, and $R$.

\underbar{file 00368}
%(END_QUESTION)





%(BEGIN_ANSWER)

$$I_R = I_{total} \Bigl({R_{total} \over R}\Bigr)$$

How is this formula similar, and how is it different, from the "voltage divider" formula?

%(END_ANSWER)





%(BEGIN_NOTES)

Though this "current divider formula" may be found in any number of electronics reference books, your students need to understand how to algebraically manipulate the given formulae to arrive at this one.

At first it may seem as though the two divider formulae (voltage versus current) are easy to confuse.  Is it ${R \over R_{total}}$ or ${R_{total} \over R}$?  However, there is a very simple way to remember which fraction belongs with which formula, based on the numerical value of that fraction.  Mention this to your students and at least one of them will be sure to recognize the pattern.

%INDEX% Algebra, manipulating equations
%INDEX% Algebra, substitution
%INDEX% Current divider formula

%(END_NOTES)


