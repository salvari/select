
%(BEGIN_QUESTION)
% Copyright 2003, Tony R. Kuphaldt, released under the Creative Commons Attribution License (v 1.0)
% This means you may do almost anything with this work of mine, so long as you give me proper credit

$$\epsfbox{01613x01.eps}$$

\underbar{file 01613}
\vfil \eject
%(END_QUESTION)





%(BEGIN_ANSWER)

Use circuit simulation software to verify your predicted and measured parameter values.

%(END_ANSWER)





%(BEGIN_NOTES)

Use a sine-wave function generator for the AC voltage source.  Specify standard resistor and capacitor values.

I recommend setting the function generator output for 1 volt, to make it easier for students to measure the point of "cutoff".  You may set it at some other value, though, if you so choose (or let students set the value themselves when they test the circuit!).

I also recommend having students use an oscilloscope to measure AC voltage in a circuit such as this, because some digital multimeters have difficulty accurately measuring AC voltage much beyond line frequency range.  I find it particularly helpful to set the oscilloscope to the "X-Y" mode so that it draws a thin line on the screen rather than sweeps across the screen to show an actual waveform.  This makes it easier to measure peak-to-peak voltage.

An extension of this exercise is to incorporate troubleshooting questions.  Whether using this exercise as a performance assessment or simply as a concept-building lab, you might want to follow up your students' results by asking them to predict the consequences of certain circuit faults.

%INDEX% Assessment, performance-based (Passive high-pass filter circuit, cutoff frequency calculation)

%(END_NOTES)


