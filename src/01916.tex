
%(BEGIN_QUESTION)
% Copyright 2003, Tony R. Kuphaldt, released under the Creative Commons Attribution License (v 1.0)
% This means you may do almost anything with this work of mine, so long as you give me proper credit

Another challenging sort of waveform to "lock in" on an oscilloscope display is one where a high-frequency waveform is superimposed on a low-frequency waveform.  If the two frequencies are not integer multiples (harmonics) of each other, it will be impossible to make {\it both} of them hold still on the oscilloscope display.

However, most oscilloscopes have frequency-specific {\it rejection} controls provided in the trigger circuitry to help the user discriminate between mixed frequencies.  Identify these controls on the oscilloscope panel, and explain which would be used for what circumstances.

$$\epsfbox{01916x01.eps}$$

\underbar{file 01916}
%(END_QUESTION)





%(BEGIN_ANSWER)

$$\epsfbox{01916x02.eps}$$

\vskip 10pt

Follow-up question: identify the filter circuits internal to the oscilloscope associated with each of these "rejection" controls.

%(END_ANSWER)





%(BEGIN_NOTES)

Finding the controls on the oscilloscope panel should present no difficulty for most students, at least once they realize what the controls are called.  The key to answering this question is to research the words "rejection" and "trigger" in the context of oscilloscope controls.

%INDEX% Oscilloscope, HF and LR trigger rejection

%(END_NOTES)


