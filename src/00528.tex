
%(BEGIN_QUESTION)
% Copyright 2003, Tony R. Kuphaldt, released under the Creative Commons Attribution License (v 1.0)
% This means you may do almost anything with this work of mine, so long as you give me proper credit

This voltage divider should output half the battery's voltage between points A and B:

$$\epsfbox{00528x01.eps}$$

However, if you perform this same experiment using a real voltmeter, the measurement obtained with the meter will be substantially different from what {\it should} be there, based on a prediction of ${1 \over 2}E_{battery}$.  Explain why the voltmeter registers as it does.  What is it about this circuit that causes the measurement to be so far off from the prediction, when we know full well that other voltage divider circuits we've constructed do not exhibit any significant error?

\underbar{file 00528}
%(END_QUESTION)





%(BEGIN_ANSWER)

This circuit is a demonstration of voltmeter "loading" on the circuit, by causing a falsely low measurement.

%(END_ANSWER)





%(BEGIN_NOTES)

Meter "loading" is a serious problem in electrical metrology.  It is a basic principle of measurement that a measuring instrument always impacts the quantity being measured, to some extent.  In cases like this, the extent of impact is severe.

For those instructors with some background in quantum physics, please refrain from perpetuating the myth that meter loading is an example of Heisenberg's Uncertainty Principle.  The Uncertainty Principle has nothing to do with the impact of a measuring instrument on something we measure.  Rather it describes an uncertainty {\it inherent to the quantity itself}.  If you want to share a true electrical example of this uncertainty principle with your students, wait until they study harmonics and spectrum analyzers, where you can tell them it is impossible to measure both the {\it instantaneous amplitude} of a signal and the {\it frequency} of a signal with unlimited certainty.

%INDEX% Voltmeter loading

%(END_NOTES)


