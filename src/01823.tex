
%(BEGIN_QUESTION)
% Copyright 2003, Tony R. Kuphaldt, released under the Creative Commons Attribution License (v 1.0)
% This means you may do almost anything with this work of mine, so long as you give me proper credit

$$\epsfbox{01823x01.eps}$$

\underbar{file 01823}
\vfil \eject
%(END_QUESTION)





%(BEGIN_ANSWER)

There really isn't much you can do to verify your experimental results.  That's okay, though, because the results are qualitative anyway.

%(END_ANSWER)





%(BEGIN_NOTES)

If the oscilloscope does not have its own internal square-wave signal source, use a function generator set up to output square waves at 1 volt peak-to-peak at a frequency of 1 kHz.

If this is not the first time students have done this, be sure to "mess up" the oscilloscope controls prior to them making adjustments.  Students {\it must} learn how to quickly configure an oscilloscope's controls to display any arbitrary waveform, if they are to be proficient in using an oscilloscope as a diagnostic tool.

%INDEX% Assessment, performance-based (Probe compensation for oscilloscope)

%(END_NOTES)


