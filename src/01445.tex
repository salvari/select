
%(BEGIN_QUESTION)
% Copyright 2003, Tony R. Kuphaldt, released under the Creative Commons Attribution License (v 1.0)
% This means you may do almost anything with this work of mine, so long as you give me proper credit

Suppose you need to store a text message in digital memory, consisting of 7500 ASCII characters.  What is the most logical memory organization (addresses $\times$ data lines) to do this?  How many address bits would be needed to store these 7500 characters?

\underbar{file 01445}
%(END_QUESTION)





%(BEGIN_ANSWER)

Ideal memory organization: 8k $\times$ 8, thirteen address bits required.

%(END_ANSWER)





%(BEGIN_NOTES)

Be sure to ask your students {\it how} they calculated 13 bits for the address.  Of course, there is the trial-and-error method of trying different powers of two, but there is a much more elegant solution involving logarithms to find the requisite number of bits.

%INDEX% Memory organization

%(END_NOTES)


