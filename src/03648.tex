
%(BEGIN_QUESTION)
% Copyright 2005, Tony R. Kuphaldt, released under the Creative Commons Attribution License (v 1.0)
% This means you may do almost anything with this work of mine, so long as you give me proper credit

\vbox{\hrule \hbox{\strut \vrule{} $\int f(x) \> dx$ \hskip 5pt {\sl Calculus alert!} \vrule} \hrule}

Calculus is widely (and falsely!) believed to be too complicated for the average person to understand.  Yet, anyone who has ever driven a car has an intuitive grasp of calculus' most basic concepts: {\it differentiation} and {\it integration}.  These two complementary operations may be seen at work on the instrument panel of every automobile:

$$\epsfbox{03648x01.eps}$$

On this one instrument, two measurements are given: speed in miles per hour, and distance traveled in miles.  In areas where metric units are used, the units would be kilometers per hour and kilometers, respectively.  Regardless of units, the two variables of speed and distance are related to each other over time by the calculus operations of integration and differentiation.  My question for you is which operation goes which way?

We know that speed is the rate of change of distance over time.  This much is apparent simply by examining the units (miles {\it per hour} indicates a rate of change over time).  Of these two variables, speed and distance, which is the {\it derivative} of the other, and which is the {\it integral} of the other?  Also, determine what happens to the value of each one as the other maintains a constant (non-zero) value.

\underbar{file 03648}
%(END_QUESTION)





%(BEGIN_ANSWER)

Speed is the derivative of distance; distance is the integral of speed.

\vskip 10pt

If the speed holds steady at some non-zero value, the distance will accumulate at a steady rate.  If the distance holds steady, the speed indication will be zero because the car is at rest.

%(END_ANSWER)





%(BEGIN_NOTES)

The goal of this question is to get students thinking in terms of derivative and integral every time they look at their car's speedometer/odometer, and ultimately to grasp the nature of these two calculus operations in terms they are already familiar with.

%INDEX% Calculus, derivative (applied to motion)
%INDEX% Calculus, integral (applied to motion)

%(END_NOTES)

