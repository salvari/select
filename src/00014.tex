
%(BEGIN_QUESTION)
% Copyright 2003, Tony R. Kuphaldt, released under the Creative Commons Attribution License (v 1.0)
% This means you may do almost anything with this work of mine, so long as you give me proper credit

What difference will it make if the switch is located in either of these two alternate locations in the circuit?

$$\epsfbox{00014x01.eps}$$

$$\epsfbox{00014x02.eps}$$

\underbar{file 00014}
%(END_QUESTION)





%(BEGIN_ANSWER)

The choice of switch locations shown in the two alternate diagrams makes no difference at all.  In either case, the switch exerts the same control over the light bulb.

%(END_ANSWER)





%(BEGIN_NOTES)

This is a difficult concept for some students to master.  Make sure they all understand the nature of electrical current and the importance of continuity {\it throughout} the entire circuit.  Perhaps the best way for students to master this concept is to actually build working battery-switch-lamp circuits.  Remind them that their "research" of these worksheet questions is not limited to book reading.  It is not only valid, but {\it preferable} for them to experiment on their own, so long as the voltages are low enough that no shock hazard exists.

One analogy to use for the switch's function that makes sense with the schematic is a drawbridge: when the bridge is down (closed), cars may cross; when the bridge is up (open), cars cannot.

%INDEX% Circuit, simple
%INDEX% Continuity
%INDEX% Irrelevance of switch location in simple circuit

%(END_NOTES)


