
%(BEGIN_QUESTION)
% Copyright 2003, Tony R. Kuphaldt, released under the Creative Commons Attribution License (v 1.0)
% This means you may do almost anything with this work of mine, so long as you give me proper credit

It is a well-known fact that temperature affects the operating parameters of bipolar junction transistors.  This is why grounded-emitter circuits (with no emitter feedback resistor) are not practical as stand-alone amplifier circuits.

Does temperature affect junction field-effect transistors in the same way, or to the same extent?  Design an experiment to determine the answer to this question.

\underbar{file 01182}
%(END_QUESTION)





%(BEGIN_ANSWER)

Did you really think I would tell you the answer to this question?  Build the circuit(s) and discover the answer for yourselves!

%(END_ANSWER)





%(BEGIN_NOTES)

The purpose of this question is to get students thinking in an experimental mode.  It is very important that students learn to set up and run their own experiments, so they will be able to verify (or perhaps discover!) electronic principles after they have graduated from school.  There will be times when the answers they seek are not to be found in a book, and they will have to "let the electrons teach them" what they need to know.

Remind your students that proper scientific experiments include both {\it experimental} and {\it control} subjects, so that results are based upon a comparison of measurements.

%INDEX% Temperature, effect on JFET performance

%(END_NOTES)


