
%(BEGIN_QUESTION)
% Copyright 2005, Tony R. Kuphaldt, released under the Creative Commons Attribution License (v 1.0)
% This means you may do almost anything with this work of mine, so long as you give me proper credit

The voltage gain of a common-emitter transistor amplifier is approximately equal to the collector resistance divided by the emitter resistance:

$$\epsfbox{03107x01.eps}$$

Knowing this, calculate the necessary resistance values for the following fixed-value resistors ($R_1$ and $R_2$) to give this common-emitter amplifier an adjustable voltage gain range of 4 to 7:

$$\epsfbox{03107x02.eps}$$

\underbar{file 03107}
%(END_QUESTION)





%(BEGIN_ANSWER)

$R_1$ = 13.33 k$\Omega$

\vskip 10pt

$R_2$ = 3.333 k$\Omega$

%(END_ANSWER)





%(BEGIN_NOTES)

Have your students show how they set up the system of equations to solve for the two resistor values.  This is a good exercise to do in front of the class, so everyone can see (possibly) different methods of solution.

%INDEX% Simultaneous equations
%INDEX% Systems of nonlinear equations
%INDEX% Voltage gain, common emitter amplifier

%(END_NOTES)


