
%(BEGIN_QUESTION)
% Copyright 2005, Tony R. Kuphaldt, released under the Creative Commons Attribution License (v 1.0)
% This means you may do almost anything with this work of mine, so long as you give me proper credit

Design the simplest relay circuit possible (i.e. having the fewest contacts) to implement the following truth table:

% No blank lines allowed between lines of an \halign structure!
% I use comments (%) instead, so that TeX doesn't choke.

$$\vbox{\offinterlineskip
\halign{\strut
\vrule \quad\hfil # \ \hfil & 
\vrule \quad\hfil # \ \hfil & 
\vrule \quad\hfil # \ \hfil & 
\vrule \quad\hfil # \ \hfil \vrule \cr
\noalign{\hrule}
%
% First row
A & B & C & Output \cr
%
\noalign{\hrule}
%
% Second row
0 & 0 & 0 & 0 \cr
%
\noalign{\hrule}
%
% Third row
0 & 0 & 1 & 0 \cr
%
\noalign{\hrule}
%
% Fourth row
0 & 1 & 0 & 1 \cr
%
\noalign{\hrule}
%
% Fifth row
0 & 1 & 1 & 0 \cr
%
\noalign{\hrule}
%
% Sixth row
1 & 0 & 0 & 0 \cr
%
\noalign{\hrule}
%
% Seventh row
1 & 0 & 1 & 0 \cr
%
\noalign{\hrule}
%
% Eighth row
1 & 1 & 0 & 1 \cr
%
\noalign{\hrule}
%
% Ninth row
1 & 1 & 1 & 1 \cr
%
\noalign{\hrule}
} % End of \halign 
}$$ % End of \vbox

\underbar{file 02827}
%(END_QUESTION)





%(BEGIN_ANSWER)

Simplest relay circuit possible:

$$\epsfbox{02827x01.eps}$$

%(END_ANSWER)





%(BEGIN_NOTES)

Ask your students to show all their work in designing the relay circuit.  By presenting their thought processes, not only do you help them consolidate their learning, but you also help the other students understand better by allowing them to learn from a peer.

%INDEX% Boolean algebra, relay circuit simplification
%INDEX% Sum-of-Products expression, Boolean algebra (from analysis of a truth table)
%INDEX% SOP expression, Boolean algebra (from analysis of a truth table)

%(END_NOTES)


