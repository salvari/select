
%(BEGIN_QUESTION)
% Copyright 2004, Tony R. Kuphaldt, released under the Creative Commons Attribution License (v 1.0)
% This means you may do almost anything with this work of mine, so long as you give me proper credit

If the power waveform is plotted for an AC circuit with a 90 degree phase shift between voltage and current, it will look something like this:

$$\epsfbox{02175x01.eps}$$

What is the significance of the power value oscillating equally between positive (above the zero line) and negative (below the zero line)?  How does this differ from a scenario where there is zero phase shift between voltage and current?

\underbar{file 02175}
%(END_QUESTION)





%(BEGIN_ANSWER)

A symmetrically oscillating power waveform represents energy going back and forth between source and load, never actually dissipating.

%(END_ANSWER)





%(BEGIN_NOTES)

Discuss with your students the energy-storing and energy-releasing ability of capacitors and inductors, and how they differ from resistors.  This is key to understanding the zero net power dissipation of reactive components in AC circuits.

%INDEX% Power, instantaneous, in AC circuit with phase shift between V and I

%(END_NOTES)


