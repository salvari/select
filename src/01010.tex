
%(BEGIN_QUESTION)
% Copyright 2003, Tony R. Kuphaldt, released under the Creative Commons Attribution License (v 1.0)
% This means you may do almost anything with this work of mine, so long as you give me proper credit

This opamp circuit is known as a {\it difference amplifier}, sometimes called a {\it subtractor}.  Assuming that all resistor values are equal in the circuit, write an equation expressing the output ($y$) as a function of the two input voltages ($a$ and $b$):

$$\epsfbox{01010x01.eps}$$

\underbar{file 01010}
%(END_QUESTION)





%(BEGIN_ANSWER)

$y = b - a$

%(END_ANSWER)





%(BEGIN_NOTES)

Work through some example conditions of input voltages and resistor values to calculate the output voltage using Ohm's Law and the general principle of negative feedback in an opamp circuit (namely, an assumption of zero voltage differential at the opamp inputs).  The goal here is to have students comprehend {\it why} this circuit subtracts one voltage from another, rather than just encourage rote memorization.

%INDEX% Difference amplifier, opamp
%INDEX% Subtractor circuit, opamp

%(END_NOTES)


