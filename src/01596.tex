
%(BEGIN_QUESTION)
% Copyright 2003, Tony R. Kuphaldt, released under the Creative Commons Attribution License (v 1.0)
% This means you may do almost anything with this work of mine, so long as you give me proper credit

A very powerful method for discerning cause-and-effect relationships is {\it scientific method}.  One commonly accepted algorithm (series of steps) for scientific method is the following:

\medskip
\item{$1.$} Observation
\item{$2.$} Formulate an hypothesis (an educated guess)
\item{$3.$} Predict a unique consequence of that hypothesis
\item{$4.$} Test the prediction by experiment
\item{$5.$} If test fails, go back to step \#2.  If test passes, hypothesis is provisionally confirmed.
\medskip

This methodology is also very useful in technical troubleshooting, since troubleshooting is fundamentally a determination of cause for an observed effect.  Read the following description of an experienced troubleshooter diagnosing an mechanical noise problem in a bicycle, and match the troubleshooter's steps to those five steps previously described for scientific method:

\vskip 10pt {\narrower \noindent \baselineskip5pt

One day a bicyclist called a mechanic friend of his over the telephone, and describes a problem with his bicycle.  The bicycle is making a rhythmic "clicking" sound as it is pedaled, but the bicyclist is not very mechanically inclined, and cannot determine the cause of the noise. 

\vskip 5pt

The mechanic considered some of the options.  Being a rhythmic noise, it was probably being caused by one of the bicycle's rotating objects.  This includes the wheels, crank, and chain, which all rotate at different speeds.  After a bit of thought, the mechanic asked his bicyclist friend a question.

\vskip 5pt

``Does the pace of the clicking increase as you ride faster?''  The bicyclist answered, ``Yes, it does.''  

\vskip 5pt

``If you shift into a higher gear so that your crank is turning slower for the same road speed, does the pace of the clicking change?'' asked the mechanic.  The bicyclist admitted he didn't know the answer to this question, as he hadn't thought to pay attention to this detail.  After riding the bike once again to test the mechanic's idea, the bicyclist reported back.  ``No, the pace of the clicking does not change when I shift gears.  It only changes with changes in road speed.''

\vskip 5pt

Upon hearing this, the mechanic knew the general location of the problem, and continued his troubleshooting over the telephone with further questions for the bicyclist.

\par} \vskip 10pt

Where is the clicking sound coming from on this bicycle, based on the information presented here?  How do you (and the mechanic) know?

\underbar{file 01596}
%(END_QUESTION)





%(BEGIN_ANSWER)

The clicking noise has something to do with one of the wheels, and not the chain or crank.

%(END_ANSWER)





%(BEGIN_NOTES)

Discuss with your students the relationship between the mechanic's steps and the steps given for scientific method.  Have them locate the observation, hypothesis, prediction, and test.

Once students have successfully identified the mechanic's reasoning, ask them to explain how the prediction of noise rhythm distinguishes which part of the bicycle is making the noise.

Also, discuss whether this concludes the diagnostic procedures, or if there is more troubleshooting left to do.  What steps are recommended to take next, if any?

%INDEX% Troubleshooting strategy, applying scientific method

%(END_NOTES)


