
%(BEGIN_QUESTION)
% Copyright 2003, Tony R. Kuphaldt, released under the Creative Commons Attribution License (v 1.0)
% This means you may do almost anything with this work of mine, so long as you give me proper credit

When connecting components together to build your power supply, it is important that you use the proper type(s) of wire.  Identify what characteristics are required for the wires you use in this project, for each of the following parameters:

\medskip
\item{$\bullet$} Wire gauge:
\item{$\bullet$} Insulation type:
\item{$\bullet$} Stranding (solid or stranded):
\medskip

\underbar{file 01509}
%(END_QUESTION)





%(BEGIN_ANSWER)

\medskip
\item{$\bullet$} Wire gauge: {\it sufficient for the maximum current expected}
\item{$\bullet$} Insulation type: {\it sufficient for the maximum voltage expected}
\item{$\bullet$} Stranding (solid or stranded): {\it stranded preferred}
\medskip

For the first two parameters, identify the values of expected current and voltage to be encountered in your power supply circuit, for both the AC and the DC sections.

Why do you suppose stranded wire might be preferred for a project such as this?  I'll give you a hint: it is not for any {\it electrical} property of the wire as much as it is for {\it mechanical} considerations.

%(END_ANSWER)





%(BEGIN_NOTES)

Ampacity figures may not be readily available for wires of the gauge most electronics students are accustomed to dealing with.  However, ampacity for small-gauge wire may be roughly calculated by plotting data points of published wire ampacities for wire of several larger gauges, and then extrapolating downward on the gauge scale.  If nothing else, this would be a great example to students of how to use simple statistical techniques (regression, curve fitting, plotting) to make practical estimations.

%(END_NOTES)


