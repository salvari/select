
%(BEGIN_QUESTION)
% Copyright 2003, Tony R. Kuphaldt, released under the Creative Commons Attribution License (v 1.0)
% This means you may do almost anything with this work of mine, so long as you give me proper credit

Calculate all voltages and currents in this circuit, at a power supply frequency near resonance:

$$\epsfbox{00608x01.eps}$$

Based on your calculations, what general predictions can you make about series-resonant circuits, in terms of their total impedance, their total current, and their individual component voltage drops?

\underbar{file 00608}
%(END_QUESTION)





%(BEGIN_ANSWER)

In a series LC circuit near resonance, $Z_{total}$ is nearly zero, $I_{total}$ is large, and both $E_L$ and $E_C$ are large as well.

\vskip 10pt

Follow-up question: suppose the capacitor were to fail shorted.  Identify how this failure would alter the circuit's current and voltage drops.

%(END_ANSWER)





%(BEGIN_NOTES)

This question is given without a specified source frequency for a very important reason: to encourage students to "experiment" with numbers and explore concepts on their own.  Sure, I could have given a power supply frequency as well, but I chose not to because I wanted students to set up part of the problem themselves.

In my experience teaching, students will often choose to remain passive with regard to a concept they do not understand, rather than aggressively pursue an understanding of it.  They would rather wait and see if the instructor happens to cover that concept during class time than take initiative to explore it on their own.  Passivity is a recipe for failure in life, and this includes intellectual endeavors as much as anything else.  The fundamental trait of autonomous learning is the habit of pursuing the answer to a question, without being led to do so.  Questions like this, which purposefully omit information, and thus force the student to think creatively and independently, teach them to develop this trait.

%INDEX% Series resonance
%INDEX% Resonance, series

%(END_NOTES)


