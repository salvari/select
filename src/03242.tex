
%(BEGIN_QUESTION)
% Copyright 2005, Tony R. Kuphaldt, released under the Creative Commons Attribution License (v 1.0)
% This means you may do almost anything with this work of mine, so long as you give me proper credit

Use Th\'evenin's Theorem to determine a simple equivalent circuit for the 3 mA source, the 8 k$\Omega$ resistor, and the 3.3 k$\Omega$ resistor; then calculate the voltage across the 1 k$\Omega$ load:

$$\epsfbox{03242x01.eps}$$

\underbar{file 03242}
%(END_QUESTION)





%(BEGIN_ANSWER)

$V_{load}$ = 2.756 volts

\vskip 10pt

Follow-up question: although analyzing this circuit by series-parallel analysis is probably easier than using Th\'evenin's Theorem, there is definite value to doing it the way this question instructs when considering many different load resistance possibilities.  Explain why this is.

%(END_ANSWER)





%(BEGIN_NOTES)

Ask your students to show how (step-by-step) they arrived at the equivalent circuit, prior to calculating load voltage.

In case students are unfamiliar with the "double-chevron" symbols in the schematic diagram, let them know that these represent male/female connector pairs.

%INDEX% Thevenin's Theorem, applied to resistive network

%(END_NOTES)


