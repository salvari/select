
%(BEGIN_QUESTION)
% Copyright 2003, Tony R. Kuphaldt, released under the Creative Commons Attribution License (v 1.0)
% This means you may do almost anything with this work of mine, so long as you give me proper credit

Suppose you wished to measure the current of an AC motor, the full-load current of which should be about 150 amps.  This current is much too great to measure directly with your only AC ammeter (rated 0 to 5 amps), and the only current transformers you have available are rated at 1200:5, which would not produce enough output current to drive the 0-5 amp meter movement very far at a load current of 150 amps.  Sure, you would get a measurement, but it wouldn't be very accurate.

Figure out a way to overcome this measurement problem, so that a motor current of 150 amps will produce a more substantial deflection on the 0-5 amp meter movement scale.

\underbar{file 00759}
%(END_QUESTION)





%(BEGIN_ANSWER)

Perhaps the easiest solution is to wrap the power conductor so it goes through the CT toroid several times, thus changing its effective ratio.

%(END_ANSWER)





%(BEGIN_NOTES)

The "multiple wrap" solution is a neat trick I've used more than once to measure current with an oversized current transformer.  Discuss the effect of multiple "turns" of primary conductor on the ratio of a CT with your students, calculating the new ratios formed by doing so.

Although the "multiple wrap" solution is simple, it is not the only possible solution to this problem.  Another solution would be to use multiple current transformers, but I'll leave that up to you and your students to figure out!

%(END_NOTES)


