
%(BEGIN_QUESTION)
% Copyright 2003, Tony R. Kuphaldt, released under the Creative Commons Attribution License (v 1.0)
% This means you may do almost anything with this work of mine, so long as you give me proper credit

If the numbers sixteen and nine are added in binary form, will the answer be any different than if the same quantities are added in decimal form?  Explain.

\underbar{file 01229}
%(END_QUESTION)





%(BEGIN_ANSWER)

No.  The form of numeration used to represent numbers has no bearing on the outcome of mathematical operations.

%(END_ANSWER)





%(BEGIN_NOTES)

Although this may seem like a trivial question, I've met electronics technicians who actually believed that the form of numeration affected the outcome of certain mathematical operations.  In particular, I met one fellow who believed the number $\pi$ was fundamentally different in binary form than it was in decimal form: that a binary "pi" was not the same quantity as a decimal "pi".  I challenged his belief by applying some Socratic irony:

\vskip 10pt

{\narrower 

\noindent
{\bf Me:} How do you use a hand calculator to determine the circumference of a circle, given its diameter?  For example, a circle with a diameter of 5 feet has a circumference of . . .

\vskip 10pt

\noindent
{\bf Him:} By multiplying the diameter times "pi".  5 feet times "pi" is a little over 15 feet.

\vskip 10pt

\noindent
{\bf Me:} Does a calculator give you the correct answer?

\vskip 10pt

\noindent
{\bf Him:} Of course it does.

\vskip 10pt

\noindent
{\bf Me:} Does an electronic calculator use decimal numbers, internally, to do math?

\vskip 10pt

\noindent
{\bf Him:} No, it uses binary numbers, because its circuitry is made up of logic gates . . . {\it (long pause)} . . . Oh, now I see!  If the type of number system mattered in doing math, digital computers and calculators would arrive at different answers for arithmetic problems than we would doing the math by hand!

\par} 

\vskip 10pt

Of course, those familiar with computer programming and numerical analysis understand that digital computers can introduce "artifacts" into computed results that are not mathematically correct.  However, this is not due to their use of binary numeration so much as it is limited word-widths (leading to overflow conditions), algorithmic problems converting floating-point to integer and visa-versa, and such.

%INDEX% Numerical quantities, represented in binary and decimal alike

%(END_NOTES)


