
%(BEGIN_QUESTION)
% Copyright 2005, Tony R. Kuphaldt, released under the Creative Commons Attribution License (v 1.0)
% This means you may do almost anything with this work of mine, so long as you give me proper credit

Use simultaneous equations to calculate the values of $R_1$ and $R_2$ necessary to give this voltage divider the range of adjustment specified:

$$\epsfbox{03110x01.eps}$$

$V_{out}$ (minimum) = 3 volts \hskip 30pt $V_{out}$ (maximum) = 8 volts

\vskip 20pt

$R_1$ = \hskip 80pt $R_2$ =

\vskip 10pt

\underbar{file 03110}
%(END_QUESTION)





%(BEGIN_ANSWER)

$R_1$ = 2 k$\Omega$

\vskip 10pt

$R_2$ = 3 k$\Omega$

%(END_ANSWER)





%(BEGIN_NOTES)

Have your students show their methods of solution in class, so you may observe their problem-solving ability and they may see multiple methods of solution.

%INDEX% Simultaneous equations
%INDEX% Systems of nonlinear equations

%(END_NOTES)


