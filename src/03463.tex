
%(BEGIN_QUESTION)
% Copyright 2005, Tony R. Kuphaldt, released under the Creative Commons Attribution License (v 1.0)
% This means you may do almost anything with this work of mine, so long as you give me proper credit

Given James Clerk Maxwell's prediction of electromagnetic {\it waves} arising from the self-sustenance of changing electric and magnetic fields in open space, what sort of a device or collection of devices do you think we would need to create electromagnetic waves oscillating at a frequency within the range attainable by an electric circuit?  In other words, what kind of component(s) would we attach to a source of high-frequency AC to radiate these waves?

\underbar{file 03463}
%(END_QUESTION)





%(BEGIN_ANSWER)

Ideally you will need a device that produces both electric and magnetic fields in space: something that possesses both capacitance and inductance in an unshielded form where the electric and magnetic fields would be open to space.  In other words, you will need an {\it antenna}.

%(END_ANSWER)





%(BEGIN_NOTES)

The purpose of this question is to relate the concept of distributed capacitance and inductance along a plain piece of wire to the very nature of electromagnetic waves (oscillating electric and magnetic fields).  If students suggest using capacitors and inductors, they are quite close to the mark.  Unfortunately, these devices are usually designed to contain their respective fields to {\it prevent} radiation into space.  Here, we {\it want} the fields to radiate away from the device, and so we use an open wire (or an array of open wires).

%INDEX% Antenna, used to radiate electromagnetic waves
%INDEX% Electromagnetic waves, predicted by James Clerk Maxwell

%(END_NOTES)


