
%(BEGIN_QUESTION)
% Copyright 2005, Tony R. Kuphaldt, released under the Creative Commons Attribution License (v 1.0)
% This means you may do almost anything with this work of mine, so long as you give me proper credit

A simple AC-DC power supply circuit outputs about 6.1 volts DC without a filter capacitor connected, and about 9.3 volts DC with a filter capacitor connected:

\vskip 10pt
\epsfbox{02311x01.eps}

Explain why this is.  How can the addition of nothing but a capacitor have such a great effect on the amount of DC voltage output by the circuit?

\underbar{file 02311}
%(END_QUESTION)





%(BEGIN_ANSWER)

The filter capacitor captures the peak voltage level of each pulse from the rectifier circuit, holding that peak level during the time between pulses.  

%(END_ANSWER)





%(BEGIN_NOTES)

Many new students find this phenomenon paradoxical, especially when they see a DC output voltage {\it greater} than the AC (RMS) output voltage of the transformer's secondary winding.  Case in point: being able to build a 30 volt DC power supply using a transformer with a secondary voltage rating of only 24 volts.  The purpose of this question is to get students to face this paradox if they have not recognized and resolved it already.

%INDEX% Power supply circuit, voltage rise created by filter capacitor

%(END_NOTES)


