
%(BEGIN_QUESTION)
% Copyright 2003, Tony R. Kuphaldt, released under the Creative Commons Attribution License (v 1.0)
% This means you may do almost anything with this work of mine, so long as you give me proper credit

Explain the operation of this "H-bridge" motor control circuit:

$$\epsfbox{00449x01.eps}$$

At any given moment, how many transistors are turned on and how many are turned off?  Also, explain what would happen to the function of the circuit if resistor R1 failed open.

\underbar{file 00449}
%(END_QUESTION)





%(BEGIN_ANSWER)

Two transistors are on at any given time, and the other two are off.  If R1 fails open, the motor will not be able to go in the "forward" (Fwd) direction.

\vskip 10pt

Challenge question: what type of DC motor is this drive circuit designed for?  Shunt-wound, series-wound, compound, or permanent magnet?  Explain your answer.

%(END_ANSWER)





%(BEGIN_NOTES)

The "H-drive" circuit is a very common method of reversing polarity to a DC motor (or other polarity-sensitive load), using only a single-pole switch.  Very, very large electric motor "drives" have been based on this same design.

%INDEX% H-bridge motor control circuit
%INDEX% Motor control circuit, "H-bridge"

%(END_NOTES)


