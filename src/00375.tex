
%(BEGIN_QUESTION)
% Copyright 2003, Tony R. Kuphaldt, released under the Creative Commons Attribution License (v 1.0)
% This means you may do almost anything with this work of mine, so long as you give me proper credit

Many precision resistors utilize a {\it wire-wound} construction, where the resistance is determined by the type and length of wire wrapped around a spool.  This form of construction allows for high precision of resistance, with low temperature sensitivity if certain metal alloys are used for the wire.

Unfortunately, though, wrapping wire around a spool forms a coil, which will naturally possess a significant amount of inductance.  This is generally undesirable, as we would like to have resistors possessing {\it only resistance}, with no "parasitic" properties.

There is, however, a special way in which a wire coil may be wound so as to have almost no inductance.  This method is called {\it bifilar} winding, and it is common in wire-wound resistor construction.  Describe how bifilar winding works, and why it eliminates parasitic inductance.

\underbar{file 00375}
%(END_QUESTION)





%(BEGIN_ANSWER)

I won't directly describe how a bifilar winding is made, but I'll give you a hint.  Compare the inductance of a straight piece of wire, versus one that is folded in half:

$$\epsfbox{00375x01.eps}$$

Now, how could a non-inductive {\it coil} of wire be made using the same principle?

%(END_ANSWER)





%(BEGIN_NOTES)

This technique is very useful in reducing or eliminating parasitic inductance.  Typically, parasitic inductance is not a problem unless very high rates of current change are involved, such as in high-frequency AC circuits (radio, high-speed digital logic, etc.).  In such applications, knowing how to control stray inductance is very important to proper circuit operation.

%INDEX% Inductance, parasitic
%INDEX% Bifilar winding

%(END_NOTES)


