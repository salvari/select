
%(BEGIN_QUESTION)
% Copyright 2003, Tony R. Kuphaldt, released under the Creative Commons Attribution License (v 1.0)
% This means you may do almost anything with this work of mine, so long as you give me proper credit

The simple switch-and-diode circuit shown here is an example of a digital {\it encoder}.  Explain what this circuit does, as the switch is moved from position to position:

$$\epsfbox{01408x01.eps}$$

\underbar{file 01408}
%(END_QUESTION)





%(BEGIN_ANSWER)

This encoder generates a three-bit binary code corresponding to the switch position (one out of eight positions).

\vskip 10pt

Follow-up question: trace the path of electron flow through the circuit with the switch in position \#3.

\vskip 10pt

Challenge question: are there other codes (besides binary) that could possibly be generated with a circuit of this general design?

%(END_ANSWER)





%(BEGIN_NOTES)

Ask your students to explain how the term "encoder" applies to this simple circuit.  What, exactly, is being encoded, and what form of code is the data being converted to?

%INDEX% Encoder, simple

%(END_NOTES)


