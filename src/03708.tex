
%(BEGIN_QUESTION)
% Copyright 2005, Tony R. Kuphaldt, released under the Creative Commons Attribution License (v 1.0)
% This means you may do almost anything with this work of mine, so long as you give me proper credit

Suppose this power supply circuit was working fine for several years, then one day failed to output any DC voltage at all:

$$\epsfbox{03708x01.eps}$$

When you open the case of this power supply, you immediately notice the strong odor of burnt components.  From this information, determine some likely component faults and explain your reasoning.

\underbar{file 03708}
%(END_QUESTION)





%(BEGIN_ANSWER)

Shorted capacitor, open transformer winding (as a result of overloading), shorted diode(s) resulting in blown fuse.

%(END_ANSWER)





%(BEGIN_NOTES)

Troubleshooting scenarios are always good for stimulating class discussion.  Be sure to spend plenty of time in class with your students developing efficient and logical diagnostic procedures, as this will assist them greatly in their careers.

Remind your students that test instrument readings are not the only viable source of diagnostic data!  Burnt electronic components usually produce a strong and easily-recognized odor, always indicative of overheating.  It is important to keep in mind that often the burnt component is {\it not} the original source of trouble, but may be a casualty of some other component fault.

%INDEX% Troubleshooting, power supply (AC-DC)

%(END_NOTES)


