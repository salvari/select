
%(BEGIN_QUESTION)
% Copyright 2003, Tony R. Kuphaldt, released under the Creative Commons Attribution License (v 1.0)
% This means you may do almost anything with this work of mine, so long as you give me proper credit

Two computer technicians are called to troubleshoot malfunctioning computer systems.  Although the symptoms in each system are very similar, the histories of the two systems are not.  The first computer is a unit that has been in operation for over two years, while the second system is a brand-new prototype, still in the developmental stages.

Without knowing any more details on these two computer systems, what recommendations can you give to the two technicians about to troubleshoot them?  If you were asked to troubleshoot each system, how would you approach the two systems differently?  What ranges of problems might you expect from each system?

\underbar{file 01580}
%(END_QUESTION)





%(BEGIN_ANSWER)

I'll let you determine the answers to this question!  I do not expect that you will provide specific, technical answers, because I have given very little information about the malfunctioning systems.  What I want is for you to think in general terms: how might the scope of possible problems differ between {\it any} two similar systems, one of which is proven while the other is untried?

%(END_ANSWER)





%(BEGIN_NOTES)

As an illustration of this principle, you might want to elaborate on your own experiences as an electronics instructor.  When assisting students with lab projects, what typical problems do you encounter with the circuits they build, and how do these problems typically differ from problems you've seen in real-life electronic equipment?

%INDEX% Troubleshooting, considering history of system

%(END_NOTES)


