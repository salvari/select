
%(BEGIN_QUESTION)
% Copyright 2005, Tony R. Kuphaldt, released under the Creative Commons Attribution License (v 1.0)
% This means you may do almost anything with this work of mine, so long as you give me proper credit

Suppose a step-down transformer fails due to an accidental short-circuit on the secondary (load) side of the circuit:

$$\epsfbox{03587x01.eps}$$

That the transformer actually failed as a result of the short is without any doubt: smoke was seen coming from it, shortly before current in the circuit stopped.  A technician removes the burned-up transformer and does a quick continuity check of both windings to verify that it has failed open.  What she finds is that the primary winding is open but that the secondary winding is still continuous.  Puzzled at this finding, she asks you to explain how the {\it primary} winding could have failed open while the secondary winding is still intact, if indeed the short occurred on the secondary side of the circuit.  What would you say?  How is it possible that a fault on one side of the transformer caused the {\it other} side to be damaged?

\underbar{file 03587}
%(END_QUESTION)





%(BEGIN_ANSWER)

A short-circuit would cause current in {\it both} windings of the transformer to increase.

%(END_ANSWER)





%(BEGIN_NOTES)

It is important for students to realize that a transformer "reflects" load conditions on the secondary side to the primary side, so that the source "feels" the load in all respects.  What happens on the secondary (load) side will indeed be reflected on the primary (source) side.

%INDEX% Troubleshooting, transformer circuit

%(END_NOTES)


