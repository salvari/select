
%(BEGIN_QUESTION)
% Copyright 2004, Tony R. Kuphaldt, released under the Creative Commons Attribution License (v 1.0)
% This means you may do almost anything with this work of mine, so long as you give me proper credit

This is a schematic for a very simple VCO:

$$\epsfbox{02283x01.eps}$$

The oscillator is of the "Colpitts" design.  The key to understanding this circuit's operation is knowing how the {\it varactor} diode responds to different amounts of DC bias voltage.  Explain how this circuit works, especially how the diode exerts control over the oscillation frequency.  Why does the output frequency vary as the control voltage varies?  Does the output frequency increase or decrease as the control voltage input receives a more positive voltage?

\vskip 10pt

Note: "RFC" is an acronym standing for {\it Radio-Frequency Choke}, an iron-core inductor whose purpose it is to block radio frequency current from passing through.

\underbar{file 02283}
%(END_QUESTION)





%(BEGIN_ANSWER)

As voltage across the varactor diode changes, its capacitance changes.

The output frequency {\it increases} as the control voltage becomes more positive.

%(END_ANSWER)





%(BEGIN_NOTES)

This question is a good review of varactor diode function, as well as frequency modulation theory.

%INDEX% VCO, discrete transistor example circuit

%(END_NOTES)


