
%(BEGIN_QUESTION)
% Copyright 2003, Tony R. Kuphaldt, released under the Creative Commons Attribution License (v 1.0)
% This means you may do almost anything with this work of mine, so long as you give me proper credit

A technique commonly used in special-effects lighting is to sequence the on/off blinking of a string of light bulbs, to produce the effect of motion without any moving objects:

$$\epsfbox{00734x01.eps}$$

What would the effect be if this string of lights were arranged in a {\it circle} instead of a line?  Also, explain what would have to change electrically to alter the "speed" of the blinking lights' "motion".

\underbar{file 00734}
%(END_QUESTION)





%(BEGIN_ANSWER)

If arranged in a circle, the lights would appear to rotate.  The speed of this "rotation" depends on the {\it frequency} of the on/off blinking.

\vskip 10pt

Follow-up question: what electrical change(s) would you have to make to reverse the direction of the lights' apparent motion?

\vskip 10pt

Challenge question: what would happen to the apparent motion of the lights if one of the phases (either 1, 2, or 3) were to fail, so that none of the bulbs with that number would ever light up?

%(END_ANSWER)





%(BEGIN_NOTES)

Ask your students to describe what would happen to the blinking lights if the {\it voltage} were increased or decreased.  Would this alter the perceived speed of motion?

Although this question may seem insultingly simple to many, its purpose is to introduce other sequenced-based phenomenon such as polyphase electric motor theory, where the answers to analogous questions are not so obvious.

%INDEX% Phase rotation, conceptual

%(END_NOTES)


