
%(BEGIN_QUESTION)
% Copyright 2003, Tony R. Kuphaldt, released under the Creative Commons Attribution License (v 1.0)
% This means you may do almost anything with this work of mine, so long as you give me proper credit

$$\epsfbox{01968x01.eps}$$

\underbar{file 01968}
\vfil \eject
%(END_QUESTION)





%(BEGIN_ANSWER)

Use circuit simulation software to verify your predicted and measured parameter values.

%(END_ANSWER)





%(BEGIN_NOTES)

Use a dual-voltage, regulated power supply to supply power to the opamp.

I have had good success using the following values:

\medskip
\item{$\bullet$} +V = +12 volts
\item{$\bullet$} -V = -12 volts
\item{$\bullet$} $V_{TP1}$ = Any voltage well between +V and -V
\item{$\bullet$} $R_{pot}$ = 10 k$\Omega$ linear potentiometer
\item{$\bullet$} $U_1$ = TL081 BiFET operational amplifier (or one-half of a TL082)
\medskip

In order to demonstrate latch-up, you must have an op-amp capable of latching up.  Thus, you should avoid op-amps such as the LM741 and LM1458.  I recommend using an op-amp such as the TL082 for this exercise because it not only latches up, but also does not swing its output voltage rail-to-rail.  Students need to see both these common limitations when they first learn how to use op-amps.

In case your students ask, test point {\bf TP1} is for measuring the output of the potentiometer rather than as a place to inject external signals into.  All you need to connect to {\bf TP1} is a voltmeter!

An extension of this exercise is to incorporate troubleshooting questions.  Whether using this exercise as a performance assessment or simply as a concept-building lab, you might want to follow up your students' results by asking them to predict the consequences of certain circuit faults.

%INDEX% Assessment, performance-based (Opamp voltage follower)

%(END_NOTES)


