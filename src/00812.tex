
%(BEGIN_QUESTION)
% Copyright 2003, Tony R. Kuphaldt, released under the Creative Commons Attribution License (v 1.0)
% This means you may do almost anything with this work of mine, so long as you give me proper credit

A {\it shunt-wound} generator has an electromagnet "field" winding providing the stationary magnetic field in which the armature rotates:

$$\epsfbox{00812x01.eps}$$

Like all electromagnets, the magnetic field strength produced is in direct proportion to the amount of current through the wire coil.  But when the generator is sitting still, its output voltage is zero, and therefore there will be no current through the field winding to energize it and produce a magnetic field for the armature to rotate through.  This causes a problem, since the armature will not have any voltage induced in its windings until it is rotating {\it and} it has a stationary magnetic field from the field winding to rotate through.

It seems like we have a catch-22 situation here: the generator cannot output a voltage until its field winding is energized, but its field winding will not be energized until the generator (armature) outputs some voltage.  How can this generator ever begin to output voltage, given this predicament?

\underbar{file 00812}
%(END_QUESTION)





%(BEGIN_ANSWER)

Usually, there is enough {\it residual} magnetism left in the field poles to initiate some generator action when turned.

\vskip 10pt

Challenge question: what we could do if the generator's field poles ever totally lost their residual magnetism?  How could the generator ever be started?

%(END_ANSWER)





%(BEGIN_NOTES)

Back in the days when generators were common in automotive electrical systems, this used to be a fairly common problem.  However, generators could be "flashed" so as to re-establish this residual magnetic field once again.

%INDEX% Generator, self-excited
%INDEX% Self-excited generator

%(END_NOTES)


