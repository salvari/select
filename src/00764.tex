
%(BEGIN_QUESTION)
% Copyright 2003, Tony R. Kuphaldt, released under the Creative Commons Attribution License (v 1.0)
% This means you may do almost anything with this work of mine, so long as you give me proper credit

Most electromechanical meter movements are inherently {\it average-responding}.  They display their indications in units of volts or amps "RMS" only because they have been calibrated to do so for sinusoidal waveforms.

Some electromechanical meter movements, though, are true-RMS responding.  For example, electrodynamometer movements, when connected as either voltmeters or ammeters (not as wattmeters), naturally provide indications proportional to the voltage's or current's true RMS value.

Based on the inherent differences between these meter movements, describe how you could use electromechanical meter movements to perform qualitative assessments of waveform distortion.  In other words, how could you use electromechanical meters to tell whether an AC waveform was sinusoidal or not?

\underbar{file 00764}
%(END_QUESTION)





%(BEGIN_ANSWER)

Take an "average-responding" and a "true-RMS" meter that indicate equally when measuring sinusoidal waveforms, and compare their readings when measuring the AC waveform in question.  The greater the difference between the two meter readings, the greater the distortion (from a sine-wave ideal).

%(END_ANSWER)





%(BEGIN_NOTES)

Given the prevalence of harmonics in modern AC power systems, this "trick" can be quite useful in making qualitative assessments of harmonic distortion.  Of course, expensive test equipment will give {\it quantitative} measurements of distortion, but it's always nice to know how to use lesser test equipment just in case the expensive equipment is not available.

%(END_NOTES)


