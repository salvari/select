
%(BEGIN_QUESTION)
% Copyright 2006, Tony R. Kuphaldt, released under the Creative Commons Attribution License (v 1.0)
% This means you may do almost anything with this work of mine, so long as you give me proper credit

Electrical engineers usually express the frequency of an AC circuit in terms of {\it angular velocity}, measured in units of {\it radians per second} rather than cycles per second (Hertz, or Hz).

First, explain what a {\it radian} is.  Next, write an equation relating frequency ($f$) in Hertz to angular velocity ($\omega$) in radians per second.  Hint: the relationship between the two is perhaps most easily understood in terms of a two-pole AC generator, or alternator, where each revolution of the rotor generates one full cycle of AC.

\underbar{file 04060}
%(END_QUESTION)





%(BEGIN_ANSWER)

A {\it radian} is that angle describing a sector of a circle, whose arc length is equal to the radius of the circle:

$$\epsfbox{04060x01.eps}$$

\vskip 10pt

Next, the equivalence between angular velocity ($\omega$) and frequency ($f$):

$$\omega = 2 \pi f$$

%(END_ANSWER)





%(BEGIN_NOTES)

Personally, I find the rotating alternator model the best way to comprehend the relationship between angular velocity and frequency.  If each turn of the rotor is one cycle ($2 \pi$ radians), and frequency is cycles per second, then one revolution per second will be 1 Hertz, which will be $2 \pi$ radians per second.

%INDEX% Angular velocity, (lower-case omega)

%(END_NOTES)


