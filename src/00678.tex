
%(BEGIN_QUESTION)
% Copyright 2003, Tony R. Kuphaldt, released under the Creative Commons Attribution License (v 1.0)
% This means you may do almost anything with this work of mine, so long as you give me proper credit

In electric circuits, the three basic quantities are {\it voltage} ($E$ or $V$), {\it current} ($I$), and {\it resistance} ($R$), corresponding to the general concepts of {\it cause}, {\it effect}, and {\it opposition}, respectively.

$$I = {E \over R}$$

$$\hbox{effect} = {\hbox{cause} \over \hbox{opposition}}$$

$$\epsfbox{00678x02.eps}$$

Magnetic "circuits" also have quantities corresponding to "cause," "effect," and "opposition."  Identify these quantities, along with their respective symbols, and write an "Ohm's Law" equation relating them mathematically.  Also, identify the units of measurement associated with each, in three systems of measurement: CGS ("old" metric), SI ("new" metric), and English.

$$\epsfbox{00678x03.eps}$$

\underbar{file 00678}
%(END_QUESTION)





%(BEGIN_ANSWER)

"Cause" = Magnetomotive force (MMF) = ${\cal F}$

"Effect" = Magnetic flux = $\Phi$

"Opposition" = Reluctance = $\Re$

\vskip 10pt

This relationship is known as {\it Rowland's Law}, and it bears striking similarity to Ohm's Law in electric circuits:

$$\Phi = {{\cal F} \over \Re}$$

$$\epsfbox{00678x01.eps}$$

\vskip 10pt

Follow-up question: algebraically manipulate the Rowland's Law equation shown above to solve for $\cal F$ and to solve for $\cal \Re$.

%(END_ANSWER)





%(BEGIN_NOTES)

Magnetism, while commonly experienced in the form of permanent magnets and magnetic compasses, is just as "strange" a concept as electricity for the new student.  At this point in their education, however, they should be familiar enough with voltage, current, and resistance to reflect upon them as analogous quantities to these new magnetic quantities of MMF, flux, and reluctance.  Emphasize the analogical similarities of basic electrical quantities in your discussion with students.  Not only will this help students understand magnetism better, but it will also reinforce their comprehension of electrical quantities.

%INDEX% Magnetism, "Ohm's Law" for
%INDEX% Rowland's Law (magnetism)

%(END_NOTES)


