
%(BEGIN_QUESTION)
% Copyright 2005, Tony R. Kuphaldt, released under the Creative Commons Attribution License (v 1.0)
% This means you may do almost anything with this work of mine, so long as you give me proper credit

$$\epsfbox{01952x01.eps}$$

\underbar{file 01952}
\vfil \eject
%(END_QUESTION)





%(BEGIN_ANSWER)

Use circuit simulation software to verify your predicted and measured parameter values.

%(END_ANSWER)





%(BEGIN_NOTES)

I have had great success with the following values:

\medskip
\item{$\bullet$} $V_{CC}$ = 7 to 24 volts
\item{$\bullet$} $C_1$ and $C_2$ = 0.22 $\mu$F
\item{$\bullet$} $C_3$ = 0.47 $\mu$F
\item{$\bullet$} $L_1$ = 100 $\mu$H (ferrite core RF choke)
\item{$\bullet$} $R_1$ = 22 k$\Omega$
\item{$\bullet$} $R_2$ = 1.5 M$\Omega$
\item{$\bullet$} $Q_1$ = part number 2N3403
\medskip

With these component values, the output waveform was quite clean and the frequency was very close to predicted:

$$f_{out} = {1 \over {2 \pi \sqrt{L C_1 C_2 \over C_1 + C_2}}}$$

You might want to quiz your students on the purpose of resistor $R_2$, since it usually only has to be present at power-up to initiate oscillation!

An extension of this exercise is to incorporate troubleshooting questions.  Whether using this exercise as a performance assessment or simply as a concept-building lab, you might want to follow up your students' results by asking them to predict the consequences of certain circuit faults.

%INDEX% Assessment, performance-based (Colpitts oscillator, BJT)

%(END_NOTES)


