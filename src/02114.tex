
%(BEGIN_QUESTION)
% Copyright 2004, Tony R. Kuphaldt, released under the Creative Commons Attribution License (v 1.0)
% This means you may do almost anything with this work of mine, so long as you give me proper credit

A technician needs to know the value of a capacitor, but does not have a capacitance meter nearby.  In lieu of this, the technician sets up the following circuit to measure capacitance:

$$\epsfbox{02114x01.eps}$$

You happen to walk by this technician's workbench and ask, "How does this measurement setup work?"  The technician responds, "You connect a resistor of known value ($R$) in series with the capacitor of unknown value ($C_x$), then adjust the generator frequency until the oscilloscope shows the two voltage drops to be equal, and then you calculate $C_x$."

Explain how this system works, in your own words.  Also, write the formula you would use to calculate the value of $C_x$ given $f$ and $R$.

\underbar{file 02114}
%(END_QUESTION)





%(BEGIN_ANSWER)

I'll let you figure out how to explain the operation of this test setup.  The formula you would use looks like this:

$$C_x = {1 \over {2 \pi f R}}$$

\vskip 10pt

Follow-up question: could you use a similar setup to measure the inductance of an unknown inductor $L_x$?  Why or why not?  

\vskip 10pt

Challenge question: astute observers will note that this setup might not work in real life because the ground connection of the oscilloscope is {\it not} common with one of the function generator's leads.  Explain why this might be a problem, and suggest a practical solution for it.

%(END_ANSWER)





%(BEGIN_NOTES)

This method of measuring capacitance (or inductance for that matter) is fairly old, and works well if the unknown component has a high $Q$ value.

%INDEX% Algebra, substitution
%INDEX% Capacitance, inferred by measurements of AC voltage in a series RC circuit
%INDEX% Measuring capacitance by AC voltage drops in a series RC circuit

%(END_NOTES)


