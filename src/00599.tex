
%(BEGIN_QUESTION)
% Copyright 2003, Tony R. Kuphaldt, released under the Creative Commons Attribution License (v 1.0)
% This means you may do almost anything with this work of mine, so long as you give me proper credit

Explain what {\it surface-mount devices} are, and how the soldering and desoldering processes for them differ from the processes used to solder and de-solder "through-hole" components on printed circuit boards.  The following illustration contrasts "through-hole" component construction versus surface-mount construction for a voltage divider circuit (three resistors connected in series):

$$\epsfbox{00599x01.eps}$$

\underbar{file 00599}
%(END_QUESTION)





%(BEGIN_ANSWER)

"Surface-mount" components solder to the surface of a printed circuit board, and have no mechanical means of attachment, unlike "through-hole" components.  Because of their small size and lack of attachment prior to soldering, special soldering equipment must be used. 

%(END_ANSWER)





%(BEGIN_NOTES)

Ask your students to describe some of the special soldering equipment that must be used for surface-mount devices?  How does it differ in terms of power from regular soldering irons and guns?  Why are surface-mount soldering tools rated at a different power level?

Ask your students to describe how surface-mount components may be held to the board during the soldering process.  Are all surface-mount components identical, or are there different types?  What special soldering tool attachments are available for working with surface-mount devices?

%INDEX% Surface-mount device, defined
%INDEX% SMD, defined

%(END_NOTES)


