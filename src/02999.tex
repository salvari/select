
%(BEGIN_QUESTION)
% Copyright 2005, Tony R. Kuphaldt, released under the Creative Commons Attribution License (v 1.0)
% This means you may do almost anything with this work of mine, so long as you give me proper credit

\noindent
NAME: \underbar{\hskip 150pt}

\centerline{\bf Problem Log} \medskip

{\it To help you measure and improve your own learning, it is good to keep track of areas where you have been prone to miscomprehension.  This way, you may recognize your weaknesses, which is the first step in overcoming them.}

\vskip 10pt

\noindent
1.  Identify concepts you had difficulty understanding, citing specific questions (or parts of questions) from this worksheet where these difficulties were apparent.  If you are tempted to write, "Everything!," you need to see your instructor for help in answering this question.

\vskip 100pt

\noindent
2.  Are there any common points of difficulty you see between this worksheet's subjects and subjects from past worksheets?

\vskip 100pt

\noindent
3.  Identify specific steps you will take (or continue to take) to overcome your greatest academic weaknesses.

\vskip 100pt

\underbar{file 02999}
%(END_QUESTION)





%(BEGIN_ANSWER)

When answering question \#1, avoid being too specific or too vague.  What you want to do is focus on the particular concept, not the details of the question or the general subject of the worksheet.  Look for the root cause of your problems.  A good answer to this question is one that leads directly to a workable plan of action in overcoming it, and may usually be found by asking yourself "why" often enough.  Read these examples:

\vskip 10pt {\narrower \noindent \baselineskip5pt

{\bf Too specific} -- {\it I entered the 300 ohms in my calculator instead of 330 ohms as I should have on question 34, and that made me get the wrong answer for total resistance.}

\par} \vskip 10pt



\vskip 10pt {\narrower \noindent \baselineskip5pt

{\bf Too vague} -- {\it Math confuses me.}

\par} \vskip 10pt



\vskip 10pt {\narrower \noindent \baselineskip5pt

{\bf Fair answer} -- {\it I kept entering the wrong numbers in my calculator as I tried to solve for total resistance in the circuit of question 34.}

\par} \vskip 10pt



\vskip 10pt {\narrower \noindent \baselineskip5pt

{\bf Better answer} -- {\it I kept confusing numbers from the last problem (number 33) on my scratch paper as I tried to solve for total resistance on question 34.}

\par} \vskip 10pt



\vskip 10pt {\narrower \noindent \baselineskip5pt

{\bf Best answer} -- {\it Question 34 revealed a common problem of mine: I try to cram too much work on one sheet of paper, and I end up confusing myself by entering numbers from a previous problem when I'm working on solving a different one.}

\par} \vskip 10pt


%(END_ANSWER)





%(BEGIN_NOTES)

Helping students overcome academic problems is very similar to the task counselors face in helping their clients overcome personal problems of all kinds.  Neither counselors nor teachers can fix anyone's problems directly, but they can help the person see what their problems are.  Identifying the nature of a problem is absolutely crucial in overcoming it!

%INDEX% Metacognitive survey

%(END_NOTES)


