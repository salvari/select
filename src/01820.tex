
%(BEGIN_QUESTION)
% Copyright 2003, Tony R. Kuphaldt, released under the Creative Commons Attribution License (v 1.0)
% This means you may do almost anything with this work of mine, so long as you give me proper credit

Most oscilloscopes can only directly measure voltage, not current.  One way to measure AC current with an oscilloscope is to measure the voltage dropped across a {\it shunt resistor}.  Since the voltage dropped across a resistor is proportional to the current through that resistor, whatever wave-shape the current is will be translated into a voltage drop with the exact same wave-shape.

However, one must be very careful when connecting an oscilloscope to any part of a grounded system, as many electric power systems are.  Note what happens here when a technician attempts to connect the oscilloscope across a shunt resistor located on the "hot" side of a grounded 120 VAC motor circuit:

$$\epsfbox{01820x01.eps}$$

Here, the reference lead of the oscilloscope (the small alligator clip, not the sharp-tipped probe) creates a short-circuit in the power system.  Explain why this happens.

\underbar{file 01820}
%(END_QUESTION)





%(BEGIN_ANSWER)

The "ground" clip on an oscilloscope probe is electrically common with the metal chassis of the oscilloscope, which in turn is connected to earth ground by the three-prong (grounded) power plug.

%(END_ANSWER)





%(BEGIN_NOTES)

This is a very important lesson for students to learn about line-powered oscilloscopes.  If necessary, discuss the wiring of the power system, drawing a schematic showing the complete short-circuit fault current path, from AC voltage source to "hot" lead to ground clip to chassis to ground prong to ground wire to neutral wire to AC voltage source.

%INDEX% Oscilloscope, earth-grounded probe reference lead

%(END_NOTES)


