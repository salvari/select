
%(BEGIN_QUESTION)
% Copyright 2003, Tony R. Kuphaldt, released under the Creative Commons Attribution License (v 1.0)
% This means you may do almost anything with this work of mine, so long as you give me proper credit

In this variable-voltage transformer circuit, the input voltage (120 VAC) is switched to different "taps" on the transformer's primary winding to create different step-down ratios.

$$\epsfbox{00050x01.eps}$$

While it is possible to "tap" the secondary winding of the transformer to achieve different output voltages instead of the primary, there is a good reason for locating the switch in the primary side of the circuit.  Identify this practical reason.
 
\underbar{file 00050}
%(END_QUESTION)





%(BEGIN_ANSWER)

To minimize the amount of current the switch contacts have to handle.

%(END_ANSWER)





%(BEGIN_NOTES)

It is important to always keep in mind practical limitations of components such as switch contacts when designing circuits.  Sure, there may be many alternative ways of building a working circuit, but some ways will be more practical than others.

In some cases, it might be better to locate the switch (and winding taps) on the secondary side of a step-down transformer rather than the primary.  Imagine if the primary winding voltage was 100 kVAC instead of 120 VAC.  Pose this scenario to your students and ask them what practical switch limitations might force re-location to the secondary winding of the transformer.

%INDEX% Transformer taps
%INDEX% Transformer ratio
%INDEX% Winding taps, transformer
%INDEX% Taps, transformer winding

%(END_NOTES)


