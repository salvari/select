
%(BEGIN_QUESTION)
% Copyright 2006, Tony R. Kuphaldt, released under the Creative Commons Attribution License (v 1.0)
% This means you may do almost anything with this work of mine, so long as you give me proper credit

Examine these checkplot images from a PCB drafting program, for a control board based on this inverter circuit design.  Both the top and bottom copper layer plots are shown from the perspective of the board's top side.  The six large "pads" around the periphery of the board are actually holes for mounting screws:

\vskip 10pt

$$\hbox{\bf Assembly drawing}$$

$$\epsfbox{03898x03.eps}$$

\vfil \eject

$$\hbox{\bf Top copper layer}$$

$$\epsfbox{03898x01.eps}$$

\vfil \eject

$$\hbox{\bf Bottom copper layer}$$

$$\epsfbox{03898x02.eps}$$

%\noindent
%\epsfbox{03898x01.eps} \hskip 60pt  \epsfbox{03898x02.eps} \hskip 30pt \epsfbox{03898x03.eps}

\vskip 10pt

Mark where discrete components (resistors, capacitors, and diodes) go into the PCB, and identify which integrated circuits on the board layout are performing which functions in the schematic.  Note: the square pad on each IC marks pin number 1.

\underbar{file 03898}
%(END_QUESTION)





%(BEGIN_ANSWER)

I'll let you do the work on this one.  Discuss your answers with your classmates!

%(END_ANSWER)





%(BEGIN_NOTES)

This is an exercise in datasheet research and layout tracing.  One potentially confusing aspect of the PCB shown is that I have a diode placed in the circuit for "idiot-proofing" in the event of reverse power supply connections.

In case you were wondering, the reason IC part numbers appear on the "top copper" checkplot even though silkscreening is not shown is because I actually wrote the part numbers as part of the copper layer.  This was done purely for purposes of economy: my PCB supplier offers a "bare bones" deal with no silkscreening capability, and I still wanted to have IC labels on my boards.

%(END_NOTES)


