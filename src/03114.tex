
%(BEGIN_QUESTION)
% Copyright 2004, Tony R. Kuphaldt, released under the Creative Commons Attribution License (v 1.0)
% This means you may do almost anything with this work of mine, so long as you give me proper credit

The {\it Pythagorean Theorem} is used to calculate the length of the hypotenuse of a right triangle given the lengths of the other two sides:

$$\epsfbox{03114x01.eps}$$

Manipulate the standard form of the Pythagorean Theorem to produce a version that solves for the length of $A$ given $B$ and $C$, and also write a version of the equation that solves for the length of $B$ given $A$ and $C$.

\underbar{file 03114}
%(END_QUESTION)





%(BEGIN_ANSWER)

Standard form of the Pythagorean Theorem:

$$C = \sqrt{A^2 + B^2}$$

\vskip 10pt

Solving for $A$:

$$A = \sqrt{C^2 - B^2}$$

\vskip 10pt

Solving for $B$:

$$B = \sqrt{C^2 - A^2}$$

%(END_ANSWER)





%(BEGIN_NOTES)

The Pythagorean Theorem is easy enough for students to find on their own that you should not need to show them.  A memorable illustration of this theorem are the side lengths of a so-called {\it 3-4-5} triangle.  Don't be surprised if this is the example many students choose to give.

%INDEX% Pythagorean Theorem

%(END_NOTES)


