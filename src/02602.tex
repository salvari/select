
%(BEGIN_QUESTION)
% Copyright 2005, Tony R. Kuphaldt, released under the Creative Commons Attribution License (v 1.0)
% This means you may do almost anything with this work of mine, so long as you give me proper credit

An important AC performance parameter for operational amplifiers is {\it slew rate}.  Explain what "slew rate" is, and what causes it to be less than optimal for an opamp.

\underbar{file 02602}
%(END_QUESTION)





%(BEGIN_ANSWER)

{\it Slew rate} is the maximum rate of change of output voltage over time (${dv \over dt}\big|_{max}$) that an opamp can muster.

\vskip 10pt

Follow-up question: what would the output waveform of an opamp look like if it were trying to amplify a square wave signal with a frequency and amplitude exceeding the amplifier's slew rate?

%(END_ANSWER)





%(BEGIN_NOTES)

The follow-up question is very important, as it asks students to apply the concept of a maximum ${dv \over dt}$ to actual waveshapes.  This is often discussed by introductory textbooks, though, so it should not be difficult for students to find good information to help them formulate an answer.

%INDEX% Slew rate, cause of (opamp)

%(END_NOTES)


