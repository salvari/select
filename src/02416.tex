
%(BEGIN_QUESTION)
% Copyright 2005, Tony R. Kuphaldt, released under the Creative Commons Attribution License (v 1.0)
% This means you may do almost anything with this work of mine, so long as you give me proper credit

E-type MOSFETs are {\it normally-off} devices just like bipolar junction transistors, the natural state of their channels strongly resisting the passage of electric currents.  Thus, a state of conduction will only occur on command from an external source.

Explain what must be done to an E-type MOSFET, specifically, to drive it into a state of conduction (where a channel forms to conduct current between source and drain).

\underbar{file 02416}
%(END_QUESTION)





%(BEGIN_ANSWER)

A voltage must be applied between gate and substrate (or gate and source if the substrate is connected to the source terminal) in such a way that the polarity of the gate terminal electrostatically attracts the channel's majority charge carriers (forming an inversion layer directly underneath the insulating layer separating gate from channel).

%(END_ANSWER)





%(BEGIN_NOTES)

This is perhaps the most important question your students could learn to answer when first studying E-type MOSFETs.  What, exactly, is necessary to turn one on?  Have your students draw diagrams to illustrate their answers as they present in front of the class.

Ask them specifically to identify what polarity of $V_{GS}$ would have to be applied to turn on an N-channel E-type MOSFET, and also a P-channel E-type MOSFET.

%INDEX% MOSFET, conditions necessary for conduction (E-type)
%INDEX% Gate voltage, for E-type MOSFET during conduction

%(END_NOTES)


