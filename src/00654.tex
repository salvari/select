
%(BEGIN_QUESTION)
% Copyright 2003, Tony R. Kuphaldt, released under the Creative Commons Attribution License (v 1.0)
% This means you may do almost anything with this work of mine, so long as you give me proper credit

Radio communication functions on the general principle of high-frequency AC power being {\it modulated} by low-frequency data.  Two common forms of modulation are {\it Amplitude Modulation} (AM) and {\it Frequency Modulation} (FM).  In both cases, the modulation of a high frequency waveform by a lower-frequency waveform produces something called {\it sidebands}.

Describe what "sidebands" are, to the best of your ability.

\underbar{file 00654}
%(END_QUESTION)





%(BEGIN_ANSWER)

"Sidebands" are sinusoidal frequencies just above and just below the carrier frequency, produced as a result of the modulation process.  On a spectrum analyzer, they show up as peaks to either side of the main (carrier) peak.  Their quantity, frequencies, and amplitudes are all a function of the data signals modulating the carrier.

%(END_ANSWER)





%(BEGIN_NOTES)

Be sure to ask your students what "AM" and "FM" mean, before they present their answers on sidebands.

The answer makes frequent use of the word {\it carrier} without defining it.  This is another intentional "omission" designed to make students do their research.  If they have taken the time to find information on sidebands, they will surely discover what the word "carrier" means.  Ask them to define this word, in addition to their description of sidebands.

%INDEX% Sideband, defined

%(END_NOTES)


