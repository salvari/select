
%(BEGIN_QUESTION)
% Copyright 2005, Tony R. Kuphaldt, released under the Creative Commons Attribution License (v 1.0)
% This means you may do almost anything with this work of mine, so long as you give me proper credit

Write equations expressing each of the four "total" variables ($V_{total}$, $I_{total}$, $R_{total}$, and $P_{total}$) in a {\bf simple parallel circuit} with one voltage source and three load components.  Note that the fourth equation is already given to you as an example:

\vskip 10pt

\medskip
\item{$\bullet$} $V_{total}$ =
\vskip 5pt
\item{$\bullet$} $I_{total}$ =
\vskip 5pt
\item{$\bullet$} $R_{total}$ =
\vskip 5pt
\item{$\bullet$} $P_{total} = P_1 + P_2 + P_3$
\medskip

\vskip 10pt

For a circuit variable that is equal for all components, write the equation in this form: 

$x_{total} = x_1 = x_2 = x_3$

\underbar{file 03424}
%(END_QUESTION)





%(BEGIN_ANSWER)

\medskip
\item{$\bullet$} $V_{total} = V_1 = V_2 = V_3$
\vskip 5pt
\item{$\bullet$} $I_{total} = I_1 + I_2 + I_3$
\vskip 5pt
\item{$\bullet$} $R_{total} = {1 \over {{1 \over R_1} + {1 \over R_2} + {1 \over R_3}}}$
\vskip 5pt
\item{$\bullet$} $P_{total} = P_1 + P_2 + P_3$
\medskip

%(END_ANSWER)





%(BEGIN_NOTES)

{\bf This question is intended for exams only and not worksheets!}.

%(END_NOTES)


