
%(BEGIN_QUESTION)
% Copyright 2003, Tony R. Kuphaldt, released under the Creative Commons Attribution License (v 1.0)
% This means you may do almost anything with this work of mine, so long as you give me proper credit

Suppose you have found a sensitive meter movement to be the heart of your voltmeter project, but you don't know much about its electrical characteristics.  How would you experimentally determine the two most important characteristics of the meter movement: its {\it full-scale current rating} and its {\it coil resistance}?

\underbar{file 01500}
%(END_QUESTION)





%(BEGIN_ANSWER)

A precise measurement of coil resistance may be obtained using a digital multimeter (ohmmeter) connected to the meter movement.  Determining the meter movement's full-scale current rating, especially without danger of damaging the meter from accidental overcurrent, entails that you design a test circuit specifically for this purpose.

%(END_ANSWER)





%(BEGIN_NOTES)

As the instructor, you may be called on for frequent assistance by students as they prepare to test their meter movements for full-scale current.  You may be called upon by less thoughtful students as they ask you to explain why their galvanometer won't move anymore after being directly connected to a 6 volt battery.

The exercise of experimentally determining a meter's full-scale current rating is a great way to prepare students for the creative challenges of the workplace.  Technicians are frequently asked to invent test jigs and other custom circuitry to measure quantities for which there are no "off-the-shelf" meters to measure, or to perform specialized tasks for which there are no ready-made devices to do it.  This is where technicians really prove their worth to a business enterprise, and where competent technicians distinguish themselves from the incompetent.

%(END_NOTES)


