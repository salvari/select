
%(BEGIN_QUESTION)
% Copyright 2004, Tony R. Kuphaldt, released under the Creative Commons Attribution License (v 1.0)
% This means you may do almost anything with this work of mine, so long as you give me proper credit

The three different types of power in AC circuits are as follows:

\medskip
\goodbreak
\item{$\bullet$} S = apparent power, measured in Volt-Amps (VA)
\item{$\bullet$} P = true power, measured in Watts (W)
\item{$\bullet$} Q = reactive power, measured in Volt-Amps reactive (VAR)
\medskip

Explain the names of each of these power types.  Why are they called "apparent," "true," and "reactive"?

\underbar{file 02178}
%(END_QUESTION)





%(BEGIN_ANSWER)

"Apparent" power is {\it apparently} the total circuit power when volts and amps are multiplied together.  "Reactive" power is due to reactive components ($L$ and $C$) only, and "True" power is the only type that actually accounts for energy leaving the circuit through a load component.

%(END_ANSWER)





%(BEGIN_NOTES)

These definitions may be found in any number of textbooks, but that does not mean they are easy to understand.  Be sure to discuss these very important concepts with your students, given their tendency to generate confusion!

%INDEX% Apparent power, defined
%INDEX% Reactive power, defined
%INDEX% True power, defined

%(END_NOTES)


