
%(BEGIN_QUESTION)
% Copyright 2003, Tony R. Kuphaldt, released under the Creative Commons Attribution License (v 1.0)
% This means you may do almost anything with this work of mine, so long as you give me proper credit

Suppose you tried to measure the voltage at test point 2 (TP2) with a digital voltmeter having an input resistance of 10 M$\Omega$.  How much voltage would it indicate?  How much voltage {\it should} it ideally indicate?

$$\epsfbox{01795x01.eps}$$

\underbar{file 01795}
%(END_QUESTION)





%(BEGIN_ANSWER)

Ideally, of course, this voltage divider circuit should exhibit 7.5 volts at test point 2.  The voltmeter, however, will register only 6.76 volts.

\vskip 10pt

Follow-up question: is the voltmeter registering inaccurately, or is its connection to the circuit actually changing $V_{TP2}$?  In other words, what is the actual voltage at TP2 with the voltmeter connected as shown?

%(END_ANSWER)





%(BEGIN_NOTES)

An analogy I often use to explain meter loading is the use of a pressure gauge to measure the air pressure in a pneumatic tire.  In order to measure the pressure, some of the air must be let out of the tire, which of course changes the tire's air pressure.

And in case you are wondering: no, this is {\it not} an example of Heisenberg's Uncertainty Principle, popularly misunderstood as error introduced by measurement.  The Uncertainty Principle is far more profound than this!

%INDEX% Voltmeter loading

%(END_NOTES)


