
%(BEGIN_QUESTION)
% Copyright 2005, Tony R. Kuphaldt, released under the Creative Commons Attribution License (v 1.0)
% This means you may do almost anything with this work of mine, so long as you give me proper credit

Ammeters must be connected in {\it series} with the current to be measured, to ensure that all the current moves through the meter:

$$\epsfbox{03638x01.eps}$$

In order to practically function, an ammeter must have some internal resistance.  It is usually a very small amount, but it does exist.  It should be apparent to you that the presence of this resistance will have some effect on the circuit current, when compared to the amount of current in the circuit without any meter connected:

$$\epsfbox{03638x02.eps}$$

Explain why it is usually safe to ignore the internal resistance of an ammeter, though, when it is in a circuit.  A common term used in electrical engineering to describe this intentional oversight is {\it swamping}.  In this particular circuit an engineer would say, ``The resistance of the light bulb {\it swamps} the internal resistance of the ammeter.''

\underbar{file 03638}
%(END_QUESTION)





%(BEGIN_ANSWER)

When one quantity ``swamps'' another, we mean that its effect is huge compared to the effect of the other, so much so that we may safely ignore it in our calculations and still arrive at a reasonably accurate result.

%(END_ANSWER)





%(BEGIN_NOTES)

I have found that the concept of ``swamping'' is extremely useful when making estimations.  To be able to ignore the values of some components allows one to simplify a great many circuits, enabling easier calculations to be performed. 

%INDEX% Ammeter, internal resistance of
%INDEX% Swamping, ammeter resistance swamped by load resistance

%(END_NOTES)


