
%(BEGIN_QUESTION)
% Copyright 2003, Tony R. Kuphaldt, released under the Creative Commons Attribution License (v 1.0)
% This means you may do almost anything with this work of mine, so long as you give me proper credit

A capacitor rated at 2.2 microfarads is subjected to a sinusoidal AC voltage of 24 volts RMS, at a frequency of 60 hertz.  Write the formula for calculating capacitive reactance ($X_C$), and solve for current through the capacitor.

\underbar{file 00583}
%(END_QUESTION)





%(BEGIN_ANSWER)

$$X_C = {1 \over {2 \pi f C}}$$

The current through this capacitor is 19.91 mA RMS.

%(END_ANSWER)





%(BEGIN_NOTES)

I have consistently found that qualitative (greater than, less than, or equal) analysis is much more difficult for students to perform than quantitative (punch the numbers on a calculator) analysis.  Yet, I have consistently found on the job that people lacking qualitative skills make more "silly" quantitative errors because they cannot validate their calculations by estimation.

In light of this, I always challenge my students to qualitatively analyze formulae when they are first introduced to them.  Ask your students to identify what will happen to one term of an equation if another term were to either increase, or decrease (you choose the direction of change).  Use up and down arrow symbols if necessary to communicate these changes graphically.  Your students will greatly benefit in their conceptual understanding of applied mathematics from this kind of practice!

%(END_NOTES)


