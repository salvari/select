
%(BEGIN_QUESTION)
% Copyright 2003, Tony R. Kuphaldt, released under the Creative Commons Attribution License (v 1.0)
% This means you may do almost anything with this work of mine, so long as you give me proper credit

Calibration laboratories often make reference to {\it Test Uncertainty Ratios} (TUR's), normally holding to a ratio of 4:1 or more when performing calibration work on instruments.  What does this figure mean?

\underbar{file 00689}
%(END_QUESTION)





%(BEGIN_ANSWER)

The "Test Uncertainty Ratio" refers to how much more precise the calibrating instrument is compared to the instrument being calibrated.

%(END_ANSWER)





%(BEGIN_NOTES)

While your students may never have to calculate TUR's, it is still important for them to know what the general principle is.  If any of them experience difficulty understanding the concept, ask them if it makes any sense to use a ruler to check the calibration of a micrometer, or to use a wristwatch to check the long-term stability of a laboratory-grade chronograph.

%INDEX% Test Uncertainty Ratio (TUR), defined
%INDEX% TUR (Test Uncertainty Ratio), defined

%(END_NOTES)


