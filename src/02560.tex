
%(BEGIN_QUESTION)
% Copyright 2005, Tony R. Kuphaldt, released under the Creative Commons Attribution License (v 1.0)
% This means you may do almost anything with this work of mine, so long as you give me proper credit

$$\epsfbox{02560x01.eps}$$

\underbar{file 02560}
\vfil \eject
%(END_QUESTION)





%(BEGIN_ANSWER)

Use circuit simulation software to verify your predicted and measured parameter values.

%(END_ANSWER)





%(BEGIN_NOTES)

Choose both positive input voltage values and negative input voltage values, so that students may predict and measure the output of this circuit under both types of conditions.  The choice of diodes is not critical, as any rectifier diodes will work.  All resistor values need to be equal, and at least as high as the potentiometer value.  I recommend a 10 k$\Omega$ potentiometer and 15 k$\Omega$ resistors.

A good follow-up question to ask is what would be required to change the polarity of this full-wave precision rectifier circuit.

An extension of this exercise is to incorporate troubleshooting questions.  Whether using this exercise as a performance assessment or simply as a concept-building lab, you might want to follow up your students' results by asking them to predict the consequences of certain circuit faults.

%INDEX% Assessment, performance-based (Precision rectifier, full wave)

%(END_NOTES)


