
%(BEGIN_QUESTION)
% Copyright 2003, Tony R. Kuphaldt, released under the Creative Commons Attribution License (v 1.0)
% This means you may do almost anything with this work of mine, so long as you give me proper credit

The following circuit is a three-channel audio {\it mixer} circuit, used to blend and amplify three different audio signals (coming from microphones or other signal sources):

$$\epsfbox{00825x01.eps}$$

Suppose we measured a 9 kHz sinusoidal voltage of 0.5 volts (peak) at point "A" in the diagram, using an oscilloscope.  Determine the voltage at point "B" in the circuit, after this AC signal voltage "passes through" the voltage divider biasing network.  

The voltage at point "B" will be a mix of AC and DC, so be sure to express both quantities!  Ignore any "loading" effects of the transistor's base current on the voltage divider.

\underbar{file 00825}
%(END_QUESTION)





%(BEGIN_ANSWER)

$V_B =$ 1.318 VDC + 0.5 VAC (peak)

%(END_ANSWER)





%(BEGIN_NOTES)

Ask your students what purpose the 47 $\mu$F capacitor serves.  Since its presence does not noticeably attenuate the AC signal at point "A" (the whole 0.5 volts AC getting to point B), why not just replace it with a straight piece of wire?

%INDEX% Bias network, amplifier

%(END_NOTES)


