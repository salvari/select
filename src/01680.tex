
%(BEGIN_QUESTION)
% Copyright 2003, Tony R. Kuphaldt, released under the Creative Commons Attribution License (v 1.0)
% This means you may do almost anything with this work of mine, so long as you give me proper credit

A common term used in semiconductor circuit engineering is {\it small signal analysis}.  What, exactly, is "small signal" analysis, and how does it contrast with {\it large signal analysis}?

\underbar{file 01680}
%(END_QUESTION)





%(BEGIN_ANSWER)

{\it Small signal analysis} is where the signals are presumed to be small enough in magnitude that the active device(s) should respond in a nearly linear manner.  {\it Large signal analysis} is where the signals are presumed to be large enough that component nonlinearities become significant.

\vskip 10pt

Follow-up question: why would engineers bother with two modes of analysis instead of just one (large signal), where the components' true (nonlinear) behavior is taken into account?  Explain this in terms of network theorems and other mathematical "tools" available to engineers for circuit analysis.

%(END_ANSWER)





%(BEGIN_NOTES)

When researching engineering textbooks and other resources, these terms are quite often used without introduction, leaving many beginning students confused.

%INDEX% Small signal analysis, versus large signal analysis
%INDEX% Large signal analysis, versus small signal analysis

%(END_NOTES)


