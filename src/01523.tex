
%(BEGIN_QUESTION)
% Copyright 2003, Tony R. Kuphaldt, released under the Creative Commons Attribution License (v 1.0)
% This means you may do almost anything with this work of mine, so long as you give me proper credit

The following schematic diagram shows a simple {\it common-collector} transistor amplifier circuit:

$$\epsfbox{01523x01.eps}$$

Explain why the AC voltage gain ($A_{V(AC)}$) of such an amplifier is approximately 1, using any or all of these general "rules" of transistor behavior:

\medskip
\item{$\bullet$} $I_E = I_C + I_B$
\item{$\bullet$} $I_E \approx I_C$
\item{$\bullet$} $V_{BE} \approx 0.7$ volts
\item{$\bullet$} $\beta = {I_C \over I_B}$
\medskip

Remember that AC voltage gain is defined as ${\Delta V_{out} \over \Delta V_{in}}$.

\underbar{file 01523}
%(END_QUESTION)





%(BEGIN_ANSWER)

Since $V_{BE}$ is relatively constant, $\Delta V_{in} \approx \Delta V_{out}$.
 
\vskip 10pt

For your discussion response, be prepared to explain why, in mathematical terms, the above statement is true.  You will have to use Kirchhoff's Voltage Law as part of your explanation.

%(END_ANSWER)





%(BEGIN_NOTES)

Although the given answer seems complete, what I'm looking for here is a good analytical understanding of why the voltage gain is approximately 1.  Placing the requirement of using KVL on the students' answers ensures that they will have to explore the concept further than the given answer does.

%INDEX% Voltage gain, common collector amplifier

%(END_NOTES)


