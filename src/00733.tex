
%(BEGIN_QUESTION)
% Copyright 2003, Tony R. Kuphaldt, released under the Creative Commons Attribution License (v 1.0)
% This means you may do almost anything with this work of mine, so long as you give me proper credit

For any given range of current measurement, what design parameter(s) of an electromechanical ammeter influence its input resistance?  In other words, to approach the "ideal" input resistance of an ammeter, for any given range, what component values are optimum?

\underbar{file 00733}
%(END_QUESTION)





%(BEGIN_ANSWER)

To achieve the lowest possible input resistance, without changing the range of the ammeter, you need a meter movement with a minimum full-scale current rating and a minimum amount of coil resistance.

\vskip 10pt

Challenge question: is it possible to improve the performance of an ammeter's meter movement, as per the recommendations given here, by adding resistors to it?  If so, how?

%(END_ANSWER)





%(BEGIN_NOTES)

If your students have already studied voltmeter design, you might want to ask them to compare the (single) design factor influencing sensitivity ("ohms-per-volt") in an electromechanical voltmeter with the two factors listed in the answer to this question.  Why is the meter movement coil resistance not a factor in voltmeter sensitivity, but it is in ammeter sensitivity?  Challenge your students with this question, by having them propose some example voltmeter circuits and ammeter circuits with different coil resistances.  Let them figure out how to set up the problems, rather than you setting up the problems for them!

Some students may suggest that the effective coil resistance of a meter movement may be decreased with the addition of a shunt resistance inside the movement.  If anyone proposes this solution, work through the calculations of an example ammeter circuit on the whiteboard with the class and see what the effect is!

%(END_NOTES)


