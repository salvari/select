
%(BEGIN_QUESTION)
% Copyright 2003, Tony R. Kuphaldt, released under the Creative Commons Attribution License (v 1.0)
% This means you may do almost anything with this work of mine, so long as you give me proper credit

Many years ago, I decided to experiment with electromagnetism by making an electromagnet out of a spool of wire.  I placed a steel bolt through the center of the spool so as to have a core of high permeability, and passed current from a battery through the wire to make a magnetic field.  Not having any "jumper" wires, I held the wire ends of the spool in contact with the 9-volt battery terminals, one in each hand.

The electromagnet worked just fine, and I was able to move some steel paperclips with the magnetic field generated by it.  However, when I broke the circuit by releasing one of the wire ends from the battery terminal it was touching, I received a small electric shock!  Shown here is a schematic diagram of me, in the circuit:

$$\epsfbox{00376x01.eps}$$

At the time, I didn't understand how inductance worked.  I only understood how to make magnetism with electricity, but I didn't realize a coil of wire could generate (high voltage!) electricity from its own magnetic field.  I did know, however, that the 9 volts output by the battery was much too weak to shock me (yes, I touched the battery terminals directly to verify this fact), so {\it something} in the circuit must have generated a voltage greater than 9 volts.

If you had been there to explain what just happened to me, what would you say?

\underbar{file 00376}
%(END_QUESTION)





%(BEGIN_ANSWER)

There are a couple of different ways to explain how an electromagnet coil can generate a much greater voltage than what it is energized from (the battery).  One way is to explain the origin of the high voltage using Faraday's Law of electromagnetic induction ($e = N{d\phi \over dt}$, or $e = L{di \over dt}$).  Another way is to explain how it is the nature of an inductor to oppose any {\it change} in current over time.  I'll leave it to you to figure out the exact words to say!

%(END_ANSWER)





%(BEGIN_NOTES)

One way to help understand how an inductor could produce such large voltages is to consider it as a {\it temporary current source}, which will output as much voltage as necessary in an effort to maintain constant current.  Just as ideal current sources are dangerous to open-circuit, current-carrying inductors are likewise capable of generating tremendous transient voltages.

Although there was no real safety hazard with my experiment, there potentially could have been, provided different circumstances.  Discuss with your students what would have been necessary to create an actual safety hazard.

%INDEX% Inductive "kickback"
%INDEX% Electromagnetic induction
%INDEX% Safety, electrical
%INDEX% Electric shock

%(END_NOTES)


