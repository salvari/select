
%(BEGIN_QUESTION)
% Copyright 2003, Tony R. Kuphaldt, released under the Creative Commons Attribution License (v 1.0)
% This means you may do almost anything with this work of mine, so long as you give me proper credit

What will happen to each resistor's voltage in this circuit if resistor R4 fails shorted?  Provide individual answers for each resistor, please.

$$\epsfbox{01773x01.eps}$$

Also, comment on the practical likelihood of a resistor failing shorted, as opposed to failing open.

\underbar{file 01773}
%(END_QUESTION)





%(BEGIN_ANSWER)

If resistor R4 fails shorted . . .

\medskip
\item{$\bullet$} $V_{R4}$ will decrease to zero
\item{$\bullet$} $V_{R1}$ will increase
\item{$\bullet$} $V_{R2}$ will decrease
\item{$\bullet$} $V_{R3}$ will increase
\medskip

\vskip 10pt

Follow-up question: resistors are actually far less likely to fail shorted as they are to fail {\it open}.  However, this does not mean something else on a circuit board cannot go wrong to make it appear as though a resistor failed shorted!  One example of such a fault is called a {\it solder bridge}.  Explain what this is, any why it could produce the same effect as a resistor failing shorted.

%(END_ANSWER)





%(BEGIN_NOTES)

I have found in teaching that many students loathe qualitative analysis, because they cannot let their calculators do the thinking for them.  However, being able to judge whether a circuit parameter will increase, decrease, or remain the same after a component fault is an {\it essential} skill for proficient troubleshooting.

%INDEX% Resistor, probable failure mode
%INDEX% Series-parallel circuit; effect of shorted fault

%(END_NOTES)


