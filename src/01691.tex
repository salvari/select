
%(BEGIN_QUESTION)
% Copyright 2003, Tony R. Kuphaldt, released under the Creative Commons Attribution License (v 1.0)
% This means you may do almost anything with this work of mine, so long as you give me proper credit

$$\epsfbox{01691x01.eps}$$

\underbar{file 01691}
\vfil \eject
%(END_QUESTION)





%(BEGIN_ANSWER)

Use circuit simulation software to verify your predicted and measured parameter values.

%(END_ANSWER)





%(BEGIN_NOTES)

This is perhaps the most reliable means of measuring inductance without an impedance bridge or an LCR meter.  You may wish to ask your students to explain {\it why} this method of measurement is so good (hint: they must explain why the inductor's intrinsic resistance has no effect on the measurement).

I prefer this particular circuit design for $L$ measurement because series resistance does not skew the resonant point, and because the series capacitor prevents any possible DC current from "biasing" the inductor's core.

If your students own high-quality multimeters capable of measuring audio-frequency AC current and frequency, then the best way to do this is to replace the resistor $R_1$ with their ammeters.  Otherwise, use an oscilloscope to measure (maximum) voltage dropped across $R_1$.

In order to obtain good measurements using this technique, I recommend following these guidelines:

\medskip
\item{$\bullet$} Choose a value of $R_1$ low enough to give a sharp bandwidth, but not so low that the voltage signal dropped across it is "fuzzy" with noise and difficult to accurately discern the period of.
\item{$\bullet$} Choose a value for $C_1$ as low as possible to give sharp bandwidth (thereby maximizing the ${L \over C}$ ratio), without pushing the circuit's resonant frequency too close to the inductor's self-resonant frequency.
\item{$\bullet$} Avoid frequencies above the audio range, lest your students measure the inductor's self-resonant point!
\item{$\bullet$} Use minimal output from the signal generator, to avoid voltage and current levels that will approach core saturation in the inductor.
\medskip

I've had fair results using one of the windings of a small audio output transformer (center-tapped 1000 $\Omega$ primary winding, with 8 $\Omega$ secondary winding) as the inductor, connected in series with either a 0.1 $\mu$F or a 0.47 $\mu$F metal-film capacitor, all in series with a 100 ohm resistor.  For best results, of course, pre-measure the value of the capacitor rather than go by its advertised value.

I have also used a 100 mH inductor (nominal), in series with a 0.033 $\mu$F capacitor and 100 $\Omega$ resistor, with good results.

%INDEX% Assessment, performance-based (Measuring inductance via series LC resonance)

%(END_NOTES)


