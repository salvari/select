
%(BEGIN_QUESTION)
% Copyright 2003, Tony R. Kuphaldt, released under the Creative Commons Attribution License (v 1.0)
% This means you may do almost anything with this work of mine, so long as you give me proper credit

$$\epsfbox{02170x01.eps}$$

\underbar{file 02170}
\vfil \eject
%(END_QUESTION)





%(BEGIN_ANSWER)

Use circuit simulation software to verify your predicted and measured parameter values.

%(END_ANSWER)





%(BEGIN_NOTES)

Here, students must choose the right type of series RC circuit configuration to provide the requested phase shift.  This, of course, also involves choosing proper values for $C_1$ and $R_1$, and being able to successfully measure phase shift with an oscilloscope.

I recommend selecting a phase shift angle ($\Theta$) somewhere between 15$^{o}$ and 75$^{o}$.  Angles too close to 90$^{o}$ will result in small output voltages that are difficult to measure through the noise.

An extension of this exercise is to incorporate troubleshooting questions.  Whether using this exercise as a performance assessment or simply as a concept-building lab, you might want to follow up your students' results by asking them to predict the consequences of certain circuit faults.

%INDEX% Assessment, performance-based (Phase shift circuit)

%(END_NOTES)


