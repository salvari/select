
%(BEGIN_QUESTION)
% Copyright 2003, Tony R. Kuphaldt, released under the Creative Commons Attribution License (v 1.0)
% This means you may do almost anything with this work of mine, so long as you give me proper credit

Suppose you tried to measure the voltage at all three test points with an analog voltmeter having a sensitivity rating of 20 k$\Omega$ per volt, set on the 10 volt scale.  How much voltage would it indicate at each test point?  How much voltage {\it should} it ideally indicate at each test point?

$$\epsfbox{01796x01.eps}$$

% No blank lines allowed between lines of an \halign structure!
% I use comments (%) instead, so that TeX doesn't choke.

$$\vbox{\offinterlineskip
\halign{\strut
\vrule \quad\hfil # \ \hfil & 
\vrule \quad\hfil # \ \hfil & 
\vrule \quad\hfil # \ \hfil \vrule \cr
\noalign{\hrule}
%
% First row
Test point & Ideal voltage & Meter indication \cr
%
\noalign{\hrule}
%
% Second row
TP1 &  &  \cr
%
\noalign{\hrule}
%
% Third row
TP2 &  &  \cr
%
\noalign{\hrule}
%
% Fourth row
TP3 &  &  \cr
%
\noalign{\hrule}
} % End of \halign 
}$$ % End of \vbox

\underbar{file 01796}
%(END_QUESTION)





%(BEGIN_ANSWER)

% No blank lines allowed between lines of an \halign structure!
% I use comments (%) instead, so that TeX doesn't choke.

$$\vbox{\offinterlineskip
\halign{\strut
\vrule \quad\hfil # \ \hfil & 
\vrule \quad\hfil # \ \hfil & 
\vrule \quad\hfil # \ \hfil \vrule \cr
\noalign{\hrule}
%
% First row
Test point & Ideal voltage & Meter indication \cr
%
\noalign{\hrule}
%
% Second row
TP1 & 5 V & 5 V \cr
%
\noalign{\hrule}
%
% Third row
TP2 & 4.138 V & 0.805 V \cr
%
\noalign{\hrule}
%
% Fourth row
TP3 & 1.293 V & 0.197 V \cr
%
\noalign{\hrule}
} % End of \halign 
}$$ % End of \vbox

%(END_ANSWER)





%(BEGIN_NOTES)

An analogy I often use to explain meter loading is the use of a pressure gauge to measure the air pressure in a pneumatic tire.  In order to measure the pressure, some of the air must be let out of the tire, which of course changes the tire's air pressure.

And in case you are wondering: no, this is {\it not} an example of Heisenberg's Uncertainty Principle, popularly misunderstood as error introduced by measurement.  The Uncertainty Principle is far more profound than this!

%INDEX% Voltmeter loading, analog

%(END_NOTES)


