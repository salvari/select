
%(BEGIN_QUESTION)
% Copyright 2005, Tony R. Kuphaldt, released under the Creative Commons Attribution License (v 1.0)
% This means you may do almost anything with this work of mine, so long as you give me proper credit

The simplest types of programmable logic ICs are called PLDs (Programmable Logic Devices), PALs (Programmable Array Logic), PLAs (Programmable Logic Array), and GALs (Generic Array Logic).  While each acronym represents a slightly different internal design architecture, these devices share a common feature of using inverters, AND gates, and OR gates to implement any desired combinational logic function.

Explain how it is possible to generate any arbitrary logic function with just these gate types (inverter, AND, OR), without any others.  What principle or convention of Boolean algebra is used by these devices to do this?

\underbar{file 03049}
%(END_QUESTION)





%(BEGIN_ANSWER)

With a sufficient number of AND, OR, and inverter gates, any SOP or POS expression may be generated.

%(END_ANSWER)





%(BEGIN_NOTES)

This question requires students to review the principles of how SOP and POS expressions relate to truth tables, and in doing so explain how any arbitrary truth table may be fulfilled.

%INDEX% GAL, defined
%INDEX% PAL, defined
%INDEX% PLA, defined
%INDEX% PLD, defined
%INDEX% Programmable logic, use of SOP/POS expressions

%(END_NOTES)


