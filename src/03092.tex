
%(BEGIN_QUESTION)
% Copyright 2005, Tony R. Kuphaldt, released under the Creative Commons Attribution License (v 1.0)
% This means you may do almost anything with this work of mine, so long as you give me proper credit

Plot the solutions to the equation $y + x = 8$ on a graph:

$$\epsfbox{03092x01.eps}$$

On the same graph, plot the solutions to the equation $y - x = 3$.  What is the significance of the point where the two lines cross?

\underbar{file 03092}
%(END_QUESTION)





%(BEGIN_ANSWER)

$$\epsfbox{03092x02.eps}$$

The point of intersection between the two lines represents the one solution set that satisfies {\it both} equations (where $x = 2.5$ and $y = 5.5$).

%(END_ANSWER)





%(BEGIN_NOTES)

The purpose of this question is to gently introduce students to the concept of simultaneous systems of equations, where a set of solutions satisfies more than one equation at a time.  It is important for students to understand the basic concepts of graphing before they try to answer this question, though.

%INDEX% Algebra, graphing simple functions
%INDEX% Graphing simple functions, algebra
%INDEX% Simultaneous equations
%INDEX% Systems of linear equations

%(END_NOTES)


