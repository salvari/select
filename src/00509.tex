
%(BEGIN_QUESTION)
% Copyright 2003, Tony R. Kuphaldt, released under the Creative Commons Attribution License (v 1.0)
% This means you may do almost anything with this work of mine, so long as you give me proper credit

The electrical resistance of a conductor at any temperature may be calculated by the following equation:

$$R_T = R_r + R_r \alpha T - R_r \alpha T_r$$

\noindent
Where,

$R_T =$ Resistance of conductor at temperature $T$

$R_r =$ Resistance of conductor at reference temperature $T_r$

$\alpha =$ Temperature coefficient of resistance at reference temperature $T_r$

\vskip 10pt

Simplify this equation by means of factoring.

\underbar{file 00509}
%(END_QUESTION)





%(BEGIN_ANSWER)

$$R_T = R_r [1 + \alpha(T - T_r)]$$

\vskip 10pt

Follow-up question: when plotted on a graph with temperature ($T$) as the independent variable and resistance ($R_T$) as the dependent variable (i.e. a two-axis graph with $T$ on the horizontal and $R$ on the vertical), is the resulting plot linear?  Why or why not?  How is it possible to tell just by looking at the equation, prior to actually plotting on a graph?

%(END_ANSWER)





%(BEGIN_NOTES)

Just an exercise in algebra here!

%INDEX% Resistance, relation to conductor temperature
%INDEX% Algebra, manipulating equations

%(END_NOTES)


