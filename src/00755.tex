
%(BEGIN_QUESTION)
% Copyright 2003, Tony R. Kuphaldt, released under the Creative Commons Attribution License (v 1.0)
% This means you may do almost anything with this work of mine, so long as you give me proper credit

In power distribution systems, it is very important to be able to measure line voltage.  You cannot control what you cannot measure, and it is important to control power line voltage so as to not exceed the insulators' ratings.

But how do you safely measure the voltage of a 750 kV power line?  Obviously, no voltmeter small enough to be located on a control panel could safely handle 750,000 volts applied to it, as a voltage that high is capable of arcing several feet through the air (not to mention the safety hazards of having wires behind the meter panel connecting straight to the power line!).

In industry, specialized {\it transformers} are used to safely measure the high voltages on power lines.  Describe what is special about these "potential transformers," and how they are implemented to measure dangerous voltages.

\underbar{file 00755}
%(END_QUESTION)





%(BEGIN_ANSWER)

A "potential transformer," or "PT," is a step-down transformer with a very precise winding turns ratio, so that the secondary voltage is a precise and known fraction of the primary voltage.

\vskip 10pt

Follow-up question: in addition to stepping the line voltage down to relatively safe levels, potential transformers also provide one more important safety feature for voltage measurement.  Describe what this extra feature is, and why it is important.  Hint: all transformers except for autotransformers provide this feature!

%(END_ANSWER)





%(BEGIN_NOTES)

Ask students to draw a rough schematic diagram of how a potential transformer would be placed in a complete voltage-measurement circuit, with power lines, panel-mounted voltmeter mechanism, safety fuses, etc.  

%INDEX% Potential transformer, high voltage metering

%(END_NOTES)


