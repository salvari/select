
%(BEGIN_QUESTION)
% Copyright 2003, Tony R. Kuphaldt, released under the Creative Commons Attribution License (v 1.0)
% This means you may do almost anything with this work of mine, so long as you give me proper credit

$$\epsfbox{01677x01.eps}$$

\underbar{file 01677}
\vfil \eject
%(END_QUESTION)





%(BEGIN_ANSWER)

The real-life measurements you take constitute the "final word" on which sources generate the most significant voltages.

%(END_ANSWER)





%(BEGIN_NOTES)

In this performance assessment, different electricity sources are suggested by way of conversion phenomena.  In other words, the instructor will list such things as {\it photovoltaic} and {\it piezoelectric}, and students will have to choose the correct components to demonstrate conversion of energy into electrical form.  Then, students will demonstrate each conversion for the instructor, ranking them in order of the voltage magnitude generated by each demonstration.

The purpose of this exercise is not only for students to obtain a practical understanding of electricity sources, but also to understand the relative magnitudes of each one.  It is important for students to know, for instance, that thermoelectricity is a rather weak effect compared to piezoelectricity.  This will help them understand the relative sensitivity of sensors and other electrical devices in the future.

Possible sources to list for student demonstration are:

\medskip
\item{$\bullet$} Photovoltaic
\item{$\bullet$} Piezoelectric
\item{$\bullet$} Electromagnetic
\item{$\bullet$} Chemical
\item{$\bullet$} Thermoelectric
\medskip

Of course, your selection of sources for student demonstration depends on the parts and equipment available to them.

%INDEX% Assessment, performance-based (Identifying sources of electricity)

%(END_NOTES)


