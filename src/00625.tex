
%(BEGIN_QUESTION)
% Copyright 2003, Tony R. Kuphaldt, released under the Creative Commons Attribution License (v 1.0)
% This means you may do almost anything with this work of mine, so long as you give me proper credit

What happens to the magnetic lines of flux emanating from a magnet, when an unmagnetized piece of iron is placed near it?

$$\epsfbox{00625x01.eps}$$

\underbar{file 00625}
%(END_QUESTION)





%(BEGIN_ANSWER)

$$\epsfbox{00625x02.eps}$$

\vskip 10pt

Challenge question: can you think of any practical applications of this field-distortion effect?

%(END_ANSWER)





%(BEGIN_NOTES)

A previously unmagnetized piece of iron will become magnetized once exposed to the magnetic field of a nearby magnet.  This fact leads to another question, namely, what magnetic pole assignments does the iron piece assume in this condition?  Ask your students to identify which end of the iron becomes "North" and which end of the iron becomes "South".

One application of magnetic field distortion is the remote detection of vehicles and other objects made of ferrous metals.  There is an historical wartime application of this principle, which may prove interesting for discussion purposes!

%INDEX% Magnetic lines of flux
%INDEX% Flux, magnetic lines

%(END_NOTES)


