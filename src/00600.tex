
%(BEGIN_QUESTION)
% Copyright 2003, Tony R. Kuphaldt, released under the Creative Commons Attribution License (v 1.0)
% This means you may do almost anything with this work of mine, so long as you give me proper credit

If a metal bar is struck against a hard surface, the bar will "ring" with a characteristic frequency.  This is the fundamental principle upon which {\it tuning forks} work:

$$\epsfbox{00600x01.eps}$$

The ability of any physical object to "ring" like this after being struck is dependent upon two complementary properties: {\it mass} and {\it elasticity}.  An object must possess both mass and a certain amount of "springiness" in order to physically resonate.

Describe what would happen to the resonant frequency of a metal bar if it were made of a more elastic (less "stiff") metal?  What would happen to the resonant frequency if an extra amount of mass were added to the end being struck?

\underbar{file 00600}
%(END_QUESTION)





%(BEGIN_ANSWER)

In either case, the resonant frequency of the bar would {\it decrease}.

%(END_ANSWER)





%(BEGIN_NOTES)

Electrical resonance is so closely related to physical resonance, that I believe questions like this help students grasp the concept better.  Everyone knows what resonance is in the context of a vibrating object (tuning fork, bell, wind chime, guitar string, cymbal head), even if they have never heard of the term "resonance" before.  Getting them to understand that mechanical resonance depends on the complementary qualities of mass and elasticity primes their minds for understanding that electrical resonance depends on the complementary qualities of inductance and capacitance.

%INDEX% Resonance, mechanical

%(END_NOTES)


