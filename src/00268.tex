
%(BEGIN_QUESTION)
% Copyright 2003, Tony R. Kuphaldt, released under the Creative Commons Attribution License (v 1.0)
% This means you may do almost anything with this work of mine, so long as you give me proper credit

Determine how much voltage a voltmeter would indicate when connected between the following points in this circuit:

$$\epsfbox{00268x01.eps}$$

\medskip
\item{$\bullet$} Between TB1-1 and TB1-3
\item{$\bullet$} Between TB1-4 and TB2-4
\item{$\bullet$} Between TB2-3 and TB2-1
\item{$\bullet$} Between TB1-1 and TB2-1
\medskip

Hint: it might help to draw a neat schematic diagram of this circuit first, with all connection points labeled!

\underbar{file 00268}
%(END_QUESTION)





%(BEGIN_ANSWER)

\medskip
\item{$\bullet$} Between TB1-1 and TB1-3 ({\it Voltmeter measures 9 volts})
\item{$\bullet$} Between TB1-4 and TB2-4 ({\it Voltmeter measures 0 volts})
\item{$\bullet$} Between TB2-3 and TB2-1 ({\it Voltmeter measures 0 volts})
\item{$\bullet$} Between TB1-1 and TB2-1 ({\it Voltmeter measures 9 volts})
\medskip

%(END_ANSWER)





%(BEGIN_NOTES)

This question provides an opportunity to discuss the concept of {\it electrically common points}: namely, that there can be no substantial voltage built up between points that are made "electrically common" by means of low-resistance connections between them.

This is also an opportunity to develop the skill of drawing a schematic diagram for a real-life circuit.  Schematic diagrams, of course, are very helpful in that they provide a nice, neat layout of all circuit components, making visualization of voltage drops and other quantities easier.

%INDEX% Voltmeter usage

%(END_NOTES)


