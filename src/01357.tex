
%(BEGIN_QUESTION)
% Copyright 2003, Tony R. Kuphaldt, released under the Creative Commons Attribution License (v 1.0)
% This means you may do almost anything with this work of mine, so long as you give me proper credit

A variation on the gated S-R latch circuit is something called the {\it D-latch}:

$$\epsfbox{01357x01.eps}$$

Complete the truth table for this D latch circuit, and identify which rows in the truth table represent the {\it set}, {\it reset}, and {\it latch} states, respectively.

\underbar{file 01357}
%(END_QUESTION)





%(BEGIN_ANSWER)

$$\epsfbox{01357x02.eps}$$

%(END_ANSWER)





%(BEGIN_NOTES)

Since this gate does not actually have "Set" and "Reset" inputs, ask your students to explain what conditions define the "set" and "reset" states.  Note that these state labels may be applied to {\it any} type of latch circuit.

To many of your students, this latch circuit may seem rather useless.  Explain to them that this basic latch may be used to form {\it memory cells}, with each D latch storing 1 binary bit of information!  Ask your students to explain, in their own words, how the latching action of this circuit constitutes a memory function.  Under what condition(s) will the stored information in a D latch memory cell be lost?

%(END_NOTES)


