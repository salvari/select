
%(BEGIN_QUESTION)
% Copyright 2003, Tony R. Kuphaldt, released under the Creative Commons Attribution License (v 1.0)
% This means you may do almost anything with this work of mine, so long as you give me proper credit

At some point in time, it was decided that the unit of the "Bel" was too large.  Instead, the {\it deci}-Bel became the most common usage of the unit.  Modify these equations to include $A_P$ figures cast in units of decibels (dB) instead of Bels:

$$A_{P(ratio)} = 10^{A_{P(Bels)}}$$

$$A_{P(Bels)} = \log A_{P(ratio)}$$

Then, calculate the decibel figures that correspond to a power gain of 2 (ratio), and a power loss of 50\%, respectively.

\underbar{file 00677}
%(END_QUESTION)





%(BEGIN_ANSWER)

$$A_{P(ratio)} = 10^{A_{P(dB)} \over 10}$$

$$A_{P(dB)} = 10 \log A_{P(ratio)}$$

\vskip 10pt

Power gain of 2 (ratio) $\approx$ 3 dB

\vskip 10pt

Power loss of 50\% (ratio) $\approx$ -3 dB

%(END_ANSWER)





%(BEGIN_NOTES)

It is important that students work through the original equations algebraically to obtain the answers rather than just look up these formulae in a book.  Have your students write their work on the whiteboard in front of the other students, so that everyone has the opportunity to examine the technique(s) and ask pertinent questions.

Be sure to let your students know that the figure of "3 dB", either positive or negative, is very common in electronics calculations.  Your students might remember this expression used to describe the cutoff frequency of a filter circuit ($f_{-3 dB}$).

%INDEX% Decibel, defined

%(END_NOTES)


