
%(BEGIN_QUESTION)
% Copyright 2005, Tony R. Kuphaldt, released under the Creative Commons Attribution License (v 1.0)
% This means you may do almost anything with this work of mine, so long as you give me proper credit

$$\epsfbox{03468x01.eps}$$

\underbar{file 03468}
\vfil \eject
%(END_QUESTION)





%(BEGIN_ANSWER)

Use circuit simulation software to verify your predicted and measured parameter values.

%(END_ANSWER)





%(BEGIN_NOTES)

This is a very interesting circuit to built and test.  You may build one using 1 $\mu$F capacitors, 2.7 k$\Omega$ resistors, and a 100 k$\Omega$ potentiometer that will successfully operate on 60 Hz power-line excitation.  If you prefer to use audio frequency power, try 0.047 $\mu$F capacitors, 1 k$\Omega$ resistors, a 100 k$\Omega$ potentiometer, and 3.386 kHz for the source frequency.

An interesting thing to note about using line power is that any distortions in the excitation sine-wave will become obvious when the potentiometer wiper is turned toward the differentiating position (where $\Theta$ is positive).  If listened to with an audio detector, you may even hear the change in timbre while moving the wiper from one extreme to the other.  If excited by a "clean" sine-wave, however, no change in timbre should be heard because there are no harmonics present.

%INDEX% Assessment, performance-based (Phase shift circuit)

%(END_NOTES)


