
%(BEGIN_QUESTION)
% Copyright 2003, Tony R. Kuphaldt, released under the Creative Commons Attribution License (v 1.0)
% This means you may do almost anything with this work of mine, so long as you give me proper credit

Calculate the voltage indicated by the voltmeter in this circuit for the following voltage inputs:

$$\epsfbox{00475x01.eps}$$

\medskip
\item{} $V_1 =$ 4.0 volts
\item{} $V_2 =$ 5.0 volts
\item{} $V_3 =$ 12.0 volts
\medskip

What do you notice about the output voltage of this circuit?  What mathematical function does this circuit perform?

\underbar{file 00475}
%(END_QUESTION)





%(BEGIN_ANSWER)

$V_{out} =$ 7.0 volts

\vskip 5pt

This circuit is a very simple form of {\it analog computer}, because it has the ability to perform a mathematical operation, with voltages representing numerical quantities!

%(END_ANSWER)





%(BEGIN_NOTES)

Not only does this simple circuit provide an excellent opportunity to practice using Millman's theorem, but it also illustrates the important principle of using resistor networks to perform mathematical functions.  In essence, this circuit is a form of computer (an {\it analog} computer), capable of "calculating" at a rate of speed unmatched by any digital computer.

Ask your students to think of the advantages an analog computer such as this would have over a digital computer, and visa-versa.  How come analog computers are seldom used, and digital technology is so prevalent?  Does this mean analog computer technology has no place in modern electronics?

%INDEX% Passive averager circuit
%INDEX% Averager circuit, passive

%(END_NOTES)


