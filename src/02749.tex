
%(BEGIN_QUESTION)
% Copyright 2005, Tony R. Kuphaldt, released under the Creative Commons Attribution License (v 1.0)
% This means you may do almost anything with this work of mine, so long as you give me proper credit

Just as certain assumptions are often made for bipolar transistors in order to simplify their analysis in circuits (an ideal BJT has negligible base current, $I_C = I_E$, constant $\beta$, etc.), we often make assumptions about operational amplifiers so we may more easily analyze their behavior in closed-loop circuits.  Identify some of these ideal opamp assumptions as they relate to the following parameters:

\medskip
\goodbreak
\item{$\bullet$} Magnitude of input terminal currents:
\item{$\bullet$} Input impedance:
\item{$\bullet$} Output impedance:
\item{$\bullet$} Input voltage range:
\item{$\bullet$} Output voltage range:
\item{$\bullet$} Differential voltage (between input terminals) with negative feedback:
\medskip

\underbar{file 02749}
%(END_QUESTION)





%(BEGIN_ANSWER)

\medskip
\goodbreak
\item{$\bullet$} Magnitude of input terminal currents: {\it infinitesimal}
\item{$\bullet$} Input impedance: {\it infinite}
\item{$\bullet$} Output impedance: {\it infinitesimal}
\item{$\bullet$} Input voltage range: {\it never exceeding +V/-V}
\item{$\bullet$} Output voltage range: {\it never exceeding +V/-V}
\item{$\bullet$} Differential voltage (between input terminals) with negative feedback: {\it infinitesimal}
\medskip

%(END_ANSWER)





%(BEGIN_NOTES)

Just in case your students are unfamiliar with the words {\it infinite} and {\it infinitesimal}, tell them they simply mean "bigger than big" and "smaller than small", respectively.

%INDEX% Differential voltage, for opamp with negative feedback
%INDEX% Opamps, general rules for circuit analysis

%(END_NOTES)


