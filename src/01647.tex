
%(BEGIN_QUESTION)
% Copyright 2003, Tony R. Kuphaldt, released under the Creative Commons Attribution License (v 1.0)
% This means you may do almost anything with this work of mine, so long as you give me proper credit

$$\epsfbox{01647x01.eps}$$

\underbar{file 01647}
\vfil \eject
%(END_QUESTION)





%(BEGIN_ANSWER)

Use circuit simulation software to verify your predicted and measured parameter values.

%(END_ANSWER)





%(BEGIN_NOTES)

Two very important "given" parameters are the relay coil resistance ($R_{coil}$) and the relay dropout voltage ($V_{dropout}$).  These are best determined experimentally.

Many students fail to grasp the purpose of this exercise until it is explained.  The idea here is to predict {\it when} the relay will "drop out" after the switch is opened.  This means solving for $t$ in the time-constant (decay) equation given the initial capacitor voltage, time constant ($\tau$), and the capacitor voltage at time $t$.  Because this involves the use of logarithms, students may be perplexed until given assistance.

An extension of this exercise is to incorporate troubleshooting questions.  Whether using this exercise as a performance assessment or simply as a concept-building lab, you might want to follow up your students' results by asking them to predict the consequences of certain circuit faults.

%INDEX% Assessment, performance-based (RC time-delay relay)

%(END_NOTES)


