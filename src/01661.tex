
%(BEGIN_QUESTION)
% Copyright 2003, Tony R. Kuphaldt, released under the Creative Commons Attribution License (v 1.0)
% This means you may do almost anything with this work of mine, so long as you give me proper credit

$$\epsfbox{01661x01.eps}$$

\underbar{file 01661}
\vfil \eject
%(END_QUESTION)





%(BEGIN_ANSWER)

Use circuit simulation software to verify your predicted and measured parameter waveforms.

\vskip 10pt

Note: this circuit works on the same basic principle as the compensation adjustment on high-quality oscilloscope probes, except that here the resistor is variable and not the capacitor.

%(END_ANSWER)





%(BEGIN_NOTES)

Use a square-wave function generator for the AC voltage source.  I have built this circuit using resistances of approximately 4.7 k$\Omega$ and capacitances of approximately 0.1 $\mu$F, and achieved good results with a square-wave (fundamental) frequency of about 300 Hz.  Of course, the $R_1$ resistance calculation is more challenging if the two capacitor values are unequal!

Incidentally, this circuit may be used to comparatively measure reactive components such as capacitors and inductors, since the square wave signal will be faithfully reproduced only when the $R_1 \over R_2$ ratio equals the $X_{C1} \over X_{C2}$ or $X_{L1} \over X_{L2}$ ratio.  To be honest, the idea was not mine.  I found it in an old book, the {\it Electronics Manual for Radio Engineers}, by Vin Zeluff and John Markus, first edition (1949), page 427.  Apparently, the inventor of this ingenious impedance measurement technique was an employee of Allen B. DuMont Lab., Inc. named Peter S. Christaldi, who obtained a patent for the technique on October 15, 1946 (patent number 2,409,419 for those who are interested).

An extension of this exercise is to incorporate troubleshooting questions.  Whether using this exercise as a performance assessment or simply as a concept-building lab, you might want to follow up your students' results by asking them to predict the consequences of certain circuit faults.

%INDEX% Assessment, performance-based (Balanced attenuator circuit)

%(END_NOTES)


