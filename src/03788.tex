
%(BEGIN_QUESTION)
% Copyright 2005, Tony R. Kuphaldt, released under the Creative Commons Attribution License (v 1.0)
% This means you may do almost anything with this work of mine, so long as you give me proper credit

Given the fact that a switched-capacitor network has the ability to emulate a variable resistance, what advantage(s) can you think of for using switched-capacitor networks inside of analog integrated circuits?  Identify some practical applications of this technology.

\underbar{file 03788}
%(END_QUESTION)





%(BEGIN_ANSWER)

Switched-capacitor networks allow us to have {\it electronically variable resistors} inside integrated circuits, with no moving parts, which is a technological advantage over standard resistors.  Another advantage of switched-capacitor circuits is that they typically require less substrate area on an integrated circuit than an equivalent resistor.  A huge advantage is that switched-capacitor networks may be manufactured to give much more accurate resistances than plain resistors, which is important in filtering applications.

\vskip 10pt

I'll let you research (or dream up) some practical applications for this technology.

%(END_ANSWER)





%(BEGIN_NOTES)

This question could very well lead to an interesting and lively discussion with your students.  Once students recognize the equivalence between switched-capacitor networks and resistors, it is a short cognitive leap from there to practical applications where variable resistances would be useful (especially in an integrated circuit environment, where moving parts have traditionally been out of the question).

%INDEX% Switched capacitor network, practical applications for

%(END_NOTES)


