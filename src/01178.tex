
%(BEGIN_QUESTION)
% Copyright 2005, Tony R. Kuphaldt, released under the Creative Commons Attribution License (v 1.0)
% This means you may do almost anything with this work of mine, so long as you give me proper credit

This is a schematic of an RF amplifier using a JFET as the active element:

$$\epsfbox{01178x01.eps}$$

What configuration of JFET amplifier is this (common drain, common gate, or common source)?  Also, explain the purpose of the two iron-core inductors in this circuit.  Hint: inductors $L_1$ and $L_2$ are often referred to as {\it RF chokes}.

\underbar{file 01178}
%(END_QUESTION)





%(BEGIN_ANSWER)

This is a common-gate amplifier.  The iron-core inductors block ("choke") the high-frequency AC signals from getting to the DC power supply.

%(END_ANSWER)





%(BEGIN_NOTES)

Be sure to ask your students {\it why} it would not be good for the RF signals to find their way to the DC power supply.  There is more than one possible answer to this question!

This schematic was derived from an evaluation amplifier schematic shown in an \underbar{ON Semiconductor} J308/J309/J310 transistor datasheet.

%INDEX% Common-gate JFET amplifier, for high-frequency use
%INDEX% High-frequency JFET amplifier

%(END_NOTES)


