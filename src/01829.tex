
%(BEGIN_QUESTION)
% Copyright 2003, Tony R. Kuphaldt, released under the Creative Commons Attribution License (v 1.0)
% This means you may do almost anything with this work of mine, so long as you give me proper credit

One way to vary the amount of power delivered to a resistive AC load is by varying another resistance connected in series:

$$\epsfbox{01829x01.eps}$$

A problem with this power control strategy is that power is wasted in the series resistance ($I^2R_{series}$).  A different strategy for controlling power is shown here, using a series {\it inductance} rather than resistance:

$$\epsfbox{01829x02.eps}$$

Explain why the latter circuit is more power-efficient than the former, and draw a phasor diagram showing how changes in $L_{series}$ affect $Z_{total}$.

\underbar{file 01829}
%(END_QUESTION)





%(BEGIN_ANSWER)

Inductors are {\it reactive} rather than {\it resistive} components, and therefore do not dissipate power (ideally).

$$\epsfbox{01829x03.eps}$$

\vskip 10pt

Follow-up question: the inductive circuit is not just more energy-efficient -- it is safer as well.  Identify a potential safety hazard that the resistive power-control circuit poses due to the energy dissipation of its variable resistor.

%(END_ANSWER)





%(BEGIN_NOTES)

If appropriate, you may want to mention devices called {\it saturable reactors}, which are used to control power in AC circuits by the exact same principle: varying a series inductance.

%INDEX% Impedance in series LR circuit (qualitative, with phasor diagram)
%INDEX% Phasor diagram

%(END_NOTES)


