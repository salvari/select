
%(BEGIN_QUESTION)
% Copyright 2003, Tony R. Kuphaldt, released under the Creative Commons Attribution License (v 1.0)
% This means you may do almost anything with this work of mine, so long as you give me proper credit

Metric prefixes are nothing more than "shorthand" representations for certain powers of ten.  Express the following quantities of mass (in units of grams) using metric prefixes rather than scientific notation, and complete the "index" of metric prefixes shown below:

$$8.3 \times 10^{18} \hbox{ g} = $$

$$3.91 \times 10^{15} \hbox{ g} = $$

$$5.2 \times 10^{12} \hbox{ g} = $$

$$9.3 \times 10^9 \hbox{ g} = $$

$$6.7 \times 10^6 \hbox{ g} = $$

$$6.8 \times 10^3 \hbox{ g} = $$

$$4.5 \times 10^2 \hbox{ g} = $$

$$8.11 \times 10^1 \hbox{ g} = $$

$$2.1 \times 10^{-1} \hbox{ g} = $$

$$1.9 \times 10^{-2} \hbox{ g} = $$

$$9.32 \times 10^{-3} \hbox{ g} = $$

$$1.58 \times 10^{-6} \hbox{ g} = $$

$$8.80 \times 10^{-9} \hbox{ g} = $$

$$6.9 \times 10^{-12} \hbox{ g} = $$

$$7.2 \times 10^{-15} \hbox{ g} = $$

$$4.1 \times 10^{-18} \hbox{ g} = $$

$$\epsfbox{01709x01.eps}$$

\underbar{file 01709}
%(END_QUESTION)





%(BEGIN_ANSWER)

$$8.3 \times 10^{18} \hbox{ g} = 8.3 \hbox{ Eg}$$

$$3.91 \times 10^{15} \hbox{ g} = 3.91 \hbox{ Pg}$$

$$5.2 \times 10^{12} \hbox{ g} = 5.2 \hbox{ Tg}$$

$$9.3 \times 10^9 \hbox{ g} = 9.3 \hbox{ Gg}$$

$$6.7 \times 10^6 \hbox{ g} = 6.7 \hbox{ Mg}$$

$$6.8 \times 10^3 \hbox{ g} = 6.8 \hbox{ kg}$$

$$4.5 \times 10^2 \hbox{ g} = 4.5 \hbox{ hg}$$

$$8.11 \times 10^1 \hbox{ g} = 8.11 \hbox{ dag}$$

$$2.1 \times 10^{-1} \hbox{ g} = 2.1 \hbox{ dg}$$

$$1.9 \times 10^{-2} \hbox{ g} = 1.9 \hbox{ cg}$$

$$9.32 \times 10^{-3} \hbox{ g} = 9.32 \hbox{ mg}$$

$$1.58 \times 10^{-6} \hbox{ g} = 1.58 \> \mu \hbox{g}$$

$$8.80 \times 10^{-9} \hbox{ g} = 8.80 \hbox{ ng}$$

$$6.9 \times 10^{-12} \hbox{ g} = 6.9 \hbox{ pg}$$

$$7.2 \times 10^{-15} \hbox{ g} = 7.2 \hbox{ fg}$$

$$4.1 \times 10^{-18} \hbox{ g} = 4.1 \hbox{ ag}$$

$$\epsfbox{01709x02.eps}$$

%(END_ANSWER)





%(BEGIN_NOTES)

Once students realize metric prefixes are nothing more than shorthand for certain powers of ten, they see that scientific and metric notations are really one and the same.  The only difficult part of this is committing to memory all the different metric prefixes and their respective powers of ten.  Be sure to mention that the power of ten that are multiples of three ($10^{3}$, $10^{6}$, $10^{-12}$, etc.) are more commonly used than the other powers (h, da, d, and c).

In case anyone asks, the metric prefix for $10^1$ (deka) is sometimes spelled {\it deca}.  There seems to be no "standard" way to spell the name of this prefix.

%INDEX% Notation, scientific
%INDEX% Notation, metric

%(END_NOTES)


