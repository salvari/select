
%(BEGIN_QUESTION)
% Copyright 2003, Tony R. Kuphaldt, released under the Creative Commons Attribution License (v 1.0)
% This means you may do almost anything with this work of mine, so long as you give me proper credit

Determine the total voltage in each of these examples, drawing a phasor diagram to show how the total (resultant) voltage geometrically relates to the source voltages in each scenario:

$$\epsfbox{00498x01.eps}$$

\underbar{file 00498}
%(END_QUESTION)





%(BEGIN_ANSWER)

$$\epsfbox{00498x02.eps}$$

%(END_ANSWER)





%(BEGIN_NOTES)

At first it may confuse students to use polarity marks (+ and -) for AC voltages.  After all, doesn't the polarity of AC {\it alternate} back and forth, so as to be continuously changing?  However, when analyzing AC circuits, polarity marks are essential for giving a frame of reference to phasor voltages, which like all voltages are measured {\it between two points}, and thus may be measured two different ways.

%INDEX% Phasor addition

%(END_NOTES)


