
%(BEGIN_QUESTION)
% Copyright 2003, Tony R. Kuphaldt, released under the Creative Commons Attribution License (v 1.0)
% This means you may do almost anything with this work of mine, so long as you give me proper credit

Shown here is a simplified schematic diagram of one of the operational amplifiers inside an LM324 quad op-amp integrated circuit:

$$\epsfbox{00799x01.eps}$$

Qualitatively determine what will happen to the output voltage ($V_{out}$) if the voltage on the inverting input ($V_{in-}$) increases, and the voltage on the noninverting input ($V_{in+}$) remains the same (all voltages are positive quantities, referenced to ground).  Explain what happens at every stage of the op-amp circuit (voltages increasing or decreasing, currents increasing or decreasing) with this change in input voltage.

\underbar{file 00799}
%(END_QUESTION)





%(BEGIN_ANSWER)

Here, I've labeled a few of the important voltage changes in the circuit, resulting from the increase in inverting input voltage ($V_{in-}$):

$$\epsfbox{00799x02.eps}$$

%(END_ANSWER)





%(BEGIN_NOTES)

The answer provided here is minimal.  Challenge your students to follow the whole circuit through until the end, qualitatively assessing voltage and current changes.

Incidentally, the strange-looking double-circle symbol is a {\it current source}.  Ask your students if they were able to find a reference anywhere describing what this symbol means.

%INDEX% Opamp, simplified internal schematic of (LM324)

%(END_NOTES)


