
%(BEGIN_QUESTION)
% Copyright 2003, Tony R. Kuphaldt, released under the Creative Commons Attribution License (v 1.0)
% This means you may do almost anything with this work of mine, so long as you give me proper credit

In most high-power DC generator and motor designs, the wire used to make the field winding is much thinner gauge than the wire used to make the armature winding.  This indicates the relative magnitude of current through these respective windings, with the armature coils conducting much more current than the field coils.

That the armature conducts more current than the field is no small matter, because all current through the armature must be conducted through the brushes and commutator bars.  The more current these components have to carry, the shorter their life, all other factors being equal.

Couldn't the generator be re-designed so that the field conducted most of the current, with the armature only conducting a small amount?  This way, the brushes and commutator bars would only have to carry a fraction of their normal current, making them less expensive and longer-lived.  Explain why this is impossible to do.

\vskip 10pt

Hint: consider the design of a permanent-magnet generator.

\underbar{file 00815}
%(END_QUESTION)





%(BEGIN_ANSWER)

It is impossible for the field winding to conduct more current than the armature in a functioning DC generator, because the armature has to be the {\it source} of electrical power, while the field is only a {\it load}.

%(END_ANSWER)





%(BEGIN_NOTES)

Being that brush and commutator wear is the main reason AC motors and generators are favored over DC, any idea that may potentially reduce the "wear and tear" on DC motor or generator brushes is worth considering.  However, the idea proposed in this question will never work.  This is not necessarily an easy question to answer, as it tests the students' comprehension of generator theory.  The hint given in the question ("consider a permanent-magnet generator") is intended to force students to simplify the problem, by considering a working generator design that only has one winding (the armature).  By simplifying the problem in this way, students should see that the armature winding {\it has to} carry the bulk of the current in a DC generator.

%INDEX% Armature versus field windings, DC generator
%INDEX% Field versus armature windings, DC generator

%(END_NOTES)


