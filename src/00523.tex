
%(BEGIN_QUESTION)
% Copyright 2003, Tony R. Kuphaldt, released under the Creative Commons Attribution License (v 1.0)
% This means you may do almost anything with this work of mine, so long as you give me proper credit

In chemistry laboratories, {\it balance} scales are used to precisely measure the mass of various substances.  How, exactly, is a laboratory "balance" scale used?  What component of the scale primarily determines its accuracy?

$$\epsfbox{00523x01.eps}$$

\underbar{file 00523}
%(END_QUESTION)





%(BEGIN_ANSWER)

A "balance" scale works on the principle of masses in equilibrium: the unknown mass is countered with known quantities of mass until the scale registers a condition of balance.

%(END_ANSWER)





%(BEGIN_NOTES)

The important point I wish to communicate with this question is that the scale does nothing but indicate a condition of balance (zero excess mass on either side).  As such, the primary source of accuracy in this measurement system does not lie within the scale mechanism itself!  This is an important quality of a measurement system: to isolate sources of inaccuracy to very specific portions of the system, where they may be tightly controlled.

%(END_NOTES)


