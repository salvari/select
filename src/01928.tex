
%(BEGIN_QUESTION)
% Copyright 2003, Tony R. Kuphaldt, released under the Creative Commons Attribution License (v 1.0)
% This means you may do almost anything with this work of mine, so long as you give me proper credit

When testing this circuit for the first time, the student connects a 1 k$\Omega$ resistor to the two test leads and gets this result from the oscilloscope display:

$$\epsfbox{01928x01.eps}$$

The student was actually expecting something that looked like this:

$$\epsfbox{01928x02.eps}$$

Upon inspection, nothing appears to be wrong with the wiring of the curve tracer circuit.  Explain what the problem is so the student is able to achieve the expected results.

\underbar{file 01928}
%(END_QUESTION)





%(BEGIN_ANSWER)

One of the oscilloscope's two input channels must be {\it inverted} for the curve tracer to work as expected.  Most oscilloscopes come equipped with an "invert" control on the second input channel that is used for this purpose.

\vskip 10pt

Follow-up question: is the curve tracer circuit configured for an AC test or a DC test, based on the appearance of the oscilloscope trace?

%(END_ANSWER)





%(BEGIN_NOTES)

This circuit provides an excellent opportunity for students to discuss and review the common grounds of oscilloscope inputs, and why one of the oscilloscope's inputs must be wired "backward" in order to yield the expected trace from lower-left to upper-right.

%(END_NOTES)


