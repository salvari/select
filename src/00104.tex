
%(BEGIN_QUESTION)
% Copyright 2003, Tony R. Kuphaldt, released under the Creative Commons Attribution License (v 1.0)
% This means you may do almost anything with this work of mine, so long as you give me proper credit

The brightness of a light bulb -- or the power dissipated by any electrical load, for that matter -- may be varied by inserting a variable resistance in the circuit, like this:

$$\epsfbox{00104x01.eps}$$

This method of electrical power control is not without its disadvantages, though.  Consider an example where the circuit current is 5 amps, the variable resistance is 2 $\Omega$, and the lamp drops 20 volts of voltage across its terminals.  Calculate the power dissipated by the lamp, the power dissipated by the variable resistance, and the total power provided by the voltage source.  Then, explain why this method of power control is not ideal.

\underbar{file 00104}
%(END_QUESTION)





%(BEGIN_ANSWER)

$P_{lamp} =$ 100 watts

$P_{resistance} =$ 50 watts

$P_{total} =$ 150 watts

\vskip 10pt

Follow-up question: note how in the original question I offered a set of hypothetical values to use in figuring out why a series rheostat (variable resistance) is not an efficient means to control lamp power.  Explain how the assumption of certain values is a useful problem-solving technique in cases where no values are given to you.

%(END_ANSWER)





%(BEGIN_NOTES)

Discuss the concept of energy conservation: that energy can neither be created nor destroyed, but merely changed between different forms.  Based on this principle, the sum of all power dissipations in a circuit must equal the total amount of power supplied by the energy source, regardless of how the components are connected together.

%INDEX% Power control
%INDEX% Variable resistor

%(END_NOTES)


