
%(BEGIN_QUESTION)
% Copyright 2003, Tony R. Kuphaldt, released under the Creative Commons Attribution License (v 1.0)
% This means you may do almost anything with this work of mine, so long as you give me proper credit

Practical integrator circuits must have a {\it compensating resistor} connected in parallel with the capacitor, in the feedback loop.  Typically, this resistor value is very large: about 100 times as large as $R_{in}$.  

$$\epsfbox{01012x01.eps}$$

Describe why this is a necessity for accurate integration.  Hint: an ideal opamp would not need this resistor!

\underbar{file 01012}
%(END_QUESTION)





%(BEGIN_ANSWER)

This compensation resistor helps offset errors otherwise incurred by the opamp's {\it bias current} on the inverting input.

%(END_ANSWER)





%(BEGIN_NOTES)

Discuss where bias currents originate from in the opamp's internal circuitry, and ask your students if they have any recommendations on specific opamp types that minimize bias current.

%INDEX% Integrator circuit, opamp (with compensation resistor)

%(END_NOTES)


