
%(BEGIN_QUESTION)
% Copyright 2003, Tony R. Kuphaldt, released under the Creative Commons Attribution License (v 1.0)
% This means you may do almost anything with this work of mine, so long as you give me proper credit

Suppose you were measuring the voltage between "test points" TP1 and TP2 on this printed circuit board, where the circuit receives DC power from a voltage source.  Suddenly, this voltage decreases.  Wondering if perhaps the power supply voltage itself decreased, you go back to measure its output voltage and find that it is unchanged from before:

$$\epsfbox{00349x01.eps}$$

Based on your knowledge of Kirchhoff's Voltage Law, what must have happened to cause the voltage between TP1 and TP2 on the circuit board to suddenly decrease, given the power supply's stable voltage output?  Hint: the components on the circuit board are irrelevant to the answer!

\underbar{file 00349}
%(END_QUESTION)





%(BEGIN_ANSWER)

Voltage drop(s) somewhere in the loop between TP1/TP2 and the power supply {\it must} have increased, in order to account for $E_{TP1-TP2}$ decreasing with a constant supply voltage.

%(END_ANSWER)





%(BEGIN_NOTES)

Ask your students where the increased voltage drop(s) might be located in this circuit.  It might be helpful to draw the test points, power cable, and power supply as a circuit, using standard schematic symbols.  Ask your students, "What type of fault might cause such voltage drop increases in this loop?"

%INDEX% Kirchhoff's Voltage Law
%INDEX% Troubleshooting, simple circuit

%(END_NOTES)


