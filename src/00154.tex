
%(BEGIN_QUESTION)
% Copyright 2003, Tony R. Kuphaldt, released under the Creative Commons Attribution License (v 1.0)
% This means you may do almost anything with this work of mine, so long as you give me proper credit

How is an electrical switch constructed?  What goes on inside the switch that actually "makes" or "breaks" a path for electric current?

\underbar{file 00154}
%(END_QUESTION)





%(BEGIN_ANSWER)

Switches typically use metal {\it contacts} that are touched together or moved apart by some sort of actuating lever, shaft, or other mechanical assembly.

%(END_ANSWER)





%(BEGIN_NOTES)

An inexpensive type of switch I use for teaching basic electricity/electronics classes is a household light switch.  Available from hardware stores for very little cost, these switches are rugged and easy to connect into real circuits, large or small.  For this question, you might want to let students disassemble one or two of these switches to observe how they are constructed.

This question is also a good point to start a conversation on the {\it reliability} of switches.  Being that it has moving parts, what could possibly fail in a switch?  How about the contacts themselves: what might happen to them over time, especially if "overloaded" with too much electrical current?

%INDEX% Switch, principle of operation
%INDEX% Switch, contact
%INDEX% Contact, switch

%(END_NOTES)


