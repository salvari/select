
%(BEGIN_QUESTION)
% Copyright 2003, Tony R. Kuphaldt, released under the Creative Commons Attribution License (v 1.0)
% This means you may do almost anything with this work of mine, so long as you give me proper credit

Build a simple electric circuit using a battery as the electrical energy {\it source}, and a small light bulb as the electrical {\it load} (I suggest using a 6-volt "lantern" battery and a miniature incandescent light bulb rated for either 6 or 12 volts).  Use "jumper" wires with metal clips at the ends to join these two electrical devices together:

$$\epsfbox{01697x01.eps}$$

After connecting the components together properly so the light bulb lights up, answer the following questions:

\medskip
\item{$\bullet$} What conditions must be met for the light bulb to light up?
\item{$\bullet$} What happens if the circuit is "broken"?
\item{$\bullet$} Does it matter where the circuit is "broken"?
\medskip

Then, add a third jumper wire to the circuit so you have a ready "break" to experiment with:

$$\epsfbox{01697x02.eps}$$

Try bridging this "break" with various materials, and note whether or not the light bulb lights up:

\medskip
\item{$\bullet$} Paper
\item{$\bullet$} Steel paper clip
\item{$\bullet$} Gold ring
\item{$\bullet$} Rubber eraser
\item{$\bullet$} Pencil lead (graphite)
\medskip

Also, try touching the jumper wire ends together along their plastic exteriors, rather than at the metal "clip" ends.  Does the light bulb light up when you do this?

\vskip 10pt

Explain what this experiment demonstrates about the electrical {\it conductivity} of the various substances listed as well as the plastic coating of the jumper wires.  Also explain why electrical wires are provided with that plastic coating, instead of being bare metal.  Finally, explain what this experiment has taught you about electric circuits in general.

\underbar{file 01697}
%(END_QUESTION)





%(BEGIN_ANSWER)

Let the electrons show you the answers to these questions!

%(END_ANSWER)





%(BEGIN_NOTES)

I find that 6-volt "lantern" batteries work well for an experiment such as this, along with either 6 or 12 volt miniature light bulbs.  Sometimes the over-rated light bulbs (12 volt rated lamp powered by a 6 volt battery) work better for showing students the glowing filament.  The filament of an incandescent light bulb at full brightness is difficult to distinguish.

Please avoid using LED's or any polarity-sensitive devices until your students are ready to explore polarity!

%INDEX% Circuit assembly
%INDEX% Circuit, simple
%INDEX% Circuit, simple conductivity tester
%INDEX% Conductivity, experimentally demonstrated

%(END_NOTES)


