
%(BEGIN_QUESTION)
% Copyright 2003, Tony R. Kuphaldt, released under the Creative Commons Attribution License (v 1.0)
% This means you may do almost anything with this work of mine, so long as you give me proper credit

Define the following terms, as they relate to digital memory devices:

\medskip
\item{$\bullet$} RAM:
\item{$\bullet$} ROM:
\item{$\bullet$} Volatile:
\item{$\bullet$} Nonvolatile:
\medskip

In particular, explain why "RAM" is a misleading term.

\underbar{file 01439}
%(END_QUESTION)





%(BEGIN_ANSWER)

{\it ROM} stands for Read-Only Memory, which means it can only be written to once.  {\it Volatile} and {\it Nonvolatile} refer to whether or not stored data is lost when the device is powered down.

Technically, {\it RAM} means Random-Access Memory, where data stored in memory may be accessed without having to "sift through" all the other bits of data in sequential order.  In practice, however, the term RAM is used to designate the volatile electronic memory inside a computer, which just happens to be randomly accessible.

%(END_ANSWER)





%(BEGIN_NOTES)

The mis-use of the acronym "RAM" is another unfortunate entry in the lexicon of electronics.  Your students are sure to have questions about this term, so be prepared to discuss it with them!

%INDEX% RAM, definition
%INDEX% ROM, definition
%INDEX% Volatile memory, definition
%INDEX% Nonvolatile memory, definition

%(END_NOTES)


