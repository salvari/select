
%(BEGIN_QUESTION)
% Copyright 2003, Tony R. Kuphaldt, released under the Creative Commons Attribution License (v 1.0)
% This means you may do almost anything with this work of mine, so long as you give me proper credit

Junction field-effect transistors have the ability to perform some functions that are impossible with (single) bipolar junction transistors.  Take this circuit, for example:

$$\epsfbox{00996x01.eps}$$

What effect will opening and closing the toggle switch have on the AC signal measured at the output terminals?

\underbar{file 00996}
%(END_QUESTION)





%(BEGIN_ANSWER)

When the toggle switch is open, the output signal will fall to (nearly) 0 volts AC.  When the toggle switch is closed, the output signal will be (nearly) the same as $V_{in}$.

%(END_ANSWER)





%(BEGIN_NOTES)

Discuss how the JFET is able to perform this AC signal "shunting" function, whereas a BJT would not be able to do the same.  Can your students think of any practical applications of a circuit like this?

%INDEX% Transistor switch circuit (JFET), for AC signals

%(END_NOTES)


