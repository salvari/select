
%(BEGIN_QUESTION)
% Copyright 2003, Tony R. Kuphaldt, released under the Creative Commons Attribution License (v 1.0)
% This means you may do almost anything with this work of mine, so long as you give me proper credit

The following illustration is a cross-section of an {\it insulated gate field-effect transistor}, sometimes referred to as an {\it IGFET}:

$$\epsfbox{02070x01.eps}$$

Explain what happens when a positive voltage is applied to the gate (with reference to the substrate), with regard to electrical conductivity between the source and drain:

$$\epsfbox{02070x02.eps}$$

\underbar{file 02070}
%(END_QUESTION)





%(BEGIN_ANSWER)

When enough positive voltage is applied to the gate, an {\it inversion layer} forms just beneath it, creating an N-type channel for source-drain current:

$$\epsfbox{02070x03.eps}$$

%(END_ANSWER)





%(BEGIN_NOTES)

Ask your students to explain how the inversion layer forms, and what it means for source-drain conduction if no inversion layer is present.

Discuss with your students the fact that this inversion layer is incredibly thin; so this that it is often referred to as a two-dimensional "sheet" of charge carriers.

Also mention to your students that although "IGFET" is the general term for such a device, "MOSFET" is more commonly used as a designator due to the device's history.

%INDEX% IGFET, defined
%INDEX% Inversion layer, MOSFET
%INDEX% MOSFET, principle of operation

%(END_NOTES)


