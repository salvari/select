
%(BEGIN_QUESTION)
% Copyright 2005, Tony R. Kuphaldt, released under the Creative Commons Attribution License (v 1.0)
% This means you may do almost anything with this work of mine, so long as you give me proper credit

Some of the following transistor switch circuits are properly configured, and some are not.  Identify which of these circuits will function properly (i.e. turn on the load when the switch closes) and which of these circuits are mis-wired:

$$\epsfbox{02409x01.eps}$$

\underbar{file 02409}
%(END_QUESTION)





%(BEGIN_ANSWER)

$$\epsfbox{02409x02.eps}$$

\vskip 10pt

Follow-up question: circuit \#3 is different from the other "bad" circuits.  While the other bad circuits' lamps do not energize at all, the lamp in circuit \#3 energizes weakly when the pushbutton switch is open (not actuated).  Explain why.

%(END_ANSWER)





%(BEGIN_NOTES)

This is a very important concept for students to learn if they are to do any switch circuit design -- a task not limited to engineers.  Technicians often must piece together simple transistor switching circuits to accomplish specific tasks on the job, so it is important for them to be able to design switching circuits that will work.

Have your students describe to the class how they were able to determine the status of each circuit, so that everyone may learn new ways of looking at this type of problem.  Also have them describe what would have to be changed in the "bad" circuits to make them functional.

%INDEX% Transistor switch circuit (BJT)

%(END_NOTES)


