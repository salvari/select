
%(BEGIN_QUESTION)
% Copyright 2003, Tony R. Kuphaldt, released under the Creative Commons Attribution License (v 1.0)
% This means you may do almost anything with this work of mine, so long as you give me proper credit

Determine the phase angle ($\Theta$) of the current in this circuit, with respect to the supply voltage:

$$\epsfbox{01853x01.eps}$$

\underbar{file 01853}
%(END_QUESTION)





%(BEGIN_ANSWER)

$\Theta = 26.51^o$

\vskip 10pt

Challenge question: explain how the following phasor diagram was determined for this problem:

$$\epsfbox{01853x02.eps}$$

%(END_ANSWER)





%(BEGIN_NOTES)

This is an interesting question for a couple of reasons.  First, students must determine how they will measure phase shift with just the two voltage indications shown by the meters.  This may present a significant challenge for some.  Discuss problem-solving strategies in class so that students understand how and why it is possible to determine $\Theta$.

Secondly, this is an interesting question because it shows how something as abstract as phase angle can be measured with just a voltmeter -- no oscilloscope required!  Not only that, but we don't even have to know the component values either!  Note that this technique works only for simple circuits.

A practical point to mention here is that multimeters have frequency limits which must be considered when taking measurements on electronic circuits.  Some high-quality handheld digital meters have frequency limits of hundred of kilohertz, while others fail to register accurately at only a few thousand hertz.  Unless we knew these two digital voltmeters were sufficient for measuring at the signal frequency, their indications would be useless to us.

%INDEX% Phase shift, measuring with voltmeter in series RC circuit

%(END_NOTES)


