
%(BEGIN_QUESTION)
% Copyright 2003, Tony R. Kuphaldt, released under the Creative Commons Attribution License (v 1.0)
% This means you may do almost anything with this work of mine, so long as you give me proper credit

$$\epsfbox{01609x01.eps}$$

\underbar{file 01609}
\vfil \eject
%(END_QUESTION)





%(BEGIN_ANSWER)

Use circuit simulation software to verify your predicted and measured parameter values.

%(END_ANSWER)





%(BEGIN_NOTES)

Use a variable-voltage, regulated power supply to supply any amount of DC voltage below 30 volts.  Specify standard resistor values, all between 1 k$\Omega$ and 100 k$\Omega$ (1k5, 2k2, 2k7, 3k3, 4k7, 5k1, 6k8, 8k2, 10k, 22k, 33k, 39k 47k, 68k, 82k, etc.).

I have used this circuit as both a "quick" lab exercise and a troubleshooting exercise, using values of 10 k$\Omega$ for R1, R2, and R3; 15 k$\Omega$ for R(load1); 22 k$\Omega$ for R(load2); and 6 volts for the power supply.  Of course, these component values are not critical, but they do provide easy-to measure voltages and currents without incurring excessive impedances that would cause significant voltmeter loading problems.

An extension of this exercise is to incorporate troubleshooting questions.  Whether using this exercise as a performance assessment or simply as a concept-building lab, you might want to follow up your students' results by asking them to predict the consequences of certain circuit faults.

%INDEX% Assessment, performance-based (Loaded DC voltage divider, 3 resistors, 2 loads)

%(END_NOTES)


