
%(BEGIN_QUESTION)
% Copyright 2003, Tony R. Kuphaldt, released under the Creative Commons Attribution License (v 1.0)
% This means you may do almost anything with this work of mine, so long as you give me proper credit

Determine the output voltage of this circuit for the following input voltage conditions:

\medskip
\item{$\bullet$} $V_1 = +2 \hbox{ volts}$
\item{$\bullet$} $V_3 = -1.5 \hbox{ volts}$
\item{$\bullet$} $V_1 = +2.2 \hbox{ volts}$
\medskip

$$\epsfbox{01175x01.eps}$$

\vskip 10pt

Hint: if you find this circuit too complex to analyze all at once, think of a way to simplify it so that you may analyze it one "piece" at a time.

\underbar{file 01175}
%(END_QUESTION)





%(BEGIN_ANSWER)

The output voltage will be +2.2 volts, precisely.

\vskip 10pt

Follow-up question: what function does this circuit perform?  Can you think of any practical applications for it?

%(END_ANSWER)





%(BEGIN_NOTES)

Another facet of this question to ponder with your students is the simplification process, especially for those students who experience difficulty analyzing the whole circuit.  What simplification methods did your students think of when they approached this problem?  What conclusions may be drawn about the general concept of problem simplification (as a problem-solving technique)?

%INDEX% High-select circuit, opamp
%INDEX% Precision rectifier circuit, opamp

%(END_NOTES)


