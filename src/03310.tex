
%(BEGIN_QUESTION)
% Copyright 2005, Tony R. Kuphaldt, released under the Creative Commons Attribution License (v 1.0)
% This means you may do almost anything with this work of mine, so long as you give me proper credit

% Uncomment the following line if the question involves calculus at all:
\vbox{\hrule \hbox{\strut \vrule{} $\int f(x) \> dx$ \hskip 5pt {\sl Calculus alert!} \vrule} \hrule}

Calculus is a branch of mathematics that originated with scientific questions concerning {\it rates of change}.  The easiest rates of change for most people to understand are those dealing with time.  For example, a student watching their savings account dwindle over time as they pay for tuition and other expenses is very concerned with rates of change ({\it dollars per year} being spent).

In calculus, we have a special word to describe rates of change: {\it derivative}.  One of the notations used to express a derivative (rate of change) appears as a fraction.  For example, if the variable $S$ represents the amount of money in the student's savings account and $t$ represents time, the rate of change of dollars over time would be written like this:

$${dS \over dt}$$

\goodbreak
The following set of figures puts actual numbers to this hypothetical scenario:

\medskip
\item{$\bullet$} Date: November 20
\vskip 5pt
\item{$\bullet$} Saving account balance ($S$) = \$12,527.33
\vskip 5pt
\item{$\bullet$} Rate of spending $\left({dS \over dt}\right)$ = -5,749.01 per year
\medskip

List some of the equations you have seen in your study of electronics containing derivatives, and explain how {\it rate of change} relates to the real-life phenomena described by those equations.

\underbar{file 03310}
%(END_QUESTION)





%(BEGIN_ANSWER)

\noindent
{\bf Voltage and current for a capacitor:}

$$i = C {dv \over dt}$$

\vskip 20pt

\noindent
{\bf Voltage and current for an inductor:}

$$v = L {di \over dt}$$

\vskip 20pt

\noindent
{\bf Electromagnetic induction:}

$$v = N {d \phi \over dt}$$

\vskip 20pt

I leave it to you to describe how the rate-of-change over time of one variable relates to the other variables in each of the scenarios described by these equations.

\vskip 10pt

Follow-up question: why is the derivative quantity in the student's savings account example expressed as a negative number?  What would a positive $dS \over dt$ represent in real life?

\vskip 10pt

Challenge question: describe actual circuits you could build to demonstrate each of these equations, so that others could see what it means for one variable's rate-of-change over time to affect another variable.

%(END_ANSWER)





%(BEGIN_NOTES)

The purpose of this question is to introduce the concept of the derivative to students in ways that are familiar to them.  Hopefully the opening scenario of a dwindling savings account is something they can relate to!

A very important aspect of this question is the discussion it will engender between you and your students regarding the relationship between rates of change in the three equations given in the answer.  It is very important to your students' comprehension of this concept to be able to verbally describe how the derivative works in each of these formulae.  You may want to have them phrase their responses in realistic terms, as if they were describing how to set up an illustrative experiment for a classroom demonstration.

%INDEX% Calculus, derivative (defined)

%(END_NOTES)


