
%(BEGIN_QUESTION)
% Copyright 2003, Tony R. Kuphaldt, released under the Creative Commons Attribution License (v 1.0)
% This means you may do almost anything with this work of mine, so long as you give me proper credit

Here is an S-R latch circuit, built from NAND gates:

$$\epsfbox{01355x01.eps}$$

Add two more NAND gates to this circuit, converting it into a {\it gated} S-R latch, with an Enable (E) input, and write the truth table for the new circuit.

\underbar{file 01355}
%(END_QUESTION)





%(BEGIN_ANSWER)

$$\epsfbox{01355x02.eps}$$

\vskip 10pt

Follow-up question: explain why the inputs to the latch circuit are not active-low as they were before the addition of the two extra NAND gates.  In other words, why does this latch now have $S$ and $R$ inputs rather than $\overline{S}$ and $\overline{R}$ inputs as it did before?

%(END_ANSWER)





%(BEGIN_NOTES)

Ask your students if they see any practical advantage to this latch circuit over a gated latch built from NOR gates.  What if they had to build a latch circuit from individual gates, rather than as a complete integrated circuit in and of itself?  Would one design be preferable over the other?

Then, ask your students to compare the truth tables of the two different types of gated latches.  Is there any difference in operation at all between the latch built with NAND gates and the latch built with NOR gates?

%(END_NOTES)


