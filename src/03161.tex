
%(BEGIN_QUESTION)
% Copyright 2005, Tony R. Kuphaldt, released under the Creative Commons Attribution License (v 1.0)
% This means you may do almost anything with this work of mine, so long as you give me proper credit

{\it Multiplexing} is a technique used in most dual-trace analog oscilloscopes.  Briefly explain \underbar{how} and \underbar{why} the two channels of a dual-trace oscilloscope are multiplexed to the display (CRT).

\underbar{file 03161}
%(END_QUESTION)





%(BEGIN_ANSWER)

Multiple input channels are "painted" one at a time on the CRT display in such a manner that it appears seamless to the human eye.  This is done either by {\it alternating} traces or {\it chopping} them (alternating back and forth many times during one sweep).  The reason for this is obvious: to get more functionality out of a single-gun CRT.

%(END_ANSWER)





%(BEGIN_NOTES)

{\bf This question is intended for exams only and not worksheets!}.

%INDEX% Multiplexing, oscilloscope display

%(END_NOTES)


