
%(BEGIN_QUESTION)
% Copyright 2003, Tony R. Kuphaldt, released under the Creative Commons Attribution License (v 1.0)
% This means you may do almost anything with this work of mine, so long as you give me proper credit

Draw the Bode plot for an {\it ideal} high-pass filter circuit:

$$\epsfbox{00618x01.eps}$$

Be sure to note the "cutoff frequency" on your plot.

\underbar{file 00618}
%(END_QUESTION)





%(BEGIN_ANSWER)

$$\epsfbox{00618x02.eps}$$

\vskip 10pt

Follow-up question: a theoretical filter with this kind of idealized response is sometimes referred to as a "brick wall" filter.  Explain why this name is appropriate.

%(END_ANSWER)





%(BEGIN_NOTES)

The plot given in the answer, of course, is for an ideal high-pass filter, where all frequencies below $f_{cutoff}$ are blocked and all frequencies above $f_{cutoff}$ are passed.  In reality, filter circuits never attain this ideal "square-edge" response.  Discuss possible applications of such a filter with your students.  

Challenge them to draw the Bode plots for ideal {\it band-pass} and {\it band-stop} filters as well.  Exercises such as this really help to clarify the purpose of filter circuits.  Otherwise, there is a tendency to lose perspective of what real filter circuits, with their correspondingly complex Bode plots and mathematical analyses, are supposed to do.

%INDEX% High-pass filter, ideal frequency response

%(END_NOTES)


