
\centerline{\bf ELTR 120 (Semiconductors 1), section 3} \bigskip 
 
\vskip 10pt

\noindent
{\bf Recommended schedule}

\vskip 5pt

%%%%%%%%%%%%%%%
\hrule \vskip 5pt
\noindent
\underbar{Day 1}

\hskip 10pt Topics: {\it Clipper, clamper, and voltage multiplier circuits}
 
\hskip 10pt Questions: {\it 1 through 10}
 
\hskip 10pt Lab Exercise: {\it Diode clipper circuit (question 51)}
 
\vskip 10pt
%%%%%%%%%%%%%%%
\hrule \vskip 5pt
\noindent
\underbar{Day 2}

\hskip 10pt Topics: {\it Thyristor devices}
 
\hskip 10pt Questions: {\it 11 through 20}
 
\hskip 10pt Lab Exercise: {\it Work on project}
 
\vskip 10pt
%%%%%%%%%%%%%%%
\hrule \vskip 5pt
\noindent
\underbar{Day 3}

\hskip 10pt Topics: {\it Thyristor power control circuits}
 
\hskip 10pt Questions: {\it 21 through 30}
 
\hskip 10pt Lab Exercise: {\it SCR latch circuit (question 52)}
 
%INSTRUCTOR \hskip 10pt {\bf Demo: show students what a solid-state relay looks like}

\vskip 10pt
%%%%%%%%%%%%%%%
\hrule \vskip 5pt
\noindent
\underbar{Day 4}

\hskip 10pt Topics: {\it Pulse-width modulation power control}
 
\hskip 10pt Questions: {\it 31 through 40}
 
\hskip 10pt Lab Exercise: {\it PWM power controller, discrete (question 53)}
 
\vskip 10pt
%%%%%%%%%%%%%%%
\hrule \vskip 5pt
\noindent
\underbar{Day 5}

\hskip 10pt Topics: {\it Switching power supply circuits}
 
\hskip 10pt Questions: {\it 41 through 50}
 
\hskip 10pt Lab Exercise: {\it Work on project}
 
%INSTRUCTOR \hskip 10pt {\bf Demo: show students a switching regulator IC datasheet}

%INSTRUCTOR \hskip 10pt {\bf MIT 8.02 video clip: Disk 4, Lecture 24; Ruhmkorff coil as a DC converter 47:45 to end}

\vskip 10pt
%%%%%%%%%%%%%%%
\hrule \vskip 5pt
\noindent
\underbar{Day 6}

\hskip 10pt Exam 3: {\it includes thyristor latch circuit performance assessment}
 
\hskip 10pt {\bf Project due}

\hskip 10pt Question 54: Sample project grading criteria
 
\vskip 10pt

%%%%%%%%%%%%%%%
\hrule \vskip 5pt
\noindent
\underbar{Troubleshooting practice problems}

\hskip 10pt Questions: {\it 55 through 64}
 
\vskip 10pt
%%%%%%%%%%%%%%%
\hrule \vskip 5pt
\noindent
\underbar{General concept practice and challenge problems}

\hskip 10pt Questions: {\it 65 through the end of the worksheet}
 
\vskip 10pt
%%%%%%%%%%%%%%%












\vfil \eject

\centerline{\bf ELTR 120 (Semiconductors 1), section 3} \bigskip 
 
\vskip 10pt

\noindent
{\bf Skill standards addressed by this course section}

\vskip 5pt

%%%%%%%%%%%%%%%
\hrule \vskip 10pt
\noindent
\underbar{EIA {\it Raising the Standard; Electronics Technician Skills for Today and Tomorrow}, June 1994}

\vskip 5pt

\medskip
\item{\bf D} {\bf Technical Skills -- Discrete Solid-State Devices}
\item{\bf D.15} Understand principles and operations of thyristor circuitry (SCR, TRIAC, DIAC, etc.).
\item{\bf D.16} Fabricate and demonstrate thyristor circuitry (SCR, TRIAC, DIAC, etc.).
\item{\bf D.17} Troubleshoot and repair thyristor circuitry (SCR, TRIAC, DIAC, etc.).
\item{\bf E} {\bf Technical Skills -- Analog Circuits}
\item{\bf E.07} Understand principles and operations of linear power supplies and filters.
\item{\bf E.08} Fabricate and demonstrate linear power supplies and filters.
\item{\bf E.09} Troubleshoot and repair linear power supplies and filters.
\item{\bf E.16} Understand principles and operations of regulated and switching power supply circuits.
\item{\bf E.29} Demonstrate an understanding of motor phase shift control circuits.
\medskip

\vskip 5pt

\medskip
\item{\bf B} {\bf Basic and Practical Skills -- Communicating on the Job}
\item{\bf B.01} Use effective written and other communication skills.  {\it Met by group discussion and completion of labwork.}
\item{\bf B.03} Employ appropriate skills for gathering and retaining information.  {\it Met by research and preparation prior to group discussion.}
\item{\bf B.04} Interpret written, graphic, and oral instructions.  {\it Met by completion of labwork.}
\item{\bf B.06} Use language appropriate to the situation.  {\it Met by group discussion and in explaining completed labwork.}
\item{\bf B.07} Participate in meetings in a positive and constructive manner.  {\it Met by group discussion.}
\item{\bf B.08} Use job-related terminology.  {\it Met by group discussion and in explaining completed labwork.}
\item{\bf B.10} Document work projects, procedures, tests, and equipment failures.  {\it Met by project construction and/or troubleshooting assessments.}
\item{\bf C} {\bf Basic and Practical Skills -- Solving Problems and Critical Thinking}
\item{\bf C.01} Identify the problem.  {\it Met by research and preparation prior to group discussion.}
\item{\bf C.03} Identify available solutions and their impact including evaluating credibility of information, and locating information.  {\it Met by research and preparation prior to group discussion.}
\item{\bf C.07} Organize personal workloads.  {\it Met by daily labwork, preparatory research, and project management.}
\item{\bf C.08} Participate in brainstorming sessions to generate new ideas and solve problems.  {\it Met by group discussion.}
\item{\bf D} {\bf Basic and Practical Skills -- Reading}
\item{\bf D.01} Read and apply various sources of technical information (e.g. manufacturer literature, codes, and regulations).  {\it Met by research and preparation prior to group discussion.}
\item{\bf E} {\bf Basic and Practical Skills -- Proficiency in Mathematics}
\item{\bf E.01} Determine if a solution is reasonable.
\item{\bf E.02} Demonstrate ability to use a simple electronic calculator.
\item{\bf E.05} Solve problems and [sic] make applications involving integers, fractions, decimals, percentages, and ratios using order of operations.
\item{\bf E.06} Translate written and/or verbal statements into mathematical expressions.
\item{\bf E.09} Read scale on measurement device(s) and make interpolations where appropriate.  {\it Met by oscilloscope usage.}
\item{\bf E.12} Interpret and use tables, charts, maps, and/or graphs.
\item{\bf E.13} Identify patterns, note trends, and/or draw conclusions from tables, charts, maps, and/or graphs.
\item{\bf E.15} Simplify and solve algebraic expressions and formulas.
\item{\bf E.16} Select and use formulas appropriately.
\item{\bf E.17} Understand and use scientific notation.
\medskip

%%%%%%%%%%%%%%%




\vfil \eject

\centerline{\bf ELTR 120 (Semiconductors 1), section 3} \bigskip 
 
\vskip 10pt

\noindent
{\bf Common areas of confusion for students}

\vskip 5pt

%%%%%%%%%%%%%%%
\hrule \vskip 5pt

\vskip 10pt

\noindent
{\bf Difficult concept: } {\it Reactive components as sources versus loads.}

The relationship between direction of current and polarity of voltage drop is simple and straightforward for resistors, because resistors always dissipate power.  That is, a resistor will always act as a load.  Things are not so simple for reactive components such as inductors and capacitors, however, because these components {\it store} and {\it release} energy while (ideally) dissipating none.  This means they sometimes act as sources and other times act as loads.  Since DC-DC converter circuits exploit the energy storage inherent to inductors and/or capacitors, this becomes a very important concept for students to grasp.


