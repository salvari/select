
%(BEGIN_QUESTION)
% Copyright 2003, Tony R. Kuphaldt, released under the Creative Commons Attribution License (v 1.0)
% This means you may do almost anything with this work of mine, so long as you give me proper credit

$$\epsfbox{01953x01.eps}$$

\underbar{file 01953}
\vfil \eject
%(END_QUESTION)





%(BEGIN_ANSWER)

Use circuit simulation software to verify your predicted and measured parameter values.

%(END_ANSWER)





%(BEGIN_NOTES)

I have had great success with the following values:

\medskip
\item{$\bullet$} +V = 7 to 24 volts
\item{$\bullet$} $C_1$ and $C_2$ = 0.001 $\mu$F
\item{$\bullet$} $C_3$, $C_4$, and $C_5$ = 0.47 $\mu$F
\item{$\bullet$} $L_1$ = 100 $\mu$H (ferrite core RF choke)
\item{$\bullet$} $R_1$ = 22 k$\Omega$
\item{$\bullet$} $R_2$ = 1.5 M$\Omega$
\item{$\bullet$} $R_3$ = 6.8 k$\Omega$
\item{$\bullet$} $R_4$ = 100 k$\Omega$
\item{$\bullet$} $Q_1$ = part number 2N3403
\item{$\bullet$} $Q_2$ = part number MPF 102
\medskip

With these component values, the carrier waveform was quite clean and the frequency was almost exactly 700 kHz:

$$f_{out} = {1 \over {2 \pi \sqrt{L C_1 C_2 \over C_1 + C_2}}}$$

Modulation isn't that great, due to the crude nature of the circuit, but it is certainly good enough to hear over an appropriately tuned AM radio.  Setting $V_{signal}$ and $f_{signal}$ is a matter of experimentation, to achieve the desired degree of modulation and tone pitch.

An extension of this exercise is to incorporate troubleshooting questions.  Whether using this exercise as a performance assessment or simply as a concept-building lab, you might want to follow up your students' results by asking them to predict the consequences of certain circuit faults.

%INDEX% Assessment, performance-based (AM radio transmitter)

%(END_NOTES)


