
%(BEGIN_QUESTION)
% Copyright 2005, Tony R. Kuphaldt, released under the Creative Commons Attribution License (v 1.0)
% This means you may do almost anything with this work of mine, so long as you give me proper credit

$$\epsfbox{03113x01.eps}$$

Identify which trigonometric functions (sine, cosine, or tangent) are represented by each of the following ratios, with reference to the angle labeled with the Greek letter "Phi" ($\phi$):

$${R \over X} = $$

$${X \over Z} = $$

$${R \over Z} = $$

\underbar{file 03113}
%(END_QUESTION)





%(BEGIN_ANSWER)

$$\epsfbox{03113x01.eps}$$

$${R \over X} = \tan \phi = {\hbox{Opposite} \over \hbox{Adjacent}}$$

$${X \over Z} = \cos \phi = {\hbox{Adjacent} \over \hbox{Hypotenuse}}$$

$${R \over Z} = \sin \phi = {\hbox{Opposite} \over \hbox{Hypotenuse}}$$

%(END_ANSWER)





%(BEGIN_NOTES)

Ask your students to explain what the words "hypotenuse", "opposite", and "adjacent" refer to in a right triangle.

%INDEX% Trigonometry, cosine function defined in impedance triangle
%INDEX% Trigonometry, sine function defined in impedance triangle
%INDEX% Trigonometry, tangent function defined in impedance triangle

%(END_NOTES)


