
\centerline{\bf Design Project: Logic probe} \bigskip 
 
This worksheet and all related files are licensed under the Creative Commons Attribution License, version 1.0.  To view a copy of this license, visit http://creativecommons.org/licenses/by/1.0/, or send a letter to Creative Commons, 559 Nathan Abbott Way, Stanford, California 94305, USA.  The terms and conditions of this license allow for free copying, distribution, and/or modification of all licensed works by the general public.

\bigskip 

\hrule

\vskip 10pt

Your project is to design and build a simple logic probe, capable of displaying "high," "low," and "indeterminate" logic states for any family of logic (TTL, CMOS, ECL, etc.).  Here is a sample schematic diagram for you to follow when designing your system:

% Sample schematic diagram here
$$\epsfbox{proj_lpb.eps}$$

\medskip
\goodbreak
Suggested components:
\item{$\bullet$} $R_1$ = $R_2$ = 1 M$\Omega$
\item{$\bullet$} $R_3$ = $R_4$ = 1 k$\Omega$
\item{$\bullet$} $R_{pot1}$ = $R_{pot2}$ = 10 k$\Omega$
\item{$\bullet$} $C_1$ = 10 $\mu$F
\item{$\bullet$} $D_1$ = $D_2$ = 1N4148 switching diodes
\item{$\bullet$} $U_1$ = $U_2$ = LM339 comparator
\item{$\bullet$} $U_3$ = One gate in 4011 (CMOS) quad NAND integrated circuit
\medskip

Of course, you are not restricted to using this exact design.  Another neat feature to add to this circuit would be a "pulse" detection LED that lights up only when the input (probe voltage) changes state.

\vskip 10pt

\noindent
Deadlines (set by instructor):

\medskip
\item{$\bullet$} Project design completed: 
\item{$\bullet$} Components purchased:
\item{$\bullet$} Working prototype:
\item{$\bullet$} Finished system:
\item{$\bullet$} Full documentation:
\medskip



