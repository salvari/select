
%(BEGIN_QUESTION)
% Copyright 2003, Tony R. Kuphaldt, released under the Creative Commons Attribution License (v 1.0)
% This means you may do almost anything with this work of mine, so long as you give me proper credit

Ideally, should a voltmeter have a very low input resistance, or a very high input resistance?  (Input resistance being the amount of electrical resistance intrinsic to the meter, as measured between its test leads.)  Explain your answer.

\underbar{file 00725}
%(END_QUESTION)





%(BEGIN_ANSWER)

Ideally, a voltmeter should have the greatest amount of input resistance possible.  This is important when using it to measure voltage sources and voltage drops in circuits containing large amounts of resistance.

%(END_ANSWER)





%(BEGIN_NOTES)

The answer to this question is related to the very important principle of {\it meter loading}.  Technicians, especially, have to be very aware of meter loading, and how erroneous measurements may result from it.  The answer is also related to how voltmeters are connected with the circuits under test: {\it always in parallel!}

%INDEX% Voltmeter, input resistance of

%(END_NOTES)


