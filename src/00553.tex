
%(BEGIN_QUESTION)
% Copyright 2003, Tony R. Kuphaldt, released under the Creative Commons Attribution License (v 1.0)
% This means you may do almost anything with this work of mine, so long as you give me proper credit

How is it possible to electrically measure the torque output by a permanent-magnet DC motor?  Hint: it is very simple, and for large electric motors it involves the use of a {\it shunt resistor}.  Modify this circuit diagram to include a meter that provides indirect indication of motor torque:

$$\epsfbox{00553x01.eps}$$

\underbar{file 00553}
%(END_QUESTION)





%(BEGIN_ANSWER)

$$\epsfbox{00553x02.eps}$$

%(END_ANSWER)





%(BEGIN_NOTES)

If some of your students think the "V" symbol in the meter means it is measuring motor {\it voltage}, they need to review the purpose and function of a shunt resistor!

This method of measuring motor torque is accurate, so long as the motor is in good condition.  Ask your students what they think the meter would indicate if the motor began to develop a low-resistance fault due to carbon dust buildup from the brushes shorting some of the armature current.  Would the meter indicate falsely low, falsely high, or would it still accurately register motor torque?

%INDEX% DC electric motor, speed and torque

%(END_NOTES)


