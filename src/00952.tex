
%(BEGIN_QUESTION)
% Copyright 2003, Tony R. Kuphaldt, released under the Creative Commons Attribution License (v 1.0)
% This means you may do almost anything with this work of mine, so long as you give me proper credit

An important parameter of transistor amplifier circuits is the {\it Q point}, or {\it quiescent operating point}.  The "Q point" of a transistor amplifier circuit will be a single point somewhere along its load line.

Describe what the "Q point" actually means for a transistor amplifier circuit, and how its value may be altered.

\underbar{file 00952}
%(END_QUESTION)





%(BEGIN_ANSWER)

The "Q point" for a transistor amplifier circuit is the point along its operating region in a "quiescent" condition: when there is no input signal being amplified.

%(END_ANSWER)





%(BEGIN_NOTES)

Q points are very important in the design process of transistor amplifiers, but again students often seem to fail to grasp the actual meaning of the concept.  Ask your students to explain how the load line formed by the load resistance, and characteristic curves of the transistor, describe all the possible operating conditions of collector current and $V_{CE}$ for that amplifier circuit.  Then discuss how the status of that circuit is defined {\it at any single point in time} along those graphs (by a line, a curve, or a point?).

%INDEX% Q point, defined for transistor amplifier circuit

%(END_NOTES)


