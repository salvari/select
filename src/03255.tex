
%(BEGIN_QUESTION)
% Copyright 2005, Tony R. Kuphaldt, released under the Creative Commons Attribution License (v 1.0)
% This means you may do almost anything with this work of mine, so long as you give me proper credit

The efficiency ($\eta$) of a simple power system with losses occurring over the wires is a function of circuit current, wire resistance, and total source power:

$$\epsfbox{03255x01.eps}$$

A simple formula for calculating efficiency is given here:

$$\eta = {{P_{source} - I^2R} \over P_{source}}$$

\noindent
Where,

$P_{source}$ = the power output by the voltage source, in watts (W)

$I$ = the circuit current, in amperes (A)

$R$ = the {\it total} wire resistance ($R_{wire1}$ + $R_{wire2}$), in ohms ($\Omega$)

\vskip 10pt

Algebraically manipulate this equation to solve for wire resistance ($R$) in terms of all the other variables, and then calculate the maximum amount of allowable wire resistance for a power system where a source outputting 200 kW operates at a circuit current of 48 amps, at a minimum efficiency of 90\%.

\underbar{file 03255}
%(END_QUESTION)





%(BEGIN_ANSWER)

$$R = {{P_{source} - \eta P_{source}} \over I^2}$$

The maximum allowable (total) wire resistance is 8.681 $\Omega$.

%(END_ANSWER)





%(BEGIN_NOTES)

A common mistake for students to make here is entering 90\% as "90" rather than as "0.9" in their calculators.

%INDEX% Efficiency of power lines
%INDEX% Power line resistance
%INDEX% Resistance of power line conductors

%(END_NOTES)


