
%(BEGIN_QUESTION)
% Copyright 2003, Tony R. Kuphaldt, released under the Creative Commons Attribution License (v 1.0)
% This means you may do almost anything with this work of mine, so long as you give me proper credit

When designing a project such as this, it is a good idea to "prototype" the circuit before soldering components together, where they will be more difficult to replace, exchange, and reconfigure.  Explain how you intend to prototype your circuit prior to assembling it in its final form.

\underbar{file 01517}
%(END_QUESTION)





%(BEGIN_ANSWER)

I'll let you determine the answer to this question!

%(END_ANSWER)





%(BEGIN_NOTES)

Students are typically loathe to prototype {\it anything} for fear of "wasting time" when they could be building something in its final form.  What most students fail to realize, though, is that they will waste far more time by re-doing their "final" build than they would have spent prototyping the circuit in less permanent form.

Prototyping is an essential part of the design and build process.  You may elect to hold your students accountable to this process as part of their grade, or let them choose not to prototype (and allow them to discover firsthand why they should have prototyped their circuit).

%(END_NOTES)


