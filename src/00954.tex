
%(BEGIN_QUESTION)
% Copyright 2003, Tony R. Kuphaldt, released under the Creative Commons Attribution License (v 1.0)
% This means you may do almost anything with this work of mine, so long as you give me proper credit

{\it Load lines} are useful tools for analyzing transistor amplifier circuits, but they may be applied to other types of circuits as well.  Take for instance this diode-resistor circuit:

$$\epsfbox{00954x01.eps}$$

The diode's characteristic curve is already plotted on the following graph.  Your task is to plot the load line for the circuit on the same graph, and note where the two lines intersect:

$$\epsfbox{00954x02.eps}$$

What is the practical significance of these two plots' intersection?

\underbar{file 00954}
%(END_QUESTION)





%(BEGIN_ANSWER)

The two lines intersect at a current of approximately 1.72 mA:

$$\epsfbox{00954x03.eps}$$

\vskip 10pt

Follow-up question: explain why the use of a load line greatly simplifies the determination of circuit current in such a diode-resistor circuit.

\vskip 10pt

Challenge question: suppose the resistor value were increased from 2.5 k$\Omega$ to 10 k$\Omega$.  What difference would this make in the load line plot, and in the intersection point between the two plots?

%(END_ANSWER)





%(BEGIN_NOTES)

While this approach to circuit analysis may seem silly -- using load lines to calculate the current in a diode-resistor circuit -- it demonstrates the principle of load lines in a context that should be obvious to students at this point in their study.  Discuss with your students how the load line is obtained for this circuit, and why it is straight while the diode's characteristic curve is not.  

Also, discuss the significance of the two line intersecting.  Mathematically, what does the intersection of two graphs mean?  What do the coordinate values of the intersection point represent in a system of simultaneous functions?  How does this principle relate to an electronic circuit?

%INDEX% Load line, concept illustrated with diode-resistor circuit

%(END_NOTES)


