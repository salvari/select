
%(BEGIN_QUESTION)
% Copyright 2005, Tony R. Kuphaldt, released under the Creative Commons Attribution License (v 1.0)
% This means you may do almost anything with this work of mine, so long as you give me proper credit

How does the operation of this difference amplifier circuit compare with the resistor values given ($2R$ = twice the resistance of $R$), versus its operation with all resistor values equal?

$$\epsfbox{02525x01.eps}$$

Describe what approach or technique you used to derive your answer, and also explain how your conclusion for this circuit might be generalized for all difference amplifier circuits.

\underbar{file 02525}
%(END_QUESTION)





%(BEGIN_ANSWER)

It is very important that you develop the skill of "exploring" a circuit configuration to see what it will do, rather than having to be told what it does (either by your instructor or by a book).  All you need to have is a solid knowledge of basic electrical principles (Ohm's Law, Kirchhoff's Voltage and Current Laws) and know how opamps behave when configured for negative feedback.

\vskip 10pt

As for a generalized conclusion:

$$\epsfbox{02525x02.eps}$$

%(END_ANSWER)





%(BEGIN_NOTES)

It is easy for you (the instructor) to show how and why this circuit acts as it does.  The point of this question, however, is to get students to take the initiative to explore the circuit on their own.  It is simple enough for any student to set up some hypothetical test conditions (a {\it thought experiment}) to analyze what this circuit will do, that the only thing holding them back from doing so is attitude, not aptitude.  

This is something I have noticed over years of teaching: so many students who are more than capable of doing the math and applying well-understood electrical rules refuse to do so {\it on their own}, because years of educational tradition has indoctrinated them to wait for the instructor's lead rather than explore a concept on their own.

%INDEX% Algebra, deriving equations from data
%INDEX% Difference amplifier (with non-unity gain), opamp
%INDEX% Subtractor circuit (with non-unity gain), opamp

%(END_NOTES)


