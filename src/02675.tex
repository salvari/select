
%(BEGIN_QUESTION)
% Copyright 2005, Tony R. Kuphaldt, released under the Creative Commons Attribution License (v 1.0)
% This means you may do almost anything with this work of mine, so long as you give me proper credit

Most operational amplifiers do not have the ability to swing their output voltages rail-to-rail.  Most of those do not swing their output voltages symmetrically.  That is, a typical non-rail-to-rail opamp may be able to approach one power supply rail voltage closer than the other; e.g. when powered by a +15/-15 volt split supply, the output saturates positive at +14 volts and saturates negative at -13.5 volts.

What effect do you suppose this non-symmetrical output range will have on a typical relaxation oscillator circuit such as the following, and how might you suggest we fix the problem?

$$\epsfbox{02675x01.eps}$$

\underbar{file 02675}
%(END_QUESTION)





%(BEGIN_ANSWER)

The duty cycle will not be 50\%.  One way to fix the problem is to do something like this:

$$\epsfbox{02675x02.eps}$$

\vskip 10pt

Follow-up question: explain how and why {\it this} solution works.  Now you just {\it knew} I was going to ask this question the moment you saw the diagram, didn't you?

%(END_ANSWER)





%(BEGIN_NOTES)

Note that I added an additional resistor to the circuit, in series with the opamp output terminal.  In some cases this is not necessary because the opamp is self-limiting in output current, but it is a good design practice nonetheless.  In the event anyone ever swaps out the original opamp for a different model lacking overcurrent protection, the new opamp will not become damaged.

%INDEX% Opamp, rail-to-rail output swing
%INDEX% Relaxation oscillator, opamp (without rail-to-rail output capability)

%(END_NOTES)


