
\centerline{\bf ELTR 115 (AC 2), section 2} \bigskip 
 
\vskip 10pt

\noindent
{\bf Recommended schedule}

\vskip 5pt

%%%%%%%%%%%%%%%
\hrule \vskip 5pt
\noindent
\underbar{Day 1}

\hskip 10pt Topics: {\it Power in AC circuits}
 
\hskip 10pt Questions: {\it 1 through 20}
 
\hskip 10pt Lab Exercise: {\it Lissajous figures for phase shift measurement (question 71)}
 
\vskip 10pt
%%%%%%%%%%%%%%%
\hrule \vskip 5pt
\noindent
\underbar{Day 2}

\hskip 10pt Topics: {\it Power factor correction}
 
\hskip 10pt Questions: {\it 21 through 40}
 
\hskip 10pt Lab Exercise: {\it Power factor correction for AC motor (question 72)}
 
%INSTRUCTOR \hskip 10pt {\bf Demo: show photos of substation power factor correction equipment}

\vskip 10pt
%%%%%%%%%%%%%%%
\hrule \vskip 5pt
\noindent
\underbar{Day 3}

\hskip 10pt Topics: {\it Alternator construction and introduction to polyphase AC}
 
\hskip 10pt Questions: {\it 41 through 55}
 
\hskip 10pt Lab Exercise: {\it Power factor correction for AC motor (question 72, continued)}
 
%INSTRUCTOR \hskip 10pt {\bf Demo: show disassembled automotive alternator}

%INSTRUCTOR \hskip 10pt {\bf Demo: operate 3-phase motor/generator unit}

\vskip 10pt
%%%%%%%%%%%%%%%
\hrule \vskip 5pt
\noindent
\underbar{Day 4}

\hskip 10pt Topics: {\it AC motor construction and polyphase AC circuits}
 
\hskip 10pt Questions: {\it 56 through 70}
 
\hskip 10pt Lab Exercise: {\it work on project}
 
%INSTRUCTOR \hskip 10pt {\bf MIT 8.02 video clip: Disk 3, Lecture 18; 3-phase motor demo 33:35 to 37:13}

%INSTRUCTOR \hskip 10pt {\bf Socratic Electronics animation: Three-phase motor}

%INSTRUCTOR \hskip 10pt {\bf Demo: operate 3-phase motor/generator unit}

%INSTRUCTOR \hskip 10pt {\bf MIT 8.02 video clip: Disk 3, Lecture 18; 1-phase motor demo 49:07 to end}

\vskip 10pt
%%%%%%%%%%%%%%%
\hrule \vskip 5pt
\noindent
\underbar{Day 5}

\hskip 10pt Exam 2: {\it includes Lissajous figure phase shift measurement performance assessment}
 
\vskip 10pt
%%%%%%%%%%%%%%%
\hrule \vskip 5pt
\noindent
\underbar{Practice and challenge problems}

\hskip 10pt Questions: {\it 74 through the end of the worksheet}
 
\vskip 10pt
%%%%%%%%%%%%%%%
\hrule \vskip 5pt
\noindent
\underbar{Impending deadlines}

\hskip 10pt {\bf Project due at end of ELTR115, Section 3}
 
\hskip 10pt Question 73: Sample project grading criteria
 
\vskip 10pt
%%%%%%%%%%%%%%%











\vfil \eject

\centerline{\bf ELTR 115 (AC 2), section 2} \bigskip 
 
\vskip 10pt

\noindent
{\bf Skill standards addressed by this course section}

\vskip 5pt

%%%%%%%%%%%%%%%
\hrule \vskip 10pt
\noindent
\underbar{EIA {\it Raising the Standard; Electronics Technician Skills for Today and Tomorrow}, June 1994}

\vskip 5pt

\medskip
\item{\bf C} {\bf Technical Skills -- AC circuits}
\item{\bf C.01} Demonstrate an understanding of sources of electricity in AC circuits.
\item{\bf C.04} Demonstrate an understanding of basic motor/generator theory and operation.
\item{\bf C.05} Demonstrate an understanding of measurement of power in AC circuits.
\item{\bf C.30} Understand principles and operations of AC polyphase circuits.
\medskip

\vskip 5pt

\medskip
\item{\bf B} {\bf Basic and Practical Skills -- Communicating on the Job}
\item{\bf B.01} Use effective written and other communication skills.  {\it Met by group discussion and completion of labwork.}
\item{\bf B.03} Employ appropriate skills for gathering and retaining information.  {\it Met by research and preparation prior to group discussion.}
\item{\bf B.04} Interpret written, graphic, and oral instructions.  {\it Met by completion of labwork.}
\item{\bf B.06} Use language appropriate to the situation.  {\it Met by group discussion and in explaining completed labwork.}
\item{\bf B.07} Participate in meetings in a positive and constructive manner.  {\it Met by group discussion.}
\item{\bf B.08} Use job-related terminology.  {\it Met by group discussion and in explaining completed labwork.}
\item{\bf B.10} Document work projects, procedures, tests, and equipment failures.  {\it Met by project construction and/or troubleshooting assessments.}
\item{\bf C} {\bf Basic and Practical Skills -- Solving Problems and Critical Thinking}
\item{\bf C.01} Identify the problem.  {\it Met by research and preparation prior to group discussion.}
\item{\bf C.03} Identify available solutions and their impact including evaluating credibility of information, and locating information.  {\it Met by research and preparation prior to group discussion.}
\item{\bf C.07} Organize personal workloads.  {\it Met by daily labwork, preparatory research, and project management.}
\item{\bf C.08} Participate in brainstorming sessions to generate new ideas and solve problems.  {\it Met by group discussion.}
\item{\bf D} {\bf Basic and Practical Skills -- Reading}
\item{\bf D.01} Read and apply various sources of technical information (e.g. manufacturer literature, codes, and regulations).  {\it Met by research and preparation prior to group discussion.}
\item{\bf E} {\bf Basic and Practical Skills -- Proficiency in Mathematics}
\item{\bf E.01} Determine if a solution is reasonable.
\item{\bf E.02} Demonstrate ability to use a simple electronic calculator.
\item{\bf E.05} Solve problems and [sic] make applications involving integers, fractions, decimals, percentages, and ratios using order of operations.
\item{\bf E.06} Translate written and/or verbal statements into mathematical expressions.
\item{\bf E.09} Read scale on measurement device(s) and make interpolations where appropriate.  {\it Met by oscilloscope usage.}
\item{\bf E.12} Interpret and use tables, charts, maps, and/or graphs.
\item{\bf E.13} Identify patterns, note trends, and/or draw conclusions from tables, charts, maps, and/or graphs.
\item{\bf E.15} Simplify and solve algebraic expressions and formulas.
\item{\bf E.16} Select and use formulas appropriately.
\item{\bf E.17} Understand and use scientific notation.
\item{\bf E.26} Apply Pythagorean theorem.
\item{\bf E.27} Identify basic functions of sine, cosine, and tangent.
\item{\bf E.28} Compute and solve problems using basic trigonometric functions.
\medskip

\vskip 5pt

\medskip
\item{\bf F} {\bf Additional Skills -- Electromechanics}
\item{\bf B.01e} Types of motors.
\medskip

%%%%%%%%%%%%%%%






\vfil \eject

\centerline{\bf ELTR 115 (AC 2), section 2} \bigskip 
 
\vskip 10pt

\noindent
{\bf Common areas of confusion for students}

\vskip 5pt

%%%%%%%%%%%%%%%
\hrule \vskip 5pt

\vskip 10pt

\noindent
{\bf Difficult concept: } {\it Power factor.}

The very idea of such a thing as "imaginary power" (reactive power) is hard to grasp.  Basically, what we're talking about here is current in an AC circuit that is not contributing to work being done because it is out of phase with the voltage waveform.  Instead of contributing to useful work (energy leaving the circuit), it merely stores and releases energy from reactive components.  In a purely reactive circuit, {\it all} power in the circuit is "imaginary" and does no useful work.  In a purely resistive circuit, {\it all} power is "real" and is dissipated from the circuit by the resistance (this is also true of motor circuits operating at 100\% efficiency, where energy leaves not in the form of heat but in the form of mechanical work).  In all realistic circuits, power is some combination of "real" and "imaginary;" true and reactive.

\vskip 10pt

\noindent
{\bf Difficult concept: } {\it Measuring phase shift with an oscilloscope.}

Phase shift is not as easy to measure with an oscilloscope as is amplitude or frequency, and it takes practice to learn.  Some students have a tendency to look for memorizable formulae or step-by-step procedures for doing this rather than to figure out how and why it works.  As I am fond of telling my students, {\it memory will fail you!  Understand, don't memorize!}  Phase shift measurement is not as difficult as it may look.  Once you figure out the relationship between horizontal divisions on the oscilloscope screen and the period (360$^{o}$) of the waveforms, figuring phase shift is merely a matter of ratio: $x$ divisions of shift is to the number of divisions per cycle as $y$ degrees of shift is to 360$^{o}$.

\vskip 10pt

\noindent
{\bf Difficult concept: } {\it Polyphase electric power.}

Electrical systems having more than one phase ({\it poly-}phase) are tremendously useful and prevalent in modern industry.  The idea of having multiple voltages and currents in a system all out of phase with each other may seem a little weird and confusing, but it is very necessary to know.  Perhaps the best way to grasp what is going on in these systems is to see a video animation of three-phase power being generated, or a three-phase motor in action.  As I teacher, I like to use blinking Christmas lights or the motion of crowds in grandstands sequentially waving ("The Wave") as an illustration of the large-scale "motion" one may create with out-of-phase sequences.

When it comes to mathematically analyzing what is happening in polyphase systems, phasor diagrams are most useful.  Try to apply a phasor diagram to polyphase problems which you know the solution(s) to, and then see how the same types of diagrams may apply to problems you are currently trying to solve.




