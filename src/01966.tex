
%(BEGIN_QUESTION)
% Copyright 2005, Tony R. Kuphaldt, released under the Creative Commons Attribution License (v 1.0)
% This means you may do almost anything with this work of mine, so long as you give me proper credit

$$\epsfbox{01966x01.eps}$$

\underbar{file 01966}
\vfil \eject
%(END_QUESTION)





%(BEGIN_ANSWER)

Use circuit simulation software to verify your predicted and measured parameter values.

%(END_ANSWER)





%(BEGIN_NOTES)

Use a variable-voltage, regulated power supply to supply any amount of DC voltage below 30 volts.  Specify standard resistor values, all between 1 k$\Omega$ and 100 k$\Omega$ (1k5, 2k2, 2k7, 3k3, 4k7, 5k1, 6k8, 10k, 22k, 33k, 39k 47k, 68k, etc.).  Use a sine-wave function generator to supply an audio-frequency input signal, and make sure its amplitude isn't set so high that the amplifier clips.

I have had good success using the following values:

\medskip
\item{$\bullet$} $V_{CC}$ = 9 volts
\item{$\bullet$} $V_{in}$ = audio-frequency signal, 0.5 volt peak-to-peak
\item{$\bullet$} $R_1$ = 220 k$\Omega$
\item{$\bullet$} $R_2$ = 27 k$\Omega$
\item{$\bullet$} $R_C$ = 10 k$\Omega$
\item{$\bullet$} $R_E$ = 1.5 k$\Omega$
\item{$\bullet$} $C_1$ = 10 $\mu$F
\medskip

An extension of this exercise is to incorporate troubleshooting questions.  Whether using this exercise as a performance assessment or simply as a concept-building lab, you might want to follow up your students' results by asking them to predict the consequences of certain circuit faults.

%INDEX% Assessment, performance-based (Common-emitter class-A amplifier circuit)

%(END_NOTES)


