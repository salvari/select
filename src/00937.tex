
%(BEGIN_QUESTION)
% Copyright 2003, Tony R. Kuphaldt, released under the Creative Commons Attribution License (v 1.0)
% This means you may do almost anything with this work of mine, so long as you give me proper credit

Differential amplifiers often make use of {\it active loads}: a current mirror circuit to establish collector currents between the two transistors, rather than load resistors.

$$\epsfbox{00937x01.eps}$$

What does the current mirror "look like" to the common-emitter side of the differential amplifier circuit, when we apply the Superposition theorem?  What aspect of the differential amplifier's performance is primarily enhanced with the addition of the current mirror to the circuit?

\underbar{file 00937}
%(END_QUESTION)





%(BEGIN_ANSWER)

Adding a current mirror greatly increases the effective resistance on the collector terminal of the common-emitter side, and so greatly increases the amplifier's differential voltage gain.

%(END_ANSWER)





%(BEGIN_NOTES)

Ask your students to explain why the differential voltage gain increases.  One hint is the internal (Norton) resistance of an ideal current source: {\it infinite} ohms!  Ask your students how this equivalent resistance compares to the (finite) values of the resistors replaced by the current mirror, and what impact that change has on voltage gain.

%INDEX% Differential pair circuit, with active loads

%(END_NOTES)


