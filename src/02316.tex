
%(BEGIN_QUESTION)
% Copyright 2005, Tony R. Kuphaldt, released under the Creative Commons Attribution License (v 1.0)
% This means you may do almost anything with this work of mine, so long as you give me proper credit

Qualitatively determine what will happen to the load current and the zener diode current in this voltage regulator circuit if the source voltage suddenly {\it increases}.  Assume that the zener diode's behavior is ideal; i.e. its voltage drop holds absolutely constant throughout its operating range.

$$\epsfbox{02316x01.eps}$$

$I_{load}$ = ({\it increase, decrease, or unchanged?})

\vskip 5pt

$I_{zener}$ = ({\it increase, decrease, or unchanged?})

\vskip 5pt

\underbar{file 02316}
%(END_QUESTION)





%(BEGIN_ANSWER)

If the source voltage increases, $I_{zener}$ will increase and $I_{load}$ will remain unchanged.

\vskip 10pt

Challenge question: what do you think will happen with a {\it real} zener diode, where its voltage drop does change slightly with changes in current?

%(END_ANSWER)





%(BEGIN_NOTES)

A conceptual understanding of zener diode regulator circuits is important, perhaps even more important than a quantitative understanding.  Your students will need to understand what happens to the different variables in such a circuit when another parameter changes, in order to understand how these circuits will dynamically react to changing load or source conditions.

%INDEX% Zener diode, qualitative current estimations
%INDEX% Zener diode as a voltage regulator

%(END_NOTES)


