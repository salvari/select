
%(BEGIN_QUESTION)
% Copyright 2003, Tony R. Kuphaldt, released under the Creative Commons Attribution License (v 1.0)
% This means you may do almost anything with this work of mine, so long as you give me proper credit

Determine the number of time constants ($\tau$) that 7.5 seconds is equal to in each of the following reactive circuits:

\medskip
\item{$\bullet$} RC circuit; R = 10 k$\Omega$, C = 220 $\mu$F ; 7.5 sec = 
\item{$\bullet$} RC circuit; R = 33 k$\Omega$, C = 470 $\mu$F ; 7.5 sec =
\item{$\bullet$} RC circuit; R = 1.5 k$\Omega$, C = 100 $\mu$F ; 7.5 sec =
\item{$\bullet$} RC circuit; R = 790 $\Omega$, C = 9240 nF ; 7.5 sec =
\item{$\bullet$} RC circuit; R = 100 k$\Omega$, C = 33 pF ; 7.5 sec =

\vskip 10pt

\item{$\bullet$} LR circuit; R = 100 $\Omega$, L = 50 mH ; 7.5 sec =
\item{$\bullet$} LR circuit; R = 45 $\Omega$, L = 2.2 H ; 7.5 sec =
\item{$\bullet$} LR circuit; R = 1 k$\Omega$, L = 725 mH ; 7.5 sec =
\item{$\bullet$} LR circuit; R = 4.7 k$\Omega$, L = 325 mH ; 7.5 sec =
\item{$\bullet$} LR circuit; R = 6.2 $\Omega$, L = 25 H ; 7.5 sec =
\medskip

\underbar{file 01802}
%(END_QUESTION)





%(BEGIN_ANSWER)

\item{$\bullet$} RC circuit; R = 10 k$\Omega$, C = 220 $\mu$F ; 7.5 sec = 3.41 $\tau$
\item{$\bullet$} RC circuit; R = 33 k$\Omega$, C = 470 $\mu$F ; 7.5 sec = 0.484 $\tau$ 
\item{$\bullet$} RC circuit; R = 1.5 k$\Omega$, C = 100 $\mu$F ; 7.5 sec = 50.0 $\tau$
\item{$\bullet$} RC circuit; R = 790 $\Omega$, C = 9240 nF ; 7.5 sec = 1027 $\tau$
\item{$\bullet$} RC circuit; R = 100 k$\Omega$, C = 33 pF ; 7.5 sec = 2,272,727 $\tau$

\vskip 10pt

\item{$\bullet$} LR circuit; R = 100 $\Omega$, L = 50 mH ; 7.5 sec = 15,000 $\tau$
\item{$\bullet$} LR circuit; R = 45 $\Omega$, L = 2.2 H ; 7.5 sec = 153.4 $\tau$
\item{$\bullet$} LR circuit; R = 1 k$\Omega$, L = 725 mH ; 7.5 sec = 10,345 $\tau$
\item{$\bullet$} LR circuit; R = 4.7 k$\Omega$, L = 325 mH ; 7.5 sec = 108,462 $\tau$
\item{$\bullet$} LR circuit; R = 6.2 $\Omega$, L = 25 H ; 7.5 sec = 1.86 $\tau$

%(END_ANSWER)





%(BEGIN_NOTES)

An interesting thing to note here is the span of time constant values available from common capacitor/inductor/resistor sizes.  As students should notice, the capacitor-resistor combinations (all very practical values, I might add) create both longer and shorter time constant values than the inductor-resistor combinations, and that is even including the 25 Henry - 6.2 Ohm combination, which would be difficult (read: expensive) to achieve in real life.

%INDEX% Time constant calculation, RC or LR circuit

%(END_NOTES)


