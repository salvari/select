
%(BEGIN_QUESTION)
% Copyright 2003, Tony R. Kuphaldt, released under the Creative Commons Attribution License (v 1.0)
% This means you may do almost anything with this work of mine, so long as you give me proper credit

A common adage heard among instrument designers is the phrase, {\it ``Don't make it accurate; make it adjustable.''}  Modify the example meter design to include independent adjustment for each of the voltage ranges.

\underbar{file 01501}
%(END_QUESTION)





%(BEGIN_ANSWER)

This is easy to do with potentiometers, connected as variable resistors.  The real challenge, though, is to determine how to use available potentiometers to give adjustment ranges that are not too large or too small.  In other words, you want your potentiometers to have fine enough adjustment to allow you to precisely "dial in" a calibrated range, yet not so fine that you "run out of adjustment" if your calculations happen to be off.

%(END_ANSWER)





%(BEGIN_NOTES)

Some students may suggest using nothing but potentiometers (no fixed resistors) in their voltmeter circuits.  While this is theoretically possible, it ends up being impractical due to the adjustments being too coarse.  This becomes an excellent exercise in "bracketing" a potentiometer's range through the judicious connection of fixed-value resistors in the circuit.

Something students might not be aware of when they begin this project is multi-turn potentiometers.  If ever there was an application for them, this is it!

%(END_NOTES)


