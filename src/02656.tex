
%(BEGIN_QUESTION)
% Copyright 2005, Tony R. Kuphaldt, released under the Creative Commons Attribution License (v 1.0)
% This means you may do almost anything with this work of mine, so long as you give me proper credit

The purpose of a {\it current mirror} circuit is to maintain constant current through a load despite changes in that load's resistance:

$$\epsfbox{02656x01.eps}$$

If we were to crudely model the transistor's behavior as an automatically-varied rheostat -- constantly adjusting resistance as necessary to keep load current constant -- how would you describe this rheostat's response to changes in load resistance?

$$\epsfbox{02656x02.eps}$$

In other words, as $R_{load}$ increases, what does $R_{transistor}$ do -- increase resistance, decrease resistance, or remain the same resistance it was before?  How does the changing value of $R_{transistor}$ affect total circuit resistance?

\underbar{file 02656}
%(END_QUESTION)





%(BEGIN_ANSWER)

As $R_{load}$ increases, $R_{transistor}$ will decrease in resistance so as to maintain a constant current through the load and a constant $R_{total}$.

%(END_ANSWER)





%(BEGIN_NOTES)

This model of current mirror transistor behavior, albeit crude, serves as a good introduction to the subject of {\it active loads} in transistor amplifier circuits.  This is where a transistor is configured to operate as a constant-current regulator, then placed in series with an amplifying transistor to yield much greater voltage gains than what is possible with a passive (fixed resistor) load.

%INDEX% Current mirror circuit, BJT

%(END_NOTES)


