
%(BEGIN_QUESTION)
% Copyright 2003, Tony R. Kuphaldt, released under the Creative Commons Attribution License (v 1.0)
% This means you may do almost anything with this work of mine, so long as you give me proper credit

A style of counter circuit that completely circumvents the "ripple" effect is called the {\it synchronous} counter:

$$\epsfbox{01396x01.eps}$$

Complete a timing diagram for this circuit, and explain why this design of counter does not exhibit "ripple" on its output lines:

$$\epsfbox{01396x02.eps}$$

Challenge question: to {\it really} understand this type of counter circuit well, include propagation delays in your timing diagram.

\underbar{file 01396}
%(END_QUESTION)





%(BEGIN_ANSWER)

The timing diagram shown here is ideal, with no propagation delays shown:

$$\epsfbox{01396x03.eps}$$

However, even with propagation delays included (equal delays for each flip-flop), you should find there is still no "ripple" effect in the output count.

%(END_ANSWER)





%(BEGIN_NOTES)

"Walk" through the timing diagram given in the answer, and have students explain how the logic states correspond to a two-bit binary counting sequence.

%INDEX% Counter, synchronous

%(END_NOTES)


