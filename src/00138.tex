
%(BEGIN_QUESTION)
% Copyright 2003, Tony R. Kuphaldt, released under the Creative Commons Attribution License (v 1.0)
% This means you may do almost anything with this work of mine, so long as you give me proper credit

Many students first learning about atomic structure and electricity notice a paradox with respect to the location of all the protons in an atom: despite their electrical charge, they are tightly bound together in a "core" called the {\it nucleus}.  Based on what you know about electrical charges, explain why this is a paradox, and also what the solution to the paradox is.

\underbar{file 00138}
%(END_QUESTION)





%(BEGIN_ANSWER)

Paradox: since similar charges tend to physically repel each other, why should protons (all positively charged) cling so tightly together in the nucleus of an atom?  Why don't they fly apart from each other due to electrical repulsion?

I'll let you research the answer to this paradox yourself!

%(END_ANSWER)





%(BEGIN_NOTES)

Believe it or not, I once read a religious tract that used this paradox as proof of God's existence.  The argument went like this: everyone knows that like charges repel, so it must be God who holds the protons together!  Volumes could be written on the psychology behind this argument, but I digress . . .

The solution to this paradox is the subject of freshman-level college physics.  For those of you who don't know the answer, I'll give you a hint: it has something to do with the power of nuclear fission.

%INDEX% Charges, attraction
%INDEX% Charges, repulsion
%INDEX% Atom, nucleus
%INDEX% Nucleus

%(END_NOTES)


