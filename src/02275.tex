
%(BEGIN_QUESTION)
% Copyright 2004, Tony R. Kuphaldt, released under the Creative Commons Attribution License (v 1.0)
% This means you may do almost anything with this work of mine, so long as you give me proper credit

FM tends to be a far more noise-resistant means of signal modulation than AM.  For instance, the "crackling" form of radio interference caused by natural lightning or the "buzzing" noise produced by high-voltage power lines are both easy to hear on an AM radio, but absent on an FM radio.  Explain why.

\underbar{file 02275}
%(END_QUESTION)





%(BEGIN_ANSWER)

Radio interference manifests itself as additional peaks on the "envelope" of a modulated carrier wave.  AM reception is based on the extraction of that envelope from the modulated carrier, and so AM receivers will "pick up" unwanted noise.  FM reception is based on the extraction of information from changes in {\it frequency}, which is largely unaffected by noise.

%(END_ANSWER)





%(BEGIN_NOTES)

Ask students to explain this principle in their own words, and not just repeat the given answer.

%INDEX% AM versus FM
%INDEX% FM versus AM

%(END_NOTES)


