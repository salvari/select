
%(BEGIN_QUESTION)
% Copyright 2004, Tony R. Kuphaldt, released under the Creative Commons Attribution License (v 1.0)
% This means you may do almost anything with this work of mine, so long as you give me proper credit

Identify each of these filter types, and explain {\it how} you were able to positively identify their behaviors:

$$\epsfbox{02098x01.eps}$$

\underbar{file 02098}
%(END_QUESTION)





%(BEGIN_ANSWER)

$$\epsfbox{02098x02.eps}$$

\vskip 10pt

Follow-up question: in each of the circuits shown, identify at least one {\it single} component failure that has the ability to prevent any signal voltage from reaching the output terminals.

%(END_ANSWER)





%(BEGIN_NOTES)

Some of these filter designs are resonant in nature, while others are not.  Resonant circuits, especially when made with high-Q components, approach ideal band-pass (or -block) characteristics.  Discuss with your students the different design strategies between resonant and non-resonant band filters.  

The high-pass filter containing both inductors and capacitors may at first appear to be some form of resonant (i.e. band-pass or band-stop) filter.  It actually {\it will} resonate at some frequency(ies), but its overall behavior is still high-pass.  If students ask about this, you may best answer their queries by using computer simulation software to plot the behavior of a similar circuit (or by suggesting they do the simulation themselves).

Regarding the follow-up question, it would be a good exercise to discuss which suggested component failures are more likely than others, given the relatively likelihood for capacitors to fail shorted and inductors and resistors to fail open.

%INDEX% Filter circuits, lowpass and highpass
%INDEX% Resonant filter circuits
%INDEX% Filter circuits, resonant

%(END_NOTES)


