
%(BEGIN_QUESTION)
% Copyright 2004, Tony R. Kuphaldt, released under the Creative Commons Attribution License (v 1.0)
% This means you may do almost anything with this work of mine, so long as you give me proper credit

Identify some different ways you can create the sine-wave audio tracks necessary to turn the media player into a signal generator.

\underbar{file 02269}
%(END_QUESTION)





%(BEGIN_ANSWER)

Some computer programs will generate "wave" files ({\tt .wav}) according to the wave-shape, amplitude, and frequency specified.  Using a program such as this is one of the best ways to obtain the necessary audio tracks.

%(END_ANSWER)





%(BEGIN_NOTES)

There are other ways to obtain good audio tracks, including recording the output of another signal generator.  I have had such good success with the computer method, though, I can't imagine that recording an analog source -- no matter how good that source may be -- would do much better.

Incidentally, the program I use is a Linux application called {\tt sox}.  It is a general-purpose sound file converter utility that also happens to have a "synthesizer" mode where it can generate audio tracks.  Using this utility, I created twenty sine-wave tracks and then burnt them to a CD-R, where I could then play the tracks using a cheap (\$17.00, 2004 U.S. currency) CD-audio player.  I found the best wave quality to be when the player's "anti-skip" mode was turned {\it off}.

%(END_NOTES)


