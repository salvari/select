
%(BEGIN_QUESTION)
% Copyright 2005, Tony R. Kuphaldt, released under the Creative Commons Attribution License (v 1.0)
% This means you may do almost anything with this work of mine, so long as you give me proper credit

Convert the following amplifier gains expressed in the unit of decibels (dB), to gain figures expressed as unitless ratios:

\medskip
\goodbreak
\item{$\bullet$} $A_P$ = 5 dB ; $A_{P(ratio)}$ = 
\item{$\bullet$} $A_V$ = 23 dB ; $A_{V(ratio)}$ = 
\item{$\bullet$} $A_I$ = 20 dB ; $A_{I(ratio)}$ = 
\item{$\bullet$} $A_P$ = 2.5 dB ; $A_{P(ratio)}$ = 
\item{$\bullet$} $A_I$ = 7.4 dB ; $A_{I(ratio)}$ = 
\item{$\bullet$} $A_V$ = 45 dB ; $A_{V(ratio)}$ = 
\item{$\bullet$} $A_P$ = 12.8 dB ; $A_{P(ratio)}$ = 
\item{$\bullet$} $A_V$ = 30 dB ; $A_{V(ratio)}$ = 
\medskip

\underbar{file 02448}
%(END_QUESTION)





%(BEGIN_ANSWER)

\medskip
\goodbreak
\item{$\bullet$} $A_P$ = 5 dB ; $A_{P(ratio)}$ = 3.16
\item{$\bullet$} $A_V$ = 23 dB ; $A_{V(ratio)}$ = 14.13
\item{$\bullet$} $A_I$ = 20 dB ; $A_{I(ratio)}$ = 10
\item{$\bullet$} $A_P$ = 2.5 dB ; $A_{P(ratio)}$ = 1.78
\item{$\bullet$} $A_I$ = 7.4 dB ; $A_{I(ratio)}$ = 2.34
\item{$\bullet$} $A_V$ = 45 dB ; $A_{V(ratio)}$ = 177.8
\item{$\bullet$} $A_P$ = 12.8 dB ; $A_{P(ratio)}$ = 19.05
\item{$\bullet$} $A_V$ = 30 dB ; $A_{V(ratio)}$ = 31.62
\medskip

%(END_ANSWER)





%(BEGIN_NOTES)

Nothing special here, just straightforward decibel-to-ratio calculations.  Have your students share and discuss the steps necessary to do all these conversions.

%INDEX% Decibel (gain) calculations
%INDEX% Gain, converting decibels into ratios

%(END_NOTES)


