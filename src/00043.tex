
%(BEGIN_QUESTION)
% Copyright 2003, Tony R. Kuphaldt, released under the Creative Commons Attribution License (v 1.0)
% This means you may do almost anything with this work of mine, so long as you give me proper credit

Define the following terms:

\medskip 
\item{$\bullet$} Frequency
\item{$\bullet$} Wavelength
\item{$\bullet$} Cycle
\item{$\bullet$} Hertz
\item{$\bullet$} Amplitude
\medskip 

\underbar{file 00043}
%(END_QUESTION)





%(BEGIN_ANSWER)

These terms are very easy to find definitions for, from a wide variety of sources.  I'll let you discover what they mean for yourself!

%(END_ANSWER)





%(BEGIN_NOTES)

The unit "Hertz" is an example of how an implicitly defined label like "cycles per second" (CPS) came to be replaced by one having no objective meaning at all.  Similar to how "Celsius" replaced "Centigrade" as the official metric name for a unit of temperature measurement.  Pardon me for stepping up on my soapbox here, but changes such as these really irritate me.

It is important to understand that these terms apply to non-electrical phenomena as well as electrical.  The wide applicability of such terms allows for easier comprehension.  Ask your students to give common examples of these terms in everyday life.  Hint: music is an excellent context for this question!

%INDEX% Frequency, defined
%INDEX% Wavelength, defined
%INDEX% Cycle, defined
%INDEX% Hertz, defined
%INDEX% Amplitude, defined

%(END_NOTES)


