
%(BEGIN_QUESTION)
% Copyright 2003, Tony R. Kuphaldt, released under the Creative Commons Attribution License (v 1.0)
% This means you may do almost anything with this work of mine, so long as you give me proper credit

The relationship between voltage and current for a PN junction is described by this equation, sometimes referred to as the "diode equation," or "Shockley's diode equation" after its discoverer:

$$I_D = I_S (e^{qV_D \over NkT} - 1)$$

\noindent
Where,

$I_D =$ Current through the PN junction, in amps

$I_S =$ PN junction saturation current, in amps (typically 1 picoamp)

$e =$ Euler's number $\approx$ 2.718281828

$q =$ Electron unit charge, $1.6 \times 10^{-19}$ coulombs

$V_D =$ Voltage across the PN junction, in volts

$N =$ Nonideality coefficient, or emission coefficient (typically between 1 and 2)

$k =$ Boltzmann's constant, $1.38 \times 10^{-23}$

$T =$ Junction temperature, degrees Kelvin

\vskip 10pt

At first this equation may seem very daunting, until you realize that there are really only three variables in it: $I_D$, $V_D$, and $T$.  All the other terms are constants.  Since in most cases we assume temperature is fairly constant as well, we are really only dealing with two variables: diode current and diode voltage.  Based on this realization, re-write the equation as a proportionality rather than an equality, showing how the two variables of diode current and voltage relate:

$$I_D \propto \hbox{ . . .}$$

\vskip 10pt

Based on this simplified equation, what would an I/V graph for a PN junction look like?  How does this graph compare against the I/V graph for a resistor?

$$\epsfbox{00712x01.eps}$$

\underbar{file 00712}
%(END_QUESTION)





%(BEGIN_ANSWER)

Simplified proportionality:

$$I_D \propto e^{V_D}$$

The graph described by the "diode formula" is a standard exponential curve, rising sharply as the independent variable ($V_D$, in this case) increases.  The corresponding graph for a resistor, of course, is linear.

%(END_ANSWER)





%(BEGIN_NOTES)

Ask your students to sketch their own renditions of an exponential curve on the whiteboard for all to see.  Don't just let them get away with parroting the answer: "It's an exponential curve."

%INDEX% Diode equation
%INDEX% Shockley's diode equation
%INDEX% Graphing

%(END_NOTES)


