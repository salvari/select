
%(BEGIN_QUESTION)
% Copyright 2005, Tony R. Kuphaldt, released under the Creative Commons Attribution License (v 1.0)
% This means you may do almost anything with this work of mine, so long as you give me proper credit

Electrochemical batteries are supposed to act as constant voltage sources, outputting an unchanging voltage for a wide range of load currents.  The output voltage of real batteries, though, always "sags" to some degree under the influence of a load.

Explain why this is so, in terms of modeling the battery as an ideal voltage source combined with a resistance.  How do you suggest the internal resistance of a chemical battery be experimentally measured?

\underbar{file 03223}
%(END_QUESTION)





%(BEGIN_ANSWER)

$$\epsfbox{03223x01.eps}$$

I'll let you figure out how to measure this internal resistance!

%(END_ANSWER)





%(BEGIN_NOTES)

Although real chemical batteries do not respond as simply as this equivalent circuit would suggest, the model is accurate enough for many purposes.

%INDEX% Internal resistance of voltage sources
%INDEX% Modeling electrochemical battery as ideal voltage source with series resistance
%INDEX% Voltage sources, internal resistance of

%(END_NOTES)


