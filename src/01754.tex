
%(BEGIN_QUESTION)
% Copyright 2003, Tony R. Kuphaldt, released under the Creative Commons Attribution License (v 1.0)
% This means you may do almost anything with this work of mine, so long as you give me proper credit

$$\epsfbox{01754x01.eps}$$

\underbar{file 01754}
\vfil \eject
%(END_QUESTION)





%(BEGIN_ANSWER)

Use circuit simulation software to verify your predicted and measured parameter values.

%(END_ANSWER)





%(BEGIN_NOTES)

Be sure to remind your students that resistances $R_1$ and $R_2$ may need to be series-parallel networks in themselves, to achieve the necessary values.  An alternative you may wish to permit is the use of 10-turn (precision) potentiometers connected as rheostats for $R_1$ and $R_2$.  This way the circuit's minimum and maximum values may be precisely calibrated.  The main potentiometer, $R_{pot1}$, should be a 3/4 turn unit, to allow fast checking of minimum and maximum total resistance, and it should be some common value such as 1 k$\Omega$ or 10 k$\Omega$.

%INDEX% Assessment, performance-based (Custom rheostat range)

%(END_NOTES)


