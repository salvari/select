
%(BEGIN_QUESTION)
% Copyright 2003, Tony R. Kuphaldt, released under the Creative Commons Attribution License (v 1.0)
% This means you may do almost anything with this work of mine, so long as you give me proper credit

A very common form of {\it latch} circuit is the simple "start-stop" relay circuit used for motor controls, whereby a pair of momentary-contact pushbutton switches control the operation of an electric motor.  In this particular case, I show a low-voltage control circuit and a 3-phase, higher voltage motor:

$$\epsfbox{01347x01.eps}$$

Explain the operation of this circuit, from the time the "Start" switch is actuated to the time the "Stop" switch is actuated.  The normally-open M1 contact shown in the low-voltage control circuit is commonly called a {\it seal-in contact}.  Explain what this contact does, and why it might be called a "seal-in" contact.

\underbar{file 01347}
%(END_QUESTION)





%(BEGIN_ANSWER)

Even though the "Start" and "Stop" switches are momentary, the "seal-in" contact makes the circuit {\it latch} in one of two states: either motor energized or motor de-energized.

%(END_ANSWER)





%(BEGIN_NOTES)

Motor "start-stop" circuits are very common in industry, and apply to applications beyond electric motors.  Ask your students if they can think of any application for a circuit such as this.

%INDEX% Control circuit, AC motor
%INDEX% Seal-in contact, motor control circuit

%(END_NOTES)


