
%(BEGIN_QUESTION)
% Copyright 2003, Tony R. Kuphaldt, released under the Creative Commons Attribution License (v 1.0)
% This means you may do almost anything with this work of mine, so long as you give me proper credit

A common means of biasing a depletion-type IGFET is called {\it zero biasing}.  An example circuit is shown below:

$$\epsfbox{01193x01.eps}$$

This may appear similar to {\it self-biasing} as seen with JFET amplifier circuits, but it is not.  Zero biasing only works with IGFET amplifier circuits.  Explain why this is so.

\underbar{file 01193}
%(END_QUESTION)





%(BEGIN_ANSWER)

The natural Q-point of a depletion-type IGFET occurs with a gate-to-source voltage of 0 volts.  This is very different from either bipolar junction (BJT) or junction field-effect (JFET) transistors.

\vskip 10pt

Follow-up question: will "zero" biasing work with enhancement-mode IGFETs as well?  Explain why or why not.

%(END_ANSWER)





%(BEGIN_NOTES)

This question provides students with an opportunity to review IGFET theory, and to differentiate between depletion-mode and enhancement-mode types, which is a subject of much confusion among students new to the topic.

%INDEX% Zero-biasing, MOSFET amplifier circuit

%(END_NOTES)


