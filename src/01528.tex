
%(BEGIN_QUESTION)
% Copyright 2003, Tony R. Kuphaldt, released under the Creative Commons Attribution License (v 1.0)
% This means you may do almost anything with this work of mine, so long as you give me proper credit

A technician is given a transistor testing circuit to repair.  This simple circuit is an audio-frequency oscillator, and has the following schematic diagram:

$$\epsfbox{01528x01.eps}$$

After repairing a broken solder joint, the technician notices that the DPDT switch has lost its label.  The purpose of this switch is to allow polarity to be reversed so as to test both PNP and NPN transistor types.  However, the label showing which direction is for NPN and which direction is for PNP has fallen off.  And, to make matters worse, the schematic diagram does not indicate which position is which.

Determine what the proper DPDT switch label should be for this transistor tester, and explain how you know it is correct.  Note: you do not even have to understand how the oscillator circuit works to be able to determine the proper switch label.  All you need to know is the proper voltage polarities for NPN and PNP transistor types.

\underbar{file 01528}
%(END_QUESTION)





%(BEGIN_ANSWER)

Left is NPN, and right is PNP.

%(END_ANSWER)





%(BEGIN_NOTES)

This is a very realistic problem for a technician to solve.  Of course, one could determine the proper switch labeling experimentally (by trying a known NPN or PNP transistor and seeing which position makes the oscillator work), but students need to figure this problem out without resorting to trial and error.  It is very important that they learn how to properly bias transistors!

Be sure to ask your students to explain {\it how} they arrived at their conclusion.  It is not good enough for them to simply repeat the given answer!

%INDEX% Polarity, transistor
%INDEX% Transistor voltage polarities

%(END_NOTES)


