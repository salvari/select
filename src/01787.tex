
%(BEGIN_QUESTION)
% Copyright 2003, Tony R. Kuphaldt, released under the Creative Commons Attribution License (v 1.0)
% This means you may do almost anything with this work of mine, so long as you give me proper credit

Combining Lenz's Law with the right-hand rule (or left-hand rule, if you follow electron flow instead of conventional flow) provides a simple and effective means for determining the direction of induced current in an induction coil.  In the following examples, trace the direction of current through the load resistor:

$$\epsfbox{01787x01.eps}$$

\underbar{file 01787}
%(END_QUESTION)





%(BEGIN_ANSWER)

Note: in case it isn't clear from the illustrations, Figures 1 through 4 show the magnet moving in relation to a stationary coil.  Figures 5 and 6 show a coil moving in relation to a stationary magnet.

$$\epsfbox{01787x02.eps}$$

%(END_ANSWER)





%(BEGIN_NOTES)

An easy way I find to remember Lenz's Law is to interpret is as {\it opposition to change}.  The coil will try to become a magnet that fights the motion.  A good way to get students thinking along these lines is to ask them, "What magnetic polarity would the coil have to assume (in each case) to resist the magnet's relative motion?"  In other words, if the magnet moves closer to the coil, the coil will "magnetize" so as to push against the magnet.  If the magnet moves away from the coil, the coil will "magnetize" so as to attract the magnet.

%INDEX% Lenz's Law, direction of induced current

%(END_NOTES)


