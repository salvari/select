
%(BEGIN_QUESTION)
% Copyright 2003, Tony R. Kuphaldt, released under the Creative Commons Attribution License (v 1.0)
% This means you may do almost anything with this work of mine, so long as you give me proper credit

As an electric current is passed through a coil of wire, it creates a magnetic field.  If the magnitude of this current changes over time, so will the strength of the magnetic field.  

We also know that a magnetic field flux that changes over time will induce a voltage along the length of a wire coil.  Explain how the complementary principles of electromagnetism and electromagnetic induction manifest themselves simultaneously in the same wire coil to produce {\it self-induction}.

Also, explain how Lenz's Law relates to the polarity of the coil's self-induced voltage.

\underbar{file 00263}
%(END_QUESTION)





%(BEGIN_ANSWER)

A changing current through a coil produces a voltage drop that opposes the direction of change.

%(END_ANSWER)





%(BEGIN_NOTES)

Self-induction is not a difficult concept to grasp if one already possesses a good understanding of electromagnetism, electromagnetic induction, and Lenz's Law.  Some students may struggle understanding self-induction, because it is probably the first application they've seen where these three phenomena inter-relate simultaneously.

%INDEX% Self-induction

%(END_NOTES)


