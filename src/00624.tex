
%(BEGIN_QUESTION)
% Copyright 2003, Tony R. Kuphaldt, released under the Creative Commons Attribution License (v 1.0)
% This means you may do almost anything with this work of mine, so long as you give me proper credit

Describe the so-called "domain theory" of magnetism, as it applies to permanent magnets.

\underbar{file 00624}
%(END_QUESTION)





%(BEGIN_ANSWER)

I'll let you research the answer to this question!

%(END_ANSWER)





%(BEGIN_NOTES)

First, ask your students what a "permanent" magnet is.  Are there other kinds of magnets?  If so, what might they be called?

One of the more interesting applications of domain theory is {\it bubble memory}, a type of computer memory technology making use of localized "bubbles" of magnetism in a material corresponding to discrete magnetic domains.  You might want to bring this up in discussion, if none of your students do it first.

%INDEX% Domain, magnetic
%INDEX% Magnetic domain

%(END_NOTES)


