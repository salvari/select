
%(BEGIN_QUESTION)
% Copyright 2003, Tony R. Kuphaldt, released under the Creative Commons Attribution License (v 1.0)
% This means you may do almost anything with this work of mine, so long as you give me proper credit

Write two KVL loop equations for this circuit, using $I_1$ and $I_2$ as the only variables:

$$\epsfbox{01043x01.eps}$$

\underbar{file 01043}
%(END_QUESTION)





%(BEGIN_ANSWER)

KVL equation for left-hand loop:

$6 - 1000I_1 - 1000(I_1 + I_2) + 1 = 0$

\vskip 10pt

KVL equation for right-hand loop:

$7.2 - 1000I_2 - 1000(I_1 + I_2) + 1 = 0$

\vskip 10pt

Follow-up question: what are the values of $I_1$ and $I_2$, based on this system of equations?

%(END_ANSWER)





%(BEGIN_NOTES)

Students' equations may not look exactly like these, depending on how they "stepped" around the loops tallying voltage drops.  So long as they are all able to reach the same answers for $I_1$ and $I_2$, it does not matter.  In fact, it is a good thing to have different students propose different forms of the equations to demonstrate that the same answers are obtained every time.

%(END_NOTES)


