
%(BEGIN_QUESTION)
% Copyright 2003, Tony R. Kuphaldt, released under the Creative Commons Attribution License (v 1.0)
% This means you may do almost anything with this work of mine, so long as you give me proper credit

A very important concept to understand in digital circuitry is the difference between {\it current sourcing} and {\it current sinking}.  For instance, examine this TTL inverter gate circuit, connected to a load:

$$\epsfbox{01666x01.eps}$$

The output circuitry of this particular gate is commonly referred to as "totem-pole," because the two output transistors are stacked one above the other like figures on a totem pole.  Is a gate circuit with a totem-pole output stage able to {\it source} load current, {\it sink} load current, or do both?

\underbar{file 01666}
%(END_QUESTION)





%(BEGIN_ANSWER)

TTL gates equipped with totem-pole output circuitry are able to both {\it source} and {\it sink} load current.  In this particular case, the way the load (LED) is connected to the output of the gate, the gate will only {\it source} current.  However, the gate is capable of sinking current from a load, if only the load were connected differently.

\vskip 10pt

Follow-up question: does the input device driving this TTL gate circuit (the switch in this particular example) have to {\it source} current, {\it sink} current, or both?

\vskip 10pt

Challenge question: explain how you would calculate the current sourcing and current sinking abilities of this logic gate circuit, if you were given the internal component values and parameters.

%(END_ANSWER)





%(BEGIN_NOTES)

The very important concept of sourcing versus sinking is best understood from the perspective of {\it conventional} current flow notation.  The terms seem backward when electron flow notation is used to track current through the output transistor.

One point of confusion I've experienced among students is that current may go either direction (in or out) of a gate with totem-pole output transistors (able to sink or source current).  Some students seem to have a conceptual difficulty with current going {\it in} to the {\it output} terminal of a gate circuit, because they mistakenly associate the "out" in {\it out}put as being a reference to direction of current, rather than direction of information or data.

An analogy I've used to help students overcome this problem is that of two people carrying a long pole:

$$\epsfbox{01666x02.eps}$$

Suppose these people are in a dark, noisy room, and they use the pole as a means of simple communication between them.  For example, one person could tug on the pole to get the other person's attention.  Perhaps they could even develop a simple code system for communicating thoughts (1 tug = hello ; 2 tugs = good-bye ; 3 tugs = I think this is a silly way to communicate ; 4 tugs = let's leave this room ; etc.).  If one of the persons {\it pushes} on the pole rather than {\it pulls} on the pole to get the other person's attention, does the direction of the pole's motion change the direction of the communication between the two persons?  Of course not.  Well, then, does the direction of current through the output terminal of a gate change the direction that {\it information} flows between two interconnected gates?  Whether a gate sources current or sinks current to a load has no bearing on the "output" designation of that gate terminal.  Either way, the gate is still "telling the load what to do" by exercising control over the load current.

%INDEX% Current sink, TTL logic
%INDEX% Current source, TTL logic
%INDEX% Sinking current, TTL logic
%INDEX% Sourcing current, TTL logic
%INDEX% Totem-pole output, TTL
%INDEX% TTL gate circuit, internal schematic

%(END_NOTES)


