
%(BEGIN_QUESTION)
% Copyright 2003, Tony R. Kuphaldt, released under the Creative Commons Attribution License (v 1.0)
% This means you may do almost anything with this work of mine, so long as you give me proper credit

What will happen in this circuit as the switches are sequentially turned on, starting with switch number 1 and ending with switch number 3?  

$$\epsfbox{00296x01.eps}$$

Describe how the successive closure of these three switches will impact:

\medskip
\item{$\bullet$} The voltage drop across each resistor 
\item{$\bullet$} The current through each resistor
\item{$\bullet$} The total amount of current drawn from the battery
\item{$\bullet$} The total amount of circuit resistance "seen" by the battery
\medskip

\underbar{file 00296}
%(END_QUESTION)





%(BEGIN_ANSWER)

I won't explain what happens when each of the switches is closed, but I will describe the effects of the first switch closing:

As the first switch (SW1) is closed, the voltage across resistor R1 will increase to full battery voltage, while the voltages across the remaining resistors will remain unchanged from their previous values.  The current through resistor R1 will increase from zero to whatever value is predicted by Ohm's Law (full battery voltage divided by that resistor's resistance), and the current through the remaining resistors will remain unchanged from their previous values.  The amount of current drawn from the battery will increase.  Overall, the battery "sees" less total resistance than before.

%(END_ANSWER)





%(BEGIN_NOTES)

One problem I've encountered while teaching the "laws" of parallel circuits is that some students mistakenly think the rule of "all voltages in a parallel circuit being the same" means that the amount of voltage in a parallel circuit is fixed over time and cannot change.  The root of this misunderstanding is memorization rather than comprehension: students memorize the rule "{\it all voltages are the same}" and think this means the voltages must remain the same before and after any change is made to the circuit.  I've actually had students complain to me, saying, "But you told us all voltages are {\it the same} in a parallel circuit!", as though it were my job to decree perfect and universal Laws which would require no critical thinking on the part of the student.  But I digress . . .

This question challenges students' comprehension of parallel circuit behavior by asking what happens after a change is made to the circuit.  The purpose of the switches is to "add" resistors from the circuit, one at a time, without actually having to insert new components.

%INDEX% Parallel circuit; voltage, current, resistance, and power in
%INDEX% Troubleshooting, simple circuit

%(END_NOTES)


