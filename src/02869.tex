
%(BEGIN_QUESTION)
% Copyright 2005, Tony R. Kuphaldt, released under the Creative Commons Attribution License (v 1.0)
% This means you may do almost anything with this work of mine, so long as you give me proper credit

Explain why placing static-sensitive components (such as CMOS integrated circuits) into a block of conductive foam protects them against damage from ESD, and why this protection exists even if the entire block of foam (with chip) is brought to an elevated potential with respect to earth ground.

\underbar{file 02869}
%(END_QUESTION)





%(BEGIN_ANSWER)

The conductive foam makes the pins electrically common to one another, so no significant {\it difference} of voltage may appear between any two pins of the component.

%(END_ANSWER)





%(BEGIN_NOTES)

You may underscore this principle by stating to your students that you may walk up to a piece of conductive foam with lots of CMOS chips inserted into it, and touch it with your static-charged finger, with no damage.  Even if you draw a spark between your finger and the foam (or any chip pin stuck into the foam), the chips will all be protected because they experience no voltage {\it between} their pins.

%INDEX% ESD (Electro-Static Discharge)

%(END_NOTES)


