
%(BEGIN_QUESTION)
% Copyright 2003, Tony R. Kuphaldt, released under the Creative Commons Attribution License (v 1.0)
% This means you may do almost anything with this work of mine, so long as you give me proper credit

The ignition system in a spark-ignition automobile engine produces voltages in the range of tens of thousands of volts: greater than the voltage levels typically used to distribute electric power through neighborhoods.  Although this is capable of producing very painful electric shocks, the actual shock {\it hazard} it poses to a person is minimal.  Why is this?

\underbar{file 00006}
%(END_QUESTION)





%(BEGIN_ANSWER)

Automotive ignition systems pose little direct shock hazard because of two factors: the {\it resistance} intrinsic to the high voltage circuit limits current to a fairly low value even without the resistance of a person's body in the circuit; and the high voltage pulse lasts only a brief moment in time.

%(END_ANSWER)





%(BEGIN_NOTES)

This is not to say that ignition systems pose no hazard, though.  One of the main hazards is the reaction a shock produces in a person: namely, the jerking of limbs which could be dangerous in the proximity of moving parts.  Emphasize that hazards may often be indirect: that something in itself may not be enough to hurt you, but you body's reaction to that something is what poses the greater threat.  

An example of this general principle is an allergic reaction.  The threat here is the body's over-reaction to an allergen, not the allergen itself!

%INDEX% Electric shock
%INDEX% Safety, electrical

%(END_NOTES)


