
%(BEGIN_QUESTION)
% Copyright 2003, Tony R. Kuphaldt, released under the Creative Commons Attribution License (v 1.0)
% This means you may do almost anything with this work of mine, so long as you give me proper credit

A mechanic goes to school and takes a course in AC electric circuits.  Upon learning about step-up and step-down transformers, he makes the remark that "Transformers act like electrical versions of gears, with different ratios."  

What does the mechanic mean by this statement?  What exactly is a "gear ratio," and how does this relate to the subject of impedance matching?

\underbar{file 00671}
%(END_QUESTION)





%(BEGIN_ANSWER)

Just as meshing gears with different tooth counts transform mechanical power between different levels of speed and torque, electrical transformers transform power between different levels of voltage and current.  

The concept of "impedance" is just as valid in mechanical systems as in electrical systems: a "low impedance" mechanical load requires high speed and low torque, whereas a "high impedance" load requires high torque and low speed.  Gear systems provide impedance matching between mechanical power sources and loads in the same way that transformers provide impedance matching between (AC) electrical power sources and loads.

%(END_ANSWER)





%(BEGIN_NOTES)

Not only is this a sound analogy, but one that many mechanically-minded people relate with easily!  If you happen to have some mechanics in your classroom, provide them with the opportunity to explain the concept of gear ratios to those students who are unaware of gear system mathematics.

I normally do not elaborate this much in my answers, but in this case I believe it may be necessary, as this is quite a cognitive leap for some people.  It is a leap well worth making, however, since it connects two (seemingly) disparate phenomenon in a way that provides a sound context for understanding the concept of impedance matching.

%(END_NOTES)


