
%(BEGIN_QUESTION)
% Copyright 2005, Tony R. Kuphaldt, released under the Creative Commons Attribution License (v 1.0)
% This means you may do almost anything with this work of mine, so long as you give me proper credit

Use Boolean algebra to simplify the following relay (ladder logic) circuit:

$$\epsfbox{02815x01.eps}$$

\underbar{file 02815}
%(END_QUESTION)





%(BEGIN_ANSWER)

$$\epsfbox{02815x02.eps}$$

%(END_ANSWER)





%(BEGIN_NOTES)

Have your students explain the entire process they used in simplifying the relay circuit: developing the Boolean expression, simplifying that expression using Boolean algebra techniques, and then developing a new relay circuit from the simplified Boolean expression.  By having your students share their thought processes with the whole class, you will increase the level of learning on the parts of presenter and viewer alike.  Students presenting their solutions will gain a better understanding of how it works because the act of presenting helps consolidate what they already know.  Students viewing the presentation will get to see another person's technique (rather than just the instructor's), which will allow them to see examples of how to do these processes cast in slightly different terms.

%INDEX% Boolean algebra, relay circuit simplification

%(END_NOTES)


