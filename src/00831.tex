
%(BEGIN_QUESTION)
% Copyright 2003, Tony R. Kuphaldt, released under the Creative Commons Attribution License (v 1.0)
% This means you may do almost anything with this work of mine, so long as you give me proper credit

Specialized forms of the decibel unit have been devised to allow easy representation of quantities other than arbitrary ratios of voltage, current, or power.  Take for example these units, the first one used extensively in the telecommunications industry:

\medskip
\item{$\bullet$} dBm
\item{$\bullet$} dBW
\item{$\bullet$} dBk
\medskip

Define what each of these units represents.

\underbar{file 00831}
%(END_QUESTION)





%(BEGIN_ANSWER)

"dBm" represents the magnitude of a voltage in relation to 1 mW of power dissipated by a 600 $\Omega$ load.  "dBW" and "dBk" units represent the magnitude of a voltage in relation to 1 W and 1 kW of power dissipated by the same load, respectively.

\vskip 10pt

Follow-up question: how many volts is 2 dBm equivalent to?

%(END_ANSWER)





%(BEGIN_NOTES)

Here we see the decibel unit being used to represent {\it absolute} quantities rather than relative ratios.  Ask your students what benefit would there be in doing this.  Why not just represent signal magnitudes in units of "volts" instead?  Why would we want to use an obscure unit such as the decibel?

%INDEX% Decibel, as an absolute power unit

%(END_NOTES)


