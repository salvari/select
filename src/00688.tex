
%(BEGIN_QUESTION)
% Copyright 2003, Tony R. Kuphaldt, released under the Creative Commons Attribution License (v 1.0)
% This means you may do almost anything with this work of mine, so long as you give me proper credit

General metrology used to be based upon standard {\it artifacts}, but the modern scientific trend is toward {\it intrinsic standards}.  Explain what these two terms mean, as well as their significance in scientific work.

\underbar{file 00688}
%(END_QUESTION)





%(BEGIN_ANSWER)

An "artifact" is a non-reproduceable object that is arbitrarily deemed the "standard."  An "intrinsic" standard, by comparison, is a reproduceable standard based on immutable physical constants.

%(END_ANSWER)





%(BEGIN_NOTES)

A great example of an "artifact" standard is the metal bar which used to be the international standard for the "meter" (metric unit of length).  Ask your students how convenient it would be for scientists working around the world to calibrate their equipment if the only primary standard for length measurement were a single bar of metal.  What benefits may be derived from the use of "intrinsic" standards?

An excellent example of an easily-accessed intrinsic standard is the standard for time, available via shortwave radio: 5 kHz, 10kHz, 15 kHz, and 20 kHz (there are other frequencies, too).  Have any of your students heard of timepieces that synchronize themselves to an "atomic clock"?  With a shortwave radio, they can tune into that same atomic clock's broadcast and synchronize their own wristwatches!  This is an excellent discussion activity, and a great way to garner student interest in what is potentially a dull subject.

Ask your students whether or not the existence of intrinsic standards negates the purpose of artifacts.  That is, does anyone use artifacts anymore?  Why or why not?

%INDEX% Artifact standard, defined
%INDEX% Intrinsic standard, defined
%INDEX% Metrology, artifact versus intrinsic standard
%INDEX% Standard, artifact versus intrinsic (metrology)

%(END_NOTES)


