
%(BEGIN_QUESTION)
% Copyright 2003, Tony R. Kuphaldt, released under the Creative Commons Attribution License (v 1.0)
% This means you may do almost anything with this work of mine, so long as you give me proper credit

Sum-of-Product Boolean expressions all follow the same general form.  As such, their equivalent logic gate circuits likewise follow a common form.  Translate each of these SOP expressions into its equivalent logic gate circuit:

\vskip 10pt

$$AB + A\overline{B}$$

\vskip 10pt

$$A\overline{B} + \overline{A}B$$

\vskip 10pt

$$ABC + \overline{A}B\overline{C} + AB\overline{C}$$

\vskip 10pt

\underbar{file 01325}
%(END_QUESTION)





%(BEGIN_ANSWER)

$$\epsfbox{01325x01.eps}$$

\vskip 30pt

$$\epsfbox{01325x02.eps}$$

\vskip 30pt

$$\epsfbox{01325x03.eps}$$

%(END_ANSWER)





%(BEGIN_NOTES)

The translation from Boolean SOP to gate circuit should not be difficult.  The point of this question is to get students thinking in terms of sum-of-products form, so they will be ready for the next step: linking this concept with truth tables.

%INDEX% Sum-of-Products expression, Boolean algebra (translating into gate circuit)
%INDEX% SOP expression, Boolean algebra (translating into gate circuit)

%(END_NOTES)


