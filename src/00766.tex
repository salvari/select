
%(BEGIN_QUESTION)
% Copyright 2005, Tony R. Kuphaldt, released under the Creative Commons Attribution License (v 1.0)
% This means you may do almost anything with this work of mine, so long as you give me proper credit

In active and passive filter design literature, you often come across filter circuits classified as one of three different names:

\medskip
\item{$\bullet$} Chebyshev
\item{$\bullet$} Butterworth
\item{$\bullet$} Bessel
\medskip

Describe what each of these names means.  What, exactly, distinguishes a "Chebyshev" filter circuit from a "Butterworth" filter circuit?

\underbar{file 00766}
%(END_QUESTION)





%(BEGIN_ANSWER)

Each of these terms describes a class of filter {\it responses}, rather than a particular circuit configuration (topology).  The shape of the Bode plot for a filter circuit is the determining factor for whether it will be a "Chebyshev," "Butterworth," or "Bessel" filter.

%(END_ANSWER)





%(BEGIN_NOTES)

I purposely omitted Bode plot examples for these three filter classifications.  Presentation and examination of Bode plots is an appropriate activity for discussion time.  Draw a set of Bode plot axes on the whiteboard, and then have students draw approximate Bode plots for each filter response, as determined from their research.

%INDEX% Active filter circuit, Bessel response
%INDEX% Active filter circuit, Butterworth response
%INDEX% Active filter circuit, Chebyshev response

%(END_NOTES)


