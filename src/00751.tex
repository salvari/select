
%(BEGIN_QUESTION)
% Copyright 2003, Tony R. Kuphaldt, released under the Creative Commons Attribution License (v 1.0)
% This means you may do almost anything with this work of mine, so long as you give me proper credit

Suppose you were building a Class-A transistor amplifier for audio frequency use, but did not have an oscilloscope available to check the output waveform for the presence of "clipping" caused by improper biasing.  You do, however, have a pair of audio headphones you may use to listen to the signals.

Explain how you would use a pair of headphones to check for the presence of severe distortion in a waveform.

\underbar{file 00751}
%(END_QUESTION)





%(BEGIN_ANSWER)

Set the signal generator to "sine-wave," and the aural difference between a pure sine wave and a distorted ("clipped") sine wave will be very apparent.

%(END_ANSWER)





%(BEGIN_NOTES)

The answer I want for this question is not just a parroting of the answer I've given.  Anyone can say "a distorted wave will sound different."  I want to know {\it how} it sounds different, and this answer can only come by direct experimentation!

%(END_NOTES)


