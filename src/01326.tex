
%(BEGIN_QUESTION)
% Copyright 2003, Tony R. Kuphaldt, released under the Creative Commons Attribution License (v 1.0)
% This means you may do almost anything with this work of mine, so long as you give me proper credit

Although it is seldom done, it is possible to express a truth table in verbal form, by describing what conditions must be met in order to generate a "high" output.

Take for example this simple truth table, for an inverter circuit:

$$\epsfbox{01326x01.eps}$$

For this truth table, we could say that the output goes high when $A$ is low.  A different way of saying this would be to state that "the output is {\it true} when $\overline{A}$ is {\it true}."

\vskip 10pt

Let's look at another example, this time of an AND gate:

$$\epsfbox{01326x02.eps}$$

For this truth table, we could say that the output goes high when $A$ and $B$ are both high.  A different way of saying this would be to state that "the output is true when $A$ is true and $B$ is true."  To use a half-Boolean, half-verbal description:

$$A \hbox{ AND } B$$

\vskip 10pt

Examine this logic gate circuit and corresponding truth table:

$$\epsfbox{01326x03.eps}$$

Express the functionality of this truth table in words.  What Boolean conditions must be satisfied ("true") in order for the output to assume a high state?

\underbar{file 01326}
%(END_QUESTION)





%(BEGIN_ANSWER)

The output of this circuit is high when $\overline{A}$ is true and $\overline{B}$ is true, or when $A$ is true and $B$ is true:

$$(\overline{A} \hbox{ AND } \overline{B}) \hbox{ OR } (A \hbox{ AND } B)$$

%(END_ANSWER)





%(BEGIN_NOTES)

Expressing truth table conditions "verbally" is a way to introduce students to the concept of deriving Boolean expression from them.

%INDEX% Sum-of-Products expression, Boolean algebra (from a verbal description of gate behavior)
%INDEX% SOP expression, Boolean algebra (from a verbal description of gate behavior)

%(END_NOTES)


