
%(BEGIN_QUESTION)
% Copyright 2003, Tony R. Kuphaldt, released under the Creative Commons Attribution License (v 1.0)
% This means you may do almost anything with this work of mine, so long as you give me proper credit

A radio technician is troubleshooting a problem in a simple AM receiver, using a voltage-measuring device called an {\it oscilloscope}.  An oscilloscope is nothing more than a graphical voltmeter, indicating the "wave shape" of voltages that change rapidly over time.  The problem with this radio is that no sound at all is heard in the headphones.

When checking for a voltage signal between points A and B in the circuit, a strong signal is obtained:

$$\epsfbox{00341x01.eps}$$

However, when checking between points C and B in the circuit, no signal is measured:

$$\epsfbox{00341x02.eps}$$

What do these voltage measurements indicate about the nature of the problem in the receiver circuit?  Is there a general troubleshooting rule that may be drawn from this example?  If so, what is it?

If possible, identify the precise nature of the failure.

\underbar{file 00341}
%(END_QUESTION)





%(BEGIN_ANSWER)

The fact that there is good signal before the potentiometer, and no signal afterward, indicates that the problem is somewhere between those two sets of signal measurement points (i.e. the potentiometer itself).

%(END_ANSWER)





%(BEGIN_NOTES)

Do not worry if your students have not studied inductors, capacitors, diodes, transistors, amplifiers, or radio theory yet.  This problem focuses on the potentiometer's function, and that is all your students have to understand in order to determine an answer.  

It is very important for electronics technicians to be able to isolate portions of circuits they do understand from portions they do not, and perform as much diagnostic work as they can based on what they know.  For this reason, I believe it is a good practice to show beginning students problems such as this, where they are challenged to see beyond the complexity of the circuit, to focus only on those portions that matter.  On the job, I was frequently challenged with troubleshooting large, complex systems that I had no hope of understanding the entirety of, but which I knew enough about to isolate the problem to sections I could repair proficiently.

Ask your students to identify where, exactly, they think the potentiometer could have failed in order to cause this particular problem.  Not just any fault within the potentiometer will result in the same loss of signal!

%INDEX% Troubleshooting, radio circuit

%(END_NOTES)


