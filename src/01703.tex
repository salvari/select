
%(BEGIN_QUESTION)
% Copyright 2005, Tony R. Kuphaldt, released under the Creative Commons Attribution License (v 1.0)
% This means you may do almost anything with this work of mine, so long as you give me proper credit

An American researcher named Charles Dalziel performed experiments with both human and animal subjects to determine the effects of electric currents on the body.  A table showing his research data is presented here:

$$\epsfbox{01703x01.eps}$$

{\bf Important Note:} {\it Dalziel's human test subjects were men and women in good health, with no known heart conditions or any other abnormalities that would have compromised their safety.  In other words, these data points represent best-case scenarios, and do not necessarily reflect the risk to persons in poorer states of health.}

\vskip 10pt

Assuming a skin contact resistance of 600 $\Omega$ for a sweaty hand, 1000 $\Omega$ of resistance for foot-to-ground contact, 50 $\Omega$ internal body resistance, 70 $\Omega$ of resistance through the soil from the person's location to the earth ground point, and a male victim, calculate the amount of voltage necessary to achieve each of the listed shock conditions (threshold of perception, pain, etc.) for the following circuit:

$$\epsfbox{01703x02.eps}$$

\underbar{file 01703}
%(END_QUESTION)





%(BEGIN_ANSWER)

\medskip
\item{$\bullet$} Slight sensation at point(s) of contact: {\bf 0.69 volts}
\item{$\bullet$} Threshold of bodily perception: {\bf 1.9 volts}
\item{$\bullet$} Pain, with voluntary muscle control maintained: {\bf 15.5 volts}
\item{$\bullet$} Pain, with loss of voluntary muscle control: {\bf 27.5 volts}
\item{$\bullet$} Severe pain and difficulty breathing: {\bf 39.6 volts}
\item{$\bullet$} Possible heart fibrillation after three seconds: {\bf 172 volts}
\medskip

%(END_ANSWER)





%(BEGIN_NOTES)

Not only does this question introduce students to the various levels of shock current necessary to induce deleterious effects in the (healthy) human body, but it also serves as a good exercise for Ohm's Law, and for introducing (or reviewing) the concept of series resistances.

For the morbidly curious, Charles Dalziel's experimentation conducted at the University of California (Berkeley) began with a state grant to investigate the bodily effects of sub-lethal electric current.  His testing method was as follows: healthy male and female volunteer subjects were asked to hold a copper wire in one hand and place their other hand on a round, brass plate.  A voltage was then applied between the wire and the plate, causing electrons to flow through the subject's arms and chest.  The current was stopped, then resumed at a higher level.  The goal here was to see how much current the subject could tolerate and still keep their hand pressed against the brass plate.  When this threshold was reached, laboratory assistants forcefully held the subject's hand in contact with the plate and the current was again increased.  The subject was asked to release the wire they were holding, to see at what current level involuntary muscle contraction (tetanus) prevented them from doing so.  For each subject the experiment was conducted using DC and also AC at various frequencies.  Over two dozen human volunteers were tested, and later studies on heart fibrillation were conducted using animal subjects. 

Given that Dalziel tested subjects for the effects of a hand-to-hand shock current path, his data does not precisely match the scenario I show in the schematic diagram (hand-to-foot).  Therefore, the calculated voltages for various hand-to-foot shock conditions are {\it approximate only}.

%INDEX% Dalziel, Charles
%INDEX% Shock current experiment data

%(END_NOTES)


