
%(BEGIN_QUESTION)
% Copyright 2003, Tony R. Kuphaldt, released under the Creative Commons Attribution License (v 1.0)
% This means you may do almost anything with this work of mine, so long as you give me proper credit

The component failure rate of complex systems usually follows a trend known in the industry as the "bathtub curve":

$$\epsfbox{01332x01.eps}$$

While the "Useful life" and "Wear-out" phases of the system life-cycle are easy to understand, the initial "Infant mortality" phase is not so intuitive.  Explain what factors might lead to premature component failure during this initial phase of a system's lifespan.

\underbar{file 01332}
%(END_QUESTION)





%(BEGIN_ANSWER)

Gross manufacturing defects, incorrect installation, and design flaws, to name a few.

\vskip 10pt

Follow-up question: is it important to know which phase of the life-cycle a system is in before you begin to troubleshoot a problem in it?  Explain your answer.

%(END_ANSWER)





%(BEGIN_NOTES)

The follow-up question is especially important to discuss with your students.  Knowing what portion of the life-cycle a system is in can make a huge difference in your troubleshooting effectiveness.  Ask your students why this is.  If possible, enlighten the discussion with examples from your own professional experience.

%(END_NOTES)


