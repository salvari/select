
%(BEGIN_QUESTION)
% Copyright 2003, Tony R. Kuphaldt, released under the Creative Commons Attribution License (v 1.0)
% This means you may do almost anything with this work of mine, so long as you give me proper credit

Determine what a digital voltmeter (DVM) would indicate if connected between the following points in this circuit:

$$\epsfbox{00347x01.eps}$$

\medskip
\item{$\bullet$} Red lead on A, black lead on H
\item{$\bullet$} Red lead on C, black lead on G
\item{$\bullet$} Red lead on F, black lead on B
\item{$\bullet$} Red lead on F, black lead on A 
\medskip

\underbar{file 00347}
%(END_QUESTION)





%(BEGIN_ANSWER)

Here is a schematic diagram to help you:

$$\epsfbox{00347x02.eps}$$

\medskip
\item{$\bullet$} Red lead on A, black lead on H = +12 volts
\item{$\bullet$} Red lead on C, black lead on G = 0 volts
\item{$\bullet$} Red lead on F, black lead on B = 0 volts
\item{$\bullet$} Red lead on F, black lead on A = -6 volts
\medskip

%(END_ANSWER)





%(BEGIN_NOTES)

Kirchhoff's Voltage Law (KVL) is very easily explored in real life with a set of batteries and "jumper wire" connections.  Encourage your students to build battery circuits like the one shown in this question, to be able to see the results for themselves!

One really nice feature of digital multimeters (DMMs) is the ability to register negative as well as positive quantities.  This feature is very useful when teaching Kirchhoff's Laws, with the {\it algebraic} (sign-dependent) summation of voltages and currents.

Students may arrive at more than one method for determining voltmeter indications in problems like these.  Encourage this type of creativity during discussion time, as it both helps students gain confidence in being able to approach problems on their own terms, as well as educates those students who might be confused with the concept.  Quite often the explanation of a peer is more valuable than the explanation of an instructor.  When students are given the freedom to explore problem-solving methods, and then share those methods with their classmates, substantial learning always results.

%INDEX% Voltmeter usage
%INDEX% Voltages in series

%(END_NOTES)


