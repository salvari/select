
%(BEGIN_QUESTION)
% Copyright 2006, Tony R. Kuphaldt, released under the Creative Commons Attribution License (v 1.0)
% This means you may do almost anything with this work of mine, so long as you give me proper credit

Predict how all component voltages and currents in this circuit will be affected as a result of the following faults.  Consider each fault independently (i.e. one at a time, no multiple faults):

$$\epsfbox{03934x01.eps}$$

\medskip
\item{$\bullet$} Current source $I_1$ fails with increased output:
\vskip 5pt
\item{$\bullet$} Solder bridge (short) across resistor $R_1$: 
\vskip 5pt
\item{$\bullet$} Solder bridge (short) across resistor $R_2$: 
\vskip 5pt
\item{$\bullet$} Resistor $R_3$ fails open: 
\medskip

For each of these conditions, explain {\it why} the resulting effects will occur.

\underbar{file 03934}
%(END_QUESTION)





%(BEGIN_ANSWER)

\medskip
\item{$\bullet$} Current source $I_1$ fails with increased output: {\it All resistor currents increase, all resistor voltages increase, voltage across $I_1$ increases.}
\vskip 5pt
\item{$\bullet$} Solder bridge (short) across resistor $R_1$: {\it All resistor currents remain unchanged, voltage across resistor $R_1$ decreases to zero, voltage across $I_1$ decreases by the amount $R_1$ previously dropped.}
\vskip 5pt
\item{$\bullet$} Solder bridge (short) across resistor $R_2$: {\it All resistor currents remain unchanged, voltage across resistor $R_2$ decreases to zero, voltage across $I_1$ decreases by the amount $R_2$ previously dropped.}
\vskip 5pt
\item{$\bullet$} Resistor $R_3$ fails open: {\it Theoretically, all resistor currents remain unchanged while a large arc jumps across the failed-open $R_3$ (and possibly across $I_1$ as well).  Realistically, all resistor currents will decrease to zero while the voltages across $R_3$ and $I_1$ will increase to maximum.}
\medskip

%(END_ANSWER)





%(BEGIN_NOTES)

The purpose of this question is to approach the domain of circuit troubleshooting from a perspective of knowing what the fault is, rather than only knowing what the symptoms are.  Although this is not necessarily a realistic perspective, it helps students build the foundational knowledge necessary to diagnose a faulted circuit from empirical data.  Questions such as this should be followed (eventually) by other questions asking students to identify likely faults based on measurements.

%INDEX% Troubleshooting, predicting effects of fault in current source series circuit

%(END_NOTES)


