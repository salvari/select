
%(BEGIN_QUESTION)
% Copyright 2003, Tony R. Kuphaldt, released under the Creative Commons Attribution License (v 1.0)
% This means you may do almost anything with this work of mine, so long as you give me proper credit

Calculate the total impedance offered by these three capacitors to a sinusoidal signal with a frequency of 4 kHz:

\medskip
\item{$\bullet$} $C_1 = 0.1 \> \mu \hbox{F}$
\item{$\bullet$} $C_2 = 0.047 \> \mu \hbox{F}$
\item{$\bullet$} $C_3 = 0.033 \> \mu \hbox{F}$
\medskip

$$\epsfbox{01846x01.eps}$$

State your answer in the form of a scalar number (not complex), but calculate it using two different strategies:

\medskip
\item{$\bullet$} Calculate total capacitance ($C_{total}$) first, then total impedance ($Z_{total}$).
\item{$\bullet$} Calculate individual admittances first ($Y_{C1}$, $Y_{C2}$, and $Y_{C3}$), then total impedance ($Z_{total}$).
\medskip

\underbar{file 01846}
%(END_QUESTION)





%(BEGIN_ANSWER)

\noindent
{\bf First strategy:}

$C_{total} = 0.18 \> \mu \hbox{F}$

$Z_{total} = 221 \> \Omega$

\vskip 10pt

\goodbreak

\noindent
{\bf Second strategy:}

$Y_{C1} = 2.51 \hbox{ mS}$

$Y_{C2} = 1.18 \> \hbox{ mS}$

$Y_{C3} = 829 \> \mu \hbox{S}$

$Y_{total} = 4.52 \hbox{ mS}$

$Z_{total} = 221 \> \Omega$

%(END_ANSWER)





%(BEGIN_NOTES)

This question is another example of how multiple means of calculation will give you the same answer (if done correctly!).  Make note to your students that this indicates an answer-checking strategy!

%INDEX% Impedance in parallel C circuit

%(END_NOTES)


