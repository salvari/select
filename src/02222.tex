
%(BEGIN_QUESTION)
% Copyright 2004, Tony R. Kuphaldt, released under the Creative Commons Attribution License (v 1.0)
% This means you may do almost anything with this work of mine, so long as you give me proper credit

If we were to apply a sinusoidal AC signal to the input of this transistor amplifier circuit, the output would definitely {\it not} be sinusoidal:

$$\epsfbox{02222x01.eps}$$

It should be apparent that only portions of the input are being reproduced at the output of this circuit.  The rest of the waveform seems to be "missing," being replaced by a flat line.  Explain why this transistor circuit is not able to amplify the {\it entire} waveform.

\underbar{file 02222}
%(END_QUESTION)





%(BEGIN_ANSWER)

Transistors are essentially DC devices, not AC devices.  Consider the base-emitter PN junction that the input signal is sent to: it can only conduct in one direction (base positive and emitter negative).

%(END_ANSWER)





%(BEGIN_NOTES)

Sometimes it is helpful for students to re-draw the circuit using a transistor model showing the base-emitter junction as a diode.  If you think this model would help some of your students understand the concept here, have another student draw the transistor model on the whiteboard, and use that drawing as a discussion aid.  Like any PN junction, the base-emitter junction of a BJT only "wants" to conduct current in one direction.

%INDEX% Common-collector circuit, unbiased (clipping signal)

%(END_NOTES)


