
%(BEGIN_QUESTION)
% Copyright 2003, Tony R. Kuphaldt, released under the Creative Commons Attribution License (v 1.0)
% This means you may do almost anything with this work of mine, so long as you give me proper credit

Suppose I were about to measure an unknown voltage with a manual-range voltmeter.  This particular voltmeter has several different voltage measurement ranges to choose from:

\medskip
\item{$\bullet$} 500 volts
\item{$\bullet$} 250 volts
\item{$\bullet$} 100 volts
\item{$\bullet$} 50 volts
\item{$\bullet$} 25 volts
\item{$\bullet$} 10 volts
\item{$\bullet$} 5 volts
\medskip

What range would be best to begin with, when first measuring this unknown voltage with the meter?  Explain your answer.

\underbar{file 00270}
%(END_QUESTION)





%(BEGIN_ANSWER)

Begin by setting the voltmeter to its highest range: 500 volts.  Then, see if the movement needle registers anything with the meter leads connected to the circuit.  Decide to change the meter's range based on this first indication.

%(END_ANSWER)





%(BEGIN_NOTES)

I always like to have my students begin their test equipment familiarity by using old-fashioned analog multimeters.  Only after they have learned to be proficient with an inexpensive meter do I allow them to use anything better (digital, auto-ranging) in their work.  This forces students to appreciate what a "fancy" meter does for them, as well as teach them basic principles of instrument ranging and measurement precision.

%INDEX% Voltmeter usage

%(END_NOTES)


