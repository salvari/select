
%(BEGIN_QUESTION)
% Copyright 2003, Tony R. Kuphaldt, released under the Creative Commons Attribution License (v 1.0)
% This means you may do almost anything with this work of mine, so long as you give me proper credit

After a ROM memory has been programmed with data, it is good to verify that the data now stored is okay, and not corrupted with any errors.  A popular method of doing this is to calculate a {\it checksum} on the stored data, and compare that against the checksum for the original data.  If the checksum numbers are identical, chances are there are no corruptions in the stored data.

Explain exactly what checksum is, and how it works as an error-detection strategy.

\underbar{file 01450}
%(END_QUESTION)





%(BEGIN_ANSWER)

One way to think about checksum is to recall the error-detection strategy of {\it parity bits}.  At root, the two processes are very similar.  As for the details of what checksum is and how it is calculated, I leave that for you to research!

%(END_ANSWER)





%(BEGIN_NOTES)

Once again, there is little I can reveal in the answer without giving everything away.  There are enough resources available for students to learn about checksum on their own, that you should not have to supply additional information.

%INDEX% Checksum

%(END_NOTES)


