
%(BEGIN_QUESTION)
% Copyright 2003, Tony R. Kuphaldt, released under the Creative Commons Attribution License (v 1.0)
% This means you may do almost anything with this work of mine, so long as you give me proper credit

One of the conductors connecting the secondary of a three-phase power distribution transformer to a large office building fails open.  Upon inspection, the source of the failure is obvious: the wire overheated at a point of contact with a terminal block, until it physically separated from the terminal.

$$\epsfbox{00411x01.eps}$$

What is strange, though, is that the overheated wire is the {\it neutral} conductor, not any one of the "line" conductors.  Based on this observation, what do you think caused the failure?

After repairing the wire, what would you do to verify the cause of the failure?

\underbar{file 00411}
%(END_QUESTION)





%(BEGIN_ANSWER)

Here's a hint: if you were to repair the neutral wire and take current measurements with a digital instrument (using a clamp-on current probe, for safety), you would find that the predominant frequency of the current is 180 Hz, rather than 60 Hz.

%(END_ANSWER)





%(BEGIN_NOTES)

This scenario is all too common in modern power systems, as non-linear loads such as switching power supplies and electronic power controls become more prevalent.  Special instruments exist to measure harmonics in power systems, but a simple DMM (digital multimeter) may be used as well to make crude assessments such as the one described in the Answer.

%INDEX% Harmonic currents, neutral conductor in 3-phase Y systems

%(END_NOTES)


