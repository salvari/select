
%(BEGIN_QUESTION)
% Copyright 2003, Tony R. Kuphaldt, released under the Creative Commons Attribution License (v 1.0)
% This means you may do almost anything with this work of mine, so long as you give me proper credit

Shown here is a simple circuit for constructing an extremely high input impedance voltmeter on a wireless breadboard, using one half of a TL082 dual op-amp:

$$\epsfbox{00934x01.eps}$$

Draw a schematic diagram of this circuit, a calculate the resistor value necessary to give the meter a voltage measurement range of 0 to 5 volts.

\underbar{file 00934}
%(END_QUESTION)





%(BEGIN_ANSWER)

$$\epsfbox{00934x02.eps}$$

$R =$ 5 k$\Omega$

\vskip 10pt

Follow-up question: determine the approximate input impedance of this voltmeter, and also the maximum voltage it is able to measure with {\it any} size resistor in the circuit.

%(END_ANSWER)





%(BEGIN_NOTES)

This is a very practical circuit for your students to build, and they may find it outperforms their own (purchased) voltmeters in the parameter of input impedance!  Be sure to ask them where they found the information on input impedance for the TL082 op-amp, and how they were able to determine the maximum input voltage for a circuit like this.

%INDEX% Precision voltmeter circuit, opamp

%(END_NOTES)


