
%(BEGIN_QUESTION)
% Copyright 2005, Tony R. Kuphaldt, released under the Creative Commons Attribution License (v 1.0)
% This means you may do almost anything with this work of mine, so long as you give me proper credit

An ideal (perfect) current source is an abstraction with no accurate realization in life.  However, we may approximate the behavior of an ideal current source with a high-voltage source and large series resistance:

$$\epsfbox{03293x01.eps}$$

Such a Thevenin equivalent circuit, however imperfect, will maintain a fairly constant current through a wide range of load resistance values.

\vskip 10pt

Similarly, an ideal (perfect) voltage source is an abstraction with no accurate realization in life.  Thankfully, though, it is not difficult to build voltage sources that are relatively close to perfect: circuits with very low internal resistance such that the output voltage sags only a little under high-current conditions.

But suppose we lived in a world where things were the opposite: where close-to-ideal {\it current} sources were simpler and more plentiful than close-to-ideal voltage sources.  Draw a Norton equivalent circuit showing how to approximate an ideal voltage source using an ideal (perfect) current source and a shunt resistance.

\underbar{file 03293}
%(END_QUESTION)





%(BEGIN_ANSWER)

$$\epsfbox{03293x02.eps}$$

%(END_ANSWER)





%(BEGIN_NOTES)

This question is not so much a practical one as it is designed to get students to think a little deeper about the differences between ideal voltage and current sources.  In other words, it focuses on concepts rather than application.

%INDEX% Current source model of voltage source
%INDEX% Voltage source, modeled by Norton equivalent circuit

%(END_NOTES)


