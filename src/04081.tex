
%(BEGIN_QUESTION)
% Copyright 2006, Tony R. Kuphaldt, released under the Creative Commons Attribution License (v 1.0)
% This means you may do almost anything with this work of mine, so long as you give me proper credit

A {\it Cathode Ray Tube}, or {\it CRT}, is the heart of an analog oscilloscope.  It functions by aiming a focused beam of electrons at a phosphorescent screen, causing light at the point of impact:

$$\epsfbox{04081x01.eps}$$

What style of current notation (electron or conventional) would best suit a description for the operation of a CRT?

\underbar{file 04081}
%(END_QUESTION)





%(BEGIN_ANSWER)

As with most electron tubes, {\it electron flow} makes the most sense of a CRT's operation.

%(END_ANSWER)





%(BEGIN_NOTES)

The answer given provides a clue as to why electron flow notation is still popular among technicians and the institutions that train them.  There is a legacy of electron-flow-based instruction originating from the days when vacuum tubes were the predominant active component in electronic circuits.  If you are teaching the operation of these devices in the simplest terms, so that non-engineers can understand them, it would make the most sense to standardize on a notation for current that follows the actual electrons.  Electrical engineers, on the other hand, established their own convention for designating direction of current before the electron was even discovered, which is why that branch of electrical science still denotes the direction of current opposite the direction of electron motion.

%INDEX% Cathode Ray Tube
%INDEX% CRT (Cathode Ray Tube)
%INDEX% Conventional flow versus electron flow
%INDEX% Electron flow versus conventional flow

%(END_NOTES)


