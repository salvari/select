
%(BEGIN_QUESTION)
% Copyright 2003, Tony R. Kuphaldt, released under the Creative Commons Attribution License (v 1.0)
% This means you may do almost anything with this work of mine, so long as you give me proper credit

Given the following test circuit, with an oscilloscope used to record {\it current} from the battery to the cable (measuring voltage dropped across a shunt resistor), what sort of waveform or pulse would the oscilloscope register after switch closure?

$$\epsfbox{00130x01.eps}$$

\underbar{file 00130}
%(END_QUESTION)





%(BEGIN_ANSWER)

The oscilloscope would register a square-edged pulse of voltage approximately equal to 480 $\mu$V, which of course corresponds to a current of approximately 480 mA:

$$\epsfbox{00130x02.eps}$$

The pulse duration should range somewhere between 162.67 microseconds and 170.42 microseconds (based on two different figures I obtained for RG-58/U cable velocity factors).

%(END_ANSWER)





%(BEGIN_NOTES)

Answering this question requires several steps, and the combining of multiple concepts.  It should be apparent from the answer that Ohm's Law ($I = {E \over R}$) is sufficient for calculating pulse current, but the time delay figure given in the answer may confuse some students.  For those students who calculate a time figure that is half as much as the one given in the answer, encourage them to think of why their (incorrect) answer might have been off by 50\%.  The existence of a 2:1 ratio such as this implies a simple conceptual misunderstanding.

For the RG-58/U cable velocity factor, I obtained two different figures: 0.63 and 0.66, which accounts for the two time delay answers given.

%INDEX% Transmission line, current analysis (unterminated cable)

%(END_NOTES)


