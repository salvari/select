
%(BEGIN_QUESTION)
% Copyright 2005, Tony R. Kuphaldt, released under the Creative Commons Attribution License (v 1.0)
% This means you may do almost anything with this work of mine, so long as you give me proper credit

Some operational amplifiers come equipped with compensation capacitors built inside.  The classic 741 design is one such opamp:

$$\epsfbox{02592x01.eps}$$

Find the compensation capacitor in this schematic diagram, and identify how it provides frequency-dependent negative feedback within the opamp to reduce gain at high frequencies.

\underbar{file 02592}
%(END_QUESTION)





%(BEGIN_ANSWER)

Identifying the capacitor is easy: it is the only one in the whole circuit!  It couples signal from the collector of $Q_{17}$, which is an active-loaded common-emitter amplifier, to the base of $Q_{16}$, which is an emitter-follower driving $Q_{17}$.  Since $Q_{17}$ inverts the signal applied to $Q_{16}$'s base, the feedback is degenerative.

%(END_ANSWER)





%(BEGIN_NOTES)

Answering this question will require a review of basic transistor amplifier theory, specifically different configurations of transistor amplifiers and their respective signal phase relationships.

%INDEX% Compensation, opamp (internal)
%INDEX% Compensation capacitor, internal to opamp
%INDEX% Opamp compensation (internal)

%(END_NOTES)


