
%(BEGIN_QUESTION)
% Copyright 2003, Tony R. Kuphaldt, released under the Creative Commons Attribution License (v 1.0)
% This means you may do almost anything with this work of mine, so long as you give me proper credit

What will happen if a cable is terminated by a resistor of incorrect value (not equal to the cable's characteristic impedance)?

\underbar{file 00135}
%(END_QUESTION)





%(BEGIN_ANSWER)

Any terminating resistance not equal to the cable's characteristic resistance (either too small or too large) will result in reflected waves, albeit at lesser amplitude than if the cable were either unterminated or terminated by a direct short.

%(END_ANSWER)





%(BEGIN_NOTES)

Answering this question is an exercise in qualitative thinking: compare the results of termination with the proper amount of resistance, versus termination with infinite or zero resistance.  A terminating resistor of improper value will produce an effect somewhere between these extreme cases.

For instance, compare the cable impedance (as "seen" by the voltage source after a substantial amount of time) for a properly terminated cable, versus one that is either open-ended or shorted.  What would a cable terminated by an improper-value resistor "look" like to the source after the propagation delay time has passed?

%INDEX% Termination resistor, proper sizing (for transmission line)

%(END_NOTES)


