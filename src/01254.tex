
%(BEGIN_QUESTION)
% Copyright 2003, Tony R. Kuphaldt, released under the Creative Commons Attribution License (v 1.0)
% This means you may do almost anything with this work of mine, so long as you give me proper credit

The simplest type of digital logic circuit is an {\it inverter}, also called an {\it inverting buffer}, or {\it NOT gate}.  Here is a schematic diagram for an inverter gate constructed from complementary MOSFETs (CMOS), shown connected to a SPDT switch and an LED:

$$\epsfbox{01254x01.eps}$$

Determine the status of the LED in each of the input switch's two positions.  Denote the logic level of switch and LED in the form of a truth table:

$$\epsfbox{01254x02.eps}$$

\underbar{file 01254}
%(END_QUESTION)





%(BEGIN_ANSWER)

$$\epsfbox{01254x03.eps}$$

%(END_ANSWER)





%(BEGIN_NOTES)

Have your students explain the operation of this very simple MOSFET circuit, describing how the inverse logic state is generated at the output terminal, from a given input state.  Discuss with your students the simplicity of the CMOS inverter, especially contrasted against a TTL inverter circuit.

%INDEX% CMOS gate circuit, internal schematic

%(END_NOTES)


