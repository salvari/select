
%(BEGIN_QUESTION)
% Copyright 2003, Tony R. Kuphaldt, released under the Creative Commons Attribution License (v 1.0)
% This means you may do almost anything with this work of mine, so long as you give me proper credit

If a digital meter display has four digits, it can represent any number from 0000 to 9999.  This constitutes ten thousand unique numbers representable by the display.  How many unique numbers could be represented by five digits?  By six digits?  

If an ancient Mayan ledger had spaces for writing numbers with three "digits" each, how many unique numbers could be represented in each space?  What if the spaces were expanded to hold four "digits" each?

If a digital circuit has four {\it bits}, how many unique binary numbers can it represent?  If we expanded its capabilities to eight bits, how many unique numbers could be represented by the circuit?

\vskip 10pt

After answering these questions, do you see any mathematical pattern relating the number of "places" in a numeration system and the number of unique quantities that may be represented, given the "base" value ("radix") of the numeration system?  Write a mathematical expression that solves for the number of unique quantities representable, given the "base" of the system and the number of "places".

\underbar{file 01199}
%(END_QUESTION)





%(BEGIN_ANSWER)

Five denary digits: one hundred thousand unique numbers.  Six denary digits: one million unique numbers.

\vskip 10pt

Three vigesimal "digits": eight thousand unique numbers.  Four vigesimal "digits": one hundred sixty thousand unique numbers.

\vskip 10pt

Four binary bits: sixteen unique numbers.  Eight binary bits: two hundred fifty six unique numbers.

\vskip 10pt

$$\hbox{Unique numbers } = b^n$$

\noindent
Where,

$b =$ Radix of numeration system.

$n =$ Number of "places" given.

%(END_ANSWER)





%(BEGIN_NOTES)

Ask your students which numeration system of the three given is more efficient in representing quantities with the smallest number of places.  Then, ask them to explain {\it why} this is.

%INDEX% Maximum count, any numeration system

%(END_NOTES)


