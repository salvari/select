
%(BEGIN_QUESTION)
% Copyright 2003, Tony R. Kuphaldt, released under the Creative Commons Attribution License (v 1.0)
% This means you may do almost anything with this work of mine, so long as you give me proper credit

A balanced, three-phase power system has a line voltage of 13.8 kV volts and a line current of 150 amps.  How much power is being delivered to the load (assuming a power factor of 1)?

$$\epsfbox{00414x01.eps}$$

A 13.8 kV single-phase system could be designed to provide the same amount of power to a load, but it would require heavier-gauge (more expensive!) conductors.  Determine the extra percentage of expense in wire cost (based on the weight of the wires) resulting from the use of single-phase instead of three-phase.

$$\epsfbox{00414x02.eps}$$

\underbar{file 00414}
%(END_QUESTION)





%(BEGIN_ANSWER)

$P_{load} =$ 3.59 MW

\vskip 5pt

A single-phase system operating at 13.8 kV would require at least 1/0 ("one-ought") gauge copper conductors to transmit 3.59 MW of power.  A three-phase system operating at the same voltage would require at least \#2 gauge copper conductors.

%(END_ANSWER)





%(BEGIN_NOTES)

This question is good for provoking discussion on the practical, cost-saving benefits of three-phase power systems over single-phase.  Your students will have to do some "review" research on wire gauges and weights, but the exercise is well worth it.

%INDEX% Polyphase versus single-phase
%INDEX% Single-phase versus polyphase

%(END_NOTES)


