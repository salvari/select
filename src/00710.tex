
%(BEGIN_QUESTION)
% Copyright 2003, Tony R. Kuphaldt, released under the Creative Commons Attribution License (v 1.0)
% This means you may do almost anything with this work of mine, so long as you give me proper credit

Describe what will happen to the impedance of both the capacitor and the resistor as the input signal frequency increases:

$$\epsfbox{00710x01.eps}$$

Also, describe what result the change in impedances will have on the op-amp circuit's voltage gain.  If the input signal amplitude remains constant as frequency increases, what will happen to the amplitude of the output voltage?  What type of filtering function does this behavior represent?

\underbar{file 00710}
%(END_QUESTION)





%(BEGIN_ANSWER)

As the frequency of $V_{in}$ increases, $Z_C$ decreases and $Z_R$ remains unchanged.  This will result in a decreased $A_V$ for the amplifier circuit.

\vskip 10pt

Challenge question: explain why this type of circuit is usually equipped with a high-value resistor ($R_2$) in parallel with the feedback capacitor:

$$\epsfbox{00710x02.eps}$$

%(END_ANSWER)





%(BEGIN_NOTES)

This same op-amp circuit is known by a particular name when used with DC input signals.  Ask your students what this design of circuit is called.  When receiving a DC input signal, what function does it serve?  The answer to this is key to answering the "challenge" question.  

%(END_NOTES)


