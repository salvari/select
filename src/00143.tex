
%(BEGIN_QUESTION)
% Copyright 2003, Tony R. Kuphaldt, released under the Creative Commons Attribution License (v 1.0)
% This means you may do almost anything with this work of mine, so long as you give me proper credit

{\it Lightning} is a spectacular example of naturally-generated static electricity.  Explain how the vast static electric charges that cause lightning are formed by natural processes, and how those processes relate to the function of a device called a {\it Van de Graaff generator}.

\underbar{file 00143}
%(END_QUESTION)





%(BEGIN_ANSWER)

The static electric charges that cause lightning are generated by the action of wind and rain transporting electrons between clouds and earth, or between clouds.  Particles of dust may also transport electrons in the same manner, which accounts for the spectacular discharges of lightning often seen during massive uprisings of dust, such as during volcanic eruptions.

In a similar manner, a Van de Graaff generator moves electrons from one point to another on a moving loop of fabric: a sort of conveyor belt for electrons.

%(END_ANSWER)





%(BEGIN_NOTES)

Lighting provides a universal example of static electricity on a grand scale.  The sheer awe of lightning is itself an attention-grabbing factor making it all the more appropriate for classroom discussion.  When students' collective attention and imagination are enthralled by fascinating subjects, learning is enhanced!

It goes without saying that the presence of a Van de Graaff generator in the laboratory is an excellent enhancement for this topic, as well.

%INDEX% Lightning
%INDEX% Van de Graaff generator

%(END_NOTES)


