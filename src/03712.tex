
%(BEGIN_QUESTION)
% Copyright 2005, Tony R. Kuphaldt, released under the Creative Commons Attribution License (v 1.0)
% This means you may do almost anything with this work of mine, so long as you give me proper credit

Predict how all component voltages and currents in this circuit will be affected as a result of the following faults.  Consider each fault independently (i.e. one at a time, no multiple faults):

$$\epsfbox{03712x01.eps}$$

\medskip
\item{$\bullet$} Transistor $Q_1$ fails open (collector-to-emitter):
\vskip 5pt
\item{$\bullet$} Transistor $Q_1$ fails shorted (collector-to-emitter):
\vskip 5pt
\item{$\bullet$} Reluctor magnet weakens:
\vskip 5pt
\item{$\bullet$} Capacitor $C_1$ fails shorted:
\vskip 5pt
\item{$\bullet$} Capacitor $C_1$ fails open:
\vskip 5pt
\item{$\bullet$} Transformer ("coil") $T_1$ primary winding fails open:
\vskip 5pt
\item{$\bullet$} Transformer ("coil") $T_1$ secondary winding fails open:
\medskip

For each of these conditions, explain {\it why} the resulting effects will occur.

\underbar{file 03712}
%(END_QUESTION)





%(BEGIN_ANSWER)

\medskip
\item{$\bullet$} Transistor $Q_1$ fails open (collector-to-emitter): {\it No current through $T_1$ primary or secondary, no high-voltage pulses across $T_1$ secondary, full 12 volts (constant) across $C_1$, voltage pulses still seen across $L_1$, no spark at spark plug.}
\vskip 5pt
\item{$\bullet$} Transistor $Q_1$ fails shorted (collector-to-emitter): {\it Constant current through $T_1$ primary, no high-voltage pulses across $T_1$ secondary, nearly 0 volts across $C_1$, very weak voltage pulses across $L_1$, no spark at spark plug.}
\vskip 5pt
\item{$\bullet$} Reluctor magnet weakens: {\it Smaller voltage pulses across $L_1$, smaller current pulses through $T_1$ primary, smaller voltage pulses across $T_2$ secondary, weak or no spark at spark plug.}
\vskip 5pt
\item{$\bullet$} Capacitor $C_1$ fails shorted: {\it Constant current through $T_1$ primary, no high-voltage pulses across $T_1$ secondary, nearly 0 volts across $C_1$, normal voltage pulses across $L_1$, no spark at spark plug.}
\vskip 5pt
\item{$\bullet$} Capacitor $C_1$ fails open: {\it Excessive voltage pulses seen at $Q_1$ collector (with respect to ground), very rapid failure of $Q_1$.}
\vskip 5pt
\item{$\bullet$} Transformer ("coil") $T_1$ primary winding fails open: {\it No current through $T_1$ primary or secondary, no high-voltage pulses across $T_1$ secondary, zero volts (constant) across $C_1$, voltage pulses still seen across $L_1$, no spark at spark plug.}
\vskip 5pt
\item{$\bullet$} Transformer ("coil") $T_1$ secondary winding fails open: {\it All voltages and currents fairly normal except for no voltage across $T_1$ secondary and no spark at spark plug, perhaps slightly greater voltage pulses seen at $Q_1$ collector with respect to ground.}
\medskip

%(END_ANSWER)





%(BEGIN_NOTES)

The purpose of this question is to approach the domain of circuit troubleshooting from a perspective of knowing what the fault is, rather than only knowing what the symptoms are.  Although this is not necessarily a realistic perspective, it helps students build the foundational knowledge necessary to diagnose a faulted circuit from empirical data.  Questions such as this should be followed (eventually) by other questions asking students to identify likely faults based on measurements.

%INDEX% Troubleshooting, predicting effects of fault in simple electronic ignition circuit

%(END_NOTES)


