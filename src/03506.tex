
%(BEGIN_QUESTION)
% Copyright 2005, Tony R. Kuphaldt, released under the Creative Commons Attribution License (v 1.0)
% This means you may do almost anything with this work of mine, so long as you give me proper credit

Examine the following schematic diagram for an audio tone control circuit:

$$\epsfbox{03506x01.eps}$$

Determine which potentiometer controls the bass (low frequency) tones and which controls the treble (high frequency) tones, and explain how you made those determinations.

\underbar{file 03506}
%(END_QUESTION)





%(BEGIN_ANSWER)

$$\epsfbox{03506x02.eps}$$

%(END_ANSWER)





%(BEGIN_NOTES)

The most important answer to this question is {\it how} your students arrived at the correct potentiometer identifications.  If none of your students were able to figure out how to identify the potentiometers, give them this tip: use the {\it superposition theorem} to analyze the response of this circuit to both low-frequency signals and high-frequency signals.  Assume that for bass tones the capacitors are opaque ($Z = \infty$) and that for treble tones they are transparent ($Z = 0$).  The answers should be clear if they follow this technique.

This general problem-solving technique -- analyzing two or more "extreme" scenarios to compare the results -- is an important one for your students to become familiar with.  It is extremely helpful in the analysis of filter circuits!

%INDEX% Tone control circuit

%(END_NOTES)


