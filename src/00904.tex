
%(BEGIN_QUESTION)
% Copyright 2003, Tony R. Kuphaldt, released under the Creative Commons Attribution License (v 1.0)
% This means you may do almost anything with this work of mine, so long as you give me proper credit

In any electrically conductive substance, what are {\it charge carriers}?  Identify the charge carriers in metallic substances, semiconducting substances, and conductive liquids.

\underbar{file 00904}
%(END_QUESTION)





%(BEGIN_ANSWER)

"Charge carriers" are any particles possessing an electrical charge, whose coordinated motion through a substance constitutes an electric current.  Different types of substances have different charge carriers:

\medskip
\item{$\bullet$} Metals: "free" (conduction-band) electrons
\item{$\bullet$} Semiconductors: electrons and holes
\item{$\bullet$} Liquids: ions
\medskip

%(END_ANSWER)





%(BEGIN_NOTES)

Metals are by far the simplest materials to understand with reference to electrical conduction.  Point out to your students that it is this simplicity that makes metallic conduction so easy to mathematically model (Ohm's Law, $E = I R$).

%INDEX% Charge carriers
%INDEX% Electrons versus holes
%INDEX% Holes versus electrons

%(END_NOTES)


