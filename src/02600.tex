
%(BEGIN_QUESTION)
% Copyright 2005, Tony R. Kuphaldt, released under the Creative Commons Attribution License (v 1.0)
% This means you may do almost anything with this work of mine, so long as you give me proper credit

Some operational amplifiers are {\it internally compensated}, while others are {\it externally compensated}.  Explain the difference between the two.  Hint: examples of each include the classic LM741 and LM101 operational amplifiers.  Research their respective datasheets to see what you find on compensation!

\underbar{file 02600}
%(END_QUESTION)





%(BEGIN_ANSWER)

The difference is the physical location of the compensating capacitor, whether it is a part of the integrated circuit or external to it.

\vskip 10pt

Follow-up question: show how an external compensating capacitor may be connected to an opamp such as the LM101.

%(END_ANSWER)





%(BEGIN_NOTES)

Ask your students to explain why we might wish to use either type of opamp when building a circuit.  In what applications do they think an internally-compensated opamp would be better, and in what applications do they think an externally-compensated opamp would be preferable?

%INDEX% Compensation capacitor, internal versus external (opamp)

%(END_NOTES)


