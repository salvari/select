
%(BEGIN_QUESTION)
% Copyright 2006, Tony R. Kuphaldt, released under the Creative Commons Attribution License (v 1.0)
% This means you may do almost anything with this work of mine, so long as you give me proper credit

Predict how the operation of this "concentrator" circuit (which takes eight digital inputs and "concentrates" them into a single, multiplexed, communication line to be expanded into eight outputs at the receiving end) will be affected as a result of the following faults.  Consider each fault independently (i.e. one at a time, no multiple faults):

$$\epsfbox{03918x01.eps}$$

\medskip
\item{$\bullet$} Clock pulse generator stops pulsing:
\vskip 5pt
\item{$\bullet$} Pin breaks on the $W$ output of 74151 chip, leaving that wire floating:
\vskip 5pt
\item{$\bullet$} Pin breaks on G2A input of 74138 chip, leaving it floating:
\vskip 5pt
\item{$\bullet$} Enable pin breaks on 74151 chip, leaving it floating:
\medskip

For each of these conditions, explain {\it why} the resulting effects will occur.

\underbar{file 03918}
%(END_QUESTION)





%(BEGIN_ANSWER)

\medskip
\item{$\bullet$} Clock pulse generator stops pulsing: {\it Only one channel out of the eight will work, and it works all the time without interruption.  Data cannot get through any of the other seven channels.}
\vskip 5pt
\item{$\bullet$} Pin breaks on the $W$ output of 74151 chip, leaving that wire floating: {\it All selected outputs on the 74138 chip go low, instead of repeating the respective logic states input at the 74151 chip.}
\vskip 5pt
\item{$\bullet$} Pin breaks on G2A input of 74138 chip, leaving it floating: {\it All outputs on the 74138 chip go high, all the time.}
\vskip 5pt
\item{$\bullet$} Enable pin breaks on 74151 chip, leaving it floating: {\it All selected outputs on the 74138 chip go low, instead of repeating the respective logic states input at the 74151 chip.}
\medskip

%(END_ANSWER)





%(BEGIN_NOTES)

The purpose of this question is to approach the domain of circuit troubleshooting from a perspective of knowing what the fault is, rather than only knowing what the symptoms are.  Although this is not necessarily a realistic perspective, it helps students build the foundational knowledge necessary to diagnose a faulted circuit from empirical data.  Questions such as this should be followed (eventually) by other questions asking students to identify likely faults based on measurements.

%INDEX% Troubleshooting, predicting effects of fault in mux/demux circuit

%(END_NOTES)


