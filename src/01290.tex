
%(BEGIN_QUESTION)
% Copyright 2003, Tony R. Kuphaldt, released under the Creative Commons Attribution License (v 1.0)
% This means you may do almost anything with this work of mine, so long as you give me proper credit

A type of electrical diagram convention optimal for representing electromechanical relay circuits is the {\it ladder logic diagram}.  An example of a "ladder logic" diagram is shown here:

\vskip 20pt

$$\epsfbox{01290x01.eps}$$

\vskip 40pt

Each parallel circuit branch is represented as its own horizontal "rung" between the two vertical "rails" of the ladder.  As you may have noticed, some of the symbols resemble standard electrical/electronic schematic symbols (toggle switches, for instance), while others are unique to ladder logic diagrams (heater elements, solenoid coils, lamps).

Where do the circuits shown obtain their electrical power?  What do "L1" and "L2" represent?  How are relay coils and contacts represented in a ladder logic diagram?  Answer each of these questions by expanding upon the diagram shown above: draw the components necessary to show a complete electrical circuit (i.e. details of the power source), as well as an additional rung (or two) showing a relay coil actuated by some sort of switch contact, and the relay contact controlling power to a second indicator lamp.

\underbar{file 01290}
%(END_QUESTION)





%(BEGIN_ANSWER)

This is just one example of how the ladder logic diagram could be expanded:

$$\epsfbox{01290x02.eps}$$

"L1" and "L2" represent the "hot" and "neutral" lines, respectively, in a 120 volt AC power system.  Often, the control circuit power is obtained from a step-down transformer, which is in turn fed by a higher voltage source (usually one phase of a 480 volt AC three-phase system, in American industrial applications).

%(END_ANSWER)





%(BEGIN_NOTES)

If students don't raise this point on their own, direct their attention to the relay coil and contact symbols.  What looks strange here?  What sort of electrical component are students familiar associating with the "CR1" contact symbol?  Does it make sense to use this symbol to symbolize a normally-open switch (relay) contact?  If we wished to show a normally-closed relay contact instead, how would we modify the diagram?

%INDEX% Ladder logic diagram

%(END_NOTES)


