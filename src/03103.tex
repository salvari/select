
%(BEGIN_QUESTION)
% Copyright 2005, Tony R. Kuphaldt, released under the Creative Commons Attribution License (v 1.0)
% This means you may do almost anything with this work of mine, so long as you give me proper credit

Suppose you needed to choose a potentiometer value ($R$) to make a voltage divider circuit, given a known fixed resistor value, the source voltage value, and the desired range of adjustment:

$$\epsfbox{03103x01.eps}$$

Solve for $R$, and show the equation you set up in order to do it.

\vskip 10pt

Hint: remember the series resistor voltage divider formula . . .

$$V_R = V_{total}\left( {R \over R_{total}} \right)$$

\underbar{file 03103}
%(END_QUESTION)





%(BEGIN_ANSWER)

$R =$ 121.43 k$\Omega$

\vskip 10pt

Follow-up question: you will not be able to find a potentiometer with a full-range resistance value of exactly 121.43 k$\Omega$.  Describe how you could take a standard-value potentiometer and connect it to one or more fixed-value resistors to give it this desired full-scale range.

%(END_ANSWER)





%(BEGIN_NOTES)

Be sure to have your students set up their equations in front of the class so everyone can see how they did it.  Some students may opt to apply Ohm's Law to the solution of $R$, which is good, but for the purpose of developing equations to fit problems it might not be the best solution.  Challenge your students to come up with a {\it single equation} that solves for $R$, with all known quantities on the other side of the "equal" sign.

The follow-up question is very practical, as it is impossible to find potentiometers ready-made to arbitrary values of full-scale resistance.  Instead, you must work with what you can find, which is usually nominal values such as 10 k$\Omega$, 100 k$\Omega$, 1 M$\Omega$, etc.

%INDEX% Simultaneous equations
%INDEX% Systems of nonlinear equations

%(END_NOTES)


