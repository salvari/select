
%(BEGIN_QUESTION)
% Copyright 2005, Tony R. Kuphaldt, released under the Creative Commons Attribution License (v 1.0)
% This means you may do almost anything with this work of mine, so long as you give me proper credit

A student is asked to calculate the phase shift for the following circuit's output voltage, relative to the phase of the source voltage:

$$\epsfbox{03748x01.eps}$$

He recognizes this as a series circuit, and therefore realizes that a right triangle would be appropriate for representing component impedances and component voltage drops (because both impedance and voltage are quantities that add in series, and the triangle represents phasor addition):

$$\epsfbox{03748x02.eps}$$

The problem now is, which angle does the student solve for in order to find the phase shift of $V_{out}$?  The triangle contains two angles besides the 90$^{o}$ angle, $\Theta$ and $\Phi$.  Which one represents the output phase shift, and more importantly, {\it why}?

\underbar{file 03748}
%(END_QUESTION)





%(BEGIN_ANSWER)

The proper angle in this circuit is $\Theta$, and it will be a positive (leading) quantity.

%(END_ANSWER)





%(BEGIN_NOTES)

Too many students blindly use impedance and voltage triangles without really understand what they are and why they work.  These same students will have no idea how to approach a problem like this.  Work with them to help them understand!

%INDEX% Phase shift calculation, RC circuit

%(END_NOTES)


