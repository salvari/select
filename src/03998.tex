
%(BEGIN_QUESTION)
% Copyright 2006, Tony R. Kuphaldt, released under the Creative Commons Attribution License (v 1.0)
% This means you may do almost anything with this work of mine, so long as you give me proper credit

The following circuit generates an analog output voltage proportional to the value of the binary input, using pulse-width modulation (PWM) as an interim format.  An eight-bit binary counter ({\bf CTR}) continually counts in the "up" direction, while an 8-bit magnitude comparator ({\bf CMP}) checks when the 8-bit binary input value matches the counter's output value.  The AND gate and inverter simply prevent the S-R latch from being "set" and "reset" simultaneously (when both A and B are maximum, both at a hex value of \$FF), which would cause the output to be "invalid" when S and R were both active, and unpredictable when both S and R inputs returned to their inactive states:

$$\epsfbox{03998x01.eps}$$

Explain how this circuit works, using timing diagrams if necessary to help show the PWM signal at $\overline{Q}$ for different input values.

\vskip 150pt

\underbar{file 03998}
%(END_QUESTION)





%(BEGIN_ANSWER)

Here is a timing diagram to help get you started on a complete answer:

$$\epsfbox{03998x02.eps}$$

I'll leave it to you to explain the relationship between the input value (A), the PWM duty cycle, and the analog output voltage.

%(END_ANSWER)





%(BEGIN_NOTES)

This circuit provides students with an interesting exercise in timing analysis, as well as being a simple means of converting large binary values into analog output voltages without resorting to using {\it large} resistor networks.

%INDEX% Digital-to-analog conversion
%INDEX% Magnitude comparator, used in DAC
%INDEX% PWM (pulse-width modulation), as a means of converting digital to analog analog

%(END_NOTES)


