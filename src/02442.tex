
%(BEGIN_QUESTION)
% Copyright 2005, Tony R. Kuphaldt, released under the Creative Commons Attribution License (v 1.0)
% This means you may do almost anything with this work of mine, so long as you give me proper credit

Calculate the approximate voltage gain ($A_V$) for the following common-emitter amplifier circuit, and also calculate the quiescent DC voltages measured at the three terminals of the transistor with respect to ground ($V_B$, $V_E$, and $V_C$).  Assume a silicon transistor:

$$\epsfbox{02442x01.eps}$$

\medskip
\goodbreak
\item{$\bullet$} $A_V \approx$ 
\item{$\bullet$} $V_B \approx$  
\item{$\bullet$} $V_E \approx$ 
\item{$\bullet$} $V_C \approx$ 
\medskip

\underbar{file 02442}
%(END_QUESTION)





%(BEGIN_ANSWER)

\medskip
\goodbreak
\item{$\bullet$} $A_V \approx$ 4.55
\item{$\bullet$} $V_B \approx$ 2.125 volts
\item{$\bullet$} $V_E \approx$ 1.425 volts
\item{$\bullet$} $V_C \approx$ 9.521 volts
\medskip

%(END_ANSWER)





%(BEGIN_NOTES)

Nothing much to comment on here -- just some practice on common-emitter amplifier calculations.  Note that the approximations given here are based on the following assumptions:

\medskip
\goodbreak
\item{$\bullet$} 0.7 volts drop (exactly) across base-emitter junction.
\item{$\bullet$} Infinite DC current gain ($\beta$) for transistor ($I_B = 0 \> \mu$A ; $I_C = I_E$).
\item{$\bullet$} Negligible loading of bias voltage divider by the emitter resistance.
\item{$\bullet$} Negligible dynamic emitter resistance ($r'_e = 0 \> \Omega$ )
\medskip

This question lends itself well to group discussions on component failure scenarios.  After discussing how to calculate the requested values, you might want to ask students to consider how these values would change given some specific component failures (open resistors, primarily, since this is perhaps the most common way that a resistor could fail).

%INDEX% Approximations for common-emitter amplifier circuit
%INDEX% Common-emitter circuit, approximating voltage gain

%(END_NOTES)


