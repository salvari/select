
%(BEGIN_QUESTION)
% Copyright 2003, Tony R. Kuphaldt, released under the Creative Commons Attribution License (v 1.0)
% This means you may do almost anything with this work of mine, so long as you give me proper credit

Calculate $V_A$ (voltage at point {\bf A} with respect to ground) and $V_B$ (voltage at point {\bf B} with respect to ground) in the following circuit:

$$\epsfbox{01727x01.eps}$$

Now, calculate the voltage between points {\bf A} and {\bf B} ($V_{AB}$).

\underbar{file 01727}
%(END_QUESTION)





%(BEGIN_ANSWER)

$V_A =$ 65.28 V

$V_B =$ -76.74 V

$V_{AB} =$ 142.02 V (point {\bf A} being positive relative to point {\bf B})

\vskip 10pt

If you are experiencing difficulty in your analysis of this circuit, you might want to refer to this re-drawing:

$$\epsfbox{01727x02.eps}$$

To make it even easier to visualize, remove the ground symbols and insert a wire connecting the lower wires of each circuit together:

$$\epsfbox{01727x03.eps}$$

It's all the same circuit, just different ways of drawing it!

\vskip 10pt

Follow-up question: identify, for a person standing on the ground (with feet electrically common to the ground symbols in the circuits), all the points on the circuits which would be safe to touch without risk of electric shock. 

%(END_ANSWER)





%(BEGIN_NOTES)

New students often experience difficulty with problems such as this, where ground connections are located in "strange" places.  The alternate diagram may be helpful in this case.

The follow-up question challenges students to apply the practical rule-of-thumb (30 volts or more is considered potentially a source of electric shock) to a circuit that is otherwise quite abstract.

%INDEX% Voltage divider

%(END_NOTES)


