
%(BEGIN_QUESTION)
% Copyright 2004, Tony R. Kuphaldt, released under the Creative Commons Attribution License (v 1.0)
% This means you may do almost anything with this work of mine, so long as you give me proper credit

Define what a {\it common-source} transistor amplifier circuit is.  What distinguishes this amplifier configuration from the other single-FET amplifier configurations, namely {\it common-drain} and {\it common-gate}?  What configuration of BJT amplifier circuit does the common-source FET circuit most resemble in form and behavior?

Also, describe the typical voltage gains of this amplifier configuration, and whether it is {\it inverting} or {\it noninverting}.

\underbar{file 02247}
%(END_QUESTION)





%(BEGIN_ANSWER)

The common-source amplifier configuration is defined by having the input and output signals referenced to the gate and drain terminals (respectively), with the source terminal of the transistor typically having a low AC impedance to ground and thus being "common" to one pole of both the input and output voltages.  

The common-source amplifier configuration most resembles the common-emitter BJT amplifier configuration in both form and behavior. 

Common-source amplifiers are characterized by moderate voltage gains, and an inverting phase relationship between input and output.

%(END_ANSWER)





%(BEGIN_NOTES)

The answers to the question may be easily found in any fundamental electronics text, but it is important to ensure students know {\it why} these characteristics are such.  I always like to tell my students, "Memory {\it will} fail you, so you need to build an understanding of {\it why} things are, not just {\it what} things are."

One exercise you might have your students do is come up to the board in front of the room and draw an example of this circuit, then everyone may refer to the drawn image when discussing the circuit's characteristics.

%INDEX% Common-source amplifier, characteristics of

%(END_NOTES)


