
%(BEGIN_QUESTION)
% Copyright 2003, Tony R. Kuphaldt, released under the Creative Commons Attribution License (v 1.0)
% This means you may do almost anything with this work of mine, so long as you give me proper credit

% Uncomment the following line if the question involves calculus at all:
\vbox{\hrule \hbox{\strut \vrule{} $\int f(x) \> dx$ \hskip 5pt {\sl Calculus alert!} \vrule} \hrule}

What happens to the inductance of an inductor as its core becomes {\it saturated}?  Does the inductance value increase, decrease, or remain the same?  Explain your answer.

\underbar{file 00473}
%(END_QUESTION)





%(BEGIN_ANSWER)

As the core of an inductor becomes saturated with magnetic flux, there will be less change in flux for a given change in current (the derivative $d\phi \over di$ will be less):

$$\epsfbox{00473x01.eps}$$

This causes the inductance to {\it decrease}.

%(END_ANSWER)





%(BEGIN_NOTES)

Ask your students to identify what condition(s) might lead to a condition of saturation.  And, even if the extreme ends of the $B-H$ curve are avoided, what does the nonlinear shape of the $B-H$ plot indicate about the linearity of an inductor?  The inductance formula ($e = L{di \over dt}$) assumes perfect linearity, but is this really true for an inductor whose core exhibits this kind of magnetic flux/force relationship?

%INDEX% Saturation, magnetic
%INDEX% Saturation, effect on inductance
%INDEX% Calculus, derivative

%(END_NOTES)


