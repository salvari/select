
%(BEGIN_QUESTION)
% Copyright 2005, Tony R. Kuphaldt, released under the Creative Commons Attribution License (v 1.0)
% This means you may do almost anything with this work of mine, so long as you give me proper credit

Determine the amount of time needed for the capacitor voltage ($V_C$) to fall to the specified levels after the switch is thrown to the "discharge" position, assuming it had first been charged to full battery voltage:

$$\epsfbox{02943x01.eps}$$

% No blank lines allowed between lines of an \halign structure!
% I use comments (%) instead, so that TeX doesn't choke.

$$\vbox{\offinterlineskip
\halign{\strut
\vrule \quad\hfil # \ \hfil & 
\vrule \quad\hfil # \ \hfil \vrule \cr
\noalign{\hrule}
%
% First row
$V_C$ & Time  \cr
%
\noalign{\hrule}
%
% Second row
10 volts &  \cr
%
\noalign{\hrule}
%
% Third row
8 volts &  \cr
%
\noalign{\hrule}
%
% Fourth row
6 volts &  \cr
%
\noalign{\hrule}
%
% Fifth row
4 volts &  \cr
%
\noalign{\hrule}
%
% Sixth row
2 volts &  \cr
%
\noalign{\hrule}
} % End of \halign 
}$$ % End of \vbox

Trace the direction of electron flow in the circuit, and also mark all voltage polarities.

\underbar{file 02943}
%(END_QUESTION)





%(BEGIN_ANSWER)

$$\epsfbox{02943x02.eps}$$

% No blank lines allowed between lines of an \halign structure!
% I use comments (%) instead, so that TeX doesn't choke.

$$\vbox{\offinterlineskip
\halign{\strut
\vrule \quad\hfil # \ \hfil & 
\vrule \quad\hfil # \ \hfil \vrule \cr
\noalign{\hrule}
%
% First row
$V_C$ & Time  \cr
%
\noalign{\hrule}
%
% Second row
10 volts & 588.9 ms \cr
%
\noalign{\hrule}
%
% Third row
8 volts & 1.31 s \cr
%
\noalign{\hrule}
%
% Fourth row
6 volts & 2.24 s \cr
%
\noalign{\hrule}
%
% Fifth row
4 volts & 3.55 s \cr
%
\noalign{\hrule}
%
% Sixth row
2 volts & 5.79 s \cr
%
\noalign{\hrule}
} % End of \halign 
}$$ % End of \vbox

%(END_ANSWER)





%(BEGIN_NOTES)

Ask your students to explain how they set up each calculation.

%INDEX% Time constant calculation, RC circuit (calculating time required to discharge to specified amount)

%(END_NOTES)


