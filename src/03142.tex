
%(BEGIN_QUESTION)
% Copyright 2005, Tony R. Kuphaldt, released under the Creative Commons Attribution License (v 1.0)
% This means you may do almost anything with this work of mine, so long as you give me proper credit

The circuit shown here provides two-direction control (forward and reverse) for a three-phase electric motor:

$$\epsfbox{03142x01.eps}$$

Explain how the reversal of motor direction is accomplished with two different motor starters, M1 and M2.  Also, explain why there is only one set of overload heaters instead of two (one for forward and one for reverse).  Finally, explain the purpose of the normally-closed contacts in series with each starter coil.

\underbar{file 03142}
%(END_QUESTION)





%(BEGIN_ANSWER)

Motor reversal is accomplished by reversing the phase sequence of the three-phase power going to the motor (from ABC to ACB).  The existence of only one set (three) heaters may be adequately explained if you consider a scenario where the motor overheats after being run in the "Forward" direction, then an immediate attempt is made to run it in "Reverse."  Finally, the NC contacts (typically called {\it interlock} contacts) prevent lots of sparks from flying if both pushbuttons are simultaneously pressed!

%(END_ANSWER)





%(BEGIN_NOTES)

Ask your students to explain exactly {\it why} "sparks [would fly]" if both pushbuttons were pressed at the same time.  The name commonly given to the NC contacts is {\it interlock}, because each one "locks out" the other starter from being energized.

%INDEX% Control circuit, AC motor
%INDEX% Interlock contact, motor control circuit
%INDEX% Ladder logic diagram
%INDEX% Overload "heater," motor control circuit
%INDEX% Direction of rotation, polyphase AC motor
%INDEX% Phase rotation, polyphase motor
%INDEX% Reversing direction of polyphase AC motor
%INDEX% Rotational direction, polyphase AC motor

%(END_NOTES)


