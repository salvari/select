
%(BEGIN_QUESTION)
% Copyright 2003, Tony R. Kuphaldt, released under the Creative Commons Attribution License (v 1.0)
% This means you may do almost anything with this work of mine, so long as you give me proper credit

% Uncomment the following line if the question involves calculus at all:
\vbox{\hrule \hbox{\strut \vrule{} $\int f(x) \> dx$ \hskip 5pt {\sl Calculus alert!} \vrule} \hrule}

Ohm's Law tells us that the amount of current through a fixed resistance may be calculated as such:

$$I = {E \over R}$$

We could also express this relationship in terms of {\it conductance} rather than {\it resistance}, knowing that $G = {1 \over R}$:

$$I = EG$$

However, the relationship between current and voltage for a fixed capacitance is quite different.  The "Ohm's Law" formula for a capacitor is as such:

$$i = C {de \over dt}$$

What significance is there in the use of lower-case variables for current ($i$) and voltage ($e$)?  Also, what does the expression ${de \over dt}$ mean?  Note: in case you think that the $d$'s are variables, and should cancel out in this fraction, think again: this is no ordinary quotient!  The $d$ letters represent a calculus concept known as a {\it differential}, and a quotient of two $d$ terms is called a {\it derivative}.

\underbar{file 01380}
%(END_QUESTION)





%(BEGIN_ANSWER)

Lower-case variables represent {\it instantaneous} values, as opposed to average values.  The expression ${de \over dt}$, which may also be written as ${dv \over dt}$, represents the {\it instantaneous rate of change of voltage over time}.

\vskip 10pt

Follow-up question: manipulate this equation to solve for the other two variables (${de \over dt} = \cdots$ ; $C = \cdots$).

%(END_ANSWER)





%(BEGIN_NOTES)

I have found that the topics of capacitance and inductance are excellent contexts in which to introduce fundamental principles of calculus to students.  The time you spend discussing this question and questions like it will vary according to your students' mathematical abilities.

Even if your students are not ready to explore calculus, it is still a good idea to discuss how the relationship between current and voltage for a capacitance involves {\it time}.  This is a radical departure from the time-independent nature of resistors, and of Ohm's Law!

%INDEX% Capacitance, "Ohm's Law" for
%INDEX% Calculus, derivative

%(END_NOTES)


