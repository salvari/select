
%(BEGIN_QUESTION)
% Copyright 2003, Tony R. Kuphaldt, released under the Creative Commons Attribution License (v 1.0)
% This means you may do almost anything with this work of mine, so long as you give me proper credit

If we were to analyze the magnetic flux lines of a current-carrying conductor, oriented perpendicularly to a magnetic field between two bar magnets, the interaction would look something like this:

$$\epsfbox{00396x01.eps}$$

This interaction of magnetic flux lines (the bar magnets' straight lines versus the wire's circles) will produce a mechanical force on the wire (called the {\it Lorentz} force).  Which direction will this force act?  Also, determine the direction of current through the conductor (seen from an end-view in the above illustration) necessary to produce the circular magnetic flux shown.

\underbar{file 00396}
%(END_QUESTION)





%(BEGIN_ANSWER)

$$\epsfbox{00396x02.eps}$$

%(END_ANSWER)





%(BEGIN_NOTES)

This question serves as a good application of the right-hand rule (or left-hand rule, if you follow electron flow notation).

%INDEX% Lorentz force on current-carrying wire
%INDEX% Force created by current-carrying wire in magnetic field

%(END_NOTES)


