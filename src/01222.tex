
%(BEGIN_QUESTION)
% Copyright 2003, Tony R. Kuphaldt, released under the Creative Commons Attribution License (v 1.0)
% This means you may do almost anything with this work of mine, so long as you give me proper credit

What is the {\it one's complement} of a binary number?  If you had to describe this principle to someone who just learned what binary numbers are, what would you say?

Determine the one's complement for the following binary numbers:

\medskip
\item{$\bullet$} $10001010_2$
\item{$\bullet$} $11010111_2$
\item{$\bullet$} $11110011_2$
\item{$\bullet$} $11111111_2$
\item{$\bullet$} $11111_2$
\item{$\bullet$} $00000000_2$
\item{$\bullet$} $00000_2$
\medskip

\underbar{file 01222}
%(END_QUESTION)





%(BEGIN_ANSWER)

\medskip
\item{$\bullet$} $10001010_2$: One's complement = $01110101_2$
\item{$\bullet$} $11010111_2$: One's complement = $00101000_2$
\item{$\bullet$} $11110011_2$: One's complement = $00001100_2$
\item{$\bullet$} $11111111_2$: One's complement = $00000000_2$
\item{$\bullet$} $11111_2$: One's complement = $00000_2$
\item{$\bullet$} $00000000_2$: One's complement = $11111111_2$
\item{$\bullet$} $00000_2$: One's complement = $11111_2$
\medskip

\vskip 10pt

Follow-up question: is the one's complement $11111111_2$ identical to the one's complement of $11111_2$?  How about the one's complements of $00000000_2$ and $00000_2$?  Explain.

%(END_ANSWER)





%(BEGIN_NOTES)

The principle of a "one's complement" is very, very simple.  Don't give your students any hints at all concerning the technique for finding a one's complement.  Rather, let them research it and present it to you on their own!

Be sure to discuss the follow-up question, concerning the one's complement of different-width binary numbers.  There is a very important lesson to be learned here!

%INDEX% One's complement, defined

%(END_NOTES)


