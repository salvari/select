
%(BEGIN_QUESTION)
% Copyright 2003, Tony R. Kuphaldt, released under the Creative Commons Attribution License (v 1.0)
% This means you may do almost anything with this work of mine, so long as you give me proper credit

$$\epsfbox{01972x01.eps}$$

\underbar{file 01972}
\vfil \eject
%(END_QUESTION)





%(BEGIN_ANSWER)

Use circuit simulation software to verify your predicted and measured parameter values.

%(END_ANSWER)





%(BEGIN_NOTES)

I strongly recommend a value for R1 of 1 M$\Omega$ or more, to protect the JFET gate from overcurrent damage.  The students will calculate their own dropping resistor value, based on the supply voltage and the LED ratings.

This exercise lends itself to experimentation with static electricity.  The input impedance of an average JFET is so high that the LED may be made to turn on and off with just a touch of the probe wire to a charged object (such as a person).

Using only the components shown, students may not be able to get their JFETs to completely turn off.  This is left for them as a challenge to figure out!

I expect students to be able to figure out how to calculate the transistor's power dissipation without being told what measurements to take!

%INDEX% Assessment, performance-based (Current-sourcing JFET switch)

%(END_NOTES)


