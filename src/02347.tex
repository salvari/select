
%(BEGIN_QUESTION)
% Copyright 2005, Tony R. Kuphaldt, released under the Creative Commons Attribution License (v 1.0)
% This means you may do almost anything with this work of mine, so long as you give me proper credit

Identify three different ways that an SCR or a TRIAC may be triggered into its "on" (conducting) state:

\medskip
\goodbreak
\item{$1.$} 
\item{$2.$} 
\item{$3.$} 
\medskip

\underbar{file 02347}
%(END_QUESTION)





%(BEGIN_ANSWER)

\medskip
\goodbreak
\item{$1.$} Applying a voltage pulse at the gate terminal
\item{$2.$} Exceeding the anode-to-cathode "breakover" voltage
\item{$3.$} Exceeding the "critical rate of rise" for anode-cathode voltage ($dv \over dt$)
\medskip

%(END_ANSWER)





%(BEGIN_NOTES)

Although gate triggering is by far the most common method of initiating conduction through SCRs and TRIACs, it is important that students realize it is not the only way.  The other two methods, both involving voltage applied between the anode and cathode terminals (or MT1-MT2 terminals) of the device, are often {\it accidental} means of triggering.

Be sure to discuss with your students the reason {\it why} excessive ${dv \over dt}$ can trigger a thyristor, based on an examination of inter-electrode capacitance within the transistors of a thyristor model.

%INDEX% SCR, methods of triggering
%INDEX% TRIAC, methods of triggering

%(END_NOTES)


