
%(BEGIN_QUESTION)
% Copyright 2008, Tony R. Kuphaldt, released under the Creative Commons Attribution License (v 1.0)
% This means you may do almost anything with this work of mine, so long as you give me proper credit

$$\epsfbox{01695x01.eps}$$

\underbar{file 01695}
\vfil \eject
%(END_QUESTION)





%(BEGIN_ANSWER)

Measure the wire length to check your calculation!

%(END_ANSWER)





%(BEGIN_NOTES)

You will need to provide some spools of wire for your students to measure using this technique.  Your students will need to find wire tables correlating length with resistance (at different temperatures, if the room temperature is significantly different from the standard temperature given in the table).

\vskip 10pt

When students set their power supplies for a certain amount of current, it is often helpful to have them do so while it is powering the wire (rather than connecting their ammeter directly across the power supply to set its current output).  This helps avoid the possibility of blowing the fuse in their ammeter!

%INDEX% Assessment, performance-based (Determining wire length by resistance measurement -- 4-wire "Kelvin" technique)
%INDEX% Assessment, performance-based (4-wire "Kelvin" resistance measurement for determining wire length)

%(END_NOTES)


