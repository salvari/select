
%(BEGIN_QUESTION)
% Copyright 2005, Tony R. Kuphaldt, released under the Creative Commons Attribution License (v 1.0)
% This means you may do almost anything with this work of mine, so long as you give me proper credit

Solve for $n$ in the following equations:

\vskip 10pt

Equation 1: $-56 = -14n$

\vskip 5pt

Equation 2: $54 - n = 10$

\vskip 5pt

Equation 3: ${4 \over n} = 12$

\vskip 5pt

Equation 4: $28 = 2 - n$

\vskip 10pt

\underbar{file 03063}
%(END_QUESTION)





%(BEGIN_ANSWER)

Equation 1: $n = 4$

\vskip 5pt

Equation 2: $n = 44$

\vskip 5pt

Equation 3: $n = 0.\overline{333}$

\vskip 5pt

Equation 4: $n = -26$

%(END_ANSWER)





%(BEGIN_NOTES)

Have your students come to the front of the class and show everyone else the techniques they used to solve for the value of $a$ in each equation.  Remind them to document each and every step in the process, so that nothing is left to guess or to chance.

Equations 2 through 4 require two steps to solve for $n$.  Equation 1 only requires a single step, but the two negative numbers may be a bit confusing to some.

%INDEX% Algebra, manipulating equations

%(END_NOTES)


