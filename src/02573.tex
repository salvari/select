
%(BEGIN_QUESTION)
% Copyright 2005, Tony R. Kuphaldt, released under the Creative Commons Attribution License (v 1.0)
% This means you may do almost anything with this work of mine, so long as you give me proper credit

A popular passive filtering network called the {\it twin-tee} is often coupled with an operational amplifier to produce an active filter circuit.  Two examples are shown here:

$$\epsfbox{02573x01.eps}$$

$$\epsfbox{02573x02.eps}$$

Identify which of these circuits is band-pass, and which is band-stop.  Also, identify the type of response typically provided by the twin-tee network alone, and how that response is exploited to make {\it two} different types of active filter responses.

\underbar{file 02573}
%(END_QUESTION)





%(BEGIN_ANSWER)

The first filter shown is a band-stop, while the second filter shown is a band-pass.

%(END_ANSWER)





%(BEGIN_NOTES)

Like all the other active filter circuits, the fundamental characteristic of each filter may be determined by qualitative analysis at $f = 0$ and $f = \infty$.

An interesting concept at work here is the inversion of a function by placement inside the feedback loop of a negative-feedback opamp circuit.  What is a band-stop filter all by itself forces the opamp to be a band-pass if placed within the negative feedback signal path.  Discuss this very important concept with your students, as this is most definitely not the only application for it!

%INDEX% Active filter circuits, using twin-tee RC networks

%(END_NOTES)


