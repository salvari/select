
%(BEGIN_QUESTION)
% Copyright 2004, Tony R. Kuphaldt, released under the Creative Commons Attribution License (v 1.0)
% This means you may do almost anything with this work of mine, so long as you give me proper credit

Class-A operation may be obtained from this simple transistor circuit if the input voltage ($V_{in}$) is "biased" with a series-connected DC voltage source:

$$\epsfbox{02223x01.eps}$$

First, define what "Class A" amplifier operation is.  Then, explain {\it why} biasing is required for this transistor to achieve it.

\underbar{file 02223}
%(END_QUESTION)





%(BEGIN_ANSWER)

"Class A" amplifier operation is when the transistor remains in its "active" mode (conducting current) throughout the entire waveform.  Biasing may be thought of as a kind of "trick" used to get the transistor (a DC device) to "think" it is amplifying DC when the input signal is really AC.

%(END_ANSWER)





%(BEGIN_NOTES)

A "trick" it may be, but a very useful and very common "trick" it is!  Discuss this concept with your students at length, being sure they have ample time and opportunity to ask questions of their own.  

One question that may arise is, "how much DC bias voltage is necessary?"  If no one asks this question, ask it yourself!  Discuss with your students what would constitute the minimum amount of bias voltage necessary to ensure the transistor never goes into "cutoff" anywhere in the waveform's cycle, and also the maximum bias voltage to prevent the transistor from "saturating".

%INDEX% Biasing, defined in context of common-collector circuit
%INDEX% Class-A amplifier operation, defined

%(END_NOTES)


