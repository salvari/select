
%(BEGIN_QUESTION)
% Copyright 2005, Tony R. Kuphaldt, released under the Creative Commons Attribution License (v 1.0)
% This means you may do almost anything with this work of mine, so long as you give me proper credit

This two-stage transistor amplifier circuit is {\it transformer-coupled}:

$$\epsfbox{01121x01.eps}$$

What advantage(s) does a transformer-coupled amplifier have over circuits using other methods of coupling?  Are there any disadvantages to using a transformer for signal coupling between transistor stages?  Explain in detail.

\underbar{file 01121}
%(END_QUESTION)





%(BEGIN_ANSWER)

Transformers allow for impedance transformation between stages, as well as phase inversion (if desired).  However, their parasitic (leakage) inductance and inter-winding capacitance may cause the amplifier to have strange frequency response characteristics.

\vskip 10pt

Follow-up question: label the transformer's polarity using "dot" notation in order to achieve no inversion of signal from input to output (as shown).

%(END_ANSWER)





%(BEGIN_NOTES)

Ask your students to explain what {\it impedance transformation} is and why it is important, especially in amplifier circuits.  This will be a good review of both transformer theory and the maximum power transfer theorem.

Regarding phase inversion, a fun challenge here is to have students specify the "dot convention" necessary for this particular transformer to obtain the non-inverting characteristic of this two-stage amplifier circuit.  In other words, have them draw dots near the transformer windings (with the proper relative relationship) to produce the phasing shown by the sine-wave symbols in the diagram.

%INDEX% Amplifier, interstage coupling
%INDEX% Amplifier, transformer coupling

%(END_NOTES)


