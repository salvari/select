
%(BEGIN_QUESTION)
% Copyright 2003, Tony R. Kuphaldt, released under the Creative Commons Attribution License (v 1.0)
% This means you may do almost anything with this work of mine, so long as you give me proper credit

Counter circuits built by cascading the output of one flip-flop to the clock input of the next flip-flop are generally referred to as {\it ripple} counters.  Explain why this is so.  What happens in such a circuit that earns it the label of "ripple"?  Is this effect potentially troublesome in circuit operation, or is it something of little or no consequence?

\underbar{file 01388}
%(END_QUESTION)





%(BEGIN_ANSWER)

When these counters increment or decrement, they do so in such a way that the respective output bits change state in rapid sequence ("rippling") rather than all at the same time.  This creates false count outputs for very brief moments of time.

Whether or not this constitutes a problem in a digital circuit depends on the circuit's tolerance of false counts.  In many circuits, there are ways to avoid this problem without resorting to a re-design of the counter.

%(END_ANSWER)





%(BEGIN_NOTES)

If your students have studied binary adder circuits, they should recognize the term "ripple" in a slightly different context.  Different circuit, same problem.

%INDEX% Counter circuit, ripple
%INDEX% Ripple counter circuit

%(END_NOTES)


