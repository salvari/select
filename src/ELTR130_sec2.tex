
\centerline{\bf ELTR 130 (Operational Amplifiers 1), section 2} \bigskip 
 
\vskip 10pt

\noindent
{\bf Recommended schedule}

\vskip 5pt

%%%%%%%%%%%%%%%
\hrule \vskip 5pt
\noindent
\underbar{Day 1}

\hskip 10pt Topics: {\it Using the operational amplifier as a noninverting voltage amplifier}
 
\hskip 10pt Questions: {\it 1 through 15}
 
\hskip 10pt Lab Exercise: {\it Opamp as noninverting amplifier (question 61)}
 
%INSTRUCTOR \hskip 10pt {\bf MIT 6.002 video clip: Disk 4, Lecture 21; Noninverting opamp 41:30 to 43:10}

\vskip 10pt
%%%%%%%%%%%%%%%
\hrule \vskip 5pt
\noindent
\underbar{Day 2}

\hskip 10pt Topics: {\it Using the operational amplifier as an inverting voltage amplifier}
 
\hskip 10pt Questions: {\it 16 through 30}
 
\hskip 10pt Lab Exercise: {\it Opamp as inverting amplifier (question 62)}
 
 
\vskip 10pt
%%%%%%%%%%%%%%%
\hrule \vskip 5pt
\noindent
\underbar{Day 3}

\hskip 10pt Topics: {\it Voltage/current converter and summer circuits}
 
\hskip 10pt Questions: {\it 31 through 40}
 
\hskip 10pt Lab Exercise: {\it Troubleshooting practice on prototyped project}
 
\vskip 10pt
%%%%%%%%%%%%%%%
\hrule \vskip 5pt
\noindent
\underbar{Day 4}

\hskip 10pt Topics: {\it Differential and instrumentation amplifier circuits}
 
\hskip 10pt Questions: {\it 41 through 50}
 
\hskip 10pt Lab Exercise: {\it Op-amp as difference amplifier (question 63)}
 
\vskip 10pt
%%%%%%%%%%%%%%%
\hrule \vskip 5pt
\noindent
\underbar{Day 5}

\hskip 10pt Topics: {\it Precision rectifier circuits}
 
\hskip 10pt Questions: {\it 51 through 60}
 
\hskip 10pt Lab Exercise: {\it Precision half-wave rectifier (question 64)}
 
\vskip 10pt
%%%%%%%%%%%%%%%
\hrule \vskip 5pt
\noindent
\underbar{Day 6}

\hskip 10pt Topics: {\it Review}
 
\hskip 10pt Lab Exercise: {\it Troubleshooting practice on prototyped project}
 
\vskip 10pt
%%%%%%%%%%%%%%%
\hrule \vskip 5pt
\noindent
\underbar{Day 7}

\hskip 10pt Exam 2: {\it includes Inverting or Noninverting amplifier circuit performance assessment}
 
\hskip 10pt {\bf Troubleshooting Assessment due:} {\it Opamp project prototype}
 
\hskip 10pt Question 65: Troubleshooting log
 
\hskip 10pt Question 66: Sample troubleshooting assessment grading criteria
 
\vskip 10pt
%%%%%%%%%%%%%%%%
\hrule \vskip 5pt
\noindent
\underbar{Troubleshooting practice problems}

\hskip 10pt Questions: {\it 67 through 76}
 
\vskip 10pt
%%%%%%%%%%%%%%%
\hrule \vskip 5pt
\noindent
\underbar{General concept practice and challenge problems}

\hskip 10pt Questions: {\it 77 through the end of the worksheet}
 
\vskip 10pt
%%%%%%%%%%%%%%%











\vfil \eject

\centerline{\bf ELTR 130 (Operational Amplifiers 1), section 2} \bigskip 
 
\vskip 10pt

\noindent
{\bf Skill standards addressed by this course section}

\vskip 5pt

%%%%%%%%%%%%%%%
\hrule \vskip 10pt
\noindent
\underbar{EIA {\it Raising the Standard; Electronics Technician Skills for Today and Tomorrow}, June 1994}

\vskip 5pt

\medskip
\item{\bf E} {\bf Technical Skills -- Analog Circuits}
\item{\bf E.10} Understand principles and operations of operational amplifier circuits.
\item{\bf E.11} Fabricate and demonstrate operational amplifier circuits.
\item{\bf E.12} Troubleshoot and repair operational amplifier circuits.
\medskip

\vskip 5pt

\medskip
\item{\bf B} {\bf Basic and Practical Skills -- Communicating on the Job}
\item{\bf B.01} Use effective written and other communication skills.  {\it Met by group discussion and completion of labwork.}
\item{\bf B.03} Employ appropriate skills for gathering and retaining information.  {\it Met by research and preparation prior to group discussion.}
\item{\bf B.04} Interpret written, graphic, and oral instructions.  {\it Met by completion of labwork.}
\item{\bf B.06} Use language appropriate to the situation.  {\it Met by group discussion and in explaining completed labwork.}
\item{\bf B.07} Participate in meetings in a positive and constructive manner.  {\it Met by group discussion.}
\item{\bf B.08} Use job-related terminology.  {\it Met by group discussion and in explaining completed labwork.}
\item{\bf B.10} Document work projects, procedures, tests, and equipment failures.  {\it Met by project construction and/or troubleshooting assessments.}
\item{\bf C} {\bf Basic and Practical Skills -- Solving Problems and Critical Thinking}
\item{\bf C.01} Identify the problem.  {\it Met by research and preparation prior to group discussion.}
\item{\bf C.03} Identify available solutions and their impact including evaluating credibility of information, and locating information.  {\it Met by research and preparation prior to group discussion.}
\item{\bf C.07} Organize personal workloads.  {\it Met by daily labwork, preparatory research, and project management.}
\item{\bf C.08} Participate in brainstorming sessions to generate new ideas and solve problems.  {\it Met by group discussion.}
\item{\bf D} {\bf Basic and Practical Skills -- Reading}
\item{\bf D.01} Read and apply various sources of technical information (e.g. manufacturer literature, codes, and regulations).  {\it Met by research and preparation prior to group discussion.}
\item{\bf E} {\bf Basic and Practical Skills -- Proficiency in Mathematics}
\item{\bf E.01} Determine if a solution is reasonable.
\item{\bf E.02} Demonstrate ability to use a simple electronic calculator.
\item{\bf E.05} Solve problems and [sic] make applications involving integers, fractions, decimals, percentages, and ratios using order of operations.
\item{\bf E.06} Translate written and/or verbal statements into mathematical expressions.
\item{\bf E.09} Read scale on measurement device(s) and make interpolations where appropriate.  {\it Met by oscilloscope usage.}
\item{\bf E.12} Interpret and use tables, charts, maps, and/or graphs.
\item{\bf E.13} Identify patterns, note trends, and/or draw conclusions from tables, charts, maps, and/or graphs.
\item{\bf E.15} Simplify and solve algebraic expressions and formulas.
\item{\bf E.16} Select and use formulas appropriately.
\item{\bf E.17} Understand and use scientific notation.
\medskip

%%%%%%%%%%%%%%%




\vfil \eject

\centerline{\bf ELTR 130 (Operational Amplifiers 1), section 2} \bigskip 
 
\vskip 10pt

\noindent
{\bf Common areas of confusion for students}

\vskip 5pt


\hrule \vskip 5pt

\vskip 10pt

\noindent
{\bf Difficult concept: } {\it Negative feedback.}

Few concepts are as fundamentally important in electronics as negative feedback, and so it is essential for the electronics student to learn well.  However, it is not an easy concept for many to grasp.  The notion that a portion of the output signal may be "fed back" into the input in a degenerative manner to stabilize gain is far from obvious.  One of the most powerfully illustrative examples I know of is the use of negative feedback in a voltage regulator circuit to compensate for the base-emitter voltage drop of 0.7 volts (see question file \#02286).

\vskip 10pt

\noindent
{\bf Common mistake: } {\it Thinking that an opamp's output current is supplied through its input terminals.}

This is a misconception that seems to have an amazing resistance to correction.  There seem to always be a few students who think that there is a direct path for current from the input terminals of an opamp to its output terminal.  It is very important to realize that {\it for most practical purposes, an opamp draws negligible current through its input terminals!}  What current does go through the output terminal is {\it always} supplied by the \underbar{power} terminals and from the power supply, never by the input signal(s).  To put this into colloquial terms, the input terminals on an opamp tell the output what to do, but they do not give the output its "muscle" (current) to do it.

I think the reason for this misconception is the fact that power terminals are often omitted from opamp symbols for brevity, and after a while of seeing this it is easy to forget they are really still there performing a useful function!

\vskip 10pt

\noindent
{\bf Difficult concept: } {\it Applying KVL and KCL to operational amplifier circuits.}

Many students have the unfortunate tendency to memorize rather than think.  When approaching all the different types of operational amplifier circuits, this tendency is a recipe for failure.  Instead of learning how to effectively apply Kirchhoff's Laws to the analysis of different opamp circuits, many students simply try to rote-memorize what all the different circuits do.  Not only will these students experience difficulty remembering all the variations in circuit behavior, but they will also be helpless in troubleshooting opamp circuits, as a faulted circuit will not behave as it should!  There is only one correct approach here, and that is to \underbar{master} the application of Kirchhoff's Voltage Law and Kirchhoff's Current Law.  Difficult?  Perhaps, but necessary.  Remember: {\it there are no shortcuts to learning!}


