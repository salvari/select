
%(BEGIN_QUESTION)
% Copyright 2003, Tony R. Kuphaldt, released under the Creative Commons Attribution License (v 1.0)
% This means you may do almost anything with this work of mine, so long as you give me proper credit

A student builds their first astable 555 timer circuit, using a TLC555CP chip.  Unfortunately, it seems to have a problem.  Sometimes, the output of the timer simply stops oscillating, with no apparent cause.  Stranger yet, the problem often occurs at the precise time anyone moves their hand within a few inches of the circuit board (without actually touching anything!).

What could the student have done wrong in assembling this circuit to cause such a problem?  What steps would you take to troubleshoot this problem?

\underbar{file 01433}
%(END_QUESTION)





%(BEGIN_ANSWER)

I won't reveal the most probable cause, but I will give you this hint: the TLC555CP integrated circuit ("chip") uses CMOS technology.

%(END_ANSWER)





%(BEGIN_NOTES)

Every year it seems I have at least one student who experiences this particular problem, usually as a result of hasty circuit assembly (not making all necessary connections to pins on the chip).  This is a good question to brainstorm with your class on, exploring possible causes and methods of diagnosis.

%INDEX% Troubleshooting, 555 timer
%INDEX% Troubleshooting, logic gate circuit (CMOS)

%(END_NOTES)


