
%(BEGIN_QUESTION)
% Copyright 2003, Tony R. Kuphaldt, released under the Creative Commons Attribution License (v 1.0)
% This means you may do almost anything with this work of mine, so long as you give me proper credit

Some of the particles comprising atoms react to each other in a way that scientists categorize as {\it electric charge}.  There are two fundamental types of electric charge: {\it positive} and {\it negative}.  Identify the respective charges of the following particles:

\medskip
\item{$\bullet$} Electrons
\item{$\bullet$} Protons
\item{$\bullet$} Neutrons
\medskip

What would happen if you placed two electrons near each other in free space?  Would they repel each other or attract each other?  How about two protons?  How about an electron and a proton?  How about a neutron and a proton?

\underbar{file 00137}
%(END_QUESTION)





%(BEGIN_ANSWER)

I'll let you research which charge type (positive or negative) is characteristic of electrons, protons, and neutrons!

As for their respective physical reactions, particles of differing charge are physically attracted to each other while particles of identical charge repel each other.

%(END_ANSWER)





%(BEGIN_NOTES)

Many students will want to know "why?" in response to electrical charges.  The technical answer has to do with electric fields extending through space, but this may be a philosophically impossible question to answer.  The concept of charge was invented to explain the physical behavior of electrical attraction and repulsion, but coining a term to explain a phenomenon does nothing to explain {\it why} that phenomenon occurs.

Still, this is a worthwhile subject for discussion, especially if students have done their research well and know something about the history of electricity.

%INDEX% Particles, subatomic
%INDEX% Subatomic particles
%INDEX% Charges, attraction
%INDEX% Charges, repulsion

%(END_NOTES)


