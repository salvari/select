
%(BEGIN_QUESTION)
% Copyright 2003, Tony R. Kuphaldt, released under the Creative Commons Attribution License (v 1.0)
% This means you may do almost anything with this work of mine, so long as you give me proper credit

How does the {\it conductance} (G) of a conductor relate to its length?  In other words, the longer the conductor, the (\underbar{fill-in-the-blank}) its conductance is, all other factors being equal.

\underbar{file 00487}
%(END_QUESTION)





%(BEGIN_ANSWER)

Conductance decreases as length increases, all other factors being equal.

\vskip 10pt

Follow-up question: how does "conductance" (G) mathematically relate to resistance (R), and what is the unit of measurement for conductance?

%(END_ANSWER)





%(BEGIN_NOTES)

There are two units of measurement for conductance: the old unit (which makes perfect sense, even though your students may laugh at it, at first), and the new unit (named after a famous electrical researcher).  Make sure your students are familiar with both.

%INDEX% Conductance, relation to conductor length

%(END_NOTES)


