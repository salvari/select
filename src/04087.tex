
%(BEGIN_QUESTION)
% Copyright 2007, Tony R. Kuphaldt, released under the Creative Commons Attribution License (v 1.0)
% This means you may do almost anything with this work of mine, so long as you give me proper credit

\centerline{\bf Animation: reflected waves on a properly terminated transmission line}

\vskip 10pt

{\it This question consists of a series of images (one per page) that form an animation.  Flip the pages with your fingers to view this animation (or click on the "next" button on your viewer) frame-by-frame.}

\vskip 10pt

The following animation models a shorted transmission line as a voltage source and traveling impedance, to show what happens to a square-wave signal as it propagates down the length of the line and reflects back.

\vfil \eject
$$\epsfbox{04087x00.eps}$$

\vfil \eject
$$\epsfbox{04087x01.eps}$$

\vfil \eject
$$\epsfbox{04087x02.eps}$$

\vfil \eject
$$\epsfbox{04087x03.eps}$$

\vfil \eject
$$\epsfbox{04087x04.eps}$$

\vfil \eject
$$\epsfbox{04087x05.eps}$$

\vfil \eject
$$\epsfbox{04087x06.eps}$$

\vfil \eject
$$\epsfbox{04087x07.eps}$$

\underbar{file 04087}

\vfil \eject

%(END_QUESTION)





%(BEGIN_ANSWER)


%(END_ANSWER)





%(BEGIN_NOTES)

The purpose of this animation is to let students study the behavior of a transmission line and reach their own conclusions.  Similar to experimentation in the lab, except that here all the data collection is done visually rather than through the use of test equipment, and the students are able to "see" things that are invisible in real life.

%INDEX% Animation, transmission line (open end)

%(END_NOTES)


