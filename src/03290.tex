
%(BEGIN_QUESTION)
% Copyright 2005, Tony R. Kuphaldt, released under the Creative Commons Attribution License (v 1.0)
% This means you may do almost anything with this work of mine, so long as you give me proper credit

If an oscilloscope is set up for "single-sweep" triggering and connected to a DC-excited resonant circuit such as the one shown in the following schematic, the resulting oscillation will last just a short time (after momentarily pressing and releasing the pushbutton switch):

$$\epsfbox{03290x01.eps}$$

Explain why the oscillations die out, rather than go on forever.  Hint: the answer is fundamentally the same as why a swinging pendulum eventually comes to a stop.

\underbar{file 03290}
%(END_QUESTION)





%(BEGIN_ANSWER)

No resonant circuit is completely free of dissipative elements, whether resistive or radiative, and so some energy is lost each cycle.

%(END_ANSWER)





%(BEGIN_NOTES)

A circuit such as this is easy to build and demonstrate, but you will need a digital storage oscilloscope to successfully capture the damped oscillations.  Also, the results may be tainted by switch "bounce," so be prepared to address that concept if you plan to demonstrate this to a live audience.

You might want to ask your students how they would suggest building a "tank circuit" that is as free from energy losses as possible.  If a perfect tank circuit could be built, how would it act if momentarily energized by a DC source such as in this setup?

%INDEX% Oscilloscope, single-sweep triggering
%INDEX% Resonant circuit, analysis in terms of energy storage
%INDEX% Resonant circuit, damped

%(END_NOTES)


