
%(BEGIN_QUESTION)
% Copyright 2003, Tony R. Kuphaldt, released under the Creative Commons Attribution License (v 1.0)
% This means you may do almost anything with this work of mine, so long as you give me proper credit

Shown here is a circuit constructed on a PCB (a "Printed Circuit Board"), with copper "traces" serving as wires to connect the components together:

$$\epsfbox{00099x01.eps}$$

How would the multimeter be used to measure the resistance of the component labeled "R1"?  Include these important points in your answer:

\item {$\bullet$} The configuration of the multimeter (selector switch position, test lead jacks)
\item {$\bullet$} The connections of the meter test leads to the circuit
\item {$\bullet$} The state of the switch on the PCB (open or closed)

\underbar{file 00099}
%(END_QUESTION)





%(BEGIN_ANSWER)

$$\epsfbox{00099x02.eps}$$

%(END_ANSWER)





%(BEGIN_NOTES)

It is very important that students understand component resistance cannot be measured when the component is energized!  In cases such as this, it is necessary to disconnect the component from the rest of the circuit so that only {\it its} resistance (and not any other components' resistance) is measured.  In other cases, though, it may be acceptable to leave the component in place to take a resistance measurement.

%INDEX% Printed circuit board
%INDEX% Ohmmeter usage

%(END_NOTES)


