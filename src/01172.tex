
%(BEGIN_QUESTION)
% Copyright 2003, Tony R. Kuphaldt, released under the Creative Commons Attribution License (v 1.0)
% This means you may do almost anything with this work of mine, so long as you give me proper credit

{\it Positive} or {\it regenerative} feedback is an essential characteristic of all oscillator circuits.  Why, then, do comparator circuits utilizing positive feedback not oscillate?  Instead of oscillating, the output of a comparator circuit with positive feedback simply saturates to one of its two rail voltage values.  Explain this.

\underbar{file 01172}
%(END_QUESTION)





%(BEGIN_ANSWER)

The positive feedback used in oscillator circuits is always phase-shifted 360$^{o}$, while the positive feedback used in comparator circuits has no phase shift at all, being direct-coupled.

%(END_ANSWER)





%(BEGIN_NOTES)

This is a challenging question, and may not be suitable for all students.  Basically, what I'm trying to get students to do here is think carefully about the nature of positive feedback as used in comparator circuits, versus as it's used in oscillator circuits.  Students who have simply memorized the concept of "positive feedback causing oscillation" will fail to understand what is being asked in this question, much less understand the given answer.

%(END_NOTES)


