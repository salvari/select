
%(BEGIN_QUESTION)
% Copyright 2003, Tony R. Kuphaldt, released under the Creative Commons Attribution License (v 1.0)
% This means you may do almost anything with this work of mine, so long as you give me proper credit

Using a {\it commutating diode} (sometimes called a {\it free-wheeling diode}) to eliminate switch contact arcing for inductive loads in a DC circuit works well, but it has an unfortunate side-effect:

$$\epsfbox{00982x01.eps}$$

With a diode in place, the release time for the solenoid increases measurably.  In other words, it takes longer for the solenoid to completely de-magnetize after the switch contacts open, than if there is no diode in the circuit.

Explain why this is, and also propose a solution for the minimizing the solenoid's release time.

\underbar{file 00982}
%(END_QUESTION)





%(BEGIN_ANSWER)

The presence of a commutating diode increases the solenoid's release time because the L/R time constant of the de-energizing circuit is made much longer than with no diode in place.  The solution to this problem is to decrease the L/R time constant of the discharge circuit (I'll let you figure out how!).

%(END_ANSWER)





%(BEGIN_NOTES)

This question is an good review of inductor time constant theory, and challenges students to put their mastery of L/R time constant circuits to the test by engineering a solution for this problem.

Once a solution has been agreed upon, ask your students if the solution introduces (or re-introduces, as the case may be) any other problems in the circuit.

%(END_NOTES)


