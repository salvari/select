
%(BEGIN_QUESTION)
% Copyright 2006, Tony R. Kuphaldt, released under the Creative Commons Attribution License (v 1.0)
% This means you may do almost anything with this work of mine, so long as you give me proper credit

The generator in this power system is disconnected for routine servicing by removing its fuse, leaving the two batteries to supply all power to the bank of light bulbs:

$$\epsfbox{03940x01.eps}$$

The two batteries are rated such that they are supposed to provide at least 10 hours of back-up power in the event of a generator "outage" such as this.  Unfortunately, the light bulbs begin to dim much sooner than expected.  Something is wrong in the system, and you are asked to figure out what.

Explain what steps you would take to diagnose the problem in this circuit, commenting on any relevant safety measures taken along the way.

\underbar{file 03940}
%(END_QUESTION)





%(BEGIN_ANSWER)

One simple procedure would be to connect a voltmeter in parallel with the light bulb bank, then remove the fuse for battery \#1 and note the decrease in bus voltage.  After that, replace the fuse for battery \#1 and then remove the fuse for battery \#2, again noting the decrease in bus voltage.  If there is a problem with one of the batteries, it will be evident in this test.  If both batteries are in good condition, but simply low on charge, that will also be evident in this test.

%(END_ANSWER)





%(BEGIN_NOTES)

I purposely avoided giving away explicit answers in the Answer section of this question, electing instead to simply provide a sound procedure.  The point here is for students to figure out on their own what the voltmeter indications would mean with regard to battery condition.

%INDEX% Troubleshooting, generator/battery power system

%(END_NOTES)


