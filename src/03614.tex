
%(BEGIN_QUESTION)
% Copyright 2005, Tony R. Kuphaldt, released under the Creative Commons Attribution License (v 1.0)
% This means you may do almost anything with this work of mine, so long as you give me proper credit

Although it may not look like it, a single-wire radio antenna is actually a resonant circuit in and of itself.  Optimum transmitting performance is reached when the transmission frequency is equal to the natural resonant frequency of the antenna:

$$\epsfbox{03614x01.eps}$$

First, explain how a single piece of wire is able to {\it resonate}, given the fact that resonance requires the existence of both inductance and capacitance, and that there are neither inductors nor capacitors attached to this wire.  Then, describe what would have to be altered about this antenna to "tune" it for resonance at a lower transmission frequency.

\underbar{file 03614}
%(END_QUESTION)





%(BEGIN_ANSWER)

The necessary inductance and capacitance are parasitic in nature: inductance existing simply from the wire's length and capacitance existing in open space between the wire and ground.  This antenna must be made {\it longer} to resonate at a lower frequency.

%(END_ANSWER)





%(BEGIN_NOTES)

{\bf This question is intended for exams only and not worksheets!}.

%(END_NOTES)


