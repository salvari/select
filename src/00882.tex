
%(BEGIN_QUESTION)
% Copyright 2003, Tony R. Kuphaldt, released under the Creative Commons Attribution License (v 1.0)
% This means you may do almost anything with this work of mine, so long as you give me proper credit

Identify the type of transistor amplifier this is (common-collector, common-emitter, or common-base), and identify whether it is {\it inverting} or {\it noninverting}.

$$\epsfbox{00882x01.eps}$$

Also, explain how to derive the voltage gain equation for this amplifier:

$$A_V = {R_C \over {R_E + r'_e}}$$

\underbar{file 00882}
%(END_QUESTION)





%(BEGIN_ANSWER)

This is a {\it common-emitter} amplifier circuit, and it is inverting.  Here is a schematic hint for explaining why the voltage gain formula is as it is.  Note that the transistor is modeled here as a resistance and a controlled current source:

$$\epsfbox{00882x02.eps}$$

\vskip 10pt

Follow-up question: the voltage gain equation for this type of transistor amplifier is usually shown in a simpler form that what I provided at the end of the question.  Write this simpler form, and explain why it is approximately true.

%(END_ANSWER)





%(BEGIN_NOTES)

In discussion with your students, analyze the response of this single-transistor amplifier circuit for a variety of input voltages.  In other words, perform a {\it thought experiment} to re-familiarize them with the behavior of this transistor amplifier configuration.  Plot the input and output voltages, if necessary, on a graph.  Discuss what the term "inverting" means in this context.

%INDEX% BJT amplifier circuit, inverting

%(END_NOTES)


