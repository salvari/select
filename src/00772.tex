
%(BEGIN_QUESTION)
% Copyright 2003, Tony R. Kuphaldt, released under the Creative Commons Attribution License (v 1.0)
% This means you may do almost anything with this work of mine, so long as you give me proper credit

In this circuit, three common AC loads are represented as resistances, combined with reactive components in two out of the three cases.  Calculate the amount of true power ($P$), apparent power ($S$), reactive power ($Q$), and power factor ($PF$) for each of the loads:

$$\epsfbox{00772x01.eps}$$

Also, draw power triangle diagrams for each circuit, showing how the true, apparent, and reactive powers trigonometrically relate.

\underbar{file 00772}
%(END_QUESTION)





%(BEGIN_ANSWER)

Fluorescent lamp: $P =$ 60 W ; $Q =$ 54.3 VAR ; $S =$ 80.9 VA ; $PF =$ 0.74, leading 

\vskip 10pt

Incandescent lamp: $P =$ 60 W ; $Q =$ 0 VAR ; $S =$ 60 VA ; $PF =$ 1.0

\vskip 10pt

Induction motor: $P =$ 52.0 W ; $Q =$ 20.4 VAR ; $S =$ 55.8 VA ; $PF =$ 0.93, lagging

%(END_ANSWER)





%(BEGIN_NOTES)

Your students should realize that the only dissipative element in each load is the resistor.  Inductors and capacitors, being reactive components, do not actually dissipate power.

Ask your students how the "excess" current drawn by each load potentially influences the size of wire needed to carry power to that load.  Suppose the impedance of each load were 100 times less, resulting in 100 times as much current for each load.  Would the "extra" current be significant then?

Being that most heavy AC loads happen to be strongly inductive in nature (large electric motors, electromagnets, and the "leakage" inductance intrinsic to large transformers), what does this mean for AC power systems in general?

%INDEX% Power, apparent versus true versus reactive

%(END_NOTES)


