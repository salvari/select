
%(BEGIN_QUESTION)
% Copyright 2004, Tony R. Kuphaldt, released under the Creative Commons Attribution License (v 1.0)
% This means you may do almost anything with this work of mine, so long as you give me proper credit

This is a schematic diagram of a {\it Delta-connected} three-phase generator (with the rotor winding shown):

$$\epsfbox{02200x01.eps}$$

How much AC current will each of the lines ($I_A$, $I_B$, or $I_C$) conduct to a load (not shown) if each stator coil inside the alternator outputs 17 amps of current?

\underbar{file 02200}
%(END_QUESTION)





%(BEGIN_ANSWER)

Phase current = 17 amps AC (given)

Line current = $I_A = I_B = I_C =$ 29.4 amps AC

\vskip 10pt

Follow-up question \#1: what is the ratio between the line and phase current magnitudes in a Delta-connected three-phase system?

\vskip 10pt

Follow-up question \#2: what would happen to the output of this alternator if the rotor winding were to fail open?  Bear in mind that the rotor winding is typically energized with DC through a pair of brushes and slip rings from an external source, the current through this winding being used to control voltage output of the alternator's three-phase "stator" windings.

%(END_ANSWER)





%(BEGIN_NOTES)

Students will quickly discover that $\sqrt{3}$ is the "magic number" for practically all balanced three-phase circuit calculations!

It should be noted that although the {\it magnitudes} of $I_A$, $I_B$, and $I_C$ are equal, their phasor angles are most definitely not.  Therefore,

$$I_A = I_B = I_C \hbox{\hskip 20pt Scalar values equal}$$

$${\bf I_A} \neq {\bf I_B} \neq {\bf I_C} \hbox{\hskip 20pt Phasors unequal}$$

%INDEX% Alternator, three-phase
%INDEX% Line current vs. phase current
%INDEX% Phase current vs. line current
%INDEX% Y connected source, three phase

%(END_NOTES)


