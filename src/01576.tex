
%(BEGIN_QUESTION)
% Copyright 2003, Tony R. Kuphaldt, released under the Creative Commons Attribution License (v 1.0)
% This means you may do almost anything with this work of mine, so long as you give me proper credit

As a technician, you are sent to troubleshoot a complex piece of electronic equipment that has stopped working.  Upon opening the cabinet door for this equipment, your nose is greeted by the pungent odor of burnt circuit board (a smell you are unlikely to forget, once having experienced it).  What does this simple fact indicate (or possibly indicate) about the nature of the equipment's fault?

\underbar{file 01576}
%(END_QUESTION)





%(BEGIN_ANSWER)

The fact that you can {\it smell} trouble indicates you are most likely dealing with a catastrophic failure caused by (or resulting in) excessive current.  When components have been heated to such a degree that they emit strong odors, the damage is often visible as well, which makes it easier to locate problem areas.

\vskip 10pt

Follow-up question: upon further investigation, you locate the charred remains of an electronic component, located on one of the system's circuit boards.  Is this the only fault, being that it is the only component visibly damaged?  Explain why or why not.

%(END_ANSWER)





%(BEGIN_NOTES)

It is important for students to understand that not all faults become visible, even if catastrophic!

Be sure to discuss with your students that the burnt component may very well be a {\it victim} of another component failure, and not the {\it cause} of the system fault.  For instance, shorted wiring located far from the equipment enclosure may have caused the components to destruct.  This is a common assumption made by beginning troubleshooters: that the most obvious failure is the only failure, or that it must be the primary failure.

%INDEX% Troubleshooting, clues given by odor

%(END_NOTES)


