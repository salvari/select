
%(BEGIN_QUESTION)
% Copyright 2005, Tony R. Kuphaldt, released under the Creative Commons Attribution License (v 1.0)
% This means you may do almost anything with this work of mine, so long as you give me proper credit

A {\it white noise} source is a special type of AC signal voltage source which outputs a broad band of frequencies ("noise") with a constant amplitude across its rated range.  Determine what the display of a spectrum analyzer would show if directly connected to a white noise source, and also if connected to a low-pass filter which is in turn connected to a white noise source:

$$\epsfbox{03621x01.eps}$$

\underbar{file 03621}
%(END_QUESTION)





%(BEGIN_ANSWER)

$$\epsfbox{03621x02.eps}$$

%(END_ANSWER)





%(BEGIN_NOTES)

The purpose of this question, besides providing a convenient way to characterize a filter circuit, is to introduce students to the concept of a {\it white noise source} and also to strengthen their understanding of a spectrum analyzer's function.

In case anyone happens to notice, be aware that the rolloff shown for this filter circuit is {\it very} steep!  This sort of sharp response could never be realized with a simple one-resistor, one-capacitor ("first order") filter.  It would have to be a multi-stage analog filter circuit or some sort of active filter circuit.

%INDEX% Frequency response, of different filter types
%INDEX% Noise, white
%INDEX% Spectrum analyzer, used to show characteristic response of a filter circuit
%INDEX% White noise

%(END_NOTES)


