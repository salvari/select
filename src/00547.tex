
%(BEGIN_QUESTION)
% Copyright 2003, Tony R. Kuphaldt, released under the Creative Commons Attribution License (v 1.0)
% This means you may do almost anything with this work of mine, so long as you give me proper credit

A {\it strain gauge} is a device used to measure the strain (compression or expansion) of a solid object by producing a resistance change proportional to the amount of strain.  As the gauge is strained, its electrical resistance alters slightly due to changes in wire cross-section and length.

The following strain gauge is shown connected in a "quarter-bridge" circuit (meaning only one-quarter of the bridge actively senses strain, while the other three-quarters of the bridge are fixed in resistance):

$$\epsfbox{00547x01.eps}$$

Explain what would happen to the voltage measured across this bridge circuit ($V_{AB}$) if the strain gauge were to be {\it compressed}, assuming that the bridge begins in a balanced condition with no strain on the gauge.

\underbar{file 00547}
%(END_QUESTION)





%(BEGIN_ANSWER)

The bridge circuit will become more unbalanced, with more strain experienced by the strain gauge.  I will not tell you what the voltmeter's {\it polarity} will be, however!

%(END_ANSWER)





%(BEGIN_NOTES)

Be sure to have your students explain how they arrived at their answers for polarity across the voltmeter terminals.  This is the most important part of the question!

%INDEX% Bridge circuit, used to measure strain
%INDEX% Strain gauge

%(END_NOTES)


