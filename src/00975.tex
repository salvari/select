
%(BEGIN_QUESTION)
% Copyright 2003, Tony R. Kuphaldt, released under the Creative Commons Attribution License (v 1.0)
% This means you may do almost anything with this work of mine, so long as you give me proper credit

A popular variation of the Class B amplifier is the {\it Class AB} amplifier, designed to eliminate any trace of crossover distortion.  What makes the difference between a Class B and a Class AB amplifier?  Why do Class AB amplifiers have less crossover distortion than Class B amplifiers?  And, is there any disadvantage to changing from Class B to Class AB operation?

\underbar{file 00975}
%(END_QUESTION)





%(BEGIN_ANSWER)

The fundamental difference between Class B and Class AB operation is biasing: both transistors are "on" for a brief moment in time around the zero-crossover point in a Class AB circuit, where only one transistor is supposed to be on at any given time in a Class B circuit.

Amplifiers operating in Class AB mode are less power-efficient than pure Class B operation.

%(END_ANSWER)





%(BEGIN_NOTES)

Ask your students to specifically identify the change(s) that would have to be made in the following Class B circuit to make it operate as a Class AB amplifier:

$$\epsfbox{00975x01.eps}$$

Discuss why the name "Class AB" is given to this mode of operation.  How does Class AB operation differ from pure Class A or pure Class B?

%INDEX% Class-AB amplifier operation, defined
%INDEX% Class-AB versus class-B push-pull amplifiers

%(END_NOTES)


