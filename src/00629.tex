
%(BEGIN_QUESTION)
% Copyright 2003, Tony R. Kuphaldt, released under the Creative Commons Attribution License (v 1.0)
% This means you may do almost anything with this work of mine, so long as you give me proper credit

Devise a method of identifying the poles of a magnet that is too large and heavy to move.

\underbar{file 00629}
%(END_QUESTION)





%(BEGIN_ANSWER)

I'll give you a hint: you could use another magnet!

%(END_ANSWER)





%(BEGIN_NOTES)

This is another question designed to provoke students to think creatively in problem-solving.  They should already understand the principle of suspending a magnet and using the Earth's natural magnetic field to identify the magnet's poles.  However, this requires allowing the magnet to freely rotate (hence the requirement of suspension) so it can act as a compass.  Using the same process to identify the poles of an immovable magnet may require the use of an intermediary magnet, combined with the knowledge of magnet pole interactions (repulsion versus attraction).

%INDEX% Magnetic south
%INDEX% Magnetic north

%(END_NOTES)


