
%(BEGIN_QUESTION)
% Copyright 2005, Tony R. Kuphaldt, released under the Creative Commons Attribution License (v 1.0)
% This means you may do almost anything with this work of mine, so long as you give me proper credit

This phase-shifting bridge circuit is supposed to provide an output voltage with a variable phase shift from -45$^{o}$ (lagging) to +45$^{o}$ (leading), depending on the position of the potentiometer wiper:

$$\epsfbox{03465x01.eps}$$

Suppose, though, that the output signal is stuck at -45$^{o}$ lagging the source voltage, no matter where the potentiometer is set.  Identify a likely failure that could cause this to happen, and explain why this failure could account for the circuit's strange behavior.

\underbar{file 03465}
%(END_QUESTION)





%(BEGIN_ANSWER)

A broken connection between the right-hand terminal of the potentiometer and the bridge could cause this to happen:

$$\epsfbox{03465x02.eps}$$

I'll let you figure out why!

%(END_ANSWER)





%(BEGIN_NOTES)

It is essential, of course, that students understand the operational principle of this circuit before they may even attempt to diagnose possible faults.  You may find it necessary to discuss this circuit in detail with your students before they are ready to troubleshoot it.

In case anyone asks, the symbolism $R_{pot} >> R$ means "potentiometer resistance is {\it much} greater than the fixed resistance value."

%INDEX% Phase shifter circuit, adjustable
%INDEX% Troubleshooting, phase-shifter bridge circuit

%(END_NOTES)


