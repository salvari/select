
%(BEGIN_QUESTION)
% Copyright 2003, Tony R. Kuphaldt, released under the Creative Commons Attribution License (v 1.0)
% This means you may do almost anything with this work of mine, so long as you give me proper credit

Shown here are two digital components: a D-type {\it latch} and a D-type {\it flip-flop}:

$$\epsfbox{01365x01.eps}$$

Other than the silly name, what distinguishes a "flip-flop" from a latch?  How do the two circuits differ in function? 

\underbar{file 01365}
%(END_QUESTION)





%(BEGIN_ANSWER)

A "flip-flop" is a latch that changes output only at the rising or falling {\it edge} of the clock pulse.

%(END_ANSWER)





%(BEGIN_NOTES)

Note to your students that the timing input of a flip-flop is called a {\it clock} rather than an {\it enable}.  Ask them to identify what differences in symbolism show this distinction between the two devices.

%INDEX% Flip-flop, defined
%INDEX% Latch versus flip-flop
%INDEX% Flip-flop versus latch

%(END_NOTES)


