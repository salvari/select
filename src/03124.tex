
%(BEGIN_QUESTION)
% Copyright 2005, Tony R. Kuphaldt, released under the Creative Commons Attribution License (v 1.0)
% This means you may do almost anything with this work of mine, so long as you give me proper credit

Compare and contrast these three techniques for printed circuit board soldering:

\medskip
\item{$\bullet$} Hand
\item{$\bullet$} Wave
\item{$\bullet$} Reflow
\medskip

\underbar{file 03124}
%(END_QUESTION)





%(BEGIN_ANSWER)

{\it Hand} soldering is exactly what is sounds like: soldering done manually, by a human being.

\vskip 10pt

{\it Wave} soldering momentarily passes the PCB just over a bath of molten solder, allowing the solder to "wick" onto the pads and through via holes.

\vskip 10pt

{\it Reflow} soldering uses solder "paste" applied to each pad, and a hot oven to melt the solder for joining components to the board.

\vskip 20pt

Follow-up question: which technique is most appropriate for the following applications?

\medskip
\item{$\bullet$} Mass production with through-hole components
\item{$\bullet$} Mass production with surface-mount components
\item{$\bullet$} Single-board prototyping
\medskip

%(END_ANSWER)





%(BEGIN_NOTES)

If possible, show videos of automated soldering processes to your students.  It is quite educational (and awe-inspiring) for students familiar with tedious hand-soldering techniques to see how large volumes of circuit boards can be soldered using modern wave and reflow techniques.

%INDEX% Reflow soldering, defined
%INDEX% Soldering, reflow
%INDEX% Soldering, wave
%INDEX% Wave soldering, defined

%(END_NOTES)


