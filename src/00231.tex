
%(BEGIN_QUESTION)
% Copyright 2003, Tony R. Kuphaldt, released under the Creative Commons Attribution License (v 1.0)
% This means you may do almost anything with this work of mine, so long as you give me proper credit

What, exactly, is necessary to establish electrical {\it continuity} between two wires?  If I want to have an electric current flow out of one wire and into another, what must be done with those two wires to make that flow path complete?

Conversely, what things might prevent continuity from being established between two wires when they are supposed to electrically connect with one another?

\underbar{file 00231}
%(END_QUESTION)





%(BEGIN_ANSWER)

There must be metal-to-metal contact between the two wires in order to establish electrical continuity between the two.  Anything preventing this clean contact between metal surfaces will inhibit continuity.  This includes dirt, dust, oil, corrosion, misplaced insulation, and the like.

%(END_ANSWER)





%(BEGIN_NOTES)

It might be helpful to show students what real pieces of wire look like, in order for them to better understand the nature of the problem.  Most electrical wire is {\it insulated} in one form or another, and this insulation must not be removed (or "stripped") in order to establish bare metal-to-metal contact.

%INDEX% Continuity

%(END_NOTES)


