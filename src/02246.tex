
%(BEGIN_QUESTION)
% Copyright 2004, Tony R. Kuphaldt, released under the Creative Commons Attribution License (v 1.0)
% This means you may do almost anything with this work of mine, so long as you give me proper credit

The following graph is a family of characteristic curves for a particular transistor:

$$\epsfbox{02246x01.eps}$$

Superimpose on that graph a load line for the following common-emitter amplifier circuit using the same transistor:

$$\epsfbox{02246x02.eps}$$

Also determine some bias resistor values ($R_1$ and $R_2$) that will cause the Q-point to rest approximately mid-way on the load line.

\vskip 10pt

$R_1$ = \hskip 80pt $R_2$ =

\underbar{file 02246}
%(END_QUESTION)





%(BEGIN_ANSWER)

$$\epsfbox{02246x03.eps}$$

There are several pairs of resistor values that will work adequately to position the Q-point at the center of the load line.  I leave this an an exercise for you to work through and discuss with your classmates!

\vskip 10pt

Follow-up question: determine what would happen to the Q-point if resistor $R_2$ (the 2.2 k$\Omega$ biasing resistor) were to fail open.

%(END_ANSWER)





%(BEGIN_NOTES)

This is a very practical question, as technicians and engineers alike need to choose proper biasing so their amplifier circuits will operate in the intended class (A, in this case).  There is more than one proper answer for the resistor values, so be sure to have your students share their solutions with the whole class so that many options may be explored.

%INDEX% Q point, illustrated on common-emitter amplifier circuit

%(END_NOTES)


