
%(BEGIN_QUESTION)
% Copyright 2005, Tony R. Kuphaldt, released under the Creative Commons Attribution License (v 1.0)
% This means you may do almost anything with this work of mine, so long as you give me proper credit

Match the following time-delay relay contact type symbols and labels:

$$\epsfbox{03134x01.eps}$$

\medskip
\item{$\bullet$} Normally-open, on-delay
\item{$\bullet$} Normally-open, off-delay
\item{$\bullet$} Normally-closed, on-delay
\item{$\bullet$} Normally-closed, off-delay
\medskip

\underbar{file 03134}
%(END_QUESTION)





%(BEGIN_ANSWER)

$$\epsfbox{03134x02.eps}$$

\vskip 10pt

Follow-up question: how do you make sense of the arrow in each contact symbol, with regard to whether the contact is an "on-delay" or an "off-delay"?

%(END_ANSWER)





%(BEGIN_NOTES)

Ask your students to present their personal explanations of how to make sense of the arrow directions, in relation to whether the relay is an "on-delay" or an "off-delay."  The correlation is really not that complex, but it is a good thing to clearly elaborate on it for the benefit of the whole class.  You may want to re-phrase the question like this: "Is it possible to determine whether each contact is on- or off-delay merely by looking at the arrow, or must one also consider the "normal" status?"

If some students believe this may be determined by arrow direction alone, show them these symbols:

$$\epsfbox{03134x03.eps}$$

%INDEX% Relay contact, time delay (types)
%INDEX% Time delay relay contact types

%(END_NOTES)


