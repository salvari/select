
%(BEGIN_QUESTION)
% Copyright 2003, Tony R. Kuphaldt, released under the Creative Commons Attribution License (v 1.0)
% This means you may do almost anything with this work of mine, so long as you give me proper credit

Write the Boolean expression for this TTL logic gate circuit, then reduce that expression to its simplest form using any applicable Boolean laws and theorems.  Finally, draw a new gate circuit diagram based on the simplified Boolean expression, that performs the exact same logic function.

$$\epsfbox{01317x01.eps}$$

\underbar{file 01317}
%(END_QUESTION)





%(BEGIN_ANSWER)

Original Boolean expression: $A + \overline{\overline{AB}C}$

\vskip 10pt

Reduced gate circuit:

$$\epsfbox{01317x02.eps}$$

\vskip 10pt

Challenge question: implement this reduced circuit, using the only remaining gates between the two integrated circuits shown on the original breadboard.

%(END_ANSWER)





%(BEGIN_NOTES)

Ask your students to explain what advantages there may be to using the simplified gate circuit rather than the original (more complex) gate circuit shown in the question.  What significance does this lend to learning Boolean algebra?

This is what Boolean algebra is really for: reducing the complexity of logic circuits.  It is far too easy for students to lose sight of this fact, learning all the abstract rules and laws of Boolean algebra.  Remember, in teaching Boolean algebra, you are supposed to be preparing students to perform manipulations of {\it electronic circuits}, not just equations.

%INDEX% Boolean algebra, gate circuit simplification

%(END_NOTES)


