
%(BEGIN_QUESTION)
% Copyright 2003, Tony R. Kuphaldt, released under the Creative Commons Attribution License (v 1.0)
% This means you may do almost anything with this work of mine, so long as you give me proper credit

A qualitative model for transistor behavior is that of a variable resistor.  While grossly inaccurate in a quantitative (numbers) sense, it at least provides a means of assessing its behavior in the sense of "more conductive" or "less conductive".

$$\epsfbox{00885x01.eps}$$

Apply this "model" of transistor behavior to this single-transistor amplifier circuit, and describe what happens to the collector voltage ($V_C$) and emitter voltage ($V_E$) when the input voltage ($V_{in}$) increases and decreases:

$$\epsfbox{00885x02.eps}$$

\underbar{file 00885}
%(END_QUESTION)





%(BEGIN_ANSWER)

When $V_{in}$ increases, $V_E$ increases and $V_C$ decreases.  When $V_{in}$ decreases, $V_E$ decreases and $V_C$ increases.

%(END_ANSWER)





%(BEGIN_NOTES)

If this concept is confusing to any of your students, draw a pair of three-resistor voltage divider circuits on the whiteboard for everyone to see, and then have the students analyze the voltage drops in two conditions: when the middle resistor is a low value, and when the middle resistor is a high value.

Astute students will recognize this transistor circuit configuration as a {\it phase splitter}.

%INDEX% BJT model

%(END_NOTES)


