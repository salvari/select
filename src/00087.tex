
%(BEGIN_QUESTION)
% Copyright 2003, Tony R. Kuphaldt, released under the Creative Commons Attribution License (v 1.0)
% This means you may do almost anything with this work of mine, so long as you give me proper credit

Plot the relationship between power and current for a 2 $\Omega$ resistor on this graph:

$$\epsfbox{00087x01.eps}$$

What pattern do you see represented by plot?  How does this compare with the graphical relationship between {\it voltage} and current for a resistor?

\underbar{file 00087}
%(END_QUESTION)





%(BEGIN_ANSWER)

The more current through the resistor, the more power dissipated.  However, this is {\it not} a linear function!

%(END_ANSWER)





%(BEGIN_NOTES)

Students need to become comfortable with graphs, and creating their own simple graphs is an excellent way to develop this understanding.  A graphical representation of the Ohm's Law (actually, Joule's Law) power function allows students another "view" of the concept.

If students have access to either a graphing calculator or computer software capable of drawing 2-dimensional graphs, encourage them to plot the functions using these technological resources.

%INDEX% Ohm's Law
%INDEX% Joule's Law
%INDEX% Graphing

%(END_NOTES)


