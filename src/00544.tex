
%(BEGIN_QUESTION)
% Copyright 2003, Tony R. Kuphaldt, released under the Creative Commons Attribution License (v 1.0)
% This means you may do almost anything with this work of mine, so long as you give me proper credit

In general terms, describe what must be done to {\it balance} this bridge circuit.  What, exactly, does the term "balance" mean in this context?

$$\epsfbox{00544x01.eps}$$

Also, write an equation containing only the four resistor values ($R_1$, $R_2$, $R_3$, and $R_4$) showing their relationship to one another in a balanced condition.

\underbar{file 00544}
%(END_QUESTION)





%(BEGIN_ANSWER)

For a bridge circuit to be "balanced" means that there is zero voltage between the two opposite corners of the circuit (where the battery does {\it not}) connect.  Achieving a condition of "balance" in a bridge circuit requires that the resistance ratios of the four "arms" of the circuit be in proportion:

$${R_1 \over R_3} = {R_2 \over R_4}$$

\vskip 10pt

Follow-up question: the bridge-balance equation shown above may also be written in a slightly different form:

$${R_1 \over R_2} = {R_3 \over R_4}$$

Show algebraically how the first equation may be manipulated to take the form of the second equation, thus demonstrating these two equations' equivalence.

%(END_ANSWER)





%(BEGIN_NOTES)

Challenge your students to write a "balance equation" describing how the ratios must relate to each other in order to achieve balance.

%INDEX% Bridge circuit

%(END_NOTES)


