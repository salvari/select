
%(BEGIN_QUESTION)
% Copyright 2003, Tony R. Kuphaldt, released under the Creative Commons Attribution License (v 1.0)
% This means you may do almost anything with this work of mine, so long as you give me proper credit

In this AC circuit, the resistor offers 3 k$\Omega$ of resistance, and the capacitor offers 4 k$\Omega$ of reactance.  Together, their series opposition to alternating current results in a current of 1 mA from the 5 volt source:

$$\epsfbox{00585x01.eps}$$

How many ohms of opposition does the series combination of resistor and capacitor offer?  What name do we give to this quantity, and how do we symbolize it, being that it is composed of both resistance ($R$) and reactance ($X$)?

\underbar{file 00585}
%(END_QUESTION)





%(BEGIN_ANSWER)

$Z_{total} =$ 5 k$\Omega$.

%(END_ANSWER)





%(BEGIN_NOTES)

Students may experience difficulty arriving at the same quantity for impedance shown in the answer.  If this is the case, help them problem-solve by suggesting they {\bf simplify the problem}: short past one of the load components and calculate the new circuit current.  Soon they will understand the relationship between total circuit opposition and total circuit current, and be able to apply this concept to the original problem.

Ask your students why the quantities of 3 k$\Omega$ and 4 k$\Omega$ do not add up to 7 k$\Omega$ like they would if they were both resistors.  Does this scenario remind them of another mathematical problem where $3 + 4 = 5$?  Where have we seen this before, especially in the context of electric circuits?

Once your students make the cognitive connection to trigonometry, ask them the significance of these numbers' addition.  Is it enough that we say a component has an opposition to AC of 4 k$\Omega$, or is there more to this quantity than a single, scalar value?  What type of number would be suitable for representing such a quantity, and how might it be written?

%INDEX% Impedance, resulting from combination of X(C) and R in series
%INDEX% Impedance, defined in a circuit with 3-4-5 proportions

%(END_NOTES)


