
%(BEGIN_QUESTION)
% Copyright 2003, Tony R. Kuphaldt, released under the Creative Commons Attribution License (v 1.0)
% This means you may do almost anything with this work of mine, so long as you give me proper credit

When Digital Audio Tape (DAT) was first introduced to the American public, it was touted as delivering superior sound quality.  Most importantly, this high quality of sound was not supposed to degrade over time like standard (analog) audio cassette tape recordings.

The magnetic media from which DAT was manufactured was basically the same stuff used to make {\it analog} audio tape.  Explain why the encoding of audio data {\it digitally} on the same media would provide superior resistance to degradation over analog recordings even though the recording media was the same.  Also, explain how this is significant to modern digital data storage technologies such as those used to store photographic images and numerical data.

\underbar{file 01441}
%(END_QUESTION)





%(BEGIN_ANSWER)

The answer to why digital recordings retain their quality longer lies in the bivalent nature of digital data, being comprised of either "high" or "low" states, with nothing in between.  Consider a sine wave, directly recorded in analog form on magnetic tape, versus a {\it digitized} representation of a sine wave, recorded as a series of 1's and 0's on the same type of tape.  Now introduce some "noise" to each of the signals, and consider the results upon playback.

%(END_ANSWER)





%(BEGIN_NOTES)

Challenge students to come up with some {\it disadvantages} of digital recordings, now that they understand the difference between analog and digital data storage.  While digital technology certainly enjoys some advantages over analog, it is not necessarily superior in all aspects!

%INDEX% Digital versus analog, recording
%INDEX% Analog versus digital, recording

%(END_NOTES)


