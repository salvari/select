
%(BEGIN_QUESTION)
% Copyright 2003, Tony R. Kuphaldt, released under the Creative Commons Attribution License (v 1.0)
% This means you may do almost anything with this work of mine, so long as you give me proper credit

Is it possible to regard this complex network of power sources and resistances as a simple voltage source (one ideal voltage source in series with an internal resistance), or as a simple current source (one ideal current source in parallel with an internal resistance)?  Why or why not?

$$\epsfbox{01036x01.eps}$$

\underbar{file 01036}
%(END_QUESTION)





%(BEGIN_ANSWER)

Yes, because all the constituent components are {\it linear} and {\it bilateral}.

\vskip 10pt

Follow-up question: why would anyone want to represent this complex network as either a simple voltage source or a simple current source?

%(END_ANSWER)





%(BEGIN_NOTES)

Ask your students what it means for an electrical or electronic component to be considered "linear," and also what it means to be considered "bilateral."  Can they give examples of components that are nonlinear, and/or unilateral?

%INDEX% Bilateral, condition to apply Thevenin's or Norton's theorems
%INDEX% Linearity, condition to apply Thevenin's or Norton's theorems

%(END_NOTES)


