
%(BEGIN_QUESTION)
% Copyright 2003, Tony R. Kuphaldt, released under the Creative Commons Attribution License (v 1.0)
% This means you may do almost anything with this work of mine, so long as you give me proper credit

Calculate the power supply's output (total) current:

$$\epsfbox{01766x01.eps}$$

\underbar{file 01766}
%(END_QUESTION)





%(BEGIN_ANSWER)

$I_{total} = 4.69 \hbox{ mA}$

\vskip 10pt

Follow-up question: explain why the voltage across the 1500 $\Omega$ resistor would remain unchanged if the 4700 $\Omega$ resistor were to fail open.

\vskip 10pt

Challenge question: what crucial assumptions underlie the calculated figure for current shown here?  In other words, what unknown quantities can affect the accuracy of our predicted current value?

%(END_ANSWER)





%(BEGIN_NOTES)

This is an interesting series-parallel circuit problem to solve, and it shows once again how a good understanding of circuit theory enables unmeasured variables to be inferred.

\vskip 10pt

Students often have difficulty formulating a method of solution: determining what steps to take to get from the given conditions to a final answer.  While it is helpful at first for you (the instructor) to show them, it is bad for you to show them too often, lest they stop thinking for themselves and merely follow your lead.  A teaching technique I have found very helpful is to have students come up to the board (alone or in teams) in front of class to write their problem-solving strategies for all the others to see.  They don't have to actually do the math, but rather outline the steps they would take, in the order they would take them.

By having students \underbar{outline their problem-solving strategies}, everyone gets an opportunity to see multiple methods of solution, and you (the instructor) get to see how (and if!) your students are thinking.  An especially good point to emphasize in these "open thinking" activities is how to check your work to see if any mistakes were made.

%INDEX% Series-parallel circuit, voltage and current calculations in

%(END_NOTES)


