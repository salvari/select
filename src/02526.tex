
%(BEGIN_QUESTION)
% Copyright 2005, Tony R. Kuphaldt, released under the Creative Commons Attribution License (v 1.0)
% This means you may do almost anything with this work of mine, so long as you give me proper credit

The following circuit is known as an {\it instrumentation amplifier}:

$$\epsfbox{02526x01.eps}$$

Suppose a DC voltage were to be applied to the noninverting input terminal, +1 volt at $V_{in(+)}$, and the inverting input terminal grounded.  Complete the following table showing the output voltage of this circuit for different values of $m$:

% No blank lines allowed between lines of an \halign structure!
% I use comments (%) instead, so that TeX doesn't choke.

$$\vbox{\offinterlineskip
\halign{\strut
\vrule \quad\hfil # \ \hfil & 
\vrule \quad\hfil # \ \hfil \vrule \cr
\noalign{\hrule}
%
% First row
$m$ & $V_{out}$ \cr
%
\noalign{\hrule}
%
% Second row
1 &  \cr
%
\noalign{\hrule}
%
% Third row
2 &  \cr
%
\noalign{\hrule}
%
% Fourth row
3 &  \cr
%
\noalign{\hrule}
%
% Fifth row
4 &  \cr
%
\noalign{\hrule}
%
% Sixth row
5 &  \cr
%
\noalign{\hrule}
%
% Seventh row
6 &  \cr
%
\noalign{\hrule}
} % End of \halign 
}$$ % End of \vbox

\underbar{file 02526}
%(END_QUESTION)





%(BEGIN_ANSWER)

% No blank lines allowed between lines of an \halign structure!
% I use comments (%) instead, so that TeX doesn't choke.

$$\vbox{\offinterlineskip
\halign{\strut
\vrule \quad\hfil # \ \hfil & 
\vrule \quad\hfil # \ \hfil \vrule \cr
\noalign{\hrule}
%
% First row
$m$ & $V_{out}$ \cr
%
\noalign{\hrule}
%
% Second row
1 & 3 volts \cr
%
\noalign{\hrule}
%
% Third row
2 & 2 volts \cr
%
\noalign{\hrule}
%
% Fourth row
3 & 1.66 volts \cr
%
\noalign{\hrule}
%
% Fifth row
4 & 1.5 volts \cr
%
\noalign{\hrule}
%
% Sixth row
5 & 1.4 volts \cr
%
\noalign{\hrule}
%
% Seventh row
6 & 1.33 volts \cr
%
\noalign{\hrule}
} % End of \halign 
}$$ % End of \vbox

$$A_{V(diff)} = {{2 + m} \over m}$$

\vskip 10pt

Follow-up question: why did I choose to set the noninverting input voltage at +1 volts and ground the inverting input?  Should we not be able to calculate gain given {\it any} two input voltages and a value for $m$?  Explain the purpose behind my choice of input voltage conditions for this "thought experiment."

%(END_ANSWER)





%(BEGIN_NOTES)

While the relationship between instrumentation amplifier differential gain and $m$ may be looked up in any good opamp circuit textbook, it is something that your students should learn to figure out on their own from the data in the table.

%INDEX% Algebra, deriving equations from data
%INDEX% Instrumentation amplifier

%(END_NOTES)


