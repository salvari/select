
%(BEGIN_QUESTION)
% Copyright 2003, Tony R. Kuphaldt, released under the Creative Commons Attribution License (v 1.0)
% This means you may do almost anything with this work of mine, so long as you give me proper credit

An {\it octave} is a type of harmonic frequency.  Suppose an electronic circuit operates at a fundamental frequency of 1 kHz.  Calculate the frequencies of the following octaves:

\medskip
\item{$\bullet$} 1 octave greater than the fundamental = 
\item{$\bullet$} 2 octaves greater than the fundamental = 
\item{$\bullet$} 3 octaves greater than the fundamental = 
\item{$\bullet$} 4 octaves greater than the fundamental = 
\item{$\bullet$} 5 octaves greater than the fundamental = 
\item{$\bullet$} 6 octaves greater than the fundamental = 
\medskip

\underbar{file 01891}
%(END_QUESTION)





%(BEGIN_ANSWER)

\medskip
\item{$\bullet$} 1 octave greater than the fundamental = 2 kHz
\item{$\bullet$} 2 octaves greater than the fundamental = 4 kHz
\item{$\bullet$} 3 octaves greater than the fundamental = 8 kHz
\item{$\bullet$} 4 octaves greater than the fundamental = 16 kHz
\item{$\bullet$} 5 octaves greater than the fundamental = 32 kHz
\item{$\bullet$} 6 octaves greater than the fundamental = 64 kHz
\medskip

%(END_ANSWER)





%(BEGIN_NOTES)

Ask your students if they can determine the mathematical relationship between octave number, octave frequency, and fundamental frequency.  This is a bit more difficult to do than for integer harmonics, but not beyond reason if students are familiar with exponents.

Clarify for your students the fact that "octave" is not just a musical term.  In electronic circuit analysis (especially filter circuits), the word "octave" is often used to represent multiples of a given frequency, usually in reference to a bandwidth (i.e. "This filter's passband response is essentially flat over two octaves!").

%INDEX% Octave, defined as a special type of harmonic

%(END_NOTES)


