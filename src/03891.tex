
%(BEGIN_QUESTION)
% Copyright 2006, Tony R. Kuphaldt, released under the Creative Commons Attribution License (v 1.0)
% This means you may do almost anything with this work of mine, so long as you give me proper credit

Identify at least one component fault that would cause the "$Q$" LED to always stay on, no matter what was done with the input switches.

$$\epsfbox{03891x01.eps}$$

For each of your proposed faults, explain {\it why} it will cause the described problem.

\underbar{file 03891}
%(END_QUESTION)





%(BEGIN_ANSWER)

\medskip
\item{$\bullet$} NOR gate $U_2$ output failed high.
\vskip 5pt
\item{$\bullet$} Wire break between "Reset" switch and resistor $R_2$ (although if this was the only fault it may allow the $Q$ LED to energize at power-up, just not de-energize after the "Set" button had been pressed).
\medskip

\vskip 10pt

Follow-up question: explain why the nature of the problem rules out the possibility of the only fault being something related to the feedback connections between $U_1$ and $U_2$.

%(END_ANSWER)





%(BEGIN_NOTES)

Latch circuits can be confusing due to their use of positive feedback.  Questions such as this are important tools for helping develop your students' understanding of latch circuits.

%INDEX% Troubleshooting, S-R latch circuit

%(END_NOTES)


