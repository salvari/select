
%(BEGIN_QUESTION)
% Copyright 2003, Tony R. Kuphaldt, released under the Creative Commons Attribution License (v 1.0)
% This means you may do almost anything with this work of mine, so long as you give me proper credit

Shown here is a simplified representation of an {\it atom}: the smallest division of matter that may be isolated through physical or chemical methods.

$$\epsfbox{00110x01.eps}$$

Inside of each atom are several smaller bits of matter called {\it particles}.  Identify the three different types of "elementary" particles inside an atom, their electrical properties, and their respective locations within the atom.

\underbar{file 00110}
%(END_QUESTION)





%(BEGIN_ANSWER)

$$\epsfbox{00110x02.eps}$$

Neutrons reside in the center ("nucleus") of the atom, as do protons.  Neutrons are electrically neutral (no charge), while protons have a positive electrical charge.  Electrons, which reside outside the nucleus, have negative electrical charges.

%(END_ANSWER)





%(BEGIN_NOTES)

Most, if not all, students will be familiar with the "solar system" model of an atom, from primary and secondary science education.  In reality, though, this model of atomic structure is not that accurate.  As far as anyone knows, the actual physical layout of an atom is much, much weirder than this!

A question that might come up in discussion is the definition of "charge."  I'm not sure if it is possible to fundamentally define what "charge" is.  Of course, we may discuss "positive" and "negative" charges in operational terms: that like charges repel and opposite charges attract.  However, this does not really tell us {\it what} charge actually is.  This philosophical quandary is common in science: to be able to describe what something is in terms of its behavior but not its identity or nature.

%INDEX% Atom
%INDEX% Particles, subatomic
%INDEX% Subatomic particles
%INDEX% Proton
%INDEX% Neutron
%INDEX% Electron

%(END_NOTES)


