
%(BEGIN_QUESTION)
% Copyright 2003, Tony R. Kuphaldt, released under the Creative Commons Attribution License (v 1.0)
% This means you may do almost anything with this work of mine, so long as you give me proper credit

Suppose you were troubleshooting the following amplifier circuit, and found the output signal to be symmetrically "clipped" on both the positive and negative peaks:

$$\epsfbox{01582x01.eps}$$

$$\epsfbox{01582x02.eps}$$

If you knew that this amplifier was a new design, and might not have all its components properly sized, what type of problem would you suspect in the circuit?  Please be as specific as possible.

\underbar{file 01582}
%(END_QUESTION)





%(BEGIN_ANSWER)

This amplifier suffers from excessive gain, which may be remedied by changing the value of $R_C$ or $R_E$.  (I'll let you determine which way the chosen resistor value must be altered, increase or decrease!)

Of course, changing either of these resistor values will alter the bias ("Q") point of the amplifier, which may necessitate subsequent changes in the value of either $R_1$ or $R_2$!

%(END_ANSWER)





%(BEGIN_NOTES)

Discuss with your students how to determine the necessary changes in resistor values, based on the determination that the gain is excessive.  This is actually very easy to do just by examining the gain formula for a common-emitter amplifier.

Another option to consider here is the addition of a negative feedback signal path to tame the amplifier's gain.  This modification would have the added benefit of improving circuit linearity.

%INDEX% Troubleshooting, common-emitter transistor amplifier circuit

%(END_NOTES)


