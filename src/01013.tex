
%(BEGIN_QUESTION)
% Copyright 2003, Tony R. Kuphaldt, released under the Creative Commons Attribution License (v 1.0)
% This means you may do almost anything with this work of mine, so long as you give me proper credit

Draw the appropriate wires in this diagram to complete the circuit for a {\it differentiator} circuit (the opposite function of an {\it integrator} circuit):

$$\epsfbox{01013x01.eps}$$

The particular opamp used in this circuit is an LM324 (quad) operational amplifier.  Of course, you need only show how one of the four opamps within the IC may be connected to form a differentiator.

\underbar{file 01013}
%(END_QUESTION)





%(BEGIN_ANSWER)

$$\epsfbox{01013x02.eps}$$

\vskip 10pt

Challenge question: given the single-supply configuration (+9 volts only), this differentiator circuit can only respond to one direction of input voltage change (${dv_{in} \over dt}$).  Which direction is this: positive or negative?

%(END_ANSWER)





%(BEGIN_NOTES)

As always with problems centered around a pictorial diagram, the solution is more apparent if a schematic diagram is drawn as a "map" first.  Ask your students to explain how they transitioned from the original diagram, to their own schematic, back to a completed diagram.

Also discuss the similarities between differentiator circuits and integrator circuits.  The two mathematical functions are inverse operations of each other.  How is this symmetry reflected in the respective circuit configurations?

%INDEX% Differentiator circuit, opamp

%(END_NOTES)


