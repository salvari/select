
%(BEGIN_QUESTION)
% Copyright 2005, Tony R. Kuphaldt, released under the Creative Commons Attribution License (v 1.0)
% This means you may do almost anything with this work of mine, so long as you give me proper credit

The following recommendations came from a flyer published by an electric power utility.  Read and comment on their instructions regarding downed power lines:

\vskip 10pt {\narrower \noindent \baselineskip5pt

{\it Assume any downed line is an energized power line.  If a power line falls on your car while you are driving, slowly drive on until you are completely clear of the line (but do not drive over it).  If your car is immobilized, stay in it until help arrives.  Call for help from a cell phone if you have one.}

{\it If you need to escape from a vehicle, such as for a car fire, jump clear of the car.  Electricity is not only traveling through the vehicle, but may also be traveling in the ground around the area.  Keep your arms crossed over your chest while you jump, and both feet together.  Do NOT touch the vehicle and the ground at the same time.  Once you land on the ground, shuffle clear of the area, keeping both feet together, on the ground, and touching at all times.  Continue shuffling until you're at least 30 feet from the accident site.}

\par} \vskip 10pt

Why do you suppose the following actions were recommended?

\medskip
\goodbreak
\item{$\bullet$} Do not drive over a downed power line.
\item{$\bullet$} Stay in the car if possible.
\item{$\bullet$} Do not touch the car and the ground at the same time.
\item{$\bullet$} Shuffle away from the car (rather than walk), with both feet together.
\medskip

\underbar{file 02911}
%(END_QUESTION)





%(BEGIN_ANSWER)

These are all interesting points to consider as a group.  I'll let you figure out possible answers to these questions together in class.

%(END_ANSWER)





%(BEGIN_NOTES)

The italicized text was taken directly from Puget Sound Energy's April 2005 "Energywise" mail flyer.  The points brought up were, I thought, very appropriate for discussion of electrical safety and theory.  Personally, I question the suggestion of shuffling on both feet.  I would suspect that running full-speed, where only one foot touches the ground at a time, and where you would leave the dangerous area faster, would be the safest option.  I would be very interested to see if there is any scientific test data available on this subject!

%INDEX% Safety, electrical

%(END_NOTES)


