
%(BEGIN_QUESTION)
% Copyright 2005, Tony R. Kuphaldt, released under the Creative Commons Attribution License (v 1.0)
% This means you may do almost anything with this work of mine, so long as you give me proper credit

In ladder logic symbolism, an electromechanical relay coil is shown as a circle, and the contact(s) actuated by the coil as two parallel lines, almost like a capacitor symbol.  Given this knowledge, interpret the following ladder logic diagram:

$$\epsfbox{02774x01.eps}$$

How do we know which relay contact is actuated by which relay coil?  How does this convention differ from that of standard electrical/electronic schematic diagrams, where the relay coil is shown as an actual coil of wire (inductor symbol) with the contact "linked" to the coil by a dashed line?  Also, what type of logic function behavior (AND, OR, NAND, or NOR) does the above circuit exhibit?

\underbar{file 02774}
%(END_QUESTION)





%(BEGIN_ANSWER)

In ladder logic diagrams, relay coils are associated with their respective contacts by {\it name} rather than by proximity.  In this particular circuit, the logic function represented is the AND function.

%(END_ANSWER)





%(BEGIN_NOTES)

Many students find it confusing that relay contacts and coils need not be drawn next to one another in a ladder logic diagram, because it is so different from the schematic diagrams they are accustomed to.  The non-necessity of proximity in a ladder logic diagram does have its advantages, though!  It is simply a matter of getting used to a new way of drawing things.

%INDEX% Ladder logic diagram (with relays)
%INDEX% OR logic function, implemented with electromechanical relays
%INDEX% Relay logic, OR function

%(END_NOTES)


