
%(BEGIN_QUESTION)
% Copyright 2003, Tony R. Kuphaldt, released under the Creative Commons Attribution License (v 1.0)
% This means you may do almost anything with this work of mine, so long as you give me proper credit

An essential part of an AC-DC power supply circuit is the {\it filter}, used to separate the residual AC (called the "ripple" voltage) from the DC voltage prior to output.  Here are two simple AC-DC power supply circuits, one without a filter and one with:

$$\epsfbox{00789x01.eps}$$

Draw the respective output voltage waveforms of these two power supply circuits ($V_{unfiltered}$ versus $V_{filtered}$).  Also identify the type of filter circuit needed for the task (low pass, high pass, band pass, or band stop), and explain why that type of filter circuit is needed.

\underbar{file 00789}
%(END_QUESTION)





%(BEGIN_ANSWER)

$$\epsfbox{00789x02.eps}$$

A low pass filter is the kind needed to filter "ripple voltage" from the power supply output.

%(END_ANSWER)





%(BEGIN_NOTES)

Many years ago, when I was first learning about power supplies, I tried to power an automotive radio with voltage from a battery charger.  The battery charger was a simple power supply suitable for charging 12-volt automotive batteries, I reasoned, so what harm would there be in using it to power an automotive radio?  After a few moments of LOUD humming from the radio speaker, my rhetorical question was answered by a puff of smoke from the radio, then silence.

Part of the problem was the output voltage of the battery charger, but a large part of the problem was the fact that the charger's output was {\it unfiltered} as well.  For the same reasons my radio did not function properly on unfiltered, rectified AC, many electronic circuits will not function on it either.

%INDEX% Filter, power supply

%(END_NOTES)


