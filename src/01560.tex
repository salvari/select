
%(BEGIN_QUESTION)
% Copyright 2003, Tony R. Kuphaldt, released under the Creative Commons Attribution License (v 1.0)
% This means you may do almost anything with this work of mine, so long as you give me proper credit

\vbox{\hrule \hbox{\strut \vrule{} $\int f(x) \> dx$ \hskip 5pt {\sl Calculus alert!} \vrule} \hrule}

In a simple resistor circuit, the current may be calculated by dividing applied voltage by resistance:

$$\epsfbox{01560x03.eps}$$

Although an analysis of this circuit probably seems trivial to you, I would like to encourage you to look at what is happening here from a fresh perspective.  An important principle observed many times in the study of physics is that of {\it equilibrium}, where quantities naturally "seek" a state of balance.  The balance sought by this simple circuit is equality of voltage: the voltage across the resistor must settle at the same value as the voltage output by the source:

$$\epsfbox{01560x04.eps}$$

If the resistor is viewed as a source of voltage seeking equilibrium with the voltage source, then current {\it must} converge at whatever value necessary to generate the necessary balancing voltage across the resistor, according to Ohm's Law ($V = IR$).  In other words, {\it the resistor's current achieves whatever magnitude it has to in order to generate a voltage drop equal to the voltage of the source}.

This may seem like a strange way of analyzing such a simple circuit, with the resistor "seeking" to generate a voltage drop equal to the source, and current "magically" assuming whatever value it must to achieve that voltage equilibrium, but it is helpful in understanding other types of circuit elements.

\vskip 10pt

For example, here we have a source of DC voltage connected to a large coil of wire through a switch.  Assume that the wire coil has negligible resistance ($0 \> \Omega$):

$$\epsfbox{01560x01.eps}$$

Like the resistor circuit, the coil will "seek" to achieve voltage equilibrium with the voltage source once the switch is closed.  However, we know that the voltage induced in a coil is not directly proportional to current as it is with a resistor -- instead, a coil's voltage drop is proportional to the {\it rate of change of magnetic flux over time} as described by Faraday's Law of electromagnetic induction:

$$v_{coil} = N {d \phi \over dt}$$

\noindent
Where,

$v_{coil} =$ Instantaneous induced voltage, in volts

$N =$ Number of turns in wire coil

${d \phi \over dt} =$ Instantaneous rate of change of magnetic flux, in webers per second

\vskip 10pt

Assuming a linear relationship between coil current and magnetic flux (i.e. $\phi$ doubles when $i$ doubles), describe this simple circuit's current over time after the switch closes.

\underbar{file 01560}
%(END_QUESTION)





%(BEGIN_ANSWER)

When the switch closes, current will steadily increase at a linear rate over time:

$$\epsfbox{01560x02.eps}$$

\vskip 10pt

Challenge question: real wire coils contain electrical resistance (unless they're made of superconducting wire, of course), and we know how voltage equilibrium occurs in resistive circuits: the current converges at a value necessary for the resistance to drop an equal amount of voltage as the source.  Describe, then, what the current does in a circuit with a {\it real} wire coil, not a superconducting wire coil.

%(END_ANSWER)





%(BEGIN_NOTES)

Students who do not yet understand the concept of inductance may be inclined to suggest that the current in this circuit will be infinite, following Ohm's Law ($I = E/R$).  One of the purposes of this question is to reveal such misunderstandings, so that they may be corrected.

This circuit provides an excellent example of the calculus principle {\it integration}, where the application of a steady voltage across the inductor results in a steadily {\it increasing} current.  Whether or not you should touch on this subject depends on the mathematical aptitude of your students.

%INDEX% Self-induction

%(END_NOTES)


