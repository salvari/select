
%(BEGIN_QUESTION)
% Copyright 2003, Tony R. Kuphaldt, released under the Creative Commons Attribution License (v 1.0)
% This means you may do almost anything with this work of mine, so long as you give me proper credit

Explain why Binary Coded Decimal (BCD) is sometimes referred to as the "8421" code.  Why is this code used at all?

\underbar{file 01239}
%(END_QUESTION)





%(BEGIN_ANSWER)

BCD uses groups of four binary bits to represent each digit of a decimal number.  The LSD place weights are 8-4-2-1, while the next significant digit's place weightings are 80-40-20-10, and so on.

\vskip 10pt

Follow-up question: the four bits used for each BCD character could be called half of a byte (8 bits).  There is a special word for a four-bit grouping.  What is that word?

%(END_ANSWER)





%(BEGIN_NOTES)

Discuss with your students the purpose of using BCD to represent decimal quantities.  While not an efficient usage of bits, BCD certainly is convenient for representing decimal figures with discrete (0 or 1) logic states.

%INDEX% BCD code, defined

%(END_NOTES)


