
%(BEGIN_QUESTION)
% Copyright 2005, Tony R. Kuphaldt, released under the Creative Commons Attribution License (v 1.0)
% This means you may do almost anything with this work of mine, so long as you give me proper credit

This vocal eliminator circuit used to work just fine, but then one day it seemed to lose a lot of its bass.  It still did its job of eliminating the vocal track, but instead of hearing the full range of musical tones it only reproduced the high frequencies, while the low frequency tones were lost:

$$\epsfbox{03796x01.eps}$$

Identify the following fault possibilities:

\vskip 10pt

\noindent
One resistor failure (either open or shorted) that could cause this to happen:

\vskip 15pt

\noindent
One capacitor failure (either open or shorted) that could cause this to happen:

\vskip 15pt

\noindent
One opamp failure that could cause this to happen:

\vskip 15pt

For each of these proposed faults, explain {\it why} the bass tones would be lost.

\underbar{file 03796}
%(END_QUESTION)





%(BEGIN_ANSWER)

Please note that the following list is not exhaustive.  That is, other component faults may be possible!

\vskip 10pt

\noindent
One resistor failure (either open or shorted) that could cause this to happen: {\it $R_8$ failed open.}

\vskip 15pt

\noindent
One capacitor failure (either open or shorted) that could cause this to happen: {\it $C_2$ failed shorted.}

\vskip 15pt

\noindent
One opamp failure that could cause this to happen: {\it $U_4$ failed.}

%(END_ANSWER)





%(BEGIN_NOTES)

Ask your students to explain how they identified their proposed faults, and also how they were able to identify component that {\it must} still be working properly.

%INDEX% Troubleshooting, vocal eliminator circuit

%(END_NOTES)


