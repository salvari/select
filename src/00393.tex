
%(BEGIN_QUESTION)
% Copyright 2003, Tony R. Kuphaldt, released under the Creative Commons Attribution License (v 1.0)
% This means you may do almost anything with this work of mine, so long as you give me proper credit

Define the following DC motor terms:

\medskip
\item{$\bullet$} Field
\item{$\bullet$} Armature
\item{$\bullet$} Commutator
\item{$\bullet$} Brush
\medskip

\underbar{file 00393}
%(END_QUESTION)





%(BEGIN_ANSWER)

\medskip
\item{$\bullet$} Field: {\it the portion of the motor creating the stationary magnetic field}
\item{$\bullet$} Armature: {\it the rotating portion of the motor}
\item{$\bullet$} Commutator: {\it copper strips where the armature coil leads terminate, usually located at one end of the shaft}
\item{$\bullet$} Brush: {\it a stationary carbon block designed to electrically contact the moving commutator bars}
\medskip

%(END_ANSWER)





%(BEGIN_NOTES)

Students may find pictures of DC electric motors in their search for these terms' definitions.  Have them show these pictures to the class if possible.  Also, a disassembled electric motor is a great "prop" for discussion on electric motor nomenclature.

%INDEX% Armature (motor), defined
%INDEX% Commutator (motor), defined
%INDEX% Brush (motor), defined
%INDEX% Field (motor), defined

%(END_NOTES)


