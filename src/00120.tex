
%(BEGIN_QUESTION)
% Copyright 2003, Tony R. Kuphaldt, released under the Creative Commons Attribution License (v 1.0)
% This means you may do almost anything with this work of mine, so long as you give me proper credit

In this circuit, where would you {\it not} expect to measure significant voltage (between what pairs of test points)?

$$\epsfbox{00120x01.eps}$$

\underbar{file 00120}
%(END_QUESTION)





%(BEGIN_ANSWER)

You should not measure any significant voltage between any of the test points along the upper wire (A to B, A to C, A to D, etc.), nor between any of the test points along the lower wire (F to G, F to H, F to I, etc.).  As a general rule, points in a circuit that are {\it electrically common} to each other should never have voltage between them.

%(END_ANSWER)





%(BEGIN_NOTES)

The answer uses a concept which I've found to be very helpful in understanding electrical circuits: the idea of points in a circuit being {\it electrically common} to each other.  Simply put, this means the points are connected together by conductors of negligible resistance.  Having nearly 0 ohms of resistance between points assures insignificant voltage drop, even for large currents.

Conversely, if significant voltage is measured between points in a circuit, you can be assured that those points are {\it not} electrically common to each other.  Engage your students in a discussion of electrical commonality and expected voltage drops: 

\vskip 10pt

\item {$\bullet$} If voltage is measured between two points in a circuit, are those two points electrically common to each other?  Why or why not?
\item {$\bullet$} If no voltage is measured between two points in a circuit, are those two points electrically common to each other?  Why or why not?

%INDEX% Troubleshooting, simple circuit
%INDEX% Electrically common points

%(END_NOTES)


