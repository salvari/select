
%(BEGIN_QUESTION)
% Copyright 2004, Tony R. Kuphaldt, released under the Creative Commons Attribution License (v 1.0)
% This means you may do almost anything with this work of mine, so long as you give me proper credit

In this circuit, the values of capacitor $C_1$ and resistor $R_1$ are chosen to provide a short time constant, so they act as a differentiator network.  This results in a brief pulse of voltage across $R_1$ at each leading edge of the square wave input.  Capacitor $C_2$ and resistor $R_2$ are sized to provide a long time constant, so as to form an integrator network.  This time-averages the brief pulses into a final DC output voltage relatively free of ripple.

$$\epsfbox{02129x01.eps}$$

Explain what happens to the output voltage as the input frequency is increased, assuming the input voltage amplitude does not change.  Can you think of any practical applications for a circuit such as this?

\underbar{file 02129}
%(END_QUESTION)





%(BEGIN_ANSWER)

The DC output voltage will increase as the input signal frequency is increased.  This lends itself to frequency measurement applications.

%(END_ANSWER)





%(BEGIN_NOTES)

Do not accept an answer from students along the lines of "frequency measurement."  Ask them to provide some {\it practical} examples of systems where frequency measurement is important.  If they have difficulty thinking of anything practical, suggest that the input (square wave) signal might come from a sensor detecting shaft rotation (one pulse per revolution), then ask them to think of possible applications for a circuit such as this.

%INDEX% Counter circuit, passive RC

%(END_NOTES)


