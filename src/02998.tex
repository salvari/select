
%(BEGIN_QUESTION)
% Copyright 2005, Tony R. Kuphaldt, released under the Creative Commons Attribution License (v 1.0)
% This means you may do almost anything with this work of mine, so long as you give me proper credit

An early form of digital, serial data communication was {\it Morse code}.  Explain what "Morse code" is (or was), and how it compares to more modern codes such as ASCII.

\underbar{file 02998}
%(END_QUESTION)





%(BEGIN_ANSWER)

Morse code was a simple convention used to represent alphanumeric characters for telegraph data transmission.  At first, human operators served the task of parallel-to-serial-to-parallel data converters, but then machines were built to do this automatically.

%(END_ANSWER)





%(BEGIN_NOTES)

An interesting feature of Morse code likely not recognized by your students is inherent compression.  Because some Morse characters are shorter than others (fewer pulses) as opposed to ASCII where all characters are the same length, messages sent in Morse tend to require fewer bits than messages sent in ASCII.

%INDEX% Morse code

%(END_NOTES)


