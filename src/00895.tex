
%(BEGIN_QUESTION)
% Copyright 2003, Tony R. Kuphaldt, released under the Creative Commons Attribution License (v 1.0)
% This means you may do almost anything with this work of mine, so long as you give me proper credit

Based on what you know about bipolar junction transistors, what will the collector current do (increase, decrease, or remain the same) if the variable voltage source increases in voltage?  The small, fixed voltage source (0.7 volts) is just enough to make the transistor conduct, but not enough to fully saturate it.

$$\epsfbox{00895x01.eps}$$

From the variable voltage source's perspective, what does the transistor circuit "look" like?  It certainly does not look resistive, because a resistive circuit would increase current linearly with increases in applied voltage!  If you could relate the behavior of this circuit to a common idealized electrical component, what would it be?

\underbar{file 00895}
%(END_QUESTION)





%(BEGIN_ANSWER)

The collector current will remain (approximately) the same as the variable voltage source increases in magnitude.  In this manner, the transistor circuit "looks" like a {\it current source}.

%(END_ANSWER)





%(BEGIN_NOTES)

This question is really nothing more than review of a transistor's characteristic curves.  You might want to ask your students to relate this circuit's behavior to the common characteristic curves shown in textbooks for bipolar junction transistors.  What portion of the characteristic curve is this transistor operating in while it regulates current?

%INDEX% BJT, as current regulator

%(END_NOTES)


