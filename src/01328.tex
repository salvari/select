
%(BEGIN_QUESTION)
% Copyright 2003, Tony R. Kuphaldt, released under the Creative Commons Attribution License (v 1.0)
% This means you may do almost anything with this work of mine, so long as you give me proper credit

Explain what the following statement means, with regard to the design of electronic circuits:

\vskip 10pt {\narrower \noindent \baselineskip5pt
\hskip 20pt {\it Faults are inevitable, but failure is not.}
\par} \vskip 10pt

Specifically, what does this philosophy mean for your career as an electronics professional, entrusted with the installation, maintenance, and possibly design of complex systems?

\underbar{file 01328}
%(END_QUESTION)





%(BEGIN_ANSWER)

Components can and will fail, but the system as a whole does not have to be so fragile as to fail with a single component fault.

%(END_ANSWER)





%(BEGIN_NOTES)

Discuss with your students the ramifications of the "fault, but no failure" philosophy with regard to their daily jobs.  How do their actions impact the reliability of systems, and what can they do to minimize the chances of system failures?

%(END_NOTES)


