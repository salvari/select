
%(BEGIN_QUESTION)
% Copyright 2003, Tony R. Kuphaldt, released under the Creative Commons Attribution License (v 1.0)
% This means you may do almost anything with this work of mine, so long as you give me proper credit

Calculate the approximate input impedance of this JFET amplifier circuit:

$$\epsfbox{01177x01.eps}$$

Explain why it is easier to calculate the $Z_{in}$ of a JFET circuit like this than it is to calculate the $Z_{in}$ of a similar bipolar transistor amplifier circuit.  Also, explain how calculation of this amplifier's {\it output} impedance compares with that of a similar BJT amplifier circuit -- same approach or different approach?

\underbar{file 01177}
%(END_QUESTION)





%(BEGIN_ANSWER)

$Z_{in} =$ 89.2 k$\Omega$

%(END_ANSWER)





%(BEGIN_NOTES)

Ask your students to explain why input impedance is an important factor in amplifier design.  Why should we care how much input impedance an amplifier has?

Also, ask your students to explain why such high-value bias resistors (150 k$\Omega$ and 220 k$\Omega$) would probably not be practical in a BJT amplifier circuit.

%INDEX% Impedance, JFET amplifier input
%INDEX% Input impedance, JFET amplifier

%(END_NOTES)


