
%(BEGIN_QUESTION)
% Copyright 2006, Tony R. Kuphaldt, released under the Creative Commons Attribution License (v 1.0)
% This means you may do almost anything with this work of mine, so long as you give me proper credit

In some applications where transistors must amplify very high currents, bipolar transistors are paralleled together so that their current ratings add.  When this is done, it is a good idea to use {\it swamping resistors} at the transistor emitter connections to help ensure even balancing of currents:

$$\epsfbox{03900x01.eps}$$

However, if we use MOSFETs instead of BJTs, we do not have to use swamping resistors:

$$\epsfbox{03900x02.eps}$$

Explain why MOSFETs do not require swamping resistors to help evenly distribute current, while BJTs do.

\underbar{file 03900}
%(END_QUESTION)





%(BEGIN_ANSWER)

The amount of controlling voltage varies with temperature for the BJT, but not for the MOSFET.

%(END_ANSWER)





%(BEGIN_NOTES)

The answer given here is purposefully vague.  Let your students do the necessary research!  Tell them that manufacturers' application notes are valuable sources of information for questions such as this.

%INDEX% BJTs, paralleling for extra current
%INDEX% MOSFETs, paralleling for extra current
%INDEX% Transistors, paralleling for extra current

%(END_NOTES)


