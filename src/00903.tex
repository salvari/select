
%(BEGIN_QUESTION)
% Copyright 2003, Tony R. Kuphaldt, released under the Creative Commons Attribution License (v 1.0)
% This means you may do almost anything with this work of mine, so long as you give me proper credit

If a pure ("intrinsic") semiconductor material is heated, the thermal energy liberates some valence-band electrons into the conduction band.  The vacancies left behind in the valence band are called {\it holes}:

$$\epsfbox{00903x01.eps}$$

If an electrical voltage is applied across the heated semiconducting substance, with positive on the left and negative on the right, what will this do to the energy bands, and how will this affect both the electrons and the holes?

$$\epsfbox{00903x02.eps}$$

\underbar{file 00903}
%(END_QUESTION)





%(BEGIN_ANSWER)

$$\epsfbox{00903x03.eps}$$

The presence of an electric field across the length of the material will cause the bands to slope, electrons moving toward the positive side and holes toward the negative.

%(END_ANSWER)





%(BEGIN_NOTES)

Holes are difficult concepts to grasp for some students.  An analogy I find helpful for explaining how the {\it absence} of an electron may be though of as a particle is to refer to bubbles of air in water.  When viewing bubbles of air in a clear, water-filled tube, it sure seems as though the bubbles are {\it discrete particles}, even though we know them to actually be {\it voids} where there is no water.  And no one balks at the idea of assigning direction and speed to bubbles, even though they are really nothing rather than something!

The principle of energy bands sloping due to the presence of an electric field is vitally important for students to understand if they are to grasp the operation of a PN junction.  An analogy that helps to visualize the electron and hole motion is to think of the two bands (conduction and valence) as two different pipes that can carry water.  The upper pipe (the conduction band) is mostly empty, with only droplets of water running downhill.  The bottom pipe (the valence band) is mostly full of water, with air bubbles running uphill.

One major point I wish to communicate here is that "hole flow" is not just a mirror-image of electron conduction.  "Hole flow" is a fundamentally different mechanism of electron motion.  Electrons are the only true charge carriers in any solid material, but "holes" are commonly referred to as "carriers" because they represent an easy-to-follow marker of valence electron motion.  By referring to "holes" as entities unto themselves, it better distinguishes the two forms of electron motion (conduction-band versus valence-band).

Something you might want to point out to students, if they haven't already discovered it through their own research, is that there is no such thing as "hole flow" in metals.  In metals, 100\% of the conduction occurs through conduction-band electrons.  This phenomenon of dual-mode electron flow only occurs when there is a band gap separating the valence and conduction bands.  This is interesting to note, because many texts (even some high-level engineering textbooks!) refer to "conventional flow" current notation as "hole flow," even when the current exists in metal wires.

%INDEX% Electron bands
%INDEX% Electrons versus holes
%INDEX% Holes versus electrons
%INDEX% Diagram, electron band
%INDEX% Energy diagram, semiconductor with external voltage applied

%(END_NOTES)


