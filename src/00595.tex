
%(BEGIN_QUESTION)
% Copyright 2003, Tony R. Kuphaldt, released under the Creative Commons Attribution License (v 1.0)
% This means you may do almost anything with this work of mine, so long as you give me proper credit

Both soldering {\it irons} and soldering {\it guns} are tools used in the process of electrical soldering.  Describe what each of these tools looks like, and how they function.

\underbar{file 00595}
%(END_QUESTION)





%(BEGIN_ANSWER)

A modern soldering "iron" is a pencil-shaped device with an electrical heating element in its end.  Soldering "guns" have a pistol-shaped body, with a loop of thick copper wire at the end acting directly as a heating element.  Soldering irons have no on/off switch, while soldering guns do, which should tell you something about the speed at which they heat.

%(END_ANSWER)





%(BEGIN_NOTES)

Antique soldering irons were really nothing more than iron wedges with wooden handles, which were set directly in a fire for heat, much like clothes irons used to be nothing more than an anvil-shaped mass of iron that was set on top of a hot stove for heat.  I've used one of these old soldering irons to solder sheet metal pieces together, and the technique of use was exactly the same as it is for a modern (electric) soldering iron.

If you have soldering irons and guns available in your classroom, show them to your students during discussion.

%INDEX% Soldering gun, defined
%INDEX% Soldering iron, defined

%(END_NOTES)


