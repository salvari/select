
%(BEGIN_QUESTION)
% Copyright 2004, Tony R. Kuphaldt, released under the Creative Commons Attribution License (v 1.0)
% This means you may do almost anything with this work of mine, so long as you give me proper credit

Power calculation in DC circuits is simple.  There are three formulae that may be used to calculate power:

$$P = IV \hbox{\hskip 30pt} P = I^2 R \hbox{\hskip 30pt} P = {V^2 \over R} \hbox{\hskip 30pt Power in DC circuits}$$

Calculating power in AC circuits is much more complex, because there are three different types of power: {\it apparent} power ($S$), {\it true} power ($P$), and {\it reactive} power ($Q$).  Write equations for calculating each of these types of power in an AC circuit:

\vskip 40pt

\underbar{file 02181}
%(END_QUESTION)





%(BEGIN_ANSWER)

$$S = IV \hbox{\hskip 30pt} S = I^2 Z \hbox{\hskip 30pt} S = {V^2 \over Z} \hbox{\hskip 30pt Apparent power in AC circuits}$$

$$P = IV \cos \Theta \hbox{\hskip 30pt} P = I^2 R \hbox{\hskip 30pt} P = {V^2 \over R} \hbox{\hskip 30pt True power in AC circuits}$$

$$Q = IV \sin \Theta \hbox{\hskip 30pt} Q = I^2 X \hbox{\hskip 30pt} Q = {V^2 \over X} \hbox{\hskip 30pt Reactive power in AC circuits}$$

\vskip 10pt

Follow-up question \#1: algebraically manipulate each of the following equations to solve for all the other variables in them:

$$Q = I^2 X \hbox{\hskip 30pt} S = {V^2 \over Z} \hbox{\hskip 30pt} P = I^2 R$$

\vskip 10pt

Follow-up question \#2: substitute $\pi$, $f$, and either $L$ or $C$ into the reactive power equations so that one may calculate $Q$ without having to directly know the value of $X$.

%(END_ANSWER)





%(BEGIN_NOTES)

Nothing much to comment on here, as these equations may be found in any number of texts.  One thing you might consider doing to encourage participation from your students is to ask three of them to write these equations on the board in front of class, one student per power type ($S$, $P$, and $Q$).  This would be an ideal question for your more timid students, because there is little explanation involved and therefore little chance of embarrassment.

%INDEX% Power, apparent versus true versus reactive
%INDEX% Power formulae, apparent versus true versus reactive

%(END_NOTES)


