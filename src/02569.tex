
%(BEGIN_QUESTION)
% Copyright 2005, Tony R. Kuphaldt, released under the Creative Commons Attribution License (v 1.0)
% This means you may do almost anything with this work of mine, so long as you give me proper credit

$$\epsfbox{02569x01.eps}$$

\underbar{file 02569}
\vfil \eject
%(END_QUESTION)





%(BEGIN_ANSWER)

Use circuit simulation software to verify your predicted and measured parameter values.

%(END_ANSWER)





%(BEGIN_NOTES)

Use a dual-voltage, regulated power supply to supply power to the opamp.  Specify standard resistor values, all between 1 k$\Omega$ and 100 k$\Omega$ (1k5, 2k2, 2k7, 3k3, 4k7, 5k1, 6k8, 10k, 22k, 33k, 39k 47k, 68k, etc.).

I have had good success using the following values:

\medskip
\item{$\bullet$} +V = +12 volts
\item{$\bullet$} -V = -12 volts
\item{$\bullet$} $R_1$ = 10 k$\Omega$
\item{$\bullet$} $R_2$ = 10 k$\Omega$
\item{$\bullet$} $R_3$ = 10 k$\Omega$
\item{$\bullet$} $R_4$ = 10 k$\Omega$
\item{$\bullet$} $R_5$ = 100 k$\Omega$
\item{$\bullet$} $C_1$ = 0.1 $\mu$F
\item{$\bullet$} $C_2$ = 0.47 $\mu$F
\item{$\bullet$} $U_1$ = one-half of LM1458 dual operational amplifier
\item{$\bullet$} $U_2$ = other half of LM1458 dual operational amplifier
\medskip

It is a good idea to choose capacitor $C_2$ as a larger value than capacitor $C_1$, so that the second opamp does not saturate.

An extension of this exercise is to incorporate troubleshooting questions.  Whether using this exercise as a performance assessment or simply as a concept-building lab, you might want to follow up your students' results by asking them to predict the consequences of certain circuit faults.

%INDEX% Assessment, performance-based (Opamp triangle wave oscillator)

%(END_NOTES)


