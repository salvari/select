
%(BEGIN_QUESTION)
% Copyright 2006, Tony R. Kuphaldt, released under the Creative Commons Attribution License (v 1.0)
% This means you may do almost anything with this work of mine, so long as you give me proper credit

One of the idiosyncrasies of analog-to-digital conversion is a phenomenon known as {\it aliasing}.  It happens when an ADC attempts to digitize a waveform with too high of a frequency.  

Explain what aliasing is, how it happens, and what may be done to prevent it from happening to an ADC circuit.

\underbar{file 04040}
%(END_QUESTION)





%(BEGIN_ANSWER)

As the saying goes, a picture is worth a thousand words:

$$\epsfbox{04040x01.eps}$$

%(END_ANSWER)





%(BEGIN_NOTES)

The point of this question (and of the answer given) is to have students put this important concept into their own words.

Something noteworthy for students and instructors alike is that aliasing may be visually experienced using digital oscilloscopes.  Setting the timebase (seconds/division) control too slow may result in a false (aliased) waveform displayed in the oscilloscope.  Not only does this make a good classroom demonstration, but it also is a great lesson to learn if one expects to use digital oscilloscopes on a regular basis!

%INDEX% Aliasing, ADC

%(END_NOTES)


