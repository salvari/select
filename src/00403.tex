
%(BEGIN_QUESTION)
% Copyright 2003, Tony R. Kuphaldt, released under the Creative Commons Attribution License (v 1.0)
% This means you may do almost anything with this work of mine, so long as you give me proper credit

Is the deflection of an analog AC meter movement proportional to the peak, average, or RMS value of the waveform measured?  Explain your answer.

\underbar{file 00403}
%(END_QUESTION)





%(BEGIN_ANSWER)

Analog meter deflection is proportional to the {\it average} value of the AC waveform measured, for most AC meter movement types.  There are some meter movement designs, however, that give indications proportional to the RMS value of the waveform: {\it hot-wire} and {\it electrodynamometer} movements are of this nature.

\vskip 10pt

Follow-up question: does this mean an average-responding meter movement cannot be calibrated to indicate in RMS units?

\vskip 10pt

Challenge question: why do hot-wire and electrodynamometer meter movements provide true RMS indications, while most other movement designs indicate based on the signal's average value?

%(END_ANSWER)





%(BEGIN_NOTES)

Students often confuse the terms "average" and "RMS", thinking they are interchangeable.  Discuss the difference between these two terms, both mathematically and practically.  While the concepts may seem similar at first, the details are actually quite different.

The question of whether an average-responding instrument can be calibrated to register in RMS units is very practical, since the vast majority of multimeters are calibrated this way.  Because the proportionality between the average and RMS values of an AC waveform are dependent on the shape of the waveform, a certain wave-shape must be assumed in order to accurately calibrate an average-responding meter movement for RMS measurement.  The assumed wave-shape, of course, is sinusoidal.

%INDEX% Meter response, peak vs. RMS vs. average

%(END_NOTES)


