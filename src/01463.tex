
%(BEGIN_QUESTION)
% Copyright 2003, Tony R. Kuphaldt, released under the Creative Commons Attribution License (v 1.0)
% This means you may do almost anything with this work of mine, so long as you give me proper credit

Identify the polarity of voltage across the load resistor in the following switched capacitor circuit (called a {\it transresistor} circuit).  Note: $\phi_1$ and $\phi_2$ are two-phase, non-overlapping clock signals, and the switches are just generic representations of transistors.

$$\epsfbox{01463x01.eps}$$

Identify the polarity of voltage across the load resistor in the following switched capacitor circuit.  Note: the only difference between this circuit and the last is the switching sequence.

$$\epsfbox{01463x02.eps}$$

What difference would it make to the output signal of this operational amplifier circuit if the switching sequence of the switched capacitor network were changed?  What difference would it make if the switching {\it frequency} were changed?

$$\epsfbox{01463x04.eps}$$

\underbar{file 01463}
%(END_QUESTION)





%(BEGIN_ANSWER)

$$\epsfbox{01463x03.eps}$$

The op-amp circuit will act as an inverting or non-inverting amplifier, depending on the switching sequence.  Gain will be affected by switching frequency.

\vskip 10pt

Challenge question: the second switched capacitor network is often referred to as a {\it negative resistor} equivalent.  Explain why.

%(END_ANSWER)





%(BEGIN_NOTES)

Some of the versatility of switched capacitor networks can be seen in these two circuit examples.  Really, they're the same circuit, just operated differently.  Discuss with your students how this versatility may be an advantage in circuit design.

%INDEX% Switched capacitor network, resistor equivalent
%INDEX% Switched capacitor network, transresistor type

%(END_NOTES)


