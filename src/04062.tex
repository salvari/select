
%(BEGIN_QUESTION)
% Copyright 2006, Tony R. Kuphaldt, released under the Creative Commons Attribution License (v 1.0)
% This means you may do almost anything with this work of mine, so long as you give me proper credit

A fascinating mathematical identity discovered by Leonhard Euler (1707-1783), regarded by some as the most beautiful equation in all of mathematics, relates five of mathematics' fundamental constants together:

$$e^{i \pi} + 1 = 0$$

Use {\it Euler's relation} to translate this identity into trigonometric terms, where the truth of this identity will become more evident.

\underbar{file 04062}
%(END_QUESTION)





%(BEGIN_ANSWER)

$$\cos \pi + j \sin \pi + 1 = 0$$

%(END_ANSWER)





%(BEGIN_NOTES)

The beauty of this identity cannot be denied, relating five fundamental constants of mathematics ($e$, $i$, $\pi$, 1, and 0) together in one simple equation.

Note: here I break stylistic convention by using the more traditionally mathematical $i$ instead of the traditionally electrical $j$ to represent $\sqrt{-1}$.  For those who just can't stand to see $i$ represent anything other than instantaneous current, here you go:

$$e^{j \pi} + 1 = 0$$

Are you happy now?

%INDEX% Euler's identity

%(END_NOTES)


