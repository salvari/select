
%(BEGIN_QUESTION)
% Copyright 2003, Tony R. Kuphaldt, released under the Creative Commons Attribution License (v 1.0)
% This means you may do almost anything with this work of mine, so long as you give me proper credit

Describe what "electricity" is, in your own words.

\underbar{file 00114}
%(END_QUESTION)





%(BEGIN_ANSWER)

If you're having difficulty formulating a definition for "electricity," a simple definition of "electric current" will suffice.  What I'm looking for here is a description of how an electric current may exist within a solid material such as a metal wire.

%(END_ANSWER)





%(BEGIN_NOTES)

This question is not as easy to answer as it may first appear.  Certainly, electric current is defined as the "flow" of electrons, but how do electrons "flow" through a solid material such as copper?  How does {\it anything} flow through a solid material, for that matter?

Many scientific disciplines challenge our "common sense" ideas of reality, including the seemingly solid nature of certain substances.  One of the liberating aspects of scientific investigation is that it frees us from the limitations of direct sense perception.  Through structured experimentation and rigorous thinking, we are able to "see" things that might otherwise be impossible to see.  We certainly cannot see electrons with our eyes, but we can detect their presence with special equipment, measure their motion by inference from other effects, and prove empirically that they do in fact exist.

In this regard, scientific method is a tool for the expansion of human ability.  Your students will begin to experience the thrill of "working with the invisible" as they explore electricity and electric circuits.  It is your task as an instructor to foster and encourage this sense of wonder in your students' work.

%INDEX% Electricity, defined

%(END_NOTES)


