
%(BEGIN_QUESTION)
% Copyright 2003, Tony R. Kuphaldt, released under the Creative Commons Attribution License (v 1.0)
% This means you may do almost anything with this work of mine, so long as you give me proper credit

The {\it Superposition Theorem} is a very important concept used to analyze both DC and AC circuits.  Define this theorem in your own words, and also state the necessary conditions for it to be freely applied to a circuit.

\underbar{file 02036}
%(END_QUESTION)





%(BEGIN_ANSWER)

There are plenty of textbook references to the Superposition Theorem and where it may be applied.  I'll let you do your own research here!

%(END_ANSWER)





%(BEGIN_NOTES)

As the answer states, students have a multitude of resources to consult on this topic.  It should not be difficult for them to ascertain what this important theorem is and how it is applied to the analysis of circuits.

Be sure students understand what the terms {\it linear} and {\it bilateral} mean with reference to circuit components and the necessary conditions for Superposition Theorem to be applied to a circuit.  Point out that it is still possible to apply the Superposition Theorem to a circuit containing nonlinear or unilateral components if we do so carefully (i.e. under narrowly defined conditions).

%INDEX% Superposition theorem, definition

%(END_NOTES)


