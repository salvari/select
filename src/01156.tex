
%(BEGIN_QUESTION)
% Copyright 2003, Tony R. Kuphaldt, released under the Creative Commons Attribution License (v 1.0)
% This means you may do almost anything with this work of mine, so long as you give me proper credit

Find a magnet and bring it with you to class for discussion.  Identify as much information as you can about your magnet prior to discussion:

\medskip
\item{$\bullet$} Location of poles
\item{$\bullet$} Which pole is North, and which pole is South
\item{$\bullet$} Type (metal, ceramic, etc.)
\medskip

Please be careful to keep any magnets away from cassette tapes, computer disks, credit cards (with magnetic information strips) or any other form of magnetic media!

\underbar{file 01156}
%(END_QUESTION)





%(BEGIN_ANSWER)

If possible, find a manufacturer's datasheet for your component (or at least a datasheet for a similar component) to discuss with your classmates.

Be prepared to explain to your classmates how you were able to determine the polarity (North and South) of your magnet!

%(END_ANSWER)





%(BEGIN_NOTES)

The purpose of this question is to get students to kinesthetically interact with the subject matter.  It may seem silly to have students engage in a "show and tell" exercise, but I have found that activities such as this greatly help some students.  For those learners who are kinesthetic in nature, it is a great help to actually {\it touch} real components while they're learning about their function.

Another good learning experience with this question is how your students determined the polarity of their magnet.

%INDEX% Magnetic south
%INDEX% Magnetic north

%(END_NOTES)


