
%(BEGIN_QUESTION)
% Copyright 2004, Tony R. Kuphaldt, released under the Creative Commons Attribution License (v 1.0)
% This means you may do almost anything with this work of mine, so long as you give me proper credit

Determine the current through the series LR branch in this series-parallel circuit:

$$\epsfbox{02116x01.eps}$$

Hint: convert the series LR sub-network into a parallel equivalent first.

\underbar{file 02116}
%(END_QUESTION)





%(BEGIN_ANSWER)

$I_{LR}$ = 3.290 mA

%(END_ANSWER)





%(BEGIN_NOTES)

Yes, that is an AC current source shown in the schematic!  In circuit analysis, it is quite common to have AC current sources representing idealized portions of an actual component.  For instance {\it current transformers} (CT's) act very close to ideal AC current sources.  Transistors in amplifier circuits also act as AC current sources, and are often represented as such for the sake of analyzing amplifier circuits.

Although there are other ways to calculate this voltage drop, it is good for students to learn the method of series-parallel subcircuit equivalents.  If for no other reason, this method has the benefit of requiring less tricky math (no complex numbers needed!).

Have your students explain the procedures they used to find the answer, so that all may benefit from seeing multiple methods of solution and multiple ways of explaining it.

%INDEX% AC current source
%INDEX% Converting series impedances to parallel impedances, and visa-versa
%INDEX% Current source, AC
%INDEX% Equivalent networks, series and parallel impedances

%(END_NOTES)


