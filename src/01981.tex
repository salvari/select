
%(BEGIN_QUESTION)
% Copyright 2005, Tony R. Kuphaldt, released under the Creative Commons Attribution License (v 1.0)
% This means you may do almost anything with this work of mine, so long as you give me proper credit

$$\epsfbox{01981x01.eps}$$

\underbar{file 01981}
\vfil \eject
%(END_QUESTION)





%(BEGIN_ANSWER)

Use circuit simulation software to verify your predicted and measured parameter values.

%(END_ANSWER)





%(BEGIN_NOTES)

Use a variable-voltage, regulated power supply to supply any amount of DC voltage below 30 volts.  Specify standard resistor values, all between 1 k$\Omega$ and 100 k$\Omega$ (1k5, 2k2, 2k7, 3k3, 4k7, 5k1, 6k8, 10k, 22k, 33k, 39k 47k, 68k, etc.).  Use a sine-wave function generator to supply an audio-frequency input signal, and make sure its amplitude isn't set so high that the amplifier clips.

The voltage gain of this amplifier configuration tends to be very high, approximately equal to $R_C \over r'_e$.  Your students will have to use fairly low input voltages to achieve class A operation with this amplifier circuit.  I have had good success using the following values:

\medskip
\item{$\bullet$} $V_{CC}$ = 12 volts
\item{$\bullet$} $V_{in}$ = 20 mV peak-to-peak, at 5 kHz
\item{$\bullet$} $R_1$ = 1 k$\Omega$
\item{$\bullet$} $R_2$ = 4.7 k$\Omega$
\item{$\bullet$} $R_C$ = 100 $\Omega$
\item{$\bullet$} $R_E$ = 1 k$\Omega$
\item{$\bullet$} $C_1$ = 33 $\mu$F
\medskip

Your students will find the actual voltage gain deviates somewhat from predicted values with this circuit, largely because it is so dependent on the value of $r'_e$, and that parameter tends to be unpredictable.

An extension of this exercise is to incorporate troubleshooting questions.  Whether using this exercise as a performance assessment or simply as a concept-building lab, you might want to follow up your students' results by asking them to predict the consequences of certain circuit faults.

%INDEX% Assessment, performance-based (Common-base class-A amplifier circuit)

%(END_NOTES)


