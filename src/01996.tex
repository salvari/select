
%(BEGIN_QUESTION)
% Copyright 2003, Tony R. Kuphaldt, released under the Creative Commons Attribution License (v 1.0)
% This means you may do almost anything with this work of mine, so long as you give me proper credit

$$\epsfbox{01996x01.eps}$$

\underbar{file 01996}
\vfil \eject
%(END_QUESTION)





%(BEGIN_ANSWER)

Use circuit simulation software to verify your predicted and measured parameter values.

%(END_ANSWER)





%(BEGIN_NOTES)

Use a variable-voltage, regulated power supply to supply any amount of DC voltage below 30 volts.  Specify standard resistor values, all between 1 k$\Omega$ and 100 k$\Omega$ (1k5, 2k2, 2k7, 3k3, 4k7, 5k1, 6k8, 10k, 22k, 33k, 39k 47k, 68k, etc.).  Use a sine-wave function generator to supply an audio-frequency input signal.

If you lack a spectrum analyzer in your lab, fear not!  There are free software packages in existence allowing you to use the audio input of a personal computer's sound card as a (limited) spectrum analyzer and oscilloscope!  You may find some of these packages by searching on the Internet.  One that I've used (2002) successfully in my own class is called WinScope.

An extension of this exercise is to incorporate troubleshooting questions.  Whether using this exercise as a performance assessment or simply as a concept-building lab, you might want to follow up your students' results by asking them to predict the consequences of certain circuit faults.

%INDEX% Assessment, performance-based (Amplifier distortion testing)

%(END_NOTES)


