
%(BEGIN_QUESTION)
% Copyright 2003, Tony R. Kuphaldt, released under the Creative Commons Attribution License (v 1.0)
% This means you may do almost anything with this work of mine, so long as you give me proper credit

$$\epsfbox{02023x01.eps}$$

Note: when testing the frequency response of the tone control circuit, you may need to replace the transformer/speaker assembly with a non-inductive resistor of equivalent impedance, and measure $V_{out}$ across it. 

\underbar{file 02023}
\vfil \eject
%(END_QUESTION)





%(BEGIN_ANSWER)

Use circuit simulation software to verify your predicted and measured parameter values.

%(END_ANSWER)





%(BEGIN_NOTES)

A good source of audio signal is the headphone output jack of almost any radio, media player, or other portable audio device.  Students like being able to do a lab exercise that directly relates to technology they're already familiar with.

I have experienced good success with the following component values:

\medskip
\item{$\bullet$} $C_1$ = 0.1 $\mu$F 
\item{$\bullet$} $L_1$ = 200 mH (actually two 100 mH inductors in series)
\item{$\bullet$} $R_1$ = $R_2$ = 1 k$\Omega$
\item{$\bullet$} $T_1$ = 1000:8 ohm audio output transformer
\item{$\bullet$} $R_{pot}$ = $R_{pot2}$ = 10 k$\Omega$
\item{$\bullet$} Speaker = small 8 $\Omega$ unit (salvaged from an old clock radio or other inexpensive audio device)
\medskip

An extension of this exercise is to incorporate troubleshooting questions.  Whether using this exercise as a performance assessment or simply as a concept-building lab, you might want to follow up your students' results by asking them to predict the consequences of certain circuit faults.

%INDEX% Assessment, performance-based (Tone balance control circuit)

%(END_NOTES)


