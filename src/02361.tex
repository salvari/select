
%(BEGIN_QUESTION)
% Copyright 2005, Tony R. Kuphaldt, released under the Creative Commons Attribution License (v 1.0)
% This means you may do almost anything with this work of mine, so long as you give me proper credit

The "substrate" connection in a MOSFET is often internally connected to the source, like this:

$$\epsfbox{02361x01.eps}$$

This turns the MOSFET from a four-terminal device into a three-terminal device, making it easier to use.  One consequence of this internal connection, though, is the creation of a {\it parasitic diode} between the source and drain terminals: a PN junction that exists whether we want it to or not.

Add this parasitic diode to the MOSFET symbol shown here (representing the MOSFET cross-section shown above), and explain how its presence affects the transistor's use in a real circuit:

$$\epsfbox{02361x02.eps}$$

\underbar{file 02361}
%(END_QUESTION)





%(BEGIN_ANSWER)

$$\epsfbox{02361x03.eps}$$

\vskip 10pt

Follow-up question: how does the presence of this parasitic diode allow us to positively distinguish the source terminal from the gate terminal when identifying the terminals of a MOSFET with a multimeter?

%(END_ANSWER)





%(BEGIN_NOTES)

The presence of this diode is a very important concept for students to grasp, as it makes the MOSFET a unilateral device for most practical purposes.  Discuss the significance of this diode, and contrast the characteristics of a three-terminal MOSFET against the characteristics of a three-terminal JFET, which is a truly bilateral device.

%INDEX% MOSFET, parasitic diode
%INDEX% Parasitic diode, MOSFET

%(END_NOTES)


