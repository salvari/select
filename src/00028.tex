
%(BEGIN_QUESTION)
% Copyright 2003, Tony R. Kuphaldt, released under the Creative Commons Attribution License (v 1.0)
% This means you may do almost anything with this work of mine, so long as you give me proper credit

What is the difference between {\it DC} and {\it AC} electricity?  Identify some common sources of each type of electricity.

\underbar{file 00028}
%(END_QUESTION)





%(BEGIN_ANSWER)

{\it DC} is an acronym meaning {\it Direct Current}: that is, electrical current that moves in one direction only.  {\it AC} is an acronym meaning {\it Alternating Current}: that is, electrical current that periodically reverses direction ("alternates").

Electrochemical batteries generate DC, as do solar cells.  Microphones generate AC when sensing sound waves (vibrations of air molecules).  There are many, many other sources of DC and AC electricity than what I have mentioned here!

%(END_ANSWER)





%(BEGIN_NOTES)

Discuss a bit of the history of AC versus DC in early power systems.  In the early days of electric power in the United States of America, there was a heated debate between the use of DC versus AC.  Thomas Edison championed DC, while George Westinghouse and Nikola Tesla advocated AC.

It might be worthwhile to mention that almost all the electric power in the world is generated and distributed as AC (Alternating Current), and not as DC (in other words, Thomas Edison lost the AC/DC battle!).  Depending on the level of the class you are teaching, this may or may not be a good time to explain {\it why} most power systems use AC.  Either way, your students will probably ask why, so you should be prepared to address this question in some way (or have them report any findings of their own!).

%INDEX% DC versus AC
%INDEX% AC versus DC
%INDEX% Sources of AC and DC electricity

%(END_NOTES)


