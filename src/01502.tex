
%(BEGIN_QUESTION)
% Copyright 2003, Tony R. Kuphaldt, released under the Creative Commons Attribution License (v 1.0)
% This means you may do almost anything with this work of mine, so long as you give me proper credit

The armature windings of your DC motor will have very low resistance.  Too low, in fact, to accurately measure with an ohmmeter connected directly to the commutator terminals, because the contact resistance between the meter probes and terminals will be a large percentage of the resistance read by the meter.  

One way to avoid this contact resistance problem and measure very low resistances using normal meters is to employ what is known as the {\it Kelvin resistance measurement technique}: measuring the voltage dropped by the low-resistance element when a substantial DC current is passed through it.

Draw a diagram showing how this measurement technique works, and explain {\it why} it works.

\underbar{file 01502}
%(END_QUESTION)





%(BEGIN_ANSWER)

$$\epsfbox{01502x01.eps}$$

$R$ is calculated by way of Ohm's Law.

%(END_ANSWER)





%(BEGIN_NOTES)

Kelvin resistance measurement is a very common technique used in electrical metrology, because it neatly sidesteps the problem of contact resistance in the path of high current.  It is a technique well worth learning for the serious student of electronics.

%(END_NOTES)


