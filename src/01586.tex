
%(BEGIN_QUESTION)
% Copyright 2003, Tony R. Kuphaldt, released under the Creative Commons Attribution License (v 1.0)
% This means you may do almost anything with this work of mine, so long as you give me proper credit

The three-stage amplifier shown here has a problem.  Despite being supplied with good, "clean" DC power and an adequate input signal to amplify, there is no output signal whatsoever:

$$\epsfbox{01586x01.eps}$$

Explain how you would use the "divide and conquer" or "divide by two" strategy of troubleshooting to locate the amplification stage where the fault is.  (This is where you divide the signal path into different sections, then test for good signal at points along that path so as to narrow the problem down to one-half of the circuit, then to one-quarter of the circuit, etc.)

Show the lines of demarcation where you would divide the circuit into distinct sections, and identify input and output test points for each of those sections.

\underbar{file 01586}
%(END_QUESTION)





%(BEGIN_ANSWER)

$$\epsfbox{01586x02.eps}$$

\vskip 10pt

Challenge question: how well do you suppose this same troubleshooting strategy would work to locate the fault {\it within} a particular amplification stage?

%(END_ANSWER)





%(BEGIN_NOTES)

Multi-stage amplifier circuits lend themselves well to the "divide and conquer" strategy of troubleshooting, especially when the stages are as symmetrical as these.

%INDEX% Troubleshooting, multi-stage transistor amplifier circuit
%INDEX% Troubleshooting strategy, "divide and conquer"

%(END_NOTES)


