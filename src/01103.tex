
%(BEGIN_QUESTION)
% Copyright 2003, Tony R. Kuphaldt, released under the Creative Commons Attribution License (v 1.0)
% This means you may do almost anything with this work of mine, so long as you give me proper credit

The schematic diagram shown here is for a {\it "boost" converter circuit}, a type of DC-DC "switching" power conversion circuit:

$$\epsfbox{01103x01.eps}$$

In this circuit, the transistor is either fully on or fully off; that is, driven between the extremes of saturation or cutoff.  By avoiding the transistor's "active" mode (where it would drop substantial voltage while conducting current), very low transistor power dissipations can be achieved.  With little power wasted in the form of heat, "switching" power conversion circuits are typically very efficient.

Trace all current directions during both states of the transistor.  Also, mark the inductor's voltage polarity during both states of the transistor.

\underbar{file 01103}
%(END_QUESTION)





%(BEGIN_ANSWER)

$$\epsfbox{01103x02.eps}$$

\vskip 10pt

Follow-up question: how does the load voltage of this converter relate to the supply (battery) voltage?  Does the load receive more or less voltage than that provided by the battery?

\vskip 10pt

Challenge question: why do you suppose a Schottky diode is used in this circuit, as opposed to a regular (PN) rectifying diode?

%(END_ANSWER)





%(BEGIN_NOTES)

Ask your students why they think this circuit is called a {\it boost} converter.  "Boost" usually refers to something that is aiding something else.  What is being aided in this circuit?

%INDEX% Boost converter circuit

%(END_NOTES)


