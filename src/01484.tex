
%(BEGIN_QUESTION)
% Copyright 2003, Tony R. Kuphaldt, released under the Creative Commons Attribution License (v 1.0)
% This means you may do almost anything with this work of mine, so long as you give me proper credit

A length of copper wire ($\alpha = 0.004041$ at 20$^{o}$ C) has a resistance of 5 ohms at 20 degrees Celsius.  Calculate its resistance if the temperature were to increase to 50 degrees Celsius.

\vskip 10pt

Now, take that calculated resistance, and that new temperature of 50$^{o}$ C, and calculate what the resistance of the wire should go to if it cools back down to 20$^{o}$ C.  Treat this as a separate problem, working through all the calculations, and don't just say "5 ohms" because you know the original conditions!

\underbar{file 01484}
%(END_QUESTION)





%(BEGIN_ANSWER)

$R_{50^o C} = 5.606 \> \Omega$

\vskip 10pt

If you got an answer of $R_{20^o C} = 4.927 \> \Omega$ for the second calculation, you made a common mistake that is not always warned against in textbooks!  Try the math again.  If you got the proper answer of $5 \> \Omega$ upon doing the second calculation, try to figure out why someone may have calculated $4.927 \> \Omega$ taking the temperature from 50$^{o}$ C down to 20$^{o}$ C.

%(END_ANSWER)





%(BEGIN_NOTES)

One thing students need to learn is they can't simply use the resistance-temperature formula as it is normally given if the "reference" (starting) temperature is not the same as the temperature at which $\alpha$ is specified at!

%INDEX% Resistance, relation to conductor temperature

%(END_NOTES)


