
%(BEGIN_QUESTION)
% Copyright 2005, Tony R. Kuphaldt, released under the Creative Commons Attribution License (v 1.0)
% This means you may do almost anything with this work of mine, so long as you give me proper credit

Predict how the operation of this clipper circuit will be affected as a result of the following faults.  Consider each fault independently (i.e. one at a time, no multiple faults):

$$\epsfbox{03727x01.eps}$$

\medskip
\item{$\bullet$} Diode $D_1$ fails open:
\vskip 5pt
\item{$\bullet$} Diode $D_1$ fails shorted:
\vskip 5pt
\item{$\bullet$} Resistor $R_1$ fails open:
\vskip 5pt
\item{$\bullet$} Resistor $R_1$ fails shorted:
\medskip

For each of these conditions, explain {\it why} the resulting effects will occur.

\underbar{file 03727}
%(END_QUESTION)





%(BEGIN_ANSWER)

\medskip
\item{$\bullet$} Diode $D_1$ fails open: {\it No output voltage at all.}
\vskip 5pt
\item{$\bullet$} Diode $D_1$ fails shorted: {\it Full AC signal at output (no clipping at all).}
\vskip 5pt
\item{$\bullet$} Resistor $R_1$ fails open: {\it No change (if diode is indeed ideal), but realistically there may not be much clipping if the receiving circuit has an extremely large input impedance.}
\vskip 5pt
\item{$\bullet$} Resistor $R_1$ fails shorted: {\it No output voltage at all.}
\medskip

%(END_ANSWER)





%(BEGIN_NOTES)

The purpose of this question is to approach the domain of circuit troubleshooting from a perspective of knowing what the fault is, rather than only knowing what the symptoms are.  Although this is not necessarily a realistic perspective, it helps students build the foundational knowledge necessary to diagnose a faulted circuit from empirical data.  Questions such as this should be followed (eventually) by other questions asking students to identify likely faults based on measurements.

%INDEX% Troubleshooting, predicting effects of fault in series diode clipper circuit

%(END_NOTES)


