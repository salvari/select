
%(BEGIN_QUESTION)
% Copyright 2005, Tony R. Kuphaldt, released under the Creative Commons Attribution License (v 1.0)
% This means you may do almost anything with this work of mine, so long as you give me proper credit

In this circuit, three resistors receive the same amount of current (4 amps) from a single source.  Calculate the amount of voltage "dropped" by each resistor, as well as the amount of power dissipated by each resistor:

$$\epsfbox{00090x01.eps}$$

\underbar{file 00090}
%(END_QUESTION)





%(BEGIN_ANSWER)

$E_{1 \Omega} = 4$ volts

$E_{2 \Omega} = 8$ volts

$E_{3 \Omega} = 12$ volts

\vskip 10pt

$P_{1 \Omega} = 16$ watts

$P_{2 \Omega} = 32$ watts

$P_{3 \Omega} = 48$ watts

\vskip 10pt

Follow-up question: Compare the direction of current through all components in this circuit with the polarities of their respective voltage drops.  What do you notice about the relationship between current direction and voltage polarity for the battery, versus for all the resistors?  How does this relate to the identification of these components as either {\it sources} or {\it loads}?

%(END_ANSWER)





%(BEGIN_NOTES)

The answers to this question should not create any surprises, especially when students understand electrical resistance in terms of {\it friction}: resistors with greater resistance (more friction to electron motion) require greater voltage (push) to get the same amount of current through them.  Resistors with greater resistance (friction) will also dissipate more power in the form of heat, given the same amount of current.

Another purpose of this question is to instill in students' minds the concept of components in a simple series circuit all sharing the same amount of current.

Challenge your students to recognize any mathematical patterns in the respective voltage drops and power dissipations.  What can be said, mathematically, about the voltage drop across the 2 $\Omega$ resistor versus the 1 $\Omega$ resistor, for example?

%INDEX% Ohm's Law
%INDEX% Joule's Law
%INDEX% Series circuit

%(END_NOTES)


