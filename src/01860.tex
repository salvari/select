
%(BEGIN_QUESTION)
% Copyright 2003, Tony R. Kuphaldt, released under the Creative Commons Attribution License (v 1.0)
% This means you may do almost anything with this work of mine, so long as you give me proper credit

Engineers often calculate the impedance of pure capacitances and pure inductances in a way that directly provides results in rectangular (complex) form:

$${\bf Z}_L = j \omega L$$

$${\bf Z}_C = -j {1 \over {\omega C}}$$

The bold-faced type (${\bf Z}$ instead of $Z$) signifies the calculated impedance as a complex rather than a scalar quantity.  Given these equations' forms, what is the mathematical definition of $\omega$?  In other words, what combination of variables and constants comprise "$\omega$", and what unit is it properly expressed in?

Also, determine what the equations would look like for calculating the impedance of these series networks:

$$\epsfbox{01860x01.eps}$$

\underbar{file 01860}
%(END_QUESTION)





%(BEGIN_ANSWER)

$\omega = 2 \pi f$ is called the {\it angular velocity} of the circuit, and it is expressed in units of {\it radians per second}.

\vskip 10pt

The impedance equations for the series LR and RC networks are as follows:

$${\bf Z}_{LR} = R + j \omega L$$

$${\bf Z}_{RC} = R - j {1 \over {\omega C}}$$

%(END_ANSWER)





%(BEGIN_NOTES)

Students who have taken trigonometry should recognize the {\it radian} as a unit for measuring angles.  Discuss with your students why multiplying frequency ($f$, cycles per second) by the constant $2 \pi$ results in the unit changing to "radians per second".

Engineers often refer to $\omega$ as the {\it angular velocity} of an AC system.  Discuss why the term "velocity" is appropriate for $\omega$.

%INDEX% Angular velocity, (lower-case omega)

%(END_NOTES)


