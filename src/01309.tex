
%(BEGIN_QUESTION)
% Copyright 2003, Tony R. Kuphaldt, released under the Creative Commons Attribution License (v 1.0)
% This means you may do almost anything with this work of mine, so long as you give me proper credit

Implement the following Boolean expression in the form of a digital logic circuit:

$$\overline{(\overline{AB} + C)B}$$

Form the circuit by making the necessary connections between pins of these integrated circuits on a solderless breadboard:

$$\epsfbox{01309x01.eps}$$

\underbar{file 01309}
%(END_QUESTION)





%(BEGIN_ANSWER)

The circuit shown is not the only possible solution to this problem:

$$\epsfbox{01309x02.eps}$$

%(END_ANSWER)





%(BEGIN_NOTES)

First things first: did students remember to include the power supply connections to each IC?  This is a very common mistake!

In order to successfully develop a solution to this problem, of course, students must research the "pinouts" of each integrated circuit.  If most students simply present the answer shown to them in the worksheet, challenge them during discussion to present alternative solutions.

Also, ask them this question: "should we connect the unused inputs to either ground or $V_{CC}$, or is it permissible to leave the inputs floating?"  Students should not just give an answer to this question, but be able to support their answer(s) with reasoning based on the construction of this type of logic circuit.

%INDEX% Boolean algebra, conversion of expression into gate logic

%(END_NOTES)


