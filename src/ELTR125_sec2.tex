
\centerline{\bf ELTR 125 (Semiconductors 2), section 2} \bigskip 
 
\vskip 10pt

\noindent
{\bf Recommended schedule}

\vskip 5pt

%%%%%%%%%%%%%%%
\hrule \vskip 5pt
\noindent
\underbar{Day 1}

\hskip 10pt Topics: {\it Load lines and amplifier bias calculations}
 
\hskip 10pt Questions: {\it 1 through 15}
 
\hskip 10pt Lab Exercise: {\it Common-drain amplifier circuit (question 61)}
 
\vskip 10pt
%%%%%%%%%%%%%%%
\hrule \vskip 5pt
\noindent
\underbar{Day 2}

\hskip 10pt Topics: {\it FET amplifier configurations}
 
\hskip 10pt Questions: {\it 16 through 25}
 
\hskip 10pt Lab Exercise: {\it Common-source amplifier circuit (question 62)}
 
 
\vskip 10pt
%%%%%%%%%%%%%%%
\hrule \vskip 5pt
\noindent
\underbar{Day 3}

\hskip 10pt Topics: {\it Push-pull amplifier circuits}
 
\hskip 10pt Questions: {\it 26 through 35}
 
\hskip 10pt Lab Exercise: {\it Audio intercom circuit, push-pull output (question 63)}
 
%INSTRUCTOR \hskip 10pt {\bf Socratic Electronics animation: Push-pull transistor amplifier with crossover distortion}

\vskip 10pt
%%%%%%%%%%%%%%%
\hrule \vskip 5pt
\noindent
\underbar{Day 4}

\hskip 10pt Topics: {\it Multi-stage and high-frequency amplifier designs}
 
\hskip 10pt Questions: {\it 36 through 50}
 
\hskip 10pt Lab Exercise: {\it Audio intercom circuit, push-pull output (question 63, continued)}
 
\vskip 10pt
%%%%%%%%%%%%%%%
\hrule \vskip 5pt
\noindent
\underbar{Day 5}

\hskip 10pt Topics: {\it Amplifier troubleshooting}
 
\hskip 10pt Questions: {\it 51 through 60}
 
\hskip 10pt Lab Exercise: {\it Troubleshooting practice (oscillator/amplifier circuit -- question 64)}

\vskip 10pt
%%%%%%%%%%%%%%%
\hrule \vskip 5pt
\noindent
\underbar{Day 6}

\hskip 10pt Exam 2: {\it includes Amplifier circuit performance assessment}
 
\hskip 10pt Lab Exercise: {\it Troubleshooting practice (oscillator/amplifier circuit -- question 64)}
  
\vskip 10pt
%%%%%%%%%%%%%%%
\hrule \vskip 5pt
\noindent
\underbar{General concept practice and challenge problems}

\hskip 10pt Questions: {\it 67 through the end of the worksheet}
 
\vskip 10pt

%%%%%%%%%%%%%%%
\hrule \vskip 5pt
\noindent
\underbar{Impending deadlines}

\hskip 10pt {\bf Troubleshooting assessment (oscillator/amplifier) due at end of ELTR125, Section 3}

\hskip 10pt Question 65: Troubleshooting log
 
\hskip 10pt Question 66: Sample troubleshooting assessment grading criteria
 
\vskip 10pt
%%%%%%%%%%%%%%%










\vfil \eject

\centerline{\bf ELTR 125 (Semiconductors 2), section 2} \bigskip 
 
\vskip 10pt

\noindent
{\bf Skill standards addressed by this course section}

\vskip 5pt

%%%%%%%%%%%%%%%
\hrule \vskip 10pt
\noindent
\underbar{EIA {\it Raising the Standard; Electronics Technician Skills for Today and Tomorrow}, June 1994}

\vskip 5pt

\medskip
\item{\bf D} {\bf Technical Skills -- Discrete Solid-State Devices}
\item{\bf D.12} Understand principles and operations of single stage amplifiers.
\item{\bf D.13} Fabricate and demonstrate single stage amplifiers.
\item{\bf D.14} Troubleshoot and repair single stage amplifiers.
\item{\bf E} {\bf Technical Skills -- Analog Circuits}
\item{\bf E.01} Understand principles and operations of multistage amplifiers.
\item{\bf E.02} Fabricate and demonstrate multistage amplifiers.
\item{\bf E.03} Troubleshoot and repair multistage amplifiers.
\item{\bf E.14} Fabricate and demonstrate audio power amplifiers.
\item{\bf E.15} Troubleshoot and repair audio power amplifiers.
\medskip

\vskip 5pt

\medskip
\item{\bf B} {\bf Basic and Practical Skills -- Communicating on the Job}
\item{\bf B.01} Use effective written and other communication skills.  {\it Met by group discussion and completion of labwork.}
\item{\bf B.03} Employ appropriate skills for gathering and retaining information.  {\it Met by research and preparation prior to group discussion.}
\item{\bf B.04} Interpret written, graphic, and oral instructions.  {\it Met by completion of labwork.}
\item{\bf B.06} Use language appropriate to the situation.  {\it Met by group discussion and in explaining completed labwork.}
\item{\bf B.07} Participate in meetings in a positive and constructive manner.  {\it Met by group discussion.}
\item{\bf B.08} Use job-related terminology.  {\it Met by group discussion and in explaining completed labwork.}
\item{\bf B.10} Document work projects, procedures, tests, and equipment failures.  {\it Met by project construction and/or troubleshooting assessments.}
\item{\bf C} {\bf Basic and Practical Skills -- Solving Problems and Critical Thinking}
\item{\bf C.01} Identify the problem.  {\it Met by research and preparation prior to group discussion.}
\item{\bf C.03} Identify available solutions and their impact including evaluating credibility of information, and locating information.  {\it Met by research and preparation prior to group discussion.}
\item{\bf C.07} Organize personal workloads.  {\it Met by daily labwork, preparatory research, and project management.}
\item{\bf C.08} Participate in brainstorming sessions to generate new ideas and solve problems.  {\it Met by group discussion.}
\item{\bf D} {\bf Basic and Practical Skills -- Reading}
\item{\bf D.01} Read and apply various sources of technical information (e.g. manufacturer literature, codes, and regulations).  {\it Met by research and preparation prior to group discussion.}
\item{\bf E} {\bf Basic and Practical Skills -- Proficiency in Mathematics}
\item{\bf E.01} Determine if a solution is reasonable.
\item{\bf E.02} Demonstrate ability to use a simple electronic calculator.
\item{\bf E.05} Solve problems and [sic] make applications involving integers, fractions, decimals, percentages, and ratios using order of operations.
\item{\bf E.06} Translate written and/or verbal statements into mathematical expressions.
\item{\bf E.09} Read scale on measurement device(s) and make interpolations where appropriate.  {\it Met by oscilloscope usage.}
\item{\bf E.12} Interpret and use tables, charts, maps, and/or graphs.
\item{\bf E.13} Identify patterns, note trends, and/or draw conclusions from tables, charts, maps, and/or graphs.
\item{\bf E.15} Simplify and solve algebraic expressions and formulas.
\item{\bf E.16} Select and use formulas appropriately.
\item{\bf E.17} Understand and use scientific notation.
\medskip

%%%%%%%%%%%%%%%



\vfil \eject

\centerline{\bf ELTR 125 (Semiconductors 2), section 2} \bigskip 
 
\vskip 10pt

\noindent
{\bf Common areas of confusion for students}

\vskip 5pt


\hrule \vskip 5pt

\vskip 10pt

\noindent
{\bf Difficult concept: } {\it Inverting nature of common-emitter amplifier.}

Some students find it quite difficult to grasp why the DC output voltage of a common-emitter amplifier {\it decreases} as the DC input voltage level increases.  Step-by-step DC analysis of the circuit is the only remedy I have found to this conceptual block: getting students to carefully analyze what happens as voltages increase and decrease.

\vskip 10pt

\noindent
{\bf Difficult concept: } {\it Crossover distortion.}

Crossover distortion is always a concern with class B amplifiers, because there is that point where one transistor "hands off" to the other near the zero-crossing point of the waveform.  Ideally, the "hand off" is seamless, with one transistor beginning to conduct just as the other one cuts off, but this is difficult to achieve.  One way to grasp the nature of the problem is to imagine a class-B amplifier with little or no bias trying to amplify a {\it very} small DC input voltage.  Unless the input signal is enough to get one of the transistors conducting, there will be no resulting output!  This is not good, as an amplifier should at least do {\it something} with any input signal, no matter how small.

