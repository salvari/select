
%(BEGIN_QUESTION)
% Copyright 2003, Tony R. Kuphaldt, released under the Creative Commons Attribution License (v 1.0)
% This means you may do almost anything with this work of mine, so long as you give me proper credit

The amount of capacitance between two conductors may be calculated by the following equation:

$$C = {\epsilon A \over d}$$

\noindent
Where,

$C =$ Capacitance in Farads

$\epsilon =$ Permittivity of dielectric (absolute)

$A =$ Conductor area, in square meters

$d =$ Separation distance, in meters

\vskip 10pt

How far away from each other would two metal plates, 2 square meters in area each, have to be in order to create a capacitance of 1 $\mu$F?  Assume that the plates are separated by air.

\underbar{file 00200}
%(END_QUESTION)





%(BEGIN_ANSWER)

If you calculated a distance in the order of 2 million meters ($2 \times 10^6$ meters), you made a common mistake!  The proper answer is 17.71 micro-meters ($17.71 \times 10^{-6}$ meters), or 0.01771 millimeters.

%(END_ANSWER)





%(BEGIN_NOTES)

This problem is first and foremost an algebraic manipulation exercise.  Then, it is merely a matter of solving for $d$ given the proper values.  Finding $\epsilon$ could be difficult, though, and this is by design: I want students to learn the significance of {\it absolute} permittivity!

%INDEX% Capacitance, calculating

%(END_NOTES)


