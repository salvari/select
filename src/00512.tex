
%(BEGIN_QUESTION)
% Copyright 2003, Tony R. Kuphaldt, released under the Creative Commons Attribution License (v 1.0)
% This means you may do almost anything with this work of mine, so long as you give me proper credit

Calculate the resistance of each of these specimens, given their resistance at the reference temperature ($R_r$ @ $T_r$), and their present temperatures ($T$):

\medskip
\item{$\bullet$} Specimen 1: Copper ; $R_r =$ 200 $\Omega$ @ $T_r = 20^o$C ; $T = 45^o$C ; $R_T =$
\item{$\bullet$} Specimen 2: Copper ; $R_r =$ 10 k$\Omega$ @ $T_r = 20^o$C ; $T = 5^o$C ; $R_T =$
\item{$\bullet$} Specimen 3: Aluminum ; $R_r =$ 1,250 $\Omega$ @ $T_r = 20^o$C ; $T = 100^o$C ; $R_T =$
\item{$\bullet$} Specimen 4: Iron ; $R_r =$ 35.4 $\Omega$ @ $T_r = 20^o$C ; $T = -40^o$C ; $R_T =$
\item{$\bullet$} Specimen 5: Nickel ; $R_r =$ 525 $\Omega$ @ $T_r = 20^o$C ; $T = 70^o$C ; $R_T =$
\item{$\bullet$} Specimen 6: Gold ; $R_r =$ 25 k$\Omega$ @ $T_r = 20^o$C ; $T = 65^o$C ; $R_T =$
\item{$\bullet$} Specimen 7: Tungsten ; $R_r =$ 2.2 k$\Omega$ @ $T_r = 20^o$C ; $T = -10^o$C ; $R_T =$
\item{$\bullet$} Specimen 8: Copper ; $R_r =$ 350 $\Omega$ @ $T_r = 10^o$C ; $T = 35^o$C ; $R_T =$
\item{$\bullet$} Specimen 9: Copper ; $R_r =$ 1.5 k$\Omega$ @ $T_r = -25^o$C ; $T = -5^o$C ; $R_T =$
\item{$\bullet$} Specimen 10: Silver ; $R_r =$ 3.5 M$\Omega$ @ $T_r = 45^o$C ; $T = 10^o$C ; $R_T =$
\medskip

\underbar{file 00512}
%(END_QUESTION)





%(BEGIN_ANSWER)

\medskip
\item{$\bullet$} Specimen 1: $R_T =$ 220.2 $\Omega$
\item{$\bullet$} Specimen 2: $R_T =$ 9.394 k$\Omega$
\item{$\bullet$} Specimen 3: $R_T =$ 1.681 k$\Omega$
\item{$\bullet$} Specimen 4: $R_T =$ 23.35 $\Omega$
\item{$\bullet$} Specimen 5: $R_T =$ 679 $\Omega$
\item{$\bullet$} Specimen 6: $R_T =$ 29.18 k$\Omega$
\item{$\bullet$} Specimen 7: $R_T =$ 1.909 k$\Omega$
\item{$\bullet$} Specimen 8: $R_T =$ 386.8 $\Omega$
\item{$\bullet$} Specimen 9: $R_T =$ 1.648 k$\Omega$
\item{$\bullet$} Specimen 10: $R_T =$ 3.073 M$\Omega$
\medskip

%(END_ANSWER)





%(BEGIN_NOTES)

Students may find difficulty obtaining the proper answers for the last three specimens (8, 9, and 10).  The key to performing calculations correctly on these is the {\it assumed temperature} at which the $\alpha$ figure is given for each metal type.  This reference temperature may not be the same as the reference temperature given in the question!

Here are the $\alpha$ values I used in my calculations, {\it all at a reference temperature of 20$^{o}$ Celsius:}

\medskip
\item{$\bullet$} Copper = 0.004041
\item{$\bullet$} Aluminum = 0.004308
\item{$\bullet$} Iron = 0.005671 
\item{$\bullet$} Nickel = 0.005866
\item{$\bullet$} Gold = 0.003715
\item{$\bullet$} Tungsten = 0.004403
\item{$\bullet$} Silver = 0.003819
\medskip

Your students' sources may vary a bit from these figures.

%INDEX% Conductor resistance calculation

%(END_NOTES)


