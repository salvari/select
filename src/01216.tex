
%(BEGIN_QUESTION)
% Copyright 2003, Tony R. Kuphaldt, released under the Creative Commons Attribution License (v 1.0)
% This means you may do almost anything with this work of mine, so long as you give me proper credit

What is the ideal amount of load impedance for this amplifier circuit, so that maximum power will be delivered to it?

$$\epsfbox{01216x01.eps}$$

Suppose we wished to drive an 8 ohm audio speaker with this amplifier circuit.  How could we better match the amplifier's impedance to the speaker's?

\underbar{file 01216}
%(END_QUESTION)





%(BEGIN_ANSWER)

$R_{load} = 3.3$ k$\Omega$

\vskip 10pt

To match this amplifier to an 8 $\Omega$ speaker, we could use a matching transformer, or (better yet) a common-collector final transistor stage.

%(END_ANSWER)





%(BEGIN_NOTES)

Ask your students to explain whether they would connect a matching transformer as a step-up or a step-down to match source and load impedances in this example.  How do we know which way we need to use the transformer?

Challenge your students by asking them how they might calculate the necessary transformer winding ratio for this impedance matching application.  I wouldn't be surprised if many of your students do not remember the impedance ratio relationship to turns ratio back from their education in AC circuit theory.  However, they {\it should} remember how turns ratio relates to voltage and current ratios, and from this they should be able to figure out the impedance transformation ratio of a transformer!

An important skill to have is the ability to reconstruct forgotten information by setting up "thought experiments" and deriving results from known (remembered) principles.  I can't tell you how many times in my professional and academic life that this skill has been helpful to me.

%INDEX% Output impedance, amplifier
%INDEX% Impedance, amplifier output

%(END_NOTES)


