
%(BEGIN_QUESTION)
% Copyright 2003, Tony R. Kuphaldt, released under the Creative Commons Attribution License (v 1.0)
% This means you may do almost anything with this work of mine, so long as you give me proper credit

Note that this circuit is impossible to reduce by regular series-parallel analysis:

$$\epsfbox{01855x01.eps}$$

However, the Superposition Theorem makes it almost trivial to calculate all the voltage drops and currents:

$$\epsfbox{01855x02.eps}$$

Explain the procedure for applying the Superposition Theorem to this circuit.

\underbar{file 01855}
%(END_QUESTION)





%(BEGIN_ANSWER)

This is easy enough for you to research on your own!

%(END_ANSWER)





%(BEGIN_NOTES)

I really enjoy covering the Superposition Theorem in class with my students.  It's one of those rare analysis techniques that is intuitively obvious and yet powerful at the same time.  Because the principle is so easy to learn, I highly recommend you leave this question for your students to research, and let {\it them} fully present the answer in class rather than you explain any of it.

%INDEX% Superposition theorem, quantitative

%(END_NOTES)


