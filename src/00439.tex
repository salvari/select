
%(BEGIN_QUESTION)
% Copyright 2003, Tony R. Kuphaldt, released under the Creative Commons Attribution License (v 1.0)
% This means you may do almost anything with this work of mine, so long as you give me proper credit

When checked with an ohmmeter, how should a properly functioning inductor respond?

\underbar{file 00439}
%(END_QUESTION)





%(BEGIN_ANSWER)

A "healthy" inductor should register as a very low resistance between its terminals.  If the inductor has an iron core, there should be infinite resistance (no continuity) between either winding terminal and the core.

\vskip 10pt

Follow-up question: what do you suppose is the most likely failure "mode" of an inductor, open or shorted?  Explain your answer.

%(END_ANSWER)





%(BEGIN_NOTES)

Have your students actually test a few inductors with their ohmmeters in class.  Unlike capacitor checking with an ohmmeter, there is never a substantial "charging" period!  If your students have already learned about capacitor checking with an ohmmeter, ask them to explain why there is no "charging" action indicated by the ohmmeter when connected to an inductor.

If there were an observable "charging" time exhibited by an inductor when measured by an ohmmeter, what would it appear as, in terms of the ohmmeter's indication?

Normally, I don't give away answers to follow-up questions in the "Notes" section, but here I feel it may be necessary.  Studies have shown that inductors have about an equal chance of failing open as they do failing shorted.  Of course, this will vary with the specific design and application of the inductor, but there is no mode of failure clearly more probable than the other.

%INDEX% Inductor check with ohmmeter
%INDEX% Ohmmeter check of inductor
%INDEX% Inductor, probable failure mode

%(END_NOTES)


