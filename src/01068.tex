
%(BEGIN_QUESTION)
% Copyright 2003, Tony R. Kuphaldt, released under the Creative Commons Attribution License (v 1.0)
% This means you may do almost anything with this work of mine, so long as you give me proper credit

{\it Anti-static wrist straps} are commonly worn by technicians when performing work on circuits containing MOSFETs.  Explain how these straps are used, and how you would test one to ensure it is functioning properly.

\underbar{file 01068}
%(END_QUESTION)





%(BEGIN_ANSWER)

$$\epsfbox{01068x01.eps}$$

A simple ohmmeter test should reveal mega-ohm levels of resistance between the strap's skin contact point and the metal grounding clip.

\vskip 10pt

Follow-up question: why is there resistance intentionally placed between the wrist strap and the grounding clip?  What would be wrong with simply having a 0 $\Omega$ connection between the strap and earth ground (i.e. an uninterrupted length of wire)?

%(END_ANSWER)





%(BEGIN_NOTES)

A good question to ask your students is {\it why} anti-static protection is important when working with MOSFET devices.  You should never assume this is obvious, unless the subject was covered in a question immediately previous to this one!

Your students should have an anti-static wrist strap as part of their regular tool collection.  When discussing this question, it would be good to have students use their ohmmeters to verify the operation of their wrist straps.

%INDEX% Anti-static wrist strap
%INDEX% Electrostatic discharge
%INDEX% ESD (Electro-Static Discharge)
%INDEX% Wrist strap, anti-static

%(END_NOTES)


