
%(BEGIN_QUESTION)
% Copyright 2004, Tony R. Kuphaldt, released under the Creative Commons Attribution License (v 1.0)
% This means you may do almost anything with this work of mine, so long as you give me proper credit

Define what a {\it common-collector} transistor amplifier circuit is.  What distinguishes this amplifier configuration from the other single-BJT amplifier configurations, namely {\it common-emitter} and {\it common-base}?

Also, describe the typical gains (voltage and current) of this amplifier configuration, and whether it is {\it inverting} or {\it noninverting}.

\underbar{file 02226}
%(END_QUESTION)





%(BEGIN_ANSWER)

The common-collector amplifier configuration is defined by having the input and output signals referenced to the base and emitter terminals (respectively), with the collector terminal of the transistor typically having a low AC impedance to ground and thus being "common" to one pole of both the input and output voltages.  

Common-collector amplifiers are characterized by high current gains, voltage gains of 1 or (slightly) less, and a noninverting phase relationship between input and output.

%(END_ANSWER)





%(BEGIN_NOTES)

The answers to the question may be easily found in any fundamental electronics text, but it is important to ensure students know {\it why} these characteristics are such.  I always like to tell my students, "Memory {\it will} fail you, so you need to build an understanding of {\it why} things are, not just {\it what} things are."

%INDEX% Common-collector amplifier, characteristics of

%(END_NOTES)


