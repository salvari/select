
%(BEGIN_QUESTION)
% Copyright 2004, Tony R. Kuphaldt, released under the Creative Commons Attribution License (v 1.0)
% This means you may do almost anything with this work of mine, so long as you give me proper credit

In 1887, a German physicist named Heinrich Hertz successfully demonstrated the existence of {\it electromagnetic waves}.  Examine the following schematic of the apparatus he used to do this, and explain what significance Hertz's discovery has to do with your study of electronics:

$$\epsfbox{02278x01.eps}$$

\underbar{file 02278}
%(END_QUESTION)





%(BEGIN_ANSWER)

Hertz's experiment empirically demonstrated the theoretical discovery of James Clerk Maxwell, who concluded years before that "electromagnetic waves" comprised of electric and magnetic fields oscillating perpendicular to one another must be capable of radiating through empty space.  This is the basis of radio communication: generating these electromagnetic waves for the purpose of communicating information over long distances without wires.

%(END_ANSWER)





%(BEGIN_NOTES)

An experiment such as this is not difficult to set up.  Be sure to provide the appropriate safety precautions against electric shock, as such spark-gap transmitters (as they came to be called) require substantially high voltages to operate.

%INDEX% Hertz, Heinrich (proving existence of electromagnetic waves)

%(END_NOTES)


