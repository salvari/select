
%(BEGIN_QUESTION)
% Copyright 2005, Tony R. Kuphaldt, released under the Creative Commons Attribution License (v 1.0)
% This means you may do almost anything with this work of mine, so long as you give me proper credit

Trace the directions for all currents in this circuit, and calculate the values for voltage at the output ($V_{out}$) and at test point 1 ($V_{TP1}$) for several values of input voltage ($V_{in}$):

$$\epsfbox{02467x01.eps}$$

% No blank lines allowed between lines of an \halign structure!
% I use comments (%) instead, so that TeX doesn't choke.

$$\vbox{\offinterlineskip
\halign{\strut
\vrule \quad\hfil # \ \hfil & 
\vrule \quad\hfil # \ \hfil & 
\vrule \quad\hfil # \ \hfil \vrule \cr
\noalign{\hrule}
%
% First row
$V_{in}$ & $V_{TP1}$ & $V_{out}$ \cr
%
\noalign{\hrule}
%
% Second row
0.0 V &  & \cr
%
\noalign{\hrule}
%
% Third row
0.4 V &  & \cr
%
\noalign{\hrule}
%
% Fourth row
1.2 V &  & \cr
%
\noalign{\hrule}
%
% Fifth row
3.4 V &  & \cr
%
\noalign{\hrule}
%
% Sixth row
7.1 V &  & \cr
%
\noalign{\hrule}
%
% Seventh row
10.8 V &  & \cr
%
\noalign{\hrule}
} % End of \halign 
}$$ % End of \vbox

Then, from the table of calculated values, determine the voltage gain ($A_V$) for this amplifier circuit.

\underbar{file 02467}
%(END_QUESTION)





%(BEGIN_ANSWER)

$$\epsfbox{02467x02.eps}$$

% No blank lines allowed between lines of an \halign structure!
% I use comments (%) instead, so that TeX doesn't choke.

$$\vbox{\offinterlineskip
\halign{\strut
\vrule \quad\hfil # \ \hfil & 
\vrule \quad\hfil # \ \hfil & 
\vrule \quad\hfil # \ \hfil \vrule \cr
\noalign{\hrule}
%
% First row
$V_{in}$ & $V_{TP1}$ & $V_{out}$ \cr
%
\noalign{\hrule}
%
% Second row
0.0 V & 0.0 V & 0.0 V \cr
%
\noalign{\hrule}
%
% Third row
0.4 V & 0.0 V & -0.4 V \cr
%
\noalign{\hrule}
%
% Fourth row
1.2 V & 0.0 V & -1.2 V \cr
%
\noalign{\hrule}
%
% Fifth row
3.4 V & 0.0 V & -3.4 V \cr
%
\noalign{\hrule}
%
% Sixth row
7.1 V & 0.0 V & -7.1 V \cr
%
\noalign{\hrule}
%
% Seventh row
10.8 V & 0.0 V & -10.8 V \cr
%
\noalign{\hrule}
} % End of \halign 
}$$ % End of \vbox

$$A_V = 1 \hbox{ (ratio)} = 0 \hbox{ dB}$$

\vskip 10pt

Follow-up question: the point marked "TP1" in this circuit is often referred to as a {\it virtual ground}.  Explain why this is, based on the voltage figures shown in the above table.

%(END_ANSWER)





%(BEGIN_NOTES)

Some texts describe the voltage gain of an inverting voltage amplifier as being a negative quantity.  I tend not to look at things that way, treating all gains as positive quantities and relying on my knowledge of circuit behavior to tell whether the signal is inverted or not.  In my teaching experience, I have found that students have a tendency to blindly follow equations rather than think about what it is they are calculating, and that strict adherence to the mathematical signs of gain values only encourages this undesirable behavior ("If the sign of the gain tells me whether the circuit is inverting or not, I can just multiply input voltage by gain and the answer will always be right!").

This strategy is analogous to problem-solving in electromagnetics, where a common approach is to use math to solve for the absolute values of quantities (potential, induced voltage, magnetic flux), and then to use knowledge of physical principles (Lenz' Law, right-hand rule) to solve for polarities and directions.  The alternative -- to try to maintain proper sign convention throughout all calculations -- not only complicates the math but it also encourages students to over-focus on calculations and neglect fundamental principles.

%INDEX% Opamp, inverting amplifier circuit
%INDEX% Virtual ground, opamp circuit

%(END_NOTES)


