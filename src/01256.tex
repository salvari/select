
%(BEGIN_QUESTION)
% Copyright 2003, Tony R. Kuphaldt, released under the Creative Commons Attribution License (v 1.0)
% This means you may do almost anything with this work of mine, so long as you give me proper credit

The simplest type of digital logic circuit is an {\it inverter}, also called an {\it inverting buffer}, or {\it NOT gate}.  Here is a schematic diagram for an inverter gate constructed from bipolar transistors (transistor-to-transistor-logic, also known as {\it TTL}), shown connected to a SPDT switch and an LED:

$$\epsfbox{01256x01.eps}$$

The left-most transistor in this schematic is actually not being used as a transistor, but rather it functions as a "steering diode" network, like this:

$$\epsfbox{01256x04.eps}$$

Determine the status of the LED in each of the input switch's two positions.  Denote the logic level of switch and LED in the form of a truth table:

$$\epsfbox{01256x02.eps}$$

\underbar{file 01256}
%(END_QUESTION)





%(BEGIN_ANSWER)

$$\epsfbox{01256x03.eps}$$

%(END_ANSWER)





%(BEGIN_NOTES)

Have your students explain the operation of this TTL circuit, describing how the inverse logic state is generated at the output terminal, from a given input state.

%INDEX% TTL gate circuit, internal schematic

%(END_NOTES)


