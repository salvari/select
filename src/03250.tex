
%(BEGIN_QUESTION)
% Copyright 2005, Tony R. Kuphaldt, released under the Creative Commons Attribution License (v 1.0)
% This means you may do almost anything with this work of mine, so long as you give me proper credit

Identify each of the specified voltages in the following circuit.  The subscripts refer to the specific test points (where the red test lead of the voltmeter touches the circuit), while ground is the point where the voltmeter's black lead is assumed to be attached:

$$\epsfbox{03250x01.eps}$$

For example, $V_B$ means the voltage indicated by a voltmeter with the red test lead touching point B and the black test lead touching ground.

\medskip
\item{$\bullet$} $V_A$ = 
\item{$\bullet$} $V_B$ = 
\item{$\bullet$} $V_C$ = 
\item{$\bullet$} $V_D$ =
\item{$\bullet$} $V_E$ = 
\medskip

\underbar{file 03250}
%(END_QUESTION)





%(BEGIN_ANSWER)

\medskip
\item{$\bullet$} $V_A$ = +16 volts
\item{$\bullet$} $V_B$ = +16 volts
\item{$\bullet$} $V_C$ = +10 volts
\item{$\bullet$} $V_D$ = 0 volts
\item{$\bullet$} $V_E$ = -2 volts
\medskip

\vskip 10pt

Follow-up question: explain how it is possible to determine that $V_A$ and $V_B$ will be exactly the same value, prior to performing any mathematical calculations.

%(END_ANSWER)





%(BEGIN_NOTES)

This question helps to familiarize students with the concept of "ground" as the default reference point for taking voltage measurements, as well as being an application of Kirchhoff's Voltage Law.

%INDEX% Kirchhoff's Voltage Law

%(END_NOTES)


