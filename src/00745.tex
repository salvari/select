
%(BEGIN_QUESTION)
% Copyright 2003, Tony R. Kuphaldt, released under the Creative Commons Attribution License (v 1.0)
% This means you may do almost anything with this work of mine, so long as you give me proper credit

Electromechanical watt-hour meters use an aluminum disk that is spun by an electric motor.  To generate a constant "drag" on the disk necessary to limit its rotational speed, a strong magnet is placed in such a way that its lines of magnetic flux pass perpendicularly through the disk's thickness:

$$\epsfbox{00745x01.eps}$$

The disk itself need not be made of a ferromagnetic material in order for the magnet to create a "drag" force.  It simply needs to be a good conductor of electricity.

Explain the phenomenon accounting for the drag effect, and also explain what would happen if the roles of magnet and disk were reversed: if the magnet were moved in a circle around the periphery of a stationary disk.

\underbar{file 00745}
%(END_QUESTION)





%(BEGIN_ANSWER)

This is an example of {\it Lenz' Law}.  A rotating magnet would cause a torque to be generated in the disk.

%(END_ANSWER)





%(BEGIN_NOTES)

Mechanical speedometer assemblies used on many automobiles use this very principle: a magnet assembly is rotated by a cable connected to the vehicle's driveshaft.  This magnet rotates in close proximity to a metal disk, which gets "dragged" in the same direction that the magnet spins.  The disk's torque acts against the resistance of a spring, deflecting a pointer along a scale, indicating the speed of the vehicle.  The faster the magnet spins, the more torque is felt by the disk.

%INDEX% Lenz's Law
%INDEX% Watt-hour meter, "drag" magnet

%(END_NOTES)


