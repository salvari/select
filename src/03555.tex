
%(BEGIN_QUESTION)
% Copyright 2005, Tony R. Kuphaldt, released under the Creative Commons Attribution License (v 1.0)
% This means you may do almost anything with this work of mine, so long as you give me proper credit

Determine the capacitor voltage at the specified times (time $t$ = 0 milliseconds being the exact moment the switch contacts close).  Assume the capacitor begins in a fully discharged state:

$$\epsfbox{03555x01.eps}$$

% No blank lines allowed between lines of an \halign structure!
% I use comments (%) instead, so that TeX doesn't choke.

$$\vbox{\offinterlineskip
\halign{\strut
\vrule \quad\hfil # \ \hfil & 
\vrule \quad\hfil # \ \hfil \vrule \cr
\noalign{\hrule}
%
% First row
Time & $V_C$ (volts) \cr
%
\noalign{\hrule}
%
0 ms &  \cr
%
\noalign{\hrule}
%
30 ms &  \cr
%
\noalign{\hrule}
%
60 ms &  \cr
%
\noalign{\hrule}
%
90 ms &  \cr
%
\noalign{\hrule}
%
120 ms &  \cr
%
\noalign{\hrule}
%
150 ms &  \cr
%
\noalign{\hrule}
} % End of \halign 
}$$ % End of \vbox

\underbar{file 03555}
%(END_QUESTION)





%(BEGIN_ANSWER)

% No blank lines allowed between lines of an \halign structure!
% I use comments (%) instead, so that TeX doesn't choke.

$$\vbox{\offinterlineskip
\halign{\strut
\vrule \quad\hfil # \ \hfil & 
\vrule \quad\hfil # \ \hfil \vrule \cr
\noalign{\hrule}
%
% First row
Time & $V_C$ (volts) \cr
%
\noalign{\hrule}
%
0 ms & 0 \cr
%
\noalign{\hrule}
%
30 ms & 12.29 \cr
%
\noalign{\hrule}
%
60 ms & 19.71 \cr
%
\noalign{\hrule}
%
90 ms & 24.19 \cr
%
\noalign{\hrule}
%
120 ms & 26.89 \cr
%
\noalign{\hrule}
%
150 ms & 28.52 \cr
%
\noalign{\hrule}
} % End of \halign 
}$$ % End of \vbox

%(END_ANSWER)





%(BEGIN_NOTES)

Be sure to have your students share their problem-solving techniques (how they determined which equation to use, etc.) in class.

%INDEX% Time constant calculation, RC circuit

%(END_NOTES)


