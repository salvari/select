
\centerline{\bf ELTR 130 (Operational Amplifiers 1), section 1} \bigskip 
 
\vskip 10pt

\noindent
{\bf Recommended schedule}

\vskip 5pt

%%%%%%%%%%%%%%%
\hrule \vskip 5pt
\noindent
\underbar{Day 1}

\hskip 10pt Topics: {\it Differential pair circuits}
 
\hskip 10pt Questions: {\it 1 through 15}
 
\hskip 10pt Lab Exercise: {\it Discrete differential amplifier (question 56)}
 
%INSTRUCTOR \hskip 10pt {\bf Explain suggested project ideas to students}

%INSTRUCTOR \hskip 10pt {\bf Give project grading rubric to students, complete with deadlines}

\vskip 10pt
%%%%%%%%%%%%%%%
\hrule \vskip 5pt
\noindent
\underbar{Day 2}

\hskip 10pt Topics: {\it The basic operational amplifier}
 
\hskip 10pt Questions: {\it 16 through 25}
 
\hskip 10pt Lab Exercise: {\it Discrete differential amplifier (question 56, continued)}
 
\vskip 10pt
%%%%%%%%%%%%%%%
\hrule \vskip 5pt
\noindent
\underbar{Day 3}

\hskip 10pt Topics: {\it Using the operational amplifier as a comparator}
 
\hskip 10pt Questions: {\it 26 through 35}
 
\hskip 10pt Lab Exercise: {\it Comparator circuit (question 57)}
 
%INSTRUCTOR \hskip 10pt {\bf MIT 6.002 video clip: Disk 4, Lecture 21; Open-loop opamp 27:30 to 29:32}

\vskip 10pt
%%%%%%%%%%%%%%%
\hrule \vskip 5pt
\noindent
\underbar{Day 4}

\hskip 10pt Topics: {\it Using the operational amplifier as a voltage buffer}
 
\hskip 10pt Questions: {\it 36 through 45}
 
\hskip 10pt Lab Exercise: {\it Opamp voltage follower (question 58)}
 
%INSTRUCTOR \hskip 10pt {\bf MIT 6.002 video clip: Disk 4, Lecture 21; Open-loop opamp 41:30 to 43:10}

\vskip 10pt
%%%%%%%%%%%%%%%
\hrule \vskip 5pt
\noindent
\underbar{Day 5}

\hskip 10pt Topics: {\it Additional applications of feedback} (optional)
 
\hskip 10pt Questions: {\it 46 through 55}
 
\hskip 10pt Lab Exercise: {\it Linear voltage regulator circuit (question 59)}
 
%INSTRUCTOR \hskip 10pt {\bf Socratic Electronics animation: Push-pull transistor amplifier with crossover distortion}

\vskip 10pt
%%%%%%%%%%%%%%%
\hrule \vskip 5pt
\noindent
\underbar{Day 6}

\hskip 10pt Exam 1: {\it includes Comparator circuit performance assessment}
 
\hskip 10pt Lab Exercise: {\it Select an opamp project to prototype and troubleshoot by the end of the next course section (ELTR130, Section 2)}
  
\vskip 10pt
%%%%%%%%%%%%%%%
\hrule \vskip 5pt
\noindent
\underbar{Troubleshooting practice problems}

\hskip 10pt Questions: {\it 62 through 71}
 
\vskip 10pt
%%%%%%%%%%%%%%%
\hrule \vskip 5pt
\noindent
\underbar{General concept practice and challenge problems}

\hskip 10pt Questions: {\it 72 through the end of the worksheet}
 
\vskip 10pt
%%%%%%%%%%%%%%%
\hrule \vskip 5pt
\noindent
\underbar{Impending deadlines}

\hskip 10pt {\bf Troubleshooting assessment (project prototype) due at end of ELTR130, Section 2}

\hskip 10pt Question 60: Troubleshooting log
 
\hskip 10pt Question 61: Sample troubleshooting assessment grading criteria
 
\vskip 10pt
%%%%%%%%%%%%%%%




\vfil \eject

\centerline{\bf ELTR 130 (Operational Amplifiers 1), section 1} \bigskip 
 
\vskip 10pt

\noindent
{\bf Project ideas}

\vskip 5pt

\hrule \vskip 5pt

\vskip 10pt

\noindent
\underbar{Audio signal generator amplifier:} Uses an op-amp to amplify the output of a digital audio playback device (such as a CD-audio or MP3 player) for use as a sine wave signal generator.  Sine waves at different frequencies are recorded on digital media as different tracks -- the op-amp circuit providing voltage and current gain -- allowing the use of any inexpensive consumer-grade audio playback device as an audio frequency signal generator.

\vskip 10pt

\noindent
\underbar{Intercom system:} Uses an op-amp to amplify the output of a speaker (used as a microphone) to power another speaker located some distance away.  With a reversing switch, the roles of the two speakers may be reversed (send versus receive).

\vskip 10pt

\noindent
\underbar{Seismograph:} Uses an op-amp to amplify small voltages generated by a stationary "pickup" coil located near a pendulum-mounted permanent magnet.  Vibrations in the earth create motion between the magnet and the coil, inducing voltage in the coil.  The op-amp output then drives a meter, recording device, or an alarm.

\vskip 10pt

\noindent
\underbar{Pulse-width modulation signal generator:} There are many ways to make such a circuit, but almost all of them use a comparator to compare an adjustable DC reference voltage against a varying (oscillating) voltage produced by an oscillator circuit.  The resulting comparator output will be a square wave with variable duty cycle, useful for driving power transistors for PWM power control of electric loads.

\vskip 10pt

\noindent
\underbar{Series voltage regulator:} Uses an op-amp to "buffer" the reference voltage of a zener diode, driving a transistor to maintain constant DC voltage to a load.  The op-amp provides much greater precision and regulation over a wide range of load resistances than a simple zener-BJT regulator circuit could on its own.

\vskip 10pt

\noindent
\underbar{Amplified audio detector:} The "sensitive audio detector" circuit suggested in ELTR115 (AC 2) may be improved with the addition of an op-amp amplification stage.  This can drastically raise input impedance and sensitivity.

\vskip 10pt

\noindent
\underbar{Infra-red motion sensor:} Passive infra-red detectors are available for purchase (or salvaged from old motion-sensitive light controller circuits) which output a small DC voltage corresponding to IR light intensity.  By amplifying this voltage and passing it through an active differentiator circuit, an output voltage representing {\it rate of change} of IR light will be produced.  This signal may then be sent to a comparator to trigger an alarm or take some other action when a warm object {\it moves} by the sensor.

\vskip 10pt

\noindent
\underbar{High-impedance analog voltmeter:} Uses a JFET or MOSFET input op-amp to drive and analog meter movement for precise measurement of DC voltage.

\vskip 10pt






\vfil \eject

\centerline{\bf ELTR 130 (Operational Amplifiers 1), section 1} \bigskip 
 
\vskip 10pt

\noindent
{\bf Skill standards addressed by this course section}

\vskip 5pt

%%%%%%%%%%%%%%%
\hrule \vskip 10pt
\noindent
\underbar{EIA {\it Raising the Standard; Electronics Technician Skills for Today and Tomorrow}, June 1994}

\vskip 5pt

\medskip
\item{\bf E} {\bf Technical Skills -- Analog Circuits}
\item{\bf E.10} Understand principles and operations of operational amplifier circuits.
\item{\bf E.11} Fabricate and demonstrate operational amplifier circuits.
\item{\bf E.12} Troubleshoot and repair operational amplifier circuits.
\medskip

\vskip 5pt

\medskip
\item{\bf B} {\bf Basic and Practical Skills -- Communicating on the Job}
\item{\bf B.01} Use effective written and other communication skills.  {\it Met by group discussion and completion of labwork.}
\item{\bf B.03} Employ appropriate skills for gathering and retaining information.  {\it Met by research and preparation prior to group discussion.}
\item{\bf B.04} Interpret written, graphic, and oral instructions.  {\it Met by completion of labwork.}
\item{\bf B.06} Use language appropriate to the situation.  {\it Met by group discussion and in explaining completed labwork.}
\item{\bf B.07} Participate in meetings in a positive and constructive manner.  {\it Met by group discussion.}
\item{\bf B.08} Use job-related terminology.  {\it Met by group discussion and in explaining completed labwork.}
\item{\bf B.10} Document work projects, procedures, tests, and equipment failures.  {\it Met by project construction and/or troubleshooting assessments.}
\item{\bf C} {\bf Basic and Practical Skills -- Solving Problems and Critical Thinking}
\item{\bf C.01} Identify the problem.  {\it Met by research and preparation prior to group discussion.}
\item{\bf C.03} Identify available solutions and their impact including evaluating credibility of information, and locating information.  {\it Met by research and preparation prior to group discussion.}
\item{\bf C.07} Organize personal workloads.  {\it Met by daily labwork, preparatory research, and project management.}
\item{\bf C.08} Participate in brainstorming sessions to generate new ideas and solve problems.  {\it Met by group discussion.}
\item{\bf D} {\bf Basic and Practical Skills -- Reading}
\item{\bf D.01} Read and apply various sources of technical information (e.g. manufacturer literature, codes, and regulations).  {\it Met by research and preparation prior to group discussion.}
\item{\bf E} {\bf Basic and Practical Skills -- Proficiency in Mathematics}
\item{\bf E.01} Determine if a solution is reasonable.
\item{\bf E.02} Demonstrate ability to use a simple electronic calculator.
\item{\bf E.05} Solve problems and [sic] make applications involving integers, fractions, decimals, percentages, and ratios using order of operations.
\item{\bf E.06} Translate written and/or verbal statements into mathematical expressions.
\item{\bf E.09} Read scale on measurement device(s) and make interpolations where appropriate.  {\it Met by oscilloscope usage.}
\item{\bf E.12} Interpret and use tables, charts, maps, and/or graphs.
\item{\bf E.13} Identify patterns, note trends, and/or draw conclusions from tables, charts, maps, and/or graphs.
\item{\bf E.15} Simplify and solve algebraic expressions and formulas.
\item{\bf E.16} Select and use formulas appropriately.
\item{\bf E.17} Understand and use scientific notation.
\medskip

%%%%%%%%%%%%%%%




\vfil \eject

\centerline{\bf ELTR 130 (Operational Amplifiers 1), section 1} \bigskip 
 
\vskip 10pt

\noindent
{\bf Common areas of confusion for students}

\vskip 5pt


\hrule \vskip 5pt

\vskip 10pt

\noindent
{\bf Difficult concept: } {\it Inverting nature of common-emitter amplifier.}

Some students find it quite difficult to grasp why the DC output voltage of a common-emitter amplifier {\it decreases} as the DC input voltage level increases.  Step-by-step DC analysis of the circuit is the only remedy I have found to this conceptual block: getting students to carefully analyze what happens as voltages increase and decrease.

\vskip 10pt

\noindent
{\bf Difficult concept: } {\it Differential pair circuits.}

Perhaps the most difficult concept to grasp regarding differential pair circuits is that they are basically a hybrid of common-collector, common-base, and common-emitter amplifiers.  This is why a strong knowledge of the three basic amplifier types is essential for understanding how differential pairs work, and why I begin exploring differential pairs by reviewing C-C, C-E, and C-B amplifiers.

\vskip 10pt

\noindent
{\bf Difficult concept: } {\it Determining comparator output polarity.}

The key to determining the polarity of a comparator's output is applying Kirchhoff's Voltage Law to the two signals at the input terminals to find the differential input voltage, then seeing whether the differential voltage's polarity matches the polarity markings of the comparator's input terminals.  If so, the output will saturate in a positive direction.  If not, the output will saturate in a negative direction.

\vskip 10pt

\noindent
{\bf Difficult concept: } {\it Negative feedback.}

Few concepts are as fundamentally important in electronics as negative feedback, and so it is essential for the electronics student to learn well.  However, it is not an easy concept for many to grasp.  The notion that a portion of the output signal may be "fed back" into the input in a degenerative manner to stabilize gain is far from obvious.  One of the most powerfully illustrative examples I know of is the use of negative feedback in a voltage regulator circuit to compensate for the base-emitter voltage drop of 0.7 volts (see question file \#02286).

\vskip 10pt

\noindent
{\bf Common mistake: } {\it Thinking that an opamp's output current is supplied through its input terminals.}

This is a misconception that seems to have an amazing resistance to correction.  There seem to always be a few students who think that there is a direct path for current from the input terminals of an opamp to its output terminal.  It is very important to realize that {\it for most practical purposes, an opamp draws negligible current through its input terminals!}  What current does go through the output terminal is {\it always} supplied by the \underbar{power} terminals and from the power supply, never by the input signal(s).  To put this into colloquial terms, the input terminals on an opamp tell the output what to do, but they do not give the output its "muscle" (current) to do it.

I think the reason for this misconception is the fact that power terminals are often omitted from opamp symbols for brevity, and after a while of seeing this it is easy to forget they are really still there performing a useful function!


