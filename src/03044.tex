
%(BEGIN_QUESTION)
% Copyright 2005, Tony R. Kuphaldt, released under the Creative Commons Attribution License (v 1.0)
% This means you may do almost anything with this work of mine, so long as you give me proper credit

{\it Microcontrollers} are single-chip microcomputers, containing a microprocessor core, memory, I/O control, and other associated components necessary to make the system self-contained.  Simply put, a microcontroller follows sequential instructions that someone enters into its memory.

{\it Programmable logic} devices, however, are fundamentally different from microcontrollers both in how they are programmed and how they function after being programmed.  Explain what some of these differences are.

\underbar{file 03044}
%(END_QUESTION)





%(BEGIN_ANSWER)

Unlike microcontrollers, programmable logic devices are not (necessarily) sequential devices: the latter act as a collection of logic gates and other "primitive" logic elements to directly implement certain logic functions.

%(END_ANSWER)





%(BEGIN_NOTES)

Discuss with your students how programmable logic devices are more primitive and direct devices than microcontrollers, which are more abstract by comparison.  Perhaps the easiest distinction to understand is in terms of gate connections.  In a microcontroller, the connections between its constituent gates are fixed; only the software (bits stored in memory) ever change.  In a programmable logic device, it is as though you are directly forging connections between its constituent gates (as many or as few as needed), creating a hard-wired circuit by specifying connections in a "hardware description language" (HDL).

%INDEX% Microcontroller versus programmable logic
%INDEX% Programmable logic versus microcontroller

%(END_NOTES)


