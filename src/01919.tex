
%(BEGIN_QUESTION)
% Copyright 2003, Tony R. Kuphaldt, released under the Creative Commons Attribution License (v 1.0)
% This means you may do almost anything with this work of mine, so long as you give me proper credit

A student is trying to measure an AC waveform superimposed on a DC voltage, output by the following circuit:

$$\epsfbox{01919x01.eps}$$

The problem is, every time the student moves the circuit's DC bias adjustment knob, the oscilloscope loses its triggering and the waveform begins to wildly scroll across the width of the screen.  In order to get the oscilloscope to trigger on the AC signal again, the student must likewise move the trigger level knob on the oscilloscope panel.  Inspect the settings on the student's oscilloscope (shown here) and determine what could be configured differently to achieve consistent triggering so the student won't have to re-adjust the trigger level every time she re-adjusts the circuit's DC bias voltage:

$$\epsfbox{01919x02.eps}$$

\underbar{file 01919}
%(END_QUESTION)





%(BEGIN_ANSWER)

Set the trigger coupling control from "DC" to "AC".

%(END_ANSWER)





%(BEGIN_NOTES)

In order for students to successfully answer this question, they must grasp the function of the circuit itself.  Discuss with them why and how the rheostat is able to change the amount of DC "bias" voltage imposed on the AC signal, then progress to discussing the oscilloscope's triggering.

%INDEX% Oscilloscope, AC versus DC trigger coupling

%(END_NOTES)


