
%(BEGIN_QUESTION)
% Copyright 2003, Tony R. Kuphaldt, released under the Creative Commons Attribution License (v 1.0)
% This means you may do almost anything with this work of mine, so long as you give me proper credit

These two phasors are written in a form known as {\it polar notation}.  Re-write them in {\it rectangular notation}:

$$4 \> \angle \> 0^o = $$ 

$$3 \> \angle \> 90^o = $$ 

\underbar{file 00497}
%(END_QUESTION)





%(BEGIN_ANSWER)

These two phasors, written in rectangular notation, would be $4 + j0$ and $0 + j3$, respectively, although a mathematician would probably write them as $4 + i0$ and $0 + i3$, respectively.

\vskip 10pt

Challenge question: what does the lower-case $j$ or $i$ represent, in mathematical terms?

%(END_ANSWER)





%(BEGIN_NOTES)

Discuss with your students the two notations commonly used with phasors: {\it polar} and {\it rectangular} form.  They are merely two different ways of "saying" the same thing.  A helpful "prop" for this discussion is the complex number {\it plane} (as opposed to a number {\it line} -- a one-dimensional field), showing the "real" and "imaginary" axes, in addition to standard angles (right = 0$^{o}$, left = 180$^{o}$, up = 90$^{o}$, down = 270$^{o}$).  Your students should be familiar with this from their research, so have one of them draw the number plane on the whiteboard for all to view.

The challenge question regards the origin of complex numbers, beginning with the distinction of "imaginary" numbers as being a separate set of quantities from "real" numbers.  Electrical engineers, of course, avoid using the lower-case letter $i$ to denote "imaginary" because it would be so easily be confused with the standard notation for instantaneous current $i$.

%INDEX% Polar notation versus rectangular notation
%INDEX% Rectangular notation versus polar notation

%(END_NOTES)


