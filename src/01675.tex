
%(BEGIN_QUESTION)
% Copyright 2003, Tony R. Kuphaldt, released under the Creative Commons Attribution License (v 1.0)
% This means you may do almost anything with this work of mine, so long as you give me proper credit

$$\epsfbox{01675x01.eps}$$

\underbar{file 01675}
\vfil \eject
%(END_QUESTION)





%(BEGIN_ANSWER)

Use circuit simulation software to verify your predicted and measured parameter values.

%(END_ANSWER)





%(BEGIN_NOTES)

Use line-frequency power transformers for this exercise, with load resistor values low enough to "swamp" the primary winding's excitation current, so that the primary/secondary current ratio is realistic.  Choosing a resistor value low enough to load the transformer near 100 \% rated secondary current is a good start.  Be careful that your load resistor can handle the power dissipation!

It might be a good idea for students to take careful measurements of primary and secondary voltage in an {\it unloaded} condition in order to calculate the actual winding turns ratio of their transformer.  Knowing this precise ratio will be helpful to them later on when they use their transformers in other performance assessment activities, so their predictions will more closely match their actual measurements.

An extension of this exercise is to incorporate troubleshooting questions.  Whether using this exercise as a performance assessment or simply as a concept-building lab, you might want to follow up your students' results by asking them to predict the consequences of certain circuit faults.

%INDEX% Assessment, performance-based (Power transformer voltage and current ratios)

%(END_NOTES)


