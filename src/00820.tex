
%(BEGIN_QUESTION)
% Copyright 2003, Tony R. Kuphaldt, released under the Creative Commons Attribution License (v 1.0)
% This means you may do almost anything with this work of mine, so long as you give me proper credit

Assuming that the output frequency of an alternator must remain constant (as is the case in national power systems, where the frequency of all power plants must be the same), how may its output voltage be regulated?  In other words, since we do not have the luxury of increasing or decreasing its rotational speed to control voltage, since that would change the frequency, how can we coax the alternator to produce more or less voltage on demand?

\vskip 10pt

Hint: automotive alternators are manufactured with this feature, though the purpose in that application is to maintain constant voltage despite changes in engine speed.  In automotive electrical systems, the frequency of the alternator's output is irrelevant because the AC is "rectified" into DC (frequency = 0 Hz) to charge the battery.

\underbar{file 00820}
%(END_QUESTION)





%(BEGIN_ANSWER)

The rotor cannot be a permanent magnet, but must be an electromagnet, where we can change its magnetic field strength at will.

\vskip 10pt

Follow-up question: how is it possible to conduct electric power to windings on a spinning rotor?  Should we energize the rotor winding with AC or DC?  Explain your answer.

%(END_ANSWER)





%(BEGIN_NOTES)

Ask your students how this voltage regulation strategy compares with that of DC generators.  Ask them to describe the difference between "commutator bars" and "slip rings."

%INDEX% Voltage regulation, generator

%(END_NOTES)


