
%(BEGIN_QUESTION)
% Copyright 2005, Tony R. Kuphaldt, released under the Creative Commons Attribution License (v 1.0)
% This means you may do almost anything with this work of mine, so long as you give me proper credit

Give a step-by-step procedure for "Th\'evenizing" {\it any} circuit: finding the Th\'evenin equivalent voltage ($V_{Thevenin}$) and Th\'evenin equivalent resistance ($R_{Thevenin}$).

\medskip
\goodbreak
\item{$\bullet$} Step \#1:
\vskip 5pt
\item{$\bullet$} Step \#2:
\medskip

\underbar{file 02456}
%(END_QUESTION)





%(BEGIN_ANSWER)

This is easy enough for you to look up in any electronics textbook.  I'll leave you to it!

\vskip 10pt

Follow-up question: describe the difference in how one must consider {\it voltage sources} versus {\it current sources} when calculating the equivalent circuit's resistance ($R_{Thevenin}$) of a complex circuit containing both types of sources?

%(END_ANSWER)





%(BEGIN_NOTES)

I really mean what I say here about looking this up in a textbook.  Th\'evenin's Theorem is a very well-covered subject in many books, and so it is perfectly reasonable to expect students will do this research on their own and come back to class with a complete answer.

The follow-up question is very important, because some circuits (especially transistor amplifier circuits) contain {\it both} types of sources.  Knowing how to consider each one in the process of calculating the Th\'evenin equivalent resistance for a circuit is very important.  When performing this analysis on transistor amplifiers, the circuit often becomes much simpler than its original form with all the voltage sources shorted and current sources opened!

%INDEX% Thevenin equivalent circuit, step-by-step procedure for determining values of V and R

%(END_NOTES)


