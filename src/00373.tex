
%(BEGIN_QUESTION)
% Copyright 2003, Tony R. Kuphaldt, released under the Creative Commons Attribution License (v 1.0)
% This means you may do almost anything with this work of mine, so long as you give me proper credit

Two 5 H inductors connected in series are subjected to an electric current that changes at a rate of 4.5 amps per second.  How much voltage will be dropped across the series combination?

Now suppose that two 5 H inductors connected in parallel are subjected to the same total applied current (changing at a rate of 4.5 amps per second).  How much voltage will be dropped by these inductors?  Hint: the total current is divided evenly between the two inductors.

\underbar{file 00373}
%(END_QUESTION)





%(BEGIN_ANSWER)

Series connection: 45 volts total.  Parallel connection: 11.25 volts total.  

\vskip 10pt

Follow-up question: what do these figures indicate about the nature of series-connected and parallel connected inductors?  In other words, what single inductor value is equivalent to two series-connected 5 H inductors, and what single inductor value is equivalent to two parallel-connected 5 H inductors?

%(END_ANSWER)





%(BEGIN_NOTES)

If your students are having difficulty answering the follow-up question in the Answer, ask them to compare these voltage figures (45 V and 11.25 V) against the voltage that would be dropped by just one of the 5 H inductors under the same condition (an applied current changing at a rate of 4.5 amps per second).

It is, of course, important that students know how series-connected and parallel connected inductors behave.  However, this is typically a process of rote memorization for students rather than true understanding.  With this question, the goal is to have students come to a realization of inductor connections based on their understanding of series and parallel voltages and currents.

%INDEX% Series inductors
%INDEX% Inductors, series
%INDEX% Parallel inductors
%INDEX% Inductors, parallel

%(END_NOTES)


