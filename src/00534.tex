
%(BEGIN_QUESTION)
% Copyright 2003, Tony R. Kuphaldt, released under the Creative Commons Attribution License (v 1.0)
% This means you may do almost anything with this work of mine, so long as you give me proper credit

A technician prepares to use an oscilloscope to display an AC voltage signal.  After turning the oscilloscope on and connecting the Y input probe to the signal source test points, this display appears:

$$\epsfbox{00534x01.eps}$$

What display control(s) need to be adjusted on the oscilloscope in order to show a normal-looking wave on the screen?

\underbar{file 00534}
%(END_QUESTION)





%(BEGIN_ANSWER)

The "vertical" control needs to be adjusted for a greater number of volts per division.

%(END_ANSWER)





%(BEGIN_NOTES)

Discuss the function of both these controls with your students.  If possible, demonstrate this scenario using a real oscilloscope and function generator, and have students adjust the controls to get the waveform to display optimally.  Challenge your students to think of ways the {\it signal source} (function generator) may be adjusted to produce the display, then have them think of ways the oscilloscope controls could be adjusted to fit.

%INDEX% Oscilloscope, vertical range controls (qualitative)

%(END_NOTES)


