
%(BEGIN_QUESTION)
% Copyright 2004, Tony R. Kuphaldt, released under the Creative Commons Attribution License (v 1.0)
% This means you may do almost anything with this work of mine, so long as you give me proper credit

Another name for "capacitor" is {\it condenser}.  Explain what a {\it synchronous condenser} is, and how it is used to correct power factor in AC power systems.

\underbar{file 02194}
%(END_QUESTION)





%(BEGIN_ANSWER)

A "synchronous condenser" is a special type of AC electric motor that happens to have a variable power factor.  They are used as variable capacitors to correct for changing power factors.

\vskip 10pt

Challenge question: capacitors are considered reactive devices because they have to ability to store and release energy.  How would a synchronous condenser store and release energy, seeing as it does not make use of electric fields as capacitors do?

%(END_ANSWER)





%(BEGIN_NOTES)

There is a fair amount of information available on the internet and also in power engineering texts on the subject of synchronous condensers, although this mature technology is being superseded by solid-state {\it static VAR compensator} circuits which have no moving parts.

You might wish to mention that most AC generators (alternators) have the ability to run as synchronous motors, and therefore as synchronous condensers.  It is commonplace for spare generators at power plants to be "idled" as electric motors and used to generate leading VARs to reduce heating in the windings of the other generators.  This is especially true at hydroelectric dams, where frequent shut-downs and start-ups of generator units is discouraged due to the massive size of the units and the physical wear incurred during that cycling.

%INDEX% Synchronous condenser

%(END_NOTES)


