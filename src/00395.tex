
%(BEGIN_QUESTION)
% Copyright 2003, Tony R. Kuphaldt, released under the Creative Commons Attribution License (v 1.0)
% This means you may do almost anything with this work of mine, so long as you give me proper credit

When the switch closes, the ammeter will initially register a large amount of current, then the current will decay to a much lesser value over time as the motor speeds up:

$$\epsfbox{00395x01.eps}$$

In view of Ohm's Law, where current is supposed to be a direct function of voltage and resistance ($I = {E \over R}$), explain why this happens.  After all, the motor's winding resistance does not change as it spins, and the battery voltage is fairly constant.  Why, then, does the current vary so greatly between initial start-up and full operating speed?

What do you think the ammeter will register after the motor has achieved full (no-load) speed, if a mechanical load is placed on the motor shaft, forcing it to slow down?

\underbar{file 00395}
%(END_QUESTION)





%(BEGIN_ANSWER)

Motor current is inversely proportional to speed, due to the counter-EMF produced by the armature as it rotates.

\vskip 10pt

Follow-up question: draw a schematic diagram showing the equivalent circuit of battery, switch, ammeter, and motor, with the counter-EMF of the motor represented as another battery symbol.  Which way must the counter-EMF voltage face, {\it opposed} to the battery voltage, or {\it aiding} the battery voltage?

%(END_ANSWER)





%(BEGIN_NOTES)

The so-called "inrush" current of an electric motor during startup can be quite substantial, upwards of ten times the normal full-load current!

%INDEX% Inrush current, electric motor
%INDEX% Counter-EMF, electric motor

%(END_NOTES)


