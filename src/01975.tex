
%(BEGIN_QUESTION)
% Copyright 2003, Tony R. Kuphaldt, released under the Creative Commons Attribution License (v 1.0)
% This means you may do almost anything with this work of mine, so long as you give me proper credit

$$\epsfbox{01975x01.eps}$$

\underbar{file 01975}
\vfil \eject
%(END_QUESTION)





%(BEGIN_ANSWER)

Use circuit simulation software to verify your predicted and measured parameter values.

%(END_ANSWER)





%(BEGIN_NOTES)

Use a variable-voltage, regulated power supply to supply any amount of DC voltage below 30 volts.  Specify standard resistor values, all between 1 k$\Omega$ and 100 k$\Omega$ (1k5, 2k2, 2k7, 3k3, 4k7, 5k1, 6k8, 10k, 22k, 33k, 39k 47k, 68k, etc.). 

$R_{pot}$ serves the purpose of providing variable AC gain in the first amplifier stage to meet the Barkhausen criterion.

I have had good success with the following values:

\medskip
\item{$\bullet$} $V_{CC}$ = 12 volts
\item{$\bullet$} $C_1$ and $C_2$ = 0.001 $\mu$F
\item{$\bullet$} $C_3$ = 47 $\mu$F
\item{$\bullet$} $C_4$ = 0.47 $\mu$F
\item{$\bullet$} $R_1$ and $R_2$ = 4.7 k$\Omega$
\item{$\bullet$} $R_3$ = 4.7 k$\Omega$
\item{$\bullet$} $R_4$ = 39 k$\Omega$
\item{$\bullet$} $R_5$ = 22 k$\Omega$
\item{$\bullet$} $R_6$ = 27 k$\Omega$
\item{$\bullet$} $R_7$ = 3.3 k$\Omega$
\item{$\bullet$} $R_{pot}$ = 10 k$\Omega$, linear
\item{$\bullet$} $Q_1$ and $Q_2$ = part number 2N2222
\medskip

An extension of this exercise is to incorporate troubleshooting questions.  Whether using this exercise as a performance assessment or simply as a concept-building lab, you might want to follow up your students' results by asking them to predict the consequences of certain circuit faults.

%INDEX% Assessment, performance-based (Wien bridge oscillator, BJT)

%(END_NOTES)


