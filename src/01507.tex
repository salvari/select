
%(BEGIN_QUESTION)
% Copyright 2003, Tony R. Kuphaldt, released under the Creative Commons Attribution License (v 1.0)
% This means you may do almost anything with this work of mine, so long as you give me proper credit

When you measure the low-voltage AC output of your transformer and compare that measurement to the rectified and filtered DC output voltage, you will notice something very counter-intuitive.  The DC output voltage reading is substantially {\it greater} than the unrectified AC voltage from the transformer's secondary winding!

Explain why this is, and then show mathematical calculations that relate the two voltage measurements together with reasonable accuracy.

\underbar{file 01507}
%(END_QUESTION)





%(BEGIN_ANSWER)

This is something that catches almost all students by surprise when they first measure the voltages.  I'll give you a big hint why the rectified and filtered (DC) output voltage is so much greater than the unrectified (AC) secondary voltage: the AC voltage measurement you make with your voltmeter is most likely an {\it RMS} measurement, not a {\it peak} measurement.

%(END_ANSWER)





%(BEGIN_NOTES)

Let students figure out this mystery on their own: it is one of those phenomena that really reveals the nature of RMS measurements in contrast to peak measurements.

%(END_NOTES)


