
%(BEGIN_QUESTION)
% Copyright 2003, Tony R. Kuphaldt, released under the Creative Commons Attribution License (v 1.0)
% This means you may do almost anything with this work of mine, so long as you give me proper credit

This circuit uses a {\it unijunction transistor} (UJT) to latch an LED in the "on" state with a positive pulse at the input terminal.  A negative voltage pulse at the input terminal turns the LED off:

$$\epsfbox{01128x01.eps}$$

Explain how the unijunction transistor functions in this circuit. 

\underbar{file 01128}
%(END_QUESTION)





%(BEGIN_ANSWER)

Unijunction transistors are hysteretic, like all thyristors.  A positive pulse to the emitter terminal latches the UJT, and a negative pulse makes it "drop out".

\vskip 10pt

Challenge question: what is the purpose of resistor $R_3$ in this circuit?

%(END_ANSWER)





%(BEGIN_NOTES)

Ask your students to identify the terminals on the UJT.  The designations for each terminal may be surprising to your students, given the names of bipolar transistor terminals!

The challenge question may only be answered if one carefully considers the characteristics of an LED.  Resistor $R_3$ helps overcome problems that might potentially arise due to the nonlinearities of the diode in its off state.

I got this circuit from the October 2003 issue of \underbar{Electronics World} magazine, in their regular "Circuit Ideas" section.  The design is attributed to Andr\'e de Gu\'erin. 

%INDEX% UJT
%INDEX% Unijunction transistor

%(END_NOTES)


