
%(BEGIN_QUESTION)
% Copyright 2003, Tony R. Kuphaldt, released under the Creative Commons Attribution License (v 1.0)
% This means you may do almost anything with this work of mine, so long as you give me proper credit

Electrical capacitance has a close mechanical analogy: {\it elasticity}.  Explain what the term "elasticity" means for a mechanical spring, and how the quantities of velocity and force applied to a spring are respectively analogous to current and voltage applied to a capacitance.

\underbar{file 01139}
%(END_QUESTION)





%(BEGIN_ANSWER)

As a spring is compressed at a constant velocity, the amount of reaction force it generates increases at a linear rate:

$$v = {1 \over k}{dF \over dt}$$

\noindent
Where,

$v = $ Velocity of spring compression

$k = $ Spring "stiffness" constant

$F = $ Reaction force generated by the spring's compression

$t = $ Time

\vskip 10pt

In a similar manner, a pure capacitance experiencing a constant current will exhibit a constant rate of voltage change over time:

$$i = C{de \over dt}$$

%(END_ANSWER)





%(BEGIN_NOTES)

Note to your students that spring stiffness ($k$) and capacitance ($C$) are inversely proportional to one another in this analogy.

Explain to your students how the similarities between inertia and capacitance are so close, that capacitors can be used to electrically model mechanical springs!

%INDEX% Capacitance, analogous to mechanical elasticity

%(END_NOTES)


