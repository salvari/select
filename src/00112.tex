
%(BEGIN_QUESTION)
% Copyright 2003, Tony R. Kuphaldt, released under the Creative Commons Attribution License (v 1.0)
% This means you may do almost anything with this work of mine, so long as you give me proper credit

Of the three types of "elementary particles" constituting atoms, determine which type(s) influence the following properties of an element:

\vskip 10pt

\item {$\bullet$} The chemical identity of the atoms (whether it is an atom of {\it nitrogen}, {\it iron}, {\it silver}, or some other element).
\item {$\bullet$} The mass of the atom.
\item {$\bullet$} The electrical charge of the atom.
\item {$\bullet$} Whether or not it is radioactive (spontaneous disintegration of the nucleus).

\vskip 10pt

\underbar{file 00112}
%(END_QUESTION)





%(BEGIN_ANSWER)

\item {$\bullet$} The chemical identity of the atoms: {\bf protons}.
\item {$\bullet$} The mass of the atom: {\bf neutrons} and {\bf protons}, and to a much lesser extent, {\bf electrons}.
\item {$\bullet$} The electrical charge of the atom: {\bf electrons} and {\bf protons} (whether or not the numbers are equal).
\item {$\bullet$} Whether or not it is radioactive: {\bf neutrons}, although one might also say {\bf protons} in some cases, as there are no known "stable" (non-radioactive) isotopes of certain elements, the identity of an element being determined strictly by the number of protons.

%(END_ANSWER)





%(BEGIN_NOTES)

It never ceases to fascinate me how many of the basic properties of elements is determined by a simple integer count of particles within each atom's nucleus.

In the answer, I introduce the word {\it isotope}.  Let students research what this term means.  Don't simply tell them!

%INDEX% Particles, subatomic
%INDEX% Subatomic particles
%INDEX% Proton
%INDEX% Neutron
%INDEX% Electron

%(END_NOTES)


