
%(BEGIN_QUESTION)
% Copyright 2003, Tony R. Kuphaldt, released under the Creative Commons Attribution License (v 1.0)
% This means you may do almost anything with this work of mine, so long as you give me proper credit

In this circuit there is at least one ammeter that is not reading correctly:

$$\epsfbox{01742x01.eps}$$

Apply Kirchhoff's Current Law to this circuit to prove why all three current measurements shown here cannot be correct.

\underbar{file 01742}
%(END_QUESTION)





%(BEGIN_ANSWER)

Kirchhoff's Current Law renders this scenario impossible: "the algebraic sum of all currents at a node {\it must} be zero."

%(END_ANSWER)





%(BEGIN_NOTES)

Another way of stating KCL is to say, "What goes in must come out."  When continuous DC currents are involved, this law is really nothing more than a restatement of the Conservation of Charge.  Of course, there are transient exceptions to this law (static electric charge and discharge, for example) as well as interesting AC exceptions (current "through" a capacitor), so a hard-literal interpretation of KCL may cause confusion later one.  However, it is a fairly intuitive Law to grasp, and consequently I seldom find students experiencing difficulty with it.

%INDEX% Kirchhoff's Current Law

%(END_NOTES)


