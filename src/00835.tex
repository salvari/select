
%(BEGIN_QUESTION)
% Copyright 2003, Tony R. Kuphaldt, released under the Creative Commons Attribution License (v 1.0)
% This means you may do almost anything with this work of mine, so long as you give me proper credit

Interpret this AC motor control circuit diagram, explaining the meaning of each symbol:

$$\epsfbox{00835x01.eps}$$

Also, explain the operation of this motor control circuit.  What happens when someone actuates the "Run" switch?  What happens when they let go of the "Run" switch?

\underbar{file 00835}
%(END_QUESTION)





%(BEGIN_ANSWER)

In this circuit, the motor will start once the "Run" switch is actuated.  When the "Run" switch is released, the motor continues to run.

\vskip 10pt

Follow-up question: this circuit has no "stop" switch!  What would have to be modified in the ladder logic circuit to provide "stop" control?

%(END_ANSWER)





%(BEGIN_NOTES)

This circuit is known as a {\it latching} circuit, because it "latches" in the "on" state after a momentary action.  The contact in parallel with the "Run" switch is often referred to as a {\it seal-in contact}, because it "seals" the momentary condition of the Run switch closure after that switch is de-actuated.

The follow-up question of how we may make the motor stop running is a very important one.  Spend time with your students discussing this practical design problem, and implement a solution.

%INDEX% Control circuit, AC motor
%INDEX% Ladder logic diagram
%INDEX% Seal-in contact, motor control circuit

%(END_NOTES)


