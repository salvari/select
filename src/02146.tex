
%(BEGIN_QUESTION)
% Copyright 2004, Tony R. Kuphaldt, released under the Creative Commons Attribution License (v 1.0)
% This means you may do almost anything with this work of mine, so long as you give me proper credit

Optically-isolated TRIACs are available for use as {\it solid-state relays}, suitable for replacing electromechanical relays in many AC power switching applications:

$$\epsfbox{02146x01.eps}$$

Describe some of the advantages of using a solid-state relay for switching AC power instead of using an electromechanical relay as shown here:

$$\epsfbox{02146x02.eps}$$

Also describe any disadvantages to using a solid-state relay, if they exist.

\underbar{file 02146}
%(END_QUESTION)





%(BEGIN_ANSWER)

\medskip
\goodbreak
\item{} {\bf Advantages}
\item{$\bullet$} Less DC drive current required
\item{$\bullet$} No moving parts to wear
\item{$\bullet$} Zero-crossing turn-off naturally provided by the TRIAC
\item{$\bullet$} {\it Any others you can think of . . . ?}
\medskip

\medskip
\goodbreak
\item{} {\bf Disadvantages}
\item{$\bullet$} "Off" state not as high-impedance as an electromechanical relay
\item{$\bullet$} Susceptible to ${dv \over dt}$-induced turn-on
\item{$\bullet$} {\it Any others you can think of . . . ?}
\medskip

\vskip 10pt

Follow-up question: what is {\it zero-crossing turn-off}, and what type of load might benefit most from this feature?

%(END_ANSWER)





%(BEGIN_NOTES)

It should be noted that the label "solid-state relay" is not exclusively reserved for opto-TRIAC devices.  Many different types of solid-state relays exist, including opto-BJT, opto-FET, and opto-SCR.  Be sure to mention this to your students.

%INDEX% Opto-TRIAC
%INDEX% Power control circuit, opto-TRIAC
%INDEX% Solid state relay

%(END_NOTES)


