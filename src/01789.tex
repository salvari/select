
%(BEGIN_QUESTION)
% Copyright 2003, Tony R. Kuphaldt, released under the Creative Commons Attribution License (v 1.0)
% This means you may do almost anything with this work of mine, so long as you give me proper credit

One context in which to understand Lenz's Law is the well-known physical law called the Conservation of Energy, which states that energy can neither be created (from nothing) nor destroyed (to nothing).  This well-founded law of physics is the general principle forbidding so-called "over-unity" or "free energy" machines, where energy would supposedly be produced from no source whatsoever.

Demonstrate that if Lenz's Law were reversed, the Conservation of Energy principle would be violated.  In other words, imagine what would happen if the effects of Lenz's Law were exactly opposite in direction, and show how this would result in more energy produced by a system than what is input to that system.

\underbar{file 01789}
%(END_QUESTION)





%(BEGIN_ANSWER)

There are several ways to demonstrate this.  Perhaps the easiest to visualize (from an energy perspective) is a rotary magnetic "drag" disk, where the perpendicular intersection of a magnetic field and an electrically conductive disk creates a resistive drag (opposing) torque when the disk is rotated.  The effects of reversing the Lenz force direction should be obvious here.

%(END_ANSWER)





%(BEGIN_NOTES)

This question may very well lead to a fruitful discussion on perpetual motion and claims of "free energy" machines, the very existence of such claims in modern times being outstanding evidence of scientific illiteracy.  Not only do a substantial number of people seem ignorant of the Conservation of Energy principle and just how well it is founded, but also seem unable to grasp the importance of the ultimate test for such a device: to be able to power itself (and a load) indefinitely.  But I digress . . .

%INDEX% Lenz's Law, relationship to conservation of energy

%(END_NOTES)


