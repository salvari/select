
%(BEGIN_QUESTION)
% Copyright 2003, Tony R. Kuphaldt, released under the Creative Commons Attribution License (v 1.0)
% This means you may do almost anything with this work of mine, so long as you give me proper credit

There is more than one way to electrically "brake" (slow down) an electric motor.  Three methods in common use are:

\medskip
\item{$\bullet$} Dynamic braking
\item{$\bullet$} Regenerative braking
\item{$\bullet$} Plugging
\medskip

Describe how each of these methods work.

\underbar{file 00697}
%(END_QUESTION)





%(BEGIN_ANSWER)

Dynamic braking is re-connecting the motor as a generator, and dissipating the generated power into a resistive load.  Regenerative braking is an extension of dynamic braking, using the generated power to do something useful.  Plugging is the temporary application of {\it reverse} polarity to the motor to forcefully stop it.

%(END_ANSWER)





%(BEGIN_NOTES)

All three methods of motor braking are used in industry, each with its own benefits.  Discuss the relative merits of each method with your students, with regard to simplicity, braking power, power consumption, etc.  

Be sure to discuss the advantages and disadvantages of electrical braking over mechanical braking.  What advantage(s) does electrical braking (using the motor as a brake) have over mechanical braking (using a separate brake mechanism attached to the motor shaft)?  Which type of braking do you think might be more reliable?

%INDEX% Braking techniques, electric motor
%INDEX% Motor braking techniques

%(END_NOTES)


