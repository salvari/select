
%(BEGIN_QUESTION)
% Copyright 2003, Tony R. Kuphaldt, released under the Creative Commons Attribution License (v 1.0)
% This means you may do almost anything with this work of mine, so long as you give me proper credit

If a sinusoidal voltage is applied to an impedance with a phase angle of 90$^{o}$, the resulting voltage and current waveforms will look like this:

$$\epsfbox{00632x01.eps}$$

Given that power is the product of voltage and current ($p = i e$), plot the waveform for power in this circuit.  Also, explain how the mnemonic phrase {\it "ELI the ICE man"} applies to these waveforms.

\underbar{file 00632}
%(END_QUESTION)





%(BEGIN_ANSWER)

$$\epsfbox{00632x02.eps}$$

The mnemonic phrase, "ELI the ICE man" indicates that this phase shift is due to an inductance rather than a capacitance.

%(END_ANSWER)





%(BEGIN_NOTES)

Ask your students to observe the waveform shown in the answer closely, and determine what {\it sign} the power values are.  Note how the power waveform alternates between positive and negative values, just as the voltage and current waveforms do.  Ask your students to explain what {\it negative} power could possibly mean.

Of what significance is this to us?  What does this indicate about the nature of a load with an impedance phase angle of 90$^{o}$?

\vskip 10pt

The phrase, "ELI the ICE man" has been used be generations of technicians to remember the phase relationships between voltage and current for inductors and capacitors, respectively.  One area of trouble I've noted with students, though, is being able to interpret which waveform is leading and which one is lagging, from a time-domain plot such as this.

%INDEX% "ELI the ICE man" mnemonic
%INDEX% Power, instantaneous, in AC circuit with phase shift between V and I

%(END_NOTES)


