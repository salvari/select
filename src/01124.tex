
%(BEGIN_QUESTION)
% Copyright 2003, Tony R. Kuphaldt, released under the Creative Commons Attribution License (v 1.0)
% This means you may do almost anything with this work of mine, so long as you give me proper credit

The following amplifier circuit has a problem.  Despite the presence of a strong input signal (as verified by an oscilloscope measurement at TP1), there is no sound coming from the speaker:

$$\epsfbox{01124x01.eps}$$

\vskip 10pt

Explain a logical, step-by-step approach to identifying the source of the problem, by taking voltage signal measurements.  Remember, the more efficient your troubleshooting technique is (the fewer measurements taken), the better!

\underbar{file 01124}
%(END_QUESTION)





%(BEGIN_ANSWER)

I'll let you have fun determining your own strategies here!

%(END_ANSWER)





%(BEGIN_NOTES)

This question can easily occupy a large portion of your discussion time, so be sure to make room for it in your schedule!

A great way to help students grasp the concepts involved in this circuit as well as improve their troubleshooting technique, is to make a large-scale demonstration board of this circuit, which you can fault on the back side by disconnecting wires, opening or closing switches, etc.  Then, have the students take an oscilloscope and practice finding problems in the circuit by voltage measurement only.  I have built similar demonstration boards for my own classroom, and have found them to be extremely useful in building and assessing troubleshooting skills.

%INDEX% Amplifier, multiple-stage
%INDEX% Troubleshooting, amplifier

%(END_NOTES)


