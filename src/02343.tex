
%(BEGIN_QUESTION)
% Copyright 2005, Tony R. Kuphaldt, released under the Creative Commons Attribution License (v 1.0)
% This means you may do almost anything with this work of mine, so long as you give me proper credit

There is a problem somewhere in this electronic ignition circuit.  The "coil" does not output high voltage as it should when the reluctor spins.  A mechanic already changed the coil and replaced it with a new one, but this did not fix the problem.

$$\epsfbox{02343x01.eps}$$

You are then asked to look at the circuit to see if you can figure out what is wrong.  Using your multimeter, you measure voltage between TP1 and ground (12 volts DC) and also between TP2 and ground (0 volts DC).  These voltage readings do not change at all as the reluctor spins.

From this information, identify two possible faults that could account for the problem and all measured values in this circuit, and also identify two circuit elements that could not possibly be to blame (i.e. two things that you know {\it must} be functioning properly, no matter what else may be faulted).  The circuit elements you identify as either possibly faulted or properly functioning can be wires, traces, and connections as well as components.  Be as specific as you can in your answers, identifying both the circuit element and the type of fault.

\medskip
\goodbreak
\item{$\bullet$} Circuit elements that are possibly faulted
\item{1.}
\item{2.} 
\medskip

\medskip
\goodbreak
\item{$\bullet$} Circuit elements that must be functioning properly
\item{1.} 
\item{2.} 
\medskip

\underbar{file 02343}
%(END_QUESTION)





%(BEGIN_ANSWER)

Note: the following answers are not exhaustive.  There may be more circuit elements possibly at fault and more circuit elements known to be functioning properly!

\medskip
\goodbreak
\item{$\bullet$} Circuit elements that are possibly faulted
\item{1.} Transistor failed shorted between collector and emitter
\item{2.} Capacitor failed shorted
\medskip

\medskip
\goodbreak
\item{$\bullet$} Circuit elements that must be functioning properly
\item{1.} Battery
\item{2.} Ignition switch
\medskip

%(END_ANSWER)





%(BEGIN_NOTES)

Ask your students to identify means by which they could confirm suspected circuit elements, by measuring something other than what has already been measured.

Troubleshooting scenarios are always good for stimulating class discussion.  Be sure to spend plenty of time in class with your students developing efficient and logical diagnostic procedures, as this will assist them greatly in their careers.

%INDEX% Electronic ignition system, engine
%INDEX% Ignition system (electronic), engine
%INDEX% Transistor switch circuit (BJT)
%INDEX% Troubleshooting, electronic engine ignition system (BJT switch)

%(END_NOTES)


