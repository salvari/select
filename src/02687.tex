
%(BEGIN_QUESTION)
% Copyright 2005, Tony R. Kuphaldt, released under the Creative Commons Attribution License (v 1.0)
% This means you may do almost anything with this work of mine, so long as you give me proper credit

Examine this progression of mathematical statements:

$$({1000})^2 = 1000000$$

$$({1000})^2 = 10^6$$

$$\log [({1000})^2] = \log 10^6$$

$$(2)(\log 1000) = \log 10^6$$

$$(2)(\log 10^3) = \log 10^6$$

$$(2)(3) = 6$$

What began as an exponential problem ended up as a multiplication problem, through the application of logarithms.  What does this tell you about the utility of logarithms as an arithmetic tool?

\underbar{file 02687}
%(END_QUESTION)





%(BEGIN_ANSWER)

That logarithms can reduce the complexity of an equation from exponentiation, down to multiplication, indicates its usefulness as a tool to {\it simplify} arithmetic problems.  Specifically, the logarithm of a number raised to a power is equal to that power multiplied by the logarithm of the number.

%(END_ANSWER)





%(BEGIN_NOTES)

In mathematics, any procedure that reduces a complex type of problem into a simpler type of problem is called a {\it transform function}, and logarithms are one of the simplest types of transform functions in existence.

%INDEX% Logarithms, used to transform a power problem into a multiplication problem

%(END_NOTES)


