
%(BEGIN_QUESTION)
% Copyright 2003, Tony R. Kuphaldt, released under the Creative Commons Attribution License (v 1.0)
% This means you may do almost anything with this work of mine, so long as you give me proper credit

What does it mean if a meter movement is described as being {\it RMS indicating, average responding}?

\underbar{file 00762}
%(END_QUESTION)





%(BEGIN_ANSWER)

It means that the meter movement's indication is naturally proportional to the {\it average} value of the measured AC signal, but its calibration is skewed to represent RMS value when measuring a sinusoidal signal.

\vskip 10pt

Challenge question: if one of these meters is subjected to a square-wave AC signal, will its "RMS" indication be falsely low, falsely high, or accurate?

%(END_ANSWER)





%(BEGIN_NOTES)

This concept is confusing to many students, yet important for them to understand.  I suggest you illuminate the subject by asking a series of questions:

\medskip
\item{$\bullet$} What would one of these meters indicate if the amplitude of a sinusoidal signal were doubled?  {\it Answer: the indication would double.}
\item{$\bullet$} What would one of these meters indicate if the signal amplitude and wave-shape were to change in such a way that the average value of the signal doubled, but the RMS value did not increase as much?  {\it Answer: the indication would double.}
\item{$\bullet$} What would one of these meters indicate if the signal amplitude and wave-shape were to change in such a way that the RMS value of the signal doubled, but the average value did not increase as much?  {\it Answer: the indication would increase only as much as the average value increased.}
\medskip

%(END_NOTES)


