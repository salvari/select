
%(BEGIN_QUESTION)
% Copyright 2005, Tony R. Kuphaldt, released under the Creative Commons Attribution License (v 1.0)
% This means you may do almost anything with this work of mine, so long as you give me proper credit

Note the following logarithmic identities, using the "common" (base 10) logarithm:

$$\log 10 = 1$$

$$\log 100 = 2$$

$$\log 1000 = 3$$

$$\log 10000 = 4$$

In the first equation, the numbers 10 and 1 were related together by the log function.  In the second equation, the numbers 100 and 2 were related together by the same log function, and so on.

Rewrite the four equations together in such a way that the same numbers are related to each other, but without writing "log".  In other words, represent the same mathematical relationships using some mathematical function other than the common logarithm function.

\underbar{file 02680}
%(END_QUESTION)





%(BEGIN_ANSWER)

$$10^1 = 10$$

$$10^2 = 100$$

$$10^3 = 1000$$

$$10^4 = 10000$$

%(END_ANSWER)





%(BEGIN_NOTES)

An illustration like this helps students comprehend what the "log" function actually does.

%INDEX% Exponents and logarithms, as inverse functions 
%INDEX% Logarithms and exponents, as inverse functions 

%(END_NOTES)


