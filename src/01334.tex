
%(BEGIN_QUESTION)
% Copyright 2003, Tony R. Kuphaldt, released under the Creative Commons Attribution License (v 1.0)
% This means you may do almost anything with this work of mine, so long as you give me proper credit

Write an SOP expression for this truth table, and then draw a ladder logic (relay) circuit diagram corresponding to that SOP expression:

$$\epsfbox{01334x01.eps}$$

Implement the SOP logic function using contacts of relays CR1, CR2, and CR3.  A partial ladder logic diagram has been provided for you.

Finally, simplify this expression using Boolean algebra, and draw a simplified ladder logic diagram based on this new (reduced) Boolean expression.  When deciding "how far" to reduce the Boolean expression, choose a form that results in the minimum number of relay contacts in the simplified ladder logic diagram.

\underbar{file 01334}
%(END_QUESTION)





%(BEGIN_ANSWER)

Original SOP expression and relay circuit:

$$\epsfbox{01334x02.eps}$$

\vskip 10pt

Reduced expression and relay circuit:

$$\epsfbox{01334x03.eps}$$

%(END_ANSWER)





%(BEGIN_NOTES)

Discuss with your students the utility of Boolean algebra as a circuit simplification tool.  Ask your students to compare the original and reduced logic gate circuits, and comment on such performance metrics as reliability, power consumption, maximum operating speed, etc.

%INDEX% Boolean algebra, conversion of expression into relay logic
%INDEX% Boolean algebra, relay circuit simplification
%INDEX% Sum-of-Products expression, Boolean algebra (generated from a truth table)
%INDEX% SOP expression, Boolean algebra (generated from a truth table)

%(END_NOTES)


