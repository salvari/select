
%(BEGIN_QUESTION)
% Copyright 2005, Tony R. Kuphaldt, released under the Creative Commons Attribution License (v 1.0)
% This means you may do almost anything with this work of mine, so long as you give me proper credit

Plot the equation $y = x^2$ on the following graph:

$$\epsfbox{03101x01.eps}$$

On the same graph, plot the equation $y = x + 2$.  What is the significance of the point where the two plots cross?

\underbar{file 03101}
%(END_QUESTION)





%(BEGIN_ANSWER)

$$\epsfbox{03101x02.eps}$$

Here there are {\it two} points of intersection between the parabola (curve) and the straight line, representing two different solution sets that satisfy {\it both} equations.

\vskip 10pt

Challenge question: solve this simultaneous system of equations without graphing, but by symbolically manipulating the equations!

%(END_ANSWER)





%(BEGIN_NOTES)

Here, solution by graphing may be a bit easier than the symbolic solution.  In principle we may determine solutions for {\it any} pair of equations by graphing, with about equal difficulty.  The only real problem is precision: how closely we may interpret to points of intersection.  A practical example of non-linear simultaneous function solution is {\it load line} analysis in semiconductor circuitry.

%INDEX% Algebra, graphing simple functions
%INDEX% Graphing simple functions, algebra
%INDEX% Simultaneous equations
%INDEX% Systems of nonlinear equations

%(END_NOTES)


