
%(BEGIN_QUESTION)
% Copyright 2003, Tony R. Kuphaldt, released under the Creative Commons Attribution License (v 1.0)
% This means you may do almost anything with this work of mine, so long as you give me proper credit

{\it Boolean algebra} is a strange sort of math.  For example, the complete set of rules for Boolean addition is as follows:

$$0 + 0 = 0$$

$$0 + 1 = 1$$

$$1 + 0 = 1$$

$$1 + 1 = 1$$

Suppose a student saw this for the very first time, and was quite puzzled by it.  What would you say to him or her as an explanation for this?  How in the world can $1 + 1 = 1$ and not 2?  And why are there no more rules for Boolean addition?  Where is the rule for $1 + 2$ or $2 + 2$?

\underbar{file 01297}
%(END_QUESTION)





%(BEGIN_ANSWER)

Boolean quantities can only have one out of two possible values: either 0 or 1.  There is no such thing as "2" in the set of Boolean numbers.

%(END_ANSWER)





%(BEGIN_NOTES)

Boolean algebra is a strange math, indeed.  However, once students understand the limited scope of Boolean quantities, the rationale for Boolean rules of arithmetic make sense.  1 + 1 {\it must} equal 1, because there is no such thing as "2" in the Boolean world, and the answer certainly can't be 0.

%INDEX% Boolean algebra, rules of addition

%(END_NOTES)


