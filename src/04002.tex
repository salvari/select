
%(BEGIN_QUESTION)
% Copyright 2006, Tony R. Kuphaldt, released under the Creative Commons Attribution License (v 1.0)
% This means you may do almost anything with this work of mine, so long as you give me proper credit

A two-conductor cable of uniform construction will exhibit a uniform {\it characteristic impedance} ($Z_0$) due to its intrinsic, distributed inductance and capacitance:

$$\epsfbox{04002x01.eps}$$

What would happen to the value of this characteristic impedance if we were to shorten the cable's length, all other dimensions remaining the same?

$$\epsfbox{04002x02.eps}$$

\underbar{file 04002}
%(END_QUESTION)





%(BEGIN_ANSWER)

$Z_0$ would remain exactly the same!

\vskip 10pt

Follow-up question: what electrical characteristics {\it would} change for this shortened cable?

%(END_ANSWER)





%(BEGIN_NOTES)

This is sort of a "trick" question, designed to make students {\it think} about characteristic impedance, and to test their real comprehension of it.  If a student properly understands the physics resulting in characteristic impedance, they will realize length has nothing whatsoever to do with it.  Although the cable's total capacitance will change as a result of shortening the cable's length, and the cable's total inductance will likewise decrease for the same reason, these electrical changes should not present a conceptual difficulty to students unless they are modeling the cable in terms of {\it one} lumped capacitance and {\it one} (or two) lumped inductance(s).  If they are thinking in these terms, they have not yet fully grasped the reason why characteristic impedance exists at all.

%INDEX% Transmission line, dimensional effects on characteristic impedance
%INDEX% Characteristic impedance, effects of changing cable dimensions on
%INDEX% Surge impedance, effects of changing cable dimensions on

%(END_NOTES)


