
%(BEGIN_QUESTION)
% Copyright 2006, Tony R. Kuphaldt, released under the Creative Commons Attribution License (v 1.0)
% This means you may do almost anything with this work of mine, so long as you give me proper credit

If you have studied complex numbers, you know that the same complex quantity may be written in two different forms: {\it rectangular} and {\it polar}.  Take for example the complex quantity ${\sqrt{3} \over 2} + j{1 \over 2}$.  The following illustration shows this point located on the complex plane, along with its rectangular dimensions:

$$\epsfbox{04059x01.eps}$$

\vskip 10pt

Next, we see the same point, on the same complex plane, along with its polar coordinates:

$$\epsfbox{04059x02.eps}$$

\vskip 10pt

Written out, we might express the equivalence of these two notations as such:

$${\sqrt{3} \over 2} + j{1 \over 2} = 1 \angle {\pi \over 6}$$

\vskip 30pt

\goodbreak
Expressed in a more general form, the equivalence between rectangular and polar notations would look like this:

$$a + jb = c \angle \Theta$$

However, a problem with the "angle" symbol ($\angle$) is that we have no standardized way to deal with it mathematically.  We would have to invent special rules to describe how to add, subtract, multiply, divide, differentiate, integrate, or otherwise manipulate complex quantities expressed using this symbol.  A more profitable alternative to using the "angle" symbol is shown here:

$$a + jb = c e^{j\Theta}$$

Explain why this equivalence is mathematically sound.

\underbar{file 04059}
%(END_QUESTION)





%(BEGIN_ANSWER)

The equivalence shown is based on {\it Euler's relation}, which is left to you as an exercise to prove.

%(END_ANSWER)





%(BEGIN_NOTES)

This question should probably be preceded by \#04058, which asks students to explore the relationship between the infinite series for $e^x$, $\cos x$, and $\sin x$.  In any case, your students will need to know Euler's relation:

$$e^{jx} = \cos x + j \sin x$$

%INDEX% Polar notation, as a complex exponential

%(END_NOTES)


