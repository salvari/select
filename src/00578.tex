
%(BEGIN_QUESTION)
% Copyright 2003, Tony R. Kuphaldt, released under the Creative Commons Attribution License (v 1.0)
% This means you may do almost anything with this work of mine, so long as you give me proper credit

As a general rule, inductors oppose change in ({\bf choose:} \underbar{voltage} or \underbar{current}), and they do so by . . . (complete the sentence).

\vskip 10pt

Based on this rule, determine how an inductor would react to a constant AC current that increases in frequency.  Would an inductor drop more or less voltage, given a greater frequency?  Explain your answer.

\underbar{file 00578}
%(END_QUESTION)





%(BEGIN_ANSWER)

As a general rule, inductors oppose change in \underbar{current}, and they do so by producing a voltage.

\vskip 10pt

An inductor will drop a greater amount of AC voltage, given the same AC current, at a greater frequency.

%(END_ANSWER)





%(BEGIN_NOTES)

This question is an exercise in qualitative thinking: relating rates of change to other variables, without the use of numerical quantities.  The general rule stated here is very, very important for students to master, and be able to apply to a variety of circumstances.  If they learn nothing about inductors except for this rule, they will be able to grasp the function of a great many inductor circuits.

%INDEX% Inductive reactance, relationship to frequency
%INDEX% Reactance, inductive (relationship to frequency)

%(END_NOTES)


