
%(BEGIN_QUESTION)
% Copyright 2003, Tony R. Kuphaldt, released under the Creative Commons Attribution License (v 1.0)
% This means you may do almost anything with this work of mine, so long as you give me proper credit

During the early development of telephone technology, a unit was invented for representing power gain (or loss) in an electrical system.  It was called the {\it Bel}, in honor of Alexander Graham Bell, the telecommunications pioneer.

"Bels" relate to power gain ratios by the following equation:

$$A_{P(ratio)} = 10^{A_{P(Bels)}}$$

Given this mathematical relationship, translate these power gain figures given in units of Bels, into ratios:

\medskip
\item{$\bullet$} $A_P =$ 3 B ; $A_P =$
\vskip 5pt
\item{$\bullet$} $A_P =$ 2 B ; $A_P =$
\vskip 5pt
\item{$\bullet$} $A_P =$ 1 B ; $A_P =$
\vskip 5pt
\item{$\bullet$} $A_P =$ 0 B ; $A_P =$
\vskip 5pt
\item{$\bullet$} $A_P =$ -1 B ; $A_P =$
\vskip 5pt
\item{$\bullet$} $A_P =$ -2 B ; $A_P =$
\vskip 5pt
\item{$\bullet$} $A_P =$ -3 B ; $A_P =$
\medskip

\underbar{file 00675}
%(END_QUESTION)





%(BEGIN_ANSWER)

\medskip
\item{$\bullet$} $A_P =$ 3 B ; $A_P =$ 1000
\vskip 5pt
\item{$\bullet$} $A_P =$ 2 B ; $A_P =$ 100
\vskip 5pt
\item{$\bullet$} $A_P =$ 1 B ; $A_P =$ 10
\vskip 5pt
\item{$\bullet$} $A_P =$ 0 B ; $A_P =$ 1
\vskip 5pt
\item{$\bullet$} $A_P =$ -1 B ; $A_P =$ $1 \over 10$
\vskip 5pt
\item{$\bullet$} $A_P =$ -2 B ; $A_P =$ $1 \over 100$
\vskip 5pt
\item{$\bullet$} $A_P =$ -3 B ; $A_P =$ $1 \over 1000$
\medskip

\vskip 10pt

Follow-up question: a geologist, taking a class on electronics, sees this mathematical pattern and remarks, "This is just like the {\it Richter} scale!"  Explain what the geologist means.

%(END_ANSWER)





%(BEGIN_NOTES)

Ask your students how these two systems of power gain expression (Bels versus ratios) compare in terms of {\it range}.  Which system of expression encompasses the greatest range of power gains or losses, with the smallest changes in numerical value?

%INDEX% Bel, defined as unit of power gain
%INDEX% Gain, converting power gains in Bels to power gain ratios

%(END_NOTES)


