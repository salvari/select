
%(BEGIN_QUESTION)
% Copyright 2003, Tony R. Kuphaldt, released under the Creative Commons Attribution License (v 1.0)
% This means you may do almost anything with this work of mine, so long as you give me proper credit

One analogy used to explain and contrast negative feedback versus positive feedback is that of a round stone, placed on either a hilltop or a valley:

$$\epsfbox{01149x01.eps}$$

The stability of the stone in each of these scenarios represents the stability of a specific type of electrical feedback system.  Which of these scenarios represents negative feedback, which represents positive feedback, and why?

\underbar{file 01149}
%(END_QUESTION)





%(BEGIN_ANSWER)

The valley represents negative feedback, while the hill represents positive feedback.

%(END_ANSWER)





%(BEGIN_NOTES)

I have found this simple analogy to be most helpful when explaining feedback systems to students, because the behavior of each is intuitively obvious.

%INDEX% Negative feedback versus positive feedback
%INDEX% Positive feedback versus negative feedback

%(END_NOTES)


