
%(BEGIN_QUESTION)
% Copyright 2003, Tony R. Kuphaldt, released under the Creative Commons Attribution License (v 1.0)
% This means you may do almost anything with this work of mine, so long as you give me proper credit

$$\epsfbox{01986x01.eps}$$

\underbar{file 01986}
\vfil \eject
%(END_QUESTION)





%(BEGIN_ANSWER)

Use circuit simulation software to verify your predicted and measured parameter values.

%(END_ANSWER)





%(BEGIN_NOTES)

Any diodes will work for this, so long as the source frequency is not too high.

I have had good success with the following values:

\medskip
\item{$\bullet$} $V_{source}$ = 4 volts (peak)
\item{$\bullet$} $f_{source}$ = 3 kHz
\item{$\bullet$} $V_{DC}$ = 6 volts
\item{$\bullet$} $C_1$ = 0.47 $\mu$F
\item{$\bullet$} $R_1$ = 100 k$\Omega$
\item{$\bullet$} Potentiometer = 10 k$\Omega$, linear
\item{$\bullet$} $D_1$ = part number 1N4004 (any 1N400x diode should work)
\medskip

An extension of this exercise is to incorporate troubleshooting questions.  Whether using this exercise as a performance assessment or simply as a concept-building lab, you might want to follow up your students' results by asking them to predict the consequences of certain circuit faults.

%INDEX% Assessment, performance-based (Diode clipper circuit, adjustable)

%(END_NOTES)


