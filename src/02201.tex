
%(BEGIN_QUESTION)
% Copyright 2004, Tony R. Kuphaldt, released under the Creative Commons Attribution License (v 1.0)
% This means you may do almost anything with this work of mine, so long as you give me proper credit

Label where each of the following electrical quantities would be found in both the "Y" and "Delta" three-phase configurations:

\medskip
\goodbreak
\item{$\bullet$} Phase voltage
\item{$\bullet$} Line voltage
\item{$\bullet$} Phase current
\item{$\bullet$} Line current
\medskip

$$\epsfbox{02201x01.eps}$$

In which circuit (Y or Delta) are the phase and line currents equal?  In which circuit (Y or Delta) are the phase and line voltages equal?  Explain both answers, in terms that anyone with a basic knowledge of electricity could understand.

Where phase and line quantities are {\it un}equal, determine which is larger.

\underbar{file 02201}
%(END_QUESTION)





%(BEGIN_ANSWER)

$$\epsfbox{02201x02.eps}$$

\medskip
\goodbreak
\item{} {\bf Y configuration} 
\item{$\bullet$} $I_{phase} = I_{line}$
\item{$\bullet$} $V_{phase} < V_{line}$ 
\medskip

\medskip
\goodbreak
\item{} {\bf Delta configuration} 
\item{$\bullet$} $V_{phase} = V_{line}$ 
\item{$\bullet$} $I_{phase} < I_{line}$
\medskip

\vskip 10pt

Follow-up question: how do Kirchhoff's Voltage and Current Laws explain the relationships between unequal quantities in "Y" and "Delta" configurations?

%(END_ANSWER)





%(BEGIN_NOTES)

Your students will need to know what "phase" and "line" represents in both types of polyphase configurations, especially when using formulae that reference quantities by these labels.  

%INDEX% Line current vs. phase current
%INDEX% Line voltage vs. phase voltage
%INDEX% Phase current vs. line current
%INDEX% Phase voltage vs. line voltage

%(END_NOTES)


