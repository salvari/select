
%(BEGIN_QUESTION)
% Copyright 2005, Tony R. Kuphaldt, released under the Creative Commons Attribution License (v 1.0)
% This means you may do almost anything with this work of mine, so long as you give me proper credit

In the following circuit, an adjustable voltage source is connected in series with a resistive load and another voltage source:

$$\epsfbox{03224x01.eps}$$

Determine what will happen to the current in this circuit if the adjustable voltage source is increased.

\vskip 10pt

In this next circuit, an adjustable voltage source is connected in series with a resistive load and a {\it current} source:

$$\epsfbox{03224x02.eps}$$

Now determine what will happen to the current in this second circuit if the adjustable voltage source is increased.

\vskip 10pt

One way to define electrical resistance is by comparing the {\it change} in applied voltage ($\Delta V$) to the {\it change} in resultant current ($\Delta I$).  This is mathematically expressed by the following ratio:

$$R = {\Delta V \over \Delta I}$$

From the perspective of the adjustable voltage source ($V_{adjust}$), and as defined by the above equation, which of these two circuits has the greatest resistance?  What does this result suggest about the equivalent resistance of a constant-voltage source versus the equivalent resistance of a constant-current source?

\underbar{file 03224}
%(END_QUESTION)





%(BEGIN_ANSWER)

In the first circuit, current will increase as $V_{adjust}$ is increased, yielding a finite total resistance.  In the second circuit, current will remain constant as $V_{adjust}$ is increased, yielding an infinite total resistance.

\vskip 10pt

Follow-up question: calculate $R$ as defined by the formula ${\Delta V \over \Delta I}$ for these two circuits, assuming $V_{adjust}$ changes from 15 volts to 16 volts (1 volt $\Delta V$):

$$\epsfbox{03224x03.eps}$$

%(END_ANSWER)





%(BEGIN_NOTES)

If students are unable to analyze the two circuits qualitatively as suggested in the question, the follow-up question should clear things up.  The point of all this, of course, is for students to see that a constant voltage source has zero internal resistance and that a constant current source has infinite internal resistance.

In case anyone should ask, the proper definition for resistance is expressed as a derivative.  That is, instead of $R = {\Delta V \over \Delta I}$ we should have $R = {dV \over dI}$.

%INDEX% Current sources, internal resistance of
%INDEX% Internal resistance of current sources
%INDEX% Internal resistance of voltage sources
%INDEX% Voltage sources, internal resistance of

%(END_NOTES)


