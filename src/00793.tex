
%(BEGIN_QUESTION)
% Copyright 2003, Tony R. Kuphaldt, released under the Creative Commons Attribution License (v 1.0)
% This means you may do almost anything with this work of mine, so long as you give me proper credit

AC-DC power supplies are a cause of harmonic currents in AC power systems, especially large AC-DC power supplies used in motor control circuits and other high-power controls.  In this example, I show the waveforms for output voltage and input current for an unloaded AC-DC power supply with a step-down transformer, full-wave rectifier, and capacitive filter circuit (the unfiltered DC voltage waveform is shown as a dashed line for reference):

$$\epsfbox{00793x01.eps}$$

As you can see, the input current waveform lags the voltage waveform by 90$^{o}$, because when the power supply is unloaded, the only input current is the {\it magnetizing current} of the transformer's primary winding.

With increased loading, the output ripple voltage becomes more pronounced.  This also changes the input current waveform significantly, making it non-sinusoidal.  Trace the shape of the input current waveform, given the output voltage waveform and magnetizing current waveform (dotted line) shown here:

$$\epsfbox{00793x02.eps}$$

The non-filtered DC output waveform is still shown as a dotted line, for reference purposes.

\underbar{file 00793}
%(END_QUESTION)





%(BEGIN_ANSWER)

$$\epsfbox{00793x03.eps}$$

\vskip 10pt

Challenge question: does the input current waveform shown here contain even-numbered harmonics (i.e. 120 Hz, 240 Hz, 360 Hz)?

%(END_ANSWER)





%(BEGIN_NOTES)

In a filtered DC power supply, the only time current is drawn from the rectifier is when the filter capacitor charges.  Thus, the only time you see input current above and beyond the magnetizing current waveform is when the capacitor voltage requires charging.

Note that although the (sinusoidal) magnetizing current waveform is 90$^{o}$ out of phase with the voltage waveform, the input current transients are precisely in-phase with the current transients on the transformer's secondary winding.  This reviews an important principle of transformers: that whatever primary current is the result of secondary winding load is in-phase with that secondary load current.  In this regard, a transformer does not act as a reactive device, but a direct power-coupling device.

Note also that after the initial surge (rising pulse edge) of current, the input current waveform follows a different curve from the voltage waveform, because $i = C{dv \over dt}$ for a capacitor.

\vskip 10pt

In case you haven't guessed by now, there is a lot of stuff happening in this circuit!  I would consider this question to be "advanced" for most introductory-level courses, and may be skipped at your discretion.

%INDEX% Harmonics, generated by AC-DC power supply circuit

%(END_NOTES)


