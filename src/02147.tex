
%(BEGIN_QUESTION)
% Copyright 2004, Tony R. Kuphaldt, released under the Creative Commons Attribution License (v 1.0)
% This means you may do almost anything with this work of mine, so long as you give me proper credit

{\it Commutation} is an important issue in any kind of thyristor circuit, due to the "latching" nature of these devices.  Explain what "commutation" means, and how it may be achieved for various thyristors.

\underbar{file 02147}
%(END_QUESTION)





%(BEGIN_ANSWER)

{\it Commutation} is nothing more than a fancy word for "switching" (think of the {\it commutator} in a DC electric motor -- its purpose being to {\it switch} polarity of voltage applied to the armature windings).  In the context of thyristors, "commutation" refers to the issue of how to turn the device(s) off after they have been triggered on.

\vskip 10pt

Follow-up question: in some circuits, commutation occurs naturally.  In other circuits, special provisions must be made to force the thyristor(s) to turn off.  Identify at least one example of a thyristor circuit with {\it natural commutation} and at least one example of a thyristor circuit using {\it forced commutation}.

%(END_ANSWER)





%(BEGIN_NOTES)

An important feature of all thyristors is that they {\it latch} in the "on" state once having been triggered.  This point needs to be emphasized multiple times for some students to grasp it, as they are accustomed to thinking in terms of transistors which do not latch.

%INDEX% Commutation of thyristor, defined
%INDEX% Forced commutation of thyristor, defined
%INDEX% Natural commutation of thyristor, defined

%(END_NOTES)


