
%(BEGIN_QUESTION)
% Copyright 2003, Tony R. Kuphaldt, released under the Creative Commons Attribution License (v 1.0)
% This means you may do almost anything with this work of mine, so long as you give me proper credit

The circuit shown here is called a "bridge rectifier," and its purpose is to convert alternating current (from the "power-supply" unit) into direct current.  Suppose you were instructed to check the continuity of the switch (SW1) mounted on the printed circuit board.  What would be a fast and effective way of testing this switch's continuity (ideally, without removing the switch from the circuit board)?

$$\epsfbox{00100x01.eps}$$

\underbar{file 00100}
%(END_QUESTION)





%(BEGIN_ANSWER)

Disconnect the power supply from the circuit board (only one wire need be disconnected), and then use an ohmmeter to measure continuity across the switch terminals when in the "ON" position and when in the "OFF" position.  Incidentally, this is not the only way to check the switch's continuity, but it is the most direct.

%(END_ANSWER)





%(BEGIN_NOTES)

Challenge your students to think of other methods which could be used to check the switch's continuity.  There is often more than one way to perform a certain check of component function, if you are knowledgeable in electrical theory and creative in your use of test equipment!

%INDEX% Switch
%INDEX% Continuity
%INDEX% Ohmmeter usage
%INDEX% Troubleshooting, simple circuit

%(END_NOTES)


