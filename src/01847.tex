
%(BEGIN_QUESTION)
% Copyright 2003, Tony R. Kuphaldt, released under the Creative Commons Attribution License (v 1.0)
% This means you may do almost anything with this work of mine, so long as you give me proper credit

Due to the effects of a changing electric field on the dielectric of a capacitor, some energy is dissipated in capacitors subjected to AC.  Generally, this is not very much, but it is there.  This dissipative behavior is typically modeled as a series-connected resistance:

$$\epsfbox{01847x01.eps}$$

Calculate the magnitude and phase shift of the current through this capacitor, taking into consideration its equivalent series resistance (ESR):

$$\epsfbox{01847x02.eps}$$

Compare this against the magnitude and phase shift of the current for an ideal 0.22 $\mu$F capacitor.

\underbar{file 01847}
%(END_QUESTION)





%(BEGIN_ANSWER)

${\bf I} =$ 3.732206 mA $\angle$ 89.89$^{o}$ for the real capacitor with ESR.

\vskip 10pt

${\bf I} =$ 3.732212 mA $\angle$ 90.00$^{o}$ for the ideal capacitor.

\vskip 10pt

Follow-up question \#1: can this ESR be detected by a DC meter check of the capacitor?  Why or why not?

\vskip 10pt

Follow-up question \#2: explain how the ESR of a capacitor can lead to physical heating of the component, especially under high-voltage, high-frequency conditions.  What safety concerns might arise as a result of this?

%(END_ANSWER)





%(BEGIN_NOTES)

Although capacitors do contain their own parasitic effects, ESR being one of them, they still tend to be much "purer" components than inductors for general use.  This is another reason why capacitors are generally favored over inductors in applications where either will suffice.

%INDEX% ESR, capacitor
%INDEX% Impedance calculation, series RC circuit

%(END_NOTES)


