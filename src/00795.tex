
%(BEGIN_QUESTION)
% Copyright 2003, Tony R. Kuphaldt, released under the Creative Commons Attribution License (v 1.0)
% This means you may do almost anything with this work of mine, so long as you give me proper credit

A technician is troubleshooting a power supply circuit with no DC output voltage.  The output voltage is supposed to be 15 volts DC:

$$\epsfbox{00795x01.eps}$$

The technician begins making voltage measurements between some of the test points (TP) on the circuit board.  What follows is a sequential record of his measurements:

\medskip
\goodbreak
\item{1.} $V_{TP9-TP10} =$ 0 volts DC
\item{2.} $V_{TP8-TP7} =$ 0 volts DC
\item{3.} $V_{TP8-TP5} =$ 0 volts DC
\item{4.} $V_{TP6-TP7} =$ 0 volts DC
\item{5.} $V_{TP4-TP5} =$ 0 volts AC
\item{6.} $V_{TP1-TP3} =$ 0 volts AC
\item{7.} $V_{TP1-TP2} =$ 116 volts AC
\medskip

Based on these measurements, what do you suspect has failed in this supply circuit?  Explain your answer.  Also, critique this technician's troubleshooting technique and make your own suggestions for a more efficient pattern of steps.

\underbar{file 00795}
%(END_QUESTION)





%(BEGIN_ANSWER)

The fuse is blown open.  

\vskip 10pt

Follow-up question: with regard to the troubleshooting technique, this technician seems to have started from one end of the circuit and moved incrementally toward the other, checking voltage at almost every point in between.  Can you think of a more efficient strategy than to start at one end and work slowly toward the other?

%(END_ANSWER)





%(BEGIN_NOTES)

Troubleshooting scenarios are always good for stimulating class discussion.  Be sure to spend plenty of time in class with your students developing efficient and logical diagnostic procedures, as this will assist them greatly in their careers.

%INDEX% Troubleshooting, power supply (AC-DC)

%(END_NOTES)


