
%(BEGIN_QUESTION)
% Copyright 2003, Tony R. Kuphaldt, released under the Creative Commons Attribution License (v 1.0)
% This means you may do almost anything with this work of mine, so long as you give me proper credit

What will happen when the pushbutton switch is actuated in this circuit?

$$\epsfbox{00284x01.eps}$$

\underbar{file 00284}
%(END_QUESTION)





%(BEGIN_ANSWER)

The light bulb will de-energize when the pushbutton switch is actuated.

\vskip 10pt

Follow-up question: is the contact inside the relay {\it normally-open} or is it {\it normally-closed}?

%(END_ANSWER)





%(BEGIN_NOTES)

There is a sequence of events to the final result of the pushbutton's actuation.  Be sure to ask you students to explain all the steps, from beginning to end, of this relay circuit's operation.  Test their comprehension of this circuit, to ensure they fully understand what is taking place.

A logical question your students may ask is, "What is the point?"  After all, a circuit with no relay at all (just a switch, battery, and lamp) could accomplish the same task!  What is the point of having an extra battery and this device called a relay?  Resist the temptation to tell them why, and let them figure out some possible reasons for using a relay.

%INDEX% Relay circuit, simple
%INDEX% Normally open contact
%INDEX% Normally closed contact

%(END_NOTES)


