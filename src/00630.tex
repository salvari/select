
%(BEGIN_QUESTION)
% Copyright 2003, Tony R. Kuphaldt, released under the Creative Commons Attribution License (v 1.0)
% This means you may do almost anything with this work of mine, so long as you give me proper credit

A mechanic visits you one day, carrying a large wrench.  She says the wrench became magnetized after setting it near a large magnet.  Now the wrench has become an annoyance, attracting all the other tools in her toolbox toward it.  Can you think of a way to {\it demagnetize} the wrench for her?

\underbar{file 00630}
%(END_QUESTION)





%(BEGIN_ANSWER)

An object may be de-magnetized by careful magnetization in the opposite direction.  Incidentally, this is generally {\it not} how de-magnetization is done in industry, but it is valid.

%(END_ANSWER)





%(BEGIN_NOTES)

Students may want to test their hypotheses on magnetized paper clips, which are easy to work with and quite inexpensive.  It will be interesting to see how many of your students actually researched modern de-magnetization techniques in preparation for answering this question.

%INDEX% Demagnetization

%(END_NOTES)


