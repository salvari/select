
%(BEGIN_QUESTION)
% Copyright 2005, Tony R. Kuphaldt, released under the Creative Commons Attribution License (v 1.0)
% This means you may do almost anything with this work of mine, so long as you give me proper credit

Determine the output voltage of this circuit for two different input voltage values: +4 volts, and -4 volts.  Determine the voltage at each and every node with respect to ground as part of your analysis:

$$\epsfbox{01026x01.eps}$$

Based on this data (and any other input conditions you wish to test this circuit under), describe what the function of this circuit is.

\underbar{file 01026}
%(END_QUESTION)





%(BEGIN_ANSWER)

$$\epsfbox{01026x02.eps}$$

$$\epsfbox{01026x03.eps}$$

\vskip 10pt

This circuit is a precision full-wave rectifier.

%(END_ANSWER)





%(BEGIN_NOTES)

It is much easier to analyze the behavior of this circuit with a positive input voltage than it is to analyze it with a negative input voltage!  There is a tendency for students to reach this conclusion when analyzing the circuit's behavior with a negative input voltage:

$$\epsfbox{01026x04.eps}$$

The error seems reasonable until an analysis of {\it current} is made.  If these voltages were true, Kirchhoff's Current Law would be violated at the first opamp's virtual ground:

$$\epsfbox{01026x05.eps}$$

%INDEX% Precision rectifier circuit, opamp

%(END_NOTES)


