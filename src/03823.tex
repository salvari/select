
%(BEGIN_QUESTION)
% Copyright 2005, Tony R. Kuphaldt, released under the Creative Commons Attribution License (v 1.0)
% This means you may do almost anything with this work of mine, so long as you give me proper credit

Predict how the operation of this logic gate circuit will be affected as a result of the following faults.  Consider each fault independently (i.e. one at a time, no multiple faults):

$$\epsfbox{03823x01.eps}$$

\medskip
\item{$\bullet$} Diode $D_1$ fails open:
\vskip 5pt
\item{$\bullet$} Diode $D_1$ fails shorted:
\vskip 5pt
\item{$\bullet$} Diode $D_2$ fails open:
\vskip 5pt
\item{$\bullet$} Resistor $R_1$ fails open:
\vskip 5pt
\item{$\bullet$} Resistor $R_2$ fails open:
\vskip 5pt
\item{$\bullet$} Transistor $Q_2$ emitter terminal fails open:
\vskip 5pt
\item{$\bullet$} Transistor $Q_3$ emitter terminal fails open:
\medskip

For each of these conditions, explain {\it why} the resulting effects will occur.

\underbar{file 03823}
%(END_QUESTION)





%(BEGIN_ANSWER)

\medskip
\item{$\bullet$} Diode $D_1$ fails open: {\it No effect.}
\vskip 5pt
\item{$\bullet$} Diode $D_1$ fails shorted: {\it Output always in high state, possible damage to circuit when input switch is in high state.}
\vskip 5pt
\item{$\bullet$} Diode $D_2$ fails open: {\it No effect.}
\vskip 5pt
\item{$\bullet$} Resistor $R_1$ fails open: {\it Output always in high state.}
\vskip 5pt
\item{$\bullet$} Resistor $R_2$ fails open: {\it Gate can sink some current in low output state, but cannot source current in high output state.  Gate may have trouble attaining a solid "low" output state as well.}
\vskip 5pt
\item{$\bullet$} Transistor $Q_2$ emitter terminal fails open: {\it Output always in high state.}
\vskip 5pt
\item{$\bullet$} Transistor $Q_3$ emitter terminal fails open: {\it Gate can sink current in low output state, but cannot source current in high output state.}
\medskip

%(END_ANSWER)





%(BEGIN_NOTES)

The purpose of this question is to approach the domain of circuit troubleshooting from a perspective of knowing what the fault is, rather than only knowing what the symptoms are.  Although this is not necessarily a realistic perspective, it helps students build the foundational knowledge necessary to diagnose a faulted circuit from empirical data.  Questions such as this should be followed (eventually) by other questions asking students to identify likely faults based on measurements.

%INDEX% Troubleshooting, predicting effects of fault inside TTL logic gate IC

%(END_NOTES)


