
%(BEGIN_QUESTION)
% Copyright 2003, Tony R. Kuphaldt, released under the Creative Commons Attribution License (v 1.0)
% This means you may do almost anything with this work of mine, so long as you give me proper credit

Develop a verbal description of this truth table, specifying what conditions must be met ("{\it true}" in a Boolean sense) in order for the output to assume a high state:

% Graphic image replaced by TeX table, April 5, 2005
%$$\epsfbox{01327x01.eps}$$

% No blank lines allowed between lines of an \halign structure!
% I use comments (%) instead, so that TeX doesn't choke.

$$\vbox{\offinterlineskip
\halign{\strut
\vrule \quad\hfil # \ \hfil & 
\vrule \quad\hfil # \ \hfil & 
\vrule \quad\hfil # \ \hfil & 
\vrule \quad\hfil # \ \hfil \vrule \cr
\noalign{\hrule}
%
% First row
A & B & C & Output \cr
%
\noalign{\hrule}
%
% Second row
0 & 0 & 0 & 0 \cr
%
\noalign{\hrule}
%
% Third row
0 & 0 & 1 & 0 \cr
%
\noalign{\hrule}
%
% Fourth row
0 & 1 & 0 & 0 \cr
%
\noalign{\hrule}
%
% Fifth row
0 & 1 & 1 & 0 \cr
%
\noalign{\hrule}
%
% Sixth row
1 & 0 & 0 & 0 \cr
%
\noalign{\hrule}
%
% Seventh row
1 & 0 & 1 & 1 \cr
%
\noalign{\hrule}
%
% Eighth row
1 & 1 & 0 & 0 \cr
%
\noalign{\hrule}
%
% Ninth row
1 & 1 & 1 & 0 \cr
%
\noalign{\hrule}
} % End of \halign 
}$$ % End of \vbox

\vskip 10pt

Do the same for this truth table as well:

% Graphic image replaced by TeX table, April 5, 2005
%$$\epsfbox{01327x02.eps}$$

% No blank lines allowed between lines of an \halign structure!
% I use comments (%) instead, so that TeX doesn't choke.

$$\vbox{\offinterlineskip
\halign{\strut
\vrule \quad\hfil # \ \hfil & 
\vrule \quad\hfil # \ \hfil & 
\vrule \quad\hfil # \ \hfil & 
\vrule \quad\hfil # \ \hfil \vrule \cr
\noalign{\hrule}
%
% First row
A & B & C & Output \cr
%
\noalign{\hrule}
%
% Second row
0 & 0 & 0 & 0 \cr
%
\noalign{\hrule}
%
% Third row
0 & 0 & 1 & 1 \cr
%
\noalign{\hrule}
%
% Fourth row
0 & 1 & 0 & 1 \cr
%
\noalign{\hrule}
%
% Fifth row
0 & 1 & 1 & 0 \cr
%
\noalign{\hrule}
%
% Sixth row
1 & 0 & 0 & 0 \cr
%
\noalign{\hrule}
%
% Seventh row
1 & 0 & 1 & 0 \cr
%
\noalign{\hrule}
%
% Eighth row
1 & 1 & 0 & 0 \cr
%
\noalign{\hrule}
%
% Ninth row
1 & 1 & 1 & 0 \cr
%
\noalign{\hrule}
} % End of \halign 
}$$ % End of \vbox

\underbar{file 01327}
%(END_QUESTION)





%(BEGIN_ANSWER)

For the first truth table: the output of this circuit is high when $A$ is true and $\overline{B}$ is true and $C$ is true:

$$A \hbox{ AND } \overline{B} \hbox{ AND } C$$

\vskip 10pt

For the second truth table: the output of this circuit is high when $\overline{A}$ is true and $\overline{B}$ is true and $C$ is true, or when $\overline{A}$ is true and $B$ is true and $\overline{C}$ is true:

$$(\overline{A} \hbox{ AND } \overline{B} \hbox{ AND } C) \hbox{ OR } (\overline{A} \hbox{ AND } B \hbox{ AND } \overline{C})$$

\vskip 10pt

Follow-up question: do you suspect we could write a formal Boolean expression for each of these truth tables?  What would those expressions be, and what form would they be in (SOP or POS)?

%(END_ANSWER)





%(BEGIN_NOTES)

I find this "verbal" approach works well to introduce students to the concept of deriving Boolean expressions from truth tables.

Be sure to ask your students what Boolean expressions they derived for both these truth tables.  Given the answers in "verbal" form, this should not be difficult for them!

%INDEX% Sum-of-Products expression, Boolean algebra (from a verbal description of a truth table)
%INDEX% SOP expression, Boolean algebra (from a verbal description of a truth table)

%(END_NOTES)


