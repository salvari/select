
%(BEGIN_QUESTION)
% Copyright 2005, Tony R. Kuphaldt, released under the Creative Commons Attribution License (v 1.0)
% This means you may do almost anything with this work of mine, so long as you give me proper credit

Capacitors and inductors alike have the ability to both {\it store} and {\it release} energy.  This makes them more complicated than resistors, which merely dissipate energy.  As a consequence, the relationship between direction of current and polarity of voltage is a bit more complex for capacitors and inductors than it is for power sources and resistors:

$$\epsfbox{03725x01.eps}$$

Draw current arrows and voltage polarity marks (+ and - symbols) next to each of the following components, the left group representing a battery, capacitor, and inductor all acting as {\it energy sources} and the right group representing a resistor, capacitor, and inductor all acting as {\it energy loads}:

$$\epsfbox{03725x02.eps}$$

\underbar{file 03725}
%(END_QUESTION)





%(BEGIN_ANSWER)

$$\epsfbox{03725x03.eps}$$

\vskip 10pt

Follow-up question: what form does the stored energy take inside each type of reactive component?  In other words, {\it how} does an inductor store energy, and {\it how} does a capacitor store energy?

%(END_ANSWER)





%(BEGIN_NOTES)

Although the answer may seem a bit too easy -- nay, even obvious -- the point here is to get students to correlate the behavior of capacitors and inductors in terms of components they already understand very well.  Then, they may correctly associate direction of current with polarity of voltage drop according to the direction of energy flow (source versus load) the reactive component is subjected to.

%INDEX% Capacitance, voltage versus current in
%INDEX% Inductance, voltage versus current in

%(END_NOTES)


