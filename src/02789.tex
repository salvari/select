
%(BEGIN_QUESTION)
% Copyright 2005, Tony R. Kuphaldt, released under the Creative Commons Attribution License (v 1.0)
% This means you may do almost anything with this work of mine, so long as you give me proper credit

$$\epsfbox{02789x01.eps}$$

\underbar{file 02789}
\vfil \eject
%(END_QUESTION)





%(BEGIN_ANSWER)

Use circuit simulation software to verify your predicted and actual truth tables.

%(END_ANSWER)





%(BEGIN_NOTES)

The purpose of this exercise is for students to research what type of IC this is (from the given part number for $U_1$), its pinout, and then predict and prove its operation using truth tables to document the results.  You, as the instructor, may select any 14-pin CMOS or TTL logic IC that you wish.  Students are to draw the logic gate symbol within the rectangle of $U_1$, then connect that symbol to the input switches and output LED.

It needs to be understood that the "0" and "1" states are defined by voltage levels with respect to ground, and not by switch actuation.  Many students assume an actuated (pushed) switch is a "1" input and a de-actuated (unpushed) switch is a "0" input.  Not necessarily so!  In this circuit, the switches are connecting inputs to {\it ground}.  This means a closed (actuated) switch provides a low (0) input state, while an open (unactuated) switch provides a high (1) input state.

%INDEX% Assessment, performance-based (IC logic gate usage)

%(END_NOTES)


