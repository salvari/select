
%(BEGIN_QUESTION)
% Copyright 2005, Tony R. Kuphaldt, released under the Creative Commons Attribution License (v 1.0)
% This means you may do almost anything with this work of mine, so long as you give me proper credit

Conduction of an electric current through the collector terminal of a bipolar junction transistor requires that minority carriers be "injected" into the base region by a base-emitter current.  Only after being injected into the base region may these charge carriers be swept toward the collector by the applied voltage between emitter and collector to constitute a collector current:

$$\epsfbox{02482x01.eps}$$

\vskip 10pt

$$\epsfbox{02482x02.eps}$$

An analogy to help illustrate this is a person tossing flower petals into the air above their head, while a breeze carries the petals horizontally away from them.  None of the flower petals may be "swept" away by the breeze until the person releases them into the air, and the velocity of the breeze has no bearing on how many flower petals are swept away from the person, since they must be released from the person's grip before they can go anywhere.

By referencing either the energy diagram or the flower petal analogy, explain why the collector current for a BJT is strongly influenced by the base current and only weakly influenced by the collector-to-emitter voltage.

\underbar{file 02482}
%(END_QUESTION)





%(BEGIN_ANSWER)

The action of tossing flower petals into the air is analogous to base current injecting charge carriers into the base region of a transistor.  The drifting of those tossed petals by the wind is analogous to the sweeping of charge carriers across the base and into the collector by $V_{CE}$.  Like the number of flower petals drifting, the amount of collector current does not depend much on the strength of $V_{CE}$ (the strength of the wind), but rather on the rate of charge carriers injected (the number of petals tossed upward per second).

%(END_ANSWER)





%(BEGIN_NOTES)

This is one of my better analogies for explaining BJT operation, especially for illustrating the why $I_C$ is almost independent of $V_{CE}$.  It also helps to explain reverse recovery time for transistors: imagine how long it takes the air to clear of tossed flower petals after you stop tossing them, analogous to latent charge carriers having to be swept out of the base region by $V_{CE}$ after base current stops.

%INDEX% Active mode, BJT
%INDEX% BJT, as current regulator
%INDEX% BJT, flower petal analogy for injection/collection currents

%(END_NOTES)


