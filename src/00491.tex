
%(BEGIN_QUESTION)
% Copyright 2003, Tony R. Kuphaldt, released under the Creative Commons Attribution License (v 1.0)
% This means you may do almost anything with this work of mine, so long as you give me proper credit

Before two or more operating alternators (AC generators) may be electrically coupled, they must be brought into synchronization with each other.  If two alternators are out of "sync" (or out of {\it phase}) with each other, the result will be a large fault current when the disconnect switch is closed.

A simple and effective means of checking for "sync" prior to closing the disconnect switch for an alternator is to have light bulbs connected in parallel with the disconnect switch contacts, like this:

$$\epsfbox{00491x01.eps}$$

What should the alternator operator look for before closing the alternator switch?  Do bright lights indicate a condition of being "in-phase" with the bus, or do dim lights indicate this?  What does the operator have to do in order to bring an alternator into "phase" with the bus voltage?

Also, describe what the light bulbs would do if the two alternators were spinning at slightly different speeds.

\underbar{file 00491}
%(END_QUESTION)





%(BEGIN_ANSWER)

Dim lights indicate a condition of being "in-phase" with the bus.  If the two alternators are spinning at slightly different speeds, there will be a {\it heterodyne} effect to the light bulbs' brightness: alternately growing brighter, then dimmer, then brighter again.

%(END_ANSWER)





%(BEGIN_NOTES)

Proper synchronization of alternators with bus voltage is a task that used to be performed exclusively by human operators, but may now be accomplished by automatic controls.  It is still important, though, for students of electricity to understand the principles involved in alternator synchronization, and the simple light bulb technique of sync-indication is an excellent means of clarifying the concept.

Discuss with your students the {\it means} of bringing an alternator into phase with an AC bus.  If the light bulbs are glowing brightly, what should the operator do to make them dim?

It might also be a good idea to discuss with your students what happens once two synchronized alternators become electrically coupled: the two machines become "locked" together as though they were mechanically coupled, thus maintaining synchronization from that point onward.

%INDEX% Synchronization, multiple alternators

%(END_NOTES)


