
%(BEGIN_QUESTION)
% Copyright 2003, Tony R. Kuphaldt, released under the Creative Commons Attribution License (v 1.0)
% This means you may do almost anything with this work of mine, so long as you give me proper credit

An alternative to the Darlington pair is the {\it Sziklai pair}, formed by a complementary pair of bipolar transistors:

$$\epsfbox{02049x01.eps}$$

Complete the following circuits, showing how a switch would be connected to each of the transistor pairs to exert control over the electric motor:

$$\epsfbox{02049x02.eps}$$

\underbar{file 02049}
%(END_QUESTION)





%(BEGIN_ANSWER)

$$\epsfbox{02049x03.eps}$$

Follow-up question: why would anyone want to use either a Darlington or Sziklai pair when a single transistor is able to switch current on its own?  What advantage do either of these transistor pair configurations give over a single transistor?

%(END_ANSWER)





%(BEGIN_NOTES)

Sziklai pair circuits are often not discussed in electronics texts until the subject of audio power amplifiers, where the Sziklai pair is offered as an alternative to complementary power transistors in a push-pull circuit (where the final output transistors are both NPN instead of one being NPN and the other PNP).  There is nothing wrong, however, with introducing the Sziklai pair configuration in the context of a simple switching circuit as it is done here.

%INDEX% Darlington pair
%INDEX% Sziklai pair
%INDEX% Transistor switch circuit (BJT)

%(END_NOTES)


