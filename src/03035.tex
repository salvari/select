
%(BEGIN_QUESTION)
% Copyright 2005, Tony R. Kuphaldt, released under the Creative Commons Attribution License (v 1.0)
% This means you may do almost anything with this work of mine, so long as you give me proper credit

It is usually necessary to have more than one display digit for a digital system.  The most obvious and direct way of driving multiple 7-segment display units is to use an equal number of BCD-to-7-segment decoders like this:

$$\epsfbox{03035x01.eps}$$

If we are driving the decoder ICs with a microprocessor or microcontroller, this direct technique unfortunately uses a lot of I/O pins.  In this particular case, with three 7-segment displays, we would need to use {\it twelve} output pins on the microcontroller for the three BCD numbers:

$$\epsfbox{03035x02.eps}$$

Due to limited pin count on most MPU and MCU chips, I/O lines are precious.  It would be a shame to waste so many on a simple function such as driving display digits when we could use them for other tasks such as interfacing with memory devices, receiving real-world data from sensors, driving discrete control devices such as lights and solenoids, or communicating with other MPU/MCU systems.  But if each digit requires four output lines for the BCD number, how can we possibly use less than twelve output lines on the processor?

One clever way to do just this exploits {\it persistence of human vision}, by driving only one digit at a time.  Examine the following circuit, then explain how this "multiplexed" display system works with so few output lines.  Also identify what steps the MCU/MPU must take to successfully drive all three digits so the display looks continuous:

$$\epsfbox{03035x03.eps}$$

\underbar{file 03035}
%(END_QUESTION)





%(BEGIN_ANSWER)

\goodbreak

MCU/MPU steps:

\medskip
\item{$1.$} Select digit number 1
\item{$2.$} Output BCD code for digit number 1
\item{$3.$} Pause for very brief (milliseconds) amount of time
\item{$4.$} Select digit number 2
\item{$5.$} Output BCD code for digit number 2
\item{$6.$} Pause for very brief (milliseconds) amount of time
\item{$7.$} Select digit number 3
\item{$8.$} Output BCD code for digit number 3
\item{$9.$} Pause for very brief (milliseconds) amount of time
\item{$10.$} Repeat cycle
\medskip

\vskip 10pt

Follow-up question: what would have to be changed in this circuit to use common-cathode LED 7-segment displays instead of common-anode displays?

%(END_ANSWER)





%(BEGIN_NOTES)

Be sure to ask your students where they were able to research multiplexed 7-segment displays, and what they think about this particular technique of producing a "continuous" three-digit decimal display by flashing them very rapidly.  Clever techniques such as this are often necessary to make the most of limited hardware.

By the way, I have omitted the customary LED current-limiting resistors from the schematic diagrams, for brevity's sake.  See if any of your students are able to catch this omission!

%INDEX% Displays, multiplexed
%INDEX% Multiplexing 7-segment displays

%(END_NOTES)


