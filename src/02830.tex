
%(BEGIN_QUESTION)
% Copyright 2005, Tony R. Kuphaldt, released under the Creative Commons Attribution License (v 1.0)
% This means you may do almost anything with this work of mine, so long as you give me proper credit

Use DeMorgan's Theorem, as well as any other applicable rules of Boolean algebra, to simplify the following expression so there are no more complementation bars extending over multiple variables:

$$\overline{ \overline{J + K} JL }$$

\underbar{file 02830}
%(END_QUESTION)





%(BEGIN_ANSWER)

Simplified expression:

$$\hbox{(Expression is always equal to 1)}$$

%(END_ANSWER)





%(BEGIN_NOTES)

Have your students demonstrate exactly what they did (step by step) to simplify this expression, sharing their problem-solving strategies with the whole class.

Ask your students to determine what a non-variable solution means for a circuit such as this in a practical sense.  What would they suspect if they tried to simplify a digital circuit and obtained this kind of result?

%INDEX% Boolean algebra, simplification of expression (using DeMorgan's Theorem)

%(END_NOTES)


