
%(BEGIN_QUESTION)
% Copyright 2003, Tony R. Kuphaldt, released under the Creative Commons Attribution License (v 1.0)
% This means you may do almost anything with this work of mine, so long as you give me proper credit

A parasitic property of semiconductor PN junctions is {\it capacitance} across the depletion regions.  This is often referred to as the {\it Miller Effect}.  In transistor circuits, the Miller effect contributes to a decrease in voltage gain as signal frequency increases.

Explain why junction capacitances make the voltage gain of an amplifier decrease with increasing frequency.

\underbar{file 00979}
%(END_QUESTION)





%(BEGIN_ANSWER)

The Miller capacitance between collector and base in a transistor forms a {\it negative feedback loop} for AC signals.

\vskip 10pt

Challenge question: is there any way you can think of to cancel out this negative feedback in an amplifier circuit?

%(END_ANSWER)





%(BEGIN_NOTES)

Ask your students to explain what a "negative feedback loop" is, and how exactly the base-collector junction capacitance forms one in a transistor circuit.  Also, review the formula for capacitive reactance ($X_C$), and ask your students to relate this frequency dependence to the degree of negative feedback established in an amplifier circuit.

The challenge question may be answered with a little research into the Miller-effect.  There is a method for cancellation of this unwanted negative feedback loop, but it may not be possible to implement in all amplifier circuit topologies.

%INDEX% Miller effect
%INDEX% Voltage gain, reduction due to Miller effect capacitance

%(END_NOTES)


