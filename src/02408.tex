
%(BEGIN_QUESTION)
% Copyright 2005, Tony R. Kuphaldt, released under the Creative Commons Attribution License (v 1.0)
% This means you may do almost anything with this work of mine, so long as you give me proper credit

Choose the right type of bipolar junction transistor for each of these switching applications, drawing the correct transistor symbol inside each circle:

$$\epsfbox{02408x01.eps}$$

Also, explain why resistors are necessary in both these circuits for the transistors to function without being damaged.

\underbar{file 02408}
%(END_QUESTION)





%(BEGIN_ANSWER)

$$\epsfbox{02408x02.eps}$$

\vskip 10pt

Follow-up question: explain why neither of the following transistor circuits will work.  When the pushbutton switch is actuated, the load remains de-energized:

$$\epsfbox{02408x03.eps}$$

%(END_ANSWER)





%(BEGIN_NOTES)

Discuss with your students the meaning of the words "sourcing" and "sinking" in case they are not yet familiar with them.  These are very common terms used in electronics (especially digital and power circuitry!), and they make the most sense in the context of conventional flow current notation.

In order for students to properly choose and place each transistor to make the circuits functional, they must understand how BJTs are triggered on (forward-biasing of the base-emitter junction) and also which directions the currents move through BJTs.  The two example circuits shown in this question are very realistic.

%INDEX% Sourcing versus sinking current
%INDEX% Transistor switch circuit (BJT)

%(END_NOTES)


