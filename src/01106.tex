
%(BEGIN_QUESTION)
% Copyright 2003, Tony R. Kuphaldt, released under the Creative Commons Attribution License (v 1.0)
% This means you may do almost anything with this work of mine, so long as you give me proper credit

The output voltage of a buck converter is a direct function of the switching transistor's duty cycle.  Specifically, $V_{out} = V_{in}\bigl({t_{on} \over t_{total}}\bigr)$.  Explain how the following PWM control circuit regulates the output voltage of the buck converter:

$$\epsfbox{01106x01.eps}$$

\underbar{file 01106}
%(END_QUESTION)





%(BEGIN_ANSWER)

If the load (output) voltage sags, the PWM circuit generates an output signal with a greater duty cycle, which then drives the power transistor to provide more voltage to the load.

\vskip 10pt

Follow-up question: what is the purpose of the potentiometer in this circuit?

%(END_ANSWER)





%(BEGIN_NOTES)

Here, students see a PWM control circuit coupled with a buck converter to provide {\it voltage-regulated} power conversion.  Ask them what form of feedback (positive or negative?) is used in this circuit to regulate the output voltage at a steady value.

Let your students know that the PWM and feedback functions for switching regulator circuits are often provided in a single, application-specific integrated circuit rather than by a collection of discrete components and IC's as shown in the question.

%INDEX% Buck converter circuit
%INDEX% PWM, used to control buck converter output voltage

%(END_NOTES)


