
\centerline{\bf Light-pulse switch} \bigskip 
 
This worksheet and all related files are licensed under the Creative Commons Attribution License, version 1.0.  To view a copy of this license, visit http://creativecommons.org/licenses/by/1.0/, or send a letter to Creative Commons, 559 Nathan Abbott Way, Stanford, California 94305, USA.  The terms and conditions of this license allow for free copying, distribution, and/or modification of all licensed works by the general public.

\bigskip 

\hrule

\vskip 10pt

This circuit toggles a J-K flip-flop when it senses a {\it pulse} of light.  Instead of a photocell or a solar cell, paralleled LEDs are used as the photo-detecting component.  LEDs are typically much cheaper and more readily available than either photocells or solar cells, which is why I chose to use them here.

% Sample schematic diagram here
$$\epsfbox{proj_lps.eps}$$

Of course, you are not restricted to using this exact design.  One variation is to place a monostable 555 timer circuit at the output of the comparator instead of a J-K flip-flop.  This would activate the output for a certain time period after detecting a pulse of light, rather than using one light pulse to turn the output on and another light pulse to turn the output off.

The potentiometer should be adjusted for good sensitivity in ambient light conditions.  Obviously, the darker the ambient conditions, the more sensitive the circuit is to light pulses.

My original application for this circuit was a means to control an electronic display located inside a glass-fronted display case.  By briefly pulsing a bright flashlight at the LED cluster, one could turn the display on and off through the glass with no pushbuttons or other contact-type interface devices.

\vskip 10pt

\goodbreak
\noindent
Deadlines (set by instructor):

\medskip
\item{$\bullet$} Project design completed: 
\item{$\bullet$} Components purchased:
\item{$\bullet$} Working prototype:
\item{$\bullet$} Finished system:
\item{$\bullet$} Full documentation:
\medskip


