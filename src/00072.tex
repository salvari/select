
%(BEGIN_QUESTION)
% Copyright 2003, Tony R. Kuphaldt, released under the Creative Commons Attribution License (v 1.0)
% This means you may do almost anything with this work of mine, so long as you give me proper credit

What is the difference between materials classified as {\it conductors} versus those classified as {\it insulators}, in the electrical sense of these words?

\underbar{file 00072}
%(END_QUESTION)





%(BEGIN_ANSWER)

Electrical "conductors" offer easy passage of electric current through them, while electrical "insulators" do not.  The fundamental difference between an electrical "conductor" and an electrical "insulator" is how readily electrons may drift away from their respective atoms.

For an illustration of electron mobility within a metallic substance, research the terms {\it electron gas} and {\it "sea of electrons"} in a chemistry reference book.

%(END_ANSWER)





%(BEGIN_NOTES)

It is important to realize that electrical "conductors" and "insulators" are not the same as thermal "conductors" and "insulators."  Materials that are insulators in the electrical sense may be fair conductors of heat (certain silicone gels used as heat-transfer fluids for heat sinks, for instance).  Materials that are conductors in the electrical sense may be fair insulators in the thermal sense (conductive plastics, for example).

%INDEX% Conductivity
%INDEX% Conductor versus insulator
%INDEX% Insulator versus conductor

%(END_NOTES)


