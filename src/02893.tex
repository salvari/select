
%(BEGIN_QUESTION)
% Copyright 2005, Tony R. Kuphaldt, released under the Creative Commons Attribution License (v 1.0)
% This means you may do almost anything with this work of mine, so long as you give me proper credit

Troubleshooting a system of any kind requires scientific thinking: sound deductive reasoning from effect to cause, and cause to effect.  One of the principles frequently applied in science is {\it Ockham's Razor}, named after Sir William of Ockham (1284-1350).  In Ockham's own words, the principle is as follows:

\vskip 10pt

$$\hbox{\it ``A plurality is not to be posited without necessity''}$$

\vskip 10pt

Applied to troubleshooting electric circuits, one could re-phrase Ockham's Razor as such:

$$\hbox{\it ``Look for single faults before considering multiple, simultaneous faults.''}$$

\vskip 10pt

Justify the use of Ockham's Razor in troubleshooting circuits.  Why should we first consider single faults to account for the problems the circuit is having rather than considering interesting combinations of faults which would account for the same problems?

\underbar{file 02893}
%(END_QUESTION)





%(BEGIN_ANSWER)

Because it is simply more likely that one thing has failed, than that multiple (unrelated) things have failed in just the right way to cause the problem to occur.

%(END_ANSWER)





%(BEGIN_NOTES)

A mistake common to new students is to consider wild combinations of faults in a broken system before thoroughly considering all the simpler possibilities.  This seems especially true when students answer troubleshooting questions on written exams.  When actually working on real circuits, students seem more likely to first look for simple causes.

%INDEX% Ockham's Razor, applied to troubleshooting

%(END_NOTES)


