
%(BEGIN_QUESTION)
% Copyright 2005, Tony R. Kuphaldt, released under the Creative Commons Attribution License (v 1.0)
% This means you may do almost anything with this work of mine, so long as you give me proper credit

Here is an eight-bit counter comprised of two four-bit 74HCT163 synchronous binary counters cascaded together:

$$\epsfbox{02952x01.eps}$$

Explain how this counter circuit works, and also determine which output bit is the LSB and which is the MSB.

\vskip 10pt

Now, examine this eight-bit counter comprised of the same two ICs:

$$\epsfbox{02952x02.eps}$$

Explain how this counter circuit works, and how its operation differs from the previous eight-bit counter circuit.

\underbar{file 02952}
%(END_QUESTION)





%(BEGIN_ANSWER)

The first circuit shows two four-bit counters cascaded together in a {\it ripple} fashion.  The second circuit shows the same two four-bit counters cascaded in a {\it synchronous} fashion.  In both cases, $Q_0$ of the left counter is the LSB and $Q_3$ of the right counter is the MSB.

\vskip 10pt

Follow-up question: comment on which method of cascading is preferred for this type of counter IC.  Is the functional difference between the two circuits significant enough to warrant concern?

%(END_ANSWER)





%(BEGIN_NOTES)

It is important for students to consult the datasheet for the 74HCT163 counter circuit in order to fully understand what is going on in these two cascaded counter circuits.  

%INDEX% Counter cascading

%(END_NOTES)


