
%(BEGIN_QUESTION)
% Copyright 2004, Tony R. Kuphaldt, released under the Creative Commons Attribution License (v 1.0)
% This means you may do almost anything with this work of mine, so long as you give me proper credit

% Error correction by Sam Cheung: March 18, 2005.  I had originally calculated a phase angle of 42.4
% degrees, based on the phase shift of 0.5 divisions divided by HALF of one wavelength at 4.25 
% divisions, times 360 degrees.  This mistake of mine led to every answer being wrong except for 
% apparent power (S).  The correct approach is to take 0.5 divisions shift and divide by 8.5 
% divisions for the full wavelength, then multiply by 360 degrees.  This yields the proper phase 
% shift angle of 21.2 degrees.

A dual-trace oscilloscope is used to measure the phase shift between voltage and current for an inductive AC load:

$$\epsfbox{02192x01.eps}$$

Calculate the following, given a load voltage of 110 volts, a load current of 3.2 amps, and a frequency of 60 Hz:

\medskip
\goodbreak
\item{$\bullet$} Apparent power ($S$) =
\item{$\bullet$} True power ($P$) =
\item{$\bullet$} Reactive power ($Q$) =
\item{$\bullet$} $\Theta$ =
\item{$\bullet$} Power factor =
\item{$\bullet$} Necessary parallel $C$ size to correct power factor to unity = 
\medskip

\underbar{file 02192}
%(END_QUESTION)





%(BEGIN_ANSWER)

\medskip
\goodbreak
\item{$\bullet$} Apparent power ($S$) = 352 VA
\item{$\bullet$} True power ($P$) = 328.2 W
\item{$\bullet$} Reactive power ($Q$) = 127.2 VAR
\item{$\bullet$} $\Theta$ = 21.2$^{o}$
\item{$\bullet$} Power factor = 0.932
\item{$\bullet$} Necessary parallel $C$ size to correct power factor to unity = 27.9 $\mu$F
\medskip

\vskip 10pt

Follow-up question: identify which waveform represents voltage and which waveform represents current on the oscilloscope display.

%(END_ANSWER)





%(BEGIN_NOTES)

There are multiple methods of solution for this problem, so be sure to have your students present their thoughts and strategies during discussion!

%INDEX% Power factor correction

%(END_NOTES)


