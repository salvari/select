
%(BEGIN_QUESTION)
% Copyright 2003, Tony R. Kuphaldt, released under the Creative Commons Attribution License (v 1.0)
% This means you may do almost anything with this work of mine, so long as you give me proper credit

How does the rate of charge flow (current) into and out of a capacitor relate to the amount of voltage across its terminals?  How does the rate of water flow into and out of a vessel relate to the amount of water stored in that vessel?

$$\epsfbox{00193x01.eps}$$

\underbar{file 00193}
%(END_QUESTION)





%(BEGIN_ANSWER)

Rather than simply give you an answer here, I'll let you figure this out for yourself.  Think very carefully about the water-in-a-vessel analogy when answering this question!  Fill a glass with water, if necessary, to gain an intuitive understanding of these quantities.

%(END_ANSWER)





%(BEGIN_NOTES)

The existence of such an appropriate analogy for capacitor action makes an explanation unnecessary, even if the concept takes a bit of thought to comprehend at first.  It is important that students clearly distinguish the quantities of {\it current}, {\it voltage}, and {\it charge} in a capacitor circuit just as they clearly distinguish the quantities of {\it liquid height}, {\it flow rate}, and {\it liquid volume} in a hydraulic system.

%INDEX% Capacitance, voltage versus current in

%(END_NOTES)


