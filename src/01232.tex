
%(BEGIN_QUESTION)
% Copyright 2003, Tony R. Kuphaldt, released under the Creative Commons Attribution License (v 1.0)
% This means you may do almost anything with this work of mine, so long as you give me proper credit

What is a {\it floating-point} number in a digital system?

\underbar{file 01232}
%(END_QUESTION)





%(BEGIN_ANSWER)

"Floating-point" numbers are the binary equivalent of scientific notation: certain bits are used to represent the mantissa, another collection of bits represents the exponent, and (usually) there is a single bit representing sign.  Unfortunately, there are several different "standards" for representing floating-point numbers.

%(END_ANSWER)





%(BEGIN_NOTES)

Ask your students why computer systems would have need for floating point numbers.  What's wrong with the standard forms of binary numbers that we've explored thus far?

%INDEX% Floating-point number, defined

%(END_NOTES)


