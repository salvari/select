
%(BEGIN_QUESTION)
% Copyright 2003, Tony R. Kuphaldt, released under the Creative Commons Attribution License (v 1.0)
% This means you may do almost anything with this work of mine, so long as you give me proper credit

$$\epsfbox{01644x01.eps}$$

\underbar{file 01644}
\vfil \eject
%(END_QUESTION)





%(BEGIN_ANSWER)

The magnitude of the induced voltage is a direct function of the magnetic flux's rate of change over time ($d \phi \over dt$).

%(END_ANSWER)





%(BEGIN_NOTES)

Old solenoid valve coils work very well for this exercise, as do spools of wire with large steel bolts passed through the center.  Students may also wind their own coils using small-gauge magnet wire and a steel bolt.

Please note that students will not be able to {\it predict} the polarity of the induced voltage unless they know the rotation of the coil windings and the polarity of their magnet.  This will only be possible if the windings are exposed to view or if the students wind their own coils, and if the magnet has its poles labeled "North" and "South" (or if this is determined experimentally by using the magnet as a compass).

Use magnets that are as strong as possible, and that have their poles on the physical ends.  This may seem like a strange request, but I've seen students bring some unusual magnets to class for this experiment, whose poles are {\it not} located on the ends.  One type of magnet that works well is the so-called "cow magnet," used by cattle ranchers to protect cows' multiple stomachs from injury from ingestion of fence staples and other ferromagnetic objects.  These are a few inches long, cylindrical in shape (so the cow can swallow it like a big pill), and quite strong.  

If students are using analog multimeters to measure the coil's induced voltage, be sure to keep the multimeter far away from the magnet.  Analog meter movements are generally quite sensitive to external magnetic fields and may register falsely if positioned too close to a strong magnet.

%INDEX% Assessment, performance-based (Electromagnetic induction)

%(END_NOTES)


