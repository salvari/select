
%(BEGIN_QUESTION)
% Copyright 2003, Tony R. Kuphaldt, released under the Creative Commons Attribution License (v 1.0)
% This means you may do almost anything with this work of mine, so long as you give me proper credit

Suppose you had the boring job of manually maintaining the output voltage of a DC generator constant.  Your one and only control over voltage is the setting of a rheostat:

$$\epsfbox{00889x01.eps}$$

What would you have to do to maintain the load voltage constant if the load resistance changed so as to draw more current?  Being that your only control over load voltage is the adjustment of a variable resistance in parallel with the load, what does this imply about the generator's output voltage (directly across the generator terminals), compared to the target load voltage?

\underbar{file 00889}
%(END_QUESTION)





%(BEGIN_ANSWER)

In order to increase the load voltage, you must increase the resistance of the rheostat.  In order for this scheme to work, the generator's voltage must be greater than the target load voltage.

Note: this general voltage control scheme is known as {\it shunt regulation}, where a parallel (shunt) resistance is varied to control voltage to a load.

\vskip 10pt

Follow-up question: assuming the load voltage is maintained at a constant value by an astute rheostat operator despite fluctuations in load current, how would you characterize the current through the generator's windings?  Does it increase with load current, decrease with load current, or remain the same?  Why?

%(END_ANSWER)





%(BEGIN_NOTES)

The direction of rheostat adjustment should be obvious, as is the fact that the generator's voltage must be at least as high as the intended (target) load voltage.  However, it may not be obvious to all that the generator's voltage cannot merely be equal to the intended load voltage.

To illustrate the necessity of this, ask your students how the system would work if the generator's output voltage was exactly equal to the intended load voltage.  Emphasize the fact that the generator is not perfect: it has its own internal resistance, the value of which cannot be changed by you.  What position would the rheostat have to be in, under these conditions, in order to maintain target voltage at the load?  Could the target voltage be maintained at all?

A helpful analogy for students is that of a car with an automatic transmission, with its speed being controlled by the brake pedal while the accelerator pedal is maintained at a constant position.  This is not the most energy-efficient method of speed control, but it will work within certain limits!

%INDEX% Voltage regulator, shunt (manual)

%(END_NOTES)


