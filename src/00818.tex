
%(BEGIN_QUESTION)
% Copyright 2003, Tony R. Kuphaldt, released under the Creative Commons Attribution License (v 1.0)
% This means you may do almost anything with this work of mine, so long as you give me proper credit

We know that in order to induce a sinusoidal voltage in a wire coil, the magnetic flux linking the turns of wire in the coil must follow a sinusoidal path over time, phase-shifted 90$^{o}$ from the voltage waveform.  This relationship between flux and induced voltage is expressed in Faraday's equation $v = N {d\phi \over dt}$:

$$\epsfbox{00818x01.eps}$$

Based on this fact, draw the position of the magnetic rotor in this alternator when the voltage is at one of its peaks:

$$\epsfbox{00818x02.eps}$$

\underbar{file 00818}
%(END_QUESTION)





%(BEGIN_ANSWER)

The alternator voltage peaks when the magnetic flux is at the zero-crossover point:

$$\epsfbox{00818x03.eps}$$

(The actual magnet polarities are not essential to the answer.  Without knowing which way the coils were wound and which way the rotor is spinning, it is impossible to specify an exact magnetic polarity, so if your answer had "N" facing down and "S" facing up, it's still acceptable.)

%(END_ANSWER)





%(BEGIN_NOTES)

This question challenges students to relate the magnetic flux waveform ($\phi$) to an instantaneous rotor position.  The answer may come as a surprise to some, who expected maximum induced voltage to occur when the rotor is in-line with the stator poles.  This answer, however, makes the mistake of confusing flux ($\phi$) with rate-of-flux-change over time ($d\phi \over dt$).  A rotor lined up with the stator poles would result in maximum flux ($\phi$) through those poles, but not maximum rate-of-flux-change over time ($d\phi \over dt$).

%INDEX% Magnetic flux, relation to induced voltage in alternator windings

%(END_NOTES)


