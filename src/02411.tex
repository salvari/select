
%(BEGIN_QUESTION)
% Copyright 2005, Tony R. Kuphaldt, released under the Creative Commons Attribution License (v 1.0)
% This means you may do almost anything with this work of mine, so long as you give me proper credit

Junction field-effect transistors (JFETs) are {\it normally-on} devices, the natural state of their channels being passable to electric currents.  Thus, a state of cutoff will only occur on command from an external source.

Explain what must be done to a JFET, specifically, to drive it into a state of cutoff.

\underbar{file 02411}
%(END_QUESTION)





%(BEGIN_ANSWER)

The gate-channel PN junction must be {\it reverse-biased}: a voltage applied between gate and source such that the negative side is connected to the "P" material and the positive side to the "N" material.

\vskip 10pt

Follow-up question: is any gate {\it current} required to drive a JFET into the cutoff state?  Why or why not?

%(END_ANSWER)





%(BEGIN_NOTES)

This is perhaps the most important question your students could learn to answer when first studying JFETs.  What, exactly, is necessary to turn one off?  Have your students draw diagrams to illustrate their answers as they present in front of the class.

%INDEX% JFET, conditions necessary for cutoff
%INDEX% Junction biasing, for JFET during cutoff

%(END_NOTES)


