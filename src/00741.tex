
%(BEGIN_QUESTION)
% Copyright 2003, Tony R. Kuphaldt, released under the Creative Commons Attribution License (v 1.0)
% This means you may do almost anything with this work of mine, so long as you give me proper credit

An interesting variation on the induction motor theme is the {\it wound-rotor} induction motor.  In the simplest form of a wound-rotor motor, the rotor's electromagnet coil terminates on a pair of slip rings which permit contact with stationary carbon brushes, allowing an external circuit to be connected to the rotor coil:

$$\epsfbox{00741x01.eps}$$

Explain how this motor can be operated as either a synchronous motor or a "plain" induction motor.

\underbar{file 00741}
%(END_QUESTION)





%(BEGIN_ANSWER)

A wound-rotor motor with a single rotor coil may be operated as a synchronous motor by energizing the rotor coil with direct current (DC).  Induction operation is realized by short-circuiting the slip rings together, through the brush connections.

\vskip 10pt

Challenge question: what will happen to this motor if a {\it resistance} is connected between the brushes, instead of a DC source or a short-circuit?

%(END_ANSWER)





%(BEGIN_NOTES)

In reality, almost all large synchronous motors are built this way, with an electromagnetic rotor rather than a permanent-magnet rotor.  This allows the motor to start much easier.  Ask your students why they think this would be an important feature in a large synchronous motor, to be able to start it as an induction motor.  What would happen if AC power were suddenly applied to a large synchronous motor with its rotor already magnetized?

If a resistance is connected between the brushes, it allows for an even easier start-up.  By "easier," I mean a start-up that draws less inrush current, resulting in a gentler ramp up to full speed.

%INDEX% Wound-rotor induction motor

%(END_NOTES)


