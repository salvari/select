
%(BEGIN_QUESTION)
% Copyright 2005, Tony R. Kuphaldt, released under the Creative Commons Attribution License (v 1.0)
% This means you may do almost anything with this work of mine, so long as you give me proper credit

The following ladder logic diagram is for a reversing motor control circuit:

$$\epsfbox{03148x01.eps}$$

Study this diagram, then explain how motor reversal is accomplished.  Also, identify the function of each "M" contact in the control circuit, especially those normally-closed contacts in series with the motor starter coils.

\vskip 10pt

\goodbreak

Now consider the following modification made to the reversing motor control circuit (motor and power contacts not shown here):

$$\epsfbox{03148x02.eps}$$

What extra functionality do the time-delay relays contribute to this motor control circuit?

\underbar{file 03148}
%(END_QUESTION)





%(BEGIN_ANSWER)

The normally-open and normally-closed "M" contacts provide seal-in and interlock functions, respectively.  The time-delay relays prevent the motor from being {\it immediately} reversed.

\vskip 10pt

Follow-up question: figure out how to simplify the time-delay relay circuit.  Hint: integrate the time-delay and interlocking functions into a single contact (per rung).

%(END_ANSWER)





%(BEGIN_NOTES)

This circuit provides students an opportunity to analyze the workings of a delayed-start, reversing motor control circuit.  Have your students present both their analyses and the methods behind the analyses as you work through this question with them.

%INDEX% Interlock contact, motor control circuit
%INDEX% Time delay relay circuit, electric motor control

%(END_NOTES)


