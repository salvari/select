
%(BEGIN_QUESTION)
% Copyright 2005, Tony R. Kuphaldt, released under the Creative Commons Attribution License (v 1.0)
% This means you may do almost anything with this work of mine, so long as you give me proper credit

The following schematic diagram is for a {\it two-input selector} circuit, which (as the name implies) selects one of two inputs to be sent to the output:

$$\epsfbox{03065x01.eps}$$

Determine what state the "select control" input line has to be in to select Input$_{A}$ to be sent to the output, and what state it has to be in to select Input$_{B}$ to go to the output.

\underbar{file 03065}
%(END_QUESTION)





%(BEGIN_ANSWER)

A high signal on the "select control" line selects Input$_{A}$, while a low signal on that same line selects Input$_{B}$.

%(END_ANSWER)





%(BEGIN_NOTES)

Selector circuits are widely used internally in counter and shift register circuits where digital signals must be selected from multiple sources to achieve certain functions.  Be sure your students understand how it works, for they will surely see it later in some application!

%INDEX% Selector circuit (two input), digital

%(END_NOTES)


