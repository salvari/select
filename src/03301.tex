
%(BEGIN_QUESTION)
% Copyright 2005, Tony R. Kuphaldt, released under the Creative Commons Attribution License (v 1.0)
% This means you may do almost anything with this work of mine, so long as you give me proper credit

Suppose someone were to ask you to differentiate electrical reactance ($X$) from electrical resistance ($R$).  How would you distinguish these two similar concepts from one another, using your own words?

\underbar{file 03301}
%(END_QUESTION)





%(BEGIN_ANSWER)

It is really important for you to frame this concept in your own words, so be sure to check with your instructor on the accuracy of your answer to this question!  To give you a place to start, I offer this distinction: resistance is electrical {\it friction}, whereas reactance is electrical {\it energy storage}.  Fundamentally, the difference between $X$ and $R$ is a matter of energy exchange, and it is understood most accurately in those terms.

%(END_ANSWER)





%(BEGIN_NOTES)

This is an excellent point of crossover with your students' studies in elementary physics, if they are studying physics now or have studied physics in the past.  The energy-storing actions of inductors and capacitors are quite analogous to the energy-storing actions of masses and springs (respectively, if you associate velocity with current and force with voltage).  In the same vein, resistance is analogous to kinetic friction between a moving object and a stationary surface.  The parallels are so accurate, in fact, that the electrical properties of $R$, $L$, and $C$ have been exploited to model mechanical systems of friction, mass, and resilience in circuits known as {\it analog computers}.

%INDEX% Reactance versus resistance
%INDEX% Resistance versus reactance 

%(END_NOTES)


