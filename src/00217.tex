
%(BEGIN_QUESTION)
% Copyright 2003, Tony R. Kuphaldt, released under the Creative Commons Attribution License (v 1.0)
% This means you may do almost anything with this work of mine, so long as you give me proper credit

What does the {\it last} color band represent on a color-coded resistor?

\underbar{file 00217}
%(END_QUESTION)





%(BEGIN_ANSWER)

There is more than one answer to this question!  On some resistors, the last band represents the {\it tolerance} (also known as {\it precision}) for that resistor, expressed as a percentage.  On other resistors, the last band represents a {\it reliability} rating for that resistor.

%(END_ANSWER)





%(BEGIN_NOTES)

This question is worded simply and directly enough that students might think there is only one right answer.  However, upon doing some research they should find that there is more involved than one simple answer can encompass!  Discuss with your students the different color code types, and what applications one might find resistors with "reliability" color codes in.

Regarding precision, nothing in life is perfectly accurate.  However, the absence of perfect accuracy does not necessarily imply total uncertainty.  In science, especially, it is important that all data be qualified by a statement of precision (or tolerance).  Your students may be familiar with "margins of error" stated for public opinion polls.  With resistors, this "margin of error" (expression of uncertainty) is explicitly given in the form of a separate color band.

%INDEX% Color code, resistor (tolerance band)

%(END_NOTES)


