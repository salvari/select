
%(BEGIN_QUESTION)
% Copyright 2003, Tony R. Kuphaldt, released under the Creative Commons Attribution License (v 1.0)
% This means you may do almost anything with this work of mine, so long as you give me proper credit

A very common type of amplifier used in electronic circuits is the {\it voltage buffer}, sometimes called a {\it voltage follower}.  There are two simple forms of this circuit, one using a single transistor and the other using an integrated circuit called an {\it operational amplifier}:

$$\epsfbox{00865x01.eps}$$

The voltage gain of each of these devices is unity ($A_V = 1$).  My question to you is this: what possible use is an amplifier that doesn't even amplify the voltage of its input signal?  If the output voltage is the same magnitude as the input voltage, then is this circuit really amplifying anything at all?  A straight piece of wire outputs the same voltage that it receives in!

Explain the practical purpose for these very popular amplifier circuit configurations.

\underbar{file 00865}
%(END_QUESTION)





%(BEGIN_ANSWER)

While a "voltage buffer" does not amplify the voltage level of a signal, it does amplify the {\it current} level of a signal.

%(END_ANSWER)





%(BEGIN_NOTES)

Voltage buffers are almost ubiquitous in modern electronic circuitry, so they cannot be dismissed as useless.  Discuss with your students some possible applications of voltage buffers.  When would we want to amplify a signal's current without amplifying the voltage?  Do your students think there might be any application for this type of circuit in electronic test equipment (voltmeters, especially?).

%(END_NOTES)


