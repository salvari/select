
%(BEGIN_QUESTION)
% Copyright 2005, Tony R. Kuphaldt, released under the Creative Commons Attribution License (v 1.0)
% This means you may do almost anything with this work of mine, so long as you give me proper credit

A student makes a mistake somewhere in the process of simplifying the following Boolean expression:

$$AB + A(B+C)$$

$$AB + AB + C$$

$$AB + C$$

Determine where the mistake was made, and what the proper sequence of steps should be to simplify the original expression.

\underbar{file 02804}
%(END_QUESTION)





%(BEGIN_ANSWER)

An error was made in the second step (distribution).  The correct sequence of steps is as follows: 

$$AB + A(B+C)$$

$$AB + AB + AC$$

$$AB + AC$$

$$A(B + C)$$

%(END_ANSWER)





%(BEGIN_NOTES)

An interesting way to sharpen students' understanding of algebraic techniques is to have them view someone else's incorrect work and find the error(s).  Ultimately, algebraic reduction is really just an exercise in pattern recognition.  Anything you can do to help your students recognize the correct patterns will help them become better at using algebra.

%INDEX% Boolean algebra, simplification of expression

%(END_NOTES)


