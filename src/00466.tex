
%(BEGIN_QUESTION)
% Copyright 2004, Tony R. Kuphaldt, released under the Creative Commons Attribution License (v 1.0)
% This means you may do almost anything with this work of mine, so long as you give me proper credit

Suppose two wire coils are wound around a common iron core, the "primary" coil with 100 turns of wire and the "secondary" coil with 300 turns of wire:

$$\epsfbox{00466x01.eps}$$

If the inductance of the primary coil is 2 H, what is the inductance of the secondary coil, assuming that it "sees" the exact same magnetic circuit as the first coil (same permeability, same cross-sectional area, same length)?

If an electric current changing at a rate of 30 amps per second passes through the primary coil, how much voltage will be induced in {\it each} coil?

\vskip 10pt

If only half the lines of magnetic flux from the primary coil "coupled" with the secondary coil, how much voltage would be induced in the secondary coil, given a primary current rate-of-change of 30 amps per second?

\underbar{file 00466}
%(END_QUESTION)





%(BEGIN_ANSWER)

$L_s =$ 18 H

$e_p =$ 60 volts

$e_s =$ 180 volts

\vskip 10pt

If only half the lines of flux coupled the two coils ($k = 0.5$), then $e_s = 90$ volts.

\vskip 10pt

Follow-up question: what do you notice about the ratio of primary and secondary {\it inductances} compared with primary and secondary {\it winding turns}?  Can you generalize this in the form of an equation?

%(END_ANSWER)





%(BEGIN_NOTES)

The key to this question is determining the ratio of inductances, based on the ratio of turns in the windings.  As the answer reveals, it is a nonlinear proportionality.  The sentence where I specify the "same permeability, same cross-sectional area, same length" is a hint to students for what equation they need to find in order to identify the relationship between wire turns and inductance.

%INDEX% k, coefficient of magnetic coupling
%INDEX% Coefficient of magnetic coupling, k
%INDEX% Inductance ratio, transformer

%(END_NOTES)


