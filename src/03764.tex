
%(BEGIN_QUESTION)
% Copyright 2005, Tony R. Kuphaldt, released under the Creative Commons Attribution License (v 1.0)
% This means you may do almost anything with this work of mine, so long as you give me proper credit

Predict how the operation of this differential pair circuit will be affected as a result of the following faults.  Consider each fault independently (i.e. one at a time, no multiple faults):

$$\epsfbox{03764x01.eps}$$

\medskip
\item{$\bullet$} Resistor $R_1$ fails open:
\vskip 5pt
\item{$\bullet$} Resistor $R_2$ fails open:
\vskip 5pt
\item{$\bullet$} Resistor $R_3$ fails open:
\vskip 5pt
\item{$\bullet$} Solder bridge (short) across resistor $R_3$:
\medskip

For each of these conditions, explain {\it why} the resulting effects will occur.

\underbar{file 03764}
%(END_QUESTION)





%(BEGIN_ANSWER)

\medskip
\item{$\bullet$} Resistor $R_1$ fails open: {\it More current will be drawn from $V_1$; if $V_1$ sags as a result, $V_{out}$ will decrease.}
\vskip 5pt
\item{$\bullet$} Resistor $R_2$ fails open: {\it $V_{out}$ will assume a voltage level approximately equal to $V_2$ - 0.7 volts.}
\vskip 5pt
\item{$\bullet$} Resistor $R_3$ fails open: {\it $V_{out}$ saturates to +V (positive supply rail).}
\vskip 5pt
\item{$\bullet$} Solder bridge (short) across resistor $R_3$: {\it $V_{out}$ saturates to within a few tenths of a volt from -V (negative supply rail).}
\medskip

%(END_ANSWER)





%(BEGIN_NOTES)

The purpose of this question is to approach the domain of circuit troubleshooting from a perspective of knowing what the fault is, rather than only knowing what the symptoms are.  Although this is not necessarily a realistic perspective, it helps students build the foundational knowledge necessary to diagnose a faulted circuit from empirical data.  Questions such as this should be followed (eventually) by other questions asking students to identify likely faults based on measurements.

%INDEX% Troubleshooting, predicting effects of fault in differential pair (BJT) circuit

%(END_NOTES)


