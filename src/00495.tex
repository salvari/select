
%(BEGIN_QUESTION)
% Copyright 2003, Tony R. Kuphaldt, released under the Creative Commons Attribution License (v 1.0)
% This means you may do almost anything with this work of mine, so long as you give me proper credit

Determine the sum of these two phasors, and draw a phasor diagram showing their geometric addition:

\vskip 10pt

(4 $\angle$ 0$^o$) + (3 $\angle$ 90$^o$)

\vskip 10pt

How might a phasor arithmetic problem such as this relate to an AC circuit?

\underbar{file 00495}
%(END_QUESTION)





%(BEGIN_ANSWER)

(4 $\angle$ 0$^o$) + (3 $\angle$ 90$^o$) = (5 $\angle$ 36.87$^o$)

$$\epsfbox{00495x01.eps}$$

%(END_ANSWER)





%(BEGIN_NOTES)

It is very helpful in a question such as this to graphically depict the phasors.  Have one of your students draw a phasor diagram on the whiteboard for the whole class to observe and discuss.

The relation of this arithmetic problem to an AC circuit is a very important one for students to grasp.  It is one thing for students to be able to mathematically manipulate and combine phasors, but quite another for them to smoothly transition between a phasor operation and comprehension of voltages and/or currents in an AC circuit.  Ask your students to describe what the {\it magnitude} of a phasor means (in this example, the number 5), if that phasor represents an AC voltage.  Ask your students to describe what the {\it angle} of an AC voltage phasor means, as well (in this case, 36.87$^{o}$), for an AC voltage.

%INDEX% Phasor diagram

%(END_NOTES)


