
%(BEGIN_QUESTION)
% Copyright 2003, Tony R. Kuphaldt, released under the Creative Commons Attribution License (v 1.0)
% This means you may do almost anything with this work of mine, so long as you give me proper credit

Observe the following sequence of numbers:

\vskip 10pt

00

01

02

03

04

05

06

07

08

09

10

11

12

13

14

15

16

17

18

19

20

21

.

.

.

What pattern(s) do you notice in the digits, as we count upward from 0 to 21 (and beyond)?  This may seem like a very simplistic question (and it is!), but it is important to recognize any inherent patterns so that you may understand counting sequences in numeration systems with bases other than ten.

\underbar{file 01196}
%(END_QUESTION)





%(BEGIN_ANSWER)

Note the repeating sequence of digits in the one's place, and the developing pattern of incrementation in the ten's place.  How many numbers may be counted without repeating digits in the one's place? 

%(END_ANSWER)





%(BEGIN_NOTES)

Some students will think this question to be ridiculously simple, but there are some important lessons to be learned here.  The greatest problem students face while learning binary, octal, and hexadecimal numeration systems is the overpowering familiarity with base-ten numeration.  We are {\it so} accustomed to base-ten that we don't bother to recognize its basic properties, and typically believe that this is the only way numbers may be written!

%(END_NOTES)


