
%(BEGIN_QUESTION)
% Copyright 2003, Tony R. Kuphaldt, released under the Creative Commons Attribution License (v 1.0)
% This means you may do almost anything with this work of mine, so long as you give me proper credit

If you have ever used a public address ("PA") amplifier, where sounds detected by a microphone are amplified and reproduced by speakers, you know how these systems can create "screeching" or "howling" sounds if the microphone is held too close to one of the speakers.

The noise created by a system like this is an example of {\it oscillation}: where the amplifier circuit spontaneously outputs an AC voltage, with no external source of AC signal to "drive" it.  Explain what necessary condition(s) allow an amplifier to act as an {\it oscillator}, using a "howling" PA system as the example.  In other words, what exactly is going on in this scenario, that makes an amplifier generate its own AC output signal?

\underbar{file 01074}
%(END_QUESTION)





%(BEGIN_ANSWER)

The amplifier receives {\it positive feedback} from the output (speaker) to the input (microphone).

%(END_ANSWER)





%(BEGIN_NOTES)

Ask your students to define what "positive feedback" is.  In what way is the feedback in this system "positive," and how does this feedback differ from the "negative" variety commonly seen within amplifier circuitry?

%INDEX% Positive feedback, causing "howling" in PA systems

%(END_NOTES)


