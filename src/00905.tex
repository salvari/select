
%(BEGIN_QUESTION)
% Copyright 2003, Tony R. Kuphaldt, released under the Creative Commons Attribution License (v 1.0)
% This means you may do almost anything with this work of mine, so long as you give me proper credit

What effect does temperature have on the electrical conductivity of a semiconducting material?  How does this compare with the effect of temperature on the electrical conductivity of a typical metal?

\underbar{file 00905}
%(END_QUESTION)





%(BEGIN_ANSWER)

Semiconducting materials have negative temperature coefficient of resistance ($\alpha$) values, meaning that their resistance decreases with increasing temperature.

%(END_ANSWER)





%(BEGIN_NOTES)

The answer to this question is a short review on temperature coefficients of resistance ($\alpha$), for those students who may not recall the subject from their DC circuit studies.  As always, though, the most important point of this question is {\it why} conductivity increases for semiconductors.  Ask your students to relate their answer to the concept of {\it charge carriers} in semiconducting substances.

An interesting bit of trivia you could mention to your students is that glass -- normally an excellent insulator of electricity -- may be made electrically conductive by heating.  Glass must be heated until it is red-hot before it becomes really conductive, so it is not an easy phenomenon to demonstrate.  I found this gem of an experiment in an old book: \underbar{Demonstration Experiments in Physics}, first edition (fourth impression), copyright 1938, by Richard Manliffe Sutton, Ph.D.

%INDEX% Conductivity

%(END_NOTES)


