
%(BEGIN_QUESTION)
% Copyright 2003, Tony R. Kuphaldt, released under the Creative Commons Attribution License (v 1.0)
% This means you may do almost anything with this work of mine, so long as you give me proper credit

As an instructor of electronics, I am called upon frequently to help students troubleshoot their malfunctioning lab circuits.  When I approach a student's self-built circuit to troubleshoot it, though, I often begin the process with a very different mindset than if I were troubleshooting a malfunctioning circuit on a real job site.

Aside from different safety considerations and a very different work environment, what else do you think I might consider differently when approaching a student-built circuit?  Specifically, how might the range of probable faults differ between a professionally-installed electronic system that malfunctions and a student's lab project that malfunctions?  What generalizations might you make about this difference in troubleshooting perspective, regarding the construction and operational history of the circuit in question?

\underbar{file 01583}
%(END_QUESTION)





%(BEGIN_ANSWER)

If the circuit in question is untried, literally {\it anything} could be wrong with it.

%(END_ANSWER)





%(BEGIN_NOTES)

Although sound troubleshooting technique will ultimately yield a solution, asking "pre-diagnostic" questions such as this will greatly enhance your efficiency as a troubleshooter.  Discuss this with your students, enlightening them if possible with anecdotes from your own troubleshooting experiences.

%INDEX% Troubleshooting, considering history of system

%(END_NOTES)


