
%(BEGIN_QUESTION)
% Copyright 2006, Tony R. Kuphaldt, released under the Creative Commons Attribution License (v 1.0)
% This means you may do almost anything with this work of mine, so long as you give me proper credit

A {\it differential} relay is a common type of protective relay used in power systems.  One of the more common forms is the differential current relay.  A very common example of a differential current relay -- so common, in fact, that nearly every house is equipped with at least one -- is the {\it GFCI}, or {\it Ground Fault Current Interrupter}.  Explain what a GFCI is, and then in a larger context, explain what a differential relay protects against.

\underbar{file 04023}
%(END_QUESTION)





%(BEGIN_ANSWER)

The function of a GFCI is very easy to research, so I'll leave that to you.  In a more general sense, a {\it differential relay} protects against conditions where two or more electrical quantities (usually current) are not in phasor balance.  That is, a differential relay will trip when two or more electrical quantities do not precisely balance one another when they should.

%(END_ANSWER)





%(BEGIN_NOTES)

This question affords students the opportunity to relate something they probably never have had exposure to (a differential protective relay) to something they may see every day (a GFCI-protected power receptacle).  The purpose of this comparison, of course, is to give students a familiar context in which to understand something new.

%INDEX% GFCI
%INDEX% Ground Fault Current Interrupter
%INDEX% Protective relay, differential (defined)

%(END_NOTES)


