
%(BEGIN_QUESTION)
% Copyright 2003, Tony R. Kuphaldt, released under the Creative Commons Attribution License (v 1.0)
% This means you may do almost anything with this work of mine, so long as you give me proper credit

Describe how a zener diode is able to maintain regulated (nearly constant) voltage across the load, despite changes in load current:

$$\epsfbox{00890x01.eps}$$

\underbar{file 00890}
%(END_QUESTION)





%(BEGIN_ANSWER)

The zener draws more or less current as necessary from the generator (through the series resistor) to maintain voltage at a nearly constant value.

\vskip 10pt

Follow-up question \#1: if the generator happens to output some ripple voltage (as all electromechanical DC generators do), will any of that ripple voltage appear at the load, after passing through the zener diode voltage regulator circuit?

\vskip 10pt

Follow-up question \#2: would you classify the zener diode in this circuit as a {\it series} voltage regulator or a {\it shunt} voltage regulator?  Explain your answer.

\vskip 10pt

Challenge question: at what point is the zener diode unable to regulate load voltage?  Is there some critical load condition at which the diode ceases to regulate voltage?

%(END_ANSWER)





%(BEGIN_NOTES)

Ask your students to describe how energy-efficient they think this circuit is.  Do they suspect it would be more suitable for low-current applications or high-current applications?

%INDEX% Voltage regulator, shunt (zener diode)
%INDEX% Zener diode as a voltage regulator

%(END_NOTES)


