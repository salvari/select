
%(BEGIN_QUESTION)
% Copyright 2003, Tony R. Kuphaldt, released under the Creative Commons Attribution License (v 1.0)
% This means you may do almost anything with this work of mine, so long as you give me proper credit

Alternating current produced by electromechanical generators (or {\it alternators} as they are sometimes designated) typically follows a sine-wave pattern over time.  Using a calculator, or a set of "trig tables," plot a sine wave on the following graph:

$$\epsfbox{00052x01.eps}$$

\underbar{file 00052}
%(END_QUESTION)





%(BEGIN_ANSWER)

$$\epsfbox{00052x02.eps}$$

%(END_ANSWER)





%(BEGIN_NOTES)

For many students, this might be the first time they realize trigonometry functions have anything to do with electricity!  That voltage and current in an AC circuit might alternate according to a mathematical function available in their calculators is something of a revelation.  Be prepared to discuss {\it why} rotating electromagnetic machines naturally produce such waveforms.  Also, encourage students to make the cognitive connection between the independent variable of a sine function (angle, expressed in units of degrees in this question) to actual shaft rotation in a real generator.

%INDEX% Sine wave, defined
%INDEX% Alternator

%(END_NOTES)


