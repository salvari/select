
%(BEGIN_QUESTION)
% Copyright 2003, Tony R. Kuphaldt, released under the Creative Commons Attribution License (v 1.0)
% This means you may do almost anything with this work of mine, so long as you give me proper credit

{\it Logarithms} have interesting properties, which we may exploit in electronic circuits to perform certain complex operations.  In this question, I recommend you use a hand calculator to explore these properties.

Calculate the following:

\medskip
\item{$\bullet$} $10^{\log 3} =$ 
\item{$\bullet$} $\log (10^8) =$

\vskip 10pt

\item{$\bullet$} $e^{\ln 3} =$ 
\item{$\bullet$} $\ln (e^8) =$

\vskip 10pt

\item{$\bullet$} $10^{(\log 3 + \log 5)} =$
\item{$\bullet$} $e^{(\ln 3 + \ln 5)} =$

\vskip 10pt

\item{$\bullet$} $10^{(\log 2.2 + \log 4)} =$
\item{$\bullet$} $e^{(\ln 2.2 + \ln 4)} =$

\vskip 10pt

\item{$\bullet$} $10^{(\log 12 - \log 4)} =$
\item{$\bullet$} $e^{(\ln 12 - \ln 4)} =$

\vskip 10pt

\item{$\bullet$} $10^{(2 \log 3)} =$
\item{$\bullet$} $e^{(2 \ln 3)} =$

\vskip 10pt

\item{$\bullet$} $10^{({{\log 25} \over 2})} =$
\item{$\bullet$} $e^{({{\ln 25} \over 2})} =$
\medskip

\underbar{file 01018}
%(END_QUESTION)





%(BEGIN_ANSWER)

\medskip
\item{$\bullet$} $10^{\log 3} = 3$  
\item{$\bullet$} $\log (10^8) = 8$

\vskip 10pt

\item{$\bullet$} $e^{\ln 3} = 3$ 
\item{$\bullet$} $\ln (e^8) = 8$

\vskip 10pt

\item{$\bullet$} $10^{(\log 3 + \log 5)} = 15$
\item{$\bullet$} $e^{(\ln 3 + \ln 5)} = 15$

\vskip 10pt

\item{$\bullet$} $10^{(\log 2.2 + \log 4)} = 8.8$
\item{$\bullet$} $e^{(\ln 2.2 + \ln 4)} = 8.8$

\vskip 10pt

\item{$\bullet$} $10^{(\log 12 - \log 4)} = 3$
\item{$\bullet$} $e^{(\ln 12 - \ln 4)} = 3$

\vskip 10pt

\item{$\bullet$} $10^{(2 \log 3)} = 9$
\item{$\bullet$} $e^{(2 \ln 3)} = 9$

\vskip 10pt

\item{$\bullet$} $10^{({{\log 25} \over 2})} = 5$
\item{$\bullet$} $e^{({{\ln 25} \over 2})} = 5$
\medskip

%(END_ANSWER)





%(BEGIN_NOTES)

Discuss what mathematical operations are being done with the constants in these equations, by using logarithms.  What patterns do your students notice?  Also, discuss the terms "log" and "antilog," and relate them to opamp circuits they've seen.

Ask your students whether or not they think it matters what "base" of logarithm is used in these equations.  Can they think of any other arithmetic operations to try using logarithms in this manner?

%INDEX% Exponents and logarithms, as inverse functions 
%INDEX% Logarithms, used to perform multiplication and division
%INDEX% Logarithms and exponents, as inverse functions

%(END_NOTES)


