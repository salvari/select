
%(BEGIN_QUESTION)
% Copyright 2005, Tony R. Kuphaldt, released under the Creative Commons Attribution License (v 1.0)
% This means you may do almost anything with this work of mine, so long as you give me proper credit

Plain S-R latch circuits are "set" by activating the $S$ input and de-activating the $R$ input.  Conversely, they are "reset" by activating the $R$ input and de-activating the $S$ input.  Gated latches and flip-flops, however, are a little more complex:

$$\epsfbox{02935x01.eps}$$

Describe what input conditions have to be present to force each of these multivibrator circuits to {\it set} and to {\it reset}.

\vskip 10pt

For the S-R gated latch:

\medskip
\item{$\bullet$} Set by . . .
\item{$\bullet$} Reset by . . .
\medskip

\vskip 10pt

For the S-R flip-flop:

\medskip
\item{$\bullet$} Set by . . .
\item{$\bullet$} Reset by . . .
\medskip

\underbar{file 02935}
%(END_QUESTION)





%(BEGIN_ANSWER)

For the S-R gated latch:

\medskip
\item{$\bullet$} Set by making $S$ high, $R$ low, and $E$ high.
\item{$\bullet$} Reset by making $R$ high, $S$ low, and $E$ high.
\medskip

\vskip 10pt

For the S-R flip-flop:

\medskip
\item{$\bullet$} Set by making $S$ high, $R$ low, and $C$ transition from low to high.
\item{$\bullet$} Reset by making $R$ high, $S$ low, and $C$ transition from low to high.
\medskip

%(END_ANSWER)





%(BEGIN_NOTES)

The purpose of this question is to review the definitions of "set" and "reset," as well as to differentiate latches from flip-flops.

%INDEX% Latch versus flip-flop
%INDEX% Flip-flop versus latch

%(END_NOTES)


