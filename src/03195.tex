
%(BEGIN_QUESTION)
% Copyright 2005, Tony R. Kuphaldt, released under the Creative Commons Attribution License (v 1.0)
% This means you may do almost anything with this work of mine, so long as you give me proper credit

What shape of voltage waveform would you expect to measure (using an oscilloscope) across capacitor $C_1$?  How does this waveform interact with the DC reference voltage at the wiper of $R_{pot2}$ to produce a pulse-width modulated square wave output?

\underbar{file 03195}
%(END_QUESTION)





%(BEGIN_ANSWER)

The waveform will be a "sawtooth" shape.  When compared against the DC reference voltage at $R_{pot2}$'s wiper by the LM339 comparator IC, the result is a square wave of varying duty cycle.

%(END_ANSWER)





%(BEGIN_NOTES)

Many students will initially be puzzled by the operating principle of this circuit.  The best way I have found to answer their questions, I have found, is with a multi-trace oscilloscope (preferably one that can show three traces simultaneously).  Connect one channel to the top of $C_1$, the next channel to $R_{pot2}$'s wiper, and the third to the comparator's output terminal.  A picture, as they say, is worth a thousand words.

%(END_NOTES)


