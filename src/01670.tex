
%(BEGIN_QUESTION)
% Copyright 2003, Tony R. Kuphaldt, released under the Creative Commons Attribution License (v 1.0)
% This means you may do almost anything with this work of mine, so long as you give me proper credit

\vbox{\hrule \hbox{\strut \vrule{} $\int f(x) \> dx$ \hskip 5pt {\sl Calculus alert!} \vrule} \hrule}

Plot the magnetic flux ($\Phi$) over time in the core of an ideal transformer, given a square-wave voltage applied to the primary winding:

$$\epsfbox{01670x01.eps}$$

Important: note the point in time where the square-wave source is energized.  The first pulse of applied voltage to the primary winding is not full-duration!

\underbar{file 01670}
%(END_QUESTION)





%(BEGIN_ANSWER)

$$\epsfbox{01670x02.eps}$$

Follow-up question: explain why the flux waveform is symmetrical about the zero line (perfectly balanced between positive and negative half-cycles) in this particular scenario.  How would this situation differ if the square-wave voltage source were energized at a slightly different point in time?

%(END_ANSWER)





%(BEGIN_NOTES)

Have students relate the equation $E_{L} = N {d \phi \over dt}$ to this problem, discussing the flux wave-shape in terms of rate-of-change over time.

%INDEX% Magnetic flux, relation to coil voltage over time

%(END_NOTES)


