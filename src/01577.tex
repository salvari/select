
%(BEGIN_QUESTION)
% Copyright 2003, Tony R. Kuphaldt, released under the Creative Commons Attribution License (v 1.0)
% This means you may do almost anything with this work of mine, so long as you give me proper credit

A very powerful method for discerning cause-and-effect relationships is {\it scientific method}.  One commonly accepted algorithm (series of steps) for scientific method is the following:

\medskip
\item{$1.$} Observation
\item{$2.$} Formulate an hypothesis (an educated guess)
\item{$3.$} Predict a unique consequence of that hypothesis
\item{$4.$} Test the prediction by experiment
\item{$5.$} If test fails, go back to step \#2.  If test passes, hypothesis is provisionally confirmed.
\medskip

This methodology is also very useful in technical troubleshooting, since troubleshooting is fundamentally a determination of cause for an observed effect.  Read the following description of an experienced troubleshooter diagnosing an automotive electrical problem, and match the troubleshooter's steps to those five steps previously described for scientific method:

\vskip 10pt {\narrower \noindent \baselineskip5pt

One day a car owner approached a mechanic friend of theirs with a problem.  The battery in this car seemed to be dying, requiring frequent jump-starts from other vehicles, or the application of a battery charger overnight, to be able to start reliably.  ``What could be the problem?'' asked the car owner to the mechanic.

\vskip 5pt

The mechanic considered some of the options.  One possibility was that a parasitic load was draining the battery of its charge when the car was shut off.  Another possibility was that the car's charging system (the engine-driven generator and its associated circuitry) was faulty and not charging the battery when the engine was running.  A third possibility was that the battery itself was defective, and unable to hold a charge.

\vskip 5pt

``Let's check the battery voltage with the engine stopped, and with the engine running,'' said the mechanic.  The two walked over to the car and opened the hood, then the mechanic connected a voltmeter to the battery's terminals.  It read 11.3 volts DC.  This was a 12-volt (nominal) battery.

\vskip 5pt

``Start the car,'' said the mechanic, still watching the voltmeter.  As the electric starting motor labored to turn the engine, the voltmeter's reading sagged to 9 volts.  Once the engine started and the electric starter disengaged, the voltmeter rebounded to 11.2 volts.

\vskip 5pt

``That's the problem!'' shouted the mechanic.  With that, the owner stopped the car's engine.

\par} \vskip 10pt

Explain which of the three hypotheses was confirmed by the voltmeter's reading, and how the mechanic was able to know this.

\underbar{file 01577}
%(END_QUESTION)





%(BEGIN_ANSWER)

Steps in the scientific method are indicated by superscript numbers at the end of sentences in the original narrative:

\vskip 10pt {\narrower \noindent \baselineskip5pt

One day a car owner approached a mechanic friend of theirs with a problem.  The battery in this car seemed to be dying, requiring frequent jump-starts from other vehicles, or the application of a battery charger overnight, to be able to start reliably.$^{1}$  ``What could be the problem?'' asked the car owner to the mechanic.

\vskip 5pt

The mechanic considered some of the probable causes.  One possibility was that a parasitic load was draining the battery of its charge when the car was shut off.$^{2}$  Another possibility was that the car's charging system (the engine-drive generator and its associated circuitry) was faulty and not charging the battery when the engine was running.$^{2}$  A third possibility was that the battery itself was defective, and unable to hold a charge.$^{2}$ 

\vskip 5pt

``Let's check the battery voltage with the engine stopped, and with the engine running,'' said the mechanic.$^{(3)}$  The two walked over to the car and opened the hood, then the mechanic connected a voltmeter to the battery's terminals.  It read 11.3 volts DC.  This was a 12-volt (nominal) battery.

\vskip 5pt

``Start the car,'' said the mechanic, still watching the voltmeter.  As the electric starting motor labored to turn the engine, the voltmeter's reading sagged to 9 volts.  Once the engine started and the electric starter disengaged, the voltmeter rebounded to 11.2 volts.$^{4}$

\vskip 5pt

``That's the problem!'' shouted the mechanic.$^{(5)}$  With that, the owner stopped the car's engine.

\par} \vskip 10pt

Steps 3 and 5 are labeled parenthetically because the story does not tell what the mechanic was thinking.  It doesn't indicate, for example, what the mechanic's prediction was when deciding to do a voltage check of the battery with the engine stopped and with the engine running.  I've left these steps for {\it you} to elaborate.

%(END_ANSWER)





%(BEGIN_NOTES)

Once students have successfully identified the mechanic's reasoning, ask them to explain how the prediction of battery voltage {\it uniquely} relates to only one of the three hypotheses stated.

Also, discuss whether this concludes the diagnostic procedures, or if there is more troubleshooting left to do.  What steps are recommended to take next, if any?

%INDEX% Troubleshooting strategy, applying scientific method

%(END_NOTES)


