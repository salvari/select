
%(BEGIN_QUESTION)
% Copyright 2003, Tony R. Kuphaldt, released under the Creative Commons Attribution License (v 1.0)
% This means you may do almost anything with this work of mine, so long as you give me proper credit

Explain how the addition of a propagation-delay-based one-shot circuit to the enable input of an S-R latch changes its behavior:

$$\epsfbox{01366x01.eps}$$

Specifically, reference your answer to a truth table for this circuit.

\underbar{file 01366}
%(END_QUESTION)





%(BEGIN_ANSWER)

The outputs of this device are allowed to change state only when the "clock" signal (C) is transitioning from low to high:

$$\epsfbox{01366x02.eps}$$

\vskip 10pt

Challenge question: what exactly happens in the "invalid" state for this S-R flip-flop?

%(END_ANSWER)





%(BEGIN_NOTES)

Discuss with your students what happens in this circuit when the clock signal is doing anything other than transitioning from low to high.  What condition(s) are equivalent in a regular S-R gated latch circuit?

The challenge question is especially tricky to answer.  "Invalid" states are easy to determine in regular S-R latch circuits, gated or ungated.  However, because an S-R {\it flip-flop} is only momentarily "gated" by the edge of the clock signal, the states its outputs fall to after that edge event has passed is much more difficult to determine.

%INDEX% S-R flip-flop, edge triggered
%INDEX% One-shot circuit, exploiting gate propagation delays
%INDEX% Propagation delay, gate circuits

%(END_NOTES)


