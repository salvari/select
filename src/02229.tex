
%(BEGIN_QUESTION)
% Copyright 2004, Tony R. Kuphaldt, released under the Creative Commons Attribution License (v 1.0)
% This means you may do almost anything with this work of mine, so long as you give me proper credit

Explain how the following bias networks function:

$$\epsfbox{02229x01.eps}$$

Each one has the same basic purpose, but works in a different way to accomplish it.  Describe the purpose of any biasing network in an AC signal amplifier, and comment on the different means of accomplishing this purpose employed by each of the three circuits.

\vskip 10pt

Hint: imagine if the AC signal source in each circuit were turned off (replaced with a short).  Explain how each biasing network maintains the transistor in a partially "on" state at all times even with no AC signal input.

\underbar{file 02229}
%(END_QUESTION)





%(BEGIN_ANSWER)

The purpose of any biasing network in an AC signal amplifier is to provide just enough quiescent current through the base to keep the transistor between the extremes of cutoff and saturation throughout the input signal's waveform cycle.

%(END_ANSWER)





%(BEGIN_NOTES)

All three biasing techniques are commonly used in transistor amplifier circuitry, so it behooves each student to understand them well.  In each case, resistors provide a "trickle" of current through the base of the transistor to keep it turned partially "on" at all times.

One exercise you might have your students do is come up to the board in front of the room and draw an example of this circuit, then everyone may refer to the drawn image when discussing the circuit's characteristics.

%INDEX% Biasing techniques, common-collector amplifier circuits

%(END_NOTES)


