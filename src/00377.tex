
%(BEGIN_QUESTION)
% Copyright 2003, Tony R. Kuphaldt, released under the Creative Commons Attribution License (v 1.0)
% This means you may do almost anything with this work of mine, so long as you give me proper credit

Suppose this circuit were constructed, using an inductor and an ammeter connected in series with it to measure its current:

$$\epsfbox{00377x01.eps}$$

What will happen to the inductor's current after the switch is closed?  Be as precise as you can with your answer, and explain why it does what it does.

\vskip 30pt

\underbar{file 00377}
%(END_QUESTION)





%(BEGIN_ANSWER)

The inductor's current will increase asymptotically over time when the switch is closed, with an ultimate current equal to the voltage of the source (battery) divided by the total resistance of the circuit.

%(END_ANSWER)





%(BEGIN_NOTES)

A qualitative analysis of this circuit's behavior may be performed without using any calculus, or even algebra.  Ask students to explain what happens to the voltage across the inductor as the circuit current beings to increase, and what effect that has on the rate of current rise, and so on, graphing the results for all too see.

Hint: the rate of current rise through an inductor is in direct proportion to the quantity of voltage dropped across the inductor.

%INDEX% LR time constant circuit

%(END_NOTES)


