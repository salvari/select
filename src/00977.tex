
%(BEGIN_QUESTION)
% Copyright 2003, Tony R. Kuphaldt, released under the Creative Commons Attribution License (v 1.0)
% This means you may do almost anything with this work of mine, so long as you give me proper credit

In a properly biased Class B amplifier circuit, how much voltage should be between the points {\bf A} and {\bf B}?

$$\epsfbox{00977x01.eps}$$

\underbar{file 00977}
%(END_QUESTION)





%(BEGIN_ANSWER)

0 volts, at all moments in time!

\vskip 10pt

Follow-up questions: explain how this fact may be useful in troubleshooting push-pull amplifier circuits, and also explain how this fact proves the amplifier has a voltage gain of unity (1).

%(END_ANSWER)





%(BEGIN_NOTES)

Challenge your students to apply Kirchhoff's Voltage Law to the "loops" around these two points, as such:

$$\epsfbox{00977x02.eps}$$

Knowing that the voltage between points {\bf A} and {\bf B} is zero, ask your students what the voltage drops {\it must} be across the biasing resistors.  This question foreshadows the concept of a "virtual ground" in operational amplifier circuits.  It is the idea that two or more points in a circuit may be held at the same potential without actually being connected together.  In other words, the points are {\it virtually common} rather than being {\it actually common} with one another.

%INDEX% Push-pull amplifier, biasing

%(END_NOTES)


