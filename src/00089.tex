
%(BEGIN_QUESTION)
% Copyright 2003, Tony R. Kuphaldt, released under the Creative Commons Attribution License (v 1.0)
% This means you may do almost anything with this work of mine, so long as you give me proper credit

In this circuit, three resistors receive the same amount of voltage (24 volts) from a single source.  Calculate the amount of current "drawn" by each resistor, as well as the amount of power dissipated by each resistor:

$$\epsfbox{00089x01.eps}$$

\underbar{file 00089}
%(END_QUESTION)





%(BEGIN_ANSWER)

$I_{1 \Omega} = 24$ amps

$I_{2 \Omega} = 12$ amps

$I_{3 \Omega} = 8$ amps

\vskip 10pt

$P_{1 \Omega} = 576$ watts

$P_{2 \Omega} = 288$ watts

$P_{3 \Omega} = 192$ watts

%(END_ANSWER)





%(BEGIN_NOTES)

The answers to this question may seem paradoxical to students: {\it the lowest value of resistor dissipates the greatest power}.  Math does not lie, though.

Another purpose of this question is to instill in students' minds the concept of components in a simple parallel circuit all sharing the same amount of voltage.

Challenge your students to recognize any mathematical patterns in the respective currents and power dissipations.  What can be said, mathematically, about the current drawn by the 2 $\Omega$ resistor versus the 1 $\Omega$ resistor, for example?

You might want to mention that in electrical parlance, a "heavy" load is one that draws a large amount of current, and thus has a large resistance.  This circuit, which shows how the lowest resistance in a parallel circuit consumes the most power, gives practical support to the term "heavy" used to describe loads.

%INDEX% Ohm's Law
%INDEX% Joule's Law
%INDEX% Parallel circuit

%(END_NOTES)


