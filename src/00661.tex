
%(BEGIN_QUESTION)
% Copyright 2003, Tony R. Kuphaldt, released under the Creative Commons Attribution License (v 1.0)
% This means you may do almost anything with this work of mine, so long as you give me proper credit

If a step-up transformer has a turns ratio of 3:1, calculate the following:

\medskip
\item{$\bullet$} The voltage ratio (secondary:primary)
\item{$\bullet$} The current ratio (secondary:primary)
\item{$\bullet$} The winding inductance ratio (secondary:primary)
\item{$\bullet$} The load impedance ratio (secondary:primary)
\medskip

What mathematical pattern(s) do you see between the turns ratio and these four ratios?

\underbar{file 00661}
%(END_QUESTION)





%(BEGIN_ANSWER)

\medskip
\item{$\bullet$} The voltage ratio (secondary:primary) = 3:1
\item{$\bullet$} The current ratio (secondary:primary) = 1:3
\item{$\bullet$} The winding inductance ratio (secondary:primary) = 9:1
\item{$\bullet$} The load impedance ratio (secondary:primary) = 9:1
\medskip

%(END_ANSWER)





%(BEGIN_NOTES)

Determining the voltage and current ratios should be trivial.  Calculating the impedance ratio will likely require the set-up of an example problem, based on known values of voltage and current.

The most important part of this question is the identification of mathematical patterns and trends relating the turns ratio to the requested ratios.  Of particular note are the inductance and impedance ratios.  Why are they 9:1 and not 3:1?  Ask your students what mathematical operation relates the number 3 to the number 9?  If necessary, have them work through another example problem (with a different turns ratio) to see the impedance transformation ratio there, and the resulting relationship between that ratio and the turns ratio.

%INDEX% Impedance ratio, transformer
%INDEX% Inductance ratio, transformer
%INDEX% Transformer ratios, turns vs. voltage vs. current vs. impedance vs. inductance

%(END_NOTES)


