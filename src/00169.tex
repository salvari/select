
%(BEGIN_QUESTION)
% Copyright 2003, Tony R. Kuphaldt, released under the Creative Commons Attribution License (v 1.0)
% This means you may do almost anything with this work of mine, so long as you give me proper credit

What disadvantages might there be to using oversized wire in an electrical power distribution system, where electricity is transmitted from generators (sources) to "loads" (devices requiring power) far away?  What might be bad about using wires with more than sufficient cross-sectional area for the purpose?

\underbar{file 00169}
%(END_QUESTION)





%(BEGIN_ANSWER)

An analogy: what would be bad about using oversized water pipes in a municipal water supply system?  Nothing, perhaps, once the system was operational.  However, what about the cost of building the system?

%(END_ANSWER)





%(BEGIN_NOTES)

The bad effects of oversized wiring are more difficult to discern than the bad effects of undersized wiring.  With oversized wiring, there is little electrical inefficiency, but this doesn't mean the system is inefficient in other ways!  The cost of the wiring itself, of course, is obvious, but there are other costs as well.  Brainstorm ideas with your students about the cost-inefficiencies of oversized wiring in an electrical system.

%INDEX% Wire, oversized
%INDEX% Conductors, oversized

%(END_NOTES)


