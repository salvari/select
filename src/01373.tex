
%(BEGIN_QUESTION)
% Copyright 2003, Tony R. Kuphaldt, released under the Creative Commons Attribution License (v 1.0)
% This means you may do almost anything with this work of mine, so long as you give me proper credit

Count from zero to fifteen, in binary, keeping the bits lined up in vertical columns like this:

\vskip 10pt

{\tt 0000}

{\tt 0001}

{\tt 0010}

 . . .

\vskip 150pt

Now, reading from top to bottom, notice the alternating patterns of 0's and 1's in each place (i.e. one's place, two's place, four's place, eight's place) of the four-bit binary numbers.  Note how the least significant bit alternates more rapidly than the most significant bit.  Draw a timing diagram showing the respective bits as waveforms, alternating between "low" and "high" states, and comment on the {\it frequency} of each of the bits.

\underbar{file 01373}
%(END_QUESTION)





%(BEGIN_ANSWER)

$$\epsfbox{01373x01.eps}$$

%(END_ANSWER)





%(BEGIN_NOTES)

The purpose of this question is to get students to relate the well-known binary counting sequence to electrical events: in this case, square-wave signals of different frequency.

%INDEX% Count sequence, binary

%(END_NOTES)


