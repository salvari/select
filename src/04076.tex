
%(BEGIN_QUESTION)
% Copyright 2006, Tony R. Kuphaldt, released under the Creative Commons Attribution License (v 1.0)
% This means you may do almost anything with this work of mine, so long as you give me proper credit

In metallic conductors, the dominant carriers of electric charge are {\it free electrons}, which of course are negatively charged.  Are there any examples of electric conduction where electric charge is carried by positively-charged particles?

\underbar{file 04076}
%(END_QUESTION)





%(BEGIN_ANSWER)

One example is conduction in a fluid electrolyte solution, where you often have both positively-charged ions and negatively-charged ions (moving in opposite directions!) constituting the motion of electric charge.

%(END_ANSWER)





%(BEGIN_NOTES)

Other examples exist, so do not accept the given answer as the only answer!

\vskip 10pt

Note: some students may suggest holes in semiconductors as an example of positive charge-carriers.  This is technically not true, though.  A "hole" does not exist as a real particle of matter.  It is an abstraction, used by solid-state physicists and engineers to differentiate conduction-band electron motion ("electrons") from valence-band electron motion ("holes").

%INDEX% Conventional flow versus electron flow
%INDEX% Electron flow versus conventional flow

%(END_NOTES)


