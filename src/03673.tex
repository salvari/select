
%(BEGIN_QUESTION)
% Copyright 2005, Tony R. Kuphaldt, released under the Creative Commons Attribution License (v 1.0)
% This means you may do almost anything with this work of mine, so long as you give me proper credit

A very common misconception students have about potentiometers is the relationship between resistance and direction of wiper motion.  For instance, it is common to hear a student say something like this, ``Turning the potentiometer so the wiper moves {\it up} will increase the resistance of the potentiometer.''

$$\epsfbox{03673x01.eps}$$

Explain why it does not really make sense to say something like this.

\underbar{file 03673}
%(END_QUESTION)





%(BEGIN_ANSWER)

Moving a potentiometer wiper changes {\it two} resistances in complementary directions: one resistance will increase as the other will decrease.

%(END_ANSWER)





%(BEGIN_NOTES)

Ask your students to identify which resistance (which two connection points on the potentiometer) increases and which decreases, and how they know this from the "internal views" of the potentiometers.  This is a very important thing for your students to learn.

Year after year of teaching has revealed that a great many students have difficulty grasping this concept.  This is especially true when they become accustomed to using a potentiometer as a rheostat and not as a voltage divider.  The more you can make them think carefully about the operation of a potentiometer, the better!

%INDEX% Potentiometer, concept of increasing/decreasing resistance

%(END_NOTES)


