
%(BEGIN_QUESTION)
% Copyright 2003, Tony R. Kuphaldt, released under the Creative Commons Attribution License (v 1.0)
% This means you may do almost anything with this work of mine, so long as you give me proper credit

Based on your knowledge of Lenz's Law, explain how one could construct an {\it electromagnetic brake}, whereby the energization of an electromagnet coil would produce mechanical "drag" on a rotating shaft without the need for contact between the shaft and a brake pad.

\underbar{file 01786}
%(END_QUESTION)





%(BEGIN_ANSWER)

$$\epsfbox{01786x01.eps}$$

Follow-up question: describe some of the advantages and disadvantages that a magnetic brake would have, compared to mechanical brakes (where physical contact produces friction on the shaft).

\vskip 10pt

Challenge question: normal (mechanical) brakes become hot during operation, due to the friction they employ to produce drag.  Will an electromechanical brake produce heat as well, given that there is no physical contact to create friction?

%(END_ANSWER)





%(BEGIN_NOTES)

Electromagnetic brakes are very useful devices in industry.  One interesting application I've seen for this technology is the mechanical load for an automotive dynamometer, where a car is driven onto a set of steel rollers, with one roller coupled to a large metal disk (with electromagnets on either side).  By varying the amount of current sent to the electromagnets, the degree of mechanical drag may be varied.

Incidentally, this disk becomes very hot when in use, because the automobile's power output cannot simply vanish -- it must be converted into a different form of energy in the braking mechanism, and heat it is.

%INDEX% Brake, electromagnetic
%INDEX% Electromagnetic brake
%INDEX% Lenz's Law

%(END_NOTES)


