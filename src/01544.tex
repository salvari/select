
%(BEGIN_QUESTION)
% Copyright 2003, Tony R. Kuphaldt, released under the Creative Commons Attribution License (v 1.0)
% This means you may do almost anything with this work of mine, so long as you give me proper credit

Define the following noise types, according to how each one is generated in electronic circuits:

\medskip
\item{$\bullet$} Shot noise
\item{$\bullet$} Thermal ("Johnson") noise
\item{$\bullet$} Flicker (or 1/f) noise
\medskip

Also, identify the major factor responsible for the amplitude of each noise type.

\underbar{file 01544}
%(END_QUESTION)





%(BEGIN_ANSWER)

{\it Shot noise} is generated by individual electrons "jumping" across some sort of barrier potential as they travel through a conducting substance.  Shot noise is proportional to the amount of electric {\it current} going through a conductor.

\vskip 10pt

{\it Thermal noise}, also known as {\it Johnson noise}, is caused by the random motion of electrons due to thermal energy.  As one might guess, this type of noise is proportional to conductor {\it temperature}.

\vskip 10pt

{\it Flicker noise}, or {\it 1/f noise}, is characterized by a magnitude that is inversely proportional to frequency.  Little is known about the origins of this type of noise, but it is proportional to the amount of DC {\it current}, just like shot noise, and so may be mitigated using the same controls.

%(END_ANSWER)





%(BEGIN_NOTES)

Noise is a very complex subject in electrical engineering, and most likely beyond the scope of your students' coursework.  This question is really just scratching the surface of noise theory. 

For a good overview of electrical noise, consult Texas Instruments' online manual, {\it Op Amps For Everyone}, sections 10-1 through 10-12.

%INDEX% Noise types, defined

%(END_NOTES)


