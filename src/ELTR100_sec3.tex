
\centerline{\bf ELTR 100 (DC 1), section 3} \bigskip 
 
\vskip 10pt

\noindent
{\bf Recommended schedule}

\vskip 5pt

%%%%%%%%%%%%%%%
\hrule \vskip 5pt
\noindent
\underbar{Day 1}

\hskip 10pt Topics: {\it Parallel circuits, current sources, and troubleshooting}
 
\hskip 10pt Questions: {\it 1 through 20}
 
\hskip 10pt Lab Exercises: {\it Parallel resistances (question 61)}
 
%INSTRUCTOR \hskip 10pt {\bf Demo: Show how to set up regulated power supply for sourcing fixed amount of current}

\vskip 10pt
%%%%%%%%%%%%%%%
\hrule \vskip 5pt
\noindent
\underbar{Day 2}

\hskip 10pt Topics: {\it Parallel circuits, Kirchhoff's Current Law, and chemical batteries}
 
\hskip 10pt Questions: {\it 21 through 40}
 
\hskip 10pt Lab Exercise: {\it Parallel DC resistor circuit (question 62)}
 
%INSTRUCTOR \hskip 10pt {\bf MIT 6.002 video clip: Disk 1, Lecture 2; Kirchhoff's Current Law 10:36 to 12:51}

%INSTRUCTOR \hskip 10pt {\bf MIT 8.02 video clip: Disk 2, Lecture 10; Battery demonstration 11:38 to 15:00}

%INSTRUCTOR \hskip 10pt {\bf MIT 8.02 video clip: Disk 2, Lecture 10; Shorting a 9-V battery 26:47 to 28:29}

%INSTRUCTOR \hskip 10pt {\bf MIT 8.02 video clip: Disk 2, Lecture 10; Shorting a 12-V car battery 29:22 to 31:18}

\vskip 10pt
%%%%%%%%%%%%%%%
\hrule \vskip 5pt
\noindent
\underbar{Day 3}

\hskip 10pt Topics: {\it Parallel circuits, current divider circuits, and temperature coefficient of resistance}
 
\hskip 10pt Questions: {\it 41 through 60}
 
\hskip 10pt Lab Exercise: {\it Parallel DC resistor circuit (question 63)}
 
%INSTRUCTOR \hskip 10pt {\bf Demo: Measure resistance of a magnet wire spool, both cold and room temperature}

\vskip 10pt
%%%%%%%%%%%%%%%
\hrule \vskip 5pt
\noindent
\underbar{Day 4}

\hskip 10pt Exam 3: {\it includes Parallel DC resistor circuit performance assessment}
 
\hskip 10pt {\bf Troubleshooting assessment due:} {\it Simple lamp circuit}
 
\hskip 10pt Question 64: Troubleshooting log
 
\hskip 10pt Question 65: Sample troubleshooting assessment grading criteria
 
\hskip 10pt {\bf Project due:} {\it Solder-together electronic kit}
 
\vskip 10pt
%%%%%%%%%%%%%%%
\hrule \vskip 5pt
\noindent
\underbar{Troubleshooting practice problems}

\hskip 10pt Questions: {\it 66 through 75}
 
\vskip 10pt
%%%%%%%%%%%%%%%
\hrule \vskip 5pt
\noindent
\underbar{General concept practice and challenge problems}

\hskip 10pt Questions: {\it 76 through the end of the worksheet}
 
\vskip 10pt
%%%%%%%%%%%%%%%





\vfil \eject

\centerline{\bf ELTR 100 (DC 1), section 3} \bigskip 
 
\vskip 10pt

\noindent
{\bf Skill standards addressed by this course section}

\vskip 5pt

%%%%%%%%%%%%%%%
\hrule \vskip 10pt
\noindent
\underbar{EIA {\it Raising the Standard; Electronics Technician Skills for Today and Tomorrow}, June 1994}

\vskip 5pt

\medskip
\item{\bf B} {\bf Technical Skills -- DC circuits}
\item{\bf B.02} Demonstrate an understanding of principles and operation of batteries.
\item{\bf B.03} Demonstrate an understanding of the meaning of and relationships among and between voltage, current, resistance and power in DC circuits.
\item{\bf B.05} Demonstrate an understanding of application of Ohm's Law to series, parallel, and series-parallel circuits.  {\it Partially met -- parallel circuits only.}
\item{\bf B.11} Understand principles and operations of DC parallel circuits.
\item{\bf B.12} Fabricate and demonstrate DC parallel circuits.
\item{\bf B.13} Troubleshoot and repair DC parallel circuits.
\medskip

\vskip 5pt

\medskip
\item{\bf B} {\bf Basic and Practical Skills -- Communicating on the Job}
\item{\bf B.01} Use effective written and other communication skills.  {\it Met by group discussion and completion of labwork.}
\item{\bf B.03} Employ appropriate skills for gathering and retaining information.  {\it Met by research and preparation prior to group discussion.}
\item{\bf B.04} Interpret written, graphic, and oral instructions.  {\it Met by completion of labwork.}
\item{\bf B.06} Use language appropriate to the situation.  {\it Met by group discussion and in explaining completed labwork.}
\item{\bf B.07} Participate in meetings in a positive and constructive manner.  {\it Met by group discussion.}
\item{\bf B.08} Use job-related terminology.  {\it Met by group discussion and in explaining completed labwork.}
\item{\bf B.10} Document work projects, procedures, tests, and equipment failures.  {\it Met by project construction and/or troubleshooting assessments.}
\item{\bf C} {\bf Basic and Practical Skills -- Solving Problems and Critical Thinking}
\item{\bf C.01} Identify the problem.  {\it Met by research and preparation prior to group discussion.}
\item{\bf C.03} Identify available solutions and their impact including evaluating credibility of information, and locating information.  {\it Met by research and preparation prior to group discussion.}
\item{\bf C.07} Organize personal workloads.  {\it Met by daily labwork, preparatory research, and project management.}
\item{\bf C.08} Participate in brainstorming sessions to generate new ideas and solve problems.  {\it Met by group discussion.}
\item{\bf D} {\bf Basic and Practical Skills -- Reading}
\item{\bf D.01} Read and apply various sources of technical information (e.g. manufacturer literature, codes, and regulations).  {\it Met by research and preparation prior to group discussion.}
\item{\bf E} {\bf Basic and Practical Skills -- Proficiency in Mathematics}
\item{\bf E.01} Determine if a solution is reasonable.
\item{\bf E.02} Demonstrate ability to use a simple electronic calculator.
\item{\bf E.05} Solve problems and [sic] make applications involving integers, fractions, decimals, percentages, and ratios using order of operations.
\item{\bf E.06} Translate written and/or verbal statements into mathematical expressions.
\item{\bf E.12} Interpret and use tables, charts, maps, and/or graphs.
\item{\bf E.13} Identify patterns, note trends, and/or draw conclusions from tables, charts, maps, and/or graphs.
\item{\bf E.15} Simplify and solve algebraic expressions and formulas.
\item{\bf E.16} Select and use formulas appropriately.
\item{\bf E.17} Understand and use scientific notation.
\medskip

%%%%%%%%%%%%%%%





\vfil \eject

\centerline{\bf ELTR 100 (DC 1), section 3} \bigskip 
 
\vskip 10pt

\noindent
{\bf Common areas of confusion for students}

\vskip 5pt

%%%%%%%%%%%%%%%
\hrule \vskip 5pt

\vskip 10pt

\noindent
{\bf Difficult concept: } {\it Using Ohm's Law in context.}

When applying Ohm's Law ($E = IR$ ; $I = {E \over R}$ ; $R = {E \over I}$) to circuits containing multiple resistances, students often mix contexts of voltage, current, and resistance.  Whenever you use any equation describing a physical phenomenon, be sure that each variable of that equation relates to the proper real-life value in the problem you're working on solving.  For example, when calculating the current through resistor $R_2$, you must be sure that the values for voltage and resistance are appropriate for that resistor and not some other resistor in the circuit.  If you are calculating $I_{R_2}$ using the Ohm's Law equation $I = {E \over R}$, then you must use the value of {\it that resistor's} voltage ($E_{R_2}$) and {\it that resistor's} resistance ($R_2$), not some other voltage and/or resistance value(s).  Some students have an unfortunate tendency to overlook context when seeking values to substitute in place of variables in Ohm's Law problems, and this leads to incorrect results.

\vskip 10pt

\noindent
{\bf Difficult concept: } {\it The path of least resistance.}

A common electrical adage is that ``Electricity always follows the path of least resistance.''  This is only partly true.  Given two or more paths of differing resistance, more current will flow through the path of least resistance, but this does not mean there will be zero current through the other path(s)!  The saying should probably be revised to state that ``Electricity {\it proportionately} favors the path of least resistance.''  Be careful not to blindly follow some truism about electricity, but always think quantitatively about voltage and current values if possible!  Perform some Ohm's Law calculations on parallel circuit branches to convince yourself of how current "favors" lesser resistances, or measure currents in your own safe, low-voltage circuits to see firsthand how it works.

\vskip 10pt

\noindent
{\bf Difficult concept: } {\it Parallel resistances diminishing.}

Series resistances are easy to conceptually manage: it just makes sense that multiple resistors connected end-to-end will result in a total resistance equal to the sum of the individual resistances.  However, parallel resistances are not as intuitive.  That some total quantity (total resistance) actually {\it decreases} as individual quantities accumulate (resistors connected in parallel) may seem impossible, usually because the word "total" is generally associated with addition.

