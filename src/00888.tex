
%(BEGIN_QUESTION)
% Copyright 2003, Tony R. Kuphaldt, released under the Creative Commons Attribution License (v 1.0)
% This means you may do almost anything with this work of mine, so long as you give me proper credit

Suppose you had the boring job of manually maintaining the output voltage of a DC generator constant.  Your one and only control over voltage is the setting of a rheostat:

$$\epsfbox{00888x01.eps}$$

What would you have to do to maintain the load voltage constant if the load resistance changed so as to draw more current?  Being that your only control over load voltage is the adjustment of a variable resistance in series with the generator, what does this imply about the generator's output voltage (directly across the generator terminals), compared to the target load voltage?

\underbar{file 00888}
%(END_QUESTION)





%(BEGIN_ANSWER)

In order to increase the load voltage, you must decrease the resistance of the rheostat.  In order for this scheme to work, the generator's voltage must be greater than the target load voltage.

Note: this general voltage control scheme is known as {\it series regulation}, where a series resistance is varied to control voltage to a load.

%(END_ANSWER)





%(BEGIN_NOTES)

The direction of rheostat adjustment should be obvious, as is the fact that the generator's voltage must be at least as high as the intended (target) load voltage.  However, it may not be obvious to all that the generator's voltage cannot merely be equal to the intended load voltage.

To illustrate the necessity of this, ask your students how the system would work if the generator's output voltage was exactly equal to the intended load voltage.  Emphasize the fact that the generator is not perfect: it has its own internal resistance, the value of which cannot be changed by you.  What position would the rheostat have to be in, under these conditions, in order to maintain target voltage at the load?  Could the target voltage be maintained at all?

%INDEX% Voltage regulator, series (manual)

%(END_NOTES)


