
%(BEGIN_QUESTION)
% Copyright 2003, Tony R. Kuphaldt, released under the Creative Commons Attribution License (v 1.0)
% This means you may do almost anything with this work of mine, so long as you give me proper credit

The power dissipation of a JFET may be calculated by the following formula:

$$P = V_{DS}I_D + V_{GS}I_G$$

For all practical purposes, though, this formula may be simplified and re-written as follows:

$$P = V_{DS}I_D$$

Explain why the second term of the original equation ($V_{GS}I_G$) may be safely ignored for a junction field-effect transistor.

\underbar{file 02063}
%(END_QUESTION)





%(BEGIN_ANSWER)

$I_G$ is zero for all practical purposes.

%(END_ANSWER)





%(BEGIN_NOTES)

This question asks students to look beyond the equation to the device itself and think about the relative magnitudes of each variable.  Many equations in electronics (and other sciences!) may be similarly simplified by recognizing the relative magnitudes of variables and eliminating those whose overall effect on the equation's result will be negligible.  Of course, what constitutes "negligible" will vary from context to context.

%INDEX% Power dissipation, JFET

%(END_NOTES)


