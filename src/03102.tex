
%(BEGIN_QUESTION)
% Copyright 2005, Tony R. Kuphaldt, released under the Creative Commons Attribution License (v 1.0)
% This means you may do almost anything with this work of mine, so long as you give me proper credit

Suppose you needed to choose a fixed resistor value ($R$) to make a voltage divider circuit, given a known potentiometer resistance value, the source voltage value, and the desired range of adjustment:

$$\epsfbox{03102x01.eps}$$

Solve for $R$, and show the equation you set up in order to do it.

\vskip 10pt

Hint: remember the series resistor voltage divider formula . . .

$$V_R = V_{total}\left( {R \over R_{total}} \right)$$

\underbar{file 03102}
%(END_QUESTION)





%(BEGIN_ANSWER)

$R =$ 20.588 k$\Omega$

%(END_ANSWER)





%(BEGIN_NOTES)

Be sure to have your students set up their equations in front of the class so everyone can see how they did it.  Some students may opt to apply Ohm's Law to the solution of $R$, which is good, but for the purpose of developing equations to fit problems it might not be the best solution.  Challenge your students to come up with a {\it single equation} that solves for $R$, with all known quantities on the other side of the "equal" sign.

%INDEX% Simultaneous equations
%INDEX% Systems of nonlinear equations

%(END_NOTES)


