
%(BEGIN_QUESTION)
% Copyright 2004, Tony R. Kuphaldt, released under the Creative Commons Attribution License (v 1.0)
% This means you may do almost anything with this work of mine, so long as you give me proper credit

When a capacitor is to be connected in parallel with an inductive AC load to correct for lagging power factor, it is important to be able to calculate the reactive power of the capacitor ($Q_C$).  Write at least one equation for calculating the reactive power of a capacitor (in VARs) given the capacitor's reactance ($X_C$) at the line frequency.

\underbar{file 02189}
%(END_QUESTION)





%(BEGIN_ANSWER)

$$Q_C = {E^2 \over X_C} \hbox{\hskip 40pt} Q_C = I^2 X_C$$

\vskip 10pt

Follow-up question: which of the two equations shown above would be easiest to use in calculating the reactive power of a capacitor given the following information?

$$\epsfbox{02189x01.eps}$$

%(END_ANSWER)





%(BEGIN_NOTES)

This step seems to be one of the most difficult for students to grasp as they begin to learn to correct for power factor in AC circuits, so I wrote a question specifically focusing on it.  Once students calculate the amount of reactive power consumed by the load ($Q_{load}$), they may realize the capacitor needs to produce the same ($Q_C$), but they often become mired in confusion trying to take the next step(s) in determining capacitor size.

%INDEX% Reactive power, calculating

%(END_NOTES)


