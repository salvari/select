
%(BEGIN_QUESTION)
% Copyright 2003, Tony R. Kuphaldt, released under the Creative Commons Attribution License (v 1.0)
% This means you may do almost anything with this work of mine, so long as you give me proper credit

There is a problem somewhere in this relay logic circuit.  Lamp 2 operates exactly as it should, but lamp 1 never turns on.  Identify all possible failures in the circuit that could cause this problem, and then explain how you would troubleshoot the problem as efficiently as possible (taking the least amount of electrical measurements to identify the specific problem).

$$\epsfbox{01296x01.eps}$$

\underbar{file 01296}
%(END_QUESTION)





%(BEGIN_ANSWER)

This is a problem worthy of a good in-class discussion with your peers!  Of course, several things could be wrong in this circuit to cause lamp 1 to never energize.  When you explain what measurements you would take in isolating the problem, be sure to describe whether or not you are actuating either of the pushbutton switches when you take those measurements. 

%(END_ANSWER)





%(BEGIN_NOTES)

Be sure to leave plenty of classroom time for a discussion on troubleshooting this circuit.  Electrical troubleshooting is a difficult-to-develop skill, and it takes lots of time for some people to acquire.  Being one of the most valuable skills a technical person can possess, it is well worth the time invested!

The challenge question is very practical.  Too many times I have seen students take meter measurements when their other senses provide enough data to render that step unnecessary.  While there is nothing wrong with using your meter to confirm a suspicion, the best troubleshooters use all their senses (safely, of course) in the isolation of system faults.

%INDEX% Troubleshooting, relay circuit

%(END_NOTES)


