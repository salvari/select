
%(BEGIN_QUESTION)
% Copyright 2003, Tony R. Kuphaldt, released under the Creative Commons Attribution License (v 1.0)
% This means you may do almost anything with this work of mine, so long as you give me proper credit

Shown here is an illustration of a large "stud mount" type of SCR, where the body is threaded so as to be fastened to a metal base like a bolt threads into a nut:

$$\epsfbox{01087x01.eps}$$

With no test instrument other than a simple continuity tester (battery and light bulb connected in series, with two test leads), how could you determine the identities of the three terminals on this SCR? 

\vskip 10pt

Hint:  The threaded metal base of the SCR constitutes one of the three terminals.

\underbar{file 01087}
%(END_QUESTION)





%(BEGIN_ANSWER)

The smallest terminal (on top) is the gate.  The identities of cathode and anode may be determined by connecting one test lead to the gate terminal, and touching the other test lead to either of the other terminals.

%(END_ANSWER)





%(BEGIN_NOTES)

Ask your students how they know the gate terminal is the smallest one.  Why would it be the smallest?  Does it {\it have} to be the smallest terminal?  Why?  Also, ask them what continuity indication would distinguish cathode from anode in the continuity test described in the answer.

%INDEX% SCR, terminal identification

%(END_NOTES)


