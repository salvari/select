
%(BEGIN_QUESTION)
% Copyright 2003, Tony R. Kuphaldt, released under the Creative Commons Attribution License (v 1.0)
% This means you may do almost anything with this work of mine, so long as you give me proper credit

How many binary bits are needed to count up to the number one million three hundred thousand seven hundred sixty two?  Try to answer this question without converting this quantity into binary form, and then explain the mathematical procedure!

\underbar{file 01228}
%(END_QUESTION)





%(BEGIN_ANSWER)

Twenty one bits.

\vskip 10pt

Hint: the answer consists in solving this equation:

$$1300762 = 2^n$$

\noindent
Where,

$n =$ The number of binary bits necessary to count up to 1,300,762.

%(END_ANSWER)





%(BEGIN_NOTES)

This question is a test of whether or not students know how to algebraically solve for variable exponents.  I'm not going to suggest how this is done (lest I rob a student of the learning experience that comes from researching the answer on their own), but I will say that it is a {\it very} useful algebraic technique, once mastered.

If students still haven't found a solution after doing their research, suggest they try to solve a simpler problem:

\vskip 10pt {\narrower \noindent \baselineskip5pt

How many {\it decimal digits} are needed to count up to the number one million three hundred thousand seven hundred sixty two?  

\par} \vskip 10pt

This question is trivial to answer (7 decimal digits), since we're all familiar with decimal numeration.  However, the real learning takes place when students write a mathematical expression for solving this problem, similar to the one written in the answer for the binary problem.  Once they have that expression written, ask them what algebraic techniques could be used to solve for the exponent's value.

%INDEX% Maximum count, binary

%(END_NOTES)


