
%(BEGIN_QUESTION)
% Copyright 2003, Tony R. Kuphaldt, released under the Creative Commons Attribution License (v 1.0)
% This means you may do almost anything with this work of mine, so long as you give me proper credit

When measuring the resistance of a JFET from source to drain, the ohmmeter reading may be seen to change simply by touching the gate terminal with your finger:

$$\epsfbox{02062x01.eps}$$

Such a degree of sensitivity is unheard of in bipolar junction transistors.  Explain why the JFET is so sensitive, and comment on what advantages and disadvantages this gives the JFET as an electronic device.

\underbar{file 02062}
%(END_QUESTION)





%(BEGIN_ANSWER)

The gate-channel PN junction of a JFET normally operates in reverse-bias mode, whereas bipolar transistors require a forward current for emitter-collector conduction.

%(END_ANSWER)





%(BEGIN_NOTES)

The stated answer is purposefully vague, to force students to think and to express the answer in their own words.  

It is quite easy to demonstrate this sensitivity in the classroom with nothing more than an ohmmeter and a JFET, and so I encourage you to set this up as a demonstration for your students.

%INDEX% JFET, sensitivity to gate voltage

%(END_NOTES)


