
%(BEGIN_QUESTION)
% Copyright 2005, Tony R. Kuphaldt, released under the Creative Commons Attribution License (v 1.0)
% This means you may do almost anything with this work of mine, so long as you give me proper credit

Suppose you needed to choose two resistance values to make a voltage divider with a limited adjustment range: 

$$\epsfbox{03105x01.eps}$$

Set up a system of simultaneous equations to solve for both $R_1$ and $R_2$, and show how you arrived at the solutions for each.

\vskip 10pt

Hint: remember the series resistor voltage divider formula . . .

$$V_R = V_{total}\left( {R \over R_{total}} \right)$$

\underbar{file 03105}
%(END_QUESTION)





%(BEGIN_ANSWER)

$R_1$ = 5.25 k$\Omega$

\vskip 10pt

$R_2$ = 2.25 k$\Omega$

%(END_ANSWER)





%(BEGIN_NOTES)

Be sure to have your students set up their equations in front of the class so everyone can see how they did it.  Some students may opt to apply Ohm's Law to the solution of both resistors, which is good, but for the purpose of developing equations to fit problems it might not be the best solution.  Challenge your students to come up with a {\it set of equations} that solve for $R_1$ and $R_2$, then use techniques for solution of simultaneous equations to arrive at solutions for each.

%INDEX% Simultaneous equations
%INDEX% Systems of nonlinear equations

%(END_NOTES)


