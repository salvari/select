
%(BEGIN_QUESTION)
% Copyright 2004, Tony R. Kuphaldt, released under the Creative Commons Attribution License (v 1.0)
% This means you may do almost anything with this work of mine, so long as you give me proper credit

Suppose that a single-phase AC electric motor is performing mechanical work at a rate of 45 horsepower.  This equates to 33.57 kW of power, given the equivalence of watts to horsepower (1 HP $\approx$ 746 W).

Calculate the amount of line current necessary to power this motor if the line voltage is 460 volts, assuming 100\% motor efficiency and a power factor of 1.

Now re-calculate the necessary line current for this motor if its power factor drops to 0.65.  Assume the same efficiency (100\%) and the same amount of mechanical power (45 HP).

What do these calculations indicate about the importance of maintaining a high power factor value in an AC circuit?

\underbar{file 02182}
%(END_QUESTION)





%(BEGIN_ANSWER)

P.F. = 1 ; current = 72.98 amps

\vskip 10pt

P.F. = 0.65 ; current = 112.3 amps

\vskip 10pt

Follow-up question: what is the "extra" current in the 0.65 power factor scenario doing, if not contributing to the motor's mechanical power output?

%(END_ANSWER)





%(BEGIN_NOTES)

Points of discussion for this question should be rather obvious: why is a current of 112.3 amps worse than a current of 72.98 amps when the exact same amount of mechanical work is being done?  What circuit components would have to be oversized to accommodate this extra current?

%INDEX% Power factor, effect on line current
%INDEX% Power factor, undesirability of low values

%(END_NOTES)


