
%(BEGIN_QUESTION)
% Copyright 2005, Tony R. Kuphaldt, released under the Creative Commons Attribution License (v 1.0)
% This means you may do almost anything with this work of mine, so long as you give me proper credit

Complete the table of voltages for this opamp "voltage follower" circuit:

$$\epsfbox{02289x01.eps}$$

% No blank lines allowed between lines of an \halign structure!
% I use comments (%) instead, so that TeX doesn't choke.

$$\vbox{\offinterlineskip
\halign{\strut
\vrule \quad\hfil # \ \hfil & 
\vrule \quad\hfil # \ \hfil \vrule \cr
\noalign{\hrule}
%
% First row
$V_{in}$ & $V_{out}$ \cr
%
\noalign{\hrule}
%
% Second row
0 volts & 0 volts \cr
%
\noalign{\hrule}
%
% Third row
+5 volts &  \cr
%
\noalign{\hrule}
%
% Fourth row
+10 volts &  \cr
%
\noalign{\hrule}
%
% Fifth row
+15 volts &  \cr
%
\noalign{\hrule}
%
% Sixth row
+20 volts &  \cr
%
\noalign{\hrule}
%
% Seventh row
-5 volts &  \cr
%
\noalign{\hrule}
%
% Eighth row
-10 volts &  \cr
%
\noalign{\hrule}
%
% Ninth row
-15 volts &  \cr
%
\noalign{\hrule}
%
% Tenth row
-20 volts &  \cr
%
\noalign{\hrule}
} % End of \halign 
}$$ % End of \vbox

\underbar{file 02289}
%(END_QUESTION)





%(BEGIN_ANSWER)

% No blank lines allowed between lines of an \halign structure!
% I use comments (%) instead, so that TeX doesn't choke.

$$\vbox{\offinterlineskip
\halign{\strut
\vrule \quad\hfil # \ \hfil & 
\vrule \quad\hfil # \ \hfil \vrule \cr
\noalign{\hrule}
%
% First row
$V_{in}$ & $V_{out}$ \cr
%
\noalign{\hrule}
%
% Second row
0 volts & 0 volts \cr
%
\noalign{\hrule}
%
% Third row
+5 volts & +5 volts \cr
%
\noalign{\hrule}
%
% Fourth row
+10 volts & +10 volts \cr
%
\noalign{\hrule}
%
% Fifth row
+15 volts & +15 volts \cr
%
\noalign{\hrule}
%
% Sixth row
+20 volts & +15 volts \cr
%
\noalign{\hrule}
%
% Seventh row
-5 volts & -5 volts \cr
%
\noalign{\hrule}
%
% Eighth row
-10 volts & -10 volts \cr
%
\noalign{\hrule}
%
% Ninth row
-15 volts & -15 volts \cr
%
\noalign{\hrule}
%
% Tenth row
-20 volts & -15 volts \cr
%
\noalign{\hrule}
} % End of \halign 
}$$ % End of \vbox

\vskip 10pt

Follow-up question: the output voltage values given in this table are ideal.  A real opamp would probably not be able to achieve even what is shown here, due to idiosyncrasies of these amplifier circuits.  Explain what would probably be different in a {\it real} opamp circuit from what is shown here.

%(END_ANSWER)





%(BEGIN_NOTES)

A common mistake I see students new to opamps make is assuming that the output voltage will magically attain whatever value the gain equation predicts, with no regard for power supply rail voltage limits.

Another good follow-up question to ask your students is this: "How much voltage is there between the two input terminals in each of the situations described in the table?"  They will find that the "golden rule" of closed-loop opamp circuits can be violated!

If students have difficulty answering the follow-up question, drop these two hints: (1) {\it Rail-to-rail output swing} and (2) {\it Latch-up}.

%INDEX% Opamp, output voltage limits

%(END_NOTES)


