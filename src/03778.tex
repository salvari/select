
%(BEGIN_QUESTION)
% Copyright 2005, Tony R. Kuphaldt, released under the Creative Commons Attribution License (v 1.0)
% This means you may do almost anything with this work of mine, so long as you give me proper credit

Predict how the operation of this current regulator circuit will be affected as a result of the following faults.  Consider each fault independently (i.e. one at a time, no multiple faults):

$$\epsfbox{03778x01.eps}$$

\medskip
\item{$\bullet$} Resistor $R_1$ fails open:
\vskip 5pt
\item{$\bullet$} Resistor $R_2$ fails open:
\vskip 5pt
\item{$\bullet$} Solder bridge (short) across resistor $R_2$:
\vskip 5pt
\item{$\bullet$} Zener diode $D_1$ fails shorted:
\vskip 5pt
\item{$\bullet$} Zener diode $D_1$ fails open:
\vskip 5pt
\item{$\bullet$} Load fails shorted:
\vskip 5pt
\item{$\bullet$} Wire between opamp output and transistor base breaks open:
\medskip

For each of these conditions, explain {\it why} the resulting effects will occur.

\underbar{file 03778}
%(END_QUESTION)





%(BEGIN_ANSWER)

\medskip
\item{$\bullet$} Resistor $R_1$ fails open: {\it Load current falls to zero.}
\vskip 5pt
\item{$\bullet$} Resistor $R_2$ fails open: {\it Load current falls to zero.}
\vskip 5pt
\item{$\bullet$} Solder bridge (short) across resistor $R_2$: {\it Load current increases.}
\vskip 5pt
\item{$\bullet$} Zener diode $D_1$ fails shorted: {\it Load current falls to zero.}
\vskip 5pt
\item{$\bullet$} Zener diode $D_1$ fails open: {\it Load current increases.}
\vskip 5pt
\item{$\bullet$} Load fails shorted: {\it Load current remains the same.}
\vskip 5pt
\item{$\bullet$} Wire between opamp output and transistor base breaks open: {\it Load current falls to zero.}
\medskip

\vskip 10pt

Follow-up question: which of the two opamp power terminals ($V_{supply}$ or Ground) carries more current during normal operation, and why?

%(END_ANSWER)





%(BEGIN_NOTES)

The purpose of this question is to approach the domain of circuit troubleshooting from a perspective of knowing what the fault is, rather than only knowing what the symptoms are.  Although this is not necessarily a realistic perspective, it helps students build the foundational knowledge necessary to diagnose a faulted circuit from empirical data.  Questions such as this should be followed (eventually) by other questions asking students to identify likely faults based on measurements.

%INDEX% Troubleshooting, predicting effects of fault in current regulator circuit (sourcing)

%(END_NOTES)


