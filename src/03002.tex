
%(BEGIN_QUESTION)
% Copyright 2005, Tony R. Kuphaldt, released under the Creative Commons Attribution License (v 1.0)
% This means you may do almost anything with this work of mine, so long as you give me proper credit

Calculate the voltage gain of the following opamp circuit with the potentiometer turned fully up, precisely mid-position, and fully down:

$$\epsfbox{03002x01.eps}$$

\medskip
\goodbreak
\item{$\bullet$} $A_V$ (pot fully up) = 
\item{$\bullet$} $A_V$ (pot mid-position) = 
\item{$\bullet$} $A_V$ (pot fully down) = 
\medskip

\underbar{file 03002}
%(END_QUESTION)





%(BEGIN_ANSWER)

\medskip
\item{$\bullet$} $A_V$ (pot fully up) = +1
\item{$\bullet$} $A_V$ (pot mid-position) = 0
\item{$\bullet$} $A_V$ (pot fully down) = -1
\medskip

\vskip 10pt

Follow-up question: can you think of any interesting applications for a circuit such as this?

\vskip 10pt

Challenge question: modify the circuit so that the range of voltage gain adjustment is -6 to +6 instead of -1 to +1.

%(END_ANSWER)





%(BEGIN_NOTES)

Ask your students how they approached this problem.  How, exactly, did they choose to set it up so the solution became most apparent?

%INDEX% Opamp voltage amplifier circuit (gain variable from -1 to +1)
%INDEX% Opamp, noninverting versus inverting amplifier circuits

%(END_NOTES)


