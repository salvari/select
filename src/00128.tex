
%(BEGIN_QUESTION)
% Copyright 2003, Tony R. Kuphaldt, released under the Creative Commons Attribution License (v 1.0)
% This means you may do almost anything with this work of mine, so long as you give me proper credit

If a battery and switch were connected to one end of a 10-mile long cable, and two oscilloscopes were used to record voltage at either end of the cable, how far apart in time would those two pulses be, assuming a propagation velocity equal to 69\% the speed of light (in other words, the cable has a {\it velocity factor} equal to 0.69)?

$$\epsfbox{00128x01.eps}$$

\underbar{file 00128}
%(END_QUESTION)





%(BEGIN_ANSWER)

77.80 microseconds

%(END_ANSWER)





%(BEGIN_NOTES)

Students should realize by the wording of the question that the voltage signal probably does not arrive at the far end of the cable instantaneously after the switch is closed.  Although 69\% of the speed of light is still very, very fast, it is not instant: there will be a measurable time delay.

%INDEX% Transmission line, as delay element
%INDEX% Velocity factor, transmission line

%(END_NOTES)


