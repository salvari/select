
%(BEGIN_QUESTION)
% Copyright 2004, Tony R. Kuphaldt, released under the Creative Commons Attribution License (v 1.0)
% This means you may do almost anything with this work of mine, so long as you give me proper credit

The following circuit is useful as a {\it current regulator}, the regulated current setpoint being established by the value of the resistor and the JFET's $V_{GS(off)}$ parameter:

$$\epsfbox{02251x01.eps}$$

Despite wide variations in $V_{DD}$, the current in this circuit will remain relatively constant.  The reason this circuit works as it does is {\it negative feedback}.  Explain what causes negative feedback to occur in this circuit, and why it has a stabilizing effect on the current.

\underbar{file 02251}
%(END_QUESTION)





%(BEGIN_ANSWER)

The resistor creates a voltage drop that tries to pinch off the JFET as current increases.  To put it in colloquial terms, "when the current {\it zigs}, the JFET {\it zags}."

%(END_ANSWER)





%(BEGIN_NOTES)

This is a simple example of negative feedback in action: one that does not require any AC analysis and all its associated considerations.  Here it is just DC at work, and the students must determine what it is the JFET/resistor combination does in response to attempted changes in current that makes the current so stable.

%INDEX% Current regulator, JFET
%INDEX% Current source, JFET
%INDEX% Negative feedback, in JFET current regulator circuit

%(END_NOTES)


