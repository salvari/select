
%(BEGIN_QUESTION)
% Copyright 2005, Tony R. Kuphaldt, released under the Creative Commons Attribution License (v 1.0)
% This means you may do almost anything with this work of mine, so long as you give me proper credit

A student builds the following voltage divider circuit so she can power a 6-volt lamp from a 15-volt power supply:

$$\epsfbox{03332x01.eps}$$

When built, the circuit works just as it should.  However, after operating successfully for hours, the lamp suddenly goes dark.  Identify all the possible faults you can think of in this circuit which could account for the lamp not glowing anymore.

\underbar{file 03332}
%(END_QUESTION)





%(BEGIN_ANSWER)

Here are a few possibilities (by no means exhaustive):

\medskip
\item{$\bullet$} Lamp burned out
\item{$\bullet$} 15-volt power supply failed
\item{$\bullet$} Resistors $R_1$ {\it and} $R_2$ simultaneously failed open
\medskip

\vskip 10pt

Follow-up question: although the third possibility mentioned here is certainly valid, it is less likely than any {\it single} failure.  Explain why, and how this general principle of considering single faults first is a good rule to follow when troubleshooting systems.

%(END_ANSWER)





%(BEGIN_NOTES)

Have fun with your students figuring out all the possible faults which could account for the lamp going dark!  Be sure to include wires and wire connections in your list.

The follow-up question is intended to get students to come up with their own version of Occam's Razor: the principle that the simplest explanation for an observed phenomenon is probably the correct one.

%INDEX% Troubleshooting, loaded voltage divider

%(END_NOTES)


