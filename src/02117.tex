
%(BEGIN_QUESTION)
% Copyright 2004, Tony R. Kuphaldt, released under the Creative Commons Attribution License (v 1.0)
% This means you may do almost anything with this work of mine, so long as you give me proper credit

Electrical engineers often represent impedances in rectangular form for the sake of algebraic manipulation: to be able to construct and manipulate equations involving impedance, in terms of the components' fundamental values (resistors in ohms, capacitors in farads, and inductors in henrys).

For example, the impedance of a series-connected resistor ($R$) and inductor ($L$) would be represented as follows, with angular velocity ($\omega$) being equal to $2 \pi f$:

$${\bf Z} = R + j \omega L$$

Using the same algebraic notation, represent each of the following complex quantities:

\medskip
\goodbreak
\item{$\bullet$} Impedance of a single capacitor ($C$) = 
\item{$\bullet$} Impedance of a series resistor-capacitor ($R$, $C$) network =
\item{$\bullet$} Admittance of a parallel inductor-resistor ($L$, $R$) network =
\item{$\bullet$} Admittance of a parallel resistor-capacitor ($R$, $C$) network =
\medskip

\underbar{file 02117}
%(END_QUESTION)





%(BEGIN_ANSWER)

\medskip
\goodbreak
\item{$\bullet$} Impedance of a single capacitor ($C$) = $-j {1 \over \omega C}$
\vskip 5pt
\item{$\bullet$} Impedance of a series resistor-capacitor ($R$, $C$) network = $R -j {1 \over \omega C}$
\vskip 5pt
\item{$\bullet$} Admittance of a parallel inductor-resistor ($L$, $R$) network = ${1 \over R} -j {1 \over \omega L}$
\vskip 5pt
\item{$\bullet$} Admittance of a parallel resistor-capacitor ($R$, $C$) network = ${1 \over R} +j \omega C$
\medskip

%(END_ANSWER)





%(BEGIN_NOTES)

One possible point of confusion here is the sign of $j$ after inversion.  If it is not evident from the answers, ${1 \over j}$ is equal to $-j$, so that the impedance of an inductor ($j \omega L$) becomes $-j {1 \over {\omega L}}$ when converted into an admittance.

%INDEX% Admittance, in rectangular algebraic form
%INDEX% Impedance, in rectangular algebraic form

%(END_NOTES)


