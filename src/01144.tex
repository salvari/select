
%(BEGIN_QUESTION)
% Copyright 2003, Tony R. Kuphaldt, released under the Creative Commons Attribution License (v 1.0)
% This means you may do almost anything with this work of mine, so long as you give me proper credit

Suppose a technician connected a potentiometer to the input of a VCO, to act as a manually-variable voltage source.  Then, the technician compares the output of the VCO with an external signal source that slowly changes frequency, adjusting the potentiometer to keep the two frequencies equal:

$$\epsfbox{01144x01.eps}$$

If a voltmeter were connected between the potentiometer wiper and ground, what would its indication represent (with regard to the external audio signal)?

\vskip 10pt

Explain where {\it feedback} occurs in this system, and how the system may be automated so as to not require the continual attention of a technician.

\underbar{file 01144}
%(END_QUESTION)





%(BEGIN_ANSWER)

The wiper voltage in this system is representative of the external signal's frequency.  Feedback occurs in the "loop" formed by the technician's hand, adjusting the potentiometer in response to changes in differential frequency.

%(END_ANSWER)





%(BEGIN_NOTES)

Tell your students that voltage-controlled oscillators are sometimes called {\it voltage-to-frequency converters}.  Then, ask them to explain how the presence of feedback in this system provides a {\it frequency-to-voltage conversion} function.

The question regarding automation of this system may be challenging for some of your students.  If they cannot give a technical answer for this question, ask them to explain "in layman's terms" what the additional components must do.  Accurately identifying the function of a human operator is the first step in designing an automation system!

%(END_NOTES)


