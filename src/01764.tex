
%(BEGIN_QUESTION)
% Copyright 2003, Tony R. Kuphaldt, released under the Creative Commons Attribution License (v 1.0)
% This means you may do almost anything with this work of mine, so long as you give me proper credit

Calculate the voltage drops $V_{AB}$, $V_{BC}$, and $V_{CD}$ in the following circuit:

$$\epsfbox{01764x01.eps}$$

\underbar{file 01764}
%(END_QUESTION)





%(BEGIN_ANSWER)

$V_{AB} = 461 \hbox{ mV}$

$V_{BC} = 0 \hbox{ V}$

$V_{CD} = 1.039 \hbox{ V}$

\vskip 10pt

Follow-up question: explain why the voltage between points A and B ($V_{AB}$) would {\it increase} if the 1200 $\Omega$ resistor were to fail shorted.  Hint: imagine a "jumper" wire connected across that resistor to simulate a shorted failure.

\vskip 10pt

Challenge question: explain how you can calculate these same answers without ever having to calculate total circuit current.

%(END_ANSWER)





%(BEGIN_NOTES)

Ask your students how they could tell $V_{BC}$ must be zero, just by examining the circuit (without doing any math).  If some students experience difficulty answering this question on their own, have them translate the drawing into a proper schematic diagram.

\vskip 10pt

Students often have difficulty formulating a method of solution: determining what steps to take to get from the given conditions to a final answer.  While it is helpful at first for you (the instructor) to show them, it is bad for you to show them too often, lest they stop thinking for themselves and merely follow your lead.  A teaching technique I have found very helpful is to have students come up to the board (alone or in teams) in front of class to write their problem-solving strategies for all the others to see.  They don't have to actually do the math, but rather outline the steps they would take, in the order they would take them.

By having students \underbar{outline their problem-solving strategies}, everyone gets an opportunity to see multiple methods of solution, and you (the instructor) get to see how (and if!) your students are thinking.  An especially good point to emphasize in these "open thinking" activities is how to check your work to see if any mistakes were made.

%INDEX% Series-parallel circuit, voltage and current calculations in

%(END_NOTES)


