
%(BEGIN_QUESTION)
% Copyright 2005, Tony R. Kuphaldt, released under the Creative Commons Attribution License (v 1.0)
% This means you may do almost anything with this work of mine, so long as you give me proper credit

One of the fundamental equations used in electricity and electronics is {\it Ohm's Law}: the relationship between voltage ($E$ or $V$, measured in units of {\it volts}), current ($I$, measured in units of {\it amperes}), and resistance ($R$, measured in units of {\it ohms}): 

$$E = IR \hskip 20pt I = {E \over R} \hskip 20pt R = {E \over I}$$

\noindent
Where,

$E = $ Voltage in units of volts (V)

$I = $ Current in units of amps (A)

$R = $ Resistance in units of ohms ($\Omega$)

\vskip 10pt

Solve for the unknown quantity ($E$, $I$, or $R$) given the other two, and express your answer in both scientific and metric notations:

\vskip 10pt

$I$ = 45 mA, $R$ = 3.0 k$\Omega$; \hskip 15pt $E$ = 

\vskip 10pt

$I$ = 10 kA, $R$ = 0.5 m$\Omega$; \hskip 15pt $E$ = 

\vskip 10pt

$E$ = 45 V, $R$ = 4.7 k$\Omega$; \hskip 15pt $I$ = 

\vskip 10pt

$E$ = 13.8 kV, $R$ = 8.1 k$\Omega$; \hskip 15pt $I$ = 

\vskip 10pt

$E$ = 500 $\mu$V, $I$ = 36 nA; \hskip 15pt $R$ = 

\vskip 10pt

$E$ = 14 V, $I$ = 110 A; \hskip 15pt $R$ = 

\vskip 10pt

$I$ = 0.001 A, $R$ = 922 $\Omega$; \hskip 15pt $E$ = 

\vskip 10pt

$I$ = 825 A, $R$ = 15.0 m$\Omega$; \hskip 15pt $E$ = 

\vskip 10pt

$E$ = 1.2 kV, $R$ = 30 M$\Omega$; \hskip 15pt $I$ = 

\vskip 10pt

$E$ = 750 mV, $R$ = 86 $\Omega$; \hskip 15pt $I$ = 

\vskip 10pt

$E$ = 30.0 V, $I$ = 0.0025 A; \hskip 15pt $R$ = 

\vskip 10pt

$E$ = 0.00071 V, $I$ = 3389 A; \hskip 15pt $R$ =

\vskip 10pt

\underbar{file 03247}
%(END_QUESTION)





%(BEGIN_ANSWER)

$I$ = 45 mA, $R$ = 3.0 k$\Omega$; \hskip 15pt $E$ = 140 V = $1.4 \times 10^2$ V

\vskip 10pt

$I$ = 10 kA, $R$ = 0.5 m$\Omega$; \hskip 15pt $E$ = 5 V = $5 \times 10^0$ V

\vskip 10pt

$E$ = 45 V, $R$ = 4.7 k$\Omega$; \hskip 15pt $I$ = 9.6 mA = $9.6 \times 10^{-3}$ A 

\vskip 10pt

$E$ = 13.8 kV, $R$ = 8.1 k$\Omega$; \hskip 15pt $I$ = 1.7 A = $1.7 \times 10^0$ A 

\vskip 10pt

$E$ = 500.0 $\mu$V, $I$ = 36 nA; \hskip 15pt $R$ = 14 k$\Omega$ = $1.4 \times 10^{4}$ $\Omega$

\vskip 10pt

$E$ = 14 V, $I$ = 110 A; \hskip 15pt $R$ = 130 m$\Omega$ = $1.3 \times 10^{-1}$ $\Omega$

\vskip 10pt

$I$ = 0.001 A, $R$ = 922 $\Omega$; \hskip 15pt $E$ = 900 mV = $9 \times 10^{-1}$ V

\vskip 10pt

$I$ = 825 A, $R$ = 15.0 m$\Omega$; \hskip 15pt $E$ = 12.4 V = $1.24 \times 10^{1}$ V

\vskip 10pt

$E$ = 1.2 kV, $R$ = 30 M$\Omega$; \hskip 15pt $I$ = 40 $\mu$A = $4 \times 10^{-5}$ A

\vskip 10pt

$E$ = 750 mV, $R$ = 86 $\Omega$; \hskip 15pt $I$ = 8.7 mA = $8.7 \times 10^{-3}$ A

\vskip 10pt

$E$ = 30.0 V, $I$ = 0.0025 A; \hskip 15pt $R$ = 12 k$\Omega$ = $1.2 \times 10^{4}$ $\Omega$

\vskip 10pt

$E$ = 0.00071 V, $I$ = 3389 A; \hskip 15pt $R$ = 210 n$\Omega$ = $2.1 \times 10^{-7}$ $\Omega$

%(END_ANSWER)





%(BEGIN_NOTES)

In calculating the answers, I held to proper numbers of significant digits.  This question is little more than drill for students learning how to express quantities in scientific and metric notations.

%INDEX% Ohm's Law
%INDEX% Notation, scientific
%INDEX% Notation, metric

%(END_NOTES)


