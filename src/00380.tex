
%(BEGIN_QUESTION)
% Copyright 2003, Tony R. Kuphaldt, released under the Creative Commons Attribution License (v 1.0)
% This means you may do almost anything with this work of mine, so long as you give me proper credit

Suppose a short-circuit were to develop in this electric power system:

$$\epsfbox{00380x01.eps}$$

The purpose of the circuit breaker, of course, is to open the circuit automatically, to prevent damage to the power conductors.  In a large electric power system, the magnitudes of such short-circuit currents can be enormous.

Large inductors, commonly called {\it reactors}, are often installed in series with power conductors in high-voltage power systems in order to "soften" the onset of short-circuit currents:

$$\epsfbox{00380x02.eps}$$

Explain how the addition of a "reactor" helps to minimize the magnitude of the short-circuit current the breaker has to interrupt.

\underbar{file 00380}
%(END_QUESTION)





%(BEGIN_ANSWER)

At the moment that a short-circuit fault occurs, the sudden increase in current constitutes a very large ${di \over dt}$ value, which the inductor momentarily "opposes" by dropping voltage.

\vskip 10pt

Follow-up question: why use an inductor to limit the short-circuit fault current?  Why not use a resistor instead?

%(END_ANSWER)





%(BEGIN_NOTES)

Power system reactors are usually installed in substations, where they appear as coils of wire (no iron core) a few feet in diameter, usually located near circuit breakers.

%INDEX% Reactor, fault limiting

%(END_NOTES)


