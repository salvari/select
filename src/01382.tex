
%(BEGIN_QUESTION)
% Copyright 2003, Tony R. Kuphaldt, released under the Creative Commons Attribution License (v 1.0)
% This means you may do almost anything with this work of mine, so long as you give me proper credit

% Uncomment the following line if the question involves calculus at all:
\vbox{\hrule \hbox{\strut \vrule{} $\int f(x) \> dx$ \hskip 5pt {\sl Calculus alert!} \vrule} \hrule}

One of the fundamental principles of calculus is a process called {\it integration}.  This principle is important to understand because it is manifested in the behavior of inductance.  Thankfully, there are more familiar physical systems which also manifest the process of integration, making it easier to comprehend.

If we introduce a constant flow of water into a cylindrical tank with water, the water level inside that tank will rise at a constant rate over time:

$$\epsfbox{01382x01.eps}$$

In calculus terms, we would say that the tank {\it integrates} water flow into water height.  That is, one quantity (flow) dictates the rate-of-change over time of another quantity (height).

\vskip 10pt

Like the water tank, electrical {\bf inductance} also exhibits the phenomenon of integration with respect to time.  Which electrical quantity (voltage or current) dictates the rate-of-change over time of which other quantity (voltage or current) in an inductance?  Or, to re-phrase the question, which quantity (voltage or current), when maintained at a constant value, results in which other quantity (current or voltage) steadily ramping either up or down over time?

\underbar{file 01382}
%(END_QUESTION)





%(BEGIN_ANSWER)

In an inductance, current is the time-integral of voltage.  That is, the applied voltage across the inductor dictates the rate-of-change of current through the inductor over time.

\vskip 10pt

Challenge question: can you think of a way we could exploit the similarity of inductive voltage/current integration to {\it simulate} the behavior of a water tank's filling, or any other physical process described by the same mathematical relationship?

%(END_ANSWER)





%(BEGIN_NOTES)

The concept of integration doesn't have to be overwhelmingly complex.  Electrical phenomena such as capacitance and inductance may serve as excellent contexts in which students may explore and comprehend the abstract principles of calculus.  The amount of time you choose to devote to a discussion of this question will depend on how mathematically adept your students are.

%INDEX% Calculus, integral
%INDEX% Integration, with inductor

%(END_NOTES)


