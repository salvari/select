
%(BEGIN_QUESTION)
% Copyright 2003, Tony R. Kuphaldt, released under the Creative Commons Attribution License (v 1.0)
% This means you may do almost anything with this work of mine, so long as you give me proper credit

An important parameter of logic gate circuitry is {\it noise margin}.  What exactly is "noise margin," and how is it defined for logic gates?

Specifically, how much noise margin do digital circuits exclusively composed of TTL gates have?

\vskip 10pt

Note: you will need to consult TTL gate datasheets to answer this question properly.

\underbar{file 01269}
%(END_QUESTION)





%(BEGIN_ANSWER)

Noise margin is the difference between the acceptable voltage limits for corresponding input and output logic states.

%(END_ANSWER)





%(BEGIN_NOTES)

This question, to be answered properly, involves more than just a definition of "noise margin."  Students must first discover that there is a difference between voltage compliance levels for gate inputs versus outputs, then recognize that the difference constitutes a "margin" that imposed AC voltage ("noise") must not exceed.  They must then present their answer in terms of manufacturer specifications, obtained in datasheets.  In summary, there is a lot of research that must occur to answer this question, but the results will be worth it!

%INDEX% Noise margin, defined for TTL digital gates

%(END_NOTES)


