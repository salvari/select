
%(BEGIN_QUESTION)
% Copyright 2003, Tony R. Kuphaldt, released under the Creative Commons Attribution License (v 1.0)
% This means you may do almost anything with this work of mine, so long as you give me proper credit

This circuit shown here is for a {\it timing light}: a device that uses a pulsed strobe lamp to "freeze" the motion of a rotating object.

$$\epsfbox{01078x01.eps}$$

Which component(s) in this circuit form the oscillator section?  What type of oscillator is used in this circuit?  Which component values have a direct influence on the frequency of the flash tube's output?

\underbar{file 01078}
%(END_QUESTION)





%(BEGIN_ANSWER)

The heart of the oscillator circuit is unijunction transistor $Q_1$.  Together with some other components (I'll let you figure out which!), this transistor forms a {\it relaxation oscillator} circuit.  $R_1$, $R_2$, and $C_1$ have direct influence over the oscillation frequency.

\vskip 10pt

Challenge question: what purpose does resistor $R_2$ serve?  It would seem at first glance that it serves no useful purpose, as potentiometer $R_1$ is capable of providing any desired amount of resistance for the RC time constant circuit on its own -- $R_2$'s resistance is simply added to it.  However, there is an important, practical reason for including $R_2$ in the circuit.  Explain what that reason is.

%(END_ANSWER)





%(BEGIN_NOTES)

Ask your students to explain what the other transistors do in this circuit.  If time permits, explore the operation of the entire circuit with your students, asking them to explain the purpose and function of all components in it.

After they identify which components control the frequency of oscillation, ask them to specifically identify which direction each of those component values would need to be changed in order to increase (or decrease) the flash rate.

%INDEX% Relaxation oscillator, with strobe light
%INDEX% Timing light circuit

%(END_NOTES)


