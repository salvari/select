
%(BEGIN_QUESTION)
% Copyright 2003, Tony R. Kuphaldt, released under the Creative Commons Attribution License (v 1.0)
% This means you may do almost anything with this work of mine, so long as you give me proper credit

Determine whether this amplifier circuit is inverting or noninverting (i.e. the phase shift between input and output waveforms):

$$\epsfbox{01183x01.eps}$$

Be sure to explain, step by step, {\it how} you were able to determine the phase relationship between input and output in this circuit.  Also identify the type of amplifier each transistor represents (common-???).

\underbar{file 01183}
%(END_QUESTION)





%(BEGIN_ANSWER)

Noninverting.  The JFET is connected as a common-source, while the BJT is connected as a common-emitter.

%(END_ANSWER)





%(BEGIN_NOTES)

There are several other questions you could ask about this amplifier circuit.  For example:

\medskip
\item{$\bullet$} How is the Q-point bias established for the JFET?
\item{$\bullet$} How is the Q-point bias established for the BJT?
\item{$\bullet$} What purpose does the potentiometer serve?
\item{$\bullet$} Is there another possible location for the potentiometer that would perform the same function?
\medskip

Note: the schematic diagram for this circuit was derived from one found on page 36 of John Markus' \underbar{Guidebook of Electronic Circuits}, first edition.  Apparently, the design originated from a Motorola publication on using field effect transistors ("Tips on using FET's," HMA-33, 1971).

%INDEX% Phase relationship, discrete JFET/BJT amplifier circuit

%(END_NOTES)


