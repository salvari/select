
%(BEGIN_QUESTION)
% Copyright 2003, Tony R. Kuphaldt, released under the Creative Commons Attribution License (v 1.0)
% This means you may do almost anything with this work of mine, so long as you give me proper credit

If an oscilloscope is connected to a series combination of AC and DC voltage sources, what is displayed on the oscilloscope screen depends on where the "coupling" control is set.

With the coupling control set to "DC", the waveform displayed will be elevated above (or depressed below) the "zero" line:

$$\epsfbox{00538x01.eps}$$

Setting the coupling control to "AC", however, results in the waveform automatically centering itself on the screen, about the zero line.

$$\epsfbox{00538x02.eps}$$

Based on these observations, explain what the "DC" and "AC" settings on the coupling control actually mean.

\underbar{file 00538}
%(END_QUESTION)





%(BEGIN_ANSWER)

The "DC" setting allows the oscilloscope to display {\it all} components of the signal voltage, both AC and DC, while the "AC" setting blocks all DC within the signal, to only display the varying (AC) portion of the signal on the screen.

%(END_ANSWER)





%(BEGIN_NOTES)

A common misconception among students is that the "DC" setting is used for measuring DC signals only, and that the "AC" setting is used for measuring AC signals only.  I often refer to the "DC" setting as {\it direct coupling} in order to avoid the connotation of "direct current," in an attempt to reinforce the idea that with "DC" coupling, what you see is all that's really there.  With "AC" coupling, only part of the signal is being coupled to the input amplifier circuitry.

%INDEX% Oscilloscope, AC/DC coupling (conceptual)

%(END_NOTES)


