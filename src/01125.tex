
%(BEGIN_QUESTION)
% Copyright 2003, Tony R. Kuphaldt, released under the Creative Commons Attribution License (v 1.0)
% This means you may do almost anything with this work of mine, so long as you give me proper credit

A common wideband transistor amplifier circuit is the {\it cascode} design, using common-emitter and common-base transistor stages:

$$\epsfbox{01125x01.eps}$$

What advantage(s) does the cascode amplifier have over "normal" single- or multi-stage amplifier designs?  What, specifically, makes it well suited for high-frequency applications, such as RF (Radio Frequency) signal amplifiers?

\underbar{file 01125}
%(END_QUESTION)





%(BEGIN_ANSWER)

The combination of a common-collector first stage and a common-base second stage significantly reduces the debilitating effects of interjunction capacitance within the two transistors.  Most cascode amplifiers require no {\it neutralization}, either: a testament to the effectiveness of the design.

%(END_ANSWER)





%(BEGIN_NOTES)

This is one of the few popular applications for the common-base transistor amplifier configuration, and it is a solution that has been implemented with field-effect transistors as well as bipolar transistors (and even electron tubes, before that!).  Ask your students to explain how the circuit works, especially noting the voltage gain of each stage, and the locations of interjunction (Miller-effect) capacitances in the circuit.

%INDEX% Cascode amplifier
%INDEX% Amplifier, cascode

%(END_NOTES)


