
%(BEGIN_QUESTION)
% Copyright 2003, Tony R. Kuphaldt, released under the Creative Commons Attribution License (v 1.0)
% This means you may do almost anything with this work of mine, so long as you give me proper credit

A very simple circuit that may be used as a {\it current regulator} in a DC circuit is this:

\vskip 10pt

$$\epsfbox{00997x01.eps}$$

\vskip 10pt

Draw a battery symbol and any necessary connecting wires to form a complete {\it DC current source}, that will attempt to supply a regulated amount of DC current through any given load.

\underbar{file 00997}
%(END_QUESTION)





%(BEGIN_ANSWER)

$$\epsfbox{00997x02.eps}$$

%(END_ANSWER)





%(BEGIN_NOTES)

The direction of current through the JFET, when used as a current regulator in this fashion, is very important.  Not so much for the JFET's sake, as for the sake of establishing the correct type of feedback (negative) to make the circuit self-regulating.

Discuss this circuit's operation with your students, asking them to determine the polarity of the resistor's voltage drop, and how that relates to the "pinching off" of the JFET.

%INDEX% Current regulator, JFET
%INDEX% Current source, JFET

%(END_NOTES)


