
%(BEGIN_QUESTION)
% Copyright 2005, Tony R. Kuphaldt, released under the Creative Commons Attribution License (v 1.0)
% This means you may do almost anything with this work of mine, so long as you give me proper credit

Determine the inductor voltage and inductor current at the specified times (time $t$ = 0 milliseconds being the exact moment the switch contacts close):

$$\epsfbox{03590x01.eps}$$

% No blank lines allowed between lines of an \halign structure!
% I use comments (%) instead, so that TeX doesn't choke.

$$\vbox{\offinterlineskip
\halign{\strut
\vrule \quad\hfil # \ \hfil & 
\vrule \quad\hfil # \ \hfil & 
\vrule \quad\hfil # \ \hfil \vrule \cr
\noalign{\hrule}
%
% First row
Time & $V_L$ (volts) & $I_L$ (mA) \cr
%
\noalign{\hrule}
%
0 ns &  &  \cr
%
\noalign{\hrule}
%
250 ns &  &  \cr
%
\noalign{\hrule}
%
500 ns &  &  \cr
%
\noalign{\hrule}
%
750 ns &  &  \cr
%
\noalign{\hrule}
%
1.00 $\mu$s &  &  \cr
%
\noalign{\hrule}
%
1.25 $\mu$s &  &  \cr
%
\noalign{\hrule}
} % End of \halign 
}$$ % End of \vbox

\underbar{file 03590}
%(END_QUESTION)





%(BEGIN_ANSWER)

% No blank lines allowed between lines of an \halign structure!
% I use comments (%) instead, so that TeX doesn't choke.

$$\vbox{\offinterlineskip
\halign{\strut
\vrule \quad\hfil # \ \hfil & 
\vrule \quad\hfil # \ \hfil & 
\vrule \quad\hfil # \ \hfil \vrule \cr
\noalign{\hrule}
%
% First row
Time & $V_L$ (volts) & $I_L$ (mA) \cr
%
\noalign{\hrule}
%
0 ns & 12 & 0 \cr
%
\noalign{\hrule}
%
250 ns & 6.110 & 21.82 \cr
%
\noalign{\hrule}
%
500 ns & 3.111 & 32.92 \cr
%
\noalign{\hrule}
%
750 ns & 1.584 & 38.58 \cr
%
\noalign{\hrule}
%
1.00 $\mu$s & 0.8065 & 41.46 \cr
%
\noalign{\hrule}
%
1.25 $\mu$s & 0.4106 & 42.93 \cr
%
\noalign{\hrule}
} % End of \halign 
}$$ % End of \vbox

%(END_ANSWER)





%(BEGIN_NOTES)

Be sure to have your students share their problem-solving techniques (how they determined which equation to use, etc.) in class.

%INDEX% Time constant calculation, LR circuit

%(END_NOTES)


