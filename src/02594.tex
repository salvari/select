
%(BEGIN_QUESTION)
% Copyright 2005, Tony R. Kuphaldt, released under the Creative Commons Attribution License (v 1.0)
% This means you may do almost anything with this work of mine, so long as you give me proper credit

In a common-emitter transistor amplifier circuit, the presence of capacitance between the collector and base terminals -- whether intrinsic to the transistor or externally connected -- has the effect of turning the amplifier circuit into a low-pass filter, with voltage gain being inversely proportional to frequency:

$$\epsfbox{02594x01.eps}$$

Explain why this is.  Why, exactly, does a capacitance placed in this location affect voltage gain?  Hint: it has something to do with {\it negative feedback}!

\underbar{file 02594}
%(END_QUESTION)





%(BEGIN_ANSWER)

The capacitance provides a path for an AC feedback signal to go from the collector to the base.  Given the inverting phase relationship between collector voltage and base voltage, this feedback is degenerative.

%(END_ANSWER)





%(BEGIN_NOTES)

Students should realize that this is no hypothetical question.  Intrinsic capacitance does indeed exist between the collector and base of a bipolar junction transistor (called the {\it Miller capacitance}), and this has a degenerating effect on voltage gain with increasing frequency.  If time permits, you may wish to discuss how the common-collector and common-base amplifier configurations naturally avoid this problem.

%INDEX% Voltage gain, reduction due to collector-base capacitance

%(END_NOTES)


