
%(BEGIN_QUESTION)
% Copyright 2005, Tony R. Kuphaldt, released under the Creative Commons Attribution License (v 1.0)
% This means you may do almost anything with this work of mine, so long as you give me proper credit

A common mistake made by students new to digital circuits is to misplace the pullup or pulldown resistors in schematic diagrams, and also in the circuits they build.  Study the following schematics and determine whether the resistor in each one is a properly-placed {\it pullup} or {\it pulldown} resistor, or if it is improperly placed:

$$\epsfbox{03181x01.eps}$$

\underbar{file 03181}
%(END_QUESTION)





%(BEGIN_ANSWER)

$$\epsfbox{03181x02.eps}$$

\vskip 10pt

Follow-up question: specifically identify what would be wrong with each of the "improper" circuits.

%(END_ANSWER)





%(BEGIN_NOTES)

This is one concept I have found many students have difficulty grasping, essentially because it involves the determination of a voltage drop between two points (the "arrow" wire and ground).  It is a spatial-relations problem, similar to Kirchhoff's Voltage Law problems where students need to figure out how much voltage is between two specified points given voltage drops across several other pairs of points.  Spend time with your students discussing these circuits, because several of your students will probably not understand this concept the first, second, or even third time through.

I strongly recommend students take the approach of a "thought experiment" in determining the efficacy of each circuit shown here: analyze the output voltage (logic state) for each of the switch's two positions.  This simple approach usually helps clarify what each circuit does and why the "improper" circuits do not work.

%INDEX% Pulldown resistor
%INDEX% Pullup resistor
%INDEX% Resistor, pulldown (digital)
%INDEX% Resistor, pullup (digital)

%(END_NOTES)


