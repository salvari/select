
%(BEGIN_QUESTION)
% Copyright 2003, Tony R. Kuphaldt, released under the Creative Commons Attribution License (v 1.0)
% This means you may do almost anything with this work of mine, so long as you give me proper credit

An important equation in predicting the capacity of a lead-acid cell is called {\it Peukert's formula}:

$$I^n t = C$$

\noindent
Where,

$I =$ Discharge current, in amps

$n =$ Constant, particular to the cell (typically between 1.1 and 1.8)

$t =$ Time of discharge

$C =$ Constant, particular to the battery (varies with the amp-hour capacity)

\vskip 5pt

Explain how Peukert's formula relates to the simpler "amp-hour" formula.

\vskip 10pt

Also, calculate the discharge time for a battery with an $n$ value of 1.2 and a $C$ value of 200, given a discharge current of 25 amps.

\underbar{file 00516}
%(END_QUESTION)





%(BEGIN_ANSWER)

The "amp-hour" formula is linear, while Peukert's formula is not (at least for values of $n$ not equal to unity).

\vskip 10pt

The discharge time for the example problem is 4.2 hours.

%(END_ANSWER)





%(BEGIN_NOTES)

Ask one of your students to write Peukert's equation on the whiteboard in front of class, along with the standard amp-hour formula.  Viewed side-by-side, there is but one difference between the two equations, and it is the exponent $n$.

%INDEX% Peukert's formula

%(END_NOTES)


