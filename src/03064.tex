
%(BEGIN_QUESTION)
% Copyright 2005, Tony R. Kuphaldt, released under the Creative Commons Attribution License (v 1.0)
% This means you may do almost anything with this work of mine, so long as you give me proper credit

{\it Substitution} is the term we give to the mathematical equivalence of one variable to one or more other variables in an expression.  It is a fundamental principle used to combine two or more equations into a single equation (among other things).

For example, we know that the formula for calculating current in a simple one-resistor circuit is as follows:

$$\epsfbox{03064x01.eps}$$

We also know that the total resistance ($R$) of a three-resistor series circuit is as follows:

$$\epsfbox{03064x02.eps}$$

Combine these two equations together using substitution so that we have a single equation for calculating current $I$ in a three-resistor series circuit given the source voltage $V$ and each resistance value $R_1$, $R_2$, and $R_3$:

$$\epsfbox{03064x03.eps}$$

In other words, you need to have as your answer a single equation that begins with "$I =$" and has all the variables $V$, $R_1$, $R_2$, and $R_3$ on the other side of the "equal" sign.

\underbar{file 03064}
%(END_QUESTION)





%(BEGIN_ANSWER)

$$I = {V \over {R_1 + R_2 + R_3}}$$

%(END_ANSWER)





%(BEGIN_NOTES)

I like to speak of the process of substitution in terms of {\it definitions} for variables.  In this particular case, $R_1 + R_2 + R_3$ is a {\it definition} for $R$ that we put in $R$'s place in the first equation ($I = {V \over R}$).

The notation shown in the third schematic, $I = f(V, R_1, R_2, R_3)$, is known as {\it function notation}.  It merely means that the value of $I$ is determined by the values of all those variables within the parentheses, rather than just one.

%INDEX% Algebra, substitution

%(END_NOTES)


