
%(BEGIN_QUESTION)
% Copyright 2005, Tony R. Kuphaldt, released under the Creative Commons Attribution License (v 1.0)
% This means you may do almost anything with this work of mine, so long as you give me proper credit

When the pushbutton is pressed, the relay immediately energizes and sends power to the electric horn.  When the pushbutton is released, the horn remains on for a few moments before turning off, due to the capacitor's stored charge continuing to power the relay coil.  So, the capacitor and relay form a {\it time-delay} control circuit for the horn:

$$\epsfbox{03441x01.eps}$$

Suppose this circuit has functioned as designed for quite some time, then one day develops a problem.  The horn sounds immediately when the pushbutton is pressed (as it should), bit it immediately silences instead of continuing to sound for a few more moments when the pushbutton is released.  Based on this information, identify these things:

\vskip 10pt

\medskip
\item{$\bullet$} \underbar{Two} components or wires in the circuit that you know must be in good working condition.
\vskip 40pt
\item{$\bullet$} \underbar{Two} components or wires in the circuit that could possibly be bad (and thus cause the off-delay action to fail).
\medskip

\underbar{file 03441}
%(END_QUESTION)





%(BEGIN_ANSWER)

Obviously the horn, relay, switch, and battery are all working normally.  The capacitor could be bad (failed open), as well as the wiring/connections between the capacitor and the coil.

%(END_ANSWER)





%(BEGIN_NOTES)

The purpose of this troubleshooting question is to get students to think in terms of fault elimination: deciding what things {\it cannot} be bad in order to better isolate what might be bad.

%INDEX% Time delay relay circuit, using capacitor

%(END_NOTES)


