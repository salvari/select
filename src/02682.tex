
%(BEGIN_QUESTION)
% Copyright 2005, Tony R. Kuphaldt, released under the Creative Commons Attribution License (v 1.0)
% This means you may do almost anything with this work of mine, so long as you give me proper credit

Examine the following progression of mathematical statements:

$$(10^2)(10^3) = 100000$$

$$10^{2+3} = 100000$$

$$10^5 = 100000$$

What does this pattern indicate?  What principle of algebra is illustrated by these three equations?

Next, examine this progression of mathematical statements:

$$\log 10^5 = \log 100000 = 5$$

$$\log 10^{2+3} = \log 100000 = 5$$

$$\log 10^2 + \log 10^3 = \log 100000 = 5$$

What does this pattern indicate?  What principle of algebra is illustrated by these three equations?

\underbar{file 02682}
%(END_QUESTION)





%(BEGIN_ANSWER)

\noindent
First pattern:

The product of two base numbers with different exponents is equal to that base number raised to the power of the exponents' sum.

\vskip 10pt

\noindent
Second pattern:

The sum of two logarithms is equal to the logarithm of those two numbers' product. 

%(END_ANSWER)





%(BEGIN_NOTES)

In this question, I want students to begin to see how logarithms relate multiplication to addition, and how powers relate addition to multiplication.  This is an initial step to students recognizing logarithms as {\it transform functions}: a means to transform one type of mathematical problem into a simpler type of mathematical problem.

%INDEX% Logarithms, used to transform a multiplication problem into an addition problem

%(END_NOTES)


