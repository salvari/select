
%(BEGIN_QUESTION)
% Copyright 2003, Tony R. Kuphaldt, released under the Creative Commons Attribution License (v 1.0)
% This means you may do almost anything with this work of mine, so long as you give me proper credit

Suppose we needed to shield a sensitive electronic instrument from external magnetic fields.  How would you suggest we do such a thing?  How can we keep stray magnetic fields away from this instrument?

\underbar{file 00656}
%(END_QUESTION)





%(BEGIN_ANSWER)

Magnetic shielding requires that the instrument be completely surrounded by a high-permeability enclosure, such that the enclosure will "conduct" any and all magnetic lines of flux away from the instrument.

\vskip 10pt

Challenge question: should the enclosure material have a high {\it retentivity}, or a low retentivity?  Explain your answer.

%(END_ANSWER)





%(BEGIN_NOTES)

This question provides a good review of the terms "permeability" and "retentivity".  Discuss these terms as they apply to the subject of magnetic shielding.  After discussing these concepts, ask your students to give examples of suitable enclosure materials.  What metals would be especially good for shielding purposes?  Where did they obtain their information about magnetic shielding?

%INDEX% Magnetic shielding

%(END_NOTES)


