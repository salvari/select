
%(BEGIN_QUESTION)
% Copyright 2004, Tony R. Kuphaldt, released under the Creative Commons Attribution License (v 1.0)
% This means you may do almost anything with this work of mine, so long as you give me proper credit

A very useful MOSFET circuit is the {\it bilateral switch}, an example shown here for you to analyze:

$$\epsfbox{01132x01.eps}$$

The "dual inverter" circuit simply ensures the two control lines A and B will always be opposite polarities (one at $V_{dd}$ potential, the other at ground potential).

What is the purpose of a "bilateral switch" circuit?  Hint: there are two integrated circuit implementations of the bilateral switch -- the 4016 and the 4066.  Investigate the datasheets for these integrated circuits to learn more!

\underbar{file 01132}
%(END_QUESTION)





%(BEGIN_ANSWER)

I'll let you research this one yourself!

%(END_ANSWER)





%(BEGIN_NOTES)

If your students have not yet learned about digital transistor circuits, this would be a good time to introduce the concept of "high" and "low" logic states, in this case as control signals to the bilateral switch cell.

Ask your students what the purpose of a bilateral switch might be, since we already have mechanical switches capable of switching almost any type of electrical signal known.

%INDEX% Bilateral switch circuit, CMOS

%(END_NOTES)


