
%(BEGIN_QUESTION)
% Copyright 2003, Tony R. Kuphaldt, released under the Creative Commons Attribution License (v 1.0)
% This means you may do almost anything with this work of mine, so long as you give me proper credit

A common sub-circuit in power supplies of all kinds is an {\it EMI/RFI} filter.  This LC network is all but "transparent" to the 50 or 60 Hz power line frequency, so that the transformer sees full line voltage at all times:

$$\epsfbox{02009x01.eps}$$

If this EMI/RFI filter does nothing to or with the line power, what purpose does it serve?

\underbar{file 02009}
%(END_QUESTION)





%(BEGIN_ANSWER)

An EMI/RFI filter has no purpose in the process of converting AC power into DC power.  It does, however, help prevent the power supply circuitry from interfering with {\it other} equipment energized by the same AC line power, by filtering out unwanted high-frequency noise generated within the power supply by diode switching.

\vskip 10pt

Follow-up question: calculate the total inductive reactance posed by the two inductors to 60 Hz power if the inductance of each is 100 $\mu$H.

%(END_ANSWER)





%(BEGIN_NOTES)

These filters are very common in switch-mode power supplies, but they are not out of place in linear ("brute-force") power supply circuits such as this one.  Mention to your students how the abatement of electromagnetic and radio frequency interference is a high priority in all kinds of electronic device design.

%INDEX% Filter, RFI
%INDEX% Harmonics, generated by AC-DC power supply circuit
%INDEX% RFI filter

%(END_NOTES)


