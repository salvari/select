
%(BEGIN_QUESTION)
% Copyright 2003, Tony R. Kuphaldt, released under the Creative Commons Attribution License (v 1.0)
% This means you may do almost anything with this work of mine, so long as you give me proper credit

An alternative to {\it soldering} components into printed circuit boards is to use a technique called {\it wire wrap}.  Describe what "wire wrap" is and what applications it might be suitable for.

\underbar{file 00236}
%(END_QUESTION)





%(BEGIN_ANSWER)

"Wire wrap" is a connection technique for joining components together on a circuit board without need for pre-printed copper "traces" on the board.  Instead, small-gauge wire is tightly wrapped around the pins of each component, with the wire making point-to-point connections between components.

%(END_ANSWER)





%(BEGIN_NOTES)

Though wire-wrapping is not as prevalent now as it was a few decades ago, it still has merit for certain applications, notably prototyping.  If possible, have some wire-wrapped circuit boards available to show your students during discussion time.

%INDEX% Wire wrap, defined

%(END_NOTES)


