
%(BEGIN_QUESTION)
% Copyright 2005, Tony R. Kuphaldt, released under the Creative Commons Attribution License (v 1.0)
% This means you may do almost anything with this work of mine, so long as you give me proper credit

Suppose a capacitor is charged to a voltage of exactly 100 volts, then connected to a resistor so it discharges slowly.  Calculate the amount of voltage remaining across the capacitor terminals at the following points in time:

\medskip
\item{$\bullet$} 1 time constant ($\tau$) after connecting the resistor:
\item{$\bullet$} 2 time constants (2$\tau$) after connecting the resistor:
\item{$\bullet$} 3 time constants (3$\tau$) after connecting the resistor:
\item{$\bullet$} 4 time constants (4$\tau$) after connecting the resistor:
\item{$\bullet$} 5 time constants (5$\tau$) after connecting the resistor:
\medskip

\underbar{file 03551}
%(END_QUESTION)





%(BEGIN_ANSWER)

\medskip
\item{$\bullet$} 1 time constant ($\tau$) after connecting the resistor: $V_C$ = 36.79 volts
\item{$\bullet$} 2 time constants (2$\tau$) after connecting the resistor: $V_C$ = 13.53 volts
\item{$\bullet$} 3 time constants (3$\tau$) after connecting the resistor: $V_C$ = 4.979 volts
\item{$\bullet$} 4 time constants (4$\tau$) after connecting the resistor: $V_C$ = 1.832 volts
\item{$\bullet$} 5 time constants (5$\tau$) after connecting the resistor: $V_C$ = 0.6738 volts
\medskip

\vskip 10pt

Follow-up question: write an equation solving for these voltages at the specified times.

%(END_ANSWER)





%(BEGIN_NOTES)

Although students should be able to look up approximate answers to this question from almost any beginning electronics textbook, the point here is to get them to relate the question to an actual formula so they may calculate this on their own.

%INDEX% RC time constant circuit

%(END_NOTES)


