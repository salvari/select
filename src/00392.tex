
%(BEGIN_QUESTION)
% Copyright 2003, Tony R. Kuphaldt, released under the Creative Commons Attribution License (v 1.0)
% This means you may do almost anything with this work of mine, so long as you give me proper credit

Suppose we were planning to use a photovoltaic panel to generate electricity and electrolyze water into hydrogen and oxygen gas:

$$\epsfbox{00392x01.eps}$$

Our goal is to electrolyze as much water as possible, and this means we must maximize the electrolysis cell's power dissipation.  Explain how we could experimentally determine the optimum internal resistance of the electrolysis cell, prior to actually building it, using nothing but the solar panel, a rheostat, and a DMM (digital multimeter).

\underbar{file 00392}
%(END_QUESTION)





%(BEGIN_ANSWER)

Experimentally determine what amount load resistance drops exactly one-half of the panel's open-circuit voltage.

\vskip 10pt

Follow-up question: assuming that the open-circuit voltage of this solar panel were high enough to pose a shock hazard, describe a procedure you might use to safely connect a "test load" to the panel.

%(END_ANSWER)





%(BEGIN_NOTES)

Students should at this point understand the maximum power transfer theorem, and also the concept of a voltage source having a certain amount of internal resistance.  The "trick" of this question is, of course, how to determine the panel's internal resistance.  Do not be surprised if a student suggests using the meter to measure the panel's resistance directly (though this will not work with a real photovoltaic panel).

Regarding the safety-oriented follow-up question, you might want to ask your students what the commonly accepted "shock hazard" voltage level is (30 volts).

%INDEX% Maximum power transfer theorem

%(END_NOTES)


