
%(BEGIN_QUESTION)
% Copyright 2003, Tony R. Kuphaldt, released under the Creative Commons Attribution License (v 1.0)
% This means you may do almost anything with this work of mine, so long as you give me proper credit

Suppose we needed to determine the resistance of the light bulb in this circuit, while energized:

$$\epsfbox{00307x01.eps}$$

Now, we know we cannot simply connect an ohmmeter to an energized circuit, so how is it possible to obtain the resistance measurement we desire?

\underbar{file 00307}
%(END_QUESTION)





%(BEGIN_ANSWER)

Measure voltage across the light bulb, and current through the light bulb, and then use Ohm's Law to calculate filament resistance from these measurements.

\vskip 10pt

Challenge question: explain how it would be possible to make these measurements without de-energizing the light bulb (not even once!).

%(END_ANSWER)





%(BEGIN_NOTES)

Quite often, resistance measurements on a component are not practical because the component cannot be "powered down," or perhaps because the component's resistance is different when energized versus when de-energized (incandescent light bulbs being an excellent example of this).  Knowing how to derive a resistance measurement from voltage and current measurements is a valuable skill.

The "challenge" question, to measure light bulb voltage and current without breaking the circuit, is also very practical.  Once students figure out how to do this (no, using an inductive ammeter is {\it not} an acceptable answer!), it would be a great classroom activity to actually build such a circuit and have students practice the technique on it.

%INDEX% Ohmmeter usage

%(END_NOTES)


