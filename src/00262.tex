
%(BEGIN_QUESTION)
% Copyright 2003, Tony R. Kuphaldt, released under the Creative Commons Attribution License (v 1.0)
% This means you may do almost anything with this work of mine, so long as you give me proper credit

A very useful method of measuring current through a wire is to measure the strength of the magnetic field around it.  This type of ammeter is known as a {\it clamp-on} ammeter:

$$\epsfbox{00262x01.eps}$$

Knowing the principle behind this meter's operation, describe what current values will be indicated by the three clamp-on ammeters in this circuit:
 
$$\epsfbox{00262x02.eps}$$

\medskip
\item{$\bullet$} Meter A = 
\item{$\bullet$} Meter B = 
\item{$\bullet$} Meter C = 
\medskip

\underbar{file 00262}
%(END_QUESTION)





%(BEGIN_ANSWER)

\medskip
\item{$\bullet$} Meter A = 2.5 amps
\item{$\bullet$} Meter B = 2.5 amps
\item{$\bullet$} Meter C = 0 amps
\medskip

%(END_ANSWER)





%(BEGIN_NOTES)

Clamp-on meters are very useful pieces of test equipment, but they must be used properly.  I have seen many people make the mistake of clamping one of these ammeters around multiple wires when trying to measure the amount of current through only one.  If you have any clamp-on meters in your classroom, have your students set up a simple circuit like this and prove the validity of the concept.

%(END_NOTES)


