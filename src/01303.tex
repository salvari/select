
%(BEGIN_QUESTION)
% Copyright 2003, Tony R. Kuphaldt, released under the Creative Commons Attribution License (v 1.0)
% This means you may do almost anything with this work of mine, so long as you give me proper credit

Like real-number algebra, Boolean algebra is subject to the laws of {\it commutation}, {\it association}, and {\it distribution}.  These laws allow us to build different logic circuits that perform the same logic function.

For each of the equivalent circuit pairs shown, write the corresponding Boolean law next to it:

$$\epsfbox{01303x01.eps}$$

$$\epsfbox{01303x02.eps}$$

Note: the three short, parallel lines represent "equivalent to" in mathematics.

\underbar{file 01303}
%(END_QUESTION)





%(BEGIN_ANSWER)

In order, from top to bottom:

$$AB = BA$$

$$(AB)C = A(BC)$$

$$(A+B)C = AC + BC$$

$$A+B = B+A$$

$$(A+C)B = AB + CB$$

$$(A+B)+C = A+(B+C)$$

%(END_ANSWER)





%(BEGIN_NOTES)

The commutative, associative, and distributive laws of Boolean algebra are identical to the respective laws in real number algebra.  These should not be difficult concepts for your students to understand.  The real benefit of working through these examples is to associate gate and relay logic circuits with Boolean expressions, and to see that Boolean algebra is nothing more than a symbolic means of representing electrical discrete-state (on/off) circuits.  In relating otherwise abstract mathematical concepts to something tangible, students build a much better comprehension of the concepts.

%INDEX% Boolean algebra; commutative, associative, and distributive laws

%(END_NOTES)


