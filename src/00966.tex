
%(BEGIN_QUESTION)
% Copyright 2005, Tony R. Kuphaldt, released under the Creative Commons Attribution License (v 1.0)
% This means you may do almost anything with this work of mine, so long as you give me proper credit

Some common-emitter amplifier circuits use {\it partial bypassing} of emitter resistance, with the bypass capacitor connected in parallel with only one of two series resistors:

$$\epsfbox{00966x01.eps}$$

Explain the purpose of this arrangement.  How does this differ in performance from the simple one-resistor emitter feedback design, or a grounded-emitter amplifier with no emitter resistor at all?

\underbar{file 00966}
%(END_QUESTION)





%(BEGIN_ANSWER)

This design is a compromise between full bypassing and no emitter resistor at all.  It provides all the DC voltage gain and Q-point stability of full bypassing, while providing more AC voltage gain stability than full bypassing.

%(END_ANSWER)





%(BEGIN_NOTES)

After reviewing the simple (no-resistor) common-emitter circuit design, and the full-bypass design, it should be apparent to students that this circuit is a hybrid of the two previous designs.  Likewise, it should come as little surprise that its performance characteristics lie somewhere between the two previous designs.

%INDEX% Partial bypassing, common-emitter amplifier
%INDEX% Bypass capacitor, common-emitter amplifier

%(END_NOTES)


