
%(BEGIN_QUESTION)
% Copyright 2003, Tony R. Kuphaldt, released under the Creative Commons Attribution License (v 1.0)
% This means you may do almost anything with this work of mine, so long as you give me proper credit

An {\it intervalometer} is a device that measures the interval of time between two events.  Such devices are commonly used to measure the speed of projectiles, given a known distance between two sensors:

$$\epsfbox{00444x01.eps}$$

A crude intervalometer may be constructed using two thin wires as sensors, which are broken by the passage of the projectile.  The two wires are connected in an RC circuit as such:

$$\epsfbox{00444x02.eps}$$

In order for this circuit to function properly as an intervalometer, which wire does the projectile need to break first?  Explain why.  Also, the voltmeter used in this instrument must be one with as high an input resistance as possible for best accuracy.  Explain why this is necessary as well.

Which will produce a greater voltage indication after the test, a fast projectile or a slow projectile?  Explain your answer.

\underbar{file 00444}
%(END_QUESTION)





%(BEGIN_ANSWER)

Wire \#1 needs to be closest to the projectile source (the gun), while wire \#2 needs to be further downrange.

%(END_ANSWER)





%(BEGIN_NOTES)

This is an interesting question because it requires the students to reason through the function of a practical measuring circuit.  Not only must students grasp the charge/discharge behavior of a capacitor, but they also must relate it to the practical purpose of the intervalometer, recognizing the importance of voltmeter characteristics as well.  Expect substantial discussion on this question.

%INDEX% Intervalometer, based on simple RC time constant circuit
%INDEX% RC time constant circuit, intervalometer

%(END_NOTES)


