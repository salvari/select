
%(BEGIN_QUESTION)
% Copyright 2003, Tony R. Kuphaldt, released under the Creative Commons Attribution License (v 1.0)
% This means you may do almost anything with this work of mine, so long as you give me proper credit

Many digital multimeters have a "diode check" range that allows the user to measure the forward voltage drop of a PN junction:

$$\epsfbox{02041x01.eps}$$

When using a multimeter with this feature to identify the terminals of a bipolar junction transistor, the forward voltage drop indication is necessary to distinguish the collector terminal from the emitter terminal.  Explain how this distinction is made on the basis of the forward voltage measurement, and also explain why this is.

\underbar{file 02041}
%(END_QUESTION)





%(BEGIN_ANSWER)

The emitter-base junction has a slightly greater forward voltage drop than the base-collector junction.  I'll let you explain why!

%(END_ANSWER)





%(BEGIN_NOTES)

I am surprised how many textbooks do not explain how to identify BJT terminals using a multimeter (especially a multimeter with the "diode check" function).  This is a very important skill for technicians to have, as they will often be faced with transistor terminal identification in the absence of datasheets or other graphical references to device terminals.

%INDEX% BJT, terminal identification

%(END_NOTES)


