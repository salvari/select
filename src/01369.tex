
%(BEGIN_QUESTION)
% Copyright 2003, Tony R. Kuphaldt, released under the Creative Commons Attribution License (v 1.0)
% This means you may do almost anything with this work of mine, so long as you give me proper credit

Suppose a student wants to build a sound-controlled lamp control circuit, whereby a single clap or other loud burst of noise turns the lamp on, and another single clap turns it off.  The sound-detection and lamp-drive circuitry is shown here:

$$\epsfbox{01369x01.eps}$$

Add a J-K flip-flop to this schematic diagram to implement the toggling function.

\underbar{file 01369}
%(END_QUESTION)





%(BEGIN_ANSWER)

$$\epsfbox{01369x02.eps}$$

%(END_ANSWER)





%(BEGIN_NOTES)

Some students may ask whether there is any significance to using the $\overline{Q}$ output rather than the $Q$.  Discuss this with your students: whether they think it will make any difference, or if it was just an arbitrary choice made by the circuit's designer.  Then, ask them how they would go about {\it proving} their judgment.

There are plenty of "what if" failure scenarios you could ask your students about here, challenging them to analyze this circuit with a troubleshooting perspective.  If time permits, have some fun with this.

%INDEX% J-K flip-flop, used as toggle for sound-activated control
%INDEX% J-K flip-flop, toggle mode
%INDEX% Toggle mode, J-K flip-flop

%(END_NOTES)


