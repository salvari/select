
%(BEGIN_QUESTION)
% Copyright 2003, Tony R. Kuphaldt, released under the Creative Commons Attribution License (v 1.0)
% This means you may do almost anything with this work of mine, so long as you give me proper credit

$$\epsfbox{01693x01.eps}$$

\underbar{file 01693}
\vfil \eject
%(END_QUESTION)





%(BEGIN_ANSWER)

You may use circuit simulation software to set up similar oscilloscope display interpretation scenarios, for practice or for verification of what you see in this exercise.

%(END_ANSWER)





%(BEGIN_NOTES)

Use a sine-wave function generator for the AC voltage source, and be sure set the frequency to some reasonable value (well within the capability of a multimeter to measure).  It is very important that students learn to convert between peak and RMS measurements for sine waves, but you might want to mix things up a bit by having them do the same with triangle waves and square waves as well!  It is vital that students realize the rule of $V_{RMS} = {V_{peak} \over \sqrt{2}}$ only holds for {\it sinusoidal} signals.

If you do choose to challenge students with non-sinusoidal waveshapes, be very sure that they do their voltmeter measurements using {\it true-RMS} meters!  This means no analog voltmeters, which are "miscalibrated" so their inherently average-responding movements register (sinusoidal) RMS accurately.  Your students must use true-RMS digital voltmeters in order for their non-sinusoidal RMS measurements to correlate with their calculations.

Incidentally, this lab exercise also works well as a demonstration of the importance of true-RMS indicating meters, comparing the indications of analog, non-true-RMS digital, and true-RMS digital on the same non-sinusoidal waveform!

%INDEX% Assessment, performance-based (RMS vs. peak voltage measurements)

%(END_NOTES)


