
%(BEGIN_QUESTION)
% Copyright 2003, Tony R. Kuphaldt, released under the Creative Commons Attribution License (v 1.0)
% This means you may do almost anything with this work of mine, so long as you give me proper credit

We know that the current in a series circuit may be calculated with this formula:

$$I = {E_{total} \over R_{total}}$$

We also know that the voltage dropped across any single resistor in a series circuit may be calculated with this formula:

$$E_R = I R$$

Combine these two formulae into one, in such a way that the $I$ variable is eliminated, leaving only $E_R$ expressed in terms of $E_{total}$, $R_{total}$, and $R$.

\underbar{file 00360}
%(END_QUESTION)





%(BEGIN_ANSWER)

$$E_R = E_{total} \Bigl({R \over R_{total}}\Bigr)$$

\vskip 10pt

Follow-up question: algebraically manipulate this equation to solve for $E_{total}$ in terms of all the other variables.  In other words, show how you could calculate for the amount of total voltage necessary to produce a specified voltage drop ($E_R$) across a specified resistor ($R$), given the total circuit resistance ($R_{total}$).

%(END_ANSWER)





%(BEGIN_NOTES)

Though this "voltage divider formula" may be found in any number of electronics reference books, your students need to understand how to algebraically manipulate the given formulae to arrive at this one.

%INDEX% Algebra, manipulating equations
%INDEX% Algebra, substitution
%INDEX% Voltage divider formula

%(END_NOTES)


