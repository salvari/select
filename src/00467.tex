
%(BEGIN_QUESTION)
% Copyright 2003, Tony R. Kuphaldt, released under the Creative Commons Attribution License (v 1.0)
% This means you may do almost anything with this work of mine, so long as you give me proper credit

% Uncomment the following line if the question involves calculus at all:
\vbox{\hrule \hbox{\strut \vrule{} $\int f(x) \> dx$ \hskip 5pt {\sl Calculus alert!} \vrule} \hrule}

If the number of turns of wire in an electromagnet coil is tripled, what happens to the magnitude of the magnetic flux ($\Phi$) generated by it, assuming that none of the other variables change (current through the coil, reluctance of magnetic circuit, etc.)?

\vskip 5pt

If the number of turns of wire in an inductor is tripled, what happens to the magnitude of the induced voltage for a given rate of magnetic flux change over time ($d\phi \over dt$)?

\vskip 5pt

If the number of turns of wire in an inductor is tripled, what happens to the magnitude of its inductance, measured in Henrys?  Explain your answer.

\underbar{file 00467}
%(END_QUESTION)





%(BEGIN_ANSWER)

If $N$ triples, then $\Phi$ triples, all other factors being equal.

\vskip 5pt

If ${d\phi \over dt}$ triples, then $e$ triples, all other factors being equal.

\vskip 5pt

If $N$ triples, then $L$ increases by a factor of {\it nine}, all other factors being equal.

%(END_ANSWER)





%(BEGIN_NOTES)

This question presents an interesting problem in qualitative mathematics.  It is closely related to the "chain rule" in calculus, where one function $y = f(x)$ is embedded within another function $z = f(y)$, such that ${dz \over dy}{dy \over dx} = {dz \over dx}$.  The purpose of this exercise is for students to gain a conceptual grasp of why inductance does not vary linearly with changes in $N$.

Of course, students can obtain the same (third) answer just by looking at the inductance formula (in terms of $N$, $\mu$, $A$, and $l$), without all the conceptual work.  It would be good, in fact, if a student happens to derive the same answer by inspection of this formula, just to add variety to the discussion.  But the real purpose of this question, again, is a conceptual understanding of that formula.

%INDEX% Electromagnetism
%INDEX% Electromagnetic induction
%INDEX% Inductance, conceptual

%(END_NOTES)


