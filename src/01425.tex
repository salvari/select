
%(BEGIN_QUESTION)
% Copyright 2003, Tony R. Kuphaldt, released under the Creative Commons Attribution License (v 1.0)
% This means you may do almost anything with this work of mine, so long as you give me proper credit

Where are the power supply pins most commonly located for integrated circuits in the standard "DIP" form?  Some manufacturers of high-speed digital ICs are now relocating power supply pins to the middle of the package, like this:

$$\epsfbox{01425x01.eps}$$

Explain the rationale for this non-traditional pin assignment.

\underbar{file 01425}
%(END_QUESTION)





%(BEGIN_ANSWER)

Power supply pins on DIP chips are usually located in the upper left and lower-right corners (pins 14 and 7 on a 14-pin DIP, pins 16 and 8 on a 16-pin DIP), {\it but this is not always the case!}.

Centered power supply pins on modern high-speed DIPs exhibit less parasitic inductance than corner-located pins.

\vskip 10pt

Follow-up question: what is the best way to determine pin assignments on any given integrated circuit?

%(END_ANSWER)





%(BEGIN_NOTES)

Let your students know that even some of the older integrated circuits had "weird" power supply pin locations, so they cannot assume corner-pins for every older DIP they see!  An example of this is the 2102L static RAM, or the 4049 CMOS hex inverting buffer.

Even stranger is DIP chips with {\it multiple} +V and GND pins.  Good examples of this include the 74AC11004 and the 74AUC16373.

%INDEX% Power supply pin locations, on integrated circuits
%INDEX% IC, power supply pin locations

%(END_NOTES)


