
%(BEGIN_QUESTION)
% Copyright 2003, Tony R. Kuphaldt, released under the Creative Commons Attribution License (v 1.0)
% This means you may do almost anything with this work of mine, so long as you give me proper credit

Voltage divider circuits may be constructed from reactive components just as easily as they may be constructed from resistors.  Take this capacitive voltage divider, for instance:

$$\epsfbox{00638x01.eps}$$

Calculate the magnitude and phase shift of $V_{out}$.  Also, describe what advantages a capacitive voltage divider might have over a resistive voltage divider.

\underbar{file 00638}
%(END_QUESTION)





%(BEGIN_ANSWER)

${\bf V_{out}}$ = 1.754 V $\angle$ 0$^{o}$

\vskip 10pt

Follow-up question \#1: explain why the division ratio of a capacitive voltage divider remains constant with changes in signal frequency, even though we know that the reactance of the capacitors ($X_{C1}$ and $X_{C2}$) will change.

\vskip 10pt

Follow-up question \#2: one interesting feature of capacitive voltage dividers is that they harbor the possibility of electric shock after being disconnected from the voltage source, if the source voltage is high enough and if the disconnection happens at just the right time.  Explain why a capacitive voltage divider poses this threat whereas a resistive voltage divider does not.  Also, identify what the {\it time} of disconnection from the AC voltage source has to do with shock hazard.

%(END_ANSWER)





%(BEGIN_NOTES)

Capacitive voltage dividers find use in high-voltage AC instrumentation, due to some of the advantages they exhibit over resistive voltage dividers.  Your students should take special note of the phase angle for the capacitor's voltage drop.  Why it is 0 degrees, and not some other angle?

%INDEX% Voltage divider, capacitive

%(END_NOTES)


