
%(BEGIN_QUESTION)
% Copyright 2005, Tony R. Kuphaldt, released under the Creative Commons Attribution License (v 1.0)
% This means you may do almost anything with this work of mine, so long as you give me proper credit

There is something wrong with this amplifier circuit.  Note the relative amplitudes of the input and output signals as measured by an oscilloscope:

$$\epsfbox{02465x01.eps}$$

This circuit used to function perfectly, but then began to malfunction in this manner: producing a "clipped" output waveform of excessive amplitude.  Determine the approximate amplitude that the output voltage waveform {\it should} be for the component values given in this circuit, and then identify possible causes of the problem and also elements of the circuit that you know cannot be at fault.

\underbar{file 02465}
%(END_QUESTION)





%(BEGIN_ANSWER)

$V_{out}$ (ideal) = 1.01 volts RMS

\vskip 10pt

I'll let you determine possible faults in the circuit!  From what we see here, we know the power supply is functioning (both +V and -V rails) and that there is good signal getting to the noninverting input of the opamp.

%(END_ANSWER)





%(BEGIN_NOTES)

There is definitely more than one possible cause for the observed problem.  Discuss alternatives with your students, involving them in the diagnosis process.  Ask them why we know that certain elements of the circuit are functioning as they should?  Of the possible causes, which are more likely, and why?

%INDEX% Troubleshooting, noninverting amplifier circuit (opamp)

%(END_NOTES)


