
%(BEGIN_QUESTION)
% Copyright 2005, Tony R. Kuphaldt, released under the Creative Commons Attribution License (v 1.0)
% This means you may do almost anything with this work of mine, so long as you give me proper credit

% Uncomment the following line if the question involves calculus at all:
\vbox{\hrule \hbox{\strut \vrule{} $\int f(x) \> dx$ \hskip 5pt {\sl Calculus alert!} \vrule} \hrule}

You are part of a team building a rocket to carry research instruments into the high atmosphere.  One of the variables needed by the on-board flight-control computer is velocity, so it can throttle engine power and achieve maximum fuel efficiency.  The problem is, none of the electronic sensors on board the rocket has the ability to directly measure velocity.  What is available is an {\it altimeter}, which infers the rocket's altitude (it position away from ground) by measuring ambient air pressure; and also an {\it accelerometer}, which infers acceleration (rate-of-change of velocity) by measuring the inertial force exerted by a small mass.

The lack of a "speedometer" for the rocket may have been an engineering design oversight, but it is still your responsibility as a development technician to figure out a workable solution to the dilemma.  How do you propose we obtain the electronic velocity measurement the rocket's flight-control computer needs?

\underbar{file 02702}
%(END_QUESTION)





%(BEGIN_ANSWER)

One possible solution is to use an electronic {\it integrator} circuit to derive a velocity measurement from the accelerometer's signal.  However, this is not the only possible solution!

%(END_ANSWER)





%(BEGIN_NOTES)

This question simply puts students' comprehension of basic calculus concepts (and their implementation in electronic circuitry) to a practical test.

%INDEX% Calculus, derivative (applied to motion)
%INDEX% Calculus, integral (applied to motion)

%(END_NOTES)


