
%(BEGIN_QUESTION)
% Copyright 2005, Tony R. Kuphaldt, released under the Creative Commons Attribution License (v 1.0)
% This means you may do almost anything with this work of mine, so long as you give me proper credit

A very popular active filter topology is called the {\it Sallen-Key}.  Two examples of Sallen-Key active filter circuits are shown here:

$$\epsfbox{00765x01.eps}$$

$$\epsfbox{00765x02.eps}$$

Determine which of these Sallen-Key filters is low pass, and which is high pass.  Explain your answers.  

\underbar{file 00765}
%(END_QUESTION)





%(BEGIN_ANSWER)

The first filter shown is low pass, while the second filter shown is high pass.

\vskip 10pt

Challenge question: what is the purpose of resistor $R_3$ in each circuit?

%(END_ANSWER)





%(BEGIN_NOTES)

The word "topology" may be strange to your students.  If any of them ask you what it means, ask them if they own a dictionary!

Like all the other active filter circuits, the fundamental characteristic of each filter may be determined by qualitative analysis at $f = 0$ and $f = \infty$.  This is a form of {\it thought experiment}: determining the characteristics of a circuit by imagining the effects of certain given conditions, following through by analysis based on "first principles" of circuits, rather than by researching what the circuit's intended function is.

Resistor $R_3$ is actually not essential to the circuit's operation, but is normally found in Sallen-Key filters anyway.  If it makes the analysis of the circuit any simpler, tell your students they may replace that resistor with a straight wire in their schematic diagrams.

%INDEX% Active filter circuit, Sallen-Key

%(END_NOTES)


