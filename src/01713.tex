
%(BEGIN_QUESTION)
% Copyright 2003, Tony R. Kuphaldt, released under the Creative Commons Attribution License (v 1.0)
% This means you may do almost anything with this work of mine, so long as you give me proper credit

When an electric current travels through an electrical resistance, not only will there be a voltage "drop" across that resistance ($V = IR$), but there will also be energy dissipated by that resistance.  We describe the rate of energy dissipation over time as {\it power} ($P$), and we express power in the unit of the {\it watt} (W).

Write the equation relating power to voltage and current, for an electrical resistance.  Also, describe what physical form this dissipated energy usually takes.

\underbar{file 01713}
%(END_QUESTION)





%(BEGIN_ANSWER)

$$P = IV$$

Energy dissipated by an electrical resistance is usually manifest as {\it heat}, although sometimes it is also in the form of {\it light}.

%(END_ANSWER)





%(BEGIN_NOTES)

Most people, even if they have no prior experience with electric circuit analysis, have some sense of what a "Watt" is, because so many consumer-grade appliances are rated in watts (light bulbs, heaters, blow driers, etc.).  Use these examples as context for your discussion on electric power.

%INDEX% Energy, versus power
%INDEX% Power, versus energy
%INDEX% Power dissipation, resistance 

%(END_NOTES)


