
%(BEGIN_QUESTION)
% Copyright 2003, Tony R. Kuphaldt, released under the Creative Commons Attribution License (v 1.0)
% This means you may do almost anything with this work of mine, so long as you give me proper credit

% Uncomment the following line if the question involves calculus at all:
\vbox{\hrule \hbox{\strut \vrule{} $\int f(x) \> dx$ \hskip 5pt {\sl Calculus alert!} \vrule} \hrule}

What would happen to the magnetic flux inside an air-core inductor made of superconducting wire (no electrical resistance at all), if a constant DC voltage were applied to that coil?  Remember, this is an ideal scenario, where the only mathematical function describing the resulting flux is that which relates magnetic flux to voltage and time!

\underbar{file 01488}
%(END_QUESTION)





%(BEGIN_ANSWER)

Ideally, the flux would increase from zero in a linear fashion over time.

\vskip 10pt

Follow-up question: what would happen with an {\it iron-cored} inductor, with the same superconducting (zero-resistance) wire?

%(END_ANSWER)





%(BEGIN_NOTES)

Discuss with your students why the flux increases linearly, as described by Faraday's Law of electromagnetic induction.  When discussing the iron-core scenario, be sure to mention magnetic saturation if your students have not considered it!

%INDEX% Magnetic flux, relation to coil voltage over time
%INDEX% Superconducting wire

%(END_NOTES)


