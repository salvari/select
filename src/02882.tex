
%(BEGIN_QUESTION)
% Copyright 2005, Tony R. Kuphaldt, released under the Creative Commons Attribution License (v 1.0)
% This means you may do almost anything with this work of mine, so long as you give me proper credit

Something is wrong with this building alarm system circuit.  The alarm siren energizes even when all the windows and doors are shut.  The only way to silence the alarm is to use the "override" key switch:

$$\epsfbox{02882x01.eps}$$

Using your logic probe, you measure a high signal at TP7 and a low signal at TP6 with all windows and doors shut, and with the key switch in the "override" position.  From this information, identify two possible faults that could account for the problem and all measured values in this circuit.  Then, choose one of those possible faults and explain why you think it could be to blame.  The circuit elements you identify as possibly faulted can be wires, traces, and connections as well as components.  Be as specific as you can in your answers, identifying both the circuit element and the type of fault.

\medskip
\goodbreak
\item{$\bullet$} Circuit elements that are possibly faulted
\item{1.}
\item{2.} 
\medskip

\medskip
\goodbreak
\item{$\bullet$} Explanation of {\it why} you think one of the above faults could be to blame

\vskip 30pt

\underbar{file 02882}
%(END_QUESTION)





%(BEGIN_ANSWER)

Note: the following answers are not exhaustive.  There may be more circuit elements possibly at fault!

\medskip
\goodbreak
\item{$\bullet$} Circuit elements that are possibly faulted
\item{1.} $SW_1$ contacts not closing (dirty or worn)
\item{2.} $SW_2$ contacts not closing (dirty or worn)
\item{3.} Broken wire between $SW_1$ and TP1
\item{4.} Broken wire between $SW_2$ and TP2
\item{5.} Broken wire between $SW_1$ and ground
\item{6.} Broken wire between $SW_2$ and ground
\item{7.} $U_3$ output stuck high
\item{8.} $U_1$ output stuck high
\medskip

%(END_ANSWER)





%(BEGIN_NOTES)

Ask your students to identify means by which they could confirm suspected circuit elements, by measuring something other than what has already been measured.

Troubleshooting scenarios are always good for stimulating class discussion.  Be sure to spend plenty of time in class with your students developing efficient and logical diagnostic procedures, as this will assist them greatly in their careers.

%INDEX% Troubleshooting, logic gate circuit

%(END_NOTES)


