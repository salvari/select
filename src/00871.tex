
%(BEGIN_QUESTION)
% Copyright 2003, Tony R. Kuphaldt, released under the Creative Commons Attribution License (v 1.0)
% This means you may do almost anything with this work of mine, so long as you give me proper credit

At the heart of every amplifier is a device that uses one signal to control another.  In electronics, this means a device that uses a small voltage or current signal to control a larger voltage or current.

The first electronic amplifying circuits were constructed with devices called {\it electron tubes} instead of {\it transistors}.  Tubes still find specialized applications in electronics, but they have largely been replaced by transistors.  Why is this?  What advantages do transistors have over tubes as amplifying devices? 

\underbar{file 00871}
%(END_QUESTION)





%(BEGIN_ANSWER)

Transistors are typically much more physically rugged than electron tubes, able to withstand greater levels of vibration and stress.  They are also smaller, and more energy efficient in most applications.

%(END_ANSWER)





%(BEGIN_NOTES)

Electron tubes used to be the "workhorses" of the electronics world, acting as power control and amplification devices for a wide range of applications.  It should be interesting to listen to your students' feedback on this question, being that there is a lot of "tube" information on the internet.  {\it IEEE Spectrum} magazine had a couple of excellent articles on electron tubes and their applications, which I would strongly encourage any interested students to read.

%INDEX% Electron tube versus transistor
%INDEX% Transistor versus Electron tube

%(END_NOTES)


