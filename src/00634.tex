
%(BEGIN_QUESTION)
% Copyright 2003, Tony R. Kuphaldt, released under the Creative Commons Attribution License (v 1.0)
% This means you may do almost anything with this work of mine, so long as you give me proper credit

Speakers used for audio reproduction systems (stereos, public address systems, etc.) act as power {\it loads} to the amplifiers which drive them.  These devices convert electrical energy into sound energy, which then dissipates into the surrounding air.  In this manner, a speaker acts much like a resistor: converting one form of energy (electrical) into another, and then dissipating that energy into the surrounding environment.  Naturally, it makes sense to describe the nature of such loads in units of "ohms" ($\Omega$), so that they may be mathematically analyzed in a manner similar to resistors.

Yet, despite the dissipative nature of audio speakers, their "ohms" rating is specified as an {\it impedance} rather than a {\it resistance} or a {\it reactance}.  Explain why this is.

\underbar{file 00634}
%(END_QUESTION)





%(BEGIN_ANSWER)

The term "resistance" refers to the very specific phenomenon of electrical "friction," converting electrical energy into thermal energy.  The term "reactance" refers to electric current opposition resulting from a non-dissipative {\it exchange} of energy between the component and the rest of the circuit.  The term "impedance" refers to any form of opposition to electric current, whether that opposition be dissipative or non-dissipative in nature.

While speakers are primarily dissipative devices, most of the energy dissipated by a speaker is {\it not} in the form of heat.

%(END_ANSWER)





%(BEGIN_NOTES)

In a sense, resistance may be though of as a special (limiting) case of impedance, just as reactance is a special case of impedance.  Discuss this concept with your students, especially with reference to devices such as speakers which are dissipative in nature (they dissipate energy) but yet not resistive in the strict sense of the term.

For this reason, the word "impedance" finds broad application in the world of electronics, and even in some sciences outside of electricity/electronics!

%INDEX% Impedance, of audio speaker

%(END_NOTES)


