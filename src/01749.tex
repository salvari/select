
%(BEGIN_QUESTION)
% Copyright 2003, Tony R. Kuphaldt, released under the Creative Commons Attribution License (v 1.0)
% This means you may do almost anything with this work of mine, so long as you give me proper credit

Calculate the proper value of resistance $R_2$ needs to be in order to draw 40\% of the total current in this circuit:

$$\epsfbox{01749x01.eps}$$

\underbar{file 01749}
%(END_QUESTION)





%(BEGIN_ANSWER)

$R_2$ = 1.5 k$\Omega$

\vskip 10pt

Follow-up question: explain how you could arrive at a rough estimate of $R_2$'s necessary value without doing any algebra.  In other words, show how you could at least set limits on $R_2$'s value (i.e. "We know it has to be less than . . ." or "We know it has to be greater than . . .").

%(END_ANSWER)





%(BEGIN_NOTES)

This is an interesting problem to solve algebraically from the current divider formula.  I recommend using the product-over-sum formula for parallel resistance if you plan on doing this algebraically.  The estimation question (in the follow-up) is also very good to discuss with your students.  It is possible to at least "bracket" the value of $R_2$ between two different resistance values without doing any math more complex than simple (fractional) arithmetic.

Of course, a less refined approach to solving this problem would be to assume a certain battery voltage and work with numerical figures -- but what fun is that?

%INDEX% Current divider

%(END_NOTES)


