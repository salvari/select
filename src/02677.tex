
%(BEGIN_QUESTION)
% Copyright 2005, Tony R. Kuphaldt, released under the Creative Commons Attribution License (v 1.0)
% This means you may do almost anything with this work of mine, so long as you give me proper credit

The concept of a mathematical {\it power} is familiar to most students of algebra.  For instance, ten to the third power means this:

$$10^3 = 10 \times 10 \times 10 = 1000$$

. . . and eight to the seventh power means this:

$$8^7 = 8 \times 8 \times 8 \times 8 \times 8 \times 8 \times 8 = 2,097,152$$

Just as subtraction is the inverse function of addition, and division is the inverse function of multiplication (because with inverse functions, one "undoes" the other), there is also an inverse function for a power and we call it the {\it logarithm}.

Re-write the expression $10^3 = 1000$ so that it uses the same quantities (10, 3, and 1000) in the context of a logarithm instead of a power, just as the subtraction is shown here to be the inverse of addition, and division is shown to be the inverse of multiplication in the following examples:

$$3 + 8 = 11 \hbox{\hskip 30pt (+ and - are inverse functions) \hskip 30pt} 11 - 3 = 8$$

$$2 \times 7 = 14 \hbox{\hskip 30pt} (\times \hbox{ and} \div \hbox{are inverse functions)} \hbox{\hskip 30pt} 14 \div 2 = 7$$

$$10^3 = 1000 \hbox{\hskip 30pt (powers and logs are inverse functions) \hskip 30pt} \log_{10} \hbox{???} = \hbox{???}$$

\underbar{file 02677}
%(END_QUESTION)





%(BEGIN_ANSWER)

$$10^3 = 1000 \hbox{\hskip 30pt (powers and logs are inverse functions) \hskip 30pt} \log_{10} 1000 = 3$$

%(END_ANSWER)





%(BEGIN_NOTES)

In my experience, most American students are woefully underprepared for the subject of logarithms when they study with me.  Admittedly, logarithms do not see as much use in everyday life as powers do (and that is very little for most people as it is!).  Logarithms used to be common fare for secondary school and college students, as they were essential for the operation of a {\it slide rule}, an elegant mechanical analog computing device popular decades ago.  

The purpose of this question is to twofold: to get students to realize what a logarithm is, and also to remind them of the concept of inverse functions, which become very important in analog computational circuits.

%INDEX% Exponents and logarithms, defined as inverse functions 
%INDEX% Logarithms and exponents, defined as inverse functions 

%(END_NOTES)


