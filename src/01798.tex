
%(BEGIN_QUESTION)
% Copyright 2003, Tony R. Kuphaldt, released under the Creative Commons Attribution License (v 1.0)
% This means you may do almost anything with this work of mine, so long as you give me proper credit

\vbox{\hrule \hbox{\strut \vrule{} $\int f(x) \> dx$ \hskip 5pt {\sl Calculus alert!} \vrule} \hrule}

Suppose a capacitor is charged by a voltage source, and then switched to a resistor for discharging:

$$\epsfbox{01798x01.eps}$$

Would a larger capacitance value result in a slower discharge, or a faster discharge?  How about a larger resistance value?  You may find the "Ohm's Law" equation for capacitance helpful in answering both these questions:

$$i = C {dv \over dt}$$

\vskip 10pt

Now consider an inductor, "charged" by a current source and then switched to a resistor for discharging:

$$\epsfbox{01798x02.eps}$$

Would a larger inductance value result in a slower discharge, or a faster discharge?  How about a larger resistance value?  You may find the "Ohm's Law" equation for inductance helpful in answering both these questions:

$$v = L {di \over dt}$$

\underbar{file 01798}
%(END_QUESTION)





%(BEGIN_ANSWER)

For the RC circuit:

\item{$\bullet$} Larger capacitance = slower discharge
\item{$\bullet$} Larger resistance = slower discharge

\vskip 10pt

For the LR circuit:

\item{$\bullet$} Larger inductance = slower discharge
\item{$\bullet$} Larger resistance = faster discharge

%(END_ANSWER)





%(BEGIN_NOTES)

Students usually want to just use the $\tau = RC$ and $\tau = {L \over R}$ formulae to answer questions like this, but unfortunately this does not lend itself to a firm conceptual understanding of time-constant circuit behavior.  

If students need hints on how to answer the capacitance and inductance questions, ask them what the fundamental definitions of "capacitance" and "inductance" are (the ability to store energy . . .).  Then ask them what takes longer to discharge (given the same power, or rate of energy release per unit time), a large reservoir of energy or a small reservoir of energy.

If students need hints on how to answer the resistance questions, ask them what each type of reactive component resists change in (voltage for capacitors and current for inductors).  Then ask them what condition(s) are necessary to cause the most rapid change in those variables (high current for capacitors and high voltage for inductors).  This is most evident by inspection of the differential equations $i = C{dv \over dt}$ and $v = L {di \over dt}$.

%INDEX% Time constant, how to calculate for RC circuit
%INDEX% Time constant, how to calculate for LR circuit
%INDEX% Inductance, "Ohm's Law" for
%INDEX% Capacitance, "Ohm's Law" for

%(END_NOTES)


