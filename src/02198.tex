
%(BEGIN_QUESTION)
% Copyright 2004, Tony R. Kuphaldt, released under the Creative Commons Attribution License (v 1.0)
% This means you may do almost anything with this work of mine, so long as you give me proper credit

Suppose a set of three neon light bulbs were connected to an alternator with three sets of windings labeled {\bf A}, {\bf B}, and {\bf C}:

$$\epsfbox{02198x01.eps}$$

The schematic diagram for this alternator/lamp system is as follows:

$$\epsfbox{02198x02.eps}$$

If the alternator spins fast enough (clockwise, as shown), the AC voltage induced in its windings will be enough to cause the neon lamps to "blink" (neon bulbs have very fast reaction times and thus cannot maintain a glow for very long without current, unlike incandescent lamps which operate on the principle of a glowing-hot metal filament).  Most likely this blinking will be too fast to discern with the naked eye.

However, if we were to videorecord the blinking and play back the recording at a slow speed, we should be able to see the sequence of light flashes.  Determine the apparent "direction" of the lamps' blinking (from right-to-left or from left-to-right), and relate that sequence to the voltage peaks of each alternator coil pair.

\underbar{file 02198}
%(END_QUESTION)





%(BEGIN_ANSWER)

The lamps will blink from left to right ({\bf C}-{\bf B}-{\bf A}-{\bf C}-{\bf B}-{\bf A}).

\vskip 10pt

Follow-up question: suppose lamp {\bf B} stopped blinking, while lamps {\bf A} and {\bf C} continued.  Identify at least two possible causes for this failure.

%(END_ANSWER)





%(BEGIN_NOTES)

This question is really a prelude to discussing {\it phase rotation} in polyphase systems.

%INDEX% Phase rotation, illustrated

%(END_NOTES)


