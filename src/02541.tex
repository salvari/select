
%(BEGIN_QUESTION)
% Copyright 2005, Tony R. Kuphaldt, released under the Creative Commons Attribution License (v 1.0)
% This means you may do almost anything with this work of mine, so long as you give me proper credit

A common type of graph used to describe the operation of an electronic component or subcircuit is the {\it transfer characteristic}, showing the relationship between input signal and output signal.  For example, the transfer characteristic for a simple resistive voltage divider circuit is a straight line:

$$\epsfbox{02541x01.eps}$$

Once a transfer characteristic has been plotted, it may be used to predict the output signal of a circuit given any particular input signal.  In this case, the transfer characteristic plot for the 2:1 voltage divider circuit tells us that the circuit will output +3 volts for an input of +6 volts:

$$\epsfbox{02541x02.eps}$$

We may use the same transfer characteristic to plot the output of the voltage divider given an AC waveform input:

$$\epsfbox{02541x03.eps}$$

While this example (a voltage divider with a 2:1 ratio) is rather trivial, it shows how transfer characteristics may be used to predict the output signal of a network given a certain input signal condition.  Where transfer characteristic graphs are more practical is in predicting the behavior of {\it nonlinear} circuits.  For example, the transfer characteristic for an ideal half-wave rectifier circuit looks like this:

$$\epsfbox{02541x04.eps}$$

Sketch the transfer characteristic for a realistic diode (silicon, with 0.7 volts forward drop), and use this characteristic to plot the half-wave rectified output waveform given a sinusoidal input:

$$\epsfbox{02541x05.eps}$$

\underbar{file 02541}
%(END_QUESTION)





%(BEGIN_ANSWER)

$$\epsfbox{02541x06.eps}$$

%(END_ANSWER)





%(BEGIN_NOTES)

Transfer characteristic graphs provide an elegant method to sketch the output waveshape for any electrical network, linear or nonlinear.  The method in which points along an input waveform are reflected and {\it transferred} to equivalent points on the output waveform justifies the name of this analytical tool.  Make sure your students get the opportunity to learn how to use this tool, as it can provide great insight into distortion in electronic and electromagnetic devices.

%INDEX% Transfer characteristic, defined using voltage divider circuit as an example

%(END_NOTES)


