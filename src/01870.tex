
%(BEGIN_QUESTION)
% Copyright 2003, Tony R. Kuphaldt, released under the Creative Commons Attribution License (v 1.0)
% This means you may do almost anything with this work of mine, so long as you give me proper credit

There is a direct, mathematical relationship between bandwidth, resonant frequency, and Q in a resonant filter circuit, but imagine for a moment that you forgot exactly what that formula was.  You think it must be one of these two, but you are not sure which:

$$\hbox{Bandwidth} = {Q \over f_r} \hskip 20pt \hbox{(or possibly)} \hskip 20pt \hbox{Bandwidth} = {f_r \over Q}$$

Based on your conceptual knowledge of how a circuit's quality factor affects its frequency response, determine which of these formulae must be incorrect.  In other words, {\it demonstrate} which of these must be correct rather than simply looking up the correct formula in a reference book.

\underbar{file 01870}
%(END_QUESTION)





%(BEGIN_ANSWER)

Hint: the greater the value of $Q$, the less bandwidth a resonant circuit will have.

%(END_ANSWER)





%(BEGIN_NOTES)

The purpose of this question is not necessarily to get students to look this formula up in a book, but rather to develop their qualitative reasoning and critical thinking skills.  Forgetting the exact form of an equation is not a rare event, and it pays to be able to select between different forms based on a conceptual understanding of what the formula is supposed to predict.

Note that the question asks students to identify the {\it wrong} formula, and not to tell which one is right.  If all we have are these to formulae to choose from, and a memory too weak to confidently recall the correct form, the best that logic can do is eliminate the wrong formula.  The formula making the most sense according to our qualitative analysis may or may not be {\it precisely} right, because we could very well be forgetting a multiplicative constant (such as $2 \pi$).

%INDEX% Quality factor of series LC resonant circuit
%INDEX% Quality factor, relationship to bandwidth

%(END_NOTES)


