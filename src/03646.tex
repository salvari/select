
%(BEGIN_QUESTION)
% Copyright 2005, Tony R. Kuphaldt, released under the Creative Commons Attribution License (v 1.0)
% This means you may do almost anything with this work of mine, so long as you give me proper credit

\vbox{\hrule \hbox{\strut \vrule{} $\int f(x) \> dx$ \hskip 5pt {\sl Calculus alert!} \vrule} \hrule}

Define what "derivative" means when applied to the graph of a function.  For instance, examine this graph:

$$\epsfbox{03646x01.eps}$$

Label all the points where the derivative of the function ($dy \over dx$) is positive, where it is negative, and where it is equal to zero.

\underbar{file 03646}
%(END_QUESTION)





%(BEGIN_ANSWER)

The graphical interpretation of "derivative" means the {\it slope} of the function at any given point.

$$\epsfbox{03646x02.eps}$$

%(END_ANSWER)





%(BEGIN_NOTES)

Usually students find the concept of the derivative easiest to understand in graphical form: being the {\it slope} of the graph.  This is true whether or not the independent variable is time (an important point given that most "intuitive" examples of the derivative are time-based!).

%INDEX% Calculus, derivative (defined in a graphical sense)

%(END_NOTES)


