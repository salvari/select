
%(BEGIN_QUESTION)
% Copyright 2003, Tony R. Kuphaldt, released under the Creative Commons Attribution License (v 1.0)
% This means you may do almost anything with this work of mine, so long as you give me proper credit

What would happen to this meter movement, if connected directly to a 6-volt battery?

$$\epsfbox{00718x01.eps}$$

\underbar{file 00718}
%(END_QUESTION)





%(BEGIN_ANSWER)

Two things would happen: first, the movement would most likely be damaged from excessive current.  Secondly, the needle would move to the left instead of the right (as it normally should), because the polarity is backward.

%(END_ANSWER)





%(BEGIN_NOTES)

When an electromechanical meter movement is overpowered, causing the needle to "slam" all the way to one extreme end of motion, it is commonly referred to as "pegging" the meter.  I've seen meter movements that have been "pegged" so badly that the needles are {\it bent} from hitting the stop!

Based on your students knowledge of meter movement design, ask them to tell you what they think might become damaged in a severe over-power incident such as this.  Tell them to be specific in their answers.

%INDEX% Overrange, meter movement

%(END_NOTES)


