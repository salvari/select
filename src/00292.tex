
%(BEGIN_QUESTION)
% Copyright 2003, Tony R. Kuphaldt, released under the Creative Commons Attribution License (v 1.0)
% This means you may do almost anything with this work of mine, so long as you give me proper credit

In a parallel circuit, certain general rules may be stated with regard to quantities of voltage, current, resistance, and power.  Express these rules, using your own words:

\vskip 10pt

\noindent
"In a parallel circuit, voltage . . ."

\vskip 10pt

\noindent
"In a parallel circuit, current . . ."

\vskip 10pt

\noindent
"In a parallel circuit, resistance . . ."

\vskip 10pt

\noindent
"In a parallel circuit, power . . ."

\vskip 10pt

For each of these rules, explain {\it why} it is true.

\underbar{file 00292}
%(END_QUESTION)





%(BEGIN_ANSWER)

\noindent
"In a parallel circuit, voltage {\it is equal across all components}."

\vskip 10pt

\noindent
"In a parallel circuit, current{\it s add to equal the total}."

\vskip 10pt

\noindent
"In a parallel circuit, resistance{\it s diminish to equal the total}."

\vskip 10pt

\noindent
"In a parallel circuit, power {\it dissipations add to equal the total}."

%(END_ANSWER)





%(BEGIN_NOTES)

Rules of series and parallel circuits are very important for students to comprehend.  However, a trend I have noticed in many students is the habit of memorizing rather than understanding these rules.  Students will work hard to memorize the rules without really comprehending {\it why} the rules are true, and therefore often fail to recall or apply the rules properly.

An illustrative technique I have found very useful is to have students create their own example circuits in which to test these rules.  Simple series and parallel circuits pose little challenge to construct, and therefore serve as excellent learning tools.  What could be better, or more authoritative, than learning principles of circuits from real experiments?  This is known as {\it primary research}, and it constitutes the foundation of scientific inquiry.  The greatest problem you will have as an instructor is encouraging your students to take the initiative to build these demonstration circuits on their own, because they are so used to having teachers simply {\it tell} them how things work.  This is a shame, and it reflects poorly on the state of modern education.

%INDEX% Parallel circuit; voltage, current, resistance, and power in

%(END_NOTES)


