
%(BEGIN_QUESTION)
% Copyright 2003, Tony R. Kuphaldt, released under the Creative Commons Attribution License (v 1.0)
% This means you may do almost anything with this work of mine, so long as you give me proper credit

Design an experiment to calculate the size of a capacitor based on its response in a time-constant circuit.  Include in your design an equation that gives the value of the capacitor in farads, based on data obtained by running the experiment.

\underbar{file 00455}
%(END_QUESTION)





%(BEGIN_ANSWER)

I recommend the following circuit for testing the capacitor:

$$\epsfbox{00455x01.eps}$$

The equation is yours to develop -- I will not reveal it here.  However, this does not mean there is no way to verify the accuracy of your equation, once you write it.  Explain how it would be possible to prove the accuracy of your algebra, once you have written the equation.

%(END_ANSWER)





%(BEGIN_NOTES)

In developing equations, students often feel "abandoned" if the instructor does not provide an answer for them.  How will they ever know if their equation is correct?  If the phenomenon the equation seeks to predict is well-understood, though, this is not a problem.

%INDEX% Calculating capacitance from time constant measurements

%(END_NOTES)


