
%(BEGIN_QUESTION)
% Copyright 2003, Tony R. Kuphaldt, released under the Creative Commons Attribution License (v 1.0)
% This means you may do almost anything with this work of mine, so long as you give me proper credit

% Uncomment the following line if the question involves calculus at all:
\vbox{\hrule \hbox{\strut \vrule{} $\int f(x) \> dx$ \hskip 5pt {\sl Calculus alert!} \vrule} \hrule}

Plot the relationships between voltage and current for resistors of three different values (1 $\Omega$, 2 $\Omega$, and 3 $\Omega$), all on the same graph:

$$\epsfbox{00086x01.eps}$$

What pattern do you see represented by your three plots?  What relationship is there between the amount of resistance and the nature of the voltage/current function as it appears on the graph?

\vskip 10pt

{\bf Advanced question:} in calculus, the instantaneous rate-of-change of an ($x,y$) function is expressed through the use of the {\it derivative} notation: ${dy \over dx}$.  How would the derivative for each of these three plots be properly expressed using calculus notation?  Explain how the derivatives of these functions relate to real electrical quantities.

\underbar{file 00086}
%(END_QUESTION)





%(BEGIN_ANSWER)

The greater the resistance, the steeper the slope of the plotted line.

\vskip 10pt

{\bf Advanced answer:} the proper way to express the derivative of each of these plots is ${dv \over di}$.  The derivative of a linear function is a constant, and in each of these three cases that constant equals the resistor resistance in ohms.  So, we could say that for simple resistor circuits, the instantaneous rate-of-change for a voltage/current function {\it is} the resistance of the circuit.

%(END_ANSWER)





%(BEGIN_NOTES)

Students need to become comfortable with graphs, and creating their own simple graphs is an excellent way to develop this understanding.  A graphical representation of the Ohm's Law function allows students another "view" of the concept, allowing them to more easily understand more advanced concepts such as {\it negative} resistance.

If students have access to either a graphing calculator or computer software capable of drawing 2-dimensional graphs, encourage them to plot the functions using these technological resources.

I have found it a good habit to "sneak" mathematical concepts into physical science courses whenever possible.  For so many people, math is an abstract and confusing subject, which may be understood only in the context of real-life application.  The studies of electricity and electronics are rich in mathematical context, so exploit it whenever possible!  Your students will greatly benefit.

%INDEX% Ohm's Law
%INDEX% Graphing
%INDEX% Calculus, derivative

%(END_NOTES)


