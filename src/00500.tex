
%(BEGIN_QUESTION)
% Copyright 2003, Tony R. Kuphaldt, released under the Creative Commons Attribution License (v 1.0)
% This means you may do almost anything with this work of mine, so long as you give me proper credit

Suppose a power plant operator was about to bring this alternator on-line (connect it to the AC bus), and happened to notice that neither one of the synchronizing lights was lit at all.  Thinking this to be unusual, the operator calls you to determine if something is wrong with the system.  Describe what you would do to troubleshoot this system.

$$\epsfbox{00500x01.eps}$$

\underbar{file 00500}
%(END_QUESTION)





%(BEGIN_ANSWER)

Before you proceed with troubleshooting steps, first try to determine if there is anything wrong with this system at all.  Could it be that the operator is just overly cautious, or is their caution justified?

%(END_ANSWER)





%(BEGIN_NOTES)

There may be some students who suggest there is nothing wrong at all with the system.  Indeed, since dim (or dark) lights normally indicate synchronization, would not the presence of two dim lights indicate that perfect synchronization had already been achieved?  Discuss the likelihood of this scenario with your students, that two independent alternators could be maintaining perfect synchronization without being coupled together.

In regard to troubleshooting, this scenario has great potential for group discussion.  Despite there being a simple, single, probable condition that could cause this problem, there are several possible component failures that could have created the condition.  Different students will undoubtedly have different methods of approaching the problem.  Let each one share their views, and discuss together what would be the best approach.

%INDEX% Troubleshooting, alternator synchronization lights

%(END_NOTES)


