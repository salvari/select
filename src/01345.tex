
%(BEGIN_QUESTION)
% Copyright 2003, Tony R. Kuphaldt, released under the Creative Commons Attribution License (v 1.0)
% This means you may do almost anything with this work of mine, so long as you give me proper credit

In this circuit, an AND gate is used to give a toggle switch control over the blinking of an LED:

$$\epsfbox{01345x01.eps}$$

The "astable multivibrator" is nothing more than an oscillator that produces a square-wave signal at a low frequency, at standard TTL voltage levels (0 and +5 volts).

Plot the output waveform for the gate (i.e. the voltage signal to the LED), given the following input conditions:

$$\epsfbox{01345x02.eps}$$

Hint: it helps in your analysis of digital waveforms if you first write a truth table for the gate under consideration, for your reference.

\underbar{file 01345}
%(END_QUESTION)





%(BEGIN_ANSWER)

$$\epsfbox{01345x03.eps}$$

%(END_ANSWER)





%(BEGIN_NOTES)

Many students find the waveform analysis of digital circuits intimidating at first, until they understand that it is nothing more than a graphical representation of "0" and "1" logic states over time.  Ask your students to share their "tips" on how to relate waveforms to truth tables, and in particular how they answered this particular question.

Just for fun, you might want to ask your students to identify where in time the toggle switch is open, and where it is closed.  Some students may answer backwards to this question, if they haven't carefully considered how the toggle switch is connected in this circuit!

%(END_NOTES)


