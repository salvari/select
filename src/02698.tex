
%(BEGIN_QUESTION)
% Copyright 2005, Tony R. Kuphaldt, released under the Creative Commons Attribution License (v 1.0)
% This means you may do almost anything with this work of mine, so long as you give me proper credit

% Uncomment the following line if the question involves calculus at all:
\vbox{\hrule \hbox{\strut \vrule{} $\int f(x) \> dx$ \hskip 5pt {\sl Calculus alert!} \vrule} \hrule}

How much current ($I$) would have to be forced through the resistor in order to generate an output voltage of 5 volts?

$$\epsfbox{02698x01.eps}$$

At what {\it rate} would $V_{in}$ have to increase in order to cause this amount of current to go "through" the capacitor, and thereby cause 5 volts to appear at the $V_{out}$ terminal?  What does this tell us about the behavior of this circuit?

\underbar{file 02698}
%(END_QUESTION)





%(BEGIN_ANSWER)

$I$ = 2.273 mA \hskip 50pt $dV_{in} \over dt$ = 227.3 volts/second

\vskip 10pt

The fact that this circuit outputs a voltage proportional to the {\it rate of change} over time of the input voltage indicates that it is a {\it differentiator}.

%(END_ANSWER)





%(BEGIN_NOTES)

This question is a good review of capacitor theory (relating voltage and current with regard to a capacitor), as well as an introduction to how op-amp circuits can perform calculus functions.

%INDEX% Differentiator circuit, op-amp

%(END_NOTES)


