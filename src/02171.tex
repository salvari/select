
%(BEGIN_QUESTION)
% Copyright 2004, Tony R. Kuphaldt, released under the Creative Commons Attribution License (v 1.0)
% This means you may do almost anything with this work of mine, so long as you give me proper credit

Power is easy to calculate in DC circuits.  Take for example this DC light bulb circuit:

$$\epsfbox{02171x01.eps}$$

Calculate the power dissipation in this circuit, and describe the transfer of energy from source to load: where does the energy come from and where does it go to?

\underbar{file 02171}
%(END_QUESTION)





%(BEGIN_ANSWER)

$P$ = 264 Watts

\vskip 10pt

If the source is a chemical battery, energy comes from the chemical reactions occurring in the battery's electrolyte, becomes transfered to electrical form, and then converted to heat and light in the bulb, all at the rate of 264 Joules per second (J/s).

%(END_ANSWER)





%(BEGIN_NOTES)

Discuss with your students the one-way flow of energy in a circuit such as this.  Although electric current takes a circular path, the actual transfer of energy is one-way: from source to load.  This is very important to understand, as things become more complex when reactive (inductive and capacitive) components are considered.

%INDEX% Power dissipation, resistance 

%(END_NOTES)


