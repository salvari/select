
%(BEGIN_QUESTION)
% Copyright 2003, Tony R. Kuphaldt, released under the Creative Commons Attribution License (v 1.0)
% This means you may do almost anything with this work of mine, so long as you give me proper credit

In DC circuits, we have Ohm's Law to relate voltage, current, and resistance together:

$$E = I R$$

In AC circuits, we similarly need a formula to relate voltage, current, and {\it impedance} together.  Write three equations, one solving for each of these three variables: a set of Ohm's Law formulae for AC circuits.  Be prepared to show how you may use algebra to manipulate one of these equations into the other two forms.

\underbar{file 00590}
%(END_QUESTION)





%(BEGIN_ANSWER)

$$E = I Z$$

$$I = {E \over Z}$$

$$Z = {E \over I}$$

\vskip 10pt

If using phasor quantities (complex numbers) for voltage, current, and impedance, the proper way to write these equations is as follows:

$${\bf E} = {\bf IZ}$$

$${\bf I} = {{\bf E} \over {\bf Z}}$$

$${\bf Z} = {{\bf E} \over {\bf I}}$$

\vskip 10pt

Bold-faced type is a common way of denoting vector quantities in mathematics.

%(END_ANSWER)





%(BEGIN_NOTES)

Although the use of phasor quantities for voltage, current, and impedance in the AC form of Ohm's Law yields certain distinct advantages over scalar calculations, this does not mean one cannot use scalar quantities.  Often it is appropriate to express an AC voltage, current, or impedance as a simple scalar number.

%INDEX% Impedance, defined
%INDEX% Ohm's Law, for AC circuits (incorporating impedance, Z)

%(END_NOTES)


