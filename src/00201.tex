
%(BEGIN_QUESTION)
% Copyright 2003, Tony R. Kuphaldt, released under the Creative Commons Attribution License (v 1.0)
% This means you may do almost anything with this work of mine, so long as you give me proper credit

Two 470 $\mu$F capacitors connected in series are subjected to a total applied voltage that changes at a rate of 200 volts per second.  How much current will there be "through" these capacitors?  Hint: the total voltage is divided evenly between the two capacitors.

Now suppose that two 470 $\mu$F capacitors connected in parallel are subjected to the same total applied voltage (changing at a rate of 200 volts per second).  How much total current will there be "through" these capacitors?

\underbar{file 00201}
%(END_QUESTION)





%(BEGIN_ANSWER)

Series connection: 47 milliamps (mA) total.  Parallel connection: 188 milliamps (mA) total.  

\vskip 10pt

Follow-up question: what do these figures indicate about the nature of series-connected and parallel connected capacitors?  In other words, what single capacitor value is equivalent to two series-connected 470 $\mu$F capacitors, and what single capacitor value is equivalent to two parallel-connected 470 $\mu$F capacitors?

%(END_ANSWER)





%(BEGIN_NOTES)

If your students are having difficulty answering the follow-up question in the Answer, ask them to compare these current figures (47 mA and 188 mA) against the current that would go through just one of the 470 $\mu$F capacitors under the same condition (an applied voltage changing at a rate of 200 volts per second).

It is, of course, important that students know how series-connected and parallel connected capacitors behave.  However, this is typically a process of rote memorization for students rather than true understanding.  With this question, the goal is to have students come to a realization of capacitor connections based on their understanding of series and parallel voltages and currents.

%INDEX% Capacitance, voltage versus current in
%INDEX% Capacitors, current "through"

%(END_NOTES)


