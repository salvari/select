
%(BEGIN_QUESTION)
% Copyright 2003, Tony R. Kuphaldt, released under the Creative Commons Attribution License (v 1.0)
% This means you may do almost anything with this work of mine, so long as you give me proper credit

Cranes used to move scrap iron and steel use electrically powered magnets to hold the metal pieces, rather than a scoop or some other mechanical grasping device:

$$\epsfbox{00171x01.eps}$$

In this illustration of a crane, superimpose a drawing showing the electromagnet, electrical power supply and wiring necessary for this to work.  Also include a switch so the crane operator can turn the magnet on and off.  Also, draw an electrical schematic diagram of the same circuit, showing all components in the crane's magnet circuit.

\underbar{file 00171}
%(END_QUESTION)





%(BEGIN_ANSWER)

Here is my schematic diagram:

$$\epsfbox{00171x02.eps}$$

I will leave it to you to draw the illustration of this circuit on the crane.  Your answer should show a wire {\it coil} embedded in the electromagnet assembly, a switch symbol near the operator, a battery symbol for the power supply, and wires carrying current to and from the electromagnet coil.

%(END_ANSWER)





%(BEGIN_NOTES)

The main purpose of this question is to have the students relate the principles of electric circuits and electromagnetism to a real-life application, and to show how the wire paths in the crane do not resemble the neat, clean layout of the schematic diagram.

%INDEX% Electromagnet

%(END_NOTES)


