
\centerline{\bf Design Project: audio media-based signal generator} \bigskip 
 
This worksheet and all related files are licensed under the Creative Commons Attribution License, version 1.0.  To view a copy of this license, visit http://creativecommons.org/licenses/by/1.0/, or send a letter to Creative Commons, 559 Nathan Abbott Way, Stanford, California 94305, USA.  The terms and conditions of this license allow for free copying, distribution, and/or modification of all licensed works by the general public.

\bigskip 

\hrule

\vskip 10pt

Inexpensive audio media players (CD-audio, MP3, etc.) are capable of outputting high-quality sine wave signals throughout the audio frequency range, and therefore are capable of performing as part of a signal generator system useful in the testing of audio circuits.  All you need is a recording of several sine wave tones of different frequencies, and an amplifier circuit to boost the media player's output to a voltage level reasonable for experimental work (at least a couple of volts RMS).

The following schematic diagram shows a suggested amplifier circuit that will work well for this purpose:

% Sample schematic diagram here
$$\epsfbox{proj_asg.eps}$$

I have used power supply voltages of 12 volts (+V = +6 volts ; -V = -6 volts) and 24 volts (+V = +12 volts; -V = -12 volts), both with good success.  Of course, you are not restricted to using this exact design -- feel free to modify it!

\vskip 10pt

\noindent
Deadlines (set by instructor):

\medskip
\item{$\bullet$} Project design completed: 
\item{$\bullet$} Components purchased:
\item{$\bullet$} Working prototype:
\item{$\bullet$} Finished system:
\item{$\bullet$} Full documentation:
\medskip





