
%(BEGIN_QUESTION)
% Copyright 2005, Tony R. Kuphaldt, released under the Creative Commons Attribution License (v 1.0)
% This means you may do almost anything with this work of mine, so long as you give me proper credit

Many different equations used in the analysis of electric circuits may be graphed.  Take for instance Ohm's Law for a variable resistor connected to a 12 volt source:

$$\epsfbox{03060x01.eps}$$

Plot this graph, following Ohm's Law.

\underbar{file 03060}
%(END_QUESTION)





%(BEGIN_ANSWER)

$$\epsfbox{03060x02.eps}$$

%(END_ANSWER)





%(BEGIN_NOTES)

Ask your students to explain how they plotted the two functions.  Did they make a table of values first?  Did they draw dots on the paper and then connect those dots with a line?  Did anyone plot dots for the endpoints and then draw a straight line in between because they knew this was a linear function?

Many students are surprised that the plot is nonlinear, being that resistors are considered linear devices!

%INDEX% Algebra, graphing simple functions
%INDEX% Graphing simple functions, algebra

%(END_NOTES)


