
%(BEGIN_QUESTION)
% Copyright 2003, Tony R. Kuphaldt, released under the Creative Commons Attribution License (v 1.0)
% This means you may do almost anything with this work of mine, so long as you give me proper credit

Write a single equation relating the resistance, specific resistance, length, and cross-sectional area of an electrical conductor together.

\underbar{file 00159}
%(END_QUESTION)





%(BEGIN_ANSWER)

$$R = \rho {l \over A}$$

\noindent
Where,

$R =$ Resistance, measured along the conductor's length

$\rho =$ Specific resistance of the substance

$l =$ Length of the conductor

$A =$ Cross-sectional area of the conductor

\vskip 10pt

Follow-up question: algebraically manipulate this equation to solve for length ($l$) instead of solving for resistance ($R$) as shown.

%(END_ANSWER)





%(BEGIN_NOTES)

A beneficial exercise to do with your students is to analyze this equation (and in fact, any equation) {\it qualitatively} instead of just {\it quantitatively}.  Ask students what will happen to $R$ if $\rho$ increases, or if $l$ decreases, or if $A$ decreases.  Many students find this a more challenging problem than working with real numbers, because they cannot use their calculators to give them qualitative answers (unless they enter random numbers into the equation, then change one of those numbers and re-calculate -- but that is twice the work of solving the equation with one set of numbers, once!).

%INDEX% Specific resistance

%(END_NOTES)


