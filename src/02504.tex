
%(BEGIN_QUESTION)
% Copyright 2005, Tony R. Kuphaldt, released under the Creative Commons Attribution License (v 1.0)
% This means you may do almost anything with this work of mine, so long as you give me proper credit

A common class of operation used in radio-frequency (RF) amplifier circuits is {\it class-C}.  Explain what this means, contrasting it against the class-A and class-B operations common in audio-frequency amplifier circuits.

\underbar{file 02504}
%(END_QUESTION)





%(BEGIN_ANSWER)

Class-C amplification is where the active device (transistor, usually) is conducting substantially less than 50\% of the waveform period.

%(END_ANSWER)





%(BEGIN_NOTES)

A natural question that arises when students first hear of class-C operation is, "How does a class-C amplifier circuit reproduce the entire waveform, if the transistor is completely off most of the time?"  The answer to this lies in {\it resonance}, usually supplied in the form of a tank circuit, to maintain oscillations while the transistor is cut off.

%INDEX% Class-C amplifier operation, defined

%(END_NOTES)


