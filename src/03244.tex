
%(BEGIN_QUESTION)
% Copyright 2005, Tony R. Kuphaldt, released under the Creative Commons Attribution License (v 1.0)
% This means you may do almost anything with this work of mine, so long as you give me proper credit

\centerline{\bf Animation: addressing 16 $\times$ 8 bit ROM memory}

\vskip 10pt

{\it This question consists of a series of images (one per page) that form an animation.  Flip the pages with your fingers to view this animation (or click on the "next" button on your viewer) frame-by-frame.}

\vskip 10pt

The following animation shows how setting the input (address) switches in particular combinations selects individual memory "cells" inside the ROM, resulting in data stored within those cells to appear at the output (data) lines.  Questions to ponder:

\medskip
\goodbreak
\item{$\bullet$} What is the organization of this particular ROM chip?  (e.g. 256 $\times$ 4, 1k $\times$ 1, etc.)
\item{$\bullet$} What is the relationship between the hexadecimal numbers stored inside each cell and the alphabetical characters shown below the chip?  What code is being used to represent these characters?
\item{$\bullet$} What character is represented by the hexadecimal code 20?  (Hint: this code is used twice in the sequence shown.)
\medskip

\vfil \eject
$$\epsfbox{03244x01.eps}$$

\vfil \eject
$$\epsfbox{03244x02.eps}$$

\vfil \eject
$$\epsfbox{03244x03.eps}$$

\vfil \eject
$$\epsfbox{03244x04.eps}$$

\vfil \eject
$$\epsfbox{03244x05.eps}$$

\vfil \eject
$$\epsfbox{03244x06.eps}$$

\vfil \eject
$$\epsfbox{03244x07.eps}$$

\vfil \eject
$$\epsfbox{03244x08.eps}$$

\vfil \eject
$$\epsfbox{03244x09.eps}$$

\vfil \eject
$$\epsfbox{03244x10.eps}$$

\vfil \eject
$$\epsfbox{03244x11.eps}$$

\vfil \eject
$$\epsfbox{03244x12.eps}$$

\vfil \eject
$$\epsfbox{03244x13.eps}$$

\vfil \eject
$$\epsfbox{03244x14.eps}$$

\vfil \eject
$$\epsfbox{03244x15.eps}$$

\vfil \eject
$$\epsfbox{03244x16.eps}$$

\underbar{file 03244}

\vfil \eject

%(END_QUESTION)





%(BEGIN_ANSWER)

Nothing to note here!

%(END_ANSWER)





%(BEGIN_NOTES)

The purpose of this animation is to let students study the behavior of this switch circuit and reach their own conclusions.  Similar to experimentation in the lab, except that here all the data collection is done visually rather than through the use of test equipment, and the students are able to "see" things that are invisible in real life.

%INDEX% Animation, ROM memory addressing

%(END_NOTES)


