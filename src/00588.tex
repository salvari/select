
%(BEGIN_QUESTION)
% Copyright 2003, Tony R. Kuphaldt, released under the Creative Commons Attribution License (v 1.0)
% This means you may do almost anything with this work of mine, so long as you give me proper credit

It is often necessary to represent AC circuit quantities as complex numbers rather than as scalar numbers, because both magnitude {\it and} phase angle are necessary to consider in certain calculations.

When representing AC voltages and currents in polar form, the angle given refers to the phase shift between the given voltage or current, and a "reference" voltage or current at the same frequency somewhere else in the circuit.  So, a voltage of $3.5 \hbox{ V} \angle -45^o$ means a voltage of 3.5 volts magnitude, phase-shifted 45 degrees behind (lagging) the reference voltage (or current), which is defined to be at an angle of 0 degrees.

But what about {\it impedance} $(Z)$?  Does impedance have a phase angle, too, or is it a simple scalar number like resistance or reactance?  

\vskip 10pt

Calculate the amount of current that would go through a 100 mH inductor with 36 volts RMS applied to it at a frequency of 400 Hz.  Then, based on Ohm's Law for AC circuits and what you know of the phase relationship between voltage and current for an inductor, calculate the impedance of this inductor {\it in polar form}.  Does a definite angle emerge from this calculation for the inductor's impedance?  Explain why or why not.

\underbar{file 00588}
%(END_QUESTION)





%(BEGIN_ANSWER)

${\bf Z_L} =$ 251.33 $\Omega$ $\angle$ 90$^{o}$

%(END_ANSWER)





%(BEGIN_NOTES)

This is a challenging question, because it asks the student to defend the application of phase angles to a type of quantity that does not really possess a wave-shape like AC voltages and currents do.  Conceptually, this is difficult to grasp.  However, the answer is quite clear through the Ohm's Law calculation ($Z = {E \over I}$).

Although it is natural to assign a phase angle of 0$^{o}$ to the 36 volt supply, making it the reference waveform, this is not actually necessary.  Work through this calculation with your students, assuming different angles for the voltage in each instance.  You should find that the impedance computes to be the same exact quantity every time.

%INDEX% Impedance, of inductor

%(END_NOTES)


