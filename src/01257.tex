
%(BEGIN_QUESTION)
% Copyright 2003, Tony R. Kuphaldt, released under the Creative Commons Attribution License (v 1.0)
% This means you may do almost anything with this work of mine, so long as you give me proper credit

Based on an analysis of a typical TTL logic gate circuit (consult a datasheet for a TTL logic gate if you need an internal schematic diagram for a gate circuit), determine what logic state is "assumed" by a TTL gate input when left "floating" (disconnected).

What ramification does this have for us when choosing input devices for TTL logic gates?  If, for instance, we wished to use a single-pole, single-throw (SPST) switch as the input device for a TTL logic gate, what is the {\it best} way to connect such a device to a TTL input?  Should the switch connect the TTL input to $V_{CC}$ when closed, or should it connect the input to $V_{EE}$ when closed?  Why does it matter?  Explain your answer in detail.

\underbar{file 01257}
%(END_QUESTION)





%(BEGIN_ANSWER)

TTL input devices must be {\it current-sinking}: that is, they must {\it ground} the TTL gate input in one of their states.  I'll let you figure out why this is so, from the schematic diagrams of TTL logic gate circuits.

%(END_ANSWER)





%(BEGIN_NOTES)

For review, ask your students what the symbols $V_{CC}$ and $V_{EE}$ mean with reference to TTL circuits.

Proper TTL "etiquette" is vitally important for students to understand, if they are to successfully build digital circuits (especially when interfacing TTL with other types of logic!).

%INDEX% Floating input, TTL

%(END_NOTES)


