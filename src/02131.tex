
%(BEGIN_QUESTION)
% Copyright 2003, Tony R. Kuphaldt, released under the Creative Commons Attribution License (v 1.0)
% This means you may do almost anything with this work of mine, so long as you give me proper credit

$$\epsfbox{02131x01.eps}$$

\underbar{file 02131}
\vfil \eject
%(END_QUESTION)





%(BEGIN_ANSWER)

Use circuit simulation software to verify your predicted and measured parameter values.

%(END_ANSWER)





%(BEGIN_NOTES)

The {\it real} challenge in this assessment is for students to determine their transformers' "polarities" before connecting them to the AC voltage source!  For this, they should have access to a small battery and a DC voltmeter (at their desks).

You may use a Variac at the test bench to provide variable-voltage AC power for the students' transformer circuits.  I recommend specifying load resistance values low enough that the load current completely "swamps" the transformer's magnetization current.  This may mean using wire-wound power resistors instead of $1 \over 4$ watt carbon composition resistors.

Note that there may very well be a shock hazard associated with this circuit!  Be sure to take this into consideration when specifying load resistor values.  You may also want to use low supply voltage levels (turn the Variac {\it way} down).

An extension of this exercise is to incorporate troubleshooting questions.  Whether using this exercise as a performance assessment or simply as a concept-building lab, you might want to follow up your students' results by asking them to predict the consequences of certain circuit faults.

%INDEX% Assessment, performance-based ("Buck' or "Boost" auto-transformer configuration)

%(END_NOTES)


