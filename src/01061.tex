
%(BEGIN_QUESTION)
% Copyright 2003, Tony R. Kuphaldt, released under the Creative Commons Attribution License (v 1.0)
% This means you may do almost anything with this work of mine, so long as you give me proper credit

Explain what {\it leakage inductance} is, in a system of two or more mutually coupled inductors (such as a {\it transformer}).  In a transformer, is leakage inductance a good thing or a bad thing?

\underbar{file 01061}
%(END_QUESTION)





%(BEGIN_ANSWER)

"Leakage inductance" is inductance that is {\it not} mutual between coupled inductors.  It is caused by magnetic flux produced by one coil that does not "link" with turns of the other coil(s).

In power distribution transformers, leakage inductance is undesirable.  However, there are some applications where leakage inductance is a desirable attribute.  Step-up transformers used to power gas-discharge lights, for example, are purposely built to have significant amounts of leakage inductance.

%(END_ANSWER)





%(BEGIN_NOTES)

After discussing the nature of leakage inductance (what causes it, and how it manifests itself in a transformer circuit), ask your students to explain why we do not want to have leakage inductance in a power distribution transformer, and why we do want to have it in a gas-discharge lighting transformer.

%(END_NOTES)


