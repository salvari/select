
%(BEGIN_QUESTION)
% Copyright 2003, Tony R. Kuphaldt, released under the Creative Commons Attribution License (v 1.0)
% This means you may do almost anything with this work of mine, so long as you give me proper credit

{\it Shunt} resistors are often used as current-measuring devices, in that they are designed to drop very precise amounts of voltage as large electric currents pass through them.  By measuring the amount of voltage dropped by a shunt resistor, you will be able to determine the amount of current going through it:

$$\epsfbox{00371x01.eps}$$

Suppose that a shunt resistance is labeled with the following rating: {\bf 150 A , 50 mV}.  What is the resistance of this shunt, in ohms?  Express your answer in metric notation, scientific notation, and plain decimal notation.

\underbar{file 00371}
%(END_QUESTION)





%(BEGIN_ANSWER)

Metric notation: 333.3 $\mu \Omega$

Scientific notation: $3.333 \times 10^{-4}$ $\Omega$

Plain decimal notation: 0.0003333 $\Omega$

%(END_ANSWER)





%(BEGIN_NOTES)

Ask your students how they think a resistor could be made with such a low resistance (a tiny fraction of an ohm!).  What do they think a shunt resistor would look like in real life?  If you happen to have a shunt resistor available in your classroom, show it to your students {\it after} they express their opinions on its construction.

%INDEX% Shunt resistor
%INDEX% Resistor, shunt
%INDEX% Current measurement with shunt

%(END_NOTES)


