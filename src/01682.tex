
%(BEGIN_QUESTION)
% Copyright 2003, Tony R. Kuphaldt, released under the Creative Commons Attribution License (v 1.0)
% This means you may do almost anything with this work of mine, so long as you give me proper credit

Shown here are two frequency response plots (known as {\it Bode plots}) for a pair of series resonant circuits with the same resonant frequency.  The "output" is voltage measured across the resistor of each circuit:

$$\epsfbox{01682x01.eps}$$

Determine which plot is associated with which circuit, and explain your answer.

\underbar{file 01682}
%(END_QUESTION)





%(BEGIN_ANSWER)

The steeper plot corresponds to the circuit with the greatest ${L \over C}$ ratio.

\vskip 10pt

Follow-up question: what kind of instrument(s) would you use to plot the response of a real resonant circuit in a lab environment?  Would an oscilloscope be helpful with this task?  Why or why not?

%(END_ANSWER)





%(BEGIN_NOTES)

Discuss with your students why the LC circuit with the greatest ${L \over C}$ ratio has the steeper response, in terms of reactances of the respective components at the resonant frequency.

The purpose of this question is to get students to realize that not all resonant circuits with identical resonant frequencies are alike!  Even with ideal components (no parasitic effects), the frequency response of a simple LC circuit varies with the particular choice of component values.  This is not obvious from inspection of the resonant frequency formula:

$$f_r = {1 \over {2 \pi \sqrt{LC}}}$$

%INDEX% L/C ratio, effect on frequency response of LC resonant circuit

%(END_NOTES)


