
%(BEGIN_QUESTION)
% Copyright 2003, Tony R. Kuphaldt, released under the Creative Commons Attribution License (v 1.0)
% This means you may do almost anything with this work of mine, so long as you give me proper credit

Draw the connecting wires on this terminal strip so that the three light bulbs are wired in parallel with each other and with the battery.

$$\epsfbox{01738x01.eps}$$

\underbar{file 01738}
%(END_QUESTION)





%(BEGIN_ANSWER)

$$\epsfbox{01738x02.eps}$$

%(END_ANSWER)





%(BEGIN_NOTES)

One of the more difficult visualization tasks for new students of electronics is translating schematic diagrams to physical layouts, and visa-versa.  This, sadly, is a skill that I don't see emphasized nearly enough in most basic electronics curricula.  It seems the majority of class time is spent mathematically analyzing useless resistor networks, and not enough time is invested building students' spatial relations skills.

While series connections are very easy to visualize on terminal strips, parallel connections are more difficult.  Work with your students through this question helping those who lack the innate spatial relations ability to see the solution quickly.

A "trick" I often use to help students build this skill is to have them first draw a nice, clean schematic diagram.  Then they over-trace each wire in the diagram as they draw it in the pictorial diagram.  In this way, they make sure not to overlook connections in the pictorial diagram.

%INDEX% Parallel connections, on terminal strip

%(END_NOTES)


