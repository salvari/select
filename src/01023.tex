
%(BEGIN_QUESTION)
% Copyright 2003, Tony R. Kuphaldt, released under the Creative Commons Attribution License (v 1.0)
% This means you may do almost anything with this work of mine, so long as you give me proper credit

Design an op-amp circuit that {\it divides} one quantity ($x$) by another quantity ($y$) using logarithms.  To give you a start on this circuit, I'll provide the initial logarithmic op-amp modules in this diagram:

\vskip 10pt

\epsfbox{01023x01.eps}

\vskip 10pt

Note: it will be helpful for your analysis to write the mathematical expression at each op-amp output in your circuit, so you may readily see how the overall math function is constructed from individual steps.

\underbar{file 01023}
%(END_QUESTION)





%(BEGIN_ANSWER)

$$\epsfbox{01023x02.eps}$$

%(END_ANSWER)





%(BEGIN_NOTES)

The circuit shown in the answer is a very common logarithmic construction: a {\it log-ratio} circuit, useful for many operations other than simple division.  This question challenges students to put together the logarithm, antilogarithm, and differential op-amp circuits in a way that achieves the final design goal.  Perhaps the most challenging aspect of this problem is managing the sign reversals.

%INDEX% Exponentiator, nonlinear opamp circuit
%INDEX% Exponents and logarithms, as inverse functions (in a real nonlinear circuit)
%INDEX% Logarithm extractor, nonlinear opamp circuit
%INDEX% Logarithms and exponents, as inverse functions (in a real nonlinear circuit)

%(END_NOTES)


