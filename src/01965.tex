
%(BEGIN_QUESTION)
% Copyright 2003, Tony R. Kuphaldt, released under the Creative Commons Attribution License (v 1.0)
% This means you may do almost anything with this work of mine, so long as you give me proper credit

$$\epsfbox{01965x01.eps}$$

\underbar{file 01965}
\vfil \eject
%(END_QUESTION)





%(BEGIN_ANSWER)

Use circuit simulation software to verify your predicted and measured parameter values.

%(END_ANSWER)





%(BEGIN_NOTES)

I have had success with the following values:

\medskip
\item{$\bullet$} $V_{CC}$ = 12 to 24 volts
\item{$\bullet$} $C_1$ = 0.47 $\mu$F
\item{$\bullet$} $C_2$ = 0.47 $\mu$F
\item{$\bullet$} $T_1$ = 1000:8 $\Omega$ audio matching transformer (used as center-tap inductor)
\item{$\bullet$} $R_1$ = 1.5 M$\Omega$
\item{$\bullet$} $Q_1$ = part number 2N3403
\medskip

Capacitors $C_1$ and $C_2$ need not be equal value, since they serve entirely different purposes: $C_1$ is the tank circuit capacitance, while $C_2$ is merely a coupling capacitor.  I just happened to be blessed with an abundance of 0.47 $\mu$F capacitors when I prototyped this circuit, so I chose that value for both capacitors!

With these component values, the output waveform I measured was not very sinusoidal, but at least it was oscillating.  The harmonic output of a Hartley oscillator is substantially greater than a Colpitts, primary because the two capacitors in the Colpitts design act as decoupling capacitances, shunting high-order harmonic signals to ground.  

Of course, in order to predict the frequency of oscillation in this Hartley oscillator circuit, you must know the inductance of the audio transformer's primary winding!

You might want to quiz your students on the purpose of resistor $R_1$, since it usually only has to be present at power-up to initiate oscillation!

An extension of this exercise is to incorporate troubleshooting questions.  Whether using this exercise as a performance assessment or simply as a concept-building lab, you might want to follow up your students' results by asking them to predict the consequences of certain circuit faults.

%INDEX% Assessment, performance-based (Hartley oscillator, series-fed BJT)

%(END_NOTES)


