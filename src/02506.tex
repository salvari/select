
%(BEGIN_QUESTION)
% Copyright 2005, Tony R. Kuphaldt, released under the Creative Commons Attribution License (v 1.0)
% This means you may do almost anything with this work of mine, so long as you give me proper credit

Shown here is a schematic diagram for a class-C RF (radio frequency) amplifier circuit:

$$\epsfbox{02506x01.eps}$$

This circuit will look very strange if you are accustomed to analyzing audio-frequency and DC amplifier circuits.  Note some of the distinct differences between this amplifier and an amplifier used to boost audio signals.  Also, explain what "class-C" operation means, and how this amplifier is able to output a continuous sine wave despite the transistor's behavior in class-C mode.

\vskip 10pt

Finally, write an equation that predicts this amplifier's operating frequency, based on certain component values which you identify.

\underbar{file 02506}
%(END_QUESTION)





%(BEGIN_ANSWER)

Operating in class-C mode, the transistor only turns on for a very brief moment in time during the waveform cycle.  The "flywheel" action of the tank circuit maintains a sinusoidal output waveform during the time the transistor is off.

$$f = {1 \over {2 \pi \sqrt{L_1 C_2}}}$$

Follow-up question: why is this mode of amplifier operation -- where the transistor is off most of the time and a tank circuit sustains sinusoidal oscillations -- desirable for an amplifier circuit?  Could this technique be applied to audio-frequency amplifier circuits?  Why or why not?

%(END_ANSWER)





%(BEGIN_NOTES)

Perhaps the most noteworthy detail of this circuit is the positive biasing voltage, despite the transistor being PNP and $V_CC$ being negative.  Ask your students to explain why this is necessary to get the transistor operating in class-C mode.

%INDEX% Class-C amplifier operation, defined

%(END_NOTES)


