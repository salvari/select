
%(BEGIN_QUESTION)
% Copyright 2003, Tony R. Kuphaldt, released under the Creative Commons Attribution License (v 1.0)
% This means you may do almost anything with this work of mine, so long as you give me proper credit

Many switched capacitor circuits require {\it non-overlapping}, two-phase clock signals.  Shown here is a schematic diagram of a gate circuit that converts a single clock signal into two complementary, non-overlapping clock signals:

$$\epsfbox{01457x01.eps}$$

Assume that the only gates possessing propagation delays are the double-NOT gates (inside the dotted boxes), whose sole purpose it is to provide a short time delay in the feedback signals.  Draw a timing diagram showing the two-phase clock signals ($\phi_1$ and $\phi_2$) in relation to the input clock waveform, and be prepared to explain how and why this circuit works:

$$\epsfbox{01457x02.eps}$$

\underbar{file 01457}
%(END_QUESTION)





%(BEGIN_ANSWER)

$$\epsfbox{01457x03.eps}$$

%(END_ANSWER)





%(BEGIN_NOTES)

Although it will be obvious to many of your students, you might want to ask ``why do we call the two clock signals {\it non-overlapping}?''  What, exactly, would an {\it overlapping} set of clock signals look like, in comparison?

%INDEX% Non-overlapping clock circuit
%INDEX% Clock circuit, non-overlapping
%INDEX% Two-phase clock circuit, non-overlapping

%(END_NOTES)


