
%(BEGIN_QUESTION)
% Copyright 2003, Tony R. Kuphaldt, released under the Creative Commons Attribution License (v 1.0)
% This means you may do almost anything with this work of mine, so long as you give me proper credit

The following circuit is known as a {\it Johnson counter}:

$$\epsfbox{01475x01.eps}$$

Describe the output of this circuit, as measured from the $Q$ output of the far right flip-flop, assuming that all flip-flops power up in the reset condition.

\vskip 10pt

Also, explain what this modified version of the above Johnson counter circuit will do, in each of the five selector switch positions:

$$\epsfbox{01475x02.eps}$$

\underbar{file 01475}
%(END_QUESTION)





%(BEGIN_ANSWER)

Johnson counters provide a divide-by-$n$ frequency reduction.  The second counter circuit shown has the ability to select different values for $n$.

%(END_ANSWER)





%(BEGIN_NOTES)

Strictly speaking, this circuit is a {\it divide-by-}$2n$ counter, because the frequency division ratio is equal to twice the number of flip-flops.

The final (\#5) switch position is interesting, and should be discussed among you and your students.

%INDEX% Counter circuit, Johnson
%INDEX% Johnson counter circuit

%(END_NOTES)


