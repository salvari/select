
%(BEGIN_QUESTION)
% Copyright 2003, Tony R. Kuphaldt, released under the Creative Commons Attribution License (v 1.0)
% This means you may do almost anything with this work of mine, so long as you give me proper credit

Write the Boolean expression for this relay logic circuit, then reduce that expression to its simplest form using any applicable Boolean laws and theorems.  Finally, draw a new relay circuit based on the simplified Boolean expression that performs the exact same logic function.

$$\epsfbox{01316x01.eps}$$

\underbar{file 01316}
%(END_QUESTION)





%(BEGIN_ANSWER)

Original Boolean expression: $\overline{\overline{AB} + C}$

\vskip 10pt

Reduced circuit (no relays needed!):

$$\epsfbox{01316x02.eps}$$

%(END_ANSWER)





%(BEGIN_NOTES)

Ask your students to explain what advantages there may be to using the simplified relay circuit rather than the original (more complex) relay circuit shown in the question.  What significance does this lend to learning Boolean algebra?

This is what Boolean algebra is really for: reducing the complexity of logic circuits.  It is far too easy for students to lose sight of this fact, learning all the abstract rules and laws of Boolean algebra.  Remember, in teaching Boolean algebra, you are supposed to be preparing students to perform manipulations of {\it electronic circuits}, not just equations.

%INDEX% Boolean algebra, relay circuit simplification
%INDEX% Boolean algebra, conversion of expression into relay logic

%(END_NOTES)


