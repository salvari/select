
%(BEGIN_QUESTION)
% Copyright 2003, Tony R. Kuphaldt, released under the Creative Commons Attribution License (v 1.0)
% This means you may do almost anything with this work of mine, so long as you give me proper credit

If a transistor is subjected to several different base currents, and the collector-emitter voltage ($V_{CE}$) "swept" through the full range for each of these base current values, data for an entire "family" of characteristic curves may be obtained and graphed:

$$\epsfbox{00941x01.eps}$$

What do these characteristic curves indicate about the base current's control over collector current?  How are the two currents related?

\underbar{file 00941}
%(END_QUESTION)





%(BEGIN_ANSWER)

The collector current is (for the most part) directly proportional to base current while in the "active" region.

%(END_ANSWER)





%(BEGIN_NOTES)

Ask your students what the characteristic curves would look like for a {\it perfect} transistor: one that was a perfect regulator of collector current over the full range of collector-emitter voltage.

%INDEX% Characteristic curve, BJT

%(END_NOTES)


