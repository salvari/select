
%(BEGIN_QUESTION)
% Copyright 2003, Tony R. Kuphaldt, released under the Creative Commons Attribution License (v 1.0)
% This means you may do almost anything with this work of mine, so long as you give me proper credit

Find a printed circuit board with soldered components on it, and bring it with you to class for discussion.  If you have access to a scrap bin from an electronics or computer repair shop, this would be an excellent place to find a printed circuit board.  On your PCB, try to identify:

\medskip
\item{$\bullet$} Number of layers on board
\item{$\bullet$} "Traces"
\item{$\bullet$} "Lands"
\item{$\bullet$} Cold solder joints (if any)
\item{$\bullet$} Solder bridges (if any)
\medskip

\underbar{file 01155}
%(END_QUESTION)





%(BEGIN_ANSWER)

Nothing to answer here!

\vskip 10pt

Follow-up question: be sure you know what each of these terms ("trace", "cold solder joint", "solder bridge", etc.) means.

%(END_ANSWER)





%(BEGIN_NOTES)

The purpose of this question is to get students to kinesthetically interact with the subject matter.  It may seem silly to have students engage in a "show and tell" exercise, but I have found that activities such as this greatly help some students.  For those learners who are kinesthetic in nature, it is a great help to actually {\it touch} real components while they're learning about their function.

%INDEX% Printed circuit board
%INDEX% Trace, printed circuit board

%(END_NOTES)


