
%(BEGIN_QUESTION)
% Copyright 2004, Tony R. Kuphaldt, released under the Creative Commons Attribution License (v 1.0)
% This means you may do almost anything with this work of mine, so long as you give me proper credit

Determine the frequency spectrum for a high-frequency, sine wave "carrier" signal that is {\it amplitude-modulated} (AM) by an audio-frequency sine wave signal, as the following block diagram shows:

$$\epsfbox{02271x04.eps}$$

The spectra for these respective waveforms are shown individually:

$$\epsfbox{02271x01.eps}$$

$$\epsfbox{02271x02.eps}$$

Plot the modulated signal spectrum here:

$$\epsfbox{02271x05.eps}$$

\underbar{file 02271}
%(END_QUESTION)





%(BEGIN_ANSWER)

$$\epsfbox{02271x03.eps}$$

\vskip 10pt

Follow-up question: if the modulating (audio) signal were increased in frequency, what would the sideband spectra do?

%(END_ANSWER)





%(BEGIN_NOTES)

The purpose of this question is to get students to recognize where sidebands come from, and how they relate to the frequency spectrum of the amplitude-modulated carrier wave.

In case anyone happens to ask, the symmetrical positioning of the sidebands around the carrier on the answer spectrum implies a {\it linear} frequency scale.

%INDEX% Amplitude Modulation (AM)
%INDEX% Sideband, AM frequency spectrum

%(END_NOTES)


