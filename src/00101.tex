
%(BEGIN_QUESTION)
% Copyright 2003, Tony R. Kuphaldt, released under the Creative Commons Attribution License (v 1.0)
% This means you may do almost anything with this work of mine, so long as you give me proper credit

Explain how an ohmmeter is able to measure the resistance of a component (in this case, a light bulb) when there is no battery or other source of power connected to it:

$$\epsfbox{00101x01.eps}$$

Also, identify the reading you would expect the ohmmeter to indicate if the light bulb were burnt out (failed "open").

\underbar{file 00101}
%(END_QUESTION)





%(BEGIN_ANSWER)

Ohmmeters always contain a battery or some other internal source of electrical power so that the component under test may be supplied with a small amount of current, in order to measure how hard it is for current to go through.

\vskip 10pt

If the light bulb were burnt open, the ohmmeter would register an extremely large (infinite) amount of resistance.

%(END_ANSWER)





%(BEGIN_NOTES)

Unlike voltmeters or ammeters, ohmmeters {\it must} contain their own power sources.  An implication of this fact is that ohmmeters must never be used to measure the resistance of an energized component.  Discuss this important caveat with your students, being sure to ask them to explain why connecting an ohmmeter to an energized component might give erroneous measurements (if it doesn't destroy the meter first!).

In regard to the ohmmeter reading for an open bulb, I have found that many math-weak students have a difficult time grasping the differentiating zero from infinity.  They recognize both as being extreme conditions (nothing versus everything), but many make the mistake of regarding "infinity" as identical to zero.  Quite to the contrary, "infinity" means {\it bigger than big, and huger than huge}.  Do not be surprised if one or more of your students harbor this same misconception, and be ready to address it!

%INDEX% Ohmmeter usage
%INDEX% Ohmmeter function

%(END_NOTES)


