
%(BEGIN_QUESTION)
% Copyright 2003, Tony R. Kuphaldt, released under the Creative Commons Attribution License (v 1.0)
% This means you may do almost anything with this work of mine, so long as you give me proper credit

A bipolar junction transistor parameter similar to $\beta$ is "alpha," symbolized by the Greek letter $\alpha$.  It is defined as the ratio between collector current and emitter current:

$$\alpha = {I_C \over I_E}$$

Apply algebraic substitution to this formula so that alpha is defined as a {\it function} of beta: $\alpha = f(\beta)$.  In other words, substitute and manipulate this equation until you have alpha by itself on one side and no variable except beta on the other.

You may find the following equations helpful in your work:

$$\beta = {I_C \over I_B} \hbox{\hskip 50pt} I_E = I_C + I_B$$

\underbar{file 02064}
%(END_QUESTION)





%(BEGIN_ANSWER)

$$\alpha = {\beta \over {\beta + 1}}$$

\vskip 10pt

Follow-up question: what range of values might you expect for $\alpha$, with a typical transistor?

%(END_ANSWER)





%(BEGIN_NOTES)

This question is nothing more than an exercise in algebraic manipulation.

%INDEX% Algebra, manipulating equations
%INDEX% Algebra, substitution
%INDEX% Alpha, defined for a BJT

%(END_NOTES)


