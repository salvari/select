
%(BEGIN_QUESTION)
% Copyright 2003, Tony R. Kuphaldt, released under the Creative Commons Attribution License (v 1.0)
% This means you may do almost anything with this work of mine, so long as you give me proper credit

A quantity often useful in electric circuit analysis is {\it conductance}, defined as the reciprocal of resistance:

$$G = {1 \over R}$$

The unit of conductance is the {\it siemens}, symbolized by the capital letter "S".  Convert the following resistance values into conductance values, expressing your answers in both scientific and metric notations:

\vskip 10pt

$R = 5 \hbox{ k} \Omega$ ; $G = $

\vskip 10pt

$R = 47 \> \Omega$ ; $G = $

\vskip 10pt

$R = 500 \hbox{ M} \Omega$ ; $G = $

\vskip 10pt

$R = 18.2 \> \mu \Omega$ ; $G = $

\vskip 10pt

Now, algebraically manipulate the given equation to solve for $R$ in terms of $G$, then use this new equation to work "backwards" through above calculations to see if you arrive at the original values of $R$ starting with your previously calculated values of $G$.

\underbar{file 01712}
%(END_QUESTION)





%(BEGIN_ANSWER)

$R = 5 \hbox{ k} \Omega$ ; $G = 200 \> \mu \hbox{S} = 2 \times 10^{-4} \hbox{ S}$

\vskip 10pt

$R = 47 \> \Omega$ ; $G = 21 \hbox{ mS} = 2.1 \times 10^{-2} \hbox{ S}$

\vskip 10pt

$R = 500 \hbox{ M} \Omega$ ; $G = 2 \hbox{ nS} = 2 \times 10^{-9} \hbox{ S}$

\vskip 10pt

$R = 18.2 \> \mu \Omega$ ; $G = 55 \hbox{ kS} = 5.5 \times 10^{4} \hbox{ S}$

\vskip 20pt

Solving for $R$ in terms of $G$:

$$R = {1 \over G}$$

%(END_ANSWER)





%(BEGIN_NOTES)

Ask your students to show you exactly how they manipulated the equation to solve for $R$.  The last instruction given in the question -- working backwards through the five calculations to see if you get the original (given) figures in degrees Celsius -- is actually a very useful way for students to check their algebraic work.  Be sure to make note of this in class!

%INDEX% Algebra, manipulating equations
%INDEX% Conductance
%INDEX% Siemens, unit

%(END_NOTES)


