
%(BEGIN_QUESTION)
% Copyright 2004, Tony R. Kuphaldt, released under the Creative Commons Attribution License (v 1.0)
% This means you may do almost anything with this work of mine, so long as you give me proper credit

Explain each of the mathematical approximations for this typical common-emitter amplifier circuit (with the dynamic emitter resistance "swamped" by $R_E$):

$$\epsfbox{02242x01.eps}$$

$$A_V \approx {R_C \> || \> R_{load} \over r'_e + R_E}$$

$$Z_{in} \approx R_1 \> || \> R_2 \> || \> (\beta + 1)(r'_e + R_E)$$

$$Z_{out} \approx R_C$$

What does each term in each expression represent, and why do they relate to one another as shown?

\underbar{file 02242}
%(END_QUESTION)





%(BEGIN_ANSWER)

The answers I leave for you to figure out!

%(END_ANSWER)





%(BEGIN_NOTES)

The approximations for voltage gain, input impedance, and output impedance vary somewhat according to how precise the author(s) intended them to be.  What you see here may be simpler or more complex than what you find in your textbook(s).  The purpose of this question is to summarize gain and impedance calculations for this type of amplifier circuit, as well as to stimulate thought and discussion on the rationale for each.  If students simply try to memorize these equations, they will forget them soon afterward.  If they {\it understand why} each one is as it is from principles previously learned, both comprehension and retention will be much improved.

%INDEX% Approximations for common-emitter amplifier circuit
%INDEX% Common-emitter circuit, approximations for multiple parameters
%INDEX% Impedance, amplifier input
%INDEX% Impedance, amplifier output
%INDEX% Input impedance, amplifier
%INDEX% Output impedance, amplifier

%(END_NOTES)


