
%(BEGIN_QUESTION)
% Copyright 2003, Tony R. Kuphaldt, released under the Creative Commons Attribution License (v 1.0)
% This means you may do almost anything with this work of mine, so long as you give me proper credit

Suppose we have an alternator with two sets of windings, {\bf A} and {\bf B}:

$$\epsfbox{01886x01.eps}$$

Each pair of windings in each set is series-connected, so they act as just two separate windings:

$$\epsfbox{01886x02.eps}$$

If one end of each winding pair were connected together at a common ground point, and each winding pair output 70 volts RMS, how much voltage would be measured between the open winding pair ends?

$$\epsfbox{01886x03.eps}$$

\underbar{file 01886}
%(END_QUESTION)





%(BEGIN_ANSWER)

99 volts

\vskip 10pt

Hint: if you don't understand how this voltage value was calculated, plot the voltage output of the two windings as if they were shown on an oscilloscope.  The {\it phase relationship} between the two voltages is key to the solution.

\vskip 10pt

Follow-up question: draw a phasor diagram showing how the difference in potential (voltage) between the wire ends is equal to 99 volts, when each winding coil's voltage is 70 volts.

%(END_ANSWER)





%(BEGIN_NOTES)

This question is a good exercise of students' knowledge of phase shift, in a very practical context.

%INDEX% Alternator, two-phase

%(END_NOTES)


