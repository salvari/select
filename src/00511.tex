
%(BEGIN_QUESTION)
% Copyright 2003, Tony R. Kuphaldt, released under the Creative Commons Attribution License (v 1.0)
% This means you may do almost anything with this work of mine, so long as you give me proper credit

Precision wire-wound resistors are often made of a special metal alloy called {\it manganin}.  What is it about this alloy that makes it preferable for use in precision resistor construction?

\underbar{file 00511}
%(END_QUESTION)





%(BEGIN_ANSWER)

The $\alpha$ value of manganin alloy is nearly zero.

%(END_ANSWER)





%(BEGIN_NOTES)

Ask your students what a wire-wound resistor made of copper or iron wire might do, if subjected to changes in temperature.

An historical side-note: during World War II, allied forces made extensive use of {\it analog computers} for directing the firing of projectiles and the dropping of bombs.  Unlike digital computers, which perform mathematical operations using on/off signals and are thus immune to errors caused by slight changes in component value, electronic analog computers represent physical variables in the form of continuous voltages and currents, and depend on the precision of its constituent resistors to produce precise results.  I remember reading one of the pioneering engineers in that field describe great gains in accuracy being due mostly to improvements in resistor construction.  Without some crucial improvements in resistor accuracy and stability, analog computers of the war-time era would have suffered from substantial inaccuracies.  Of all things, the lowly {\it resistor} was an influential piece of the allied war effort!

%INDEX% Resistance, relation to conductor temperature

%(END_NOTES)


