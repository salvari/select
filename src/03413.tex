
%(BEGIN_QUESTION)
% Copyright 2005, Tony R. Kuphaldt, released under the Creative Commons Attribution License (v 1.0)
% This means you may do almost anything with this work of mine, so long as you give me proper credit

\centerline{\bf Temperature coefficient of resistance at 20$^{o}$C}

% No blank lines allowed between lines of an \halign structure!
% I use comments (%) instead, so that TeX doesn't choke.

$$\vbox{\offinterlineskip
\halign{\strut
\vrule \quad\hfil # \ \hfil & 
\vrule \quad\hfil # \ \hfil \vrule \cr
\noalign{\hrule}
%
Metal type & $\alpha$ per degree $C$ \cr
%
\noalign{\hrule}
%
Silver (pure annealed) & 0.004000 \cr
%
\noalign{\hrule}
%
Copper (pure annealed) & 0.004280 \cr
%
\noalign{\hrule}
%
Copper (annealed) & 0.004020 \cr
%
\noalign{\hrule}
%
Copper (hard-drawn) & 0.004020 \cr
%
\noalign{\hrule}
%
Gold (99.9 \% pure) & 0.003770 \cr
%
\noalign{\hrule}
%
Aluminum (99.5 \% pure) & 0.004230 \cr
%
\noalign{\hrule}
%
Zinc (very pure) & 0.004060 \cr
%
\noalign{\hrule}
%
Iron (approx. pure) & 0.006250 \cr
%
\noalign{\hrule}
%
Platinum (pure) & 0.003669 \cr
%
\noalign{\hrule}
%
Nickel & 0.006220 \cr
%
\noalign{\hrule}
%
Tin (pure) & 0.004400 \cr
%
\noalign{\hrule}
%
Steel (wire) & 0.004630 \cr
%
\noalign{\hrule}
} % End of \halign 
}$$ % End of \vbox

$$R_T = R_r [ 1 + \alpha (T - T_r)]$$

\underbar{file 03413}
%(END_QUESTION)





%(BEGIN_ANSWER)

(There is no answer to give, for this is not even a question!)

%(END_ANSWER)





%(BEGIN_NOTES)

This table "question" exists solely for incorporation into exams or worksheets where students need a temperature coefficient table for reference.

The data for this table was taken from table 1-97 of the {\it American Electrician's Handbook} (eleventh edition) by Terrell Croft and Wilford Summers.

%INDEX% Table, specific resistance of different metals
%INDEX% Table, wire gauge

%(END_NOTES)


