
%(BEGIN_QUESTION)
% Copyright 2005, Tony R. Kuphaldt, released under the Creative Commons Attribution License (v 1.0)
% This means you may do almost anything with this work of mine, so long as you give me proper credit

Research datasheets for the 74LS184 and 74LS185 integrated circuits, and then explain how read-only memory technology is used to perform the BCD/binary conversion functions.

\underbar{file 02991}
%(END_QUESTION)





%(BEGIN_ANSWER)

These integrated circuits are really just read-only memory chips programmed with {\it look-up tables} for converting BCD to binary (74LS184) and binary to BCD (74LS185).

%(END_ANSWER)





%(BEGIN_NOTES)

Discuss with your students why someone would choose to implement these functions in a look-up table instead of using combinational logic or a microprocessor/microcontroller.  What advantage(s) might be realized with the look-up table approach?

%INDEX% BCD to binary conversion
%INDEX% Binary to BCD conversion
%INDEX% Look-up table

%(END_NOTES)


