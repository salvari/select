
%(BEGIN_QUESTION)
% Copyright 2005, Tony R. Kuphaldt, released under the Creative Commons Attribution License (v 1.0)
% This means you may do almost anything with this work of mine, so long as you give me proper credit

Bipolar junction transistors are definitely unidirectional ("polarized") devices, able to handle electric currents in one particular direction through each terminal:

$$\epsfbox{02414x01.eps}$$

What about JFETs?  Are they polarized just like BJTs?  Explain your answer, complete with arrows showing proper directions of current through these two JFETs:

$$\epsfbox{02414x02.eps}$$

\underbar{file 02414}
%(END_QUESTION)





%(BEGIN_ANSWER)

$$\epsfbox{02414x03.eps}$$

\vskip 10pt

Follow-up question: explain why there is no "controlling" current (ideally) in a JFET.

%(END_ANSWER)





%(BEGIN_NOTES)

The fact that JFETs are {\it bilateral} devices rather than unilateral as bipolar junction transistors leads to some interesting applications whereby AC currents may be controlled!  JFETs are often used in this capacity as AC signal switches, allowing a DC voltage to control passage or blocking of a low-amplitude AC signal such as an audio or RF signal prior to power amplification.

%INDEX% BJT, versus JFET
%INDEX% JFET, versus BJT

%(END_NOTES)


