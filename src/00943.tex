
%(BEGIN_QUESTION)
% Copyright 2003, Tony R. Kuphaldt, released under the Creative Commons Attribution License (v 1.0)
% This means you may do almost anything with this work of mine, so long as you give me proper credit

Though the characteristic curves for a transistor are usually generated in a circuit where base current is constant and the collector-emitter voltage ($V_{CE}$) is varied, this is usually not how transistor amplifier circuits are constructed.  Typically, the base current varies with the input signal, and the collector power supply is a fixed-voltage source:

$$\epsfbox{00943x01.eps}$$

The presence of a load resistor in the circuit adds another dynamic to the circuit's behavior.  Explain what happens to the transistor's collector-emitter voltage ($V_{CE}$) as the collector current increases (dropping more battery voltage across the load resistor), and qualitatively plot this {\it load line} on the same type of graph used for plotting transistor curves:

$$\epsfbox{00943x02.eps}$$

\underbar{file 00943}
%(END_QUESTION)





%(BEGIN_ANSWER)

Because there are no numbers along the axes of this graph, the best you can do is plot the general slope of the line, from upper-left to lower-right:

$$\epsfbox{00943x03.eps}$$

%(END_ANSWER)





%(BEGIN_NOTES)

Ask your student why this plot is straight, and not curved like the transistor's characteristic function.

%INDEX% Load line, concept illustrated with transistor amplifier circuit

%(END_NOTES)


