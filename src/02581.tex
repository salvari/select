
%(BEGIN_QUESTION)
% Copyright 2005, Tony R. Kuphaldt, released under the Creative Commons Attribution License (v 1.0)
% This means you may do almost anything with this work of mine, so long as you give me proper credit

Read the following quotation, and then research the term {\bf microcontroller} to see what relevance it has to the quote:

\vskip 10pt {\narrower \noindent \baselineskip5pt

\noindent
{\it I went to my first computer conference at the New York Hilton about 20 years ago.  When somebody there predicted the market for microprocessors would eventually be in the millions, someone else said, ``Where are they all going to go? It's not like you need a computer in every doorknob!''}

\vskip 10pt

\noindent
{\it Years later, I went back to the same hotel.  I noticed the room keys had been replaced by electronic cards you slide into slots in the doors.}

\vskip 10pt

\noindent
{\it There was a computer in every doorknob.}

\vskip 10pt

\noindent
-- Danny Hillis

\par} \vskip 10pt

\underbar{file 02581}
%(END_QUESTION)





%(BEGIN_ANSWER)

I'll let you do your homework on this question!

%(END_ANSWER)





%(BEGIN_NOTES)

Not only is the quotation funny, but it is startling as well, especially to those of us who were born without any computers in our homes at all, much less multiple personal computers.

A point I wish to make in having students research the term "microcontroller" is to see that most of the computers in existence are not of the variety one ordinarily thinks of by the label "computer."  Those doorknob computers -- as well as engine control computers in automobiles, kitchen appliances, cellular telephones, biomedical implants, talking birthday cards, and other small devices -- are much smaller and much more specialized than the "general purpose" computers people use at their desks to write documents or surf the internet.  They are the silent, unseen side of the modern "computer revolution," and in many ways are more appropriate for beginning students of digital electronics to explore than their larger, general-purpose counterparts.

%INDEX% Microcontrollers, ubiquitous

%(END_NOTES)


