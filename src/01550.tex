
%(BEGIN_QUESTION)
% Copyright 2003, Tony R. Kuphaldt, released under the Creative Commons Attribution License (v 1.0)
% This means you may do almost anything with this work of mine, so long as you give me proper credit

Explain, step by step, how to calculate the amount of current ($I$) that will go through each resistor in this parallel circuit, and also the voltage ($V$) dropped by each resistor:

$$\epsfbox{01550x01.eps}$$

\underbar{file 01550}
%(END_QUESTION)





%(BEGIN_ANSWER)

\medskip
\item {} $I_{R1} = 12$ mA ; $V_{R1} = 12$ V
\item {} $I_{R2} = 5.45$ mA ; $V_{R2} = 12$ V
\item {} $I_{R3} = 25.5$ mA ; $V_{R3} = 12$ V
\medskip

\vskip 10pt

Follow-up question: trace the direction of current through all three resistors as well as the power supply (battery symbol).  Compare these directions with the polarity of their shared voltage.  Explain how the relationship between voltage polarity and current direction relates to each component's identity as either a {\it source} or a {\it load}.

%(END_ANSWER)





%(BEGIN_NOTES)

Students often just want to memorize a procedure for determining answers to questions like these.  Challenge your students to not only understand the procedure, but to also explain {\it why} it must be followed.

Something your students will come to realize in discussion is that there is more than one way to arrive at all the answers!  While some of the steps will be common to all calculation strategies, other steps (near the end) leave room for creativity.

%INDEX% Parallel circuit; calculations for voltage, current, and resistance

%(END_NOTES)


