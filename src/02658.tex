
%(BEGIN_QUESTION)
% Copyright 2005, Tony R. Kuphaldt, released under the Creative Commons Attribution License (v 1.0)
% This means you may do almost anything with this work of mine, so long as you give me proper credit

Identify as many active loads as you can in the following (simplified) schematic of an LM324 operational amplifier circuit:

$$\epsfbox{02658x01.eps}$$

\underbar{file 02658}
%(END_QUESTION)





%(BEGIN_ANSWER)

Of course, all the current sources are active loads, but there is one more at the lower-left corner of the schematic.  I'll let you figure out where it is!

%(END_ANSWER)





%(BEGIN_NOTES)

Even if students do not yet know what an "operational amplifier" circuit is, they should still be able to identify transistor stages, configurations, and active loads.  In this case, most of the active loads are obvious (as revealed by the current source symbols).

Don't be surprised if some of your students point out that the differential pair in this opamp circuit looks "upside-down" compared to what they've seen before for differential pair circuits.  Let them know that this is not really an issue, and that the differential pair works the same in this configuration.

%INDEX% Active load, used in BJT amplifier
%INDEX% Active loads, used in operational amplifier (internal) circuitry

%(END_NOTES)


