
%(BEGIN_QUESTION)
% Copyright 2003, Tony R. Kuphaldt, released under the Creative Commons Attribution License (v 1.0)
% This means you may do almost anything with this work of mine, so long as you give me proper credit

Electric fields, like all fields, have two fundamental measures: field {\it force} and field {\it flux}.  In a capacitor, which of these field quantities is directly related to voltage between the plates, and which is directly related to the amount of charge (in coulombs) stored?

Based on this relationship, which electric field quantity changes when a sheet of glass is inserted between these two metal plates, connected to a source of constant voltage?

$$\epsfbox{00191x01.eps}$$

\underbar{file 00191}
%(END_QUESTION)





%(BEGIN_ANSWER)

Field force is a direct function of applied voltage, and field flux is a direct function of stored charge. 

If a sheet of glass is inserted between two metal plates connected to a constant voltage source, the electric field force between the plates will remain unchanged, while the electric field flux will increase (and along with it, the amount of charge stored on the plates).

\vskip 10pt

Follow-up question: explain how the variable of {\it electric permittivity} is relevant to the described situation.

%(END_ANSWER)





%(BEGIN_NOTES)

The concept of a {\it field} is quite abstract.  Electric fields in particular are abstract because they cannot be tangibly perceived, at least not outside of dangerous voltage levels.  Magnetic fields, which everyone should be familiar with from playing with magnets, may serve as an illustration of fields in general, but it is very important for students of electricity and electronics to understand that electric and magnetic fields are two different entities, albeit closely related (by Maxwell's Laws).

%INDEX% Electric field
%INDEX% Field, electric
%INDEX% Field, force versus flux

%(END_NOTES)


