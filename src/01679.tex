
%(BEGIN_QUESTION)
% Copyright 2003, Tony R. Kuphaldt, released under the Creative Commons Attribution License (v 1.0)
% This means you may do almost anything with this work of mine, so long as you give me proper credit

$$\epsfbox{01679x01.eps}$$

\underbar{file 01679}
\vfil \eject
%(END_QUESTION)





%(BEGIN_ANSWER)

Use circuit simulation software to verify your predicted and measured parameter values.

%(END_ANSWER)





%(BEGIN_NOTES)

Lissajous figures are very informative if one understands how to interpret them.  The purpose of this exercise is for students to learn exactly how to do that, in the context of measuring frequency (ratios).  Please use sinusoidal AC sources (variable-frequency signal generators) for this exercise.

The instructor will determine the voltage of the two sources (set the same), and also will graphically specify the Lissajous figure by drawing the desired figure on the oscilloscope screen illustration.  The student's task is to figure out what frequency ratio will generate this pattern on the oscilloscope screen, and set the signal generators to that ratio (one set to 1 kHz, the other to the appropriate frequency).  This works best if the student can only see the signal generator controls and frequency counter displays, and only the instructor can see the oscilloscope display.

Here are some sample Lissajous figures to draw for your students to duplicate:

$$\epsfbox{01679x02.eps}$$

%INDEX% Assessment, performance-based (Measuring frequency ratio by Lissajous figures)

%(END_NOTES)


