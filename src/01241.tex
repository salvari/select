
%(BEGIN_QUESTION)
% Copyright 2003, Tony R. Kuphaldt, released under the Creative Commons Attribution License (v 1.0)
% This means you may do almost anything with this work of mine, so long as you give me proper credit

Explain how the {\it Morse Code} resembles ASCII in structure and purpose.

\underbar{file 01241}
%(END_QUESTION)





%(BEGIN_ANSWER)

Morse Code is digital, being composed of only two types of characters, just like ASCII.  Also, its purpose is to convey alphanumeric information, just like ASCII.

%(END_ANSWER)





%(BEGIN_NOTES)

An interesting point to bring up to students about Morse Code is that it is {\it self-compressing}.  Note how different Morse characters possess different "bit" lengths, whereas ASCII characters are all 7 bits each (or 8 bits for Extended ASCII).  This makes Morse a more efficient code than ASCII, from the perspective of bit economy!

Ask your students what ramifications this "self-compressing" aspect of Morse Code would have if we were to choose it over ASCII for sending alphanumeric characters over digital communications lines, or store alphanumeric characters in some form of digital memory media.

%INDEX% ASCII, compared with Morse code
%INDEX% Morse code, compared with ASCII

%(END_NOTES)


