
%(BEGIN_QUESTION)
% Copyright 2003, Tony R. Kuphaldt, released under the Creative Commons Attribution License (v 1.0)
% This means you may do almost anything with this work of mine, so long as you give me proper credit

A technician picks up a resistor with the following color bands:

\vskip 10pt

\noindent
Color code: {\bf Org, Wht, Blu, Gld}

\vskip 10pt

Having forgotten the resistor color code, and being too lazy to research the color code in a book, the technician decides to simply measure its resistance with an ohmmeter.  Holding one lead of the resistor and one test lead of the ohmmeter between the thumb and index finger of the left hand, and the other resistor lead and meter test lead between the thumb and index finger of the right hand (to keep each test lead of the meter in firm contact with the respective leads of the resistor), the technician obtains a resistance measurement of 1.5 M$\Omega$.

What is wrong with the technician's measurement?

\underbar{file 00309}
%(END_QUESTION)





%(BEGIN_ANSWER)

The resistance measurement is much too low (it should be closer to 39 M$\Omega$) because the technician's body resistance is in parallel with the resistor.

%(END_ANSWER)





%(BEGIN_NOTES)

This is a very common mistake made by beginning students of electronics: holding resistors improperly when measuring their resistance values.  Discuss with your students some alternative methods of maintaining resistor lead contact with the meter probes so that the person's body does not become part of the resistance being measured.

%INDEX% Ohmmeter usage
%INDEX% Color code, resistor

%(END_NOTES)


