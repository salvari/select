
%(BEGIN_QUESTION)
% Copyright 2003, Tony R. Kuphaldt, released under the Creative Commons Attribution License (v 1.0)
% This means you may do almost anything with this work of mine, so long as you give me proper credit

Shown here is a simplified representation of an electrical power plant and a house, with the source of electricity shown as a battery, and the only electrical "load" in the house being a single light bulb:

$$\epsfbox{00075x01.eps}$$

Why would anyone use two wires to conduct electricity from a power plant to a house, as shown, when they could simply use one wire and a pair of {\it ground} connections, like this?
 
$$\epsfbox{00075x02.eps}$$

\underbar{file 00075}
%(END_QUESTION)





%(BEGIN_ANSWER)

This is not a practical solution, even though it would only require half the number of wires to distribute electrical power from the power plant to each house!  The reason this is not practical is because the earth (dirt) is not a good enough conductor of electricity.  Wires made of metal conduct electricity far more efficiently, which results in more electrical power delivered to the end user.

%(END_ANSWER)





%(BEGIN_NOTES)

Discuss the fact that although the earth (dirt) is a poor conductor of electricity, it may still be able to conduct levels of current lethal to the human body!  The amount of current necessary to light up a household light bulb is typically far in excess of values lethal for the human body.

%INDEX% Ground
%INDEX% Conductivity
%INDEX% Circuit, simple

%(END_NOTES)


