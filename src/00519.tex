
%(BEGIN_QUESTION)
% Copyright 2003, Tony R. Kuphaldt, released under the Creative Commons Attribution License (v 1.0)
% This means you may do almost anything with this work of mine, so long as you give me proper credit

Identify some common "primary" cell types, as well as some common "secondary" cell types.  For each of these battery types, identify their standard output voltages.

\underbar{file 00519}
%(END_QUESTION)





%(BEGIN_ANSWER)

\noindent
Primary cell types and voltages:

\item{$\bullet$} Carbon-zinc: 1.5 volts
\item{$\bullet$} Alkaline: 1.5 volts
\item{$\bullet$} Lithium: 1.8 to 3.65 volts, depending on exact chemical makeup
\item{$\bullet$} Mercuric-oxide: 1.35 to 1.4 volts

\vskip 10pt

\noindent
Secondary cell types and voltages:

\item{$\bullet$} Lead-acid: 2.1 volts
\item{$\bullet$} Nickel-cadmium: 1.2 volts
\item{$\bullet$} Silver-zinc: 1.5 volts

\vskip 10pt

Follow-up question: what are some of the safety and environmental concerns regarding the chemical substances used in batteries, both primary and secondary types?

%(END_ANSWER)





%(BEGIN_NOTES)

Challenge your students to list a few cell types other than what is listed in the answer section.

\vskip 10pt

Note: data for typical cell output voltages taken from chapter 20 of the {\it Electronic Engineer's Reference Book}, 5th edition (edited by F. Mazda).

%INDEX% Secondary cell battery
%INDEX% Primary cell battery

%(END_NOTES)


