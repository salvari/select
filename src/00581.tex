
%(BEGIN_QUESTION)
% Copyright 2003, Tony R. Kuphaldt, released under the Creative Commons Attribution License (v 1.0)
% This means you may do almost anything with this work of mine, so long as you give me proper credit

Does a capacitor's opposition to alternating current increase or decrease as the frequency of that current increases?  Also, explain why we refer to this opposition of AC current in a capacitor as {\it reactance} instead of {\it resistance}.

\underbar{file 00581}
%(END_QUESTION)





%(BEGIN_ANSWER)

The opposition to AC current ("reactance") of a capacitor decreases as frequency increases.  We refer to this opposition as "reactance" rather than "resistance" because it is non-dissipative in nature.  In other words, reactance causes no power to leave the circuit.

%(END_ANSWER)





%(BEGIN_NOTES)

Ask your students to define the relationship between capacitor reactance and frequency as either "directly proportional" or "inversely proportional".  These are two phrases used often in science and engineering to describe whether one quantity increases or decreases as another quantity increases.  Your students definitely need to be familiar with both these phrases, and be able to interpret and use them in their technical discussions.  

Also, discuss the meaning of the word "non-dissipative" in this context.  How could we prove that the opposition to current expressed by a capacitor is non-dissipative?  What would be the ultimate test of this?

%INDEX% Capacitive reactance
%INDEX% Reactance, capacitive
%INDEX% Reactance (capacitive) versus resistance
%INDEX% Resistance versus reactance (capacitive)

%(END_NOTES)


