
%(BEGIN_QUESTION)
% Copyright 2005, Tony R. Kuphaldt, released under the Creative Commons Attribution License (v 1.0)
% This means you may do almost anything with this work of mine, so long as you give me proper credit

A sine-wave signal generator is connected to an amplifier and set to 1 kHz for a performance test, while a spectrum analysis is taken of both the input and the output signals:

$$\epsfbox{02627x01.eps}$$

$$\epsfbox{02627x02.eps}$$

Answer the following questions about this amplifier test:

\medskip
\item{$1.$} Ideally, what should the output spectrum look like?
\item{$2.$} What name do we give to the additional peaks seen in the output signal spectrum analysis?
\item{$3.$} What does the presence of these peaks indicate about the amplifier circuit?
\item{$4.$} Identify some electrical causes (within a typical amplifier circuit) that could account for these additional peaks.
\medskip

\underbar{file 02627}
%(END_QUESTION)





%(BEGIN_ANSWER)

\medskip
\goodbreak
\item{$1.$} Ideally, what should the output spectrum look like? {\it The output spectrum should be identical to the input spectrum, ideally.}
\item{$2.$} What name do we give to the additional peaks seen in the output signal spectrum analysis? {\it These are called harmonics.} 
\item{$3.$} What does the presence of these peaks indicate about the amplifier circuit? {\it The amplifier is distorting the signal to some extent.}
\item{$4.$} Identify some electrical causes (within a typical amplifier circuit) that could account for these additional peaks. {\it Improper biasing, poor circuit design (non-ideal component values), excessive loading.}
\medskip

%(END_ANSWER)





%(BEGIN_NOTES)

{\bf This question is intended for exams only and not worksheets!}.

%INDEX% Harmonics, generated by amplifier distortion

%(END_NOTES)


