
%(BEGIN_QUESTION)
% Copyright 2003, Tony R. Kuphaldt, released under the Creative Commons Attribution License (v 1.0)
% This means you may do almost anything with this work of mine, so long as you give me proper credit

The first amplifier circuit shown here is {\it direct-coupled}, while the second is {\it capacitively coupled}.  

$$\epsfbox{01119x01.eps}$$

$$\epsfbox{01119x02.eps}$$

Which of these two designs would be more suitable for use in a DC voltmeter circuit (amplifying a measured DC voltage)?  What applications would the {\it other} amplifier design be suited for? 

\underbar{file 01119}
%(END_QUESTION)





%(BEGIN_ANSWER)

The direct-coupled amplifier circuit's bandwidth extends down to 0 Hz, unlike the other amplifier.  This makes it suitable for DC signal amplification.  The capacitive-coupled amplifier circuit would be better suited for applications where AC signals are solely dealt with.

Follow-up question: in each of these amplifier circuits, identify the point at which the signal's phase becomes shifted by 180$^{o}$.  In other words, show where the voltage signal becomes inverted, and then inverted again, so that the output is in phase with the input.

%(END_ANSWER)





%(BEGIN_NOTES)

A good question to ask your students is, "What is {\it bandwidth}?"  It is important that your students understand the basic concept of "bandwidth", and what factors influence it in a circuit. 

Ask your students to suggest possible values (in microfarads) for the coupling capacitor in the second circuit, based on common resistor values (between 1 k$\Omega$ and 100 k$\Omega$), and a modest audio frequency range (1 kHz to 20 kHz).  No exact values are needed here, but it is important that they be able to make an approximate estimation of the necessary (minimum) capacitance, if for no other reason than to demonstrate their comprehension of the coupling capacitor's intended purpose.

%INDEX% Amplifier, interstage coupling

%(END_NOTES)


