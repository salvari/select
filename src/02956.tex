
%(BEGIN_QUESTION)
% Copyright 2005, Tony R. Kuphaldt, released under the Creative Commons Attribution License (v 1.0)
% This means you may do almost anything with this work of mine, so long as you give me proper credit

When counters are used as frequency dividers, they are often drawn as simple boxes with one input and one output each, like this:

$$\epsfbox{02956x01.eps}$$

Calculate the four output frequencies ($f_{out1}$ through $f_{out4}$) given an input frequency of 1.5 kHz:

\medskip
\goodbreak
\item{$\bullet$} $f_{out1}$ = 
\item{$\bullet$} $f_{out2}$ = 
\item{$\bullet$} $f_{out3}$ = 
\item{$\bullet$} $f_{out4}$ = 
\medskip

\underbar{file 02956}
%(END_QUESTION)





%(BEGIN_ANSWER)

\medskip
\goodbreak
\item{$\bullet$} $f_{out1}$ = 150 Hz
\item{$\bullet$} $f_{out2}$ = 25 Hz
\item{$\bullet$} $f_{out3}$ = 12.5 Hz
\item{$\bullet$} $f_{out4}$ = 2.5 Hz
\medskip

\vskip 10pt

Follow-up question: if the clock frequency for this divider circuit is {\it exactly} 1.5 kHz, is it possible for the divided frequencies to vary from what is predicted by the modulus values (150 Hz, 25 Hz, 12.5 Hz, and 2.5 Hz)?  Explain why or why not.

%(END_ANSWER)





%(BEGIN_NOTES)

The purpose of this question is to introduce students to the schematic convention of counter/dividers as simple boxes with "MOD" specified for each one, and to provide a bit of quantitative analysis (albeit very simple).

%INDEX% Frequency division, digital
%INDEX% Modulus, digital counter

%(END_NOTES)


