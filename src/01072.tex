
%(BEGIN_QUESTION)
% Copyright 2003, Tony R. Kuphaldt, released under the Creative Commons Attribution License (v 1.0)
% This means you may do almost anything with this work of mine, so long as you give me proper credit

Complex quantities may be expressed in either {\it rectangular} or {\it polar} form.  Mathematically, it does not matter which form of expression you use in your calculations.

However, one of these forms relates better to real-world measurements than the other.  Which of these mathematical forms (rectangular or polar) relates more naturally to measurements of voltage or current, taken with meters or other electrical instruments?  For instance, which form of AC voltage expression, polar or rectangular, best correlates to the total voltage measurement in the following circuit?

$$\epsfbox{01072x01.eps}$$

\underbar{file 01072}
%(END_QUESTION)





%(BEGIN_ANSWER)

Polar form relates much better to the voltmeter's display of 5 volts.

\vskip 10pt

Follow-up question: how would you represent the total voltage in this circuit in rectangular form, given the other two voltmeter readings?

%(END_ANSWER)





%(BEGIN_NOTES)

While rectangular notation is mathematically useful, it does not apply directly to measurements taken with real instruments.  Some students might suggest that the 3.000 volt reading and the 4.000 volt reading on the other two voltmeters represent the rectangular components (real and imaginary, respectively) of voltage, but this is a special case.  In cases where resistance and reactance are mixed (e.g. a practical inductor with winding resistance), the voltage magnitude will be neither the real nor the imaginary component, but rather the polar magnitude.

%INDEX% Polar notation versus rectangular notation

%(END_NOTES)


