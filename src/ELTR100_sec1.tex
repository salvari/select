
\centerline{\bf ELTR 100 (DC 1), section 1} \bigskip 
 
\vskip 10pt

\noindent
{\bf Recommended schedule}

\vskip 5pt

%%%%%%%%%%%%%%%
\hrule \vskip 5pt
\noindent
\underbar{Day 1}

\hskip 10pt Topics: {\it Introduction to trades and programs}
 
\hskip 10pt Questions: {\it (none)}
 
\hskip 10pt Lab Exercises: {\it Gather books, tools, and parts}

\vskip 10pt
%%%%%%%%%%%%%%%
\hrule \vskip 5pt
\noindent
\underbar{Day 2}

\hskip 10pt Topics: {\it Basic concepts of electricity, simple circuits, and voltmeter/ammeter usage}
 
\hskip 10pt Questions: {\it 1 through 20}
 
\hskip 10pt Lab Exercises: {\it Voltmeter usage (question 101) and Ammeter usage (question 102)}
 
%INSTRUCTOR \hskip 10pt {\bf MIT 8.02 video clip: Disk 1, Lecture 1; Van De Graff 25:45 to 29:30}

%INSTRUCTOR \hskip 10pt {\bf MIT 8.02 video clip: Disk 1, Lecture 1; Electroscope 42:08 to end}

%INSTRUCTOR \hskip 10pt {\bf Socratic Electronics animation: Simple switch circuit}

\vskip 10pt
%%%%%%%%%%%%%%%
\hrule \vskip 5pt
\noindent
\underbar{Day 3}

\hskip 10pt Topics: {\it Ohm's Law and electrical safety}
 
\hskip 10pt Questions: {\it 21 through 40}
 
\hskip 10pt Lab Exercise: {\it Circuit with switch (question 103)}
 
%INSTRUCTOR \hskip 10pt {\bf MIT 8.02 video clip: Disk 1, Lecture 5; Faraday cage 45:40 to end}

%INSTRUCTOR \hskip 10pt {\bf MIT 6.002 video clip: Disk 1, Lecture 1; V/I plots 29:20 to 35:12}

%INSTRUCTOR \hskip 10pt {\bf MIT 6.002 video clip: Disk 1, Lecture 1; Burning pickle 35:35 to 37:28}

\vskip 10pt
%%%%%%%%%%%%%%%
\hrule \vskip 5pt
\noindent
\underbar{Day 4}

\hskip 10pt Topics: {\it Ohm's Law, Joule's Law, scientific notation, and metric prefixes}
 
\hskip 10pt Questions: {\it 41 through 60}
 
\hskip 10pt Lab Exercise: {\it Ohm's Law (question 104)}
 
\vskip 10pt
%%%%%%%%%%%%%%%
\hrule \vskip 5pt
\noindent
\underbar{Day 5}

\hskip 10pt Topics: {\it Resistors, precision, the standard color code, and ohmmeter usage}
 
\hskip 10pt Questions: {\it 61 through 80}
 
\hskip 10pt Lab Exercise: {\it Ohmmeter usage (question 105)}
 
%INSTRUCTOR \hskip 10pt {\bf Demo: Show dissected potentiometer to students}

\vskip 10pt
%%%%%%%%%%%%%%%
\hrule \vskip 5pt
\noindent
\underbar{Day 6}

\hskip 10pt Topics: {\it Circuit connections, soldering technique, and solderless breadboards}
 
\hskip 10pt Questions: {\it 81 through 100}
 
\hskip 10pt Lab Exercise: {\it Ohm's Law (question 106)}
 
%INSTRUCTOR \hskip 10pt {\bf Demo: Show a breadboard next to a terminal block and a printed circuit board}

%INSTRUCTOR \hskip 10pt {\bf Socratic Electronics animation: Soldering a wire to a lug}

\vskip 10pt
%%%%%%%%%%%%%%%
\hrule \vskip 5pt
\noindent
\underbar{Day 7}

\hskip 10pt Exam 1: {\it includes Ohm's Law performance assessment}
 
\hskip 10pt Lab Exercises: {\it PCB soldering (question 107) and exploring solder-together kit}
 
\vskip 10pt
%%%%%%%%%%%%%%%
\hrule \vskip 5pt
\noindent
\underbar{Practice and challenge problems}

\hskip 10pt Questions: {\it 110 through the end of the worksheet}
 
\vskip 10pt
%%%%%%%%%%%%%%%
\hrule \vskip 5pt
\noindent
\underbar{Impending deadlines}

\hskip 10pt {\bf Troubleshooting assessment (simple lamp circuit) due at end of ELTR100, Section 3}
 
\hskip 10pt Question 108: Troubleshooting log
 
\hskip 10pt Question 109: Sample troubleshooting assessment grading criteria
 
\hskip 10pt {\bf Solder-together kit due at end of ELTR100, Section 3}

\vskip 10pt
%%%%%%%%%%%%%%%




\vfil \eject

\centerline{\bf ELTR 100 (DC 1), section 1} \bigskip 
 
\vskip 10pt

\noindent
{\bf Skill standards addressed by this course section}

\vskip 5pt

%%%%%%%%%%%%%%%
\hrule \vskip 10pt
\noindent
\underbar{EIA {\it Raising the Standard; Electronics Technician Skills for Today and Tomorrow}, June 1994}

\vskip 5pt

\medskip
\item{\bf A} {\bf Technical Skills -- General}
\item{\bf A.05} Demonstrate an understanding of acceptable soldering/desoldering techniques, including through-hole and surface-mount devices.  {\it Partially met -- no SMD soldering/desoldering techniques.}
\item{\bf A.06} Demonstrate an understanding of proper solderless connections.
\item{\bf A.10} Demonstrate an understanding of color codes and other component descriptors.
\item{\bf B} {\bf Technical Skills -- DC circuits}
\item{\bf B.01} Demonstrate an understanding of sources of electricity in DC circuits.
\item{\bf B.03} Demonstrate an understanding of the meaning of and relationships among and between voltage, current, resistance and power in DC circuits.
\item{\bf B.04} Demonstrate an understanding of measurement of resistance of conductors and insulators and the computation of conductance. 
\item{\bf B.24} Demonstrate an understanding of measurement of power in DC circuits.
\medskip

\vskip 5pt

\medskip
\item{\bf B} {\bf Basic and Practical Skills -- Communicating on the Job}
\item{\bf B.01} Use effective written and other communication skills.  {\it Met by group discussion and completion of labwork.}
\item{\bf B.03} Employ appropriate skills for gathering and retaining information.  {\it Met by research and preparation prior to group discussion.}
\item{\bf B.04} Interpret written, graphic, and oral instructions.  {\it Met by completion of labwork.}
\item{\bf B.06} Use language appropriate to the situation.  {\it Met by group discussion and in explaining completed labwork.}
\item{\bf B.07} Participate in meetings in a positive and constructive manner.  {\it Met by group discussion.}
\item{\bf B.08} Use job-related terminology.  {\it Met by group discussion and in explaining completed labwork.}
\item{\bf B.10} Document work projects, procedures, tests, and equipment failures.  {\it Met by project construction and/or troubleshooting assessments.}
\item{\bf C} {\bf Basic and Practical Skills -- Solving Problems and Critical Thinking}
\item{\bf C.01} Identify the problem.  {\it Met by research and preparation prior to group discussion.}
\item{\bf C.03} Identify available solutions and their impact including evaluating credibility of information, and locating information.  {\it Met by research and preparation prior to group discussion.}
\item{\bf C.07} Organize personal workloads.  {\it Met by daily labwork, preparatory research, and project management.}
\item{\bf C.08} Participate in brainstorming sessions to generate new ideas and solve problems.  {\it Met by group discussion.}
\item{\bf D} {\bf Basic and Practical Skills -- Reading}
\item{\bf D.01} Read and apply various sources of technical information (e.g. manufacturer literature, codes, and regulations).  {\it Met by research and preparation prior to group discussion.}
\item{\bf E} {\bf Basic and Practical Skills -- Proficiency in Mathematics}
\item{\bf E.01} Determine if a solution is reasonable.
\item{\bf E.02} Demonstrate ability to use a simple electronic calculator.
\item{\bf E.05} Solve problems and [sic] make applications involving integers, fractions, decimals, percentages, and ratios using order of operations.
\item{\bf E.06} Translate written and/or verbal statements into mathematical expressions.
\item{\bf E.09} Read scale on measurement device(s) and make interpolations where appropriate.  {\it Met by analog multimeter usage.}
\item{\bf E.12} Interpret and use tables, charts, maps, and/or graphs.
\item{\bf E.13} Identify patterns, note trends, and/or draw conclusions from tables, charts, maps, and/or graphs.
\item{\bf E.15} Simplify and solve algebraic expressions and formulas.
\item{\bf E.16} Select and use formulas appropriately.
\item{\bf E.17} Understand and use scientific notation.
\item{\bf F} {\bf Basic and Practical Skills -- Proficiency in Physics}
\item{\bf F.04} Understand principles of electricity including its relationship to the nature of matter.
\medskip

\vskip 10pt

%%%%%%%%%%%%%%%




\vfil \eject

\centerline{\bf ELTR 100 (DC 1), section 1} \bigskip 
 
\vskip 10pt

\noindent
{\bf Common areas of confusion for students}

\vskip 5pt

%%%%%%%%%%%%%%%
\hrule \vskip 5pt

\vskip 10pt

\noindent
{\bf Difficult concept: } {\it An interruption anywhere in a simple circuit stops current everywhere.}

A common misunderstanding is that the location of a break (interruption) in a simple circuit matters to the electrons moving around in it.  To the contrary, {\it any} break in a simple circuit halts the flow of electrons {\it everywhere} in it.  A good way to grasp this concept is by experimenting with a simple one-battery, one-lamp circuit, seeing the effect that breaks in the circuit have on the lamp's illumination.

\vskip 10pt

\noindent
{\bf \underbar{Very} difficult concept: } {\it Voltage is a relative quantity -- it only exists \underbar{between} two points.}

Unlike current, which may be measured at a single point in a circuit, voltage is fundamentally relative: it only exists as a {\it difference} between two points.  In other words, there is no such thing as voltage existing at a single location.  Therefore, while we speak of current going {\it through} a component in a circuit, we speak of voltage being {\it across} a component, measured between two different points on that component.  So confusing is this concept that a significant number of students continue to harbor conceptual errors about the nature of voltage for several months after having first learned about it.  A good way to understand voltage is to experiment with a voltmeter, measuring voltage between different pairs of points in safe, low-voltage circuits.  Another good way to gain proficiency is to practice on conceptual problems relating to the measurement of voltage in circuits.

\vskip 10pt

\noindent
{\bf Difficult concept: } {\it Algebraically manipulating Ohm's Law equations.}

It is common for students uncomfortable with algebra to resort to "cheat" techniques to figure out how to write the various equations for Ohm's and Joule's Law (e.g. drawing the letters in a pyramid shape and then covering up the one you want to solve for to see where the others are in relation to each other; or using a "cheat sheet" showing all algebraic combinations of V, I, R, and P).  The real problem is lack of math skill, and the only solution is to gain proficiency by {\it using} algebra to transpose variables in these equations.  Ohm's Law is the easiest equation you will see in electronics for practicing your algebra skills, so use this as an opportunity to learn!


