
%(BEGIN_QUESTION)
% Copyright 2005, Tony R. Kuphaldt, released under the Creative Commons Attribution License (v 1.0)
% This means you may do almost anything with this work of mine, so long as you give me proper credit

The similarities and differences between microcontroller (microprocessor) systems and programmable logic devices may be illuminated by analogy.  Read the following scenarios where two different solutions are employed to solve common problems.  For each scenario, determine which solution is analogous to a microcontroller and which solution is analogous to a programmable logic device:

\vskip 10pt {\narrower \noindent \baselineskip5pt

{\it A business manager must make a hiring decision: either hire several specialty-skilled employees to perform various tasks (one task per person), or hire a few broadly-skilled people who can be given new instructions and/or training to switch between different tasks as needed.}

\par} \vskip 10pt

\vskip 10pt {\narrower \noindent \baselineskip5pt

{\it Two tinkerers are modifying pianos to play short songs automatically (with no human operator).  The first decides to build a "tape reader" device similar to that of an old mechanical player-piano, where a paper scroll bearing punched holes "tells" the piano keys when to strike and in what order.  The second decides to build a much simpler sequencing mechanism, where each key on the piano from left to right is struck in sequence, the proper ordering of notes in the song being arranged by re-connecting the keys to different hammers inside the piano.}

\par} \vskip 10pt

\underbar{file 03045}
%(END_QUESTION)





%(BEGIN_ANSWER)

First scenario: broadly-skilled employees = microcontroller; specialty-skilled employees = programmable logic.

\vskip 10pt

Second scenario: tape reader = microcontroller; re-linking keys to hammers = programmable logic.

%(END_ANSWER)





%(BEGIN_NOTES)

Understanding the distinction between microcontrollers and programmable logic devices can be difficult, especially if one has limited experience with both (as most students do).  Questions such as this, which ask students to examine opposing analogies, teaches some of the distinguishing principles without becoming mired in technical detail.

The fundamental principle I want students to see from these analogies is that microcontrollers and microprocessors are re-programmed by changing a {\it sequence} of fixed operations, while programmable logic systems are re-programmed by changing {\it associations} between fixed elements.

%INDEX% Microcontroller versus programmable logic
%INDEX% Programmable logic versus microcontroller

%(END_NOTES)


