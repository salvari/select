
%(BEGIN_QUESTION)
% Copyright 2005, Tony R. Kuphaldt, released under the Creative Commons Attribution License (v 1.0)
% This means you may do almost anything with this work of mine, so long as you give me proper credit

The ripple frequency of a half-wave rectifier circuit powered by 60 Hz AC is measured to be 60 Hz.  The ripple frequency of a full-wave rectifier circuit powered by the exact same 60 Hz AC line voltage is measured to be 120 Hz.  Explain why the ripple frequency of the full-wave rectifier is twice that of the half-wave rectifier.

\underbar{file 02312}
%(END_QUESTION)





%(BEGIN_ANSWER)

There are double the number of pulses in the full-wave rectifier's output, meaning the wave-shape repeats itself twice as often.

%(END_ANSWER)





%(BEGIN_NOTES)

I have heard students concoct very interesting (and wrong) explanations for why the two frequencies differ.  One common misconception is that is has something to do with the transformer, as though a transformer had the ability to step {\it frequency} up and down just as easily as voltage or current!  If students are not understanding why there is a frequency difference, you might want to help them out by asking two students to come up to the front of the class and draw two waveforms: half-wave and full-wave, along with their original AC (unrectified) waveforms.

%INDEX% Ripple voltage, frequency of

%(END_NOTES)


