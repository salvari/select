
%(BEGIN_QUESTION)
% Copyright 2003, Tony R. Kuphaldt, released under the Creative Commons Attribution License (v 1.0)
% This means you may do almost anything with this work of mine, so long as you give me proper credit

One of the simplest transistor "logic" circuits used in computer circuitry is the so-called {\it inverter gate}.  Its logic diagram symbol is this:

$$\epsfbox{00503x02.eps}$$

Inverters are often bundled six to a "DIP" (Dual-Inline Package) module, where all gates share the same two power supply connections ("$V_{cc}$" and "Ground"), like this:

$$\epsfbox{00503x03.eps}$$

When constructed with bipolar junction transistors, the internal schematic diagram of a single inverter gate looks something like this:

$$\epsfbox{00503x01.eps}$$

Being a digital logic circuit, it only "understands" two states: on and off.  As you can see, a SPDT switch provides signal input into the logic gate, either full supply voltage ($V_{CC}$) or no voltage at all (Gnd).

Determine the voltage across the load resistance for each of these switch states.  Based on your analysis, what is the logical function of an "inverter" gate?

\underbar{file 00503}
%(END_QUESTION)





%(BEGIN_ANSWER)

An "inverter" gate simply reverses the logic state of the input.

%(END_ANSWER)





%(BEGIN_NOTES)

There is much to analyze in this circuit, and this question is not intended to be a full introduction to digital logic.  It does, however, provide a preview of things to come, as well as an excellent opportunity to analyze diode and transistor circuitry in the simplest possible way: either fully "on" or fully "off".

One tactic I have found useful as an instructor for analyzing complex circuits in front of a class is to project the image onto a whiteboard using a bright computer projector.  Then, you may use dry-erase markers on the whiteboard to "comment" the diagram with voltage and current notations as you proceed to analyze the circuit with your students.  Analyzing the circuit for a different logic state is easy: merely erase the "comments" made previously, and the schematic itself is unchanged, ready to be "marked up" again.

%INDEX% Inverter gate, internal schematic (TTL)
%INDEX% Transistor switch circuit (BJT)

%(END_NOTES)


