
%(BEGIN_QUESTION)
% Copyright 2003, Tony R. Kuphaldt, released under the Creative Commons Attribution License (v 1.0)
% This means you may do almost anything with this work of mine, so long as you give me proper credit

A student builds the following digital circuit on a solderless breadboard (a "proto-board"):

$$\epsfbox{01253x01.eps}$$

The DIP circuit is a hex inverter (it contains {\it six} "inverter" or "NOT" logic gates), but only one of these gates is being used in this circuit.  The student's intent was to build a logic circuit that energized the LED when the pushbutton switch was unactuated, and de-energized the LED when the switch was pressed: so that the LED indicates the reverse state of the switch itself.  The student builds this circuit, and it is found to function perfectly well.

Explain the purpose of the resistor on the input of the inverter.  What is it there for?  What might happen if it were to be removed from the circuit?

Also, explain why the inputs of all the unused inverter gates in this IC have been either connected to ground or to $V_{DD}$.  Is this necessary for the circuit to work properly, or is it just a precautionary measure?

\underbar{file 01253}
%(END_QUESTION)





%(BEGIN_ANSWER)

The resistor on the input side of the gate functions as a {\it pulldown}, to provide a solid "low" state to the gate's input when the switch contacts open.

Shorting all unused gate inputs to either ground or $V_{DD}$ is merely a precautionary measure.  It prevents unnecessary power draw from the supply, and possible IC overheating.

%(END_ANSWER)





%(BEGIN_NOTES)

Discuss the problem of "floating" or "high-Z" states with your students, especially in the context of CMOS.  What is it about the nature of a CMOS circuit that makes floating inputs especially troublesome?  Ask your students to contrast this against floating TTL inputs.

%INDEX% Floating input, CMOS
%INDEX% Logic probe, use of multimeter as 
%INDEX% Multimeter used as a logic probe

%(END_NOTES)


