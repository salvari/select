
%(BEGIN_QUESTION)
% Copyright 2005, Tony R. Kuphaldt, released under the Creative Commons Attribution License (v 1.0)
% This means you may do almost anything with this work of mine, so long as you give me proper credit

Compare the following two circuits, the first one being a digital adder and the second one being an analog summer:

$$\epsfbox{02849x01.eps}$$

$$\epsfbox{02849x02.eps}$$

These two circuits perform the same mathematical function, yet the manners in which they perform this function are quite different.  Compare and contrast the digital adder and the analog summer circuits shown here, citing any advantages or disadvantages of each.

\underbar{file 02849}
%(END_QUESTION)





%(BEGIN_ANSWER)

I won't directly give away answers here, but I will list a few criteria you might want to use for comparing and contrasting:

\medskip
\item{$\bullet$} Resolution
\item{$\bullet$} Accuracy
\item{$\bullet$} Speed
\item{$\bullet$} Cost
\medskip

%(END_ANSWER)





%(BEGIN_NOTES)

This question is not really specific to adder/summer circuits, as it may first appear.  The fundamental comparison being drawn in this question is between digital and analog.  This is an important concept for students to grasp, as both have their roles in modern electronics.  A common fallacy is that "digital is better" in all circumstances, but the truth is that both digital and analog have their respective strengths and limitations.

%INDEX% Analog versus digital computation
%INDEX% Digital versus Analog computation

%(END_NOTES)


