
%(BEGIN_QUESTION)
% Copyright 2005, Tony R. Kuphaldt, released under the Creative Commons Attribution License (v 1.0)
% This means you may do almost anything with this work of mine, so long as you give me proper credit

The following set of mathematical expressions is the {\it complete} set of "times tables" for the Boolean number system:

$$0 \times 0 = 0$$

$$0 \times 1 = 0$$

$$1 \times 0 = 0$$

$$1 \times 1 = 1$$

Now, nothing seems unusual at first about this table of expressions, since they appear to be the same as multiplication understood in our normal, everyday system of numbers.  However, what is unusual is that these four statements comprise the entire set of rules for Boolean multiplication!

Explain how this can be so, being that there is no statement saying $1 \times 2 = 2$ or $2 \times 3 = 6$.  Where are all the other numbers besides 0 and 1?

\underbar{file 02777}
%(END_QUESTION)





%(BEGIN_ANSWER)

Boolean quantities can only have one out of two possible values: either 0 or 1.  There is no such thing as "2" -- or any other digit besides 0 or 1 for that matter -- in the set of Boolean numbers!

%(END_ANSWER)





%(BEGIN_NOTES)

Some students with background in computers may ask if Boolean is the same as binary.  The answer to this very good question is "no."  Binary is simply a {\it numeration system} for expressing real numbers, while Boolean is a completely different number system (like integer numbers are to irrational numbers, for example).  It is possible to count arbitrarily high in binary, but you can only count as high as "1" in Boolean.

%INDEX% Boolean algebra, 0 and 1 as the only possible numbers
%INDEX% Boolean algebra, rules of multiplication

%(END_NOTES)


