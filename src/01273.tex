
%(BEGIN_QUESTION)
% Copyright 2003, Tony R. Kuphaldt, released under the Creative Commons Attribution License (v 1.0)
% This means you may do almost anything with this work of mine, so long as you give me proper credit

In CMOS circuitry, one side of the DC power supply is usually labeled as "$V_{DD}$", while the other side is labeled as "$V_{SS}$".  Why is this?  What do the subscripts "DD" and "SS" represent?

\underbar{file 01273}
%(END_QUESTION)





%(BEGIN_ANSWER)

The labels $V_{DD}$ and $V_{SS}$ are supposed to mean "power supply to {\it drain} and {\it source} sides of MOSFETs, respectively.  This nomenclature is actually a holdover from obsolete NMOS gate designs, which used N-channel MOSFETS exclusively.  Even though it doesn't make much sense in CMOS circuits (you'll see why if you examine the internal schematic diagram for a CMOS gate), it is the standard way of denoting power supply terminals for CMOS circuits.

\vskip 10pt

Follow-up question: what polarities do these respective labels represent?

%(END_ANSWER)





%(BEGIN_NOTES)

Ahhh, the vestiges of yesterday's technology!  What can I say?  Sometimes terms "stick" even when it makes little sense for them to.

%INDEX% Vdd, defined
%INDEX% Vss, defined

%(END_NOTES)


