
%(BEGIN_QUESTION)
% Copyright 2003, Tony R. Kuphaldt, released under the Creative Commons Attribution License (v 1.0)
% This means you may do almost anything with this work of mine, so long as you give me proper credit

$$\epsfbox{01937x01.eps}$$

\underbar{file 01937}
\vfil \eject
%(END_QUESTION)





%(BEGIN_ANSWER)

Use circuit simulation software to verify your predicted and measured parameter values.

%(END_ANSWER)





%(BEGIN_NOTES)

I've experienced good results using the following component values:

\medskip
\item{$\bullet$} $V_{CC}$ = 12 volts
\item{$\bullet$} $R_1 = $ 220 k$\Omega$
\item{$\bullet$} $R_2 = $ 27 k$\Omega$
\item{$\bullet$} $R_3 = $ 10 k$\Omega$
\item{$\bullet$} $R_4 = $ 1.5 k$\Omega$
\item{$\bullet$} $R_5 = $ 1 k$\Omega$
\item{$\bullet$} $C_1 = $ 0.47 $\mu$F
\item{$\bullet$} $C_2 = $ 4.7 $\mu$F
\item{$\bullet$} $C_3 = $ 33 $\mu$F
\item{$\bullet$} $C_4 = $ 47 $\mu$F
\item{$\bullet$} $Q_1 \hbox{ and } Q_2 = $ 2N3403
\medskip

Students have a lot of fun connecting long lengths of cable between the output stage and the speaker, and using this circuit to talk (one-way, simplex communication) between rooms.

One thing I've noticed some students misunderstand in their study of electronic amplifier circuits is their practical purpose.  So many textbooks emphasize abstract analysis with sinusoidal voltage sources and resistive loads that some of the real applications of amplifiers may be overlooked by some students.  One student of mine in particular, when building this circuit, kept asking me, "so where does the signal generator connect to this amplifier?"  He was so used to seeing signal generators connected to amplifier inputs in his textbook (and lab manual!) that he never realized you could use an amplifier circuit to amplify a {\it real}, practical audio signal!!!  An extreme example, perhaps, but real nevertheless, and illustrative of the need for practical application in labwork.

In order for students to measure the voltage gain of this amplifier, they must apply a steady, sinusoidal signal to the input.  The microphone and speaker are indeed practical, but the signals produced in such a circuit are too chaotic for students to measure with simple test equipment.

An extension of this exercise is to incorporate troubleshooting questions.  Whether using this exercise as a performance assessment or simply as a concept-building lab, you might want to follow up your students' results by asking them to predict the consequences of certain circuit faults.

%INDEX% Assessment, performance-based (Audio intercom circuit)

%(END_NOTES)


