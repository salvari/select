
%(BEGIN_QUESTION)
% Copyright 2003, Tony R. Kuphaldt, released under the Creative Commons Attribution License (v 1.0)
% This means you may do almost anything with this work of mine, so long as you give me proper credit

There is a special type of diode called a {\it varactor}, which is used to create a voltage-dependent capacitance.  This function is often used in electronic radio tuner circuits:

$$\epsfbox{01386x01.eps}$$

The voltage-dependent capacitance of this diode is given by the following equation:

$$C_j = {C_o \over {\sqrt{2V + 1}}}$$

\noindent
Where,

$C_J =$ Junction capacitance

$C_o =$ Junction capacitance with no applied voltage

$V = $ Applied reverse junction voltage 

\vskip 10pt

Based on this equation, would you say that capacitance is {\it directly} or {\it inversely} related to the applied reverse-bias voltage of a varactor diode?  Based on what you know of diode theory, explain why this makes sense.

\underbar{file 01386}
%(END_QUESTION)





%(BEGIN_ANSWER)

$C_j$ is inversely related to $V$ for a varactor diode.

\vskip 10pt

Follow-up question: substitute the varactor diode capacitance equation into the standard resonant frequency equation to arrive at one equation solving for frequency in terms of $L$ and diode voltage $V$.

%(END_ANSWER)





%(BEGIN_NOTES)

This question reinforces students' understanding of the mathematical terms {\it direct} and {\it inverse}, as well as review basic PN junction theory and capacitor theory.

%INDEX% Diode, varactor
%INDEX% Varactor diode

%(END_NOTES)


