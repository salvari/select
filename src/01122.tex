
%(BEGIN_QUESTION)
% Copyright 2005, Tony R. Kuphaldt, released under the Creative Commons Attribution License (v 1.0)
% This means you may do almost anything with this work of mine, so long as you give me proper credit

One of the problems with capacitively-coupled amplifier circuits is poor low-frequency response: as the input signal frequency decreases, all capacitive reactances increase, leading to a decreased voltage gain.  One solution to this problem is the addition of a capacitor in the collector current path of the initial transistor stage:

$$\epsfbox{01122x01.eps}$$

Explain how the presence of this "compensating" capacitor helps to overcome the loss of gain normally experienced as a result of the other capacitors in the circuit.

\underbar{file 01122}
%(END_QUESTION)





%(BEGIN_ANSWER)

The additional capacitor's rising reactance at low frequencies boosts the gain of the first transistor stage by increasing the impedance from the first transistor's collector to the +V power supply rail.

%(END_ANSWER)





%(BEGIN_NOTES)

This technique is commonly used in video amplifier circuits, although a complete video amplifier circuit would not be this crude (no peaking coils).

%INDEX% Amplifier, interstage coupling
%INDEX% Amplifier, low-frequency compensation

%(END_NOTES)


