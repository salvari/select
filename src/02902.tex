
%(BEGIN_QUESTION)
% Copyright 2005, Tony R. Kuphaldt, released under the Creative Commons Attribution License (v 1.0)
% This means you may do almost anything with this work of mine, so long as you give me proper credit

\noindent
NAME: \underbar{\hskip 150pt}

\centerline{\bf Metacognitive Survey} 

\medskip
\goodbreak
\item{$1.$} Describe some {\it specific} things you learned in the last course section:
\vskip 40pt
\goodbreak
\item{$2.$} Identify one or more {\it specific} concepts in this course section you found relatively easy to understand:
\vskip 40pt
\goodbreak
\item{$3.$} Do those concepts share anything in common with other concepts you have found easy to understand in the past?  If so, what is common between them?
\vskip 40pt
\goodbreak
\item{$4.$} Identify one or more {\it specific} concepts in this course section you found confusing:
\vskip 40pt
\goodbreak
\item{$5.$} Do those concepts share anything in common with other concepts you have found confusing in the past?  If so, what is common between them?
\vskip 40pt
\goodbreak
\item{$6.$} Identify at least one concept or skill you will devote more study and practice to:
\vskip 40pt
\goodbreak
\item{$7.$} How well were you able to budget your time during this last course section?  Identify at least one thing you can do (starting {\it today}) to better manage your time: 
\vskip 40pt
\goodbreak
\item{$8.$} Describe at least one specific action you will take (starting {\it today}) to overcome your most significant learning obstacle.
\vskip 40pt
\goodbreak
\item{$9.$} If the last action you took to overcome a learning obstacle was not successful, explain why:
\vskip 40pt
\medskip

{\bf Remember: \underbar{you} are ultimately responsible for your own learning.  No one can force you to learn, and no one can achieve your goals for you.}

\vskip 10pt

\underbar{file 02902}

\vfil \eject

%(END_QUESTION)





%(BEGIN_ANSWER)

Be sure to give specific answers.  If you cannot think of something to write as an answer, you are not thinking deeply enough.  If you think of "everything" in response to questions regarding difficult concepts and learning obstacles, you either need to take a lower-level course or you are not thinking deeply enough about your own learning process.  If you cannot be specific in answering these questions, you need to see your instructor for help, because he/she will definitely be able to recognize specific weaknesses after working with you.

%(END_ANSWER)





%(BEGIN_NOTES)

{\it Metacognition} is awareness of one's own thought processes, and in my opinion is the most important skill one can gain from higher education.  To be able to monitor and assess one's own learning, and to take corrective action to overcome weaknesses, is vital in becoming a self-directed learner.  While every dedicated and competent instructor strives to develop metacognitive skills in their students, it is not often taught explicitly.  The purpose of this question is to remind students of the importance of metacognition and to help them become self-reflective by asking them to complete a survey about what they have learned, how they learned it, what obstacles they had to overcome, and what specific actions they will take to overcome remaining obstacles.

%INDEX% Metacognitive survey
%INDEX% Survey, metacognitive

%(END_NOTES)


