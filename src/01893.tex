
%(BEGIN_QUESTION)
% Copyright 2003, Tony R. Kuphaldt, released under the Creative Commons Attribution License (v 1.0)
% This means you may do almost anything with this work of mine, so long as you give me proper credit

When technicians and engineers consider harmonics in AC power systems, they usually only consider {\it odd-numbered} harmonic frequencies.  Explain why this is.

\underbar{file 01893}
%(END_QUESTION)





%(BEGIN_ANSWER)

Nonlinear loads are usually (but not always!) symmetrical in their distortion.

%(END_ANSWER)





%(BEGIN_NOTES)

I've had electrical power system experts confidently tell me that even-numbered harmonics {\it cannot} exist in AC power systems, due to some deep mathematical principle mysteriously beyond their ability to describe or explain.  Rubbish!  Even-numbered harmonics can and do appear in AC power systems, although they are typically much lower in amplitude than the odd-numbered harmonics due to the nature of most nonlinear loads.

If you ever wish to prove the existence of even-numbered harmonics in a power system, all you have to do is analyze the input current waveform of a half-wave rectifier!

%INDEX% Harmonics, odd versus even in AC power systems

%(END_NOTES)


