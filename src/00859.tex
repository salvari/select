
%(BEGIN_QUESTION)
% Copyright 2003, Tony R. Kuphaldt, released under the Creative Commons Attribution License (v 1.0)
% This means you may do almost anything with this work of mine, so long as you give me proper credit

It is not uncommon to see impedances represented in AC circuits as boxes, rather than as combinations of R, L, and/or C.  This is simply a convenient way to represent what may be complex sub-networks of components in a larger AC circuit:

$$\epsfbox{00859x01.eps}$$

We know that any given impedance may be represented by a simple, two-component circuit: either a resistor and a reactive component connected in series, or a resistor and a reactive component connected in parallel.  Assuming a circuit frequency of 250 Hz, determine what combination of series-connected components will be equivalent to this "box" impedance, and also what combination of parallel-connected components will be equivalent to this "box" impedance.

\underbar{file 00859}
%(END_QUESTION)





%(BEGIN_ANSWER)

$$\epsfbox{00859x02.eps}$$

%(END_ANSWER)





%(BEGIN_NOTES)

Once students learn to convert between complex impedances, equivalent series R-X circuits, and equivalent parallel R-X circuits, it becomes possible for them to analyze the most complex series-parallel impedance combinations imaginable {\it without} having to do arithmetic with complex numbers (magnitudes and angles at every step).  It does, however, require that students have a good working knowledge of resistance, conductance, reactance, susceptance, impedance, and admittance, and how these quantities relate mathematically to one another in scalar form.

%INDEX% Equivalent networks, series and parallel impedances

%(END_NOTES)


