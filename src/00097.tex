
%(BEGIN_QUESTION)
% Copyright 2003, Tony R. Kuphaldt, released under the Creative Commons Attribution License (v 1.0)
% This means you may do almost anything with this work of mine, so long as you give me proper credit

Shown here is a circuit constructed on a PCB (a "Printed Circuit Board"), with copper "traces" serving as wires to connect the components together:

$$\epsfbox{00097x01.eps}$$

How would the multimeter be used to measure the current through the component labeled "R1" when energized?  Include these important points in your answer:

\item {$\bullet$} The configuration of the multimeter (selector switch position, test lead jacks)
\item {$\bullet$} The connections of the meter test leads to the circuit
\item {$\bullet$} The state of the switch on the PCB (open or closed)

\underbar{file 00097}
%(END_QUESTION)





%(BEGIN_ANSWER)

$$\epsfbox{00097x02.eps}$$

In order to measure current through resistor R1, one of its leads must be de-soldered from the circuit board so that the meter may be connected directly in-line (in {\it series}) with it.

%(END_ANSWER)





%(BEGIN_NOTES)

Many multimeters use "international" symbols to label DC and AC selector switch positions.  It is important for students to understand what these symbols mean.

As you can see in this answer, measuring current through components is generally more difficult than measuring voltage across components, and involves greater risk because the meter must conduct the component's full current (which in some cases may be significant).  For this reason, technicians need to learn troubleshooting techniques prioritizing voltage measurements over current measurements.

%INDEX% Printed circuit board
%INDEX% Ammeter usage

%(END_NOTES)


