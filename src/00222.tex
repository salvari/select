
%(BEGIN_QUESTION)
% Copyright 2003, Tony R. Kuphaldt, released under the Creative Commons Attribution License (v 1.0)
% This means you may do almost anything with this work of mine, so long as you give me proper credit

When light strikes a {\it phototube}, an electric current is registered by the ammeter in this circuit:

$$\epsfbox{00222x01.eps}$$

What phenomenon is exhibited by the phototube in its function?  What possible applications of this effect can you think of?

\underbar{file 00222}
%(END_QUESTION)





%(BEGIN_ANSWER)

This phenomenon is called the {\it photoelectric effect}.  Incidentally, the mathematical expression of this effect earned Albert Einstein a Nobel Prize for Physics in the year 1921.

%(END_ANSWER)





%(BEGIN_NOTES)

There are several features of the schematic diagram worthy of note:

\medskip
\item{$\bullet$} High-voltage battery
\item{$\bullet$} Ammeter symbol (circle with letter "A" inside)
\item{$\bullet$} Twin arrows designating light
\medskip

Applications of the photoelectric effect are many and varied.  Special versions of the basic phototube (called {\it photomultiplier} tubes) are able to detect extremely small impulses of light, enabling scientists to detect small light flashes from cosmic-ray events, among other optical phenomena.

%INDEX% Photoelectricity, conceptual

%(END_NOTES)


