
%(BEGIN_QUESTION)
% Copyright 2006, Tony R. Kuphaldt, released under the Creative Commons Attribution License (v 1.0)
% This means you may do almost anything with this work of mine, so long as you give me proper credit

This operational amplifier circuit is often referred to as a {\it voltage buffer}, because it has unity gain (0 dB) and therefore simply reproduces, or "buffers," the input voltage:

$$\epsfbox{03801x01.eps}$$

What possible use is a circuit such as this, which offers no voltage gain or any other form of signal modification?  Wouldn't a straight piece of wire do the same thing?  Explain your answers.

$$\epsfbox{03801x02.eps}$$

\underbar{file 03801}
%(END_QUESTION)





%(BEGIN_ANSWER)

While this circuit offers no voltage gain, it does offer {\it current gain} and {\it impedance transformation}.  Much like the common-collector (or common-drain) single transistor amplifier circuits which also had voltage gains of (near) unity, opamp buffer circuits are useful whenever one must drive a relatively "heavy" (low impedance) load with a signal coming from a "weak" (high impedance) source.

%(END_ANSWER)





%(BEGIN_NOTES)

I have found that some students have difficulty with the terms "heavy" and "light" in reference to load characteristics.  That a "heavy" load would have very few ohms of impedance, and a "light" load would have many ohms of impedance seems counter-intuitive to some.  It all makes sense, though, once students realize the terms "heavy" and "light" refer to the amount of {\it current} drawn by the respective loads.

Ask your students to explain why the straight piece of wire fails to "buffer" the voltage signal in the same way the the opamp follower circuit does.

%INDEX% Voltage buffer, practical uses for

%(END_NOTES)


