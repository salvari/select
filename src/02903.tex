
%(BEGIN_QUESTION)
% Copyright 2005, Tony R. Kuphaldt, released under the Creative Commons Attribution License (v 1.0)
% This means you may do almost anything with this work of mine, so long as you give me proper credit

As an instructor, how to you help students develop the ability to monitor their own learning?  What specific tools do you use to teach metacognitive skills?

\underbar{file 02903}
%(END_QUESTION)





%(BEGIN_ANSWER)

If you are at a loss for an answer here, try some of these suggestions:

\medskip
\goodbreak
\item{$\bullet$} Resist the temptation to provide immediate answers to students' questions.  Silence is an underrated teaching tool -- return the question to the student in the form of an easier question, giving them time to think it over and to respond.
\item{$\bullet$} Have your students regularly share their problem-solving strategies in class so you and their peers can see what's going on inside their heads.
\item{$\bullet$} Give specific assignments that ask students to document the steps necessary to solve a problem, rather than merely asking them to provide an answer.
\medskip

%(END_ANSWER)





%(BEGIN_NOTES)

The purpose of this question is to get instructors to think about how they teach students to think about thinking.  So, I suppose you could call this a meta-meta-cognitive question.

%(END_NOTES)


