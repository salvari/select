
%(BEGIN_QUESTION)
% Copyright 2005, Tony R. Kuphaldt, released under the Creative Commons Attribution License (v 1.0)
% This means you may do almost anything with this work of mine, so long as you give me proper credit

Suppose you owned a scientific calculator with two broken buttons: the multiply ($\times$) and divide ($\div$).  Demonstrate how you could solve this simple multiplication problem using only logarithms, addition, and antilogarithms (powers):

$$12 \div 3 = \hbox{???}$$

The answer to this problem was easy enough for you to figure out without a calculator at all, so here are some more practice problems for you to try:

\medskip
\goodbreak
\item{$\bullet$} $122 \div 35$ = 
\vskip 5pt
\item{$\bullet$} $781 \div 92$ = 
\vskip 5pt
\item{$\bullet$} $19.4 \div 60$ = 
\vskip 5pt
\item{$\bullet$} $3.5 \div 0.21$ = 
\medskip

\underbar{file 02686}
%(END_QUESTION)





%(BEGIN_ANSWER)

Here I will show you the steps to using logarithms to solve the first multiplication problem:

$$12 \div 3 = \hbox{???}$$

$$12 \div 3 = 10^{\log 12 - \log 3}$$

$$12 \div 3 = 10^{1.0792 - 0.4771}$$

$$12 \div 3 = 10^{0.6021}$$

$$12 \div 3 = 4$$

Since the others are easy enough for you to check (with your non-broken calculator!), I'll leave their solutions in your capable hands.

%(END_ANSWER)





%(BEGIN_NOTES)

Incidentally, there is nothing special about the common logarithm to warrant its exclusive use in this problem.  We could have just as easily applied the natural logarithm function with the same (final) result:

$$12 \div 3 = \hbox{???}$$

$$12 \div 3 = e^{\ln 12 - \ln 3}$$

$$12 \div 3 = e^{2.4849 - 1.0986}$$

$$12 \div 3 = e^{1.3863}$$

$$12 \div 3 = 4$$

%INDEX% Logarithms, used to transform a division problem into a subtraction problem

%(END_NOTES)


