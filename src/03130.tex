
%(BEGIN_QUESTION)
% Copyright 2005, Tony R. Kuphaldt, released under the Creative Commons Attribution License (v 1.0)
% This means you may do almost anything with this work of mine, so long as you give me proper credit

Suppose a DC generator is powering an electric motor, which we model as a 100 $\Omega$ resistor:

$$\epsfbox{03130x01.eps}$$

Calculate the amount of current this generator will supply to the motor and the voltage measured across the motor's terminals, taking into account all the resistances shown (generator internal resistance $r_{gen}$, wiring resistances $R_{wire}$, and the motor's equivalent resistance).

Now suppose we connect an identical generator in parallel with the first, using connecting wire so short that we may safely discount its additional resistance:

$$\epsfbox{03130x02.eps}$$

Use the Superposition Theorem to re-calculate the motor current and motor terminal voltage, commenting on how these figures compare with the first calculation (using only one generator).

\underbar{file 03130}
%(END_QUESTION)





%(BEGIN_ANSWER)

\noindent
With only one generator connected:

$I_{motor}$ = 4.726 amps \hskip 30pt $V_{motor}$ = 472.6 volts

\vskip 10pt

\noindent
With two generators connected:

$I_{motor}$ = 4.733 amps \hskip 30pt $V_{motor}$ = 473.3 volts

\vskip 20pt

Challenge question: how much current does {\it each} generator supply to the circuit when there are two generators connected in parallel?

%(END_ANSWER)





%(BEGIN_NOTES)

Some students will erroneously leap to the conclusion that another generator will send twice the current through the load (with twice the voltage drop across the motor terminals!).  Such a conclusion is easy to reach if one does not fully understand the Superposition Theorem.

%INDEX% Superposition theorem, applied to generator/load power network

%(END_NOTES)


