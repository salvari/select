
%(BEGIN_QUESTION)
% Copyright 2003, Tony R. Kuphaldt, released under the Creative Commons Attribution License (v 1.0)
% This means you may do almost anything with this work of mine, so long as you give me proper credit

A primitive resistor may be formed by sketching a thick line on a piece of paper, using a pencil (not an ink pen!):

$$\epsfbox{00212x01.eps}$$

How may the end-to-end electrical resistance of this pencil mark be increased?  How may it be decreased?  Explain your answers.

\underbar{file 00212}
%(END_QUESTION)





%(BEGIN_ANSWER)

The electrical resistance of a pencil mark may be increased by increasing its length.  It may be decreased by increasing its width.

%(END_ANSWER)





%(BEGIN_NOTES)

Creating a resistor on paper using a pencil is a very easy experiment to perform, the resistance of which may be measured with an ohmmeter.  I strongly recommend this as a classroom exercise!

%INDEX% Resistor, made from pencil mark on paper

%(END_NOTES)


