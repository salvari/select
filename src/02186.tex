
%(BEGIN_QUESTION)
% Copyright 2004, Tony R. Kuphaldt, released under the Creative Commons Attribution License (v 1.0)
% This means you may do almost anything with this work of mine, so long as you give me proper credit

The power factor of this circuit is as low as it can possibly be, 0:

$$\epsfbox{02186x01.eps}$$

Calculate the apparent, true, and reactive power for this circuit:

\medskip
\goodbreak
\item{$\bullet$} $S$ =
\item{$\bullet$} $P$ = 
\item{$\bullet$} $Q$ =
\medskip

Now, suppose a capacitor is added in parallel with the inductor:

$$\epsfbox{02186x02.eps}$$

Re-calculate the apparent, true, and reactive power for this circuit with the capacitor connected:

\medskip
\goodbreak
\item{$\bullet$} $S$ = 
\item{$\bullet$} $P$ = 
\item{$\bullet$} $Q$ = 
\medskip

\underbar{file 02186}
%(END_QUESTION)





%(BEGIN_ANSWER)

\medskip
\goodbreak
\item{} {\bf Without capacitor}
\item{$\bullet$} $S$ = 1.432 VA
\item{$\bullet$} $P$ = 0 W
\item{$\bullet$} $Q$ = 1.432 VAR
\medskip

\medskip
\goodbreak
\item{} {\bf With capacitor}
\item{$\bullet$} $S$ = 0.369 VA
\item{$\bullet$} $P$ = 0 W
\item{$\bullet$} $Q$ = 0.369 VAR
\medskip

%(END_ANSWER)





%(BEGIN_NOTES)

Although the power factor of this circuit is still 0, the total current drawn from the source has been substantially reduced.  This is the essence of power factor correction, and the point of this question.

%INDEX% Power calculation, LC circuit

%(END_NOTES)


