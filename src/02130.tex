
%(BEGIN_QUESTION)
% Copyright 2004, Tony R. Kuphaldt, released under the Creative Commons Attribution License (v 1.0)
% This means you may do almost anything with this work of mine, so long as you give me proper credit

One of the practical uses of transformers is to adapt equipment to conditions not anticipated in their original design.  For instance, a heating element (which is essentially nothing more than a resistor with an unusually high power dissipation rating) may need to be operated at a lower power dissipation than designed for.  

For example, suppose you have a 1 kW electric heater rated for 208 volt operation, which you intend to operate at a reduced power dissipation of 750 watts.  Calculate the proper amount of voltage you would need to achieve this reduced power dissipation, and explain how you could use a transformer to supply this reduced voltage to the heater.

\underbar{file 02130}
%(END_QUESTION)





%(BEGIN_ANSWER)

The necessary voltage to make this 1 kW heater operate at only 750 W is approximately 180 volts.

%(END_ANSWER)





%(BEGIN_NOTES)

Some students may struggle in calculating the necessary voltage, because this problem does not exactly match most voltage/current/power calculations problems they've seen in the past.  The necessary math is almost trivial, but the "trick" is applying well-known equations to something unfamiliar.  This is an excellent opportunity to discuss problem-solving strategies, so be sure to have students share their ideas on how to solve for the necessary voltage.

%INDEX% Transformer, used for voltage reduction

%(END_NOTES)


