
%(BEGIN_QUESTION)
% Copyright 2003, Tony R. Kuphaldt, released under the Creative Commons Attribution License (v 1.0)
% This means you may do almost anything with this work of mine, so long as you give me proper credit

{\it Solderless breadboards} provide convenient means for electronics hobbyists, students, technicians, and engineers to build circuits in a non-permanent form.  The following illustration shows a three-resistor series circuit built on a breadboard:

$$\epsfbox{01761x01.eps}$$

The interconnections between the metal spring clips within the holes of the breadboard allow continuity between adjacent leads of the resistors, without the resistor leads having to be jammed into the same hole.

However, new students often get themselves into trouble when first learning how to use solderless breadboards.  One common mistake is shown here, where a student has attempted to create a simple single-resistor circuit:

$$\epsfbox{01761x02.eps}$$

What the student has actually created here is a {\it short circuit}.  Re-draw this circuit in schematic form, and explain why this circuit is faulty.

\underbar{file 01761}
%(END_QUESTION)





%(BEGIN_ANSWER)

$$\epsfbox{01761x03.eps}$$

\vskip 10pt

Follow-up question \#1: explain what might happen if a large battery or high-current power supply were powering this short circuit.

\vskip 10pt

Follow-up question \#2: show how the single-resistor circuit {\it should} have been built on the breadboard so as to avoid a short circuit.

%(END_ANSWER)





%(BEGIN_NOTES)

Situations such as this are {\it very} common among new students!  Be sure to discuss the significance of "short circuits" as well as how to avoid them.

%INDEX% Short circuit, created by improper component layout on breadboard

%(END_NOTES)


