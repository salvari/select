
%(BEGIN_QUESTION)
% Copyright 2003, Tony R. Kuphaldt, released under the Creative Commons Attribution License (v 1.0)
% This means you may do almost anything with this work of mine, so long as you give me proper credit

Calculate the voltage across the starter motor terminals of the "dead" car, and the current through the starter motor, while a second car is giving it a jump-start:

$$\epsfbox{00480x01.eps}$$

Regard the starter motor itself as a 0.15 $\Omega$ resistor, and disregard any resistance of the jumper cables connecting the two cars' electrical systems together.

\underbar{file 00480}
%(END_QUESTION)





%(BEGIN_ANSWER)

$E_{motor} =$ 9.534 V

$I_{motor} =$ 63.56 A

%(END_ANSWER)





%(BEGIN_NOTES)

For a question like this, where an equivalent schematic diagram is essential to obtaining the solution, I recommend you have a student draw their equivalent schematic on the whiteboard in front of the class, and discuss the diagram with all your students before discussing how to apply Millman's theorem.

I have found it helpful for students to have them draw diagrams and mathematical solutions on a board in front of the rest of the class.  Of course, you as the instructor must be careful to maintain a non-threatening environment in the classroom while students do this, as it tends to place a lot of stress on shy students.  However, the ability to present graphical information to a group is a valuable skill, and exercises like this help to build it in your students.

%INDEX% Millman's theorem, applied to jump-starting cars

%(END_NOTES)


