
%(BEGIN_QUESTION)
% Copyright 2003, Tony R. Kuphaldt, released under the Creative Commons Attribution License (v 1.0)
% This means you may do almost anything with this work of mine, so long as you give me proper credit

Describe the function of each component in this two-stage amplifier circuit:

$$\epsfbox{01117x01.eps}$$

Also, be prepared to explain what the effect of any one component's failure (either open or shorted) will have on the output signal.

\underbar{file 01117}
%(END_QUESTION)





%(BEGIN_ANSWER)

\medskip
\item{$\bullet$} $R_1 = $ $Q_1$ biasing
\item{$\bullet$} $R_2 = $ $Q_1$ biasing 
\item{$\bullet$} $R_3 = $ $Q_1$ load
\item{$\bullet$} $R_4 = $ $Q_1$ stability (prevents thermal runaway)
\item{$\bullet$} $R_5 = $ $Q_2$ biasing
\item{$\bullet$} $R_6 = $ $Q_2$ biasing
\item{$\bullet$} $R_7 = $ $Q_2$ load
\item{$\bullet$} $R_8 = $ $Q_2$ stability (prevents thermal runaway)
\medskip

\medskip
\item{$\bullet$} $C_1 = $ Input signal coupling to $Q_1$
\item{$\bullet$} $C_2 = $ AC bypass for $Q_1$
\item{$\bullet$} $C_3 = $ Coupling between amplifier stages
\item{$\bullet$} $C_4 = $ AC bypass for $Q_2$
\item{$\bullet$} $C_5 = $ Output signal coupling to load
\medskip

\medskip
\item{$\bullet$} $Q_1 = $ First-stage amplification
\item{$\bullet$} $Q_2 = $ Second-stage amplification
\medskip

%(END_ANSWER)





%(BEGIN_NOTES)

The answers given in the "Answers" section are minimal: just enough to help students who may be struggling with the concepts.  During discussion, I would expect more detail than these short phrases.

Be sure to challenge your students with hypothetical component failures in this circuit.  Make sure they comprehend each component's function in this circuit, beyond memorizing a phrase!

%INDEX% Amplifier, multiple-stage
%INDEX% Troubleshooting, amplifier

%(END_NOTES)


