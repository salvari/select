
%(BEGIN_QUESTION)
% Copyright 2004, Tony R. Kuphaldt, released under the Creative Commons Attribution License (v 1.0)
% This means you may do almost anything with this work of mine, so long as you give me proper credit

A circuit often used to amplitude-modulate a carrier signal is a {\it multiplier}:

$$\epsfbox{02284x01.eps}$$

Explain how the instantaneous multiplication of two sine waves results in amplitude modulation.  If possible, graph this on a graphing calculator or other computer plotting device.

\underbar{file 02284}
%(END_QUESTION)





%(BEGIN_ANSWER)

I'll let you figure this one out on your own!

%(END_ANSWER)





%(BEGIN_NOTES)

Multiplier circuits are quite useful, and not just for amplitude modulation.  The fact that they can be used as amplitude modulators, though, is a concept some students find hard to understand.  One illustration that might clear things up is an adjustable voltage divider (since multiplication and division are very closely related):

$$\epsfbox{02284x02.eps}$$

Now, such a potentiometer circuit would be totally impractical for any modulating signal frequency measured in Hertz, as the potentiometer would wear out very rapidly from all the motion.  It is the {\it principle} of modulated voltage division that this circuit helps to communicate, though.  Multiplier circuits do the same thing, only multiplying the amplitude of the carrier signal rather than dividing it.

%INDEX% Multiplier, used as amplitude modulator

%(END_NOTES)


