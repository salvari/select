
%(BEGIN_QUESTION)
% Copyright 2005, Tony R. Kuphaldt, released under the Creative Commons Attribution License (v 1.0)
% This means you may do almost anything with this work of mine, so long as you give me proper credit

There is something wrong with this amplifier circuit.  Despite an audio signal of normal amplitude detected at test point 1 (TP1), there is no output measured at the "Audio signal out" jack:

$$\epsfbox{02474x01.eps}$$

Next, you decide to check for the presence of a good signal at test point 3 (TP3).  There, you find 0 volts AC and DC no matter where the volume control is set.

From this information, formulate a plan for troubleshooting this circuit, answering the following questions:

\medskip
\goodbreak
\item{$\bullet$} What type of signal would you expect to measure at TP3?
\item{$\bullet$} What would be your next step in troubleshooting this circuit?
\item{$\bullet$} Are there any elements of this circuit you know to be working properly?
\item{$\bullet$} What do you suppose would be the most {\it likely} failure, assuming this circuit once worked just fine and suddenly stopped working all on it's own?
\medskip

\underbar{file 02474}
%(END_QUESTION)





%(BEGIN_ANSWER)

The correct voltage signal at TP3 should be an audio waveform with significant crossover distortion (specifically, a vertical "jump" at each point where the waveform crosses zero volts, about 1.4 volts peak to peak).  I'll let you figure out answers to the other questions on your own, or with classmates.

%(END_ANSWER)





%(BEGIN_NOTES)

I have found that troubleshooting scenarios are always good for stimulating class discussions, with students posing strategies for isolating the fault(s) and correcting one another on logical errors.  There is not enough information given in this question to ensure a single, correct answer.  Discuss this with your students, helping them to use their knowledge of circuit theory and opamps to formulate good diagnostic strategies.

%INDEX% Troubleshooting, inverting amplifier circuit (opamp)

%(END_NOTES)


