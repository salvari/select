
%(BEGIN_QUESTION)
% Copyright 2006, Tony R. Kuphaldt, released under the Creative Commons Attribution License (v 1.0)
% This means you may do almost anything with this work of mine, so long as you give me proper credit

The {\it Delta-Sigma} or {\it Sigma-Delta} analog-to-digital converter works on the principle of {\it oversampling}, whereby a low-resolution ADC repeatedly samples the input signal in a feedback loop.  In many cases, the ADC used is nothing more than a comparator (a 1-bit ADC!), the output of this ADC subtracted from the input signal and integrated over time in an attempt to achieve a balance near 0 volts at the output of the integrator.  The result is a {\it pulse-density modulated} (PDM) "bitstream" of 1-bit digital data which may be filtered and {\it decimated} (converted to a binary word of multiple bits):

$$\epsfbox{04011x01.eps}$$

\goodbreak
Explain what this PDM bitstream would look like for the following input voltage conditions:

\medskip
\item{$\bullet$} $V_{in}$ = 0 volts 
\item{$\bullet$} $V_{in}$ = $V_{DD}$ 
\item{$\bullet$} $V_{in}$ = $V_{ref}$ 
\medskip

\underbar{file 04011}
%(END_QUESTION)





%(BEGIN_ANSWER)

\medskip
\item{$\bullet$} $V_{in}$ = 0 volts ; bitstream = {\tt 00000000 . . .}
\item{$\bullet$} $V_{in}$ = $V_{DD}$ ; bitstream = {\tt 11111111 . . .}
\item{$\bullet$} $V_{in}$ = $V_{ref}$ ; bitstream = {\tt 01010101 . . .}
\medskip

%(END_ANSWER)





%(BEGIN_NOTES)

In order to answer this question, students must have a good grasp of how the summing integrator works.  Discuss with them how the feedback loop's "goal" is to maintain the integrator output at the reference voltage ($V_{ref}$), and how the 1-bit ADC can only make adjustments to the integrator's output by driving it upward or downward by the same analog quantity every clock pulse.

%INDEX% ADC, Delta-Sigma
%INDEX% Delta-Sigma converter, ADC
%INDEX% Pulse-density modulation (PDM), Delta-Sigma converter
%INDEX% Sigma-Delta converter, ADC

%(END_NOTES)


