
%(BEGIN_QUESTION)
% Copyright 2003, Tony R. Kuphaldt, released under the Creative Commons Attribution License (v 1.0)
% This means you may do almost anything with this work of mine, so long as you give me proper credit

If a positive charge is forcibly moved closer to another positive charge, mechanical work must be performed.  This is analogous to compressing a spring: work must be done to cause motion against the natural repulsion of the system:

$$\epsfbox{01700x01.eps}$$

This mechanical work becomes "stored" in the electric field between the charges, and is a form of {\it potential energy}.  This, again, is similar to a mechanical spring, where the work done in compressing a spring is "stored" as potential energy in its compressed state.  

Define {\it voltage} in terms of the change in potential energy resulting from this forceful motion of the charge, the way a physicist would.

\underbar{file 01700}
%(END_QUESTION)





%(BEGIN_ANSWER)

$$V_{ab} = {{U_b - U_a} \over q}$$

\noindent
Where,

$V_{ab} = $ Voltage between points A and B

$U_b - U_a = $ Difference in potential energy between points A and B 

$q = $ Quantity of charge (moved), in Coulombs

\vskip 10pt

Follow-up question: explain why, if a charge of twice the magnitude ($2q$) were moved from the same point A to the same point B, there would be twice as much work done and twice as much potential energy stored, but exactly the same voltage ({\it potential}) as gained by the smaller charge ($q$).

%(END_ANSWER)





%(BEGIN_NOTES)

If time permits, it is quite enlightening for students to discover that voltage may be defined in simple mechanical terms of repulsion and work.  Of course, a basic familiarity with {\it work} in the scientific sense of the term is a prerequisite for understanding this scenario.

%INDEX% Voltage, defined in terms of work and charge

%(END_NOTES)


