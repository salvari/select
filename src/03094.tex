
%(BEGIN_QUESTION)
% Copyright 2005, Tony R. Kuphaldt, released under the Creative Commons Attribution License (v 1.0)
% This means you may do almost anything with this work of mine, so long as you give me proper credit

Suppose you were given the following two equations and asked to find solutions for $x$ and $y$ that will satisfy {\it both} at the same time:

$$y + x = 8$$

$$y - x = 3$$

If we manipulate the second equation so as to solve for $y$, we will have a definition of $y$ in terms of $x$ that we may use for substitution in the first equation:

$$y = x + 3$$

Show the process of substitution into the first equation, and how this leads to a single solution for $x$.  Then, use that value of $x$ to solve for $y$, resulting in a solution set valid for {\it both} equations.

\underbar{file 03094}
%(END_QUESTION)





%(BEGIN_ANSWER)

If $y + x = 8$ and $y = x + 3$, then $(x + 3) + x = 8$.  Therefore, 

$$x = 2.5 \hbox{ and } y = 5.5$$

%(END_ANSWER)





%(BEGIN_NOTES)

This question demonstrates one of the (many) practical uses of algebraic substitution: solving simultaneous systems of equations.

%INDEX% Algebra, substitution
%INDEX% Simultaneous equations
%INDEX% Systems of linear equations

%(END_NOTES)


