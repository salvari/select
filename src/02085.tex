
%(BEGIN_QUESTION)
% Copyright 2003, Tony R. Kuphaldt, released under the Creative Commons Attribution License (v 1.0)
% This means you may do almost anything with this work of mine, so long as you give me proper credit

A series AC circuit contains 1125 ohms of resistance and 1500 ohms of reactance for a total circuit impedance of 1875 ohms.  This may be represented graphically in the form of an impedance triangle:

$$\epsfbox{02085x01.eps}$$

Since all side lengths on this triangle are known, there is no need to apply the Pythagorean Theorem.  However, we may still calculate the two non-perpendicular angles in this triangle using "inverse" trigonometric functions, which are sometimes called {\it arc}functions.

\goodbreak

Identify which arc-function should be used to calculate the angle $\Theta$ given the following pairs of sides:

$$R \hbox{ and } Z$$

$$X \hbox{ and } R$$

$$X \hbox{ and } Z$$

Show how three different trigonometric arcfunctions may be used to calculate the same angle $\Theta$.

\underbar{file 02085}
%(END_QUESTION)





%(BEGIN_ANSWER)

$$\arccos {R \over Z} = 53.13^o$$

$$\arctan {X \over R} = 53.13^o$$

$$\arcsin {X \over Z} = 53.13^o$$

\vskip 10pt

Challenge question: identify three {\it more} arcfunctions which could be used to calculate the same angle $\Theta$.

%(END_ANSWER)





%(BEGIN_NOTES)

Some hand calculators identify arc-trig functions by the letter "A" prepending each trigonometric abbreviation (e.g. "ASIN" or "ATAN").  Other hand calculators use the inverse function notation of a -1 exponent, which is {\it not} actually an exponent at all (e.g. $\sin^{-1}$ or $\tan^{-1}$).  Be sure to discuss function notation on your students' calculators, so they know what to invoke when solving problems such as this.

%INDEX% Trigonometry, arc-cosine function defined in impedance triangle
%INDEX% Trigonometry, arcsine function defined in impedance triangle
%INDEX% Trigonometry, arctangent function defined in impedance triangle

%(END_NOTES)


