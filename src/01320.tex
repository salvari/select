
%(BEGIN_QUESTION)
% Copyright 2003, Tony R. Kuphaldt, released under the Creative Commons Attribution License (v 1.0)
% This means you may do almost anything with this work of mine, so long as you give me proper credit

Suppose you needed an inverter gate in a logic circuit, but none were available.  You do, however, have a spare (unused) NAND gate in one of the integrated circuits.  Show how you would connect a NAND gate to function as an inverter.

Use Boolean algebra to show that your solution is valid.

\underbar{file 01320}
%(END_QUESTION)





%(BEGIN_ANSWER)

$$\epsfbox{01320x01.eps}$$

For the above solution: $\overline{AA} = \overline{A}$

\vskip 10pt

Follow-up question: are there any other ways to use a NAND gate as an inverter?  The method shown above is not the only valid solution!

%(END_ANSWER)





%(BEGIN_NOTES)

Not only is the method shown in the answer not the only valid solution, but it may even be the worst one!  Your students should be able to research or invent alternative inverter connections, so after asking them to present their alternatives, ask the class as a whole to decide which solution is better.  Ask them to consider electrical parameters, such as propagation delay time and fan-out.

%INDEX% NAND gate, used as an inverter

%(END_NOTES)


