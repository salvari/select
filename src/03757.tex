
%(BEGIN_QUESTION)
% Copyright 2005, Tony R. Kuphaldt, released under the Creative Commons Attribution License (v 1.0)
% This means you may do almost anything with this work of mine, so long as you give me proper credit

Suppose some of the turns of wire (but not all) in the primary winding of the transformer were to fail shorted in this Armstrong oscillator circuit:

$$\epsfbox{03757x01.eps}$$

How would this effective decreasing of the primary winding turns affect the operation of this circuit?  What if it were the secondary winding of the transformer to suffer this fault instead of the primary?

\underbar{file 03757}
%(END_QUESTION)





%(BEGIN_ANSWER)

A partially shorted primary winding will result in increased frequency and (possibly) increased distortion in the output signal.  A partially shorted secondary winding may result in oscillations ceasing altogether!

%(END_ANSWER)





%(BEGIN_NOTES)

The purpose of this question is to approach the domain of circuit troubleshooting from a perspective of knowing what the fault is, rather than only knowing what the symptoms are.  Although this is not necessarily a realistic perspective, it helps students build the foundational knowledge necessary to diagnose a faulted circuit from empirical data.  Questions such as this should be followed (eventually) by other questions asking students to identify likely faults based on measurements.

%INDEX% Troubleshooting, predicting effects of fault in Armstrong oscillator circuit

%(END_NOTES)


