
%(BEGIN_QUESTION)
% Copyright 2005, Tony R. Kuphaldt, released under the Creative Commons Attribution License (v 1.0)
% This means you may do almost anything with this work of mine, so long as you give me proper credit

An electrical testing laboratory has designed and built a high-current "surge" power supply for testing the effects of large electrical currents on certain types of components.  The basic idea is that a bank of capacitors are charged to a high voltage by a DC power source (through a current-limiting resistor to protect the power source from overcurrent damage), then quickly discharged through a switch to the test load:

$$\epsfbox{03442x01.eps}$$

Due to the magnitude of the transient currents generated by the discharging capacitors, a mechanical switch (as suggested by the pushbutton symbol) is impractical.  Instead, an {\it air gap switch} is constructed from two metal balls with a needle-point "ionizer" electrode between them.  The ionizer electrode connects to a simple inductor/switch circuit that generates a high-voltage pulse sufficient to create a spark.  The ionized air created in the spark provides a low-resistance path through the air which the 2000 volts from the capacitor bank can now traverse, thus completing the circuit for surge current between the capacitors and the test load:

$$\epsfbox{03442x02.eps}$$

Suppose this surge power supply circuit failed to work after several months of good service.  Nothing happens any more when the pushbutton switch is pressed and released.  No spark is heard or seen in the air gap, and the test load receives no surge of current.

\goodbreak
A voltmeter placed (carefully!) in parallel with the capacitor bank indicates a full charge voltage of 2035 volts, which is within normal parameters.  Based on this information, identify these things:

\vskip 10pt

\medskip
\item{$\bullet$} \underbar{Two} components in the circuit that you know must be in good working condition.
\vskip 20pt
\item{$\bullet$} \underbar{Two} components in the circuit that could possibly be bad, and the mode of their failure (either open or shorted).
\medskip

\underbar{file 03442}
%(END_QUESTION)





%(BEGIN_ANSWER)

The 2 kV DC power source must be functioning properly, as is its current-limiting resistor ($R_{limit1}$).  Possible failed components include:

\medskip
\item{$\bullet$} $R_{limit2}$ failed open
\item{$\bullet$} Inductor failed open or shorted
\item{$\bullet$} Pushbutton switch failed open or shorted
\item{$\bullet$} 24 VDC supply failed
\item{$\bullet$} Air gap switch "ionizing" needle tip burnt or otherwise worn
\medskip

\vskip 10pt

Follow-up question: explain how one would {\it safely} continue diagnostic measurements in a circuit such as this where there is much potential for electric shock and other hazards.

%(END_ANSWER)





%(BEGIN_NOTES)

In addition to introducing the concept of an air gap switch and the notion of a "surge" power supply, this question challenges students to envision practical problems and their respective diagnostic techniques.

It goes without saying that a circuit such as this is {\it very} dangerous, and its construction should not be attempted by anyone lacking a thorough understanding of its relevant hazards.  Having said that, I will admit to having built one myself, simply to test the validity of an air gap switch for high-voltage, high-current, transient switching.  Yes, the concept does work!

%INDEX% Surge current power supply
%INDEX% Transient current power supply

%(END_NOTES)


