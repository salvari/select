
%(BEGIN_QUESTION)
% Copyright 2003, Tony R. Kuphaldt, released under the Creative Commons Attribution License (v 1.0)
% This means you may do almost anything with this work of mine, so long as you give me proper credit

Ideally, an inverting amplifier circuit may be comprised of just one op-amp and two resistors, as such:

$$\epsfbox{01007x01.eps}$$

However, if high accuracy is desired, a third resistor must be added to the circuit, in series with the other op-amp input:

$$\epsfbox{01007x02.eps}$$

Explain what this "compensation" resistor is compensating for, and also what its value should be.

\underbar{file 01007}
%(END_QUESTION)





%(BEGIN_ANSWER)

The compensation resistor compensates for errors introduced into the voltage divider network due to input bias current.  Its value should be equal to the parallel equivalent of $R_{input}$ and $R_{feedback}$.

%(END_ANSWER)





%(BEGIN_NOTES)

First, your students will have to know what "bias currents" are in op-amp circuits, so begin your discussion of this question with a call for this definition.  Why the compensation resistor value must be equal to the {\it parallel} equivalent of the two resistors in the voltage divider is something that confuses most students.  The key to understanding it is network analysis, in particular Th\'evenin's and Norton's theorems.

%INDEX% Bias current compensation, opamp input

%(END_NOTES)


