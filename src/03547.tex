
%(BEGIN_QUESTION)
% Copyright 2005, Tony R. Kuphaldt, released under the Creative Commons Attribution License (v 1.0)
% This means you may do almost anything with this work of mine, so long as you give me proper credit

% Uncomment the following line if the question involves calculus at all:
\vbox{\hrule \hbox{\strut \vrule{} $\int f(x) \> dx$ \hskip 5pt {\sl Calculus alert!} \vrule} \hrule}

Inductors store energy in the form of a magnetic field.  We may calculate the energy stored in an inductance by integrating the product of inductor voltage and inductor current ($P = IV$) over time, since we know that power is the rate at which work ($W$) is done, and the amount of work done to an inductor taking it from zero current to some non-zero amount of current constitutes energy stored ($U$):

$$P = {dW \over dt}$$

$$dW = P \> dt$$

$$U = W = \int P \> dt$$

Find a way to substitute inductance ($L$) and current ($I$) into the integrand so you may integrate to find an equation describing the amount of energy stored in an inductor for any given inductance and current values.

\underbar{file 03547}
%(END_QUESTION)





%(BEGIN_ANSWER)

$$U = {1 \over 2}LI^2$$

%(END_ANSWER)





%(BEGIN_NOTES)

The integration required to obtain the answer is commonly found in calculus-based physics textbooks, and is an easy (power rule) integration.

%INDEX% Energy stored in an inductor
%INDEX% Inductor, energy storage in

%(END_NOTES)


