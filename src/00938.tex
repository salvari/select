
%(BEGIN_QUESTION)
% Copyright 2003, Tony R. Kuphaldt, released under the Creative Commons Attribution License (v 1.0)
% This means you may do almost anything with this work of mine, so long as you give me proper credit

This circuit is part of a weather monitoring station.  Wind speed is measured by the voltage output from a permanent-magnet DC generator, turned by a set of vanes.  A light bulb lights up when the wind speed passes a threshold ("trip") value, established by the potentiometer:

$$\epsfbox{00938x01.eps}$$

Based on your understanding of differential pair circuits, is this a high-speed wind indicating circuit or a low-speed wind indicating circuit?

\underbar{file 00938}
%(END_QUESTION)





%(BEGIN_ANSWER)

The light bulb energizes when the wind speed decreases below the threshold value.

%(END_ANSWER)





%(BEGIN_NOTES)

Ask your students to explain their reasoning in obtaining their answers.  What happens in this circuit as the generator voltage increases, and as it decreases?  Which way do we adjust the potentiometer to increase the trip point?  How do we know this?

Let your students know that this circuit is a simple example of what is called a {\it comparator}: a circuit that compares two voltages against each other and generates an output corresponding to which voltage is greater.

%INDEX% Differential pair, used as a voltage comparator

%(END_NOTES)


