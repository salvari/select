
%(BEGIN_QUESTION)
% Copyright 2003, Tony R. Kuphaldt, released under the Creative Commons Attribution License (v 1.0)
% This means you may do almost anything with this work of mine, so long as you give me proper credit

What does it mean if an operational amplifier has the ability to "swing its output rail to rail"?  Why is this an important feature to us?

\underbar{file 00844}
%(END_QUESTION)





%(BEGIN_ANSWER)

Being able to "swing" the output voltage "rail to rail" means that the full range of an op-amp's output voltage extends to within millivolts of either power supply "rail" (+V and -V).

\vskip 10pt

Challenge question: identify at least one op-amp model that has this ability, and at least one that does not.  Bring the datasheets for these op-amp models with you for reference during discussion time.

%(END_ANSWER)





%(BEGIN_NOTES)

Discuss what this feature means to us as circuit builders in a practical sense.  Ask those students who tackled the challenge question to look up the output voltage ranges of their op-amp models.  Exactly how close to +V and -V can the output voltage of an op-amp lacking "rail-to-rail" output capability "swing"?

%INDEX% Opamp, rail-to-rail output swing

%(END_NOTES)


