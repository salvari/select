
\centerline{\bf ELTR 110 (AC 1), section 2} \bigskip 
 
\vskip 10pt

\noindent
{\bf Recommended schedule}

\vskip 5pt

%%%%%%%%%%%%%%%
\hrule \vskip 5pt
\noindent
\underbar{Day 1}

\hskip 10pt Topics: {\it Capacitive reactance and impedance, trigonometry for AC circuits}
 
\hskip 10pt Questions: {\it 1 through 20}
 
\hskip 10pt Lab Exercise: {\it Capacitive reactance and Ohm's Law for AC (question 71)}
 
\vskip 10pt
%%%%%%%%%%%%%%%
\hrule \vskip 5pt
\noindent
\underbar{Day 2}

\hskip 10pt Topics: {\it Series and parallel RC circuits}
 
\hskip 10pt Questions: {\it 21 through 40}
 
\hskip 10pt Lab Exercise: {\it Series RC circuit (question 72)}
 
\vskip 10pt
%%%%%%%%%%%%%%%
\hrule \vskip 5pt
\noindent
\underbar{Day 3}

\hskip 10pt Topics: {\it Superposition principle, AC+DC oscilloscope coupling}
 
\hskip 10pt Questions: {\it 41 through 55}
 
\hskip 10pt Lab Exercise: {\it Parallel RC circuit (question 73)}
 
%INSTRUCTOR \hskip 10pt {\bf MIT 6.002 video clip: Disk 1, Lecture 3; Superposition theorem 29:18 to 34:14}

\vskip 10pt
%%%%%%%%%%%%%%%
\hrule \vskip 5pt
\noindent
\underbar{Day 4}

\hskip 10pt Topics: {\it Passive RC and LR filter circuits}
 
\hskip 10pt Questions: {\it 56 through 70}
 
\hskip 10pt Lab Exercise: {\it Time-domain phase shift measurement (question 74)}
 
%INSTRUCTOR \hskip 10pt {\bf MIT 8.02 video clip: Disk 3, Lecture 20; Lowpass filter using inductor 37:46 to 39:18}

\vskip 10pt
%%%%%%%%%%%%%%%
\hrule \vskip 5pt
\noindent
\underbar{Day 5}

\hskip 10pt Exam 2: {\it includes Series \underbar{or} Parallel RC circuit performance assessment}
 
\hskip 10pt Lab Exercise: {\it Troubleshooting practice (variable phase shift bridge circuit -- question 75)}
  
\vskip 10pt
%%%%%%%%%%%%%%%
\hrule \vskip 5pt
\noindent
\underbar{Practice and challenge problems}

\hskip 10pt Questions: {\it 78 through the end of the worksheet}
 
\vskip 10pt
%%%%%%%%%%%%%%%
\hrule \vskip 5pt
\noindent
\underbar{Impending deadlines}

\hskip 10pt {\bf Troubleshooting assessment (AC bridge circuit) due at end of ELTR110, Section 3}
 
\hskip 10pt Question 76: Troubleshooting log
 
\hskip 10pt Question 77: Sample troubleshooting assessment grading criteria
 
\vskip 10pt
%%%%%%%%%%%%%%%








\vfil \eject

\centerline{\bf ELTR 110 (AC 1), section 2} \bigskip 
 
\vskip 10pt

\noindent
{\bf Skill standards addressed by this course section}

\vskip 5pt

%%%%%%%%%%%%%%%
\hrule \vskip 10pt
\noindent
\underbar{EIA {\it Raising the Standard; Electronics Technician Skills for Today and Tomorrow}, June 1994}

\vskip 5pt

\medskip
\item{\bf C} {\bf Technical Skills -- AC circuits}
\item{\bf C.02} Demonstrate an understanding of the properties of an AC signal.
\item{\bf C.08} Understand principles and operations of AC capacitive circuits.
\item{\bf C.09} Fabricate and demonstrate AC capacitive circuits.
\item{\bf C.10} Troubleshoot and repair AC capacitive circuits.
\item{\bf C.27} Understand principles and operations of AC frequency selective filter circuits.
\medskip

\vskip 5pt

\medskip
\item{\bf B} {\bf Basic and Practical Skills -- Communicating on the Job}
\item{\bf B.01} Use effective written and other communication skills.  {\it Met by group discussion and completion of labwork.}
\item{\bf B.03} Employ appropriate skills for gathering and retaining information.  {\it Met by research and preparation prior to group discussion.}
\item{\bf B.04} Interpret written, graphic, and oral instructions.  {\it Met by completion of labwork.}
\item{\bf B.06} Use language appropriate to the situation.  {\it Met by group discussion and in explaining completed labwork.}
\item{\bf B.07} Participate in meetings in a positive and constructive manner.  {\it Met by group discussion.}
\item{\bf B.08} Use job-related terminology.  {\it Met by group discussion and in explaining completed labwork.}
\item{\bf B.10} Document work projects, procedures, tests, and equipment failures.  {\it Met by project construction and/or troubleshooting assessments.}
\item{\bf C} {\bf Basic and Practical Skills -- Solving Problems and Critical Thinking}
\item{\bf C.01} Identify the problem.  {\it Met by research and preparation prior to group discussion.}
\item{\bf C.03} Identify available solutions and their impact including evaluating credibility of information, and locating information.  {\it Met by research and preparation prior to group discussion.}
\item{\bf C.07} Organize personal workloads.  {\it Met by daily labwork, preparatory research, and project management.}
\item{\bf C.08} Participate in brainstorming sessions to generate new ideas and solve problems.  {\it Met by group discussion.}
\item{\bf D} {\bf Basic and Practical Skills -- Reading}
\item{\bf D.01} Read and apply various sources of technical information (e.g. manufacturer literature, codes, and regulations).  {\it Met by research and preparation prior to group discussion.}
\item{\bf E} {\bf Basic and Practical Skills -- Proficiency in Mathematics}
\item{\bf E.01} Determine if a solution is reasonable.
\item{\bf E.02} Demonstrate ability to use a simple electronic calculator.
\item{\bf E.05} Solve problems and [sic] make applications involving integers, fractions, decimals, percentages, and ratios using order of operations.
\item{\bf E.06} Translate written and/or verbal statements into mathematical expressions.
\item{\bf E.09} Read scale on measurement device(s) and make interpolations where appropriate.  {\it Met by oscilloscope usage.}
\item{\bf E.12} Interpret and use tables, charts, maps, and/or graphs.
\item{\bf E.13} Identify patterns, note trends, and/or draw conclusions from tables, charts, maps, and/or graphs.
\item{\bf E.15} Simplify and solve algebraic expressions and formulas.
\item{\bf E.16} Select and use formulas appropriately.
\item{\bf E.17} Understand and use scientific notation.
\item{\bf E.20} Graph functions.
\item{\bf E.26} Apply Pythagorean theorem.
\item{\bf E.27} Identify basic functions of sine, cosine, and tangent.
\item{\bf E.28} Compute and solve problems using basic trigonometric functions.
\medskip

%%%%%%%%%%%%%%%




\vfil \eject

\centerline{\bf ELTR 110 (AC 1), section 2} \bigskip 
 
\vskip 10pt

\noindent
{\bf Common areas of confusion for students}

\vskip 5pt

%%%%%%%%%%%%%%%
\hrule \vskip 5pt

\vskip 10pt

\noindent
{\bf Difficult concept: } {\it Phasors, used to represent AC amplitude and phase relations.}

A powerful tool used for understanding the operation of AC circuits is the {\it phasor diagram}, consisting of arrows pointing in different directions: the length of each arrow representing the amplitude of some AC quantity (voltage, current, or impedance), and the angle of each arrow representing the shift in phase relative to the other arrows.  By representing each AC quantity thusly, we may more easily calculate their relationships to one another, with the phasors showing us how to apply trigonometry (Pythagorean Theorem, sine, cosine, and tangent functions) to the various calculations.  An analytical parallel to the graphic tool of phasor diagrams is {\it complex numbers}, where we represent each phasor (arrow) by a pair of numbers: either a magnitude and angle (polar notation), or by "real" and "imaginary" magnitudes (rectangular notation).  Where phasor diagrams are helpful is in applications where their respective AC quantities {\it add}: the resultant of two or more phasors stacked tip-to-tail being the mathematical sum of the phasors.  Complex numbers, on the other hand, may be added, subtracted, multiplied, and divided; the last two operations being difficult to graphically represent with arrows.

\vskip 10pt

\noindent
{\bf Difficult concept: } {\it Conductance, susceptance, and admittance.}

Conductance, symbolized by the letter $G$, is the mathematical reciprocal of resistance ($1 \over R$).  Students typically encounter this quantity in their DC studies and quickly ignore it.  In AC calculations, however, conductance and its AC counterparts ({\it susceptance}, the reciprocal of reactance $B = {1 \over X}$ and {\it admittance}, the reciprocal of impedance $Y = {1 \over Z}$) are very necessary in order to draw phasor diagrams for parallel networks.

\vskip 10pt

\noindent
{\bf Difficult concept: } {\it Capacitance adding in parallel; capacitive reactance and impedance adding in series.}

When students first encounter capacitance, they are struck by how this quantity adds when capacitors are connected in {\it parallel}, not in series as it is for resistors and inductors.  They are surprised again, though, when they discover that the opposition to current offered by capacitors (either as scalar reactance or phasor impedance) adds in series just as resistance adds in series and inductive reactance/impedance adds in series.  Remember: {\it ohms} always add in series, no matter what their source(s); only {\it farads} add in parallel (omitting {\it siemens} or {\it mhos}, the units for conductance and admittance and susceptance, which of course also add in parallel).

\vskip 10pt

\noindent
{\bf Difficult concept: } {\it Identifying filter circuit types.}

Many students have a predisposition to memorization (as opposed to comprehension of concepts), and so when approaching filter circuits they try to identify the various types by memorizing the positions of reactive components.  As I like to tell my students, {\it memory will fail you}, and so a better approach is to develop analytical techniques by which you may determine circuit function based on "first principles" of circuits.  The approach I recommend begins by identifying component impedance (open or short) for very low and very high frequencies, respectively, then qualitatively analyzing voltage drops under those extreme conditions.  If a filter circuit outputs a strong voltage at low frequencies and a weak voltage at high frequencies then it must be a low-pass filter.  If it outputs a weak voltage at both low and high frequencies then it must be a band-pass filter, etc.

\vskip 10pt

\noindent
{\bf Difficult concept: } {\it The practical purpose(s) for filter circuits.}

Bode plots show how filter circuits respond to inputs of changing frequency, but this is not how filters are typically used in real applications.  Rarely does one find a filter circuit subjected to one particular frequency at a time -- usually a simultaneous mix of frequencies are seen at the input, and it is the filter's job to select a particular range of frequencies to pass through from that simultaneous mix.  Understanding the superposition theorem precedes an understanding of how filter circuits are practically used.


