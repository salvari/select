
%(BEGIN_QUESTION)
% Copyright 2003, Tony R. Kuphaldt, released under the Creative Commons Attribution License (v 1.0)
% This means you may do almost anything with this work of mine, so long as you give me proper credit

% Uncomment the following line if the question involves calculus at all:
\vbox{\hrule \hbox{\strut \vrule{} $\int f(x) \> dx$ \hskip 5pt {\sl Calculus alert!} \vrule} \hrule}

How is the parameter of {\it zener impedance} defined for a zener diode?  Should an ideal zener diode have a zener impedance figure equal to zero, or infinite?  Why?

\underbar{file 01065}
%(END_QUESTION)





%(BEGIN_ANSWER)

$Z_{zener} = {{\Delta E_{diode}} \over {\Delta I_{diode}}}$ \hskip 10pt or \hskip 10pt $Z_{zener} = {{d E_{diode}} \over {d I_{diode}}}$

\vskip 10pt

(The "d" is a calculus symbol, representing a change of infinitesimal magnitude.)

\vskip 10pt

Ideally, a zener diode will have a zener impedance of zero ohms.

%(END_ANSWER)





%(BEGIN_NOTES)

Ask your students to relate a diode's zener impedance to the slope of its characteristic curve.

%(END_NOTES)


