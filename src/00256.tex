
%(BEGIN_QUESTION)
% Copyright 2005, Tony R. Kuphaldt, released under the Creative Commons Attribution License (v 1.0)
% This means you may do almost anything with this work of mine, so long as you give me proper credit

% Uncomment the following line if the question involves calculus at all:
\vbox{\hrule \hbox{\strut \vrule{} $\int f(x) \> dx$ \hskip 5pt {\sl Calculus alert!} \vrule} \hrule}

The relationship between magnetic flux and induced voltage in a wire coil is expressed in this equation, known as {\it Faraday's Law}:

$$e = N {d \phi \over dt}$$

\noindent
Where,

$e =$ Instantaneous induced voltage, in volts

$N =$ Number of turns in wire coil

$\phi =$ Instantaneous magnetic flux, in webers

$t =$ Time, in seconds

\vskip 10pt

Explain what the mathematical expression ${d \phi \over dt}$ means, in light of what you know about electromagnetic induction.  Hint: the $d \over d$ notation is borrowed from calculus, and it is called the {\it derivative}.

Also, explain why lower-case letters are used ($e$ instead of $E$, $\phi$ instead of $\Phi$) in this equation.

\underbar{file 00256}
%(END_QUESTION)





%(BEGIN_ANSWER)

The mathematical expression ${d \phi \over dt}$ means "rate of change of magnetic flux over time."  In this particular example, the unit would be "webers per second."

The use of lower-case letters for variables indicates {\it instantaneous} values: that is, quantities expressed in terms of instantaneous moments of time.

\vskip 10pt

Follow-up question: manipulate this equation to solve for each variable (${d \phi \over dt} = \cdots$ ; $N = \cdots$).

%(END_ANSWER)





%(BEGIN_NOTES)

For students who have never studied calculus, this is an excellent opportunity to introduce the concept of the derivative, having already established the principle of induced voltage being related to how {\it quickly} magnetic flux changes over time.  In general physics studies, the quantities of position, velocity, and acceleration are similarly used to introduce the concept of the time-derivative, and later, the time-integral.  In electricity, though, we have our own unique applications!

%INDEX% Electromagnetic induction
%INDEX% Faraday's Law

%(END_NOTES)


