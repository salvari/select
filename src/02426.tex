
%(BEGIN_QUESTION)
% Copyright 2005, Tony R. Kuphaldt, released under the Creative Commons Attribution License (v 1.0)
% This means you may do almost anything with this work of mine, so long as you give me proper credit

Junction field-effect transistors are very sensitive devices, requiring practically zero current to "drive" them into either cutoff or saturation.  However, they usually cannot handle high drain currents -- in other words, they are not considered "power" switching devices.

If we combine a JFET with a BJT, though, we may realize the best features of each transistor: low drive current requirements combined with a high controlled current rating.  Examine the following hybrid JFET/BJT circuits, and explain how each one works to control power to the load:

$$\epsfbox{02426x01.eps}$$

Determine for each circuit whether the load becomes energized when the switch is {\it closed} or when it is {\it opened}, and explain how each one works.

\underbar{file 02426}
%(END_QUESTION)





%(BEGIN_ANSWER)

In each case, the load de-energizes with switch closure, and energizes when the switch is opened.

\vskip 10pt

Follow-up question: explain the purpose of the resistor in each circuit.  What might happen if it were not there?

%(END_ANSWER)





%(BEGIN_NOTES)

This question is a good review of both BJT and JFET operating theory, as well as a practical example of how "cascading" different types of transistors may result in "best of both worlds" performance.

%INDEX% Transistor switch circuit (JFET)

%(END_NOTES)


