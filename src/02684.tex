
%(BEGIN_QUESTION)
% Copyright 2004, Tony R. Kuphaldt, released under the Creative Commons Attribution License (v 1.0)
% This means you may do almost anything with this work of mine, so long as you give me proper credit

Examine this progression of mathematical statements:

$${1000 \over 100} = 10$$

$${1000 \over 100} = 10^1$$

$$\log \left( {1000 \over 100} \right) = \log 10^1$$

$$\log 1000 - \log 100 = \log 10^1$$

$$\log 10^3 - \log 10^2 = \log 10^1$$

$$3 - 2 = 1$$

What began as a division problem ended up as a subtraction problem, through the application of logarithms.  What does this tell you about the utility of logarithms as an arithmetic tool?

\underbar{file 02684}
%(END_QUESTION)





%(BEGIN_ANSWER)

That logarithms can reduce the complexity of an equation from division, down to subtraction, indicates its usefulness as a tool to {\it simplify} arithmetic problems.  Specifically, the logarithm of a quotient is equal to the difference between the logarithms of the two numbers being divided.

%(END_ANSWER)





%(BEGIN_NOTES)

In mathematics, any procedure that reduces a complex type of problem into a simpler type of problem is called a {\it transform function}, and logarithms are one of the simplest types of transform functions in existence.

%INDEX% Logarithms, used to transform a division problem into a subtraction problem

%(END_NOTES)


