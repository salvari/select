
\centerline{\bf ELTR 120 (Semiconductors 1), section 2} \bigskip 
 
\vskip 10pt

\noindent
{\bf Recommended schedule}

\vskip 5pt

%%%%%%%%%%%%%%%
\hrule \vskip 5pt
\noindent
\underbar{Day 1}

\hskip 10pt Topics: {\it Bipolar junction transistor theory}
 
\hskip 10pt Questions: {\it 1 through 15}
 
\hskip 10pt Lab Exercise: {\it BJT terminal identification (question 76)}
 
%INSTRUCTOR \hskip 10pt {\bf Demo: show that $I_C$ is nearly independent of $V_{CE}$ for a BJT}

\vskip 10pt
%%%%%%%%%%%%%%%
\hrule \vskip 5pt
\noindent
\underbar{Day 2}

\hskip 10pt Topics: {\it Bipolar junction transistor switching circuits}
 
\hskip 10pt Questions: {\it 16 through 30}
 
\hskip 10pt Lab Exercise: {\it BJT switch circuit (question 77)}
 
\vskip 10pt
%%%%%%%%%%%%%%%
\hrule \vskip 5pt
\noindent
\underbar{Day 3}

\hskip 10pt Topics: {\it Junction field-effect transistor (JFET) theory}
 
\hskip 10pt Questions: {\it 31 through 45}
 
\hskip 10pt Lab Exercise: {\it JFET switch circuit (question 78)}
 
\vskip 10pt
%%%%%%%%%%%%%%%
\hrule \vskip 5pt
\noindent
\underbar{Day 4}

\hskip 10pt Topics: {\it Insulated gate field-effect transistor (MOSFET) theory}
 
\hskip 10pt Questions: {\it 46 through 60}
 
\hskip 10pt Lab Exercise: {\it Work on project}
 
%INSTRUCTOR \hskip 10pt {\bf MIT 6.002 video clip: Disk 1, Lecture 5; MOSFET V/I characteristic 44:33 to 45:41}

\vskip 10pt
%%%%%%%%%%%%%%%
\hrule \vskip 5pt
\noindent
\underbar{Day 5}

\hskip 10pt Topics: {\it Review}
 
\hskip 10pt Questions: {\it 61 through 75}
 
\hskip 10pt Lab Exercise: {\it Work on project}
 
\vskip 10pt
%%%%%%%%%%%%%%%
\hrule \vskip 5pt
\noindent
\underbar{Day 6}

\hskip 10pt Exam 2: {\it includes transistor switch circuit performance assessment}
 
\hskip 10pt Lab Exercise: {\it Work on project}

\vskip 10pt
%%%%%%%%%%%%%%%
\hrule \vskip 5pt
\noindent
\underbar{Troubleshooting practice problems}

\hskip 10pt Questions: {\it 80 through 89}
 
\vskip 10pt
%%%%%%%%%%%%%%%
\hrule \vskip 5pt
\noindent
\underbar{General concept practice and challenge problems}

\hskip 10pt Questions: {\it 90 through the end of the worksheet}
 
\vskip 10pt
%%%%%%%%%%%%%%%
\hrule \vskip 5pt
\noindent
\underbar{Impending deadlines}

\hskip 10pt {\bf Project due at end of ELTR120, Section 3}
 
\hskip 10pt Question 79: Sample project grading criteria
 
\vskip 10pt
%%%%%%%%%%%%%%%












\vfil \eject

\centerline{\bf ELTR 120 (Semiconductors 1), section 2} \bigskip 
 
\vskip 10pt

\noindent
{\bf Skill standards addressed by this course section}

\vskip 5pt

%%%%%%%%%%%%%%%
\hrule \vskip 10pt
\noindent
\underbar{EIA {\it Raising the Standard; Electronics Technician Skills for Today and Tomorrow}, June 1994}

\vskip 5pt

\medskip
\item{\bf D} {\bf Technical Skills -- Discrete Solid-State Devices}
\item{\bf D.03} Demonstrate an understanding of bipolar transistors.
\item{\bf D.04} Demonstrate an understanding of field effect transistors (FET's/MOSFET's).
\item{\bf E} {\bf Technical Skills -- Analog Circuits}
\item{\bf E.07} Understand principles and operations of linear power supplies and filters.
\item{\bf E.08} Fabricate and demonstrate linear power supplies and filters.
\item{\bf E.09} Troubleshoot and repair linear power supplies and filters.
\medskip

\vskip 5pt

\medskip
\item{\bf B} {\bf Basic and Practical Skills -- Communicating on the Job}
\item{\bf B.01} Use effective written and other communication skills.  {\it Met by group discussion and completion of labwork.}
\item{\bf B.03} Employ appropriate skills for gathering and retaining information.  {\it Met by research and preparation prior to group discussion.}
\item{\bf B.04} Interpret written, graphic, and oral instructions.  {\it Met by completion of labwork.}
\item{\bf B.06} Use language appropriate to the situation.  {\it Met by group discussion and in explaining completed labwork.}
\item{\bf B.07} Participate in meetings in a positive and constructive manner.  {\it Met by group discussion.}
\item{\bf B.08} Use job-related terminology.  {\it Met by group discussion and in explaining completed labwork.}
\item{\bf B.10} Document work projects, procedures, tests, and equipment failures.  {\it Met by project construction and/or troubleshooting assessments.}
\item{\bf C} {\bf Basic and Practical Skills -- Solving Problems and Critical Thinking}
\item{\bf C.01} Identify the problem.  {\it Met by research and preparation prior to group discussion.}
\item{\bf C.03} Identify available solutions and their impact including evaluating credibility of information, and locating information.  {\it Met by research and preparation prior to group discussion.}
\item{\bf C.07} Organize personal workloads.  {\it Met by daily labwork, preparatory research, and project management.}
\item{\bf C.08} Participate in brainstorming sessions to generate new ideas and solve problems.  {\it Met by group discussion.}
\item{\bf D} {\bf Basic and Practical Skills -- Reading}
\item{\bf D.01} Read and apply various sources of technical information (e.g. manufacturer literature, codes, and regulations).  {\it Met by research and preparation prior to group discussion.}
\item{\bf E} {\bf Basic and Practical Skills -- Proficiency in Mathematics}
\item{\bf E.01} Determine if a solution is reasonable.
\item{\bf E.02} Demonstrate ability to use a simple electronic calculator.
\item{\bf E.05} Solve problems and [sic] make applications involving integers, fractions, decimals, percentages, and ratios using order of operations.
\item{\bf E.06} Translate written and/or verbal statements into mathematical expressions.
\item{\bf E.09} Read scale on measurement device(s) and make interpolations where appropriate.  {\it Met by oscilloscope usage.}
\item{\bf E.12} Interpret and use tables, charts, maps, and/or graphs.
\item{\bf E.13} Identify patterns, note trends, and/or draw conclusions from tables, charts, maps, and/or graphs.
\item{\bf E.15} Simplify and solve algebraic expressions and formulas.
\item{\bf E.16} Select and use formulas appropriately.
\item{\bf E.17} Understand and use scientific notation.
\medskip

%%%%%%%%%%%%%%%




\vfil \eject

\centerline{\bf ELTR 120 (Semiconductors 1), section 2} \bigskip 
 
\vskip 10pt

\noindent
{\bf Common areas of confusion for students}

\vskip 5pt

%%%%%%%%%%%%%%%
\hrule \vskip 5pt

\vskip 10pt

\noindent
{\bf Difficult concept: } {\it Necessary conditions for transistor operation.}

It is vitally important for students to understand the conditions necessary for transistor operation, both for understanding how circuits work and for troubleshooting faulty circuits.  Bipolar junction transistors require a base current (in the proper direction) to conduct, and the collector-to-emitter voltage must be of the correct polarity to push a collector current in the proper direction as well.  Both currents join at the emitter terminal, making the emitter current the sum of the base and collector currents.  Field-effect transistors are not so picky about the direction of the controlled current, and they only require the correct gate voltage (no gate current) to establish conduction.  What makes this so confusing is that there are two types of bipolar transistors (NPN and PNP), two types of junction field-effect transistors (N-channel and P-channel), and four types of MOSFETs (E-type N-channel, E-type P-channel, D-type N-channel, and D-type P-channel).

\vskip 10pt

\noindent
{\bf Difficult concept: } {\it Current sourcing versus current sinking.}

It is very common in electronics work to refer to current-controlling devices as either {\it sourcing} current to a load or {\it sinking} current from a load.  This is an overt reference to conventional-flow notation, referring to whether the conventional flow moves {\it out} of the transistor from the positive power supply terminal to the load (sourcing), or whether the conventional flow moves {\it in} to the transistor from the load and then "down" to ground (sinking).  Some students grasp this concept readily, while others seem to struggle mightily with it.  It is something rather essential to understand, because this terminology is extensively used by electronics professionals and found in electronics literature.  The key detail distinguishing the two conditions is which power supply rail (either +V or Gnd) is {\it directly} connected to the current-controlling device.

