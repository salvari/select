
%(BEGIN_QUESTION)
% Copyright 2005, Tony R. Kuphaldt, released under the Creative Commons Attribution License (v 1.0)
% This means you may do almost anything with this work of mine, so long as you give me proper credit

Research the datasheet of an integrated arithmetic logic unit such as the 74AS181, and determine how its various modes of operation (addition, subtraction, comparison) are selected.

\underbar{file 02845}
%(END_QUESTION)





%(BEGIN_ANSWER)

This is a small research project I leave up to you!  Be sure to bring a copy of your IC datasheet to class for discussion!

\vskip 10pt

Follow-up question: an interesting feature of the 74AS181 is that it provides "arithmetic" functions as well as "logic" functions.  These two modes could also be referred to as "binary" and "boolean," respectively.  Explain what distinguishes these two operating modes from one another, and why they are classified differently.

%(END_ANSWER)





%(BEGIN_NOTES)

Although the 74181 ALU is a somewhat dated IC (in fact, some versions are obsolete as of this writing -- 2005), it stands as a simple example for students to learn from.  A circuit such as this provides a good example of the power of integration, as opposed to constructing a similar logic function from individual gates (not to mention discrete transistors!).

The follow-up question brings up a point many students are confused on: the distinction between binary (numerical) and boolean (bitwise) operations.  Binary is a place-weighted {\it numeration system}, used to symbolize real numbers using only two states per place.  Boolean is a {\it number system} characterized by having only two possible values.  Since both binary and boolean have something to do with two-valued quantities, many students believe the two to be interchangeable terms and concepts.  However, they are not, and an investigation of the two operating modes of this ALU highlights the distinction.

%INDEX% Arithmetic Logic Unit (ALU), model 74AS181

%(END_NOTES)


