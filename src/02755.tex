
%(BEGIN_QUESTION)
% Copyright 2005, Tony R. Kuphaldt, released under the Creative Commons Attribution License (v 1.0)
% This means you may do almost anything with this work of mine, so long as you give me proper credit

Digital logic circuitry makes use of discrete voltage levels: each "logic gate" sub-circuit inputs and outputs voltages that are either considered "high" or "low".  Define what both of these terms means in a digital logic circuit powered by 5 volts DC.

\underbar{file 02755}
%(END_QUESTION)





%(BEGIN_ANSWER)

"High" = (nearly) 5 volts between the gate input/output and ground.

\vskip 10pt

"Low" = (nearly) 0 volts between the gate input/output and ground.

%(END_ANSWER)





%(BEGIN_NOTES)

This is a very simple concept, but worthwhile to cover in its own question just to be sure no students misunderstand when the concept is later applied.

%INDEX% High logic level, defined in a general sense
%INDEX% Logic gate voltage levels, "high" and "low" defined
%INDEX% Low logic level, defined in a general sense

%(END_NOTES)


