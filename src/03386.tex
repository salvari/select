
%(BEGIN_QUESTION)
% Copyright 2005, Tony R. Kuphaldt, released under the Creative Commons Attribution License (v 1.0)
% This means you may do almost anything with this work of mine, so long as you give me proper credit

Examine the following specific resistance table for various metals:

% No blank lines allowed between lines of an \halign structure!
% I use comments (%) instead, so that TeX doesn't choke.

$$\vbox{\offinterlineskip
\halign{\strut
\vrule \quad\hfil # \ \hfil & 
\vrule \quad\hfil # \ \hfil & 
\vrule \quad\hfil # \ \hfil \vrule \cr
\noalign{\hrule}
%
Metal type & $\rho$ in $\Omega$ $\cdot$ cmil / ft @ 32$^{o}$F & $\rho$ in $\Omega$ $\cdot$ cmil / ft @ 75$^{o}$F \cr
%
\noalign{\hrule}
%
Zinc (very pure) & 34.595 & 37.957 \cr
%
\noalign{\hrule}
%
Tin (pure) & 78.489 & 86.748 \cr
%
\noalign{\hrule}
%
Copper (pure annealed) & 9.390 & 10.351 \cr
%
\noalign{\hrule}
%
Copper (hard-drawn) & 9.810 & 10.745 \cr
%
\noalign{\hrule}
%
Copper (annealed) & 9.590 & 10.505 \cr
%
\noalign{\hrule}
%
Platinum (pure) & 65.670 & 71.418 \cr
%
\noalign{\hrule}
%
Silver (pure annealed) & 8.831 & 9.674 \cr
%
\noalign{\hrule}
%
Nickel & 74.128 & 85.138 \cr
%
\noalign{\hrule}
%
Steel (wire) & 81.179 & 90.150 \cr
%
\noalign{\hrule}
%
Iron (approx. pure) & 54.529 & 62.643 \cr
%
\noalign{\hrule}
%
Gold (99.9 \% pure) & 13.216 & 14.404 \cr
%
\noalign{\hrule}
%
Aluminum (99.5 \% pure) & 15.219 & 16.758 \cr
%
\noalign{\hrule}
} % End of \halign 
}$$ % End of \vbox

Of the metals shown, which is the best conductor of electricity?  Which is the worst?  What do you notice about the resistivity of these metals as temperature is increased from 32$^{o}$F to 75 $^{o}$F?

\underbar{file 03386}
%(END_QUESTION)





%(BEGIN_ANSWER)

Here is the same table, the order re-arranged to show resistivity from least to greatest:

% No blank lines allowed between lines of an \halign structure!
% I use comments (%) instead, so that TeX doesn't choke.

$$\vbox{\offinterlineskip
\halign{\strut
\vrule \quad\hfil # \ \hfil & 
\vrule \quad\hfil # \ \hfil & 
\vrule \quad\hfil # \ \hfil \vrule \cr
\noalign{\hrule}
%
Metal type & $\rho$ in $\Omega$ $\cdot$ cmil / ft @ 32$^{o}$F & $\rho$ in $\Omega$ $\cdot$ cmil / ft @ 75$^{o}$F \cr
%
\noalign{\hrule}
%
Silver (pure annealed) & 8.831 & 9.674 \cr
%
\noalign{\hrule}
%
Copper (pure annealed) & 9.390 & 10.351 \cr
%
\noalign{\hrule}
%
Copper (annealed) & 9.590 & 10.505 \cr
%
\noalign{\hrule}
%
Copper (hard-drawn) & 9.810 & 10.745 \cr
%
\noalign{\hrule}
%
Gold (99.9 \% pure) & 13.216 & 14.404 \cr
%
\noalign{\hrule}
%
Aluminum (99.5 \% pure) & 15.219 & 16.758 \cr
%
\noalign{\hrule}
%
Zinc (very pure) & 34.595 & 37.957 \cr
%
\noalign{\hrule}
%
Iron (approx. pure) & 54.529 & 62.643 \cr
%
\noalign{\hrule}
%
Platinum (pure) & 65.670 & 71.418 \cr
%
\noalign{\hrule}
%
Nickel & 74.128 & 85.138 \cr
%
\noalign{\hrule}
%
Tin (pure) & 78.489 & 86.748 \cr
%
\noalign{\hrule}
%
Steel (wire) & 81.179 & 90.150 \cr
%
\noalign{\hrule}
} % End of \halign 
}$$ % End of \vbox

%(END_ANSWER)





%(BEGIN_NOTES)

The data for this table was taken from table 1-97 of the {\it American Electrician's Handbook} (eleventh edition) by Terrell Croft and Wilford Summers.

It may come as a surprise to some students to find that gold is actually a worse conductor of electricity than copper, but the data doesn't lie!  {\it Silver} is actually the best, but gold is chosen for a lot of microelectronic applications because of its resistance to oxidation.

%INDEX% Specific resistance
%INDEX% Table, specific resistance of different metals

%(END_NOTES)


