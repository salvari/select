
%(BEGIN_QUESTION)
% Copyright 2005, Tony R. Kuphaldt, released under the Creative Commons Attribution License (v 1.0)
% This means you may do almost anything with this work of mine, so long as you give me proper credit

A common problem encountered in the development of transistor amplifier circuits is unwanted oscillation resulting from parasitic capacitance and inductance forming a positive feedback loop from output to input.  Often, these parasitic parameters are quite small (nanohenrys and picofarads), resulting in very high oscillation frequencies.

Another parasitic effect in transistor amplifier circuits is {\it Miller-effect} capacitance between the transistor terminals.  For common-emitter circuits, the base-collector capacitance ($C_{BC}$) is especially troublesome because it introduces a feedback path for AC signals to travel directly from the output (collector terminal) to the input (base terminal).

Does this parasitic base-to-collector capacitance encourage or discourage high-frequency oscillations in a common-emitter amplifier circuit?  Explain your answer.

\underbar{file 02601}
%(END_QUESTION)





%(BEGIN_ANSWER)

The presence of $C_{BC}$ in a common-emitter circuit mitigates high-frequency oscillations.

%(END_ANSWER)





%(BEGIN_NOTES)

Note that I chose to use a the word "mitigate" instead of give the answer in more plain English.  Part of my reasoning here is to veil the given answer from immediate comprehension so that students must {\it think} a bit more.  Another part of my reasoning is to force students' vocabularies to expand.

%INDEX% Miller effect, impact on parasitic oscillations

%(END_NOTES)


