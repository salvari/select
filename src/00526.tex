
%(BEGIN_QUESTION)
% Copyright 2003, Tony R. Kuphaldt, released under the Creative Commons Attribution License (v 1.0)
% This means you may do almost anything with this work of mine, so long as you give me proper credit

If a voltmeter is to be used to directly measure the voltage of an unknown source, it must first be {\it calibrated} so as to ensure an accurate measurement:

$$\epsfbox{00526x01.eps}$$

What is the minimum number of points along the meter's range that it needs to be calibrated at, given the assumption of perfectly linear response?

\vskip 10pt

If a voltmeter is to be used to measure the voltage of an unknown source, as a {\it differential} indicator only, what is the minimum number of points along its range that it needs to be calibrated at?

$$\epsfbox{00526x02.eps}$$

\underbar{file 00526}
%(END_QUESTION)





%(BEGIN_ANSWER)

A perfectly linear voltmeter, used to measure voltage directly, needs to be calibrated at {\it two} points along its measurement range in order to ensure measurement accuracy.

\vskip 10pt

A voltmeter used for differential measurement need only be calibrated at a single point along its range, and that single point is zero.

%(END_ANSWER)





%(BEGIN_NOTES)

Ask your students if the two calibration points on the meter's range should be close together, or far apart, for the best (most comprehensive) calibration possible.

Discuss the easier calibration requirements of the differential meter.  Challenge your students with this question: "Does the measurement linearity of a voltmeter matter as much if it is used to make a differential measurement, as compared to if it is used to make a direct measurement?"  Why or why not?

%(END_NOTES)


