
%(BEGIN_QUESTION)
% Copyright 2003, Tony R. Kuphaldt, released under the Creative Commons Attribution License (v 1.0)
% This means you may do almost anything with this work of mine, so long as you give me proper credit

$$\epsfbox{01674x01.eps}$$

\underbar{file 01674}
\vfil \eject
%(END_QUESTION)





%(BEGIN_ANSWER)

You may use circuit simulation software to set up similar oscilloscope display interpretation scenarios, for practice or for verification of what you see in this exercise.

%(END_ANSWER)





%(BEGIN_NOTES)

Use a sine-wave function generator for the AC voltage source, and be sure set the frequency to some reasonable value (well within the capability of both the oscilloscope and counter to measure).

If this is not the first time students have done this, be sure to "mess up" the oscilloscope controls prior to them making adjustments.  Students {\it must} learn how to quickly configure an oscilloscope's controls to display any arbitrary waveform, if they are to be proficient in using an oscilloscope as a diagnostic tool.

%INDEX% Assessment, performance-based (Setting up analog oscilloscope to display a sine wave)

%(END_NOTES)


