
%(BEGIN_QUESTION)
% Copyright 2003, Tony R. Kuphaldt, released under the Creative Commons Attribution License (v 1.0)
% This means you may do almost anything with this work of mine, so long as you give me proper credit

In a {\it shunt-wound} DC generator, the output voltage is determined by the rotational speed of the armature and the density of the stationary magnetic field flux.  For a given armature speed, what prevents the output voltage from "running away" to infinite levels, since the output voltage energizes the field winding, which leads to greater field flux, which leads to greater output voltage, which leads to greater field flux, which leads to . . . ?

$$\epsfbox{00814x01.eps}$$

Obviously, there must be some inherent limit to this otherwise vicious cycle.  Otherwise, the output voltage of a shunt-wound DC generator would be completely unstable.

\underbar{file 00814}
%(END_QUESTION)





%(BEGIN_ANSWER)

At a certain amount of field winding current, the generator's field poles {\it saturate}, preventing further increases in magnetic flux.

%(END_ANSWER)





%(BEGIN_NOTES)

This question provides a great opportunity to review the concept of magnetic "saturation," as well as introduce the engineering concept of {\it positive feedback}.

%INDEX% Saturation, magnetic (in the context of shunt-wound DC generators)

%(END_NOTES)


