
%(BEGIN_QUESTION)
% Copyright 2003, Tony R. Kuphaldt, released under the Creative Commons Attribution License (v 1.0)
% This means you may do almost anything with this work of mine, so long as you give me proper credit

One of the more complicated controls to master on an oscilloscope, but also one of the most useful, is the {\it triggering} control.  Without proper "triggering," a waveform will scroll horizontally across the screen rather than staying "locked" in place.

Describe how the triggering control is able to "lock" an AC waveform on the screen so that it appears stable to the human eye.  What, exactly, is the triggering function doing that makes an AC waveform appear to stand still?

\underbar{file 00537}
%(END_QUESTION)





%(BEGIN_ANSWER)

The triggering circuit inside the oscilloscope delays the initiation of a beam "sweep" across the screen until the instantaneous voltage value of the waveform has reached the same point, every time, on the wave-shape.

%(END_ANSWER)





%(BEGIN_NOTES)

For students who have every used a "strobe" or "timing" light to make a rotating object appear to "freeze" in place, the concept of oscilloscope triggering makes perfect sense.  In fact, a strobe light and a rotating object such as a fan work very well to illustrate the concept of having to "flash" at just the right times in order to make something moving appear to be still.

An interesting comparison to make is between a strobe light (freezing the motion of a fan) set to a frequency that is slightly "off" sync -- thereby causing the rotating object to appear to move {\it very slowly} -- and an oscilloscope with the triggering turned off, and the horizontal sweep speed set in the same manner, adjusted to make the AC waveform horizontally scroll across the screen.

Once your students have seen this comparison, ask them to describe what would be necessary to "trigger" a strobe light so that the moving object always appears to stand still, and cannot "scroll" due to a mismatch in frequencies.

%INDEX% Oscilloscope, trigger controls (conceptual)

%(END_NOTES)


