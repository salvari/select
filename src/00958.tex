
%(BEGIN_QUESTION)
% Copyright 2003, Tony R. Kuphaldt, released under the Creative Commons Attribution License (v 1.0)
% This means you may do almost anything with this work of mine, so long as you give me proper credit

Explain the effects of increasing R3's resistance in this amplifier circuit.  As R3 becomes more resistive, will the input signal ($V_{in}$) have more or less effect on the output voltage ($V_{out}$) than before?  Express your answer in terms of voltage gain ($A_V$).

$$\epsfbox{00958x01.eps}$$

\underbar{file 00958}
%(END_QUESTION)





%(BEGIN_ANSWER)

As resistor R3 increases in value, the voltage gain of the amplifier will decrease.

%(END_ANSWER)





%(BEGIN_NOTES)

While the answer to this question will be obvious to some, it will not be obvious to all.  Ask those students who do understand the answer to explain -- using their own words -- why the voltage gain decreases as R3's resistance increases.

Ask your students to imagine two scenarios of extreme resistance change for R3: shorted and open.  Qualitative analysis of the circuit should be rather easy given these extreme conditions!  Then, ask your students to relate the results of these hypothetical scenarios with a simple increase in resistance.  Is there a general problem-solving technique at work here?  Challenge your students to explain how this technique works.

%INDEX% Voltage gain, effects of resistor change on

%(END_NOTES)


