
%(BEGIN_QUESTION)
% Copyright 2003, Tony R. Kuphaldt, released under the Creative Commons Attribution License (v 1.0)
% This means you may do almost anything with this work of mine, so long as you give me proper credit

Observe the following "5-band" precision resistors, their color codes, and corresponding resistance values (note that the last color band is omitted, since it deals with precision and not nominal value):

$$\epsfbox{00265x01.eps}$$

What patterns do you notice between the color codes (given as three-letter abbreviations, so as to avoid interpretational errors resulting from variations in print quality) and the resistance values of each resistor?  Why do precision resistors use a "5-band" color code instead of a "4-band" color code?
 
\underbar{file 00265}
%(END_QUESTION)





%(BEGIN_ANSWER)

The first three color "bands" for precision five-band resistors denote three digits and a "multiplier" value, respectively.  A five-band color code is necessary to express resistance with a greater number of significant digits than a four-band code.

%(END_ANSWER)





%(BEGIN_NOTES)

The normal way to teach students the resistor color code is to show them the code first, then show them some resistors.  Here, the sequence is reversed: show the students some resistors, and have them figure out the code.  An important cognitive skill is the ability to detect and apply patterns in sets of data.  Exercises such as this help build that skill.

It should be noted that there is a 5-band color code for {\it non-precision} resistors as well, with the first four bands serving the same purpose as in a 4-band code, the extra band indicating resistor {\it reliability}.  This scheme was developed for military purposes and is seldom seen in civilian circuitry.

%INDEX% Color code, resistor (5-band)

%(END_NOTES)


