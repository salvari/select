
%(BEGIN_QUESTION)
% Copyright 2003, Tony R. Kuphaldt, released under the Creative Commons Attribution License (v 1.0)
% This means you may do almost anything with this work of mine, so long as you give me proper credit

As the armature coils in a DC motor rotate through the magnetic flux lines produced by the stationary field poles, voltage will be induced in those coils.  Describe how this phenomenon relates to Faraday's Law of electromagnetic induction, specifically in regard to what variables influence the magnitude of the induced voltage:

$$e = N{d\phi \over dt}$$

The self-induced voltage produced by a rotating armature is often called the {\it counter-voltage}, or {\it counter-EMF}.  Why would it be called "counter"?  What is implied by this terminology, and what electromagnetic principle is illustrated by the "counter" nature of this induced voltage?

\underbar{file 00397}
%(END_QUESTION)





%(BEGIN_ANSWER)

Counter-EMF varies directly with armature speed, with the number of turns in the armature windings, and also with field strength.  It is called "counter-" EMF because of Lenz' Law: the induced effect opposes the cause.

%(END_ANSWER)





%(BEGIN_NOTES)

The principle I wish to communicate most with this problem is that every motor, when operating, also acts as a generator (producing counter-EMF).  This concept is essential to understanding electric motor behavior, especially torque/speed curves.

%INDEX% Counter-EMF, electric motor

%(END_NOTES)


