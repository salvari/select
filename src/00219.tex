
%(BEGIN_QUESTION)
% Copyright 2003, Tony R. Kuphaldt, released under the Creative Commons Attribution License (v 1.0)
% This means you may do almost anything with this work of mine, so long as you give me proper credit

If the terminals of a DC voltage source are connected to individual wires, dipped in a container full of water, an interesting effect takes place on the submerged surfaces of the wires.  What is this phenomenon, and what is it called?

$$\epsfbox{00219x01.eps}$$

\underbar{file 00219}
%(END_QUESTION)





%(BEGIN_ANSWER)

I'll let you discover the actual phenomenon on your own (be careful to use no more than 12 volts when performing an experiment like this, for safety!), but its name is {\it electrolysis}.

%(END_ANSWER)





%(BEGIN_NOTES)

Electrochemical effects are very commonly used in industry to form chemical substances.  Ironically, the battery itself shown in the illustration, is another example of electrochemistry in action, just hidden from sight!

%INDEX% Electrolysis, conceptual

%(END_NOTES)


