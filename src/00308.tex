
%(BEGIN_QUESTION)
% Copyright 2003, Tony R. Kuphaldt, released under the Creative Commons Attribution License (v 1.0)
% This means you may do almost anything with this work of mine, so long as you give me proper credit

A technician picks up a resistor with the following color bands:

\vskip 10pt

\noindent
Color code: {\bf Brn, Blk, Sil, Gld}

\vskip 10pt

Having forgotten the resistor color code, and being too lazy to research the color code in a book, the technician decides to simply measure its resistance with an ohmmeter.  The value this technician obtains is 0.6 $\Omega$.

What is wrong with the technician's measurement?

\underbar{file 00308}
%(END_QUESTION)





%(BEGIN_ANSWER)

The resistance measurement is too high (it should be closer to 0.1 $\Omega$) because the ohmmeter was not properly set up to compensate for test lead resistance.

%(END_ANSWER)





%(BEGIN_NOTES)

Analog ohmmeters had to be "zeroed" prior to almost every measurement out of necessity, to give accurate measurements despite changes in internal battery voltage.  Modern digital ohmmeters do not have this problem, but the lack of a "zeroing" procedure can lead to errors due to uncompensated test lead resistance.  Your students should research the proper use of their own digital meters in this regard.

%INDEX% Ohmmeter usage
%INDEX% Color code, resistor

%(END_NOTES)


