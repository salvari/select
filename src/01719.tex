
%(BEGIN_QUESTION)
% Copyright 2003, Tony R. Kuphaldt, released under the Creative Commons Attribution License (v 1.0)
% This means you may do almost anything with this work of mine, so long as you give me proper credit

How much voltage does the light bulb receive in this circuit?  Explain your answer.

$$\epsfbox{01719x01.eps}$$

Also, identify the polarity of the voltage across the light bulb (mark with "+" and "-" signs).

\underbar{file 01719}
%(END_QUESTION)





%(BEGIN_ANSWER)

$$\epsfbox{01719x02.eps}$$

\vskip 10pt

Follow-up question: draw the direction of current in this circuit.

%(END_ANSWER)





%(BEGIN_NOTES)

This is a very fundamental concept that students must learn: how to determine the total voltage in a series circuit where opposing voltage sources exist.  One thing mistake students sometimes make is to try to discern polarity by looking at the polarity signs at the end terminals of the end battery; i.e. at the 3-volt battery's left-hand terminal, and the 4.5-volt battery's right-hand terminal, then try to transfer those signs down to the load terminals.  This is {\it not} an accurate way to tell polarity, but it seems to "work" for them in some situations.  This problem is one example of a situation where this faulty technique most definitely does not work!

Have your students collectively agree on a procedure they may use to accurate discern series voltage sums and polarities.  Guide their discussion, helping them identify principles that are true and valid for all series circuits.

%INDEX% Series voltages
%INDEX% Voltages in series

%(END_NOTES)


