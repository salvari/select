
%(BEGIN_QUESTION)
% Copyright 2005, Tony R. Kuphaldt, released under the Creative Commons Attribution License (v 1.0)
% This means you may do almost anything with this work of mine, so long as you give me proper credit

$$\epsfbox{02766x01.eps}$$

\underbar{file 02766}
\vfil \eject
%(END_QUESTION)





%(BEGIN_ANSWER)

Use circuit simulation software to verify your predicted and measured parameter values.

%(END_ANSWER)





%(BEGIN_NOTES)

I recommend setting the function generator output for 1 volt, to make it easier for students to measure the point of "cutoff".  You may set it at some other value, though, if you so choose (or let students set the value themselves when they test the circuit!).

I also recommend having students use an oscilloscope to measure AC voltage in a circuit such as this, because some digital multimeters have difficulty accurately measuring AC voltage much beyond line frequency range.  I find it particularly helpful to set the oscilloscope to the "X-Y" mode so that it draws a thin line on the screen rather than sweeps across the screen to show an actual waveform.  This makes it easier to measure peak-to-peak voltage.

Be sure to choose component values that will yield a frequency well within the range that the specified opamp can handle!  It would be foolish, for example, to specify a cutoff frequency in the megahertz range if the particular opamp being used was an LM741.

%INDEX% Assessment, performance-based (First order active lowpass filter)

%(END_NOTES)


