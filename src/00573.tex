
%(BEGIN_QUESTION)
% Copyright 2003, Tony R. Kuphaldt, released under the Creative Commons Attribution License (v 1.0)
% This means you may do almost anything with this work of mine, so long as you give me proper credit

When securing equipment for safe maintenance, special {\it tags} are attached with the lock(s) used to keep circuit breakers and other disconnect devices in the open (off) state.  A typical "lockout" tag looks something like this:

$$\epsfbox{00573x01.eps}$$

What is the purpose of attaching such a tag to an electrical disconnect device in addition to locking it in the open position?  Why is a lock, by itself, not sufficient from a safety perspective?

\underbar{file 00573}
%(END_QUESTION)





%(BEGIN_ANSWER)

A tag informs anyone wishing to turn the disconnect device "on" as to {\it when} it was turned off, and {\it who} placed the lock(s) on it.  Many lockout tags have space on for a written description so that the reason for the lockout may be explained.

%(END_ANSWER)





%(BEGIN_NOTES)

Discuss with your students the need for good communication between all people performing maintenance work on large and (potentially) dangerous systems.  Tags are an integral part of this communication.

%INDEX% Lock-out / tag-out

%(END_NOTES)


