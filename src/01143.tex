
%(BEGIN_QUESTION)
% Copyright 2003, Tony R. Kuphaldt, released under the Creative Commons Attribution License (v 1.0)
% This means you may do almost anything with this work of mine, so long as you give me proper credit

A block diagram of a {\it phase-locked loop} circuit looks like this:

$$\epsfbox{01143x01.eps}$$

Determine what type of electronic signals would be seen at points A and B for the following input conditions:

\medskip
\item{$\bullet$} Input = sine wave, steady frequency
\item{$\bullet$} Input = sine wave, increasing frequency
\item{$\bullet$} Input = sine wave, decreasing frequency
\item{$\bullet$} Input = sine wave, frequency increases and decreases regularly
\medskip

\underbar{file 01143}
%(END_QUESTION)





%(BEGIN_ANSWER)

\medskip

\item{$\bullet$} Input = sine wave, steady frequency
\item{ -- } {\bf A}: Steady DC voltage 
\item{ -- } {\bf B}: frequency equal to input signal

\vskip 10pt

\item{$\bullet$} Input = sine wave, increasing frequency
\item{ -- } {\bf A}: Increasing DC voltage
\item{ -- } {\bf B}: frequency equal to input signal

\vskip 10pt

\item{$\bullet$} Input = sine wave, decreasing frequency
\item{ -- } {\bf A}: Decreasing DC voltage
\item{ -- } {\bf B}: frequency equal to input signal

\vskip 10pt

\item{$\bullet$} Input = sine wave, frequency increases and decreases regularly
\item{ -- } {\bf A}: DC voltage that rises and falls with input frequency
\item{ -- } {\bf B}: frequency equal to input signal

\medskip

%(END_ANSWER)





%(BEGIN_NOTES)

The purpose of this question is to get students to recognize the function of each "block" in a phase-locked loop.  Having them predict the types of output signals at points A and B for different input signal conditions reveals whether or not they understand the concept.

Ask them where they obtained their information on phase-locked loop operation.  I've seen quite a few tutorials on the internet for this subject, so there should be no problem with students finding sources.

%(END_NOTES)


