
%(BEGIN_QUESTION)
% Copyright 2003, Tony R. Kuphaldt, released under the Creative Commons Attribution License (v 1.0)
% This means you may do almost anything with this work of mine, so long as you give me proper credit

$$\epsfbox{03456x01.eps}$$

\underbar{file 03456}
\vfil \eject
%(END_QUESTION)





%(BEGIN_ANSWER)

Use circuit simulation software to verify your predicted and measured parameter values.

%(END_ANSWER)





%(BEGIN_NOTES)

You will need an AC signal source of variable voltage, so that the reference voltage at test point {\bf A} ($V_A$) may be precisely adjusted to 1 volt RMS.  The frequency of the signal source must be within the range of the voltmeter to measure, of course.  You may use a sine-wave signal generator as the source, but a step-down transformer and series potentiometer works just as well!  The purpose of this exercise is not how to calculate voltage divider outputs, but rather how to utilize a dB-reading meter to measure signal strength in decibels rather than volts AC.

Most high-quality digital multimeters will measure dBV in addition to AC volts RMS.  Some have a "zero" or "relative" button which acts like the "tare" button on an electronic weigh scale.  With such a meter, one may set 0 dB to reference at {\it any} measurable level of AC voltage.  Of course, this means the meter will be measuring dB instead of dBV, since the reference point is made arbitrary with the press of the "relative" button.  However, this is a very useful tool for electronic signal measurements, where source strength may be set to 0 dB and then losses measured directly in -dB (relative to the source) just as easily as normal voltage measurements.

%INDEX% Assessment, performance-based (Decibel measurement of AC signals)

%(END_NOTES)


