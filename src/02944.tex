
%(BEGIN_QUESTION)
% Copyright 2005, Tony R. Kuphaldt, released under the Creative Commons Attribution License (v 1.0)
% This means you may do almost anything with this work of mine, so long as you give me proper credit

A {\it sequential} timer circuit may be constructed from multiple 555 timer ICs cascaded together.  Examine this circuit and determine how it works:

$$\epsfbox{02944x01.eps}$$

Can you think of any practical applications for a circuit such as this?

\underbar{file 02944}
%(END_QUESTION)





%(BEGIN_ANSWER)

Each 555 timer's cycle is triggered by the negative edge of the pulse on the {\it trigger} terminal.  A passive differentiator network between each 555 timer ensures that only a brief negative-going pulse is sent to the trigger terminal of the next timer from the output terminal of the one before it.

\vskip 10pt

Follow-up question: when timer circuits are cascaded like this, do their time delays {\it add} or {\it multiply} to make the total delay time?  Be sure to explain your reasoning.

%(END_ANSWER)





%(BEGIN_NOTES)

Practical applications abound for such a circuit.  One whimsical application is to energize sequential tail-light bulbs for an automobile, to give an interesting turn-signal visual effect.  A sequential timer circuit was used to do just this on certain years of (classic) Ford Cougar cars.  Other, more utilitarian, applications for sequential timers include start-up sequences for a variety of electronic systems, traffic light controls, and automated household appliances.

%INDEX% 555 timer, monostable operation
%INDEX% Sequential timer, using multiple 555 IC's

%(END_NOTES)


