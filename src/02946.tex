
%(BEGIN_QUESTION)
% Copyright 2005, Tony R. Kuphaldt, released under the Creative Commons Attribution License (v 1.0)
% This means you may do almost anything with this work of mine, so long as you give me proper credit

The following expression is frequently used to calculate values of changing variables (voltage and current) in RC and LR timing circuits:

$$e^{-{t \over \tau}} \hbox{\hskip 30pt or \hskip 30pt} {1 \over {e^{t \over \tau}}}$$

If we evaluate this expression for a time of $t = 0$, we find that it is equal to 1 (100\%).  If we evaluate this expression for increasingly larger values of time ($t \to \infty$), we find that it approaches 0 (0\%).

Based on this simple analysis, would you say that the expression $e^{-{t \over \tau}}$ describes the percentage that a variable has changed from its initial value in a timing circuit, or the percentage that it has {\it left} to change before it reaches its final value?  To frame this question in graphical terms . . .

$$\epsfbox{02946x01.eps}$$

Which percentage does the expression $e^{-{t \over \tau}}$ represent in each case?  Explain your answer.

\underbar{file 02946}
%(END_QUESTION)





%(BEGIN_ANSWER)

Whether the variable in question is increasing or decreasing over time, the expression $e^{-{t \over \tau}}$ describes the percentage that a variable has left to change before it reaches its final value.

\vskip 10pt

Follow-up question: what could you add to or modify about the expression to make it describe the percentage that a variable has already changed from its initial value?  In other words, alter the expression so that it is equal to 0\% at $t = 0$ and approaches 100\% as $t$ grows larger ($t \to \infty$).

%(END_ANSWER)





%(BEGIN_NOTES)

It is very important for students to understand what this expression means and how it works, lest they rely solely on memorization to use it in their calculations.  As I always tell my students, rote memorization {\it will} fail you!  If a student does not comprehend why the expression works as it does, they will be helpless to retain it as an effective "tool" for performing calculations in the future.

A good way to suggest students approach a problem such as this is to imagine $t$ increasing in value.  As $t$ grows larger, what happens to the expression's overall value?  Then, compare which of the two percentages (percentage traversed, or percentage remaining) follow the same trend.  One not need touch a calculator to figure this out!

%INDEX% Time constant calculation, RC or LR circuit

%(END_NOTES)


