
%(BEGIN_QUESTION)
% Copyright 2005, Tony R. Kuphaldt, released under the Creative Commons Attribution License (v 1.0)
% This means you may do almost anything with this work of mine, so long as you give me proper credit

Suppose I set a potentiometer to the 30\% position, so that the wiper is 30\% of the way up from the bottom:

$$\epsfbox{03501x01.eps}$$

Describe the mathematical relationship between $V_{in}$ and $V_{out}$ with the potentiometer in this position.  What does this suggest about the usefulness of a potentiometer as an analog computational element?

\underbar{file 03501}
%(END_QUESTION)





%(BEGIN_ANSWER)

$V_{out} = (0.3)V_{in}$, for any value of $V_{in}$ which does not overpower the potentiometer.  In essence, the potentiometer is capable of functioning as an {\it analog computer} in that it outputs a signal that is a precise, controllable ratio of the input signal.

%(END_ANSWER)





%(BEGIN_NOTES)

The important idea in this question is for students to view a potentiometer as an {\it adjustable divider} which may be used to provide a {\it ratio} function in an analog circuit.  This may be your students' first step into the world of analog computational circuitry (using voltage and current values to represent numerical quantities), so be sure to emphasize the significance of this lowly potentiometer's function.  The very idea that such a simple device can actually be used to {\it perform a mathematical operation} is profound.

%INDEX% Potentiometer, as analog computational divider

%(END_NOTES)


