
%(BEGIN_QUESTION)
% Copyright 2005, Tony R. Kuphaldt, released under the Creative Commons Attribution License (v 1.0)
% This means you may do almost anything with this work of mine, so long as you give me proper credit

In the early days of solid-state logic gate circuit technology, there was a very clear distinction between TTL and CMOS.  TTL gates were capable of switching on and off very fast, required a tightly regulated power supply voltage, and used a lot of power.  CMOS gates were not quite as fast as TTL, but could tolerate a much wider range of power supply voltages and were far less wasteful on power.

Then, during the 1980's a new technology known as {\it high-speed CMOS}, or HCMOS, entered the scene.  Explain what HCMOS is, how it compares to the older TTL and CMOS families (54/74xx and 4xxx number series, respectively), and where it is often used.  Hint: high-speed CMOS bears the same numerical codes as the old TTL 54xx and 74xx series ICs (e.g. 74HC00 instead of 7400).

\underbar{file 02773}
%(END_QUESTION)





%(BEGIN_ANSWER)

I'll let you research the answer to this question!

%(END_ANSWER)





%(BEGIN_NOTES)

High-speed CMOS was a very important developmental milestone in digital logic gate technology, and it is essential for modern (2005) students of electronics to be aware of since it is so widely used.  In many ways it blends the best of the old TTL and CMOS worlds, with few disadvantages.

%INDEX% High-speed CMOS, defined

%(END_NOTES)


