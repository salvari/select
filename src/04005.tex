
%(BEGIN_QUESTION)
% Copyright 2006, Tony R. Kuphaldt, released under the Creative Commons Attribution License (v 1.0)
% This means you may do almost anything with this work of mine, so long as you give me proper credit

Suppose we were designing a pair of BJT amplifier circuits to connect to either end of a long two-conductor cable:

$$\epsfbox{04005x01.eps}$$

How would we choose the component values in each transistor amplifier circuit to naturally terminate both ends of the 75 $\Omega$ cable?

\underbar{file 04005}
%(END_QUESTION)





%(BEGIN_ANSWER)

$R_C$ of the transmitting amplifier should be 75 $\Omega$, as should the parallel equivalent resistance $R_{B1} || R_{B2}$ of the receiving amplifier.

%(END_ANSWER)





%(BEGIN_NOTES)

This question is really a review of Th\'evenin's theorem as it applies to common-emitter, divider-biased BJT amplifier circuits. 

In case anyone asks, the "zig-zags" in the four lines for the cable represent an unspecified distance between those points.  In other words, the cable is {\it longer} than what may be proportionately represented on the schematic diagram.

%INDEX% Transmission line driver amplifier, designing proper impedance for

%(END_NOTES)


