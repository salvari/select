
%(BEGIN_QUESTION)
% Copyright 2005, Tony R. Kuphaldt, released under the Creative Commons Attribution License (v 1.0)
% This means you may do almost anything with this work of mine, so long as you give me proper credit

Suppose you were handed a black box with two metal terminals on one side, for attaching electrical (wire) connections.  Inside this box, you were told, was a voltage source connected in series with a resistance.

$$\epsfbox{03227x01.eps}$$

Your task was to experimentally determine the values of the voltage source and the resistor inside the box, and you did just that.  From your experimental data you then sketched a circuit with the following component values:

$$\epsfbox{03227x02.eps}$$

However, you later discovered that you had been tricked.  Instead of containing a single voltage source and a single resistance, the circuit inside the box actually looked like this:

$$\epsfbox{03227x03.eps}$$

Demonstrate that these two different circuits are indistinguishable from the perspective of the two metal terminals, and explain what general principle this equivalence represents.

\underbar{file 03227}
%(END_QUESTION)





%(BEGIN_ANSWER)

A good way to demonstrate the electrical equivalence of these circuits is to calculate their responses to identical load resistor values.  The equivalence you see here is an application of {\it Th\'evenin's Theorem}.

%(END_ANSWER)





%(BEGIN_NOTES)

Ask your students to clearly state Th\'evenin's Theorem, and explain how it may be applied to the two-resistor circuit to obtain the one-resistor circuit.

%INDEX% Thevenin's theorem, experimentally determining Thevenin voltage and Thevenin resistance of network

%(END_NOTES)


