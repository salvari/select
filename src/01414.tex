
%(BEGIN_QUESTION)
% Copyright 2003, Tony R. Kuphaldt, released under the Creative Commons Attribution License (v 1.0)
% This means you may do almost anything with this work of mine, so long as you give me proper credit

The truth table shown here is for a 4-line to 16-line {\it binary decoder} circuit:

$$\epsfbox{01414x01.eps}$$

For each of the sixteen output lines, there is a Boolean SOP expression describing its function.  Just for example, write the Boolean expressions for output lines 5, 8, and 13.

\underbar{file 01414}
%(END_QUESTION)





%(BEGIN_ANSWER)

Output line 5: $A\overline{B}C\overline{D}$

Output line 8: $\overline{A} \> \overline{B} \> \overline{C} D$

Output line 13: $A\overline{B}CD$

\vskip 10pt

Follow-up question: based on what you see here, what kind of logic gate circuitry is a decoder such as this comprised of?  You don't have to actually draw a schematic diagram, but just generally describe the circuitry necessary to implement sixteen different SOP expressions.

%(END_ANSWER)





%(BEGIN_NOTES)

Nothing really complex or tricky here.  Just a straightforward application of Boolean SOP expressions.

%INDEX% Decoder, digital
%INDEX% Sum-of-products expression
%INDEX% SOP expression

%(END_NOTES)


