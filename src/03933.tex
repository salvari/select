
%(BEGIN_QUESTION)
% Copyright 2006, Tony R. Kuphaldt, released under the Creative Commons Attribution License (v 1.0)
% This means you may do almost anything with this work of mine, so long as you give me proper credit

$$\epsfbox{03933x01.eps}$$

\underbar{file 03933}
\vfil \eject
%(END_QUESTION)





%(BEGIN_ANSWER)

I do not provide a grading rubric here, but elsewhere.

%(END_ANSWER)





%(BEGIN_NOTES)

The idea of a troubleshooting log is three-fold.  First, it gets students in the habit of documenting their troubleshooting procedure and thought process.  This is a valuable habit to get into, as it translates to more efficient (and easier-followed) troubleshooting on the job.  Second, it provides a way to document student steps for the assessment process, making your job as an instructor easier.  Third, it reinforces the notion that each and every measurement or action should be followed by reflection (conclusion), making the troubleshooting process more efficient.

%INDEX% Assessment, performance-based (Troubleshooting log)

%(END_NOTES)


