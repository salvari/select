
%(BEGIN_QUESTION)
% Copyright 2005, Tony R. Kuphaldt, released under the Creative Commons Attribution License (v 1.0)
% This means you may do almost anything with this work of mine, so long as you give me proper credit

% Uncomment the following line if the question involves calculus at all:
\vbox{\hrule \hbox{\strut \vrule{} $\int f(x) \> dx$ \hskip 5pt {\sl Calculus alert!} \vrule} \hrule}

Plot the relative B-H curves for a sample of air and a sample of iron, in proportion to each other (as much as possible):

$$\epsfbox{03515x01.eps}$$

What do you notice about the slope (also called the derivative, or $dB \over dH$) of each plot?

\underbar{file 03515}
%(END_QUESTION)





%(BEGIN_ANSWER)

$$\epsfbox{03515x02.eps}$$

\vskip 10pt

Follow-up question: note that the slope of both plots are approximately equal toward the far right end of the graph.  Explain this effect in terms of magnetic {\it saturation}.

%(END_ANSWER)





%(BEGIN_NOTES)

The purpose of this question is twofold: to get students to see that a ferromagnetic material such as iron is {\it much} more permeable (less "reluctant") than air, but that the great gains in $B$ realized by iron tend to disappear as soon as saturation sets in.  Once the iron is saturated, the gains in $B$ for equal advances in $H$ are the same as for air.  That is, the $dB \over dH$ for iron is equal to the $dB \over dH $ for air once the iron is saturated.

%INDEX% B-H curve, for ferrous material versus air
%INDEX% Saturation, magnetic

%(END_NOTES)


