
%(BEGIN_QUESTION)
% Copyright 2003, Tony R. Kuphaldt, released under the Creative Commons Attribution License (v 1.0)
% This means you may do almost anything with this work of mine, so long as you give me proper credit

An electromechanical alternator (AC generator) and a DC-DC inverter both output the same RMS voltage, and deliver the same amount of electrical power to two identical loads:

$$\epsfbox{00404x01.eps}$$

However, when measured by an analog voltmeter, the inverter's output voltage is slightly greater than the alternator's output voltage.  Explain this discrepancy in measurements.

\underbar{file 00404}
%(END_QUESTION)





%(BEGIN_ANSWER)

Electromechanical alternators naturally output sinusoidal waveforms.  Many DC-AC inverters do not.

%(END_ANSWER)





%(BEGIN_NOTES)

Remember, most analog meter movement designs respond to the average value of a waveform, not its RMS value.  If the proportionality between a waveform's average and RMS values ever change, the relative indications of a true-RMS instrument and an average-based (calibrated to read RMS) instrument will change as well.

%INDEX% Inverter power output, non-sinusoidal
%INDEX% Meter response, peak vs. RMS vs. average

%(END_NOTES)


