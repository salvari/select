
%(BEGIN_QUESTION)
% Copyright 2003, Tony R. Kuphaldt, released under the Creative Commons Attribution License (v 1.0)
% This means you may do almost anything with this work of mine, so long as you give me proper credit

Describe the nature of the voltage induced in the stationary ("stator") windings, as the permanent magnet rotor rotates in this machine:

$$\epsfbox{00816x01.eps}$$

What factors determine the magnitude of this voltage?  According to Faraday's Law, what factors can we alter to increase the voltage output by this generator?

Is the induced voltage AC or DC?  How can you tell?

\underbar{file 00816}
%(END_QUESTION)





%(BEGIN_ANSWER)

Increase the ${d\phi \over dt}$ rate of change, or increase the number of turns in the stator winding, to increase the magnitude of the AC voltage generated by this machine.

\vskip 10pt

Follow-up question: AC generators, or {\it alternators} as they are sometimes called, are typically long-lived machines when operated under proper conditions.  But like all machines, they will eventually fail.  Based on the illustration given in the question, identify some probable modes of failure for an alternator, and what conditions might hasten such failures.

%(END_ANSWER)





%(BEGIN_NOTES)

Ask your students to write the equation for Faraday's Law on the whiteboard, and then analyze it in a qualitative sense (with variables increasing or decreasing in value) to validate the answers.

The first answer to this question (increase ${d\phi \over dt}$) has been left purposefully vague, in order to make students think.  What, specifically, must be changed in order to increase this rate-of-change over time?  Which real-world variables are changeable after the generator has been manufactured, and which are not?

%INDEX% Alternator, permanent magnet

%(END_NOTES)


