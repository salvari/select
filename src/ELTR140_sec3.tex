
\centerline{\bf ELTR 140 (Digital 1), section 3} \bigskip 
 
\vskip 10pt

\noindent
{\bf Recommended schedule}

\vskip 5pt

%%%%%%%%%%%%%%%
\hrule \vskip 5pt
\noindent
\underbar{Day 1}

\hskip 10pt Topics: {\it Numeration systems}
 
\hskip 10pt Questions: {\it 1 through 10}
 
\hskip 10pt Lab Exercise: {\it work on project}
 
\vskip 10pt
%%%%%%%%%%%%%%%
\hrule \vskip 5pt
\noindent
\underbar{Day 2}

\hskip 10pt Topics: {\it Digital codes}
 
\hskip 10pt Questions: {\it 11 through 20}
 
\hskip 10pt Lab Exercise: {\it Gray code to binary converter (question 56)}
 
\vskip 10pt
%%%%%%%%%%%%%%%
\hrule \vskip 5pt
\noindent
\underbar{Day 3}

\hskip 10pt Topics: {\it Binary arithmetic}
 
\hskip 10pt Questions: {\it 21 through 30}
 
\hskip 10pt Lab Exercise: {\it Half adder circuit (question 57)}
 
\vskip 10pt
%%%%%%%%%%%%%%%
\hrule \vskip 5pt
\noindent
\underbar{Day 4}

\hskip 10pt Topics: {\it Binary arithmetic circuits}
 
\hskip 10pt Questions: {\it 31 through 40}
 
\hskip 10pt Lab Exercise: {\it Full adder circuit (question 58)}
 
\vskip 10pt
%%%%%%%%%%%%%%%
\hrule \vskip 5pt
\noindent
\underbar{Day 5}

\hskip 10pt Topics: {\it Digital circuit troubleshooting}
 
\hskip 10pt Questions: {\it 41 through 55}
 
\hskip 10pt Lab Exercise: {\it Analog-digital converter IC (question 59)}
 
\vskip 10pt
%%%%%%%%%%%%%%%
\hrule \vskip 5pt
\noindent
\underbar{Day 6}

\hskip 10pt Exam 3: {\it includes binary adder circuit performance assessment}
 
\hskip 10pt {\bf Project due}

\hskip 10pt Question 60: Sample project grading criteria
 
\vskip 10pt
%%%%%%%%%%%%%%%
\hrule \vskip 5pt
\noindent
\underbar{DC/AC review problems}

\hskip 10pt Questions: {\it 61 through 80}
 
\vskip 10pt
%%%%%%%%%%%%%%%
\hrule \vskip 5pt
\noindent
\underbar{General concept practice and challenge problems}

\hskip 10pt Questions: {\it 81 through the end of the worksheet}
 
%%%%%%%%%%%%%%%






\vfil \eject

\centerline{\bf ELTR 140 (Digital 1), section 3} \bigskip 
 
\vskip 10pt

\noindent
{\bf Skill standards addressed by this course section}

\vskip 5pt

%%%%%%%%%%%%%%%
\hrule \vskip 10pt
\noindent
\underbar{EIA {\it Raising the Standard; Electronics Technician Skills for Today and Tomorrow}, June 1994}

\vskip 5pt

\medskip
\item{\bf F} {\bf Technical Skills -- Digital Circuits}
\item{\bf F.05} Understand principles and operations of types of logic gates.
\item{\bf F.20} Understand principles and operations of types of arithmetic-logic circuits.
\medskip

\vskip 5pt

\medskip
\item{\bf B} {\bf Basic and Practical Skills -- Communicating on the Job}
\item{\bf B.01} Use effective written and other communication skills.  {\it Met by group discussion and completion of labwork.}
\item{\bf B.03} Employ appropriate skills for gathering and retaining information.  {\it Met by research and preparation prior to group discussion.}
\item{\bf B.04} Interpret written, graphic, and oral instructions.  {\it Met by completion of labwork.}
\item{\bf B.06} Use language appropriate to the situation.  {\it Met by group discussion and in explaining completed labwork.}
\item{\bf B.07} Participate in meetings in a positive and constructive manner.  {\it Met by group discussion.}
\item{\bf B.08} Use job-related terminology.  {\it Met by group discussion and in explaining completed labwork.}
\item{\bf B.10} Document work projects, procedures, tests, and equipment failures.  {\it Met by project construction and/or troubleshooting assessments.}
\item{\bf C} {\bf Basic and Practical Skills -- Solving Problems and Critical Thinking}
\item{\bf C.01} Identify the problem.  {\it Met by research and preparation prior to group discussion.}
\item{\bf C.03} Identify available solutions and their impact including evaluating credibility of information, and locating information.  {\it Met by research and preparation prior to group discussion.}
\item{\bf C.07} Organize personal workloads.  {\it Met by daily labwork, preparatory research, and project management.}
\item{\bf C.08} Participate in brainstorming sessions to generate new ideas and solve problems.  {\it Met by group discussion.}
\item{\bf D} {\bf Basic and Practical Skills -- Reading}
\item{\bf D.01} Read and apply various sources of technical information (e.g. manufacturer literature, codes, and regulations).  {\it Met by research and preparation prior to group discussion.}
\item{\bf E} {\bf Basic and Practical Skills -- Proficiency in Mathematics}
\item{\bf E.01} Determine if a solution is reasonable.
\item{\bf E.02} Demonstrate ability to use a simple electronic calculator.
\item{\bf E.06} Translate written and/or verbal statements into mathematical expressions.
\item{\bf E.12} Interpret and use tables, charts, maps, and/or graphs.
\item{\bf E.13} Identify patterns, note trends, and/or draw conclusions from tables, charts, maps, and/or graphs.
\item{\bf E.15} Simplify and solve algebraic expressions and formulas.
\item{\bf E.16} Select and use formulas appropriately.
\medskip

%%%%%%%%%%%%%%%




\vfil \eject

\centerline{\bf ELTR 140 (Digital 1), section 3} \bigskip 
 
\vskip 10pt

\noindent
{\bf Common areas of confusion for students}

\vskip 5pt

%%%%%%%%%%%%%%%
\hrule \vskip 5pt

\vskip 10pt

\noindent
{\bf Difficult concept: } {\it Purpose of Gray code on encoder wheels.}

The purpose of using Gray code instead of binary encoding on rotary encoder wheels is a difficult concept for some to grasp.  One aid I have found in explaining the need for Gray encoding is to build an actual rotary encoder wheel out of a piece of stiff card-stock paper, drawing the code bits on the wheel as semi-circular arcs.  By rotating this wheel and looking at the bits go by, you can better see what problems may result if the bit sensors were out of alignment.

\vskip 10pt

\noindent
{\bf Difficult concept: } {\it Arithmetic overflow.}

When we represent numbers in binary (or any other numeration system, for that matter!) given a fixed number of characters, we become bound to a certain range of representable numbers.  A 10-digit electronic calculator lacking powers-of-ten notation, for example, can only display numbers from -9,999,999,999 to +9,999,999,999.  Any attempt to calculate answers beyond this range will result in {\it overflow}, as the digital circuit will not be able to represent the result.  It is important for students to comprehend and respect the concept of overflow, because it allows intelligent interpretation of digital calculation results.  It is necessary for a digital circuit to be able to detect this condition, so that the answer may be flagged as incorrect, rather than have the answer be presented as valid.

