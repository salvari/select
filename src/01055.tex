
%(BEGIN_QUESTION)
% Copyright 2003, Tony R. Kuphaldt, released under the Creative Commons Attribution License (v 1.0)
% This means you may do almost anything with this work of mine, so long as you give me proper credit

You should know that the line voltage of a three-phase, Y-connected, balanced system is always greater than the phase voltage by a factor of $\sqrt{3}$.

$$\epsfbox{01055x01.eps}$$

Apply Kirchhoff's Voltage Law (KVL) to the upper "loop" in this Y-connected alternator schematic to prove how 120 V $\angle$ 0$^{o}$ and 120 V $\angle$ 120$^{o}$ makes 208 V.  Show the "polarity" marks for each of the voltages as part of your answer.

\underbar{file 01055}
%(END_QUESTION)





%(BEGIN_ANSWER)

(120 V $\angle$ 0$^{o}$) - (120 V $\angle$ 120$^{o}$) = 208 V $\angle$ -30$^{o}$ 

\vskip 10pt

(120 V $\angle$ 120$^{o}$) - (120 V $\angle$ 0$^{o}$) = 208 V $\angle$ 150$^{o}$ 

\vskip 10pt

$$\epsfbox{01055x02.eps}$$

%(END_ANSWER)





%(BEGIN_NOTES)

This question is highly effective in demonstrating why polarity markings are important in AC circuit analysis.  Without the polarity marks as "frames of reference" for the phase angles, it is impossible to determine the resultant line voltage from the two 120 VAC phase voltages.

%(END_NOTES)


