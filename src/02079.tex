
%(BEGIN_QUESTION)
% Copyright 2004, Tony R. Kuphaldt, released under the Creative Commons Attribution License (v 1.0)
% This means you may do almost anything with this work of mine, so long as you give me proper credit

Calculate the total impedance for these two 100 mH inductors at 2.3 kHz, and draw a phasor diagram showing circuit admittances ($Y_{total}$, $G$, and $B$):

$$\epsfbox{02079x01.eps}$$

Now, re-calculate impedance and re-draw the phasor admittance diagram supposing the second inductor is replaced by a 1.5 k$\Omega$ resistor:

$$\epsfbox{02079x02.eps}$$

\underbar{file 02079}
%(END_QUESTION)





%(BEGIN_ANSWER)

$$\epsfbox{02079x03.eps}$$

$$\epsfbox{02079x04.eps}$$

\vskip 10pt

Challenge question: why are the susceptance vectors ($B_{L1}$ and $B_{L2}$) pointed down instead of up as impedance vectors for inductances typically are?

%(END_ANSWER)





%(BEGIN_NOTES)

Phasor diagrams are powerful analytical tools, if one knows how to draw and interpret them.  With hand calculators being so powerful and readily able to handle complex numbers in either polar or rectangular form, there is temptation to avoid phasor diagrams and let the calculator handle all the angle manipulation.  However, students will have a much better understanding of phasors and complex numbers in AC circuits if you hold them accountable to representing quantities in that form.

%INDEX% Impedance in parallel LR circuit
%INDEX% Phasor diagram

%(END_NOTES)


