
%(BEGIN_QUESTION)
% Copyright 2006, Tony R. Kuphaldt, released under the Creative Commons Attribution License (v 1.0)
% This means you may do almost anything with this work of mine, so long as you give me proper credit

Th\'evenin's theorem is a powerful tool for analyzing R-2R ladder networks.  Take for instance this four-section network where the next-to-most-significant "bit" is activated, while all the other "bits" are inactive (switched to ground):

$$\epsfbox{04001x01.eps}$$

If we Th\'evenize all sections to the left of the activated section, replacing it with a single resistance to ground, we see the network becomes far simpler:

$$\epsfbox{04001x02.eps}$$

\goodbreak
Explain how we may apply Th\'evenin's theorem once again to the shaded section of this next circuit (simplified from the previous circuit shown above) to simplify it even more, obtaining a final result for $V_{out}$:

$$\epsfbox{04001x03.eps}$$

\underbar{file 04001}
%(END_QUESTION)





%(BEGIN_ANSWER)

Once you get to this point, solving for $V_{out}$ in terms of $V_{ref}$ is trivial:

$$\epsfbox{04001x04.eps}$$

%(END_ANSWER)





%(BEGIN_NOTES)

Students might not realize it is valid to iteratively apply Th\'evenin's theorem to the solution of a circuit problem.  You can, and this stands as a good example of how (and why!) you should do it.

%INDEX% R-2R ladder network

%(END_NOTES)


