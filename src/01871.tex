
%(BEGIN_QUESTION)
% Copyright 2003, Tony R. Kuphaldt, released under the Creative Commons Attribution License (v 1.0)
% This means you may do almost anything with this work of mine, so long as you give me proper credit

The {\it cutoff frequency}, also known as {\it half-power point} or {\it -3dB point}, of either a low-pass or a high-pass filter is fairly easy to define.  But what about band-pass and band-stop filter circuits?  Does the concept of a "cutoff frequency" apply to these filter types?  Explain your answer.

\underbar{file 01871}
%(END_QUESTION)





%(BEGIN_ANSWER)

Unlike low-pass and high-pass filters, band-pass and band-stop filter circuits have {\it two} cutoff frequencies ($f_{c1}$ and $f_{c2}$)!

%(END_ANSWER)





%(BEGIN_NOTES)

This question presents a good opportunity to ask students to draw the Bode plot of a typical band-pass or band-stop filter on the board in front of the class to illustrate the concept.  Don't be afraid to let students up to the front of the classroom to present their findings.  It's a great way to build confidence in them and also to help suppress the illusion that you (the teacher) are the Supreme Authority of the classroom!

%INDEX% Cutoff frequency, of band-pass or band-stop filter

%(END_NOTES)


