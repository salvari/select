
\centerline{\bf ELTR 115 (AC 2), section 1} \bigskip 
 
\vskip 10pt

\noindent
{\bf Recommended schedule}

\vskip 5pt

%%%%%%%%%%%%%%%
\hrule \vskip 5pt
\noindent
\underbar{Day 1}

\hskip 10pt Topics: {\it Mutual inductance and transformer theory}
 
\hskip 10pt Questions: {\it 1 through 15}
 
\hskip 10pt Lab Exercise: {\it Transformer voltage/current ratios (question 61)}
 
%INSTRUCTOR \hskip 10pt {\bf Explain suggested project ideas to students}

%INSTRUCTOR \hskip 10pt {\bf Give project grading rubric to students, complete with deadlines}

\vskip 10pt
%%%%%%%%%%%%%%%
\hrule \vskip 5pt
\noindent
\underbar{Day 2}

\hskip 10pt Topics: {\it Transformer step ratio}
 
\hskip 10pt Questions: {\it 16 through 30}
 
\hskip 10pt Lab Exercise: {\it Auto-transformers (question 62)}
 
%INSTRUCTOR \hskip 10pt {\bf MIT 8.02 video clip: Disk 4, Lecture 24; Transformer w/ loose wire 34:13 to 36:40}

%INSTRUCTOR \hskip 10pt {\bf MIT 8.02 video clip: Disk 4, Lecture 24; Shorted transformer 40:03 to 41:50}

%INSTRUCTOR \hskip 10pt {\bf MIT 8.02 video clip: Disk 4, Lecture 24; Ruhmkorff coil demo 47:45 to end}

\vskip 10pt
%%%%%%%%%%%%%%%
\hrule \vskip 5pt
\noindent
\underbar{Day 3}

\hskip 10pt Topics: {\it Maximum power transfer theorem and impedance matching with transformers}
 
\hskip 10pt Questions: {\it 31 through 45}
 
\hskip 10pt Lab Exercise: {\it Auto-transformers (question 63)}
 
\vskip 10pt
%%%%%%%%%%%%%%%
\hrule \vskip 5pt
\noindent
\underbar{Day 4}

\hskip 10pt Topics: {\it Transformer applications, power ratings, and core effects}
 
\hskip 10pt Questions: {\it 46 through 60}
 
\hskip 10pt Lab Exercise: {\it Differential voltage measurement using the oscilloscope (question 64)}
 
%INSTRUCTOR \hskip 10pt {\bf Demo: show a current transformer, and a clamp-on AC ammeter}

\vskip 10pt
%%%%%%%%%%%%%%%
\hrule \vskip 5pt
\noindent
\underbar{Day 5}

\hskip 10pt Exam 1: {\it includes Transformer voltage ratio performance assessment}
 
\hskip 10pt Lab Exercise: {\it work on project}
 
\hskip 10pt Project: {\it Initial project design checked by instructor and components selected (sensitive audio detector circuit recommended)}
  
\vskip 10pt
%%%%%%%%%%%%%%%
\hrule \vskip 5pt
\noindent
\underbar{Practice and challenge problems}

\hskip 10pt Questions: {\it 66 through the end of the worksheet}
 
\vskip 10pt
%%%%%%%%%%%%%%%
\hrule \vskip 5pt
\noindent
\underbar{Impending deadlines}

\hskip 10pt {\bf Project due at end of ELTR115, Section 3}
 
\hskip 10pt Question 65: Sample project grading criteria
 
\vskip 10pt
%%%%%%%%%%%%%%%










\vfil \eject

\centerline{\bf ELTR 115 (AC 2), section 1} \bigskip 
 
\vskip 10pt

\noindent
{\bf Project ideas}

\vskip 5pt

\hrule \vskip 5pt

\vskip 10pt

\noindent
\underbar{AC power supply:} {\it (Strongly Recommended!)}  This is basically one-half of an AC/DC power supply circuit, consisting of a line power plug, on/off switch, fuse, indicator lamp, and a step-down transformer.  The reason this project idea is strongly recommended is that it may serve as the basis for the recommended power supply project in the next course (ELTR120 -- Semiconductors 1).  If you build the AC section now, you will not have to re-build an enclosure or any of the line-power circuitry later!  Note that the first lab (step-down transformer circuit) may serve as a prototype for this project with just a few additional components.

\vskip 10pt

\noindent
\underbar{Sensitive audio detector:} A test device used to listen (audibly) to AC signals in the audio-frequency range.  Highly recommended, as it provides a great learning experience as well as a very useful piece of test equipment for your personal projects, especially when working on audio amplifiers.  May be expensive due to the need for a pair of closed-cup headphones.

\vskip 10pt

\noindent
\underbar{LED blinking circuit:} Recommended for advanced students only, as it involves semiconductor components which have not been covered yet.  May be designed around an astable multivibrator circuit, or a 555 timer chip.  Ask instructor for design details.

\vskip 10pt

\noindent
\underbar{Signal generator circuit:} Recommended for advanced students only, as it involves the use of at least one integrated circuit "chip."  There are multiple signal generator chips on the market, all of which are fairly easy to configure and use with no knowledge of their internal operation.

\vskip 10pt






\vfil \eject

\centerline{\bf ELTR 115 (AC 2), section 1} \bigskip 
 
\vskip 10pt

\noindent
{\bf Skill standards addressed by this course section}

\vskip 5pt

%%%%%%%%%%%%%%%
\hrule \vskip 10pt
\noindent
\underbar{EIA {\it Raising the Standard; Electronics Technician Skills for Today and Tomorrow}, June 1994}

\vskip 5pt

\medskip
\item{\bf C} {\bf Technical Skills -- AC circuits}
\item{\bf C.14} Understand principles and operations of AC circuits using transformers.
\item{\bf C.15} Demonstrate an understanding of impedance matching theory.
\item{\bf C.16} Fabricate and demonstrate AC circuits using transformers.
\item{\bf C.17} Troubleshoot and repair AC circuits using transformers.
\medskip

\vskip 5pt

\medskip
\item{\bf B} {\bf Basic and Practical Skills -- Communicating on the Job}
\item{\bf B.01} Use effective written and other communication skills.  {\it Met by group discussion and completion of labwork.}
\item{\bf B.03} Employ appropriate skills for gathering and retaining information.  {\it Met by research and preparation prior to group discussion.}
\item{\bf B.04} Interpret written, graphic, and oral instructions.  {\it Met by completion of labwork.}
\item{\bf B.06} Use language appropriate to the situation.  {\it Met by group discussion and in explaining completed labwork.}
\item{\bf B.07} Participate in meetings in a positive and constructive manner.  {\it Met by group discussion.}
\item{\bf B.08} Use job-related terminology.  {\it Met by group discussion and in explaining completed labwork.}
\item{\bf B.10} Document work projects, procedures, tests, and equipment failures.  {\it Met by project construction and/or troubleshooting assessments.}
\item{\bf C} {\bf Basic and Practical Skills -- Solving Problems and Critical Thinking}
\item{\bf C.01} Identify the problem.  {\it Met by research and preparation prior to group discussion.}
\item{\bf C.03} Identify available solutions and their impact including evaluating credibility of information, and locating information.  {\it Met by research and preparation prior to group discussion.}
\item{\bf C.07} Organize personal workloads.  {\it Met by daily labwork, preparatory research, and project management.}
\item{\bf C.08} Participate in brainstorming sessions to generate new ideas and solve problems.  {\it Met by group discussion.}
\item{\bf D} {\bf Basic and Practical Skills -- Reading}
\item{\bf D.01} Read and apply various sources of technical information (e.g. manufacturer literature, codes, and regulations).  {\it Met by research and preparation prior to group discussion.}
\item{\bf E} {\bf Basic and Practical Skills -- Proficiency in Mathematics}
\item{\bf E.01} Determine if a solution is reasonable.
\item{\bf E.02} Demonstrate ability to use a simple electronic calculator.
\item{\bf E.05} Solve problems and [sic] make applications involving integers, fractions, decimals, percentages, and ratios using order of operations.
\item{\bf E.06} Translate written and/or verbal statements into mathematical expressions.
\item{\bf E.09} Read scale on measurement device(s) and make interpolations where appropriate.  {\it Met by oscilloscope usage.}
\item{\bf E.12} Interpret and use tables, charts, maps, and/or graphs.
\item{\bf E.13} Identify patterns, note trends, and/or draw conclusions from tables, charts, maps, and/or graphs.
\item{\bf E.15} Simplify and solve algebraic expressions and formulas.
\item{\bf E.16} Select and use formulas appropriately.
\item{\bf E.17} Understand and use scientific notation.
\item{\bf E.20} Graph functions.
\item{\bf E.26} Apply Pythagorean theorem.
\item{\bf E.27} Identify basic functions of sine, cosine, and tangent.
\item{\bf E.28} Compute and solve problems using basic trigonometric functions.
\medskip

%%%%%%%%%%%%%%%






\vfil \eject

\centerline{\bf ELTR 115 (AC 2), section 1} \bigskip 
 
\vskip 10pt

\noindent
{\bf Common areas of confusion for students}

\vskip 5pt

%%%%%%%%%%%%%%%
\hrule \vskip 5pt

\vskip 10pt

\noindent
{\bf Difficult concept: } {\it Rates of change.}

When studying electromagnetic induction and Lenz's Law, one must think in terms of how fast a variable is changing.  The amount of voltage induced in a conductor is proportional to how {\it quickly} the magnetic field changes, not how strong the field is.  This is the first hurdle in calculus: to comprehend what a rate of change is, and it is not obvious.

The best examples I know of to describe rates of change are {\it velocity} and {\it acceleration}.  Velocity is nothing more than a rate of change of position: how quickly one's position is changing over time.  Therefore, if the variable $x$ describes position, then the derivative ${dx \over dt}$ (rate of change of $x$ over time $t$) must describe velocity.  Likewise, acceleration is nothing more than the rate of change of velocity: how quickly velocity changes over time.  If the variable $v$ describes velocity, then the derivative ${dv \over dt}$ must describe velocity.  Or, since we know that velocity is itself the derivative of position, we could describe acceleration as the {\it second derivative} of position: ${d^2 x \over dt^2}$

\vskip 10pt

\noindent
{\bf Common mistake: } {\it Reversing transformer ratios.}

Many transformer problems are nothing more than exercises in mathematical ratios.  Once students become comfortable manipulating these ratios to solve for voltages, currents, and/or winding turns, it is easy.  Unfortunately, it is also very easy to accidently reverse the ratios (e.g. 3 to 1 instead of 1 to 3).  With regard to voltage and current for both sides of a transformer, there is a very simple way to check your work.  Like all passive devices, transformers can never output more power than they input.  And because they are efficient, their output power is rarely less than 90\% of their input power.  For many practice problems, 100\% efficiency is assumed, so output power must be the same as input power.  To check your voltage and current calculations, figure out primary power from primary voltage and primary current, then do the same for the secondary side.  The two power calculations should match!

\vskip 10pt

\noindent
{\bf Common mistake: } {\it Failing to respect shock hazard of line-powered circuits.}

For those students who choose to build the line-powered AC power supply, it is good to review the principles of electrical safety.  Unlike nearly all the previous labs which harbored little or no shock hazard, this project can shock you.  The most important rule you can follow is to simply unplug the circuit from the AC line before reaching toward any part of the circuit with your hand or with a conductive tool.  The only things you should touch a live circuit with are test probes for measurement equipment!  Another common mistake is to fail to remove conductive jewelry (bracelets, rings, etc.) prior to working with line-powered circuits.
