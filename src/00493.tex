
%(BEGIN_QUESTION)
% Copyright 2003, Tony R. Kuphaldt, released under the Creative Commons Attribution License (v 1.0)
% This means you may do almost anything with this work of mine, so long as you give me proper credit

If we were to express the series-connected DC voltages as {\it phasors} (arrows pointing with a particular length and a particular direction, graphically expressing magnitude and polarity of an electrical signal), how would we draw them in such a way that the total (or {\it resultant}) phasors accurately expressed the total voltage of each series-connected pair?

$$\epsfbox{00493x01.eps}$$

If we were to assign angle values to each of these phasors, what would you suggest?

\underbar{file 00493}
%(END_QUESTION)





%(BEGIN_ANSWER)

$$\epsfbox{00493x02.eps}$$

In the right-hand circuit, where the two voltage sources are opposing, one of the phasors will have an angle of 0$^{o}$, while the other will have an angle of 180$^{o}$.

%(END_ANSWER)





%(BEGIN_NOTES)

Phasors are really nothing more than an extension of the familiar "number line" most students see during their primary education years.  The important difference here is that phasors are two-dimensional magnitudes, not one-dimensional, as {\it scalar} numbers are.

The use of degrees to measure angles should be familiar as well, even to those students without a strong mathematics background.  For example, what does it mean when a skateboarder or stunt bicyclist "does a {\it 180}"?  It means they turn around so as to face the opposite direction (180 degrees away from) their previous direction.

%INDEX% Addition of two DC voltage sources, phasor representation
%INDEX% Phasor diagram

%(END_NOTES)


