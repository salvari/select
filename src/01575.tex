
%(BEGIN_QUESTION)
% Copyright 2003, Tony R. Kuphaldt, released under the Creative Commons Attribution License (v 1.0)
% This means you may do almost anything with this work of mine, so long as you give me proper credit

At a construction site, workers have used power extension cords to create a "network" of power cabling for their various electric tools.  Each cord has one (male) plug and a receptacle (female) end that accepts up to three plugs:

$$\epsfbox{01575x01.eps}$$

Despite this dangerous wiring, all tools have functioned so far without trouble.  Then suddenly both the worklight and the circular saw in the lower-right corner of the illustration stop working.  All the other tools continue to function properly (including all the radios, which is very fortunate because the workers become irritable without their music).

From this information alone, determine what sections of this "network" are good, and what sections are suspect.

\underbar{file 01575}
%(END_QUESTION)





%(BEGIN_ANSWER)

$$\epsfbox{01575x02.eps}$$

Follow-up question: describe the general principle you used to locate the suspect area of this power network.

%(END_ANSWER)





%(BEGIN_NOTES)

Now, of course, it is possible that both the worklight and the saw suffered independent, simultaneous failures, and all the extension cords are good, but this is not very likely.  Be sure to discuss this possibility with your students, and the reasoning why the one extension cord would be more likely to be faulted than two separate devices.

%INDEX% Troubleshooting, extension cord scenario

%(END_NOTES)


