
%(BEGIN_QUESTION)
% Copyright 2003, Tony R. Kuphaldt, released under the Creative Commons Attribution License (v 1.0)
% This means you may do almost anything with this work of mine, so long as you give me proper credit

Given the unavoidable presence of parasitic inductance and/or capacitance in any electronic component, what does this mean in terms of {\it resonance} for single components in AC circuits?

\underbar{file 00594}
%(END_QUESTION)





%(BEGIN_ANSWER)

Parasitic reactance means that any single component is theoretically capable of resonance, all on its own!

\vskip 10pt

Follow-up question: at what frequency would you expect a component to self-resonate?  Would this be a very low frequency, a very high frequency, or a frequency within the circuit's normal operating range?  Explain your answer.

%(END_ANSWER)





%(BEGIN_NOTES)

This question grew out of several years' worth of observations, where students would discover self-resonant effects in large ($>$ 1 Henry) inductors at modest frequencies.  Being a recurring theme, I thought it prudent to include this question within my basic electronics curriculum.

One component that tends to be more immune to self-resonance than others is the lowly resistor, especially resistors of large value.  Ask your students why they think this might be.  A mechanical analogy to self-resonance is the natural frequency of vibration for an object, given the unavoidable presence of both elasticity and mass in any object.  The mechanical systems most immune to vibratory resonance, though, are those with a high degree of intrinsic {\it friction}.

%INDEX% Resonance, self (of an inductor)
%INDEX% Self-resonance, of an inductor

%(END_NOTES)


