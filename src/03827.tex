
%(BEGIN_QUESTION)
% Copyright 2006, Tony R. Kuphaldt, released under the Creative Commons Attribution License (v 1.0)
% This means you may do almost anything with this work of mine, so long as you give me proper credit

Predict how the operation of this motor control circuit will be affected as a result of the following faults.  Consider each fault independently (i.e. one at a time, no multiple faults):

$$\epsfbox{03827x01.eps}$$

\medskip
\item{$\bullet$} "Stop" pushbutton switch fails open:
\vskip 5pt
\item{$\bullet$} Relay contact CR1-1 fails open:
\vskip 5pt
\item{$\bullet$} Relay contact CR1-2 fails open:
\vskip 5pt
\item{$\bullet$} Relay coil CR1 fails open:
\medskip

For each of these conditions, explain {\it why} the resulting effects will occur.

\underbar{file 03827}
%(END_QUESTION)





%(BEGIN_ANSWER)

\medskip
\item{$\bullet$} "Stop" pushbutton switch fails open: {\it Motor cannot start, lamp never energizes.}
\vskip 5pt
\item{$\bullet$} Relay contact CR1-1 fails open: {\it Motor starts and lamp energizes when "Start" button is pressed, but both immediately de-energize when it is released.}
\vskip 5pt
\item{$\bullet$} Relay contact CR1-2 fails open: {\it "Motor run" lamp turns on and off as expected, but the motor itself never runs.}
\vskip 5pt
\item{$\bullet$} Relay coil CR1 fails open: {\it Motor cannot start, lamp never energizes.}
\medskip

%(END_ANSWER)





%(BEGIN_NOTES)

The purpose of this question is to approach the domain of circuit troubleshooting from a perspective of knowing what the fault is, rather than only knowing what the symptoms are.  Although this is not necessarily a realistic perspective, it helps students build the foundational knowledge necessary to diagnose a faulted circuit from empirical data.  Questions such as this should be followed (eventually) by other questions asking students to identify likely faults based on measurements.

%INDEX% Troubleshooting, predicting effects of fault in relay logic circuit

%(END_NOTES)


