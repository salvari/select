
%(BEGIN_QUESTION)
% Copyright 2005, Tony R. Kuphaldt, released under the Creative Commons Attribution License (v 1.0)
% This means you may do almost anything with this work of mine, so long as you give me proper credit

Capacitors often have letter codes following the three-digit number codes.  For example, here are some typical capacitor codes, complete with letters:

\medskip
\item{$\bullet$} 473K
\item{$\bullet$} 102J
\item{$\bullet$} 224M
\item{$\bullet$} 331F
\medskip

Determine the meaning of letters used on capacitor labels, what the respective numeric values are for all the available letters, and then finally what these four specific number/letter codes mean (shown above).

\underbar{file 03651}
%(END_QUESTION)





%(BEGIN_ANSWER)

Letter codes are used to designate {\it tolerance}, just like the last color band on most resistors.  I will let you research the letter code equivalencies on your own!  Same for the specific values of the four capacitor labels shown.

%(END_ANSWER)





%(BEGIN_NOTES)

The capacitor tolerance codes are easy enough for students to research on their own.  For your own reference, though:

\medskip
\item{$\bullet$} D = $\pm$ 0.5\%
\item{$\bullet$} F = $\pm$ 1\%
\item{$\bullet$} G = $\pm$ 2\%
\item{$\bullet$} H = $\pm$ 3\%
\item{$\bullet$} J = $\pm$ 5\%
\item{$\bullet$} K = $\pm$ 10\%
\item{$\bullet$} M = $\pm$ 20\%
\item{$\bullet$} P = +100\%, -0\%
\item{$\bullet$} Z = +80\%, -20\%
\medskip

Same for the four capacitor labels given in the question:

\medskip
\item{$\bullet$} 473K = 47 nF $\pm$ 10\%
\item{$\bullet$} 102J = 1 nF $\pm$ 5\%
\item{$\bullet$} 224M = 0.22 $\mu$F $\pm$ 20\%
\item{$\bullet$} 331F = 330 pF $\pm$ 1\%
\medskip

%INDEX% Capacitor value identification

%(END_NOTES)


