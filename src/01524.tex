
%(BEGIN_QUESTION)
% Copyright 2003, Tony R. Kuphaldt, released under the Creative Commons Attribution License (v 1.0)
% This means you may do almost anything with this work of mine, so long as you give me proper credit

The following schematic diagram shows a simple {\it common-emitter} transistor amplifier circuit:

$$\epsfbox{01524x01.eps}$$

Explain why the voltage gain ($A_V$) of such an amplifier is approximately ${R_C \over R_E}$, using any or all of these general "rules" of transistor behavior:

\medskip
\item{$\bullet$} $I_E = I_C + I_B$
\item{$\bullet$} $I_E \approx I_C$
\item{$\bullet$} $V_{BE} \approx 0.7$ volts
\item{$\bullet$} $\beta = {I_C \over I_B}$
\medskip

Remember that (AC) voltage gain is defined as ${\Delta V_{out} \over \Delta V_{in}}$.  Hint: this question might be easier to answer if you first consider how to explain the unity-gain of a common-collector amplifier circuit (simply eliminate $R_C$, replacing it with a direct connection to $-V$, and consider $V_E$ to be the output voltage).

\underbar{file 01524}
%(END_QUESTION)





%(BEGIN_ANSWER)

Since $V_{BE}$ is relatively constant, $\Delta V_{in} \approx \Delta V_E$.  The next essential step in the explanation for the voltage gain formula is to couple this fact with $I_E \approx I_C$.  The rest I'll leave for you to explain.
 
\vskip 10pt

For your discussion response, be prepared to explain everything in mathematical terms.  You will have to use Kirchhoff's Voltage Law at least once to be able to do this completely.

%(END_ANSWER)





%(BEGIN_NOTES)

Although the given answer seems complete, what I'm looking for here is a good analytical understanding of why the voltage gain is what it is.  Placing the requirement of using KVL on the students' answers ensures that they will have to explore the concept further than the given answer does.

%INDEX% Voltage gain, common emitter amplifier

%(END_NOTES)


