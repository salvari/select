
%(BEGIN_QUESTION)
% Copyright 2003, Tony R. Kuphaldt, released under the Creative Commons Attribution License (v 1.0)
% This means you may do almost anything with this work of mine, so long as you give me proper credit

$$\epsfbox{01692x01.eps}$$

\underbar{file 01692}
\vfil \eject
%(END_QUESTION)





%(BEGIN_ANSWER)

Use circuit simulation software to verify your predicted and measured parameter values, using an AC current source as the motor, and a multi-conductor transmission line as the cable.  Note: this may be quite complicated to set up in simulation!

%(END_ANSWER)





%(BEGIN_NOTES)

The longer the cable, the better the coupling between the motor conductors and the other conductors.  I have found that this exercise is practical with just a few feet of four-wire telephone cable.

It is very important to use a battery for this exercise, and not an AC-DC power supply.  The filter capacitors in an AC-DC power supply naturally decouple the motor's noise voltage, making it difficult to measure any at all!

Noise voltage should be measured with an oscilloscope set to AC input coupling, and recorded as peak-to-peak volts.

%INDEX% Assessment, performance-based (Noise coupling and decoupling)

%(END_NOTES)


