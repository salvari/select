
%(BEGIN_QUESTION)
% Copyright 2003, Tony R. Kuphaldt, released under the Creative Commons Attribution License (v 1.0)
% This means you may do almost anything with this work of mine, so long as you give me proper credit

Suppose an inductor is connected directly to an adjustable-current source, and the current of that source is steadily {\it increased} over time.  We know that an increasing current through an inductor will produce a magnetic field of increasing strength.  Does this increase in magnetic field constitute an {\it accumulation} of energy in the inductor, or a {\it release} of energy from the inductor?  In this scenario, does the inductor act as a {\it load} or as a {\it source} of electrical energy?

$$\epsfbox{00209x01.eps}$$

Now, suppose the adjustable current source is steadily {\it decreased} over time.  We know this will result in a magnetic field of decreasing strength in the inductor.  Does this decrease in magnetic field constitute an {\it accumulation} of energy in the inductor, or a {\it release} of energy from the inductor?  In this scenario, does the inductor act as a {\it load} or as a {\it source} of electrical energy?

$$\epsfbox{00209x02.eps}$$

For each of these scenarios, label the inductor's voltage drop polarity.

\underbar{file 00209}
%(END_QUESTION)





%(BEGIN_ANSWER)

As the applied current increases, the inductor acts as a load, accumulating additional energy from the current source.  Acting as a load, the voltage dropped by the inductor will be in the same polarity as across a resistor.

$$\epsfbox{00209x03.eps}$$

As the applied current decreases, the inductor acts as a source, releasing accumulated energy to the rest of the circuit, as though it were a current source itself of superior current.  Acting as a source, the voltage dropped by the inductor will be in the same polarity as across a battery, powering a load.

$$\epsfbox{00209x04.eps}$$

%(END_ANSWER)





%(BEGIN_NOTES)

Relating the polarity of voltage across an inductor to a change of applied current over time is a complex concept for many students.  Since it involves rates of change over time, it is an excellent opportunity to introduce calculus concepts (${d \over dt}$).

Vitally important to students' conceptual understanding of an inductor exposed to increasing or decreasing currents is the distinction between an electrical energy {\it source} versus a {\it load}.  Students need to think "battery" and "resistor," respectively when determining the relationship between direction of current and voltage drop.  The complicated aspect of inductors (and capacitors!) is that they may switch character in an instant, from being a source of energy to being a load, and visa-versa.  The relationship is not fixed as it is for resistors, which are always energy {\it loads}.

%INDEX% Inductance, voltage versus current in

%(END_NOTES)


