
%(BEGIN_QUESTION)
% Copyright 2005, Tony R. Kuphaldt, released under the Creative Commons Attribution License (v 1.0)
% This means you may do almost anything with this work of mine, so long as you give me proper credit

$$\epsfbox{02959x01.eps}$$

\underbar{file 02959}
\vfil \eject
%(END_QUESTION)





%(BEGIN_ANSWER)

Use circuit simulation software to verify your predicted and actual truth tables.

%(END_ANSWER)





%(BEGIN_NOTES)

Students are to research the datasheet for their particular IC and figure out from that what connections and timing sequences they need to make the circuit perform as requested.  It is very important for students to learn to interpret manufacturers' datasheets!

I recommend a slow clock frequency (1 Hz or so) to allow for easy viewing of the shift patterns.  To conserve the number of necessary input switches, I allow students to hard-wire the data inputs ($D_0$ through $D_3$).  This means they only need switches to control the mode of the shift register (parallel load, shift right, shift left, and shift inhibit).

%INDEX% Assessment, performance-based (Binary counter as frequency divider)

%(END_NOTES)


