
%(BEGIN_QUESTION)
% Copyright 2004, Tony R. Kuphaldt, released under the Creative Commons Attribution License (v 1.0)
% This means you may do almost anything with this work of mine, so long as you give me proper credit

Complete the table of output voltages for several given values of input voltage in this common-collector amplifier circuit.  Assume that the transistor is a standard silicon NPN unit, with a nominal base-emitter junction forward voltage of 0.7 volts:

$$\epsfbox{02224x01.eps}$$

% No blank lines allowed between lines of an \halign structure!
% I use comments (%) instead, so that TeX doesn't choke.

$$\vbox{\offinterlineskip
\halign{\strut
\vrule \quad\hfil # \ \hfil & 
\vrule \quad\hfil # \ \hfil \vrule \cr
\noalign{\hrule}
%
% First row
$V_{in}$ & $V_{out}$ \cr
%
\noalign{\hrule}
%
% Second row
0.0 V &  \cr
%
\noalign{\hrule}
%
% Third row
0.5 V &  \cr
%
\noalign{\hrule}
%
% Fourth row
1.0 V &  \cr
%
\noalign{\hrule}
%
% Fifth row
1.5 V &  \cr
%
\noalign{\hrule}
%
% Sixth row
5.0 V &  \cr
%
\noalign{\hrule}
%
% Seventh row
7.8 V &  \cr
%
\noalign{\hrule}
} % End of \halign 
}$$ % End of \vbox

Based on the values you calculate, explain why the common-collector circuit configuration is often referred to as an {\it emitter follower}.

\underbar{file 02224}
%(END_QUESTION)





%(BEGIN_ANSWER)

% No blank lines allowed between lines of an \halign structure!
% I use comments (%) instead, so that TeX doesn't choke.

$$\vbox{\offinterlineskip
\halign{\strut
\vrule \quad\hfil # \ \hfil & 
\vrule \quad\hfil # \ \hfil \vrule \cr
\noalign{\hrule}
%
% First row
$V_{in}$ & $V_{out}$ \cr
%
\noalign{\hrule}
%
% Second row
0.0 V & 0.0 V \cr
%
\noalign{\hrule}
%
% Third row
0.5 V & 0.0 V \cr
%
\noalign{\hrule}
%
% Fourth row
1.0 V & 0.3 V \cr
%
\noalign{\hrule}
%
% Fifth row
1.5 V & 0.8 V \cr
%
\noalign{\hrule}
%
% Sixth row
5.0 V & 4.3 V \cr
%
\noalign{\hrule}
%
% Seventh row
7.8 V & 7.1 V \cr
%
\noalign{\hrule}
} % End of \halign 
}$$ % End of \vbox

The voltage at the transistor's emitter terminal approximately "follows" the voltage applied to the base terminal, hence the name.

%(END_ANSWER)





%(BEGIN_NOTES)

At first, the "emitter follower" transistor circuit may seem pointless, since the output voltage practically equals the input voltage (especially for input voltages greatly exceeding 0.7 volts DC).  "What possible good is a circuit like this?" some of your students may ask.  The answer to this question, of course, has to do with {\it currents} in the circuit, and not necessarily voltages.

%INDEX% Common-collector amplifier, output voltage calculations

%(END_NOTES)


