
%(BEGIN_QUESTION)
% Copyright 2005, Tony R. Kuphaldt, released under the Creative Commons Attribution License (v 1.0)
% This means you may do almost anything with this work of mine, so long as you give me proper credit

Determine the DC (quiescent) voltage values at each of the labeled points, with respect to ground.  Assume that all conducting PN junctions drop exactly 0.7 volts, and that $I_B$ = 0 $\mu$A and $I_C = I_E$ for each transistor:

$$\epsfbox{02553x01.eps}$$

\medskip
\item{$\bullet$} $V_A$ = 
\item{$\bullet$} $V_B$ = 
\item{$\bullet$} $V_C$ = 
\item{$\bullet$} $V_D$ = 
\item{$\bullet$} $V_E$ = 
\item{$\bullet$} $V_F$ = 
\item{$\bullet$} $V_G$ = 
\item{$\bullet$} $V_H$ = 
\medskip

\underbar{file 02553}
%(END_QUESTION)





%(BEGIN_ANSWER)

\medskip
\item{$\bullet$} $V_A$ = 0 volts
\item{$\bullet$} $V_B$ = 1.5 volts
\item{$\bullet$} $V_C$ = 0.8 volts
\item{$\bullet$} $V_D$ = 5.24 volts
\item{$\bullet$} $V_E$ = 4.54 volts
\item{$\bullet$} $V_F$ = 0 volts
\item{$\bullet$} $V_G$ = 9 volts
\item{$\bullet$} $V_H$ = 0.8 volts
\medskip

%(END_ANSWER)





%(BEGIN_NOTES)

{\bf This question is intended for exams only and not worksheets!}.

%INDEX% Amplifier, multi-stage
%INDEX% Amplifier, estimating DC (quiescent) voltages in

%(END_NOTES)


