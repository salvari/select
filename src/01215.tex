
%(BEGIN_QUESTION)
% Copyright 2003, Tony R. Kuphaldt, released under the Creative Commons Attribution License (v 1.0)
% This means you may do almost anything with this work of mine, so long as you give me proper credit
% May 9, 2005 -- Sam Cheung noted error in schematic diagrams with negative feedback connection.

The circuit shown here is a {\it Wien-bridge oscillator}:

$$\epsfbox{01215x01.eps}$$

If one side of the Wien bridge is made from a potentiometer instead of two fixed-value resistors, this adjustment will affect both the {\it amplitude} and the {\it distortion} of the oscillator's output signal:

$$\epsfbox{01215x02.eps}$$

Explain why this adjustment has the effect that it does.  What, exactly, does moving the potentiometer do to the circuit to alter the output signal?  Also, calculate the operating frequency of this oscillator circuit, and explain how you would make that frequency adjustable as well.

\underbar{file 01215}
%(END_QUESTION)





%(BEGIN_ANSWER)

The potentiometer adjusts the Barkhausen criterion of the oscillator.  I'll let you figure out how to make the frequency adjustable.

\vskip 10pt

$f$ = 153.9 Hz

\vskip 10pt

Follow-up question: identify the paths of positive and negative feedback from the Wien bridge to the first amplifier stage.

%(END_ANSWER)





%(BEGIN_NOTES)

One of the advantages of the Wien bridge circuit is its ease of adjustment in this manner.  Using high-quality capacitors and resistors in the other side of the bridge, its output frequency will be very stable.

%INDEX% Wien bridge oscillator circuit
%INDEX% Oscillator circuit, Wien bridge

%(END_NOTES)


