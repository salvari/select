
%(BEGIN_QUESTION)
% Copyright 2003, Tony R. Kuphaldt, released under the Creative Commons Attribution License (v 1.0)
% This means you may do almost anything with this work of mine, so long as you give me proper credit

A student builds a four-bit asynchronous counter circuit using CMOS J-K flip-flops.  It seems to work . . . most of the time.  Every once in a while, the count suddenly and mysteriously "jumps" out of sequence, to a value that is completely wrong.  Even stranger than this is the fact that it seems to happen every time the student waves their hand next to the circuit.

What do you suspect the problem to be?

\underbar{file 01406}
%(END_QUESTION)





%(BEGIN_ANSWER)

This is a mistake I see students making all the time.  The fact that the circuit is built with CMOS components, and fails whenever an object comes near it, is a strong hint that the problem is related to stray static electric charges.  It is an easily corrected problem, caused by the student not taking time to connect {\it all} the pins of their flip-flops properly.

%(END_ANSWER)





%(BEGIN_NOTES)

I didn't exactly reveal the source of trouble in the answer, but I gave enough hints that anyone familiar with CMOS should be able to tell what it is!  This truly is a problem I've seen many times with my students!

%INDEX% Troubleshooting, logic gate circuit (CMOS)

%(END_NOTES)


