
%(BEGIN_QUESTION)
% Copyright 2003, Tony R. Kuphaldt, released under the Creative Commons Attribution License (v 1.0)
% This means you may do almost anything with this work of mine, so long as you give me proper credit

An electronic service technician prepares to work on a high-voltage power supply circuit containing one large capacitor.  On the side of this capacitor are the following specifications:

$$\hbox{3000 WVDC \hskip 10pt} 0.75 \mu \hbox{F}$$

Obviously this device poses a certain amount of danger, even with the AC line power secured (lock-out/tag-out).  Discharging this capacitor by directly shorting its terminals with a screwdriver or some other piece of metal might be dangerous due to the quantity of the stored charge.  What needs to be done is to discharge this capacitor at a modest rate.

The technician realizes that she can discharge the capacitor at any rate desired by connecting a resistor in parallel with it (holding the resistor with electrically-insulated pliers, of course, to avoid having to touch either terminal).  What size resistor should she use, if she wants to discharge the capacitor to less than 1\% charge in 15 seconds?  State your answer using the standard 4-band resistor color code (tolerance = +/- 10\%).

\underbar{file 01525}
%(END_QUESTION)





%(BEGIN_ANSWER)

Yellow, Black, Green, Silver (assuming 5 time constants' worth of time: just less than 1\% charge).  Yellow, Orange, Green, Silver for a discharge down to 1\% in 15 seconds.

%(END_ANSWER)





%(BEGIN_NOTES)

In order to answer this question, students must not only be able to calculate time constants for a simple RC circuit, but they must also remember the resistor color code so as to choose the right size based on color.  A very practical problem, and important for safety reasons too!

%INDEX% Discharging capacitors safely
%INDEX% Safety, discharging capacitors

%(END_NOTES)


