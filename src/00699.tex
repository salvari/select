
%(BEGIN_QUESTION)
% Copyright 2003, Tony R. Kuphaldt, released under the Creative Commons Attribution License (v 1.0)
% This means you may do almost anything with this work of mine, so long as you give me proper credit

Suppose someone wires a DPDT switch to an electric motor like this, hoping to achieve forward/reverse control:

$$\epsfbox{00699x01.eps}$$

Unfortunately, this switch arrangement will {\it not} reverse the motor!

\vskip 10pt

Explain why the motor will not reverse, and determine a correction to the circuit that will allow the switch to function as a forward/reverse control.

\underbar{file 00699}
%(END_QUESTION)





%(BEGIN_ANSWER)

This switching circuit will not reverse the motor, because it reverses polarity on {\it both} the armature and the field.

\vskip 10pt

Follow-up question: what does this fact tell us about the motor's ability to operate on alternating current (AC)?

%(END_ANSWER)





%(BEGIN_NOTES)

There is more than one solution for the reversing problem.  Discuss your students' solutions, and encourage the submission of multiple ideas!  One thing you might want to mention is that the field winding of a shunt-wound DC motor like this typically draws far less current than the armature winding (especially under full load).  Ask your students how this fact might influence their decision on how to re-design the switch circuit.

%INDEX% DC motor, reversing
%INDEX% Direction of rotation, DC motor
%INDEX% Reversing a DC motor
%INDEX% Rotational direction, DC motor

%(END_NOTES)


