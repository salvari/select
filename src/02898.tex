
%(BEGIN_QUESTION)
% Copyright 2005, Tony R. Kuphaldt, released under the Creative Commons Attribution License (v 1.0)
% This means you may do almost anything with this work of mine, so long as you give me proper credit

Some digital circuits are considered to have {\it active-low} inputs, while others have {\it active-high} inputs.  Explain what each of these terms means, and how we might identify which type of input(s) a digital circuit has.

\underbar{file 02898}
%(END_QUESTION)





%(BEGIN_ANSWER)

An "active-low" input is one where that particular gate function is activated or invoked on a {\it low} logic state.  Active-low inputs are identified by inversion bubbles (or inversion wedges) drawn at the IC input terminals.  For example, the Enable input (EN) for the following integrated circuit is active-low, meaning the chip is enabled when that input line is held at ground potential:

$$\epsfbox{02898x01.eps}$$

This S-R latch circuit has active-low preset ($\overline{\hbox{PRE}}$) and clear ($\overline{\hbox{CLR}}$) inputs, meaning the latch circuit will be preset and cleared when each of these inputs are grounded, respectively:

\goodbreak

$$\epsfbox{02898x02.eps}$$

Active-high inputs, conversely, engage their respective functions when brought to power supply rail ($V_{DD}$ or $V_{CC}$) potential.  As one might expect, an active-high input will {\it not} have an inversion bubble or wedge next to the input terminal.

\vskip 10pt

Challenge question: to the surprise of many students, there are a great number of digital logic circuit types built with active-low inputs.  Explain why.  Hint: most of these circuit types and functions were pioneered with TTL logic rather than CMOS logic.

%(END_ANSWER)





%(BEGIN_NOTES)

Active-low inputs tend to confuse many students, hence my unusually long and descriptive answer.

%INDEX% Active-high input, defined
%INDEX% Active-low input, defined

%(END_NOTES)


