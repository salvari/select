
%(BEGIN_QUESTION)
% Copyright 2003, Tony R. Kuphaldt, released under the Creative Commons Attribution License (v 1.0)
% This means you may do almost anything with this work of mine, so long as you give me proper credit

Shown here are two sine waves of equal frequency, superimposed on the same graph:

$$\epsfbox{00494x01.eps}$$

As accurately as possible, determine the amount of phase shift between the two waves, based on the divisions shown on the graph.

Also, plot a third sine wave that is the {\it sum} of the two sine waves shown.  Again, do this as accurately as possible, based on the divisions shown on the graph.  To give an example of how you might do this, observe the following illustration:

$$\epsfbox{00494x02.eps}$$

What is the peak value of the resultant (sum) sine-wave?  How does this compare with the peak values of the two original sine waves?

\underbar{file 00494}
%(END_QUESTION)





%(BEGIN_ANSWER)

The two (original) sine waves are 90$^{o}$ out of phase, one with a peak value of 4 volts, and the other with a peak value of 3 volts.

$$\epsfbox{00494x03.eps}$$

Together, they add to make a third sine wave with a peak value of 5 volts.

\vskip 10pt

Follow-up question: how is it possible that the sum of 3 and 4 makes 5?  Hint: you've probably seen something like this when studying trigonometry!

%(END_ANSWER)





%(BEGIN_NOTES)

Though it may seem a laborious exercise at first, the point of this question is for students to realize that the sum of two equal-frequency sine waves is another sine wave.  The peak values of 3 and 4, together with the phase-shift of 90$^{o}$, was no coincidence on my part when I wrote this question.  Where else in mathematics have students seen an example of the quantities 3 and 4 being combined (at 90 degrees) to make a quantity of 5?

%INDEX% Addition of two sine waves, with phase shift

%(END_NOTES)


