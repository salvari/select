
%(BEGIN_QUESTION)
% Copyright 2005, Tony R. Kuphaldt, released under the Creative Commons Attribution License (v 1.0)
% This means you may do almost anything with this work of mine, so long as you give me proper credit

Calculate the power dissipated by a 25 $\Omega$ resistor, when powered by a square-wave with a symmetrical amplitude of 100 volts and a frequency of 2 kHz, through a 0.22 $\mu$F capacitor:

$$\epsfbox{00652x01.eps}$$

No, I'm not asking you to calculate an infinite number of terms in the Fourier series -- that would be cruel and unusual.  Just calculate the power dissipated in the resistor by the 1st, 3rd, 5th, and 7th harmonics only.

\underbar{file 00652}
%(END_QUESTION)





%(BEGIN_ANSWER)

$P_{R(1st)} =$ 1.541 watts

$P_{R(3rd)} =$ 1.485 watts

$P_{R(5th)} =$ 1.384 watts

$P_{R(7th)} =$ 1.255 watts

$P_{R(1+3+5+7)} =$ 5.665 watts

%(END_ANSWER)





%(BEGIN_NOTES)

To calculate this power figure, students have to research the Fourier series for a square wave.  Many textbooks use square waves to introduce the subject of Fourier series, so this should not be difficult for students to find.

Ask your students how the real power dissipated by this resistor compares with the final figure of 5.665 watts.  Is the real power dissipation more, less, or equal to this figure?  If not equal, what would we have to do to arrive at a more precise figure?

%INDEX% Harmonic content of a square wave
%INDEX% Power dissipation, of square wave in an RC circuit

%(END_NOTES)


