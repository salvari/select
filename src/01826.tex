
%(BEGIN_QUESTION)
% Copyright 2003, Tony R. Kuphaldt, released under the Creative Commons Attribution License (v 1.0)
% This means you may do almost anything with this work of mine, so long as you give me proper credit

When AC power is initially applied to an electric motor (before the motor shaft has an opportunity to start moving), the motor "appears" to the AC power source to be a large inductor:

$$\epsfbox{01826x01.eps}$$

If the voltage of the 60 Hz AC power source is 480 volts RMS, and the motor initially draws 75 amps RMS when the double-pole single-throw switch closes, how much inductance ($L$) must the motor windings have?  Ignore any wire resistance, and assume the motor's only opposition to current in a locked-rotor condition is inductive reactance ($X_L$).

\underbar{file 01826}
%(END_QUESTION)





%(BEGIN_ANSWER)

$X_L = 16.98 \hbox{ mH}$

%(END_ANSWER)





%(BEGIN_NOTES)

In reality, motor winding resistance plays a substantial part in this sort of calculation, but I simplified things a bit just to give students a practical context for their introductory knowledge of inductive reactance.

%INDEX% Inductive reactance, of AC motor windings

%(END_NOTES)


