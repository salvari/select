
%(BEGIN_QUESTION)
% Copyright 2005, Tony R. Kuphaldt, released under the Creative Commons Attribution License (v 1.0)
% This means you may do almost anything with this work of mine, so long as you give me proper credit

One of the first popular operational amplifiers was manufactured by Philbrick Researches, and it was called the K2-W.  Built with two dual-triode vacuum tubes, its original schematic diagram looked like this:

$$\epsfbox{02590x01.eps}$$

To make this opamp circuit easier for modern students to understand, I'll substitute equivalent solid-state components for all tubes in the original design:

$$\epsfbox{02590x02.eps}$$

Explain the configuration (common-source, common-drain, or common-gate) of each transistor in the modernized schematic, identifying the function of each in the operational amplifier circuit.

\underbar{file 02590}
%(END_QUESTION)





%(BEGIN_ANSWER)

$Q_1$ and $Q_2$ form a {\it differential pair}, outputting a signal proportional to the difference in voltage between the two inputs.  $Q_3$ is a (bypassed) common-source voltage amplifier, while $Q_4$ is a source-follower (common-drain), providing voltage gain and current gain, respectively.

%(END_ANSWER)





%(BEGIN_NOTES)

The answer as given is incomplete.  One could elaborate more on the function of each transistor, and by doing so understand the original amplifier circuit a little better.  Explore this circuit with your students, challenging them to follow through the logic of the design, trying to figure out what the designer(s) intended.

This question also provides the opportunity to draw parallels between D-type MOSFET operation and the behavior of triode-type vacuum tubes.  As with D-type MOSFETs, triodes were "normally half-on" devices, whose plate-cathode current conduction could be enhanced or depleted by applying voltage to the grid (with respect to the cathode).

%INDEX% Opamp, internal schematic of Philbrick K2-W

%(END_NOTES)


