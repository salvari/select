
%(BEGIN_QUESTION)
% Copyright 2005, Tony R. Kuphaldt, released under the Creative Commons Attribution License (v 1.0)
% This means you may do almost anything with this work of mine, so long as you give me proper credit

Suppose a technician is checking the operation of the following electronic circuit:

$$\epsfbox{02733x01.eps}$$

She decides to measure the voltage on either side of resistor R1 with reference to ground, and obtains these readings:

$$\epsfbox{02733x02.eps}$$

On the top side of R1, the voltage with reference to ground is -5.04 volts.  On the bottom side of R1, the voltage with reference to ground is -1.87 volts.  The color code of resistor R1 is Yellow, Violet, Orange, Gold.  From this information, determine the following:

\medskip
\goodbreak
\item{$\bullet$} Voltage across R1 (between top to bottom):
\item{$\bullet$} Polarity (+ and -) of voltage across R1:
\item{$\bullet$} Current (magnitude) through R1:
\item{$\bullet$} Direction of current through R1:
\medskip

Additionally, explain how this technician would make each one of these determinations.  What rules or laws of electric circuits would she apply?

\underbar{file 02733}
%(END_QUESTION)





%(BEGIN_ANSWER)

\medskip
\item{$\bullet$} Voltage across R1 (between top to bottom): {\bf 3.17 volts}
\item{$\bullet$} Polarity (+ and -) of voltage across R1: {\bf (-) on top, (+) on bottom}
\item{$\bullet$} Current (magnitude) through R1: {\bf 67.45 $\mu$A}
\item{$\bullet$} Direction of current through R1: {\bf upward, following conventional flow}
\medskip

\vskip 10pt

Follow-up question: calculate the range of possible currents, given the specified tolerance of resistor R1 (67.45 $\mu$A assumes 0\% error).

\vskip 10pt

Challenge question: if you recognize the type of circuit this is (by the part number of the IC "chip": TL082), identify the voltage between pin 3 and ground.

%(END_ANSWER)





%(BEGIN_NOTES)

This is a good example of how Kirchhoff's Voltage Law is more than just an abstract tool for mathematical analysis -- it is also a powerful technique for practical circuit diagnosis.  Students must apply KVL to determine the voltage drop across R1, and then use Ohm's Law to calculate its current.

If students experience difficulty visualizing how KVL plays a part in the solution of this problem, show them this illustration:

$$\epsfbox{02733x03.eps}$$

By the way, the answer to the challenge question may only be realized if students recognize this circuit as a noninverting opamp voltage amplifier.  The voltage at pin 3 (noninverting input) will be the same as the voltage at pin 2 (inverting input): -1.87 volts.

%INDEX% Color code, resistor (4-band)
%INDEX% Voltmeter usage, to estimate current through a resistor

%(END_NOTES)


