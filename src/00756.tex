
%(BEGIN_QUESTION)
% Copyright 2003, Tony R. Kuphaldt, released under the Creative Commons Attribution License (v 1.0)
% This means you may do almost anything with this work of mine, so long as you give me proper credit

A common instrument used for measuring high AC currents in power systems is a {\it current transformer}, abbreviated "CT".  Current transformers usually take the form of a "donut," through which the current-carrying conductor passes:

$$\epsfbox{00756x01.eps}$$

The purpose of a current transformer is to create a secondary current that is a precise fraction of the primary current, for easier measurement of current in the power conductor.

Given this function, would current transformers be considered a "step-up" or "step-down" transformer?  Also, draw how the secondary windings of a current transformer are arranged around its toroidal core.

\underbar{file 00756}
%(END_QUESTION)





%(BEGIN_ANSWER)

From the perspective of voltage, which is usually how the terms "step-up" and "step-down" are referenced, a current transformer is a "step-up" transformer.  Its secondary windings are wound perpendicular to the magnetic flux path, as typical in all transformers.

%(END_ANSWER)





%(BEGIN_NOTES)

The question of whether the current transformer is a "step-down" or "step-up" has an important safety implication for students to realize.  Ask your students to describe what conditions might prove the most dangerous when working around current transformers, given their "step-up" nature with reference to voltage.

%INDEX% Current transformer, high current metering

%(END_NOTES)


