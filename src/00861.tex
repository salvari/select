
%(BEGIN_QUESTION)
% Copyright 2003, Tony R. Kuphaldt, released under the Creative Commons Attribution License (v 1.0)
% This means you may do almost anything with this work of mine, so long as you give me proper credit

Calculate the impedance value necessary to balance this AC bridge, expressing your answer in both polar and rectangular forms:

$$\epsfbox{00861x01.eps}$$

Also, describe what sort of device might be appropriate to serve as a "null detector" to indicate when bridge balance has been achieved, and where this device would be connected to in the bridge circuit.

\underbar{file 00861}
%(END_QUESTION)





%(BEGIN_ANSWER)

${\bf Z} =$ 337.6 $\Omega$ $\angle$ -90$^{o}$ (polar form)

${\bf Z} =$ 0 - j337.6 $\Omega$ (rectangular form)

\vskip 10pt

The simplest "null detector" for this type of AC bridge would be a sensitive pair of audio headphones, as 600 Hz is well within the audio range, and would be heard as a tone in the headphones.

\vskip 10pt

Follow-up question: what type and size of component will provide this exact amount of impedance at 600 Hz?

%(END_ANSWER)





%(BEGIN_NOTES)

So long as complex quantities are used, AC bridge circuits "balance" just the same way that DC bridge circuits balance.  Consequently, this is really nothing new for your students if they've already studied DC Wheatstone bridge circuits.

%INDEX% AC bridge circuit
%INDEX% Bridge circuit, AC

%(END_NOTES)


