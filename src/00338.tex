
%(BEGIN_QUESTION)
% Copyright 2003, Tony R. Kuphaldt, released under the Creative Commons Attribution License (v 1.0)
% This means you may do almost anything with this work of mine, so long as you give me proper credit

Potentiometers are manufactured in two different "tapers": {\it linear} and {\it audio}.  Linear taper potentiometers provide a direct, linear relationship between wiper position and resistance division, so that equal changes in wiper position result in equal changes in resistance.  Audio taper potentiometers provide a non-linear (logarithmic, to be exact) relationship between wiper position and resistance division, so that the same amount of wiper motion at one end of its range gives a much greater change in resistance than at the other end of its range.

$$\epsfbox{00338x01.eps}$$

Suppose you have a potentiometer, but do not know whether it has a linear or audio taper.  How could you determine this, using a meter? 

\underbar{file 00338}
%(END_QUESTION)





%(BEGIN_ANSWER)

A linear taper potentiometer will exhibit resistance measurements between the wiper and the other two terminals, proportional to the wiper position.

%(END_ANSWER)





%(BEGIN_NOTES)

Discuss with your students the purpose of an audio taper potentiometer: to provide logarithmically proportioned increases in audio power for volume control applications.  This is necessary for a "proportional" response when turning the volume knob on an audio amplifier, since human hearing is not linear, but logarithmic in its detection of loudness.  In order to generate a sound that the human ear perceives as twice as loud, ten times as much sound power is necessary.

A challenging question to ask your students is {\it which way} an audio taper potentiometer should be connected as a voltage divider in an audio amplifier circuit.  Being that audio taper potentiometers are non-symmetrical, it truly matters which way they are connected!

%INDEX% Linear taper potentiometer
%INDEX% Audio taper potentiometer
%INDEX% Potentiometer, linear vs. audio taper

%(END_NOTES)


