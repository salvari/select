
%(BEGIN_QUESTION)
% Copyright 2004, Tony R. Kuphaldt, released under the Creative Commons Attribution License (v 1.0)
% This means you may do almost anything with this work of mine, so long as you give me proper credit

The circuit shown here has a problem: the lamp does not light up, even though the AC power source is known to be good.  You know that the circuit used to work just fine, so it is designed properly.  Something in it has failed:

$$\epsfbox{02218x01.eps}$$

Identify one component or wire fault that could account for the lamp not lighting, and describe how you would use test equipment to verify that fault.

\vskip 10pt

\medskip
\item{$\bullet$} \underbar{One} failed wire or component in the circuit that could possibly account for the problem, and the type of fault (open or short) you suspect it would be.
\vskip 40pt
\item{$\bullet$} Identify the type of test measurement you would take on this circuit, and where you would take it (identify the test points you would measure between) to verify the suspected fault.
\medskip

\underbar{file 02218}
%(END_QUESTION)





%(BEGIN_ANSWER)

There are multiple possibilities here, and so I leave this up to you to determine!

%(END_ANSWER)





%(BEGIN_NOTES)

Of course, faults in this circuit having nothing to do with the transformer could also prevent the light bulb from lighting.  If time permits, it would be good to analyze a few failure scenarios with your students, challenging them to locate the source of the trouble as efficiently as possible.

%INDEX% Troubleshooting, transformer circuit

%(END_NOTES)


