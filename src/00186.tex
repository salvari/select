
%(BEGIN_QUESTION)
% Copyright 2003, Tony R. Kuphaldt, released under the Creative Commons Attribution License (v 1.0)
% This means you may do almost anything with this work of mine, so long as you give me proper credit

What is a {\it Leyden Jar}, and how it its construction similar to the construction of all {\it capacitors}?

\underbar{file 00186}
%(END_QUESTION)





%(BEGIN_ANSWER)

A "Leyden Jar" is a device used by early experimenters of static electricity to store electrical charges.  It is made from a glass jar, lined inside and outside with metal foil.  The glass insulates the two layers of metal foil from each other, and permits the storage of electric charge, manifested as a voltage between the two foil layers.

All capacitors share a common design feature of Leyden jars: the separation of two conductive plates by an insulating medium.

%(END_ANSWER)





%(BEGIN_NOTES)

Encourage your students to find a picture of a Leyden Jar, or even to build their own.  One can't help but notice the functional equivalence between a capacitor and a jar: storing charge versus storing substance!

A jar is not the only object which may be transformed into a capacitor.  Aluminum foil and paper sheets may also be used to make a rudimentary capacitor.  Have your students experiment with building their own capacitors, especially if they have access to a capacitance meter which may be used to compare the capacitance of different designs.

%INDEX% Capacitor construction
%INDEX% Leyden jar

%(END_NOTES)


