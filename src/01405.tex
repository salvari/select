
%(BEGIN_QUESTION)
% Copyright 2003, Tony R. Kuphaldt, released under the Creative Commons Attribution License (v 1.0)
% This means you may do almost anything with this work of mine, so long as you give me proper credit

Suppose you had an astable multivibrator circuit that output a very precise 1 Hz square-wave signal, but you had an application which requires a pulse once every {\it minute} rather than once every second.  Knowing that there are 60 seconds in a minute, can you think of a way to use digital counters to act as a "frequency divider" so that every 60 multivibrator pulses equates to 1 output pulse?

You don't have a divide-by-60 counter available, but you do have several divide-by-10 ("decade") counters at your disposal.  Engineer a solution using these counter units:

$$\epsfbox{01405x01.eps}$$

Note: assume these counter ICs have {\it asynchronous} resets.

\underbar{file 01405}
%(END_QUESTION)





%(BEGIN_ANSWER)

Cascade two decade counters together, with a NAND gate to decode when the output is equal to 60:

$$\epsfbox{01405x02.eps}$$

\vskip 10pt

Follow-up question: why can't we take the divide-by-60 pulse from the RCO output of the second counter, as we could with the divide-by-10 pulse from the first counter?

\vskip 10pt

Challenge question: re-design this circuit so that the output is a square wave with a duty cycle of 50\% ("high" for 30 seconds, then "low" for 30 seconds), rather than a narrow pulse every 60 seconds.

%(END_ANSWER)





%(BEGIN_NOTES)

Tell your students that counter circuits are quite often used as frequency dividers.  Discuss the challenge question with them, letting them propose and discuss multiple solutions to the problem.

The "note" in the question about the asynchronous nature of the counter reset inputs is very important, as synchronous-reset counter ICs would not behave the same.  Discuss this with your students, showing them how counters with synchronous reset inputs would yield a divide-by-61 ratio.

Incidentally, a divide-by-60 counter circuit is precisely what we would need to arrive at a 1 Hz pulse waveform from a 60 Hz powerline frequency signal, which is a neat "trick" for obtaining a low-speed clock of relatively good accuracy without requiring a crystal-controlled local oscillator.  (Where the "mains" power is 50 Hz instead of 60 Hz, you would need a divide-by-50 counter -- I know, I know . . .)  If time permits, ask your students to think of how they could condition the 60Hz sine-wave (120 volt!) standard powerline voltage into a 60 Hz square-wave pulse suitable for input into such a frequency divider/counter circuit.

%INDEX% Frequency division, digital
%INDEX% Modulus, digital counter

%(END_NOTES)


