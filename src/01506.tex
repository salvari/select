
%(BEGIN_QUESTION)
% Copyright 2003, Tony R. Kuphaldt, released under the Creative Commons Attribution License (v 1.0)
% This means you may do almost anything with this work of mine, so long as you give me proper credit

How do we tell which winding of the step-down power transformer is the primary, and which is the secondary, without actually powering it up with AC line power?  This is often an issue when students purchase cheap transformers that are unmarked and undocumented.

\underbar{file 01506}
%(END_QUESTION)





%(BEGIN_ANSWER)

Use an ohmmeter to measure the resistance of each winding.  The primary winding of a step-down transformer will contain a longer length of wire, of smaller cross-sectional area, than the secondary winding.  With these two factors, the resistance difference between windings should be obvious.

%(END_ANSWER)





%(BEGIN_NOTES)

Care should be taken if the transformer in question is a complete unknown.  It may be impossible to tell what the respective voltage ratings of the windings are if one does not even know whether the transformer was designed for line power (120 volts, 60 Hz AC in the United States) or not.  Small power transformers are easy enough to obtain from electronics parts suppliers and from scrap consumer electronic devices (stereos, computer accessories, etc.) that no one should have to take chances with a completely unknown transformer.

%(END_NOTES)


