
%(BEGIN_QUESTION)
% Copyright 2003, Tony R. Kuphaldt, released under the Creative Commons Attribution License (v 1.0)
% This means you may do almost anything with this work of mine, so long as you give me proper credit

Perhaps the most challenging aspect of interpreting ladder diagrams, for people more familiar with electronic schematic diagrams, is how electromechanical relays are represented.  Compare these two equivalent diagrams:

\vskip 10pt

First, the ladder diagram:

$$\epsfbox{00833x01.eps}$$

\vskip 10pt

Next, the schematic diagram:

$$\epsfbox{00833x02.eps}$$

Based on your observations of these two diagrams, explain how electromechanical relays are represented differently between ladder and schematic diagrams.

\underbar{file 00833}
%(END_QUESTION)





%(BEGIN_ANSWER)

One of the most significant differences is that in ladder diagrams, relay coils and relay contacts (the normally-open contact in this diagram shown as a capacitor-like symbol) need not be drawn near each other.

\vskip 10pt

Follow-up question: what do the two labels "L1" and "L2" represent?

%(END_ANSWER)





%(BEGIN_NOTES)

Discuss these diagrams with your students, noting any significant advantages and disadvantages of each convention.

In reference to the challenge question, the symbols "L1" and "L2" are very common designations for AC power conductors.  Be sure your students have researched this and know what these labels mean!

%INDEX% Ladder logic diagram

%(END_NOTES)


