
%(BEGIN_QUESTION)
% Copyright 2006, Tony R. Kuphaldt, released under the Creative Commons Attribution License (v 1.0)
% This means you may do almost anything with this work of mine, so long as you give me proper credit

Explain the operating principle of this analog-to-digital converter circuit, usually referred to as a {\it successive-approximation} converter:

$$\epsfbox{04015x01.eps}$$

Note: the successive-approximation register (SAR) is a special type of binary counting circuit which begins counting with the most-significant bit (MSB), then the next-less-significant bit, in order all the way down to the LSB.  At that point, it outputs a "high" signal at the "Complete" output terminal.  The operation of this register may be likened to the manual process of converting a decimal number to binary by "trial and fit" with the MSB first, through all the successive bits down to the LSB.

\underbar{file 04015}
%(END_QUESTION)





%(BEGIN_ANSWER)

The successive approximation register counts up and down as necessary to "zero in" on the analog input voltage, resulting in a binary output that locks into the correct value once every $n$ clock cycles, where $n$ is the number of bits the DAC inputs.

\vskip 10pt

Follow-up question: this form of ADC is much more effective at following fast-changing input signals than the {\it tracking} converter design.  Explain why.

%(END_ANSWER)





%(BEGIN_NOTES)

Have your students express the answer to this question in their own words, not just copying the answer I provide.  Aside from the flash converter, the tracking converter is one of the easiest ADC circuits to understand. 

%INDEX% ADC, successive-approximation
%INDEX% Successive-approximation converter, ADC

%(END_NOTES)


