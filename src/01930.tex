
%(BEGIN_QUESTION)
% Copyright 2003, Tony R. Kuphaldt, released under the Creative Commons Attribution License (v 1.0)
% This means you may do almost anything with this work of mine, so long as you give me proper credit

$$\epsfbox{01930x01.eps}$$

\underbar{file 01930}
\vfil \eject
%(END_QUESTION)





%(BEGIN_ANSWER)

Contrary to what you might think, the datasheet or cross-reference is not the "final authority" for checking your meter-based conclusions!  I have seen datasheets and cross-reference manuals wrong more than once!

%(END_ANSWER)





%(BEGIN_NOTES)

Identification of JFET terminals is a very important skill for technicians to have.  Most modern multimeters have a {\it diode check} feature which may be used to positively identify PN junction polarities, and this is what I recommend students use for identifying JFET terminals.

To make this a really good performance assessment, you might want to take several JFET's and scratch the identifying labels off, so students cannot refer to memory for pin identification (for instance, if they remember the pin assignments of a J309 because they use it so often).  Label these transistors with your own numbers ("1", "2", etc.) so {\it you} will know which is which, but not the students!

%INDEX% Assessment, performance-based (JFET terminal identification)

%(END_NOTES)


