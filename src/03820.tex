
%(BEGIN_QUESTION)
% Copyright 2005, Tony R. Kuphaldt, released under the Creative Commons Attribution License (v 1.0)
% This means you may do almost anything with this work of mine, so long as you give me proper credit

\centerline{\bf Animation: bridge rectifier circuit, with ideal diodes}

\vskip 10pt

{\it This question consists of a series of images (one per page) that form an animation.  Flip the pages with your fingers to view this animation (or click on the "next" button on your viewer) frame-by-frame.}

\vskip 10pt

The following animation shows a bridge rectifier circuit in action.  Here are some things to look for:

\medskip
\goodbreak
\item{$\bullet$} How many diodes are in conduction at any given time?
\item{$\bullet$} Is the output half-wave or full-wave?
\medskip

\vfil \eject
$$\epsfbox{03820x00.eps}$$

\vfil \eject
$$\epsfbox{03820x01.eps}$$

\vfil \eject
$$\epsfbox{03820x02.eps}$$

\vfil \eject
$$\epsfbox{03820x03.eps}$$

\vfil \eject
$$\epsfbox{03820x04.eps}$$

\vfil \eject
$$\epsfbox{03820x05.eps}$$

\vfil \eject
$$\epsfbox{03820x06.eps}$$

\vfil \eject
$$\epsfbox{03820x07.eps}$$

\vfil \eject
$$\epsfbox{03820x08.eps}$$

\vfil \eject
$$\epsfbox{03820x09.eps}$$

\vfil \eject
$$\epsfbox{03820x10.eps}$$

\vfil \eject
$$\epsfbox{03820x11.eps}$$

\vfil \eject
$$\epsfbox{03820x12.eps}$$

\vfil \eject
$$\epsfbox{03820x13.eps}$$

\vfil \eject
$$\epsfbox{03820x14.eps}$$

\vfil \eject
$$\epsfbox{03820x15.eps}$$

\vfil \eject
$$\epsfbox{03820x16.eps}$$

\vfil \eject
$$\epsfbox{03820x17.eps}$$

\vfil \eject
$$\epsfbox{03820x18.eps}$$

\vfil \eject
$$\epsfbox{03820x19.eps}$$

\vfil \eject
$$\epsfbox{03820x20.eps}$$

\underbar{file 03820}

\vfil \eject

%(END_QUESTION)





%(BEGIN_ANSWER)

Nothing to note here.

%(END_ANSWER)





%(BEGIN_NOTES)

The purpose of this animation is to let students study the behavior of this amplifier circuit and reach their own conclusions.  Similar to experimentation in the lab, except that here all the data collection is done visually rather than through the use of test equipment, and the students are able to "see" things that are invisible in real life.

%INDEX% Animation, bridge rectifier (ideal diodes)

%(END_NOTES)


