
%(BEGIN_QUESTION)
% Copyright 2003, Tony R. Kuphaldt, released under the Creative Commons Attribution License (v 1.0)
% This means you may do almost anything with this work of mine, so long as you give me proper credit

What would happen to the voltage drops across each resistor in this circuit if either resistor R2 or R3 were to fail open?

$$\epsfbox{01772x01.eps}$$

\underbar{file 01772}
%(END_QUESTION)





%(BEGIN_ANSWER)

If either resistor R2 or R3 were to fail open (internally), the voltage across both R2 {\it and} R3 would increase (but not to full battery voltage), leaving less voltage dropped across R1.

\vskip 10pt

Follow-up question: explain why it doesn't matter which resistor (R2 or R3) fails open -- the {\it qualitative} results for voltage (voltage increasing or decreasing, but not by any specific amount) will be the same.

%(END_ANSWER)





%(BEGIN_NOTES)

I have found in teaching that many students loathe qualitative analysis, because they cannot let their calculators do the thinking for them.  However, being able to judge whether a circuit parameter will increase, decrease, or remain the same after a component fault is an {\it essential} skill for proficient troubleshooting.

%INDEX% Series-parallel circuit; effect of open fault

%(END_NOTES)


