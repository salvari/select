
%(BEGIN_QUESTION)
% Copyright 2003, Tony R. Kuphaldt, released under the Creative Commons Attribution License (v 1.0)
% This means you may do almost anything with this work of mine, so long as you give me proper credit

Draw the necessary wire connections so that bridging the two contact points with your finger (creating a high-resistance connection between those points) will turn the light bulb on:

\goodbreak

\vskip 40pt

$$\epsfbox{00445x01.eps}$$

\vskip 30pt

\underbar{file 00445}
%(END_QUESTION)





%(BEGIN_ANSWER)

$$\epsfbox{00445x02.eps}$$

%(END_ANSWER)





%(BEGIN_NOTES)

Once students learn to identify the two current paths (base versus collector), especially the proper directions of current for each, the interconnections become much easier to determine.

Some students may place the light bulb on the emitter terminal of the transistor, in a common-collector configuration.  This is not recommended, since it places the light bulb in series with the controlling (base) current path, and this will have the effect of impeding base current, and therefore the controlled (light bulb) current.  Given the very high electrical resistance of human skin, this circuit needs all the gain we can possibly muster! 

This circuit works well if an LED is substituted for the incandescent lamp.

%INDEX% Transistor switch circuit (BJT)

%(END_NOTES)


