
%(BEGIN_QUESTION)
% Copyright 2005, Tony R. Kuphaldt, released under the Creative Commons Attribution License (v 1.0)
% This means you may do almost anything with this work of mine, so long as you give me proper credit

Convert the following amplifier gains (either power, voltage, or current gain ratios) into gains expressed in the unit of decibels (dB):

\medskip
\goodbreak
\item{$\bullet$} $A_P$ = 25 ; $A_{P(dB)}$ = 
\item{$\bullet$} $A_V$ = 10 ; $A_{V(dB)}$ = 
\item{$\bullet$} $A_I$ = 37 ; $A_{I(dB)}$ = 
\item{$\bullet$} $A_P$ = 150 ; $A_{P(dB)}$ = 
\item{$\bullet$} $A_I$ = 41 ; $A_{I(dB)}$ = 
\item{$\bullet$} $A_V$ = 3.4 ; $A_{V(dB)}$ = 
\item{$\bullet$} $A_P$ = 18 ; $A_{P(dB)}$ = 
\item{$\bullet$} $A_V$ = 100 ; $A_{V(dB)}$ = 
\medskip

\underbar{file 02447}
%(END_QUESTION)





%(BEGIN_ANSWER)

\medskip
\goodbreak
\item{$\bullet$} $A_P$ = 25 ; $A_{P(dB)}$ = 13.98 dB
\item{$\bullet$} $A_V$ = 10 ; $A_{V(dB)}$ = 20 dB
\item{$\bullet$} $A_I$ = 37 ; $A_{I(dB)}$ = 31.36 dB
\item{$\bullet$} $A_P$ = 150 ; $A_{P(dB)}$ = 21.76 dB
\item{$\bullet$} $A_I$ = 41 ; $A_{I(dB)}$ = 32.26 dB
\item{$\bullet$} $A_V$ = 3.4 ; $A_{V(dB)}$ = 10.63 dB
\item{$\bullet$} $A_P$ = 18 ; $A_{P(dB)}$ = 12.55 dB
\item{$\bullet$} $A_V$ = 100 ; $A_{V(dB)}$ = 40 dB
\medskip

%(END_ANSWER)





%(BEGIN_NOTES)

Nothing special here, just straightforward ratio-to-decibel calculations.  Have your students share and discuss the steps necessary to do all these conversions.

%INDEX% Decibel (gain) calculations
%INDEX% Gain, converting ratios into decibels

%(END_NOTES)


