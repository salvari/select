
%(BEGIN_QUESTION)
% Copyright 2003, Tony R. Kuphaldt, released under the Creative Commons Attribution License (v 1.0)
% This means you may do almost anything with this work of mine, so long as you give me proper credit

Plot the output waveform of a passive differentiator circuit, assuming the input is a symmetrical square wave and the circuit's RC time constant is about one-fifth of the square wave's pulse width:

$$\epsfbox{02052x01.eps}$$

$$\epsfbox{02052x02.eps}$$

\underbar{file 02052}
%(END_QUESTION)





%(BEGIN_ANSWER)

$$\epsfbox{02052x03.eps}$$

\vskip 10pt

Follow-up question \#1: what would we have to change in this passive differentiator circuit to make the output more closely resemble ideal differentiation?

\vskip 10pt

Follow-up question \#2: explain how it is possible that the differentiator's output waveform has a greater peak amplitude than the input (square) waveform.

%(END_ANSWER)





%(BEGIN_NOTES)

Ask students to contrast the behavior of this passive differentiator circuit against that of a perfect differentiator (with $\tau$ = 0).  What should the derivative plot of a square wave look like?

%INDEX% Differentiator circuit, passive
%INDEX% Passive differentiator circuit

%(END_NOTES)


