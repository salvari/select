
%(BEGIN_QUESTION)
% Copyright 2003, Tony R. Kuphaldt, released under the Creative Commons Attribution License (v 1.0)
% This means you may do almost anything with this work of mine, so long as you give me proper credit

Certain substances are known to physically deform (shorten or lengthen) when an electrical voltage is applied across their length.  The effect is known as {\it piezoelectricity}.  What types of substances are known to be piezoelectric?  Identify some possible applications of this phenomenon.

\underbar{file 00221}
%(END_QUESTION)





%(BEGIN_ANSWER)

Certain crystal substances, such as {\it quartz}, are known to be piezoelectric.

%(END_ANSWER)





%(BEGIN_NOTES)

As your students will no doubt discover in their research, piezoelectricity is a two-way effect: physical motion from electricity and visa-versa.  It will be clear how well they did their research by the types of applications they think of for the "electricity-to-motion" form of piezoelectricity, given the typical physical scale of piezoelectric displacements.

%INDEX% Piezoelectricity, conceptual

%(END_NOTES)


