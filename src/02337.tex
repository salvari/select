
%(BEGIN_QUESTION)
% Copyright 2005, Tony R. Kuphaldt, released under the Creative Commons Attribution License (v 1.0)
% This means you may do almost anything with this work of mine, so long as you give me proper credit

This is a switching circuit for video signals, or any other radio-frequency (RF), low-amplitude AC signals that one might need to switch on and off for a variety of different applications:

$$\epsfbox{02337x01.eps}$$

Identify the "on" and "off" states of the three JFETs in this circuit with the switch in the position shown in the schematic, and also determine whether the switch position shown in the diagram is for "passing" or "blocking" the RF signal from input to output.

\underbar{file 02337}
%(END_QUESTION)





%(BEGIN_ANSWER)

$$\epsfbox{02337x02.eps}$$

With the switch in the position shown, the circuit will {\it block} the RF signal from getting to the output.

%(END_ANSWER)





%(BEGIN_NOTES)

Ask your students to explain the problem-solving strategy(ies) they used to derive the correct answer for this question.  How were they able to tell whether the transistors would be on or off?  Ask them to explain the purpose of the 1 M$\Omega$ resistors.  What might happen without them?

If time permits, discuss with your students the design of this circuit: why two "pass" transistors and one "shunt" transistor?  How critical is the magnitude of the -V switching control voltage?

The idea for this circuit came from a National Semiconductor application note on JFET circuits: {\it Application Note 32, February 1970, page 13}.

%INDEX% Transistor switch circuit (JFET), for AC signals

%(END_NOTES)


