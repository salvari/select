
%(BEGIN_QUESTION)
% Copyright 2004, Tony R. Kuphaldt, released under the Creative Commons Attribution License (v 1.0)
% This means you may do almost anything with this work of mine, so long as you give me proper credit

Calculate the amount of current through this impedance, and express your answer in both polar and rectangular forms:

$$\epsfbox{02119x01.eps}$$

\underbar{file 02119}
%(END_QUESTION)





%(BEGIN_ANSWER)

$I$ = 545.45 $\mu$A $\angle$ 21$^{o}$

$I$ = 509.23 $\mu$A + $j$195.47 $\mu$A

\vskip 10pt

Follow-up question: which of these two forms is more meaningful when comparing against the indication of an AC ammeter?  Explain why.

%(END_ANSWER)





%(BEGIN_NOTES)

It is important for your students to realize that the two forms given in the answer are really the same quantity, just expressed differently.  If it helps, draw a phasor diagram showing how they are equivalent.

This is really nothing more than an exercise in complex number arithmetic.  Have your students present their solution methods on the board for all to see, and discuss how Ohm's Law and complex number formats (rectangular versus polar) relate to one another in this question.

%INDEX% AC circuit; calculation for current given impedance in complex form

%(END_NOTES)


