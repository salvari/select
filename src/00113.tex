
%(BEGIN_QUESTION)
% Copyright 2003, Tony R. Kuphaldt, released under the Creative Commons Attribution License (v 1.0)
% This means you may do almost anything with this work of mine, so long as you give me proper credit

It is much easier to electrically "charge" an atom than it is to alter its chemical identity (say, from lead into gold).  What does this fact indicate about the relative mobility of the elementary particles within an atom?

\underbar{file 00113}
%(END_QUESTION)





%(BEGIN_ANSWER)

Electrons are much easier to remove from or add to an atom than protons are.  The reason for this is also the solution to the paradox of why protons bind together tightly in the nucleus of an atom despite their identical electrical charges.

%(END_ANSWER)





%(BEGIN_NOTES)

Discuss with your students the importance of this fact: that electrons may be added to or taken from an atom rather easily, but that protons (and neutrons for that matter) are very tightly "bound" within an atom.  What might atoms behave like if their protons were not so tightly bound as they are?  

We know what happens to the electrons of some atoms when substances are rubbed together.  What might happen to those substances if protons were not as tightly bound together as they are?

%INDEX% Mobility of subatomic particles
%INDEX% Particles, subatomic
%INDEX% Subatomic particles
%INDEX% Proton
%INDEX% Neutron
%INDEX% Electron

%(END_NOTES)


