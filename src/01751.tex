
%(BEGIN_QUESTION)
% Copyright 2003, Tony R. Kuphaldt, released under the Creative Commons Attribution License (v 1.0)
% This means you may do almost anything with this work of mine, so long as you give me proper credit

A {\it thermistor} is a special resistor that dramatically changes resistance with changes in temperature.  Consider the circuit shown below, with two identical thermistors:

$$\epsfbox{01751x01.eps}$$

The "+t$^{o}$" label in each one shows that they both have positive $\alpha$ coefficients.

How much voltage would you expect the voltmeter to register when the two thermistors are at the exact same temperature?  Which thermistor would have to become hotter in order to cause the voltmeter to read a significant {\it negative} voltage?

\underbar{file 01751}
%(END_QUESTION)





%(BEGIN_ANSWER)

If the two thermistors are at equal temperature, the voltmeter should register 0 volts.  To get the voltmeter to register negative, the left-hand thermistor would have to be warmer than the right-hand thermistor.

%(END_ANSWER)





%(BEGIN_NOTES)

This circuit may be viewed from the perspective of it being two voltage dividers, or from the perspective of being a current divider.  Either way, it is a good exercise for you and your students to explore how it functions.

%INDEX% Temperature bridge circuit
%INDEX% Thermistor

%(END_NOTES)


