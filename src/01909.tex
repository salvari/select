
%(BEGIN_QUESTION)
% Copyright 2003, Tony R. Kuphaldt, released under the Creative Commons Attribution License (v 1.0)
% This means you may do almost anything with this work of mine, so long as you give me proper credit

A student is experimenting with an oscilloscope, learning how to use the triggering control.  While turning the trigger level knob clockwise, the student sees the effect it has on the waveform's position on the screen.  Then, with an additional twist of the level knob, the waveform completely disappears.  Now there is absolutely nothing shown on the screen!  Turning the level knob the other way (counter-clockwise), the waveform suddenly appears on the screen again.

Based on the described behavior, does this student have the oscilloscope trigger control set on {\it Auto} mode, or on {\it Norm} mode?  What would the oscilloscope do if the other triggering mode were set?

\underbar{file 01909}
%(END_QUESTION)





%(BEGIN_ANSWER)

This student's oscilloscope is set on the {\it Norm} mode.  If it were set on the {\it Auto} mode, the trace would default to "free-running" if ever the trigger level were set above or below the waveform's amplitude.  Instead of completely disappearing, the waveform would scroll horizontally and not "stand still" if the trigger level were set too high or too low.

%(END_ANSWER)





%(BEGIN_NOTES)

Ask your students to explain which mode they think the oscilloscope should ordinarily be set in for general-purpose use.

%INDEX% Oscilloscope, "auto" versus "normal" triggering

%(END_NOTES)


