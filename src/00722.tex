
%(BEGIN_QUESTION)
% Copyright 2003, Tony R. Kuphaldt, released under the Creative Commons Attribution License (v 1.0)
% This means you may do almost anything with this work of mine, so long as you give me proper credit

Fundamentally, what single factor in a voltmeter's design establishes its ohms-per-volt sensitivity rating?

\underbar{file 00722}
%(END_QUESTION)





%(BEGIN_ANSWER)

If your answer is, "the value of the series resistor(s)," you are incorrect.

%(END_ANSWER)





%(BEGIN_NOTES)

Students' immediate impression is that the range resistor value must establish the sensitivity rating, because they see the resistor as having the most impact on input resistance.  However, some quick calculations with different range resistor values prove otherwise!  Meter sensitivity is independent of any series-connected range resistor values.

You might want to ask your students why meter movement coil resistance is not a factor in determining voltmeter sensitivity.  Challenge your students with setting up sample circuit problems to prove the irrelevance of coil resistance on voltmeter sensitivity.  Let them figure out how to set up the problems, rather than you setting up the problems for them!

%(END_NOTES)


