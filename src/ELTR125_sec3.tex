
\centerline{\bf ELTR 125 (Semiconductors 2), section 3} \bigskip 
 
\vskip 10pt

\noindent
{\bf Recommended schedule}

\vskip 5pt

%%%%%%%%%%%%%%%
\hrule \vskip 5pt
\noindent
\underbar{Day 1}

\hskip 10pt Topics: {\it Basic oscillator theory and relaxation oscillator circuits}
 
\hskip 10pt Questions: {\it 1 through 10}
 
\hskip 10pt Lab Exercise: {\it BJT multivibrator circuit, astable (question 51)}
 
\vskip 10pt
%%%%%%%%%%%%%%%
\hrule \vskip 5pt
\noindent
\underbar{Day 2}

\hskip 10pt Topics: {\it Phase-shift and resonant oscillator circuits}
 
\hskip 10pt Questions: {\it 11 through 20}
 
\hskip 10pt Lab Exercise: {\it Wien bridge oscillator, BJT (question 52)}
 
 
\vskip 10pt
%%%%%%%%%%%%%%%
\hrule \vskip 5pt
\noindent
\underbar{Day 3}

\hskip 10pt Topics: {\it Harmonics}
 
\hskip 10pt Questions: {\it 21 through 30}
 
\hskip 10pt Lab Exercise: {\it Colpitts oscillator, BJT (question 53)}
 
%INSTRUCTOR \hskip 10pt {\bf Demo: Use graphing calculator to synthesize square wave from sinusoidal harmonics}

%INSTRUCTOR \hskip 10pt {\bf Demo: Show harmonics using a spectrum analyzer and function generator}

%INSTRUCTOR \hskip 10pt {\bf Demo: Show harmonics in power-line signal using a spectrum analyzer and transformer}

%INSTRUCTOR \hskip 10pt {\bf Demo: Show example of spectrum plot from an amplifier datasheet}

\vskip 10pt
%%%%%%%%%%%%%%%
\hrule \vskip 5pt
\noindent
\underbar{Day 4}

\hskip 10pt Topics: {\it Fundamentals of radio, amplitude modulation, and frequency modulation} (optional)
 
\hskip 10pt Questions: {\it 31 through 50}
 
\hskip 10pt Lab Exercise: {\it Troubleshooting practice (oscillator/amplifier circuit -- question 55)}

\hskip 10pt Just for fun (not required): {\it AM radio transmitter (question 54)}
 
%INSTRUCTOR \hskip 10pt {\bf Demo: Use signal generator with AM function to broadcast audio tone to radio}

\vskip 10pt
%%%%%%%%%%%%%%%
\hrule \vskip 5pt
\noindent
\underbar{Day 5}

\hskip 10pt Exam 3: {\it includes Oscillator Circuit performance assessment}
 
\hskip 10pt {\bf Troubleshooting Assessment due:} {\it oscillator/amplifier circuit (question 55)}
 
\hskip 10pt Question 56: Troubleshooting log
 
\hskip 10pt Question 57: Sample troubleshooting assessment grading criteria
 
\vskip 10pt
%%%%%%%%%%%%%%%%
\hrule \vskip 5pt
\noindent
\underbar{Troubleshooting practice problems}

\hskip 10pt Questions: {\it 58 through 67}
 
\vskip 10pt
%%%%%%%%%%%%%%%
\hrule \vskip 5pt
\noindent
\underbar{General concept practice and challenge problems}

\hskip 10pt Questions: {\it 68 through the end of the worksheet}
 
\vskip 10pt
%%%%%%%%%%%%%%%





\vfil \eject

\centerline{\bf ELTR 125 (Semiconductors 2), section 3} \bigskip 
 
\vskip 10pt

\noindent
{\bf Skill standards addressed by this course section}

\vskip 5pt

%%%%%%%%%%%%%%%
\hrule \vskip 10pt
\noindent
\underbar{EIA {\it Raising the Standard; Electronics Technician Skills for Today and Tomorrow}, June 1994}

\vskip 5pt

\item{\bf C} {\bf Technical Skills -- AC circuits}
\item{\bf C.02} Demonstrate an understanding of the properties of an AC signal.
\item{\bf C.03} Demonstrate an understanding of the principles of operation and characteristics of sinusoidal and non-sinusoidal wave forms.

\vskip 5pt

\medskip
\item{\bf E} {\bf Technical Skills -- Analog Circuits}
\item{\bf E.20} Understand principles and operations of sinusoidal and non-sinusoidal oscillator circuits.
\item{\bf E.21} Troubleshoot and repair sinusoidal and non-sinusoidal oscillator circuits.
\item{\bf E.27} Understand principles and operations of signal modulation systems (AM, FM, stereo).  {\it Partially met -- basic AM and FM only.}
\medskip

\vskip 5pt

\medskip
\item{\bf B} {\bf Basic and Practical Skills -- Communicating on the Job}
\item{\bf B.01} Use effective written and other communication skills.  {\it Met by group discussion and completion of labwork.}
\item{\bf B.03} Employ appropriate skills for gathering and retaining information.  {\it Met by research and preparation prior to group discussion.}
\item{\bf B.04} Interpret written, graphic, and oral instructions.  {\it Met by completion of labwork.}
\item{\bf B.06} Use language appropriate to the situation.  {\it Met by group discussion and in explaining completed labwork.}
\item{\bf B.07} Participate in meetings in a positive and constructive manner.  {\it Met by group discussion.}
\item{\bf B.08} Use job-related terminology.  {\it Met by group discussion and in explaining completed labwork.}
\item{\bf B.10} Document work projects, procedures, tests, and equipment failures.  {\it Met by project construction and/or troubleshooting assessments.}
\item{\bf C} {\bf Basic and Practical Skills -- Solving Problems and Critical Thinking}
\item{\bf C.01} Identify the problem.  {\it Met by research and preparation prior to group discussion.}
\item{\bf C.03} Identify available solutions and their impact including evaluating credibility of information, and locating information.  {\it Met by research and preparation prior to group discussion.}
\item{\bf C.07} Organize personal workloads.  {\it Met by daily labwork, preparatory research, and project management.}
\item{\bf C.08} Participate in brainstorming sessions to generate new ideas and solve problems.  {\it Met by group discussion.}
\item{\bf D} {\bf Basic and Practical Skills -- Reading}
\item{\bf D.01} Read and apply various sources of technical information (e.g. manufacturer literature, codes, and regulations).  {\it Met by research and preparation prior to group discussion.}
\item{\bf E} {\bf Basic and Practical Skills -- Proficiency in Mathematics}
\item{\bf E.01} Determine if a solution is reasonable.
\item{\bf E.02} Demonstrate ability to use a simple electronic calculator.
\item{\bf E.05} Solve problems and [sic] make applications involving integers, fractions, decimals, percentages, and ratios using order of operations.
\item{\bf E.06} Translate written and/or verbal statements into mathematical expressions.
\item{\bf E.09} Read scale on measurement device(s) and make interpolations where appropriate.  {\it Met by oscilloscope usage.}
\item{\bf E.12} Interpret and use tables, charts, maps, and/or graphs.
\item{\bf E.13} Identify patterns, note trends, and/or draw conclusions from tables, charts, maps, and/or graphs.
\item{\bf E.15} Simplify and solve algebraic expressions and formulas.
\item{\bf E.16} Select and use formulas appropriately.
\item{\bf E.17} Understand and use scientific notation.
\medskip

%%%%%%%%%%%%%%%




\vfil \eject

\centerline{\bf ELTR 125 (Semiconductors 2), section 3} \bigskip 
 
\vskip 10pt

\noindent
{\bf Common areas of confusion for students}

\vskip 5pt

%%%%%%%%%%%%%%%
\hrule \vskip 5pt

\vskip 10pt

\noindent
{\bf Difficult concept: } {\it Calculating phase shift of RC network.}

The only real difficulty here is the lapse in time between when most students study RC circuit analysis and the time they study phase-shift oscillator circuits.  Calculating the phase shift of a series RC circuit is as simple as drawing its impedance triangle, and then properly identifying which two sides represent the input (total) and output voltages in order to identify which angle you must calculate.

\vskip 10pt

\noindent
{\bf Difficult concept: } {\it Fourier analysis.}

No doubt about it, Fourier analysis is a strange concept to understand.  Strange, but incredibly useful!  While it is relatively easy to grasp the principle that we may create a square-shaped wave (or any other symmetrical waveshape) by mixing together the right combinations of sine waves at different frequencies and amplitudes, it is far from obvious that {\it any} periodic waveform may be decomposed into a series of sinusoidal waves the same way.  The practical upshot of this is that is it possible to consider very complex waveshapes as being nothing more than a bunch of sine waves added together.  Since sine waves are easy to analyze in the context of electric circuits, this means we have a way of simplifying what would otherwise be a dauntingly complex problem: analyzing how circuits respond to non-sinusoidal waveforms.

The actual "nuts and bolts" of Fourier analysis is highly mathematical and well beyond the scope of this course.  Right now all I want you to grasp is the concept and significance of equivalence between arbitrary waveshapes and series of sine waves.

A great way to experience this equivalence is to play with a digital oscilloscope with a built-in spectrum analyzer.  By introducing different wave-shape signals to the input and switching back and forth between the time-domain (scope) and frequency-domain (spectrum) displays, you may begin to see patterns that will enlighten your understanding.




