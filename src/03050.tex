
%(BEGIN_QUESTION)
% Copyright 2005, Tony R. Kuphaldt, released under the Creative Commons Attribution License (v 1.0)
% This means you may do almost anything with this work of mine, so long as you give me proper credit

Some programmable logic devices (and PROM memory devices as well) use tiny fuses which are intentionally "blown" in specific patterns to represent the desired program.  Programming a device by blowing tiny fuses inside of it carries certain advantages and disadvantages -- describe what some of these are.

\underbar{file 03050}
%(END_QUESTION)





%(BEGIN_ANSWER)

Certainly, the stored program will be nonvolatile, but it will also be read-only.  This is why fuse-programmed devices are sometimes called "OTP".  (I'll let you research what that acronym means.)

%(END_ANSWER)





%(BEGIN_NOTES)

It is interesting to mention that some programmable devices (Texas Instruments' TIBPAL series, for example) are built with a "security fuse" inside which prevents anyone from reverse-engineering a programmed chip!

%INDEX% Fuses, used as programming elements in programmable logic devices
%INDEX% Programmable logic, using internal fuses for programming

%(END_NOTES)


