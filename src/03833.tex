
%(BEGIN_QUESTION)
% Copyright 2006, Tony R. Kuphaldt, released under the Creative Commons Attribution License (v 1.0)
% This means you may do almost anything with this work of mine, so long as you give me proper credit

One way to think of logic gate types is to consider what input states guarantee a certain output state.  For example, we could describe the function of an AND gate as such:

$$\hbox{\it Any low input guarantees a low output.}$$

Identify what type of gate is represented by each of the following phrases:

\medskip
\item{$\bullet$} Any low input guarantees a high output.
\item{$\bullet$} Any high input guarantees a low output.
\item{$\bullet$} Any high input guarantees a high output.
\item{$\bullet$} Any difference in the inputs guarantees a high output.
\item{$\bullet$} Any difference in the inputs guarantees a low output.
\medskip

Also, explain how this sort of gate identification could be useful in troubleshooting logic gate circuits.

\underbar{file 03833}
%(END_QUESTION)





%(BEGIN_ANSWER)

\medskip
\item{$\bullet$} Any low input guarantees a high output: {\bf NAND} gate.
\item{$\bullet$} Any high input guarantees a low output: {\bf NOR} gate.
\item{$\bullet$} Any high input guarantees a high output: {\bf OR} gate.
\item{$\bullet$} Any difference in the inputs guarantees a high output: {\bf XOR} gate.
\item{$\bullet$} Any difference in the inputs guarantees a low output: {\bf XNOR} gate.
\medskip

%(END_ANSWER)





%(BEGIN_NOTES)

This is a very useful way to think of the different logic gate types, as often you are faced with a choice of which gate type to use for a specific function in a digital circuit based on a requirement cast in these terms ("Any {\it blank} input guarantees a {\it blank} output").

For example, we might need a gate to perform a "disable" function for a digital signal:

$$\epsfbox{03833x01.eps}$$

Considered in terms of what input state forces a low output, the choice to use an AND gate becomes obvious.

%INDEX% Truth tables for different gate types

%(END_NOTES)


