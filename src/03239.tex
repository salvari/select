
%(BEGIN_QUESTION)
% Copyright 2005, Tony R. Kuphaldt, released under the Creative Commons Attribution License (v 1.0)
% This means you may do almost anything with this work of mine, so long as you give me proper credit

Resistive voltage dividers are very useful and popular circuits.  However, it should be realized that their output voltages "sag" under load:

$$\epsfbox{03239x01.eps}$$

Just how much a voltage divider's output will sag under a given load may be a very important question in some applications.  Take for instance the following application where we are using a resistive voltage divider to supply an engine sensor with reduced voltage (8 volts) from the 12 volt battery potential in the automobile:

$$\epsfbox{03239x02.eps}$$

If the sensor draws no current ($I_{sensor}$ = 0 mA), then the voltage across the sensor supply terminals will be 8 volts.  However, if we were asked to predict the voltage across the sensor supply terminals for a variety of different sensor current conditions, we would be faced with a much more complex problem:

% No blank lines allowed between lines of an \halign structure!
% I use comments (%) instead, so that TeX doesn't choke.

$$\vbox{\offinterlineskip
\halign{\strut
\vrule \quad\hfil # \ \hfil & 
\vrule \quad\hfil # \ \hfil \vrule \cr
\noalign{\hrule}
%
% First row
Sensor current ($I_{sensor}$) & Sensor supply voltage \cr
%
\noalign{\hrule}
%
% Second row
0 mA & 8 volts \cr
%
\noalign{\hrule}
%
% Third row
1 mA &  \cr
%
\noalign{\hrule}
%
% Fourth row
2 mA &  \cr
%
\noalign{\hrule}
%
% Fifth row
3 mA &  \cr
%
\noalign{\hrule}
%
% Sixth row
4 mA &  \cr
%
\noalign{\hrule}
%
% Seventh row
5 mA &  \cr
%
\noalign{\hrule}
} % End of \halign 
}$$ % End of \vbox

One technique we could use to simplify this problem is to reduce the voltage divider resistor network into a Th\'evenin equivalent circuit.  With the three-resistor divider reduced to a single resistor in series with an equivalent voltage source, the calculations for sensor supply voltage become much simpler.

Show how this could be done, then complete the table of sensor supply voltages shown above.

\underbar{file 03239}
%(END_QUESTION)





%(BEGIN_ANSWER)

$$\epsfbox{03239x03.eps}$$

% No blank lines allowed between lines of an \halign structure!
% I use comments (%) instead, so that TeX doesn't choke.

$$\vbox{\offinterlineskip
\halign{\strut
\vrule \quad\hfil # \ \hfil & 
\vrule \quad\hfil # \ \hfil \vrule \cr
\noalign{\hrule}
%
% First row
Sensor current ($I_{sensor}$) & Sensor supply voltage \cr
%
\noalign{\hrule}
%
% Second row
0 mA & 8 volts \cr
%
\noalign{\hrule}
%
% Third row
1 mA & 7.333 volts \cr
%
\noalign{\hrule}
%
% Fourth row
2 mA & 6.667 volts \cr
%
\noalign{\hrule}
%
% Fifth row
3 mA & 6 volts \cr
%
\noalign{\hrule}
%
% Sixth row
4 mA & 5.333 volts \cr
%
\noalign{\hrule}
%
% Seventh row
5 mA & 4.667 volts \cr
%
\noalign{\hrule}
} % End of \halign 
}$$ % End of \vbox

\vskip 10pt

Follow-up question: if we cannot allow the sensor supply voltage to fall below 6.5 volts, what is the maximum amount of current it may draw from this voltage divider circuit?

\vskip 10pt

Challenge question: figure out how to solve for these same voltage figures without reducing the voltage divider circuit to a Th\'evenin equivalent.

%(END_ANSWER)





%(BEGIN_NOTES)

Students are known to ask, "When are we ever going to use Th\'evenin's Theorem?" as this concept is introduced in their electronics coursework.  This is a valid question, and should be answered with immediate, practical examples.  This question does exactly that: demonstrate how to predict voltage "sag" for a loaded voltage divider in such a way that is much easier than using Ohm's Law and Kirchhoff's Laws directly.

Note the usage of European schematic symbols in this question.  Nothing significant about this choice -- just an opportunity for students to see other ways of drawing schematics.

Note also how this question makes use of ground symbols, but in a way where the concept is introduced gently: the first (example) schematics do not use ground symbols, whereas the practical (automotive) circuit does.

%INDEX% Thevenin's Theorem, applied to loaded voltage divider circuit

%(END_NOTES)


