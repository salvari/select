
%(BEGIN_QUESTION)
% Copyright 2003, Tony R. Kuphaldt, released under the Creative Commons Attribution License (v 1.0)
% This means you may do almost anything with this work of mine, so long as you give me proper credit

Magnetic fields, like all fields, have two fundamental measures: field {\it force} and field {\it flux}.  In an inductor, which of these field quantities is directly related to current through the wire coil, and which is directly related to the amount of energy stored?

Based on this relationship, which magnetic field quantity changes when a bar of iron is brought closer to a wire coil, connected to a source of constant current?

$$\epsfbox{01137x01.eps}$$

\underbar{file 01137}
%(END_QUESTION)





%(BEGIN_ANSWER)

Field force is a direct function of coil current, and field flux is a direct function of stored energy. 

\vskip 10pt

If an iron bar is brought closer to a wire coil connected to a constant current source, the magnetic field force generated by the coil will remain unchanged, while the magnetic field flux will increase (and along with it, the amount of energy stored in the magnetic field).

%(END_ANSWER)





%(BEGIN_NOTES)

The concept of a {\it field} is quite abstract, but at least magnetic fields are something within most peoples' realm of experience.  This question is good for helping students distinguish between field force and field flux, in terms they should understand (constant current through a coil, versus the attractive force produced by a magnetic field flux).

%INDEX% Magnetic field
%INDEX% Field, magnetic
%INDEX% Field, force versus flux

%(END_NOTES)


