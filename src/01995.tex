
%(BEGIN_QUESTION)
% Copyright 2003, Tony R. Kuphaldt, released under the Creative Commons Attribution License (v 1.0)
% This means you may do almost anything with this work of mine, so long as you give me proper credit

$$\epsfbox{01995x01.eps}$$

\underbar{file 01995}
\vfil \eject
%(END_QUESTION)





%(BEGIN_ANSWER)

Use circuit simulation software to verify your predicted and measured parameter values.

%(END_ANSWER)





%(BEGIN_NOTES)

Use a variable-voltage, regulated power supply to supply a DC voltage safely below the maximum rating of the electret microphone (typically 10 volts).  Specify standard resistor values, all between 1 k$\Omega$ and 100 k$\Omega$ (1k5, 2k2, 2k7, 3k3, 4k7, 5k1, 6k8, 10k, 22k, 33k, 39k 47k, 68k, etc.).  Use a sine-wave function generator to supply an audio-frequency input signal, and make sure its amplitude isn't set so high that the amplifier clips.

I have had good success using the following values:

\medskip
\item{$\bullet$} $V_{CC}$ = 6 volts
\item{$\bullet$} $R_1$ = 68 k$\Omega$
\item{$\bullet$} $R_2$ = 33 k$\Omega$
\item{$\bullet$} $R_3$ = 4.7 k$\Omega$
\item{$\bullet$} $R_4$ = 1.5 k$\Omega$
\item{$\bullet$} $R_5$ = $R_6$ = 10 k$\Omega$
\item{$\bullet$} $R_7$ = $R_8$ = 10 $\Omega$
\item{$\bullet$} $C_1$ = $C_2$ = 0.47 $\mu$F
\item{$\bullet$} $C_3$ = $C_4$ = 47 $\mu$F
\item{$\bullet$} $C_5$ = 1000 $\mu$F
\item{$\bullet$} $C_6$ = 100 $\mu$F
\item{$\bullet$} $D_1$ = $D_2$ = part number 1N4001
\item{$\bullet$} $Q_1$ = part number 2N2222
\item{$\bullet$} $Q_2$ = part number 2N2222
\item{$\bullet$} $Q_3$ = part number 2N2907
\medskip

An extension of this exercise is to incorporate troubleshooting questions.  Whether using this exercise as a performance assessment or simply as a concept-building lab, you might want to follow up your students' results by asking them to predict the consequences of certain circuit faults.

%INDEX% Assessment, performance-based (Audio intercom circuit, push-pull output)

%(END_NOTES)


