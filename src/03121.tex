
%(BEGIN_QUESTION)
% Copyright 2005, Tony R. Kuphaldt, released under the Creative Commons Attribution License (v 1.0)
% This means you may do almost anything with this work of mine, so long as you give me proper credit

An important parameter when specifying traces on printed circuit boards is the {\it ounce} rating of the copper used.  "1 ounce" copper is very common for general-purpose work.  Explain how this "ounce" rating is defined, and why you might want to use "2 ounce" or "4 ounce" copper instead of "1 ounce" for certain traces on your board designs.

\underbar{file 03121}
%(END_QUESTION)





%(BEGIN_ANSWER)

The "ounce" rating of PCB copper is defined as weight of that thickness of copper trace, for one square foot of trace area.  Heavier ounce ratings are used for high-current and/or high-reliability traces.

%(END_ANSWER)





%(BEGIN_NOTES)

Ask your students to compare and contrast the "ounce" rating of PCB traces to wire "gauge."

%INDEX% Ounce rating for PCB traces

%(END_NOTES)


