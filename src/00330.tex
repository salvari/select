
%(BEGIN_QUESTION)
% Copyright 2003, Tony R. Kuphaldt, released under the Creative Commons Attribution License (v 1.0)
% This means you may do almost anything with this work of mine, so long as you give me proper credit

A large industrial electric motor is supplied power through a pair of fuses:

$$\epsfbox{00330x01.eps}$$

One day the motor suddenly stops running, even though the switch is still in the "on" position.  An electrician is summoned to troubleshoot the failed motor, and this person decides to perform some voltage measurements to determine whether or not one of the fuses has "blown" open before doing anything else.  The measurements taken by the electrician are as such (with the switch in the "on" position):

$$\epsfbox{00330x02.eps}$$

\medskip
\item{$\bullet$} Between A and ground = 120 volts AC
\item{$\bullet$} Between B and ground = 120 volts AC
\item{$\bullet$} Between C and ground = 120 volts AC
\item{$\bullet$} Between D and ground = 120 volts AC
\medskip

Based on these measurements, the electrician decides that both fuses are still in good condition, and that the problem lies elsewhere in the circuit.  Do you agree with this assessment?  Why or why not?

\underbar{file 00330}
%(END_QUESTION)





%(BEGIN_ANSWER)

So long as the switch is still in the "on" position when these measurements were taken, one of the fuses could still be blown!

\vskip 10pt

Follow-up question: what voltage measurement(s) would conclusively test the condition of both fuses?

%(END_ANSWER)





%(BEGIN_NOTES)

I have actually seen an experienced electrician make this mistake on the job!  Ask your students to explain how full voltage could be measured at points C and D, with respect to ground, even with one of the fuses blown.

%INDEX% Troubleshooting, simple circuit
%INDEX% Voltmeter usage

%(END_NOTES)


