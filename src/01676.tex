
%(BEGIN_QUESTION)
% Copyright 2003, Tony R. Kuphaldt, released under the Creative Commons Attribution License (v 1.0)
% This means you may do almost anything with this work of mine, so long as you give me proper credit

$$\epsfbox{01676x01.eps}$$

\underbar{file 01676}
\vfil \eject
%(END_QUESTION)





%(BEGIN_ANSWER)

Use circuit simulation software to verify your predicted and measured parameter values.

%(END_ANSWER)





%(BEGIN_NOTES)

Use a sine-wave function generator for the AC voltage source.  I recommend against using line-power AC because of strong harmonic frequencies which may be present (due to nonlinear loads operating on the same power circuit).  Specify standard resistor and capacitor values.

I recommend using components that produce a phase shift of approximately 45 degrees within the low audio frequency range (less than 1 kHz).  This allows most multimeters to be used for voltage measurement in conjunction with the oscilloscope.

Something to suggest that students do is use their oscilloscopes' {\it ground} position on the coupling switch, to help center the dot on the screen before they set up their Lissajous figures for phase shift measurement.

One way for students to do this assessment is to have them predict what the Lissajous figure will look like, based on circuit component values.  They sketch the predicted Lissajous figure on the grid provided (working the math "backward" to arrive at $n$ and $m$ values before actually hooking up an oscilloscope), then the instructor assesses them based on the conformity of the real oscilloscope display to their prediction.

%INDEX% Assessment, performance-based (Measuring phase shift by Lissajous figures)

%(END_NOTES)


