
%(BEGIN_QUESTION)
% Copyright 2004, Tony R. Kuphaldt, released under the Creative Commons Attribution License (v 1.0)
% This means you may do almost anything with this work of mine, so long as you give me proper credit

The "power triangle" is a very useful model for understanding the mathematical relationship between apparent power ($S$), true power ($P$), and reactive power ($Q$):

$$\epsfbox{02184x01.eps}$$

Explain what happens to the triangle if power factor correction components are added to a circuit.  What side(s) change length on the triangle, and what happens to the angle $\Theta$?

\underbar{file 02184}
%(END_QUESTION)





%(BEGIN_ANSWER)

As power factor is brought closer to unity (1), the power triangle "flattens," with $P$ remaining constant:

$$\epsfbox{02184x02.eps}$$

%(END_ANSWER)





%(BEGIN_NOTES)

Ask your students to explain what the "triangle" looks like at a power factor of unity.

%INDEX% Power factor correction, qualitative effect on power "triangle"

%(END_NOTES)


