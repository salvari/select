
%(BEGIN_QUESTION)
% Copyright 2003, Tony R. Kuphaldt, released under the Creative Commons Attribution License (v 1.0)
% This means you may do almost anything with this work of mine, so long as you give me proper credit

The {\it ignition coil} of a gasoline-powered internal combustion automobile engine is an example of a transformer, although it is not powered by alternating current.  Explain how a transformer may be operated on electricity that is {\it not} AC:

$$\epsfbox{00251x01.eps}$$

\underbar{file 00251}
%(END_QUESTION)





%(BEGIN_ANSWER)

In order for a transformer to function, the primary winding current must change rapidly with regard to time.  Whether this is a current that truly alternates, or just one that {\it pulses} in the same direction, is irrelevant.

\vskip 10pt

Challenge question: is the wave-shape of the secondary voltage sinusoidal?  Why or why not?

%(END_ANSWER)





%(BEGIN_NOTES)

This is a very common application of transformer technology: the ignition "coil" used to ignite the air-fuel mixture inside a gasoline engine's combustion chamber.  This question also addresses an issue sometimes misunderstood by students, that transformers are fundamentally AC devices, not DC.

It might be a good idea to have an automotive ignition coil available for for classroom demonstration.  In lieu of a spark plug, a neon lamp may be used to indicate the presence of high voltage.

As for answering the challenge question, an oscilloscope will quickly prove the nature of the waveshape, for any transformer energized with pulsating DC.

%INDEX% Automotive ignition system
%INDEX% Ignition coil, as a transformer

%(END_NOTES)


