
%(BEGIN_QUESTION)
% Copyright 2003, Tony R. Kuphaldt, released under the Creative Commons Attribution License (v 1.0)
% This means you may do almost anything with this work of mine, so long as you give me proper credit

Suppose a friend wanted to install filter networks in the "woofer" section of their stereo system, to prevent high-frequency power from being wasted in speakers incapable of reproducing those frequencies.  To this end, your friend installs the following resistor-capacitor networks:

$$\epsfbox{00614x01.eps}$$

After examining this schematic, you see that your friend has the right idea in mind, but implemented it incorrectly.  These filter circuits would indeed block high-frequency signals from getting to the woofers, but they would not actually accomplish the stated goal of minimizing wasted power.

What would you recommend to your friend in lieu of this circuit design?

\underbar{file 00614}
%(END_QUESTION)





%(BEGIN_ANSWER)

Rather than use a "shunting" form of low-pass filter (resistor and capacitor), a "blocking" form of low-pass filter (inductor) should be used instead.

%(END_ANSWER)





%(BEGIN_NOTES)

The reason for this choice in filter designs is very practical.  Ask your students to describe how a "shunting" form of filter works, where the reactive component is connected in parallel with the load, receiving power through a series resistor.  Contrast this against a "blocking" form of filter circuit, in which a reactive component is connected in series with the load.  In one form of filter, a resistor is necessary.  In the other form of filter, a resistor is not necessary.  What difference does this make in terms of power dissipation within the filter circuit?

%INDEX% Filter, lowpass (audio speaker application)

%(END_NOTES)


