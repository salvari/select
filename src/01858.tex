
%(BEGIN_QUESTION)
% Copyright 2003, Tony R. Kuphaldt, released under the Creative Commons Attribution License (v 1.0)
% This means you may do almost anything with this work of mine, so long as you give me proper credit

A common way of representing complex electronic systems is the {\it block diagram}, where specific functional sections of a system are outlined as squares or rectangles, each with a certain purpose and each having input(s) and output(s).  For an example, here is a block diagram of an analog ("Cathode Ray") oscilloscope, or {\it CRO}:

$$\epsfbox{01858x01.eps}$$

Block diagrams may also be helpful in representing and understanding filter circuits.  Consider these symbols, for instance:

$$\epsfbox{01858x02.eps}$$

Which of these represents a {\it low-pass} filter, and which represents a {\it high-pass filter}?  Explain your reasoning.

Also, identify the new filter functions created by the compounding of low- and high-pass filter "blocks":

$$\epsfbox{01858x03.eps}$$

\underbar{file 01858}
%(END_QUESTION)





%(BEGIN_ANSWER)

$$\epsfbox{01858x04.eps}$$

%(END_ANSWER)





%(BEGIN_NOTES)

Aside from getting students to understand that band-function filters may be built from sets of low- and high-pass filter blocks, this question is really intended to initiate problem-solving activity.  Discuss with your students how they might approach a problem like this to see how the circuits respond.  What "thought experiments" did they try in their minds to investigate these circuits?

%INDEX% Block diagram, passive filter circuits
%INDEX% Band-stop filter, built from low-pass and high-pass elements
%INDEX% Band-pass filter, built from low-pass and high-pass elements

%(END_NOTES)


