
%(BEGIN_QUESTION)
% Copyright 2003, Tony R. Kuphaldt, released under the Creative Commons Attribution License (v 1.0)
% This means you may do almost anything with this work of mine, so long as you give me proper credit

Multiplexers, or data selectors, may be used to generate arbitrary truth table functions.  Take for example this truth table, shown beside a symbol for a 16-channel multiplexer:

$$\epsfbox{01476x01.eps}$$

Show the wire connections necessary to make the multiplexer output the specified logic states in response to the data select (A, B, C, and D) inputs.

\underbar{file 01476}
%(END_QUESTION)





%(BEGIN_ANSWER)

$$\epsfbox{01476x02.eps}$$

\vskip 10pt

Follow-up question: what if this multiplexer had an active-low output, like the 74150?  How would this change your design for implementing the truth table?

%(END_ANSWER)





%(BEGIN_NOTES)

Discuss with your students the significance of using a multiplexer in this fashion: to implement arbitrary logic functions.  For those students who may not be familiar with the term, be sure to define the word "arbitrary."  It may seem silly, but students often fail to ask for the definitions of words that are new to them, for fear of sounding dumb in front of their peers and in front of you.  One more reason to model respect in your classroom, and also to create an atmosphere where students feel comfortable asking any question.

%INDEX% Multiplexer, used as arbitrary logic function generator

%(END_NOTES)


