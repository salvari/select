
%(BEGIN_QUESTION)
% Copyright 2005, Tony R. Kuphaldt, released under the Creative Commons Attribution License (v 1.0)
% This means you may do almost anything with this work of mine, so long as you give me proper credit

The voltage gain of a common-emitter transistor amplifier is approximately equal to the collector resistance divided by the emitter resistance:

$$\epsfbox{03106x01.eps}$$

Knowing this, calculate the necessary resistance values for the following fixed-value resistor ($R_2$) and potentiometer ($R_1$) to give this common-emitter amplifier an adjustable voltage gain range of 2 to 8:

$$\epsfbox{03106x02.eps}$$

\underbar{file 03106}
%(END_QUESTION)





%(BEGIN_ANSWER)

$R_1$ (pot) = 9 k$\Omega$

\vskip 10pt

$R_2$ (fixed) = 3 k$\Omega$

%(END_ANSWER)





%(BEGIN_NOTES)

Ask your students how they might make a standard-value potentiometer such as 10 k$\Omega$ have a full-scale (maximum) resistance of only 9 k$\Omega$.

%INDEX% Voltage gain, common emitter amplifier

%(END_NOTES)


