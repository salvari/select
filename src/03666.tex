
%(BEGIN_QUESTION)
% Copyright 2005, Tony R. Kuphaldt, released under the Creative Commons Attribution License (v 1.0)
% This means you may do almost anything with this work of mine, so long as you give me proper credit

An {\it electric arc welder} is a low-voltage, high-current power source used to generate hot arcs capable of melting metal.  Note the voltage and current measurements taken for this particular welder:

$$\epsfbox{03666x01.eps}$$

Determine two Th\'evenin equivalent circuits for the arc welder.  The first circuit will simply be an AC voltage source and an internal impedance.  The second circuit will be a voltage source and internal impedance connected through an ideal transformer with a step-down ratio of 8 to 1:

$$\epsfbox{03666x02.eps}$$

\underbar{file 03666}
%(END_QUESTION)





%(BEGIN_ANSWER)

\noindent
{\bf Simple equivalent circuit}

$V_{Th}$ = 48.5 volts

$Z_{Th}$ = 0.230 $\Omega$

\vskip 10pt

\noindent
{\bf Equivalent circuit with transformer}

$V_{Th}$ = 388 volts

$Z_{Th}$ = 14.72 $\Omega$

%(END_ANSWER)





%(BEGIN_NOTES)

This practical scenario shows how Th\'evenin's theorem may be used to "model" a complex device as two simple components (voltage source and resistor).  Of course, we must make certain assumptions when modeling in this fashion: we assume, for instance, that the arc welder is a linear device, which may or may not be true.

%INDEX% Thevenin's theorem, experimentally determining Thevenin voltage and Thevenin impedance of network

%(END_NOTES)


