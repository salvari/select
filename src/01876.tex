
%(BEGIN_QUESTION)
% Copyright 2003, Tony R. Kuphaldt, released under the Creative Commons Attribution License (v 1.0)
% This means you may do almost anything with this work of mine, so long as you give me proper credit

If we energize an inductor's coil with an oscillating (AC) voltage, we will generate an oscillating magnetic flux in the inductor core:

$$\epsfbox{01876x01.eps}$$

If we wrap a second coil of wire around the same magnetic core as the first (inductor) coil, we set up a situation where mutual inductance exists: a change of current through one coil induces a voltage in the other, and visa-versa.  This, obviously, will result in an AC voltage being induced in the second wire coil:

$$\epsfbox{01876x02.eps}$$

What name is given to such a device, with two coils of wire sharing a common magnetic flux?  Also, plot both the magnetic flux waveform and the secondary (induced) voltage waveform on the same graph as the primary (applied) voltage waveform:

$$\epsfbox{01876x03.eps}$$

\underbar{file 01876}
%(END_QUESTION)





%(BEGIN_ANSWER)

This device is called a {\it transformer}.

$$\epsfbox{01876x04.eps}$$

Note: the relative amplitudes of $v_p$ and $v_s$ are arbitrary.  I drew them at different amplitudes for the benefit of the reader: so the two waveforms would not perfectly overlap and become indistinguishable from one another.

%(END_ANSWER)





%(BEGIN_NOTES)

Ask your students how the secondary coil would have to be made in order to truly generate a voltage greater than the applied (primary) coil voltage.  How about generating a secondary voltage less than the primary?

%INDEX% Faraday's Law
%INDEX% Magnetic flux, relationship to counter-EMF
%INDEX% Transformer, defined

%(END_NOTES)


