
%(BEGIN_QUESTION)
% Copyright 2005, Tony R. Kuphaldt, released under the Creative Commons Attribution License (v 1.0)
% This means you may do almost anything with this work of mine, so long as you give me proper credit

$$\epsfbox{03655x01.eps}$$

\underbar{file 03655}
\vfil \eject
%(END_QUESTION)





%(BEGIN_ANSWER)

Use circuit simulation software to verify your predicted and measured parameter values.

%(END_ANSWER)





%(BEGIN_NOTES)

You may use an audio frequency impedance matching (1000:8 ohms) transformer or a step-down power transformer for this exercise.  In either case, students need to know or figure out the turns ratio before proceeding with the experiment.  If you are using an audio transformer, use a sine-wave signal generator for the voltage source, but avoid frequencies below 1kHz.  If you are using a power transformer, use a Variac to provide adjustable primary voltage.

An extension of this exercise is to incorporate troubleshooting questions.  Whether using this exercise as a performance assessment or simply as a concept-building lab, you might want to follow up your students' results by asking them to predict the consequences of certain circuit faults.

%INDEX% Assessment, performance-based (Transformer voltage ratio)

%(END_NOTES)


