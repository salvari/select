
%(BEGIN_QUESTION)
% Copyright 2005, Tony R. Kuphaldt, released under the Creative Commons Attribution License (v 1.0)
% This means you may do almost anything with this work of mine, so long as you give me proper credit

Calculate the rate of change of current ($di \over dt$) for the inductor at the exact instant in time where the switch moves to the "charge" position.

$$\epsfbox{03559x01.eps}$$

\underbar{file 03559}
%(END_QUESTION)





%(BEGIN_ANSWER)

$di \over dt$ = 280 A/s or 0.28 A/ms

\vskip 10pt

Follow-up question: does the resistor value have any effect on this initial $di \over dt$?  Explain why or why not.

%(END_ANSWER)





%(BEGIN_NOTES)

Some students may think that a rate of change of 280 amps per second might burn up the wiring, because 280 amps seems like a lot of current.  Remind them that this is a {\it rate of change} and not an actual current figure.  This number simply tells us how fast the current is changing, not how far it will rise to.  It is the difference between saying that a car travels at 75 miles per hour and that a car will travel 75 miles.

%INDEX% LR time constant circuit
%INDEX% Rate-of-change calculation, LR circuit

%(END_NOTES)


