
%(BEGIN_QUESTION)
% Copyright 2003, Tony R. Kuphaldt, released under the Creative Commons Attribution License (v 1.0)
% This means you may do almost anything with this work of mine, so long as you give me proper credit

Calculate the operating frequency of this oscillator circuit:

$$\epsfbox{01214x01.eps}$$

Explain why the operating frequency will not be the same if the transistor receives its feedback signal from the other side of the bridge, like this:

$$\epsfbox{01214x02.eps}$$

\underbar{file 01214}
%(END_QUESTION)





%(BEGIN_ANSWER)

$f = 159.155 \hbox{ Hz}$

\vskip 10pt

If the feedback signal comes from the other side of the bridge, the feedback signal's phase shift will be determined by a different set of components (primarily, the coupling capacitors and bias network resistances) rather than the reactive arms of the bridge.

%(END_ANSWER)





%(BEGIN_NOTES)

Given the phase shift requirements of a two-stage oscillator circuit such as this, some students may wonder why the circuit won't act the same in the second configuration.  If such confusion exists, clarify the concept with a question: "What is the phase relationship between input and output voltages for the bridge in these two configurations, over a wide range of frequencies?"  From this observation, your students should be able to tell that only one of these configurations will be stable at 159.155 Hz.

%INDEX% Wien bridge oscillator circuit
%INDEX% Oscillator circuit, Wien bridge

%(END_NOTES)


