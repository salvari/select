
%(BEGIN_QUESTION)
% Copyright 2003, Tony R. Kuphaldt, released under the Creative Commons Attribution License (v 1.0)
% This means you may do almost anything with this work of mine, so long as you give me proper credit

Express the impedance (${\bf Z}$) in both polar and rectangular forms for each of the following components:

\medskip
\goodbreak
\item{$\bullet$} A resistor with 500 $\Omega$ of resistance
\item{$\bullet$} An inductor with 1.2 k$\Omega$ of reactance
\item{$\bullet$} A capacitor with 950 $\Omega$ of reactance
\item{$\bullet$} A resistor with 22 k$\Omega$ of resistance
\item{$\bullet$} A capacitor with 50 k$\Omega$ of reactance
\item{$\bullet$} An inductor with 133 $\Omega$ of reactance
\medskip

\underbar{file 00591}
%(END_QUESTION)





%(BEGIN_ANSWER)

\medskip
\item{$\bullet$} A resistor with 500 $\Omega$ of resistance: 500 $\Omega$ $\angle$ 0$^{o}$ or 500 + j0 $\Omega$
\item{$\bullet$} An inductor with 1.2 k$\Omega$ of reactance: 1.2 k$\Omega$ $\angle$ 90$^{o}$ or 0 + j1.2k $\Omega$
\item{$\bullet$} A capacitor with 950 $\Omega$ of reactance: 950 $\Omega$ $\angle$ -90$^{o}$ or 0 - j950 $\Omega$
\item{$\bullet$} A resistor with 22 k$\Omega$ of resistance: 22 k$\Omega$ $\angle$ 0$^{o}$ or 22k + j0 $\Omega$
\item{$\bullet$} A capacitor with 50 k$\Omega$ of reactance: 50 k$\Omega$ $\angle$ -90$^{o}$ or 0 - j50k $\Omega$
\item{$\bullet$} An inductor with 133 $\Omega$ of reactance: 133 $\Omega$ $\angle$ 90$^{o}$ or 0 + j133 $\Omega$
\medskip

\vskip 10pt

Follow-up question: what would the phasors look like for resistive, inductive, and capacitive impedances?

%(END_ANSWER)





%(BEGIN_NOTES)

In your discussion with students, emphasize the consistent nature of phase angles for impedances of "pure" components.

%INDEX% Phase angles, of resistive, inductive, and capacitive impedances

%(END_NOTES)


