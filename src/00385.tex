
%(BEGIN_QUESTION)
% Copyright 2003, Tony R. Kuphaldt, released under the Creative Commons Attribution License (v 1.0)
% This means you may do almost anything with this work of mine, so long as you give me proper credit

When a DC motor is running, sparks may generally be seen where the carbon brushes contact the "commutator" segments.  Explain why this sparking occurs, and also define the word "commutation" in its electrical usage.

What does this phenomenon indicate about the longevity of DC motors, and their suitability in certain environments?

\underbar{file 00385}
%(END_QUESTION)





%(BEGIN_ANSWER)

To "commutate" means "to reverse direction," in the electrical sense of the word.  The result of the commutator bars and brushes alternately making and breaking the electrical circuit with the armature windings invariably causes some degree of sparking to occur.

\vskip 10pt

Follow-up question: identify an environment where a sparking motor would be unsafe.

%(END_ANSWER)





%(BEGIN_NOTES)

If your students find themselves working in some sort of electrical maintenance jobs, what types of routine maintenance do they think they might have to do on DC electric motors, given the presence of sparking at the commutator?  Ask them what safety issues this sparking could present in certain environments.  Ask them if they think there are any environments that would be especially detrimental to a motor design such as this.

%INDEX% Commutator, DC electric motor
%INDEX% Arcing, DC electric motor commutator

%(END_NOTES)


