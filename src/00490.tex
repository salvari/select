
%(BEGIN_QUESTION)
% Copyright 2003, Tony R. Kuphaldt, released under the Creative Commons Attribution License (v 1.0)
% This means you may do almost anything with this work of mine, so long as you give me proper credit

Shown here is a schematic diagram of a transformer powering a resistive load, at the exact moment in time where the primary winding's voltage is at its positive (+) peak:

$$\epsfbox{00490x01.eps}$$

Identify the polarity of voltage across the load resistor at this exact moment in time, as well as the direction of current in each of the windings.

\underbar{file 00490}
%(END_QUESTION)





%(BEGIN_ANSWER)

$$\epsfbox{00490x02.eps}$$

\vskip 10pt

Follow-up question: note the relationship between direction of current and polarity of voltage for each of the transformer windings.  What do these different relationships suggest, in regard to the "flow" of power in the circuit?

%(END_ANSWER)





%(BEGIN_NOTES)

One perspective that may help students understand the directions of current through each winding of the transformer, in relation to the voltage polarities, is to think of each winding as either being a {\it source} of electrical power or a {\it load}.  Ask your students, "which winding acts as a {\it source} in this circuit, and which one acts as a {\it load}?  Imagine these sources and loads are DC (so we may maintain the same polarity of voltage, for the sake of analysis).  Which way would you draw the currents for a DC source and for a DC load?

%INDEX% Dot convention for transformer winding "polarity"

%(END_NOTES)


