
%(BEGIN_QUESTION)
% Copyright 2003, Tony R. Kuphaldt, released under the Creative Commons Attribution License (v 1.0)
% This means you may do almost anything with this work of mine, so long as you give me proper credit

$$\epsfbox{01649x01.eps}$$

\underbar{file 01649}
\vfil \eject
%(END_QUESTION)





%(BEGIN_ANSWER)

Use circuit simulation software to verify your predicted and measured parameter values.

%(END_ANSWER)





%(BEGIN_NOTES)

Students may use potentiometers in their range resistance networks to achieve precise values.  However, they are not allowed to adjust those potentiometers after connecting them to the meter movement -- they must set their potentiometer(s) during the "prediction" step of the assessment before the circuit is completely built.

An extension of this exercise is to incorporate troubleshooting questions.  Whether using this exercise as a performance assessment or simply as a concept-building lab, you might want to follow up your students' results by asking them to predict the consequences of certain circuit faults.

%INDEX% Assessment, performance-based (DC voltmeter design)

%(END_NOTES)


