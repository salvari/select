
%(BEGIN_QUESTION)
% Copyright 2003, Tony R. Kuphaldt, released under the Creative Commons Attribution License (v 1.0)
% This means you may do almost anything with this work of mine, so long as you give me proper credit

A large audio speaker may serve to demonstrate both the principles of {\it electromagnetism} and of {\it electromagnetic induction}.  Explain how this may be done.

\underbar{file 00079}
%(END_QUESTION)





%(BEGIN_ANSWER)

I won't tell you how to set up or do the experiment, but I will show you an illustration of a typical audio speaker:

$$\epsfbox{00079x01.eps}$$

The "voice coil" is attached to the flexible speaker cone, and is free to move along the long axis of the magnet.  The magnet is stationary, being solidly anchored to the metal frame of the speaker, and is centered in the middle of the voice coil.

This experiment is most impressive when a physically large (i.e. "woofer") speaker is used.
 
\vskip 10pt

Follow-up question: identify some possible points of failure in a speaker which would prevent it from operating properly.

%(END_ANSWER)





%(BEGIN_NOTES)

Since not everyone has ready access to a large speaker for this kind of experiment, it may help to have one or two "woofer" speakers located in the classroom for students to experiment with during this phase of the discussion.  Any time you can encourage students to set up impromptu experiments in class for the purpose of exploring fundamental principles, it is a Good Thing.

%INDEX% Electromagnetism
%INDEX% Electromagnetic induction
%INDEX% Induction, electromagnetic

%(END_NOTES)


