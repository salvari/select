
%(BEGIN_QUESTION)
% Copyright 2003, Tony R. Kuphaldt, released under the Creative Commons Attribution License (v 1.0)
% This means you may do almost anything with this work of mine, so long as you give me proper credit

Determine the voltage across the capacitor three seconds after the switch is moved from the upper position to the lower position, assuming it had been left in the upper position for a long time:

$$\epsfbox{00457x01.eps}$$

\underbar{file 00457}
%(END_QUESTION)





%(BEGIN_ANSWER)

$E_C =$ 3.974 V @ 3 seconds

\vskip 10pt

Follow-up question: identify at least one failure in this circuit which would cause the capacitor to remain completely discharged no matter what position the switch was in.

%(END_ANSWER)





%(BEGIN_NOTES)

This problem is unique in that the capacitor does not discharge all the way to 0 volts when the switch is moved to the lower position.  Instead, it discharges down to a (final) value of 3 volts.  Solving for the answer requires that students be a bit creative with the common time-constant equations ($e^{-{t \over \tau}}$ and $1 - e^{-{t \over \tau}}$).

The follow-up question is simply an exercise in troubleshooting theory.

%INDEX% RC time constant circuit
%INDEX% Time constant calculation, RC circuit

%(END_NOTES)


