
\centerline{\bf ELTR 140 (Digital 1), section 1} \bigskip 
 
\vskip 10pt

\noindent
{\bf Recommended schedule}

\vskip 5pt

%%%%%%%%%%%%%%%
\hrule \vskip 5pt
\noindent
\underbar{Day 1}

\hskip 10pt Topics: {\it Logic states and simple gate circuits}
 
\hskip 10pt Questions: {\it 1 through 10}
 
\hskip 10pt Lab Exercise: {\it OR gate, diode-resistor logic (question 51)}
 
%INSTRUCTOR \hskip 10pt {\bf Explain suggested project ideas to students}

%INSTRUCTOR \hskip 10pt {\bf Give project grading rubric to students, complete with deadlines}

%INSTRUCTOR \hskip 10pt {\bf MIT 6.002 video clip: Disk 1, Lecture 4; AND gate noise immunity 48:15 to end}

\vskip 10pt
%%%%%%%%%%%%%%%
\hrule \vskip 5pt
\noindent
\underbar{Day 2}

\hskip 10pt Topics: {\it TTL logic gates and truth tables}
 
\hskip 10pt Questions: {\it 11 through 20}
 
\hskip 10pt Lab Exercise: {\it AND gate, simple BJT logic (question 52)}
 
\vskip 10pt
%%%%%%%%%%%%%%%
\hrule \vskip 5pt
\noindent
\underbar{Day 3}

\hskip 10pt Topics: {\it CMOS logic gates and truth tables}
 
\hskip 10pt Questions: {\it 21 through 30}
 
\hskip 10pt Lab Exercise: {\it IC logic gate usage (question 53)}
 
\vskip 10pt
%%%%%%%%%%%%%%%
\hrule \vskip 5pt
\noindent
\underbar{Day 4}

\hskip 10pt Topics: {\it Relay circuits and truth tables}
 
\hskip 10pt Questions: {\it 31 through 40}
 
\hskip 10pt Lab Exercise: {\it AND gate, relay logic (question 54)}
 
%INSTRUCTOR \hskip 10pt {\bf Demo: show photos of old Mark I electromechanical relay computer}

\vskip 10pt
%%%%%%%%%%%%%%%
\hrule \vskip 5pt
\noindent
\underbar{Day 5}

\hskip 10pt Topics: {\it Logic circuit performance parameters}
 
\hskip 10pt Questions: {\it 41 through 50}
 
\hskip 10pt Lab Exercise: {\it Gate-relay interposing (question 55)}
 
%INSTRUCTOR \hskip 10pt {\bf MIT 6.002 video clip: Disk 4, Lecture 24; CMOS power consumption 48:00 to 50:05}

%INSTRUCTOR \hskip 10pt {\bf MIT 6.002 video clip: Disk 4, Lecture 25; CMOS temp. vs freq. 39:03 to 40:01}

\vskip 10pt
%%%%%%%%%%%%%%%
\hrule \vskip 5pt
\noindent
\underbar{Day 6}

\hskip 10pt Exam 1: {\it includes IC logic gate performance assessment}
 
\vskip 10pt
%%%%%%%%%%%%%%%
\hrule \vskip 5pt
\noindent
\underbar{Troubleshooting practice problems}

\hskip 10pt Questions: {\it 57 through 66}
 
\vskip 10pt
%%%%%%%%%%%%%%%
\hrule \vskip 5pt
\noindent
\underbar{DC/AC review problems}

\hskip 10pt Questions: {\it 67 through 86}
 
\vskip 10pt
%%%%%%%%%%%%%%%
\hrule \vskip 5pt
\noindent
\underbar{Basic principles of microcontrollers}

\hskip 10pt Questions: {\it 87 through 96}
 
\vskip 10pt
%%%%%%%%%%%%%%%
\hrule \vskip 5pt
\noindent
\underbar{General concept practice and challenge problems}

\hskip 10pt Questions: {\it 97 through the end of the worksheet}
 
\vskip 10pt
%%%%%%%%%%%%%%%
\hrule \vskip 5pt
\noindent
\underbar{Impending deadlines}

\hskip 10pt {\bf Project due at end of ELTR140, Section 3}
 
\hskip 10pt Question 56: Sample project grading criteria
 
\vskip 10pt
%%%%%%%%%%%%%%%









\vfil \eject

\centerline{\bf ELTR 140 (Digital 1), section 1} \bigskip 
 
\vskip 10pt

\noindent
{\bf Project ideas}

\vskip 5pt

\hrule \vskip 5pt

\vskip 10pt

\noindent
\underbar{Logic probe:} Uses a comparator and at least one logic gate to drive three LEDs: "High", "Low", and "Indeterminate", showing three different voltage levels detected at the probe.  The thresholds for high and low logic levels shall be adjustable for use in different types of logic circuits.

\vskip 10pt

\noindent
\underbar{Digital combination lock:} Drives a solenoid or other power device only when the input code matches the "key" code (set by a series of switches on the circuit board), and drives an alarm siren if the input code does {\it not} match the "key" code.  The logic function may be done with Exclusive-OR gates or with a magnitude comparator IC.  Remember, the more bits in the codes, the harder it is to guess!

\vskip 10pt


\vfil \eject

\centerline{\bf ELTR 140 (Digital 1), section 1} \bigskip 
 
\vskip 10pt

\noindent
{\bf Skill standards addressed by this course section}

\vskip 5pt

%%%%%%%%%%%%%%%
\hrule \vskip 10pt
\noindent
\underbar{EIA {\it Raising the Standard; Electronics Technician Skills for Today and Tomorrow}, June 1994}

\vskip 5pt

\medskip
\item{\bf F} {\bf Technical Skills -- Digital Circuits}
\item{\bf F.01} Demonstrate an understanding of the characteristics of integrated circuit (IC) logic families.
\item{\bf F.05} Understand principles and operations of types of logic gates.
\item{\bf F.06} Fabricate and demonstrate types of logic gates.
\item{\bf F.07} Troubleshoot and repair types of logic gates.
\medskip

\vskip 5pt

\medskip
\item{\bf B} {\bf Basic and Practical Skills -- Communicating on the Job}
\item{\bf B.01} Use effective written and other communication skills.  {\it Met by group discussion and completion of labwork.}
\item{\bf B.03} Employ appropriate skills for gathering and retaining information.  {\it Met by research and preparation prior to group discussion.}
\item{\bf B.04} Interpret written, graphic, and oral instructions.  {\it Met by completion of labwork.}
\item{\bf B.06} Use language appropriate to the situation.  {\it Met by group discussion and in explaining completed labwork.}
\item{\bf B.07} Participate in meetings in a positive and constructive manner.  {\it Met by group discussion.}
\item{\bf B.08} Use job-related terminology.  {\it Met by group discussion and in explaining completed labwork.}
\item{\bf B.10} Document work projects, procedures, tests, and equipment failures.  {\it Met by project construction and/or troubleshooting assessments.}
\item{\bf C} {\bf Basic and Practical Skills -- Solving Problems and Critical Thinking}
\item{\bf C.01} Identify the problem.  {\it Met by research and preparation prior to group discussion.}
\item{\bf C.03} Identify available solutions and their impact including evaluating credibility of information, and locating information.  {\it Met by research and preparation prior to group discussion.}
\item{\bf C.07} Organize personal workloads.  {\it Met by daily labwork, preparatory research, and project management.}
\item{\bf C.08} Participate in brainstorming sessions to generate new ideas and solve problems.  {\it Met by group discussion.}
\item{\bf D} {\bf Basic and Practical Skills -- Reading}
\item{\bf D.01} Read and apply various sources of technical information (e.g. manufacturer literature, codes, and regulations).  {\it Met by research and preparation prior to group discussion.}
\item{\bf E} {\bf Basic and Practical Skills -- Proficiency in Mathematics}
\item{\bf E.01} Determine if a solution is reasonable.
\item{\bf E.02} Demonstrate ability to use a simple electronic calculator.
\item{\bf E.06} Translate written and/or verbal statements into mathematical expressions.
\item{\bf E.07} Compare, compute, and solve problems involving binary, octal, decimal, and hexadecimal numbering systems.
\item{\bf E.12} Interpret and use tables, charts, maps, and/or graphs.
\item{\bf E.13} Identify patterns, note trends, and/or draw conclusions from tables, charts, maps, and/or graphs.
\item{\bf E.15} Simplify and solve algebraic expressions and formulas.
\item{\bf E.16} Select and use formulas appropriately.
\item{\bf E.18} Use properties of exponents and logarithms.
\medskip

\vskip 5pt

\medskip
\item{\bf F} {\bf Additional Skills -- Electromechanics}
\item{\bf B.01b} Relays and relay circuits.
\medskip

%%%%%%%%%%%%%%%




\vfil \eject

\centerline{\bf ELTR 140 (Digital 1), section 1} \bigskip 
 
\vskip 10pt

\noindent
{\bf Common areas of confusion for students}

\vskip 5pt

%%%%%%%%%%%%%%%
\hrule \vskip 5pt

\vskip 10pt

\noindent
{\bf Difficult concept: } {\it Necessary conditions for transistor operation.}

It is vitally important for students to understand the conditions necessary for transistor operation, both for understanding how circuits work and for troubleshooting faulty circuits.  Bipolar junction transistors require a base current (in the proper direction) to conduct, and the collector-to-emitter voltage must be of the correct polarity to push a collector current in the proper direction as well.  Both currents join at the emitter terminal, making the emitter current the sum of the base and collector currents.  Field-effect transistors are not so picky about the direction of the controlled current, and they only require the correct gate voltage (no gate current) to establish conduction.  What makes this so confusing is that there are two types of bipolar transistors (NPN and PNP), two types of junction field-effect transistors (N-channel and P-channel), and four types of MOSFETs (E-type N-channel, E-type P-channel, D-type N-channel, and D-type P-channel).

\vskip 10pt

\noindent
{\bf Difficult concept: } {\it Current sourcing versus current sinking.}

It is very common in electronics work to refer to current-controlling devices as either {\it sourcing} current to a load or {\it sinking} current from a load.  This is an overt reference to conventional-flow notation, referring to whether the conventional flow moves {\it out} of the transistor from the positive power supply terminal to the load (sourcing), or whether the conventional flow moves {\it in} to the transistor from the load and then "down" to ground (sinking).  Some students grasp this concept readily, while others seem to struggle mightily with it.  It is something rather essential to understand, because this terminology is extensively used by electronics professionals and found in electronics literature.  The key detail distinguishing the two conditions is which power supply rail (either +V or Gnd) is {\it directly} connected to the current-controlling device.

\vskip 10pt

\noindent
{\bf Difficult concept: } {\it Pullup and pulldown resistor placement.}

In digital circuits, resistors are often used to provide a secure logic state when an input device (such as a switch) goes to a high-impedance (open) mode.  Students often have difficulty figuring out exactly where these resistors should go in a circuit.  The most common mistake I've seen is to place one of these "pullup" or "pulldown" resistors in {\it series} with a gate input, which will accomplish absolutely nothing.  The "trick" to getting this placement right, if you can call it a trick at all, is to literally follow the word "pullup" or "pulldown."  A {\it pullup} resistor pulls the logic state of a wire up to the positive supply rail, and so must connect between the gate input and +V.  A {\it pulldown} resistor pulls the logic state of a wire down to ground potential, and so must connect between the gate input and ground.  In either case, the resistor provides a sure path to the opposite power rail that the input device connects to when active (closed).

