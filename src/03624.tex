
%(BEGIN_QUESTION)
% Copyright 2005, Tony R. Kuphaldt, released under the Creative Commons Attribution License (v 1.0)
% This means you may do almost anything with this work of mine, so long as you give me proper credit

Working on a job site with an experienced technician, you are tasked with trying to determine whether the line currents going to a three-phase electric motor are balanced.  If everything is okay with the motor and the power circuitry, of course, the three line currents should be precisely equal to each other.

The problem is, neither of you brought a clamp-on ammeter for measuring the line currents.  Your multimeters are much too small to measure the large currents in this circuit, and connecting an ammeter in series with such a large motor could be dangerous anyway.  So, the experienced technician decides to try something different -- he uses his multimeter as an AC milli-voltmeter to measure the small voltage drop across each fuse, using the fuses as crude shunt resistors:

$$\epsfbox{03624x01.eps}$$

He obtains the following measurements:

% No blank lines allowed between lines of an \halign structure!
% I use comments (%) instead, so that TeX doesn't choke.

$$\vbox{\offinterlineskip
\halign{\strut
\vrule \quad\hfil # \ \hfil & 
\vrule \quad\hfil # \ \hfil \vrule \cr
\noalign{\hrule}
%
% First row
Line & Fuse voltage drop \cr
%
\noalign{\hrule}
%
% Another row
1 & 24.3 mV \cr
%
\noalign{\hrule}
%
% Another row
2 & 37.9 mV \cr
%
\noalign{\hrule}
%
% Another row
3 & 15.4 mV \cr
%
\noalign{\hrule}
} % End of \halign 
}$$ % End of \vbox

Do these voltage drop measurements suggest imbalanced motor line currents?  Why or why not?

\underbar{file 03624}
%(END_QUESTION)





%(BEGIN_ANSWER)

The results are inconclusive, because resistance for the whole fuse and holder assembly is not a reliably stable quantity.  Corrosion between one of the fuse ends and the fuse holder clip, for example, would increase resistance between the points where millivoltage is shown measured.

\vskip 10pt

Follow-up question: just because the results of these millivoltage measurements are inconclusive in this scenario does not necessarily mean the principle of using fuses as current-indicating shunt resistors is useless.  Describe one application where using a fuse as a current-indicating shunt would yield trustworthy information about the current.

\vskip 10pt

Challenge question: determine where you could measure millivoltage, that might be more reliable in terms of quantitatively indicating line current.

%(END_ANSWER)





%(BEGIN_NOTES)

While measuring millivoltage across a {\it fuse} may seem like a strange diagnostic technique, it is one I have gainfully applied for years.  The "catch" is you have to know what it is good for and what it is not.  It is {\it not} a precise, quantitative technique by any means!

%INDEX% Fuse, used as current-indicating shunt resistance

%(END_NOTES)


