
%(BEGIN_QUESTION)
% Copyright 2004, Tony R. Kuphaldt, released under the Creative Commons Attribution License (v 1.0)
% This means you may do almost anything with this work of mine, so long as you give me proper credit

Suppose a non-sinusoidal voltage source is represented by the following Fourier series:

$$v(t) = 23.2 + 30 \sin (377 t) + 15.5 \sin (1131 t + 90) + 2.7 \sin (1508 t - 40)$$

Electrically, we might represent this non-sinusoidal voltage source as a circle, like this:

$$\epsfbox{02259x01.eps}$$

Knowing the Fourier series of this voltage, however, allows us to represent the same voltage source as a set of series-connected voltage sources, each with its own (sinusoidal) frequency.  Draw the equivalent schematic in this manner, labeling each voltage source with its RMS voltage value, frequency (in Hz), and phase angle:

\vskip 60pt

Hint: $\omega = 2 \pi f$

\vskip 10pt

\underbar{file 02259}
%(END_QUESTION)





%(BEGIN_ANSWER)

$$\epsfbox{02259x02.eps}$$

%(END_ANSWER)





%(BEGIN_NOTES)

The purpose of this question is to have students relate the sinusoidal terms of a particular Fourier series to a schematic diagram, translating between angular velocity and frequency, peak values and RMS values.

%INDEX% Fourier series, electrically defined

%(END_NOTES)


