
%(BEGIN_QUESTION)
% Copyright 2003, Tony R. Kuphaldt, released under the Creative Commons Attribution License (v 1.0)
% This means you may do almost anything with this work of mine, so long as you give me proper credit

When the pushbutton switch is actuated in this circuit, the solenoid valve energizes:

$$\epsfbox{00981x01.eps}$$

The only problem with this simple circuit is that the switch contacts suffer from extensive arcing caused each time the solenoid is de-energized.  One way to combat this arcing, though, is to connect an ordinary rectifying diode in parallel with the solenoid like this:

$$\epsfbox{00981x02.eps}$$

Explain what causes the excessive arcing at the switch contacts, and how the presence of a diode in the circuit completely eliminates it.

\underbar{file 00981}
%(END_QUESTION)





%(BEGIN_ANSWER)

The arcing is caused by inductive "kickback," and the diode prevents it by providing a complete circuit for the inductor's current to discharge through when the switch opens.

%(END_ANSWER)





%(BEGIN_NOTES)

This question provides an excellent opportunity to review inductor theory, particularly the direction of current and the polarity of voltage for an inductor when charging versus when discharging.  Analysis of this circuit will be made easier by drawing a schematic diagram.

%(END_NOTES)


