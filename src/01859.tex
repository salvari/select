
%(BEGIN_QUESTION)
% Copyright 2003, Tony R. Kuphaldt, released under the Creative Commons Attribution License (v 1.0)
% This means you may do almost anything with this work of mine, so long as you give me proper credit

Explain what a {\it band-pass} filter is, and how it differs from either a low-pass or a high-pass filter circuit.  Also, explain what a {\it band-stop} filter is, and draw Bode plots representative of both band-pass and band-stop filter types.

\underbar{file 01859}
%(END_QUESTION)





%(BEGIN_ANSWER)

A band-pass filter passes only those frequencies falling within a specified range, or "band."  A band-stop filter, sometimes referred to as a {\it notch filter}, does just the opposite: it attenuates frequencies falling within a specified band.

\vskip 10pt

Challenge question: what type of filter, band-pass or band-stop, do you suppose is used in a radio receiver (tuner)?  Explain your reasoning.

%(END_ANSWER)





%(BEGIN_NOTES)

In this question, I've opted to let students draw Bode plots, only giving them written descriptions of each filter type.

%INDEX% Band-stop filter, defined
%INDEX% Band-pass filter, define

%(END_NOTES)


