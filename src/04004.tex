
%(BEGIN_QUESTION)
% Copyright 2006, Tony R. Kuphaldt, released under the Creative Commons Attribution License (v 1.0)
% This means you may do almost anything with this work of mine, so long as you give me proper credit

A two-conductor cable of uniform construction will exhibit a uniform {\it characteristic impedance} ($Z_0$) due to its intrinsic, distributed inductance and capacitance:

$$\epsfbox{04004x01.eps}$$

What would happen to the value of this characteristic impedance if we were to make the cable wider, so that the conductors were further apart, all other dimensions remaining the same?

$$\epsfbox{04004x02.eps}$$

\underbar{file 04004}
%(END_QUESTION)





%(BEGIN_ANSWER)

$Z_0$ would {\it increase}.  I will leave it to you to explain why this happens.

%(END_ANSWER)





%(BEGIN_NOTES)

Be sure to ask your students to explain why the characteristic impedance will change in the direction it does, based on the known changes to both capacitance and inductance throughout the cable.  It should fairly simple for students to explain why capacitance will increase as the two conductors are brought closer together, but it may not be as apparent why the inductance will decrease.  A good "Socratic" question to ask is about magnetic field strength, assuming one end of the cable were shorted, and a DC current source connected to the other end.  Be sure to remind them to discuss the right-hand corkscrew rule for current and magnetic fields in their answer to this follow-up question!

%INDEX% Transmission line, dimensional effects on characteristic impedance
%INDEX% Characteristic impedance, effects of changing cable dimensions on
%INDEX% Surge impedance, effects of changing cable dimensions on

%(END_NOTES)


