
%(BEGIN_QUESTION)
% Copyright 2003, Tony R. Kuphaldt, released under the Creative Commons Attribution License (v 1.0)
% This means you may do almost anything with this work of mine, so long as you give me proper credit

A numeration system often used as a "shorthand" way of writing large binary numbers is the {\it octal}, or base-eight, system.

Based on what you know of place-weighted numeration systems, describe how many valid ciphers exist in the octal system, and the respective "weights" of each place in an octal number.

Also, perform the following conversions:

\medskip
\item{$\bullet$} $35_{8}$ into decimal:
\item{$\bullet$} $16_{10}$ into octal:
\item{$\bullet$} $110010_{2}$ into octal:
\item{$\bullet$} $51_{8}$ into binary:
\medskip

\underbar{file 01286}
%(END_QUESTION)





%(BEGIN_ANSWER)

There are only eight valid ciphers in the octal system (0, 1, 2, 3, 4, 5, 6, and 7), with each successive place carrying eight times the "weight" of the place before it.

\medskip
\item{$\bullet$} $35_{8}$ into decimal: $29_{10}$
\item{$\bullet$} $16_{10}$ into octal: $20_{8}$
\item{$\bullet$} $110010_{2}$ into octal: $62_{8}$
\item{$\bullet$} $51_{8}$ into binary: $101001_2$
\medskip

\vskip 10pt

Follow-up question: why is octal considered a "shorthand" notation for binary numbers?

%(END_ANSWER)





%(BEGIN_NOTES)

There are many references from which students may learn to perform these conversions.  You assistance should be minimal, as these procedures are simple to comprehend and easy to find.

%INDEX% Conversion, numeration base

%(END_NOTES)


