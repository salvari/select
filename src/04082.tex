
%(BEGIN_QUESTION)
% Copyright 2006, Tony R. Kuphaldt, released under the Creative Commons Attribution License (v 1.0)
% This means you may do almost anything with this work of mine, so long as you give me proper credit

Two people are debating electron flow versus conventional flow.  One of them says that the you will get different results predicting polarity of voltage drops in a resistive circuit depending on which convention you use.  The other person says the convention for labeling current does not matter at all, and that the correct polarities will be predicted either way.

Which of these two people is correct?  Explain why, and give an example to prove your point.

\underbar{file 04082}
%(END_QUESTION)





%(BEGIN_ANSWER)

I will let pictures show the answer to this question:

$$\epsfbox{04082x01.eps}$$

%(END_ANSWER)





%(BEGIN_NOTES)

It is important to remember that there is only one convention for using "+" and "-" symbols to designate the polarity of a voltage drop (thankfully!).

%INDEX% Conventional flow versus electron flow
%INDEX% Electron flow versus conventional flow

%(END_NOTES)


