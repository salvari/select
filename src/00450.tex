
%(BEGIN_QUESTION)
% Copyright 2003, Tony R. Kuphaldt, released under the Creative Commons Attribution License (v 1.0)
% This means you may do almost anything with this work of mine, so long as you give me proper credit

The decay of a variable (either voltage or current) in a time-constant circuit (RC or LR) follows this mathematical expression:

$$e^{-{t \over \tau}}$$

\noindent
Where,

$e =$ Euler's constant ($\approx 2.718281828$)

$t =$ Time, in seconds

$\tau =$ Time constant of circuit, in seconds

\vskip 10pt

Calculate the value of this expression as $t$ increases, given a circuit time constant ($\tau$) of 2 seconds.  Express this value as a {\it percentage}:

\medskip
\item{} $t =$ 1 second
\item{} $t =$ 2 seconds
\item{} $t =$ 3 seconds
\item{} $t =$ 4 seconds
\item{} $t =$ 5 seconds
\item{} $t =$ 6 seconds
\item{} $t =$ 7 seconds
\item{} $t =$ 8 seconds
\item{} $t =$ 9 seconds
\item{} $t =$ 10 seconds
\medskip

Also, express the percentage value of any {\it increasing} variables (either voltage or current) in an RC or LR charging circuit, for the same conditions (same times, same time constant).

\underbar{file 00450}
%(END_QUESTION)





%(BEGIN_ANSWER)

\medskip
\item{} $t =$ 1 second ; $e^{-{t \over \tau}} =$ 60.65\% ; increasing variable = 39.35\%
\item{} $t =$ 2 seconds ; $e^{-{t \over \tau}} =$ 36.79\% ; increasing variable = 63.21\%
\item{} $t =$ 3 seconds ; $e^{-{t \over \tau}} =$ 22.31\% ; increasing variable = 77.69\%
\item{} $t =$ 4 seconds ; $e^{-{t \over \tau}} =$ 13.53\% ; increasing variable = 86.47\%
\item{} $t =$ 5 seconds ; $e^{-{t \over \tau}} =$ 8.208\% ; increasing variable = 91.79\%
\item{} $t =$ 6 seconds ; $e^{-{t \over \tau}} =$ 4.979\% ; increasing variable = 95.02\%
\item{} $t =$ 7 seconds ; $e^{-{t \over \tau}} =$ 3.020\% ; increasing variable = 96.98\%
\item{} $t =$ 8 seconds ; $e^{-{t \over \tau}} =$ 1.832\% ; increasing variable = 98.17\%
\item{} $t =$ 9 seconds ; $e^{-{t \over \tau}} =$ 1.111\% ; increasing variable = 98.89\%
\item{} $t =$ 10 seconds ; $e^{-{t \over \tau}} =$ 0.6738\% ; increasing variable = 99.33\%
\medskip

%(END_ANSWER)





%(BEGIN_NOTES)

Do not simply tell your students how to calculate the values of the increasing variable.  Based on their qualitative knowledge of time-constant circuit curves and their ability to evaluate the downward (decay) expression, they should be able to figure out how to calculate the increasing variable's value over time as well.

Some students will insist that you give them an equation to do this.  They want to be told what to do, rather than solve the problem on their own based on an observation of pattern.  It is very important that students of any science learn to recognize patterns in data, and that they learn to fit that data into a mathematical equation.  If nothing else, these figures given in the answer for both decreasing and increasing variables should be plain enough.

%INDEX% Time constant calculation, RC or LR circuit

%(END_NOTES)


