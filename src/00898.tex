
%(BEGIN_QUESTION)
% Copyright 2003, Tony R. Kuphaldt, released under the Creative Commons Attribution License (v 1.0)
% This means you may do almost anything with this work of mine, so long as you give me proper credit

In order to simplify analysis of circuits containing PN junctions, a "standard" forward voltage drop is assumed for any conducting junction, the exact figure depending on the type of semiconductor material the junction is made of.

How much voltage is assumed to be dropped across a conducting {\it silicon} PN junction?  How much voltage is assumed for a forward-biased {\it germanium} PN junction?  Identify some factors that cause the real forward voltage drop of a PN junction to deviate from its "standard" figure.

\underbar{file 00898}
%(END_QUESTION)





%(BEGIN_ANSWER)

Silicon = 0.7 volts ; Germanium = 0.3 volts.

\vskip 10pt

Temperature, current, and doping concentration all affect the forward voltage drop of a PN junction.

%(END_ANSWER)





%(BEGIN_NOTES)

I've seen too many students gain the false impression that silicon PN junctions {\it always} drop 0.7 volts, no matter what the conditions.  This "fact" is emphasized so strongly in many textbooks that students usually don't think to ask when they measure a diode's forward voltage drop and find it to be considerably different than 0.7 volts!  It is very important that students realize this figure is an approximation only, used for the sake of (greatly) simplifying junction semiconductor circuit analysis.

%INDEX% Junction, PN
%INDEX% PN junction

%(END_NOTES)


