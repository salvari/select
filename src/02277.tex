
%(BEGIN_QUESTION)
% Copyright 2004, Tony R. Kuphaldt, released under the Creative Commons Attribution License (v 1.0)
% This means you may do almost anything with this work of mine, so long as you give me proper credit

When transmitting audio information (such as music and speech) in the form of radio waves, why bother modulating a high-frequency carrier signal?  Why not just connect a powerful audio amplifier straight to an antenna and broadcast the audio frequencies directly?

\underbar{file 02277}
%(END_QUESTION)





%(BEGIN_ANSWER)

There are several reasons you would {\it not} want to try to broadcast electromagnetic (radio) waves at audio frequencies.  A few of the most important are listed here:

\medskip
\goodbreak
\item{$\bullet$} The necessary size of the antenna.
\item{$\bullet$} Low transmission efficiency from inability to match antenna length to (changing) audio frequency.
\item{$\bullet$} Interference from other (similar) radio transmitters.
\medskip

Be prepared to explain {\it why} each of these factors effectively prohibits radio broadcasts at audio frequencies.

%(END_ANSWER)





%(BEGIN_NOTES)

The purpose of this question is to have students relate their understanding of basic RF and antenna theory to a very practical problem of broadcasting low-frequency (in this case, audio) information.  A fun exercise to do along with this question is to calculate the necessary physical dimensions of a quarter-wave ($\lambda \over 4$) antenna at a frequency of 2 kHz, keeping in mind that $\lambda = {v \over f}$ and $v \approx 3 \times 10^8$ meters per second.

%INDEX% Modulation, necessity of in radio broadcasts

%(END_NOTES)


