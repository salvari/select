
%(BEGIN_QUESTION)
% Copyright 2003, Tony R. Kuphaldt, released under the Creative Commons Attribution License (v 1.0)
% This means you may do almost anything with this work of mine, so long as you give me proper credit

Observe the following two waveforms, as represented on an oscilloscope display measuring output voltage of a filtered power supply:

$$\epsfbox{00792x01.eps}$$

If both of these waveforms were measured on the same power supply circuit, at different times, determine which waveform was measured during a period of heavier "loading" (a "heavier" load being defined as a load drawing {\it greater} current).

\underbar{file 00792}
%(END_QUESTION)





%(BEGIN_ANSWER)

The left-hand waveform was measured during a period of heavier loading.

%(END_ANSWER)





%(BEGIN_NOTES)

Ask your students what the term "loading" means in this context.  Some of them may not comprehend the term accurately, and so it is good to review just to make sure.

More importantly, discuss with your students why the ripple is more severe under conditions of heavy loading.  What, exactly, is happening in the circuit to produce this kind of waveform?  If it is necessary for us to maintain a low amount of ripple under this heavy loading, what must we change in the power supply circuit?

%INDEX% Ripple voltage, power supply

%(END_NOTES)


