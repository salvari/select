
%(BEGIN_QUESTION)
% Copyright 2006, Tony R. Kuphaldt, released under the Creative Commons Attribution License (v 1.0)
% This means you may do almost anything with this work of mine, so long as you give me proper credit

Determine the voltage output by the following R-2R ladder network given the switch states shown in the table:

$$\epsfbox{04007x01.eps}$$

% No blank lines allowed between lines of an \halign structure!
% I use comments (%) instead, so that TeX doesn't choke.

$$\vbox{\offinterlineskip
\halign{\strut
\vrule \quad\hfil # \ \hfil & 
\vrule \quad\hfil # \ \hfil & 
\vrule \quad\hfil # \ \hfil & 
\vrule \quad\hfil # \ \hfil & 
\vrule \quad\hfil # \ \hfil \vrule \cr
\noalign{\hrule}
%
% First row
SW$_{0}$ & SW$_{1}$ & SW$_{2}$ & SW$_{3}$ & $V_{out}$ \cr
%
\noalign{\hrule}
%
% Another row
Ground & Ground & Ground & $V_{ref}$ &  \cr
%
\noalign{\hrule}
%
% Another row
Ground & Ground & $V_{ref}$ & Ground & \cr
%
\noalign{\hrule}
%
% Another row
Ground & $V_{ref}$ & Ground & Ground &  \cr
% 
\noalign{\hrule}
%
% Another row
$V_{ref}$ & Ground & Ground & Ground &  \cr
% 
\noalign{\hrule}
%
% Another row
Ground & Ground & Ground & Ground &  \cr
%
\noalign{\hrule}
} % End of \halign 
}$$ % End of \vbox

\underbar{file 04007}
%(END_QUESTION)





%(BEGIN_ANSWER)

$$\vbox{\offinterlineskip
\halign{\strut
\vrule \quad\hfil # \ \hfil & 
\vrule \quad\hfil # \ \hfil & 
\vrule \quad\hfil # \ \hfil & 
\vrule \quad\hfil # \ \hfil & 
\vrule \quad\hfil # \ \hfil \vrule \cr
\noalign{\hrule}
%
% First row
SW$_{0}$ & SW$_{1}$ & SW$_{2}$ & SW$_{3}$ & $V_{out}$ \cr
%
\noalign{\hrule}
%
% Another row
Ground & Ground & Ground & $V_{ref}$ & 8 volts \cr
%
\noalign{\hrule}
%
% Another row
Ground & Ground & $V_{ref}$ & Ground & 4 volts \cr
%
\noalign{\hrule}
%
% Another row
Ground & $V_{ref}$ & Ground & Ground & 2 volts \cr
% 
\noalign{\hrule}
%
% Another row
$V_{ref}$ & Ground & Ground & Ground & 1 volt \cr
% 
\noalign{\hrule}
%
% Another row
Ground & Ground & Ground & Ground & 0 volts \cr
%
\noalign{\hrule}
} % End of \halign 
}$$ % End of \vbox

\vskip 10pt

Follow-up question: the fact that an R-2R resistor network is inherently linear, we may readily apply the {\it Superposition Theorem} to figure out what happens when more than one switch is moved to the $V_{ref}$ position.  Explain how you would apply Superposition to determine all output voltages for all possible combinations of switch positions.

%(END_ANSWER)





%(BEGIN_NOTES)

As you can see, the reference voltage value of 16 volts was not chosen at random!  I wanted students to see the pattern between single switch closures and binary place-weights for a four-bit number.  The actual electrical analyses for each condition are best expedited by applying Th\'evenin's theorem repeated to the circuit, condensing sections to single resistances and voltage sources until a simple voltage divider circuit is obtained at the output terminal.

The follow-up question is quite important.  Be sure to ask your students about it, for it holds the key to figuring out all output voltage values for all binary input possibilities.

%INDEX% R-2R ladder network

%(END_NOTES)


