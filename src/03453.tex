
%(BEGIN_QUESTION)
% Copyright 2005, Tony R. Kuphaldt, released under the Creative Commons Attribution License (v 1.0)
% This means you may do almost anything with this work of mine, so long as you give me proper credit

Although the {\it toggle} function of the J-K flip-flop is one of its most popular uses, this is not the only type of flip-flop capable of performing a toggle function.  Behold the surprisingly versatile D-type flip-flop configured to do the same thing:

$$\epsfbox{03453x01.eps}$$

Explain how this circuit performs the "toggle" function more commonly associated with J-K flip-flops.

\underbar{file 03453}
%(END_QUESTION)





%(BEGIN_ANSWER)

At each clock pulse, the flip-flop must switch to the opposite state because D receives inverted feedback from $\overline{Q}$.

%(END_ANSWER)





%(BEGIN_NOTES)

The main purpose of this question is to get students to see that toggling is not the exclusive domain of J-K flip-flops.  This fact may be particularly handy to know if one needs a toggle function in a circuit but only has a D-type flip-flop available, not a J-K flip-flop.

%INDEX% D-type flip-flop, configured to toggle
%INDEX% Toggle function, D-type flip-flop

%(END_NOTES)


