
%(BEGIN_QUESTION)
% Copyright 2003, Tony R. Kuphaldt, released under the Creative Commons Attribution License (v 1.0)
% This means you may do almost anything with this work of mine, so long as you give me proper credit

Plot the transfer function ($V_{out}$ versus $V_{in}$) for this opamp circuit:

$$\epsfbox{01017x01.eps}$$

What type of mathematical function is represented by this circuit?

\underbar{file 01017}
%(END_QUESTION)





%(BEGIN_ANSWER)

This circuit (ideally) represents a {\it linear} function ($y \propto x$):

$$\epsfbox{01017x02.eps}$$

%(END_ANSWER)





%(BEGIN_NOTES)

It should be obvious from inspection that the two opamp circuits represent inverse mathematical functions.  Ask your students why the final transfer function is linear rather than nonlinear.  After all, they should realize that each of the opamp circuits, taken individually, are very nonlinear.  Why would their combined effect be linear?

An interesting exercise would be to have your students perform inverse functions like this on their hand calculators, first calculating an exponential function ($f(x) = e^x$), then a logarithmic ($g(x) = \ln x$), and verifying the combined functions' output ($f[g(x)] = x$).

%INDEX% Exponentiator, nonlinear opamp circuit
%INDEX% Exponents and logarithms, as inverse functions (in a real nonlinear circuit)
%INDEX% Logarithm extractor, nonlinear opamp circuit
%INDEX% Logarithms and exponents, as inverse functions (in a real nonlinear circuit)

%(END_NOTES)


