
%(BEGIN_QUESTION)
% Copyright 2003, Tony R. Kuphaldt, released under the Creative Commons Attribution License (v 1.0)
% This means you may do almost anything with this work of mine, so long as you give me proper credit

The following schematic shows the workings of a simple AM radio receiver, with transistor amplifier:

$$\epsfbox{00611x01.eps}$$

The "tank circuit" formed of a parallel-connected inductor and capacitor network performs a very important filtering function in this circuit.  Describe what this filtering function is.

\underbar{file 00611}
%(END_QUESTION)





%(BEGIN_ANSWER)

The "tank circuit" filters out all the unwanted radio frequencies, so that the listener hears only one radio station broadcast at a time.

\vskip 10pt

Follow-up question: how might a variable capacitor be constructed, to suit the needs of a circuit such as this?  Note that the capacitance range for a tuning capacitor such as this is typically in the pico-Farad range.

%(END_ANSWER)





%(BEGIN_NOTES)

Challenge your students to describe how to change stations on this radio receiver.  For example, if we are listening to a station broadcasting at 1000 kHz and we want to change to a station broadcasting at 1150 kHz, what do we have to do to the circuit?

Be sure to discuss with them the construction of an adjustable capacitor (air dielectric).

%INDEX% Tank circuit, used in AM radio tuner

%(END_NOTES)


