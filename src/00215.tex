
%(BEGIN_QUESTION)
% Copyright 2003, Tony R. Kuphaldt, released under the Creative Commons Attribution License (v 1.0)
% This means you may do almost anything with this work of mine, so long as you give me proper credit

Many resistors have their electrical resistance shown by a set of color codes, or "bands," imprinted around their circumference.  A standard color code associates each color with a specific decimal digit (0 through 9).  Associate each of the following digits with its respective color:

\medskip
\item 0 = 
\item 1 =  
\item 2 =  
\item 3 = 
\item 4 = 
\item 5 = 
\item 6 = 
\item 7 = 
\item 8 = 
\item 9 = 
\medskip

\underbar{file 00215}
%(END_QUESTION)



%(BEGIN_ANSWER)

\medskip
\item 0 = Black
\item 1 = Brown
\item 2 = Red
\item 3 = Orange
\item 4 = Yellow
\item 5 = Green
\item 6 = Blue
\item 7 = Violet
\item 8 = Grey
\item 9 = White
\medskip

%(END_ANSWER)





%(BEGIN_NOTES)

Several limericks have been invented to remember this color code, most of them "politically incorrect."  I often challenge students to invent their own limericks for remembering this color code, and screen the inappropriate creations from general class discussion.

%INDEX% Color code, resistor

%(END_NOTES)


