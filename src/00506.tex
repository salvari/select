
%(BEGIN_QUESTION)
% Copyright 2003, Tony R. Kuphaldt, released under the Creative Commons Attribution License (v 1.0)
% This means you may do almost anything with this work of mine, so long as you give me proper credit

It is a common phenomenon for the electrical resistance of a substance to change with changes in temperature.  Explain how you would experimentally demonstrate this effect.

\underbar{file 00506}
%(END_QUESTION)





%(BEGIN_ANSWER)

It is a simple thing to demonstrate a substance's change in resistance with temperature.  I'm interested in finding out how you might discern a {\it quantitative} measurement of this change.  That is, how would you design an experiment so as to "attach a number" to the effect of resistance changing with temperature?

%(END_ANSWER)





%(BEGIN_NOTES)

This question is an excellent starting point for an in-class experiment.  There are several ways in which this effect could be demonstrated.

%INDEX% Resistance, relation to conductor temperature

%(END_NOTES)


