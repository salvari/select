
%(BEGIN_QUESTION)
% Copyright 2003, Tony R. Kuphaldt, released under the Creative Commons Attribution License (v 1.0)
% This means you may do almost anything with this work of mine, so long as you give me proper credit

The schematic diagram shown here is for an {\it "inverting" converter circuit}, a type of DC-DC "switching" power conversion circuit:

$$\epsfbox{02285x01.eps}$$

In this circuit, the transistor is either fully on or fully off; that is, driven between the extremes of saturation or cutoff.  By avoiding the transistor's "active" mode (where it would drop substantial voltage while conducting current), very low transistor power dissipations can be achieved.  With little power wasted in the form of heat, "switching" power conversion circuits are typically very efficient.

Trace all current directions during both states of the transistor.  Also, mark the inductor's voltage polarity during both states of the transistor.

\underbar{file 02285}
%(END_QUESTION)





%(BEGIN_ANSWER)

$$\epsfbox{02285x02.eps}$$

%(END_ANSWER)





%(BEGIN_NOTES)

Ask your students why they think this circuit is called an {\it inverting} converter.  

Although it may not be evident from viewing the circuit schematic, this converter circuit is capable of stepping voltage up {\it or} down, making it quite versatile.

%INDEX% Inverting converter circuit

%(END_NOTES)


