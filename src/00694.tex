
%(BEGIN_QUESTION)
% Copyright 2003, Tony R. Kuphaldt, released under the Creative Commons Attribution License (v 1.0)
% This means you may do almost anything with this work of mine, so long as you give me proper credit

The {\it Superposition Theorem} works nicely to calculate voltages and currents in resistor circuits.  But can it be used to calculate power dissipations as well?  Why or why not?  Be specific with your answer.

\underbar{file 00694}
%(END_QUESTION)





%(BEGIN_ANSWER)

The Superposition Theorem cannot be directly used to calculate power.

%(END_ANSWER)





%(BEGIN_NOTES)

In order to answer this question correctly (without just looking up the answer in a book), students will have to perform a few power calculations in simple, multiple-source circuits.  It may be worthwhile to work through a couple of example problems during discussion time, to illustrate the answer.

Despite the fact that resistor power dissipations cannot be superimposed to obtain the answer(s), it is still possible to use the Superposition Theorem to calculate resistor power dissipations in a multiple-source circuit.  Challenge your students with the task of applying this theorem for solving power dissipations in a circuit.

%INDEX% Superposition theorem, inapplicability to (direct) power calculations

%(END_NOTES)


