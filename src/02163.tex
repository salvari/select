
%(BEGIN_QUESTION)
% Copyright 2004, Tony R. Kuphaldt, released under the Creative Commons Attribution License (v 1.0)
% This means you may do almost anything with this work of mine, so long as you give me proper credit

Explain why a bipolar junction transistor tends to regulate collector current over a wide range of collector-to-emitter voltage drops when its base current is constant.  What happens internally that makes the BJT's collector current relatively independent of collector-to-emitter voltage and strongly dependent on base current?

\underbar{file 02163}
%(END_QUESTION)





%(BEGIN_ANSWER)

Because the BJT is a minority carrier device, the vast majority of collector current is the result of charge carriers injected from the emitter into the base region.  Since this rate of charge carrier injection is a function of base-emitter junction excitation, base current (or more properly, base-to-emitter voltage) primarily determines collector current with collector-to-emitter voltage playing a relatively minor role.

%(END_ANSWER)





%(BEGIN_NOTES)

The current-regulating nature of a BJT is made more understandable by analyzing an energy band diagram of the transistor in active mode.

%INDEX% Active mode, BJT
%INDEX% BJT, as current regulator

%(END_NOTES)


