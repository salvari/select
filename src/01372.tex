
%(BEGIN_QUESTION)
% Copyright 2003, Tony R. Kuphaldt, released under the Creative Commons Attribution License (v 1.0)
% This means you may do almost anything with this work of mine, so long as you give me proper credit

If the clock frequency driving this flip-flop is 240 Hz, what is the frequency of the flip-flop's output signals (either $Q$ or $\overline{Q}$)?

$$\epsfbox{01372x01.eps}$$

\underbar{file 01372}
%(END_QUESTION)





%(BEGIN_ANSWER)

$f_{out}$ = 120 Hz

\vskip 10pt

Follow-up question: how could you use another flip-flop to obtain a square-wave signal of 60 Hz from this circuit?

%(END_ANSWER)





%(BEGIN_NOTES)

Ask your students to think of some practical applications for this type of circuit.  For those who are musically inclined, ask them what the {\it musical} relationship is between notes whose frequencies are an exact 2:1 ratio (hint: it's the same interval as {\it eight} white keys on a piano keyboard).  How could a circuit such as this possibly be used in a musical synthesizer?

%INDEX% Frequency divider, J-K flip-flop

%(END_NOTES)


