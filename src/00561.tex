
%(BEGIN_QUESTION)
% Copyright 2003, Tony R. Kuphaldt, released under the Creative Commons Attribution License (v 1.0)
% This means you may do almost anything with this work of mine, so long as you give me proper credit

A {\it Jacob's Ladder} is a novelty device designed to produce high-voltage arcs.  It consists of a source of high voltage (usually a step-up "transformer" which takes the 120 volt AC utility power voltage and increases it to several thousand volts AC) and a pair of stiff wires or metal rods with a small air gap near the bottom and a large air gap near the top:

$$\epsfbox{00561x01.eps}$$

When powered, an arc develops at the point where the two rods are closest, then travels up the length of the rods, becoming wider and wider, until it "breaks" off the top of the rods.  Once the arc is extinguished, a new arc forms at the bottom of the rods again.

Explain why the arc begins at the point where the rods are closest together.  Also, explain why the arc then moves upward along the rods' length, rather than remaining at the bottom where the gap is shorter.

\underbar{file 00561}
%(END_QUESTION)





%(BEGIN_ANSWER)

The arc begins at the shortest gap because that is where the electric field strength (volts per inch) is greatest.  Once the air ionizes, its resistance decreases dramatically, and the voltage between the two rods likewise decreases due to the "loading" effect of the arc.

\vskip 10pt

Challenge question: how do the principles behind the operation of a "Jacob's Ladder" relate to {\it arc flash} and {\it arc blast} in high-current electrical faults?

%(END_ANSWER)





%(BEGIN_NOTES)

Discuss with your students the nature of gas ionization, and why it occurs at the shortest gap first.  Also discuss why the transformer's output voltage decreases once the arc forms.  This has nothing to do with the transformer in particular as a power source, but would tend to happen with any high-voltage power source due to internal resistance. 

One method of documenting the arc process in detail is to draw a graph of voltage between the rods over time:

$$\epsfbox{00561x02.eps}$$

The most important point of this question, though, is to relate these principles to the topic of "arc blast".  How does the dramatic decrease in air resistance (after ionization) relate to the magnitude of arc current?  How does the motion of the arc in a Jacob's Ladder relate to the high temperatures of an arc?  What about the difference in gaps between the bottom of the rods versus the top -- how does this relate to the physical size of arcs in an electrical power system fault?

%INDEX% Arc, electric
%INDEX% Electric arc

%(END_NOTES)


