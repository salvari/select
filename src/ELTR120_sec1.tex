
\centerline{\bf ELTR 120 (Semiconductors 1), section 1} \bigskip 
 
\vskip 10pt

\noindent
{\bf Recommended schedule}

\vskip 5pt

%%%%%%%%%%%%%%%
\hrule \vskip 5pt
\noindent
\underbar{Day 1}

\hskip 10pt Topics: {\it Semiconductor theory and PN junctions}
 
\hskip 10pt Questions: {\it 1 through 20}
 
\hskip 10pt Lab Exercise: {\it Rectifier diode characteristics (question 91)}
 
%INSTRUCTOR \hskip 10pt {\bf Explain suggested project ideas to students}

%INSTRUCTOR \hskip 10pt {\bf Give project grading rubric to students, complete with deadlines}

%INSTRUCTOR \hskip 10pt {\bf Socratic Electronics animation: Semiconductor diode junction}

%INSTRUCTOR \hskip 10pt {\bf Demo: Set up simple curve tracer circuit and show diode curve}

\vskip 10pt
%%%%%%%%%%%%%%%
\hrule \vskip 5pt
\noindent
\underbar{Day 2}

\hskip 10pt Topics: {\it Diodes and rectifier circuits}
 
\hskip 10pt Questions: {\it 21 through 40}
 
\hskip 10pt Lab Exercise: {\it Full-wave, center-tap rectifier circuit (question 92)}
 
%INSTRUCTOR \hskip 10pt {\bf MIT 6.002 video clip: Disk 1, Lecture 1; V/I component plots 29:20 to 35:12}

%INSTRUCTOR \hskip 10pt {\bf MIT 6.002 video clip: Disk 1, Lecture 7; animated diode V/I plot 10:03 to 12:12}

%INSTRUCTOR \hskip 10pt {\bf MIT 6.002 video clip: Disk 1, Lecture 7; real diode V/I plot 12:27 to 13:53}

\vskip 10pt
%%%%%%%%%%%%%%%
\hrule \vskip 5pt
\noindent
\underbar{Day 3}

\hskip 10pt Topics: {\it AC-DC power supply circuits and troubleshooting}
 
\hskip 10pt Questions: {\it 41 through 60}
 
\hskip 10pt Lab Exercise: {\it Full-wave bridge rectifier circuit (question 93)}
 
%INSTRUCTOR \hskip 10pt {\bf Socratic Electronics animation: Bridge rectifier circuit}

\vskip 10pt
%%%%%%%%%%%%%%%
\hrule \vskip 5pt
\noindent
\underbar{Day 4}

\hskip 10pt Topics: {\it Special diodes and zener voltage regulators}
 
\hskip 10pt Questions: {\it 61 through 80}
 
\hskip 10pt Lab Exercise: {\it Zener diode voltage regulator circuit (question 94)}
 
\vskip 10pt
%%%%%%%%%%%%%%%
\hrule \vskip 5pt
\noindent
\underbar{Day 5}

\hskip 10pt Topics: {\it Electron versus Conventional flow notation}
 
\hskip 10pt Questions: {\it 81 through 90}
 
\hskip 10pt Lab Exercise: {\it LED current limiting (question 95)}
 
\vskip 10pt
%%%%%%%%%%%%%%%
\hrule \vskip 5pt
\noindent
\underbar{Day 6}

\hskip 10pt Exam 1: {\it includes rectifier circuit performance assessment}
 
\hskip 10pt Project selection: {\it Initial project design checked by instructor and components selected (Dual output AC-DC power supply \underbar{strongly} recommended)}
  
\hskip 10pt Lab Exercise: {\it Work on project}

\vskip 10pt
%%%%%%%%%%%%%%%%
\hrule \vskip 5pt
\noindent
\underbar{Troubleshooting practice problems}

\hskip 10pt Questions: {\it 97 through 106}
 
\vskip 10pt
%%%%%%%%%%%%%%%
\hrule \vskip 5pt
\noindent
\underbar{General concept practice and challenge problems}

\hskip 10pt Questions: {\it 107 through the end of the worksheet}
 
\vskip 10pt
%%%%%%%%%%%%%%%
\hrule \vskip 5pt
\noindent
\underbar{Impending deadlines}

\hskip 10pt {\bf Project due at end of ELTR120, Section 3}
 
\hskip 10pt Question 96: Sample project grading criteria
 
\vskip 10pt
%%%%%%%%%%%%%%%











\vfil \eject

\centerline{\bf ELTR 120 (Semiconductors 1), section 1} \bigskip 
 
\vskip 10pt

\noindent
{\bf Skill standards addressed by this course section}

\vskip 5pt

%%%%%%%%%%%%%%%
\hrule \vskip 10pt
\noindent
\underbar{EIA {\it Raising the Standard; Electronics Technician Skills for Today and Tomorrow}, June 1994}

\vskip 5pt

\medskip
\item{\bf D} {\bf Technical Skills -- Discrete Solid-State Devices}
\item{\bf D.01} Demonstrate an understanding of the properties of semiconducting materials.
\item{\bf D.02} Demonstrate an understanding of PN junctions.
\item{\bf D.05} Demonstrate an understanding of special diodes and transistors.  {\it Partially met -- special diodes only.}
\item{\bf D.06} Understand principles and operations of diode circuits.
\item{\bf D.07} Fabricate and demonstrate diode circuits.
\item{\bf D.08} Troubleshoot and repair diode circuits.
\item{\bf E} {\bf Technical Skills -- Analog Circuits}
\item{\bf E.07} Understand principles and operations of linear power supplies and filters.
\item{\bf E.08} Fabricate and demonstrate linear power supplies and filters.
\item{\bf E.09} Troubleshoot and repair linear power supplies and filters.
\medskip

\vskip 5pt

\medskip
\item{\bf B} {\bf Basic and Practical Skills -- Communicating on the Job}
\item{\bf B.01} Use effective written and other communication skills.  {\it Met by group discussion and completion of labwork.}
\item{\bf B.03} Employ appropriate skills for gathering and retaining information.  {\it Met by research and preparation prior to group discussion.}
\item{\bf B.04} Interpret written, graphic, and oral instructions.  {\it Met by completion of labwork.}
\item{\bf B.06} Use language appropriate to the situation.  {\it Met by group discussion and in explaining completed labwork.}
\item{\bf B.07} Participate in meetings in a positive and constructive manner.  {\it Met by group discussion.}
\item{\bf B.08} Use job-related terminology.  {\it Met by group discussion and in explaining completed labwork.}
\item{\bf B.10} Document work projects, procedures, tests, and equipment failures.  {\it Met by project construction and/or troubleshooting assessments.}
\item{\bf C} {\bf Basic and Practical Skills -- Solving Problems and Critical Thinking}
\item{\bf C.01} Identify the problem.  {\it Met by research and preparation prior to group discussion.}
\item{\bf C.03} Identify available solutions and their impact including evaluating credibility of information, and locating information.  {\it Met by research and preparation prior to group discussion.}
\item{\bf C.07} Organize personal workloads.  {\it Met by daily labwork, preparatory research, and project management.}
\item{\bf C.08} Participate in brainstorming sessions to generate new ideas and solve problems.  {\it Met by group discussion.}
\item{\bf D} {\bf Basic and Practical Skills -- Reading}
\item{\bf D.01} Read and apply various sources of technical information (e.g. manufacturer literature, codes, and regulations).  {\it Met by research and preparation prior to group discussion.}
\item{\bf E} {\bf Basic and Practical Skills -- Proficiency in Mathematics}
\item{\bf E.01} Determine if a solution is reasonable.
\item{\bf E.02} Demonstrate ability to use a simple electronic calculator.
\item{\bf E.05} Solve problems and [sic] make applications involving integers, fractions, decimals, percentages, and ratios using order of operations.
\item{\bf E.06} Translate written and/or verbal statements into mathematical expressions.
\item{\bf E.09} Read scale on measurement device(s) and make interpolations where appropriate.  {\it Met by oscilloscope usage.}
\item{\bf E.12} Interpret and use tables, charts, maps, and/or graphs.
\item{\bf E.13} Identify patterns, note trends, and/or draw conclusions from tables, charts, maps, and/or graphs.
\item{\bf E.15} Simplify and solve algebraic expressions and formulas.
\item{\bf E.16} Select and use formulas appropriately.
\item{\bf E.17} Understand and use scientific notation.
\item{\bf F} {\bf Basic and Practical Skills -- Proficiency in Physics}
\item{\bf F.04} Understand principles of electricity including its relationship to the nature of matter.
\medskip

%%%%%%%%%%%%%%%




\vfil \eject

\centerline{\bf ELTR 120 (Semiconductors 1), section 1} \bigskip 
 
\vskip 10pt

\noindent
{\bf Common areas of confusion for students}

\vskip 5pt

%%%%%%%%%%%%%%%
\hrule \vskip 5pt

\vskip 10pt

\noindent
{\bf Difficult concept: } {\it Quantum physics.}

One of my "pet peeves" regarding introductory electronics textbooks is that they commonly attempt to explain the workings of PN semiconductor junctions while holding to Rutherford's obsolete planetary model of the atom.  Electrons {\it do not} circle atomic nuclei like little planets, free to travel arbitrary orbits.  Instead, they may only assume a limited number of energy states, necessitating "quantum leaps" to change between them.  It is this discrete behavior that makes semiconductor devices possible.  Thankfully, there are a great many plainly-understandable resources on the internet and in some modern textbooks explaining this, and the relationship to current in semiconductor materials.

\vskip 10pt

\noindent
{\bf Difficult concept: } {\it RMS versus peak and average measurements.}

The very idea of assigning a fixed number for AC voltage or current that (by definition) constantly changes magnitude and direction seems strange.  Consequently, there is more than one way to do it.  We may assign that value according to the {\it highest} magnitude reached in a cycle, in which case we call it the {\it peak} measurement.  We may mathematically integrate the waveform over time to figure the mean magnitude, in which case we call it the {\it average} measurement.  Or we may figure out what level of DC (voltage or current) causes the exact same amount of average power to be dissipated by a standard resistive load, in which case we call it the {\it RMS} measurement.  One common mistake here is to think that the relationship between RMS, average, and peak measurements is a matter of fixed ratios.  The number "0.707" is memorized by every beginning electronics student as the ratio between RMS and peak, but what is commonly overlooked is that this particular ratio holds true {\it for perfect sine-waves only!}  A wave with a different shape will have a different mathematical relationship between peak and RMS values.

\vskip 10pt

\noindent
{\bf Difficult concept: } {\it Zener diode voltage regulator operation.}

Zener diode voltage regulators are often difficult for students to grasp because the diodes themselves are so highly nonlinear.  One cannot apply any variation of Ohm's Law to a zener diode, and this makes the circuit seem intractable at first glance.  A "trick" I often apply to the circuits is to first imagine the zener diode failed open and see whether or not the voltage across the (open) diode terminals exceeds the diode's zener voltage rating.  If so, then you know the diode will actually be clipping (limiting) voltage to that rated value, and you may proceed with your circuit analysis assuming that much voltage across any components parallel to the zener diode.  If not, you know the diode will not be conducting current, and you may treat it as if it is truly failed open!

\vskip 10pt

\noindent
{\bf Difficult concept: } {\it Fourier analysis.}

No doubt about it, Fourier analysis is a strange concept to understand.  Strange, but incredibly useful!  While it is relatively easy to grasp the principle that we may create a square-shaped wave (or any other symmetrical waveshape) by mixing together the right combinations of sine waves at different frequencies and amplitudes, it is far from obvious that {\it any} periodic waveform may be decomposed into a series of sinusoidal waves the same way.  The practical upshot of this is that is it possible to consider very complex waveshapes as being nothing more than a bunch of sine waves added together.  Since sine waves are easy to analyze in the context of electric circuits, this means we have a way of simplifying what would otherwise be a dauntingly complex problem: analyzing how circuits respond to non-sinusoidal waveforms.

The actual "nuts and bolts" of Fourier analysis is highly mathematical and well beyond the scope of this course.  Right now all I want you to grasp is the concept and significance of equivalence between arbitrary waveshapes and series of sine waves.

A great way to experience this equivalence is to play with a digital oscilloscope with a built-in spectrum analyzer.  By introducing different wave-shape signals to the input and switching back and forth between the time-domain (scope) and frequency-domain (spectrum) displays, you may begin to see patterns that will enlighten your understanding.

\vskip 10pt

\noindent
{\bf Common mistake: } {\it Failing to respect shock hazard of line-powered circuits.}

Students should review the principles of electrical safety prior to building the dual-output AC/DC power supply.  Unlike nearly all the previous labs which harbored little or no shock hazard, this project can shock you.  The most important rule you can follow is to simply unplug the circuit from the AC line before reaching toward any part of the circuit with your hand or with a conductive tool.  The only things you should touch a live circuit with are test probes for measurement equipment!  Another common mistake is to fail to remove conductive jewelry (bracelets, rings, etc.) prior to working with line-powered circuits.


