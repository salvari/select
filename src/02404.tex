
%(BEGIN_QUESTION)
% Copyright 2005, Tony R. Kuphaldt, released under the Creative Commons Attribution License (v 1.0)
% This means you may do almost anything with this work of mine, so long as you give me proper credit

Draw the polarities (+ and -) of the applied voltages necessary to turn both these transistors on:

$$\epsfbox{02404x01.eps}$$

Also, draw the direction of the {\it controlled} current (flowing between collector and emitter) that will result from a power source properly connected between these terminals.

\underbar{file 02404}
%(END_QUESTION)





%(BEGIN_ANSWER)

$$\epsfbox{02404x02.eps}$$

\vskip 10pt

Follow-up question: draw the voltage sources necessary for generating the "controlled" current traced in these diagrams, so that the applied voltage polarity between collector and emitter is evident.

%(END_ANSWER)





%(BEGIN_NOTES)

This is a very important concept for students to grasp: how to turn a BJT on with an applied voltage between base and emitter, and also which direction the controlled current goes through it.  Be sure to spend time discussing this, for it is fundamental to their understanding of BJT operation.

%INDEX% BJT, conditions necessary for conduction
%INDEX% Junction biasing, for BJT during conduction

%(END_NOTES)


