
%(BEGIN_QUESTION)
% Copyright 2003, Tony R. Kuphaldt, released under the Creative Commons Attribution License (v 1.0)
% This means you may do almost anything with this work of mine, so long as you give me proper credit

A student measures voltage drops in an AC circuit using three voltmeters and arrives at the following measurements:

$$\epsfbox{01566x01.eps}$$

Upon viewing these measurements, the student becomes very perplexed.  Aren't voltage drops supposed to {\it add} in series, just as in DC circuits?  Why, then, is the total voltage in this circuit only 10.8 volts and not 15.74 volts?  How is it possible for the total voltage in an AC circuit to be substantially less than the simple sum of the components' voltage drops?

Another student, trying to be helpful, suggests that the answer to this question might have something to do with RMS versus peak measurements.  A third student disagrees, proposing instead that at least one of the meters is badly out of calibration and thus not reading correctly.

When you are asked for your thoughts on this problem, you realize that neither of the answers proposed thus far are correct.  Explain the real reason for the "discrepancy" in voltage measurements, and also explain how you could experimentally disprove the other answers (RMS vs. peak, and bad calibration).

\underbar{file 01566}
%(END_QUESTION)





%(BEGIN_ANSWER)

AC voltages still add in series, but {\it phase} must also be accounted for when doing so.  Unfortunately, multimeters provide no indication of phase whatsoever, and thus do not provide us with all the information we need.  (Note: just by looking at this circuit's components, though, you should still be able to calculate the correct result for total voltage and validate the measurements.)

I'll let you determine how to disprove the two incorrect explanations offered by the other students!

\vskip 10pt

Challenge question: calculate a set of possible values for the capacitor and resistor that would generate these same voltage drops in a real circuit.  Hint: you must also decide on a value of frequency for the power source.

%(END_ANSWER)





%(BEGIN_NOTES)

This question has two different layers: first, how to reconcile the "strange" voltage readings with Kirchhoff's Voltage Law; and second, how to experimentally validate the accuracy of the voltmeters and the fact that they are all registering the same type of voltage (RMS, peak, or otherwise, it doesn't matter).  The first layer of this question regards the basic concepts of AC phase, while the second exercises troubleshooting and critical thinking skills.  Be sure to discuss both of these topics in class with your students.

%INDEX% Series voltages, AC
%INDEX% Voltages in series, AC

%(END_NOTES)


