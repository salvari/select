
%(BEGIN_QUESTION)
% Copyright 2003, Tony R. Kuphaldt, released under the Creative Commons Attribution License (v 1.0)
% This means you may do almost anything with this work of mine, so long as you give me proper credit

A DC motor may be thought of as a series of electromagnets, radially spaced around a common shaft:

$$\epsfbox{00394x01.eps}$$

This particular motor is of the "permanent magnet" type, with wire windings only on the armature.

Write the necessary magnetic polarities ("N" for north and "S" for south) on the armature's electromagnet pole tips, in order to sustain a {\it clockwise} rotation.

\underbar{file 00394}
%(END_QUESTION)





%(BEGIN_ANSWER)

$$\epsfbox{00394x02.eps}$$

\vskip 10pt

Follow-up question: suppose this motor did not rotate like it was supposed to when energized.  Identify some possible (specific) failures that could result in the motor not moving upon energization.

%(END_ANSWER)





%(BEGIN_NOTES)

The illustration shown in both the question and the answer provides a good medium for discussing commutation.  Discuss with your students how, in order for the motor's rotation to be continuous, the electromagnets radially spaced around the shaft must energize and de-energize at the right times to always be "pulling" and "pushing" in the correct direction.

Be sure to spend time on the follow-up question with your students, considering non-electrical as well as electrical fault possibilities.

%INDEX% Direction of rotation, DC motor
%INDEX% Rotational direction, DC motor

%(END_NOTES)


