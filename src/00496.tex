
%(BEGIN_QUESTION)
% Copyright 2003, Tony R. Kuphaldt, released under the Creative Commons Attribution License (v 1.0)
% This means you may do almost anything with this work of mine, so long as you give me proper credit

Is it appropriate to assign a phasor {\it angle} to a single AC voltage, all by itself in a circuit?

$$\epsfbox{00496x01.eps}$$

What if there is more than one AC voltage source in a circuit?

$$\epsfbox{00496x02.eps}$$

\underbar{file 00496}
%(END_QUESTION)





%(BEGIN_ANSWER)

Phasor angles are {\it relative}, not {\it absolute}.  They have meaning only where there is another phasor to compare against.

Angles may be associated with multiple AC voltage sources in the same circuit, but {\it only if those voltages are all at the same frequency}.

%(END_ANSWER)





%(BEGIN_NOTES)

Discuss with your students the notion of "phase angle" in relation to AC quantities.  What does it mean, exactly, if a voltage is "3 volts at an angle of 90 degrees"?  You will find that such a description only makes sense where there is another voltage (i.e., "4 volts at 0 degrees") to compare to.  Without a frame of reference, phasor angles are meaningless.

Also discuss with your students the nature of phase shifts between different AC voltage sources, if the sources are all at different frequencies.  Would the phase angles be fixed, or vary over time?  Why?  In light of this, why do we not assign phase angles when different frequencies are involved?

%INDEX% Phase angle, AC sources

%(END_NOTES)


