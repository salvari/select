
%(BEGIN_QUESTION)
% Copyright 2003, Tony R. Kuphaldt, released under the Creative Commons Attribution License (v 1.0)
% This means you may do almost anything with this work of mine, so long as you give me proper credit

How could we possibly determine the phasing of the windings in this step-down transformer (120 VAC, 60 Hz primary) so we would know how to connect them for either boosting or bucking service?

$$\epsfbox{00244x01.eps}$$

\underbar{file 00244}
%(END_QUESTION)





%(BEGIN_ANSWER)

Connect one of the secondary wires to one of the primary wires.  Then, measure voltage with a voltmeter between the other two secondary and primary winding wires (the two that are not electrically common).  The voltage reading obtained by this test will indicate the phasing.

%(END_ANSWER)





%(BEGIN_NOTES)

Note that the answer given to this question leaves the particular details of how to interpret the voltmeter readings unanswered.  Challenge your students to figure this out on their own.

Given that autotransformer connections are so practical, it is important for students to know how to test a transformer to see how its (unmarked) windings are phased.

%INDEX% Polarity, transformer windings
%INDEX% Transformer winding "polarity"

%(END_NOTES)


