
%(BEGIN_QUESTION)
% Copyright 2003, Tony R. Kuphaldt, released under the Creative Commons Attribution License (v 1.0)
% This means you may do almost anything with this work of mine, so long as you give me proper credit

The simple JFET amplifier circuit shown here (built with surface-mount components) employs a biasing technique known as {\it self-biasing}:

$$\epsfbox{01181x01.eps}$$

Self-biasing provides much greater Q-point stability than gate-biasing.  Draw a schematic diagram of this circuit, and then explain how self-biasing works.

\underbar{file 01181}
%(END_QUESTION)





%(BEGIN_ANSWER)

$$\epsfbox{01181x02.eps}$$

Self-biasing uses the negative feedback created by a source resistor to establish a "natural" Q-point for the amplifier circuit, rather than having to supply an external voltage as is done with gate biasing.

%(END_ANSWER)





%(BEGIN_NOTES)

The concept of negative feedback is extremely important in electronic circuits, but it is not easily grasped by all.  Self-biasing of JFET transistors is a relatively easy-to-understand application of negative feedback, so be sure to take advantage of this opportunity to explore the concept with your students.

Ask your students to explain why Q-point stability is a desirable feature for mass-produced amplifier circuits, as well as circuits subject to component-level repair.

%INDEX% Self-biasing, JFET amplifier circuit

%(END_NOTES)


