
%(BEGIN_QUESTION)
% Copyright 2003, Tony R. Kuphaldt, released under the Creative Commons Attribution License (v 1.0)
% This means you may do almost anything with this work of mine, so long as you give me proper credit

Find one or two real junction field-effect transistors and bring them with you to class for discussion.  Identify as much information as you can about your transistors prior to discussion:

\medskip
\item{$\bullet$} Terminal identification (which terminal is gate, source, drain)
\item{$\bullet$} Continuous power rating
\item{$\bullet$} Typical transconductance 
\medskip

\underbar{file 01165}
%(END_QUESTION)





%(BEGIN_ANSWER)

If possible, find a manufacturer's datasheet for your components (or at least a datasheet for a similar component) to discuss with your classmates.  Be prepared to {\it prove} the terminal identifications of your transistors in class, by using a multimeter!

%(END_ANSWER)





%(BEGIN_NOTES)

The purpose of this question is to get students to kinesthetically interact with the subject matter.  It may seem silly to have students engage in a "show and tell" exercise, but I have found that activities such as this greatly help some students.  For those learners who are kinesthetic in nature, it is a great help to actually {\it touch} real components while they're learning about their function.  Of course, this question also provides an excellent opportunity for them to practice interpreting component markings, use a multimeter, access datasheets, etc.

%INDEX% JFET, identification of parameters

%(END_NOTES)


