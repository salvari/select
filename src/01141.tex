
%(BEGIN_QUESTION)
% Copyright 2003, Tony R. Kuphaldt, released under the Creative Commons Attribution License (v 1.0)
% This means you may do almost anything with this work of mine, so long as you give me proper credit

If you have ever heard the sound of a dual-engine airplane flying overhead, you probably noticed an unusual "beat" pattern to the engines' tone.  The same phenomenon happens when you play the output of two audio signal generators through speakers, with the signal generators set to very similar (but not identical!) frequencies:

$$\epsfbox{01141x01.eps}$$

Explain how this phenomenon could be used by a technician to adjust two audio frequency sources to the same frequency, without the aid of any expensive test equipment such as a frequency counter or an oscilloscope.

\underbar{file 01141}
%(END_QUESTION)





%(BEGIN_ANSWER)

Adjust the frequency of one of the sources until no "beating" sound is heard anymore between the two audio tones.

%(END_ANSWER)





%(BEGIN_NOTES)

This is a very simple experiment to set up in the classroom, but beware!  I once had two audio signal generators creating low-frequency sine-wave tones for my students, shortly after their lunch break.  With full stomachs, a couple of my students had a very difficult time remaining awake while listening to the low tones.

%(END_NOTES)


