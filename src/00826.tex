
%(BEGIN_QUESTION)
% Copyright 2003, Tony R. Kuphaldt, released under the Creative Commons Attribution License (v 1.0)
% This means you may do almost anything with this work of mine, so long as you give me proper credit

Suppose an AC signal amplifier circuit has a voltage gain (ratio) of 2.  That is, $V_{out}$ is twice as large as $V_{in}$:

$$\epsfbox{00826x01.eps}$$

If we were to try to rate this amplifier's gain in terms of the relative {\it power} dissipated by a given load resistance ($P_{load}$ when powered by $V_{out}$, versus $P_{load}$ when powered by $V_{in}$), what ratio would we calculate?  In other words, what is the ratio of power for a given load resistance, when powered by a given voltage, versus when powered by a voltage that is twice as much?

\underbar{file 00826}
%(END_QUESTION)





%(BEGIN_ANSWER)

Power ratio = 4:1

%(END_ANSWER)





%(BEGIN_NOTES)

An easy way to illustrate this principle is to ask your students to calculate the power dissipation of a 1200 watt heating element rated for 120 volts, if connected to a 240 volt source.  The answer is {\it not} 2400 watts!

%INDEX% Gain, relationship of voltage gain to power gain (as ratios)

%(END_NOTES)


