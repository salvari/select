
%(BEGIN_QUESTION)
% Copyright 2003, Tony R. Kuphaldt, released under the Creative Commons Attribution License (v 1.0)
% This means you may do almost anything with this work of mine, so long as you give me proper credit

Electronic ignition systems for gasoline-powered engines typically use a device called a {\it reluctor} to trigger the transistor to turn on and off.  Shown here is a simple reluctor-based electronic ignition system:

$$\epsfbox{00448x01.eps}$$

Explain how this circuit functions.  Why do you think the triggering device is called a "reluctor"?  What advantage(s) does this circuit have over a mechanical "point" operated ignition system?

\underbar{file 00448}
%(END_QUESTION)





%(BEGIN_ANSWER)

The "reluctor" generates pulses of current to the transistor's base to turn it on and off.  The word "reluctor" is applied to this device in honor of a certain magnetic principle you should know!

%(END_ANSWER)





%(BEGIN_NOTES)

Discuss the advantages of a reluctor-triggered ignition system with your students.  As far as I am aware, the system possesses no disadvantages when compared against mechanical point-driven systems.  

An interesting side note: one method of testing a reluctor-driven ignition system at high frequencies was to hold the tip of a soldering gun (not a soldering {\it iron}!) next to the pickup coil and pull the trigger.  The strong magnetic field produced by the gun's high current (60 Hz AC) would trigger the ignition system to deliver 60 sparks per second.

Some of your students familiar with engine ignition systems will notice that there is no distributor for multiple spark plugs.  In other words, this circuit is for a single-cylinder engine!  I chose not to draw a distributor in this schematic just to keep things simple.

%INDEX% Electronic ignition system, engine
%INDEX% Ignition system (electronic), engine
%INDEX% Transistor switch circuit (BJT)

%(END_NOTES)


