
%(BEGIN_QUESTION)
% Copyright 2006, Tony R. Kuphaldt, released under the Creative Commons Attribution License (v 1.0)
% This means you may do almost anything with this work of mine, so long as you give me proper credit

Predict how all component voltages and currents in this circuit will be affected as a result of the following faults.  Consider each fault independently (i.e. one at a time, no multiple faults):

$$\epsfbox{03936x01.eps}$$

\medskip
\item{$\bullet$} Voltage source $V_1$ fails with increased output:
\vskip 5pt
\item{$\bullet$} Resistor $R_1$ fails open: 
\vskip 5pt
\item{$\bullet$} Resistor $R_3$ fails open: 
\vskip 5pt
\item{$\bullet$} Solder bridge (short) across resistor $R_2$: 
\medskip

For each of these conditions, explain {\it why} the resulting effects will occur.

\underbar{file 03936}
%(END_QUESTION)





%(BEGIN_ANSWER)

\medskip
\item{$\bullet$} Voltage source $V_1$ fails with increased output: {\it All resistor currents increase, all resistor voltages increase, current through $V_1$ increases.}
\vskip 5pt
\item{$\bullet$} Resistor $R_1$ fails open: {\it All resistor voltages remain unchanged, current through $R_1$ decreases to zero, currents through $R_2$ and $R_3$ remain unchanged, current through $V_1$ decreases by the amount that used to go through $R_1$.}
\vskip 5pt
\item{$\bullet$} Resistor $R_3$ fails open: {\it All resistor voltages remain unchanged, current through $R_3$ decreases to zero, currents through $R_1$ and $R_2$ remain unchanged, current through $V_1$ decreases by the amount that used to go through $R_3$.}
\vskip 5pt
\item{$\bullet$} Solder bridge (short) across resistor $R_2$: {\it Theoretically, all resistor voltages remain unchanged while a near-infinite amount of current goes through the shorted $R_2$.  Realistically, all resistor voltages will decrease to nearly zero while the currents through $R_2$ and $V_1$ will increase dramatically.}
\medskip

%(END_ANSWER)





%(BEGIN_NOTES)

The purpose of this question is to approach the domain of circuit troubleshooting from a perspective of knowing what the fault is, rather than only knowing what the symptoms are.  Although this is not necessarily a realistic perspective, it helps students build the foundational knowledge necessary to diagnose a faulted circuit from empirical data.  Questions such as this should be followed (eventually) by other questions asking students to identify likely faults based on measurements.

%INDEX% Troubleshooting, predicting effects of fault in voltage source parallel circuit

%(END_NOTES)


