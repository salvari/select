
%(BEGIN_QUESTION)
% Copyright 2005, Tony R. Kuphaldt, released under the Creative Commons Attribution License (v 1.0)
% This means you may do almost anything with this work of mine, so long as you give me proper credit

This circuit generates quasi-sine waves at its output.  It does so by first generating square waves, integrating those square waves (twice) with respect to time, then amplifying the double-integrated signal:

$$\epsfbox{03752x01.eps}$$

Identify the sections of this circuit performing the following functions:

\medskip
\item{$\bullet$} Square wave oscillator:
\item{$\bullet$} First integrator stage:
\item{$\bullet$} Second integrator stage:
\item{$\bullet$} Buffer stage (current amplification):
\item{$\bullet$} Final gain stage (voltage amplification):
\medskip

\underbar{file 03752}
%(END_QUESTION)





%(BEGIN_ANSWER)

\medskip
\item{$\bullet$} Square wave oscillator: $R_1$ through $R_4$, $C_1$ and $C_2$, $Q_1$ and $Q_2$
\item{$\bullet$} First integrator stage: $R_5$ and $C_3$
\item{$\bullet$} Second integrator stage: $R_6$ and $C_4$
\item{$\bullet$} Buffer stage (current amplification): $Q_3$ and $R_7$
\item{$\bullet$} Final gain stage (voltage amplification): $R_8$ and $R_9$, $R_{pot}$, $Q_4$, and $C_7$
\medskip

%(END_ANSWER)





%(BEGIN_NOTES)

The purpose of this question is to have students identify familiar sub-circuits within a larger, practical circuit.  This is a very important skill for troubleshooting, as it allows technicians to divide a malfunctioning system into easier-to-understand sections.

%INDEX% Identification of "function modules" in oscillator/waveshaper/amplifier circuit

%(END_NOTES)


