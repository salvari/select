
%(BEGIN_QUESTION)
% Copyright 2003, Tony R. Kuphaldt, released under the Creative Commons Attribution License (v 1.0)
% This means you may do almost anything with this work of mine, so long as you give me proper credit

Frequency used to be expressed in units of {\it cycles per second}, abbreviated as {\it CPS}.  Now, the standardized unit is {\it Hertz}.  Explain the meaning of the obsolete frequency unit: what, exactly, does it mean for an AC voltage or current to have $x$ number of "cycles per second?"

\underbar{file 00053}
%(END_QUESTION)





%(BEGIN_ANSWER)

Each time an AC voltage or current repeats itself, that interval is called a {\it cycle}.  Frequency, being the rate at which an AC voltage or current repeats itself over time, may be represented in terms of cycles (repetitions) per second.

%(END_ANSWER)





%(BEGIN_NOTES)

Encourage your students to discuss the origins of the new unit (Hertz), and how it actually communicates less information about the thing being measured than the old unit (CPS).

%INDEX% Hertz, defined
%INDEX% Cycle, defined

%(END_NOTES)


