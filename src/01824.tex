
%(BEGIN_QUESTION)
% Copyright 2003, Tony R. Kuphaldt, released under the Creative Commons Attribution License (v 1.0)
% This means you may do almost anything with this work of mine, so long as you give me proper credit

Determine the RMS amplitude of this square-wave signal, as displayed by an oscilloscope with a vertical sensitivity of 0.5 volts per division:

$$\epsfbox{01824x01.eps}$$

\underbar{file 01824}
%(END_QUESTION)





%(BEGIN_ANSWER)

The RMS amplitude of this waveform is 0.5 volt.

%(END_ANSWER)





%(BEGIN_NOTES)

Many electronics students I've talked to seem to think that the RMS value of a waveform is always ${\sqrt{2} \over 2}$, no matter what the waveshape.  Not true, as evidenced by the answer for this question!

Students must properly interpret the oscilloscope's display in order to obtain the correct answer for this question.  The "conversion" to RMS units is really non-existent, but I want students to be able to explain {\it why} it is and not just memorize this fact.

%INDEX% Conversion, peak to RMS (for square wave)
%INDEX% Peak versus RMS, square wave
%INDEX% RMS versus peak, square wave

%(END_NOTES)


