
%(BEGIN_QUESTION)
% Copyright 2003, Tony R. Kuphaldt, released under the Creative Commons Attribution License (v 1.0)
% This means you may do almost anything with this work of mine, so long as you give me proper credit

Complete the following sentences with one of these phrases: "shorter than," "longer than," or "equal to".  Then, explain why the time constant of each circuit type must be so.

\vskip 10pt {\narrower \noindent \baselineskip5pt

Passive integrator circuits should have time constants that are ({\it fill-in-the-blank}) the period of the waveform being integrated.

\par}


\vskip 10pt {\narrower \noindent \baselineskip5pt

Passive differentiator circuits should have time constants that are ({\it fill-in-the-blank}) the period of the waveform being differentiated.

\par} 

\underbar{file 01901}
%(END_QUESTION)





%(BEGIN_ANSWER)

Passive integrators need to have {\it slow} time constants, while passive differentiators need to have {\it fast} time constants, in order to reasonably integrate and differentiate.

%(END_ANSWER)





%(BEGIN_NOTES)

If students don't understand why this is, let them work through an example problem, to see what the output waveform(s) would look like for various periods and time constants.  Remember to stress what an ideal integrator or differentiator is supposed to do!

%INDEX% Passive integrator circuit, time constant of
%INDEX% Passive differentiator circuit, time constant of

%(END_NOTES)


