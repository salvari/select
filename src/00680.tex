
%(BEGIN_QUESTION)
% Copyright 2003, Tony R. Kuphaldt, released under the Creative Commons Attribution License (v 1.0)
% This means you may do almost anything with this work of mine, so long as you give me proper credit

Electrically conductive materials may be rated according to their relative resistance by a quantity we call {\it specific resistance} ($\rho$).  The formula relating resistance to specific resistance looks like this:

$$R = \rho{l \over A}$$

\noindent
Where,

$R =$ Electrical resistance, in ohms

$\rho =$ Specific resistance, in ohm-cmil/ft, or some other combination of units

$l =$ Length of conductor, in feet or cm (depending on units for $\rho$)

$A =$ Cross-sectional area of conductor, in cmil or cm$^{2}$ (depending on units for $\rho$)

\vskip 10pt

Magnetic materials may also be rated according to their relative reluctance by a quantity we call {\it permeability} ($\mu$).  Write the formula relating reluctance to permeability of a magnetic substance, and note whatever differences and similarities you see between it and the specific resistance formula for electrical circuits.

\underbar{file 00680}
%(END_QUESTION)





%(BEGIN_ANSWER)

$$\Re = {l \over {\mu A}}$$

%(END_ANSWER)





%(BEGIN_NOTES)

Ask your students to describe the effects on magnetic reluctance resulting from increases and decreases in all three independent variables ($\mu$, $l$, and $A$).  It is important for them to qualitatively grasp this equation, just as it is important for them to qualitatively understand Ohm's Law and the specific resistance formula.

%INDEX% Permeability, defined

%(END_NOTES)


