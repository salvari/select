
%(BEGIN_QUESTION)
% Copyright 2003, Tony R. Kuphaldt, released under the Creative Commons Attribution License (v 1.0)
% This means you may do almost anything with this work of mine, so long as you give me proper credit

What will happen to the current through R1 and R2 if resistor R3 fails open?

$$\epsfbox{00367x01.eps}$$

\underbar{file 00367}
%(END_QUESTION)





%(BEGIN_ANSWER)

If you think the currents through R1 and R2 would increase, think again!  The current through R1 and the current through R2 both remain the same as they were before R3 failed open.

%(END_ANSWER)





%(BEGIN_NOTES)

A {\it very} common mistake of beginning electronics students is to think that a failed resistor in a parallel circuit supplied by a voltage source causes current through the other resistors to change.  A simple verification using Ohm's Law will prove otherwise, though.

If this mistake is revealed during discussion, ask the class this very important question: "What assumption has to be made in order to conclude that the other two currents will change?"  Invariably, the assumption is that the source outputs a constant {\it current} rather than a constant {\it voltage}.

Another common mistake of beginning electronics students is to think that the position of the failed resistor matters.  Ask your students to determine what will happen to the current through R2 and R3 if R1 were to fail open instead.  If there are any misunderstandings, use the discussion time to correct them and improve everyone's comprehension of the concept.

%INDEX% Troubleshooting, simple circuit
%INDEX% Open component failure, in parallel circuit

%(END_NOTES)


