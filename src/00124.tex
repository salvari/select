
%(BEGIN_QUESTION)
% Copyright 2003, Tony R. Kuphaldt, released under the Creative Commons Attribution License (v 1.0)
% This means you may do almost anything with this work of mine, so long as you give me proper credit

Suppose this battery and light bulb circuit failed to work:

$$\epsfbox{00124x01.eps}$$

Using a voltmeter, a technician measures full battery voltage between the points C and H.  The result of this single measurement indicates which half of the circuit there is a definite problem in.  What would you recommend as the {\it next} voltmeter measurement to take in troubleshooting the circuit, following the same "divide in half" strategy?

\underbar{file 00124}
%(END_QUESTION)





%(BEGIN_ANSWER)

To "divide the circuit in half" again, measure voltage between points D and I.

%(END_ANSWER)





%(BEGIN_NOTES)

Some troubleshooters refer to this strategy as "divide and conquer," because it divides the possibilities of fault location by a factor of 2 with each step.  Make sure your students understand that being able to immediately determine which part of a system is {\it not} faulted is a valuable time-saver.

%INDEX% Troubleshooting, simple circuit

%(END_NOTES)


