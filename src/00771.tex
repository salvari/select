
%(BEGIN_QUESTION)
% Copyright 2003, Tony R. Kuphaldt, released under the Creative Commons Attribution License (v 1.0)
% This means you may do almost anything with this work of mine, so long as you give me proper credit

A common analogy used to describe the different types of power in AC circuits is a mug of beer that also contains foam:

$$\epsfbox{00771x01.eps}$$

Explain this analogy, relating the quantities of beer and foam to the different types of power in an AC circuit, and also why this analogy is often employed to describe the "desirability" of each power type in a circuit.

\underbar{file 00771}
%(END_QUESTION)





%(BEGIN_ANSWER)

The beer itself is "true" power ($P$, measured in Watts).  Good beer, good.  Ideally, we'd like to have a full mug of beer (true power).  Unfortunately, we also have foam in the mug, representing "reactive" power ($Q$, measured in Volt-Amps-Reactive), which does nothing but occupy space in the mug.  Bad foam, bad.  Together, their combined volume constitutes the "apparent" power in the system ($S$, measured in Volt-Amps).

\vskip 10pt

Follow-up question: can you think of any potential safety hazards that low power factor may present in a high-power circuit?  We're talking AC power circuits here, not beer!

%(END_ANSWER)





%(BEGIN_NOTES)

Ask your students to apply this analogy to the following AC circuits: how much beer versus foam exists in each one?

$$\epsfbox{00771x02.eps}$$

%INDEX% Power factor, "beer glass" analogy

%(END_NOTES)


