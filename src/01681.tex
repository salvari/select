
%(BEGIN_QUESTION)
% Copyright 2003, Tony R. Kuphaldt, released under the Creative Commons Attribution License (v 1.0)
% This means you may do almost anything with this work of mine, so long as you give me proper credit

Describe what a {\it load line} is, at it appears superimposed on this graph of characteristic transistor curves:

$$\epsfbox{01681x01.eps}$$

What exactly does the load line represent in the circuit?

\underbar{file 01681}
%(END_QUESTION)





%(BEGIN_ANSWER)

A {\it load line} is a plot showing the amount of collector-emitter voltage available to the transistor ($V_{CE}$) for any given collector current:

$$\epsfbox{01681x02.eps}$$

Follow-up question: why are load lines always straight, and not bent as the transistor characteristic curves are?  What is it that ensures load line plots will always be linear functions?

%(END_ANSWER)





%(BEGIN_NOTES)

It is very important for students to grasp the ontological nature of load lines (i.e. {\it what they are}) if they are to use them frequently in transistor circuit analysis.  This, sadly, is something often not grasped by students when they begin to study transistor circuits, and I place the blame squarely on textbooks (and instructors) who don't spend enough time introducing the concept.

My favorite way of teaching students about load lines is to have them plot load lines for non-transistor circuits, such as voltage dividers (with one of two resistors labeled as the "load," and the other resistor made variable) and diode-resistor circuits.

%INDEX% Load line, defined

%(END_NOTES)


