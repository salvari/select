
%(BEGIN_QUESTION)
% Copyright 2003, Tony R. Kuphaldt, released under the Creative Commons Attribution License (v 1.0)
% This means you may do almost anything with this work of mine, so long as you give me proper credit

Most of the simple amplifiers you will be initially studying tend to lose gain as the frequency of the amplified signal increases.  This loss of gain is sometimes quantified in terms of {\it rolloff}, usually expressed in units of decibels per octave (dB/octave).

What, exactly, is "rolloff?"  What is an "octave," in the context of the units of measurement used to specify rolloff?  If we were to plot the response of a typical amplifier in the form of a Bode plot, what type of filter circuit characteristic (band-pass, band-stop, etc.) would it best resemble?

\underbar{file 01247}
%(END_QUESTION)





%(BEGIN_ANSWER)

Most amplifiers' frequency responses resemble that of low-pass filters.  "Rolloff" is the term used to denote the steepness of the amplifier's Bode plot as it attenuates the amplified signal at ever-increasing frequencies.  

An "octave" denotes a doubling of signal frequency.  This unit applies well to logarithmic-scale Bode plots.

%(END_ANSWER)





%(BEGIN_NOTES)

Have one of your students draw a picture of a Bode plot for a (realistic) low-pass filter: that is, a non-ideal low-pass filter response.  Review with your students what a log-scale plot looks like, and ask them to relate the ratio-units of "decibel" and "octave" to such a scale.

%INDEX% Rolloff, amplifier circuit response

%(END_NOTES)


