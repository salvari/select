
%(BEGIN_QUESTION)
% Copyright 2005, Tony R. Kuphaldt, released under the Creative Commons Attribution License (v 1.0)
% This means you may do almost anything with this work of mine, so long as you give me proper credit

This passive integrator circuit is powered by a square-wave voltage source (oscillating between 0 volts and 5 volts at a frequency of 2 kHz).  Determine the output voltage ($v_{out}$) of the integrator at each instant in time where the square wave transitions (goes from 0 to 5 volts, or from 5 to 0 volts), assuming that the capacitor begins in a fully discharged state at the first transition (from 0 volts to 5 volts):

$$\epsfbox{03669x01.eps}$$

% No blank lines allowed between lines of an \halign structure!
% I use comments (%) instead, so that TeX doesn't choke.

$$\vbox{\offinterlineskip
\halign{\strut
\vrule \quad\hfil # \ \hfil & 
\vrule \quad\hfil # \ \hfil \vrule \cr
\noalign{\hrule}
%
% First row
Transition & $v_{out}$ \cr
%
\noalign{\hrule}
%
% Another row
\#1 (0 $\to$ 5 volts) & 0 volts \cr
%
\noalign{\hrule}
%
% Another row
\#2 (5 $\to$ 0 volts) &  \cr
%
\noalign{\hrule}
%
% Another row
\#3 (0 $\to$ 5 volts) &  \cr
%
\noalign{\hrule}
%
% Another row
\#4 (5 $\to$ 0 volts) &  \cr
%
\noalign{\hrule}
%
% Another row
\#5 (0 $\to$ 5 volts) &  \cr
%
\noalign{\hrule}
%
% Another row
\#6 (5 $\to$ 0 volts) &  \cr
%
\noalign{\hrule}
%
% Another row
\#7 (0 $\to$ 5 volts) &  \cr
%
\noalign{\hrule}
%
% Another row
\#8 (5 $\to$ 0 volts) &  \cr
%
\noalign{\hrule}
} % End of \halign 
}$$ % End of \vbox

\underbar{file 03669}
%(END_QUESTION)





%(BEGIN_ANSWER)

% No blank lines allowed between lines of an \halign structure!
% I use comments (%) instead, so that TeX doesn't choke.

$$\vbox{\offinterlineskip
\halign{\strut
\vrule \quad\hfil # \ \hfil & 
\vrule \quad\hfil # \ \hfil \vrule \cr
\noalign{\hrule}
%
% First row
Transition & $v_{out}$ \cr
%
\noalign{\hrule}
%
% Another row
\#1 (0 $\to$ 5 volts) & 0 volts \cr
%
\noalign{\hrule}
%
% Another row
\#2 (5 $\to$ 0 volts) & 3.395 volts \cr
%
\noalign{\hrule}
%
% Another row
\#3 (0 $\to$ 5 volts) & 1.090 volts \cr
%
\noalign{\hrule}
%
% Another row
\#4 (5 $\to$ 0 volts) & 3.745 volts \cr
%
\noalign{\hrule}
%
% Another row
\#5 (0 $\to$ 5 volts) & 1.202 volts \cr
%
\noalign{\hrule}
%
% Another row
\#6 (5 $\to$ 0 volts) & 3.781 volts \cr
%
\noalign{\hrule}
%
% Another row
\#7 (0 $\to$ 5 volts) & 1.214 volts \cr
%
\noalign{\hrule}
%
% Another row
\#8 (5 $\to$ 0 volts) & 3.785 volts \cr
%
\noalign{\hrule}
} % End of \halign 
}$$ % End of \vbox

\vskip 10pt

Challenge question: what are the final (ultimate) values for the integrator output's sawtooth-wave peak voltages?

%(END_ANSWER)





%(BEGIN_NOTES)

Be sure to have your students share their problem-solving techniques (how they determined which equation to use, etc.) in class.  See how many of them notice that the exponential portion of the equation ($e^{t \over \tau}$) is the same for each calculation, and if they find an easy way to manage the calculations by storing charge/discharge percentages in their calculator memories!

%INDEX% Time constant calculation, passive RC integrator circuit

%(END_NOTES)


