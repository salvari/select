
%(BEGIN_QUESTION)
% Copyright 2003, Tony R. Kuphaldt, released under the Creative Commons Attribution License (v 1.0)
% This means you may do almost anything with this work of mine, so long as you give me proper credit

An AC bridge circuit commonly used to make precision measurements of inductors is the {\it Maxwell-Wien} bridge.  It uses a combination of standard resistors and capacitors to "balance out" the inductor of unknown value in the opposite arm of the bridge:

$$\epsfbox{00862x01.eps}$$

Suppose this bridge circuit balances when $C_s$ is adjusted to 120 nF and $R_s$ is adjusted to 14.25 k$\Omega$.  If the source frequency is 400 Hz, and the two fixed-value resistors are 1 k$\Omega$ each, calculate the inductance ($L_x$) and resistance ($R_x$) of the inductor being tested.

\underbar{file 00862}
%(END_QUESTION)





%(BEGIN_ANSWER)

$L_x =$ 120 mH

$R_x =$ 70.175 $\Omega$

%(END_ANSWER)





%(BEGIN_NOTES)

There is actually a way to solve for the values of $L_x$ and $R_x$ in a Maxwell-Wien bridge circuit without using complex numbers at all.  If one or more of your students find out how to do so through their research, don't consider it "cheating."  Rather, applaud their research, because they found a quicker path to a solution.

This, of course, doesn't mean you don't ask them to work through the problem using complex numbers!  The benefit of students researching other means of solution simply provides more alternative solution strategies to the same problem, which is a very good thing.

%INDEX% AC bridge circuit
%INDEX% Bridge circuit, AC
%INDEX% Maxwell-Wien AC bridge

%(END_NOTES)


