
%(BEGIN_QUESTION)
% Copyright 2005, Tony R. Kuphaldt, released under the Creative Commons Attribution License (v 1.0)
% This means you may do almost anything with this work of mine, so long as you give me proper credit

$$\epsfbox{02589x01.eps}$$

\underbar{file 02589}
\vfil \eject
%(END_QUESTION)





%(BEGIN_ANSWER)

Use circuit simulation software to verify your predicted and measured parameter values.

%(END_ANSWER)





%(BEGIN_NOTES)

I also recommend having students use an oscilloscope to measure AC voltage in a circuit such as this, because some digital multimeters have difficulty accurately measuring AC voltage much beyond line frequency range.  I find it particularly helpful to set the oscilloscope to the "X-Y" mode so that it draws a thin line on the screen rather than sweeps across the screen to show an actual waveform.  This makes it easier to measure peak-to-peak voltage.

Values that have proven to work well for this exercise are given here, although of course many other values are possible:

\medskip
\goodbreak
\item{$\bullet$} +V = +12 volts
\item{$\bullet$} -V = -12 volts
\item{$\bullet$} $R_1$ = 10 k$\Omega$
\item{$\bullet$} $R_2$ = 10 k$\Omega$
\item{$\bullet$} $R_3$ = 5 k$\Omega$ (actually, two 10 k$\Omega$ resistors in parallel)
\item{$\bullet$} $R_4$ = 20 k$\Omega$ (actually, two 10 k$\Omega$ resistors in series)
\item{$\bullet$} $C_1$ = 0.001 $\mu$F
\item{$\bullet$} $C_2$ = 0.001 $\mu$F
\item{$\bullet$} $C_3$ = 0.002 $\mu$F (actually, two 0.001 $\mu$F capacitors in parallel)
\item{$\bullet$} $U_1$ = one-half of LM1458 dual operational amplifier
\medskip

This combination of components gave a predicted notch frequency of 15.92 kHz, with an actual cutoff frequency (not factoring in component tolerances) of 15.87 kHz.

%INDEX% Assessment, performance-based (Twin-T active bandstop filter)

%(END_NOTES)


