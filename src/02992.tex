
%(BEGIN_QUESTION)
% Copyright 2005, Tony R. Kuphaldt, released under the Creative Commons Attribution License (v 1.0)
% This means you may do almost anything with this work of mine, so long as you give me proper credit

If your event sensing switch is mechanical in nature, you will likely have to deal with {\it contact bounce}.  Explain what contact "bounce" is and how you can mitigate the problem.

\underbar{file 02992}
%(END_QUESTION)





%(BEGIN_ANSWER)

Contact "bounce" is the physical bouncing and interference of metal switch contacts as they come to a closed state, causing spurious pulses which will be interpreted by the counter as multiple events.  Perhaps the simplest way to mitigate the effects of contact bounce is to use a passive low-pass filter (RC integrator) circuit followed by a Schmitt trigger gate.  Using a serial-in/serial-out shift register is a more sophisticated approach, but requires more components (including a clock signal generator).

%(END_ANSWER)





%(BEGIN_NOTES)

Contact bounce can be a big problem in counter circuits getting their input from mechanical switches.  Learning how to detect and eliminate this problem is a very practical exercise for electronics technicians.

%(END_NOTES)


