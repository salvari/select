
%(BEGIN_QUESTION)
% Copyright 2003, Tony R. Kuphaldt, released under the Creative Commons Attribution License (v 1.0)
% This means you may do almost anything with this work of mine, so long as you give me proper credit

Interpret this AC motor control circuit diagram, explaining the meaning of each symbol:

$$\epsfbox{00834x01.eps}$$

Also, explain the operation of this motor control circuit.  What happens when someone actuates the "Run" switch?  What happens when they let go of the "Run" switch?

\underbar{file 00834}
%(END_QUESTION)





%(BEGIN_ANSWER)

The "Run" switch is a normally-open pushbutton.  Relay coil "M1" is energized by this switch, and actuates three normally-open contacts (also labeled "M1") to send three-phase power to the motor.  Note that the details of the power supply are not shown in these diagrams.  This is a common omission, done for the sake of simplicity.

%(END_ANSWER)





%(BEGIN_NOTES)

Discuss with your students the sources of electrical power for both circuits here: the relay control circuit and the motor itself.  Challenge your students to explore this concept by asking them the following questions: 

\medskip
\item{$\bullet$} Are the two sources necessarily the same?
\item{$\bullet$} How does the convention of linking relay coils with contacts by name (rather than by dashed lines and proximity) in ladder diagrams benefit multiple-source circuits such as this one?
\item{$\bullet$} Do these circuits even have to be drawn on the same page?
\medskip

%INDEX% Control circuit, AC motor
%INDEX% Ladder logic diagram

%(END_NOTES)


