
%(BEGIN_QUESTION)
% Copyright 2004, Tony R. Kuphaldt, released under the Creative Commons Attribution License (v 1.0)
% This means you may do almost anything with this work of mine, so long as you give me proper credit

The $Q$, or {\it quality factor}, of an inductor circuit is defined by the following equation, where $X_s$ is the series inductive reactance and $R_s$ is the series resistance:

$$Q = {X_s \over R_{s}}$$

We also know that we may convert between series and parallel equivalent AC networks with the following conversion equations:

$$R_{s} R_{p} = Z^2 \hbox{\hskip 40pt} X_{s} X_{p} = Z^2$$

$$\epsfbox{02096x01.eps}$$

Series and parallel LR networks, if truly equivalent, should share the same $Q$ factor as well as sharing the same impedance.  Develop an equation that solves for the $Q$ factor of a {\it parallel} LR circuit.

\underbar{file 02096}
%(END_QUESTION)





%(BEGIN_ANSWER)

$$Q = {R_p \over X_p}$$

\vskip 10pt

Follow-up question: what condition gives the greatest value for $Q$, a low parallel resistance or a high parallel resistance?  Contrast this against the effects of low versus high resistance in a series LR circuit, and explain both scenarios.

%(END_ANSWER)





%(BEGIN_NOTES)

This is primarily an exercise in algebraic substitution, but it also challenges students to think deeply about the nature of $Q$ and what it means, especially in the follow-up question.

%INDEX% Quality factor (Q), series versus parallel LR circuit

%(END_NOTES)


