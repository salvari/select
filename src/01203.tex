
%(BEGIN_QUESTION)
% Copyright 2003, Tony R. Kuphaldt, released under the Creative Commons Attribution License (v 1.0)
% This means you may do almost anything with this work of mine, so long as you give me proper credit

Explain why binary is a natural numeration system for expressing numbers in electronic circuits.  Why not decimal or some other base of numeration?

How do you suspect binary numbers may be {\it stored} in electronic systems, for future retrieval?  What advantages are there to the use of binary numeration in storage systems?

\underbar{file 01203}
%(END_QUESTION)





%(BEGIN_ANSWER)

The two states of an on/off circuit are equivalent to 0 and 1 ciphers in binary.  Any medium in which on/off states may be physically represented is applicable to storing binary numbers.  Optical disks (CD-ROM, DVD) are an excellent example of this, with laser-burned "pits" representing binary bits.

%(END_ANSWER)





%(BEGIN_NOTES)

Ask your students to think of different media or physical quantities which may represent binary information, especially those related to electric/electronic circuits.

%(END_NOTES)


