
%(BEGIN_QUESTION)
% Copyright 2005, Tony R. Kuphaldt, released under the Creative Commons Attribution License (v 1.0)
% This means you may do almost anything with this work of mine, so long as you give me proper credit

A Scottish physicist named James Clerk Maxwell made an astonishing theoretical prediction in the nineteenth century, which he expressed with these two equations:

$$\oint {\bf E} \cdot d{\bf l} = - {d \Phi_B \over dt}$$

$$\oint {\bf B} \cdot d{\bf l} = \mu_0 I + \mu_0 \epsilon_0 {d \Phi_E \over dt}$$

The first equation states that an electric field $({\bf E})$ will be produced in open space by a changing magnetic flux $\left({d \Phi_B \over dt}\right)$.  The second equation states than a magnetic field $({\bf B})$ will be produced in open space either by an electric current ($I$) or by a changing electric flux $\left({d \Phi_E \over dt}\right)$.  Given this complementary relationship, Maxwell reasoned, it was possible for a changing electric field to create a changing magnetic field which would then create another changing electric field, and so on.  This cause-and-effect cycle could continue, ad infinitum, with fast-changing electric and magnetic fields radiating off into open space without needing wires to carry or guide them.  In other words, the complementary fields would be self-sustaining as they traveled.

\vskip 10pt

Explain the significance of Maxwell's prediction, especially as it relates to electronics.

\underbar{file 03462}
%(END_QUESTION)





%(BEGIN_ANSWER)

What James Clerk Maxwell predicted was the existence of {\it electromagnetic waves}, the lowest-frequency type we commonly refer to as {\it radio waves}.

\vskip 10pt

Follow-up question: name the scientist who first experimentally confirmed Maxwell's prediction of electromagnetic waves.

%(END_ANSWER)





%(BEGIN_NOTES)

Not only does this question relate the concept of radio waves to concepts your students should already be familiar with (electric and magnetic fields), but it also introduces a piece of amazing scientific history.  That radio waves were first predicted mathematically rather than discovered accidently by experiment is both astonishing and enlightening.

You may find that one or more of your brighter students notice Maxwell's prediction relates a change in one type of field to a static magnitude of the other (i.e. $E \propto {d\Phi_B \over dt}$ and $B \propto {d\Phi_E \over dt}$), and that this makes it difficult to see how one changing field could create another {\it changing} field.  If anyone asks this question, point out to them that there is a set of similar mathematical functions related to one another by derivatives, and they are:

$$\sin t = - {d \over dt} \cos t  \hbox{\hskip 100pt}  \cos t = {d \over dt} \sin t$$

\vskip 10pt

Does anything look familiar (omitting the $\mu_0 I$ term from the second equation)?

\vskip 10pt

$$\oint {\bf E} \cdot d{\bf l} = - {d \Phi_B \over dt}  \hbox{\hskip 100pt}  \oint {\bf B} \cdot d{\bf l} =  \mu_0 \epsilon_0 {d \Phi_E \over dt}$$

Since we know electric flux is related to electric field by geometry ($\Phi_E = \int {\bf E} \cdot d{\bf A}$) and magnetic flux is related to magnetic field by geometry as well ($\Phi_B = \int {\bf B} \cdot d{\bf A}$), we may write the following proportionalities:

$$\Phi_E \propto - {d \Phi_B \over dt}  \hbox{\hskip 100pt}  \Phi_B \propto {d \Phi_E \over dt}$$

{\it Now} do things look similar to the sine/cosine derivative relationship?  Thus, if the electric flux $\Phi_E$ oscillates as a sine wave, the magnetic flux $\Phi_B$ will oscillate as a cosine wave, and so on.

%INDEX% Electromagnetic waves, predicted by James Clerk Maxwell
%INDEX% Maxwell's equations, relationship between electric and magnetic fields

%(END_NOTES)


