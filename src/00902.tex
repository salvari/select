
%(BEGIN_QUESTION)
% Copyright 2003, Tony R. Kuphaldt, released under the Creative Commons Attribution License (v 1.0)
% This means you may do almost anything with this work of mine, so long as you give me proper credit

In perfectly pure ("intrinsic") semiconductors, the only way charge carriers can exist is for valence electrons to "leap" into the conduction band with the application of sufficient energy, leaving a {\it hole}, or vacancy, behind in the valence band:

$$\epsfbox{00902x01.eps}$$

With sufficient thermal energy, these electron-hole pairs will form spontaneously.  At room temperature, however, this activity is slight.   

We may greatly enhance charge carrier formation by adding specific impurities to the semiconducting material.  The energy states of atoms having different electron configurations do not precisely "blend" with the electron bands of the parent semiconductor crystal, causing additional energy levels to form.

Some types of impurities will cause extra {\it donor} electrons to lurk just beneath the main conduction band of the crystal.  These types of impurities are called {\it pentavalent}, because they have 5 valence electrons per atom rather than 4 as the parent substance typically possesses:

$$\epsfbox{00902x02.eps}$$

Other types of impurities will cause vacant electron levels ({\it acceptor} "holes") to form just above the main valence band of the crystal.  These types of impurities are called {\it trivalent}, because they have 3 valence electrons per atom instead of 4:

$$\epsfbox{00902x03.eps}$$

Compare the ease of forming free (conduction-band) electrons in a semiconductor material having lots of "donor" electrons, against that of an intrinsic (pure) semiconductor material.  Which type of material will be more electrically conductive?

Likewise, compare the ease of forming valence-band holes in a semiconductor material having lots of "acceptor" holes, against that of an intrinsic (pure) semiconductor material.  Which type of material will be more electrically conductive?

\underbar{file 00902}
%(END_QUESTION)





%(BEGIN_ANSWER)

Under the influence of thermal energy from ambient sources, pentavalent "donor" atoms contribute to free electrons in the conduction band:

$$\epsfbox{00902x04.eps}$$

Likewise, trivalent "acceptor" atoms contribute to holes in the valence band:

$$\epsfbox{00902x05.eps}$$

In either case, the addition of impurities to an otherwise pure semiconductor material increases the number of available charge carriers.

%(END_ANSWER)





%(BEGIN_NOTES)

The most important concept for students to grasp here is that the addition of impurities increases the number of available {\it charge carriers} in a semiconducting substance.  What was essentially an insulator in its pure state may be made conductive to varying degrees by adding impurities.

%INDEX% Charge carriers
%INDEX% Electrons versus holes
%INDEX% Holes versus electrons
%INDEX% Electron bands
%INDEX% Diagram, electron band
%INDEX% Energy diagram, semiconductor with trivalent and pentavalent dopants

%(END_NOTES)


