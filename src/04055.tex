
%(BEGIN_QUESTION)
% Copyright 2006, Tony R. Kuphaldt, released under the Creative Commons Attribution License (v 1.0)
% This means you may do almost anything with this work of mine, so long as you give me proper credit

A common mathematical function known as {\it factorial} is represented by an exclamation point following a positive integer number.  The following are all examples of factorials:

$$1! = 1 \hbox{\hskip 30pt} 2! = 2 \hbox{\hskip 30pt} 3! = 6 \hbox{\hskip 30pt} 4! = 24 \hbox{\hskip 30pt} 5! = 120 \hbox{\hskip 30pt} 6! = 720$$

Explain in your own words what this "factorial" function represents.  What procedure (algorithm) would we use to arithmetically calculate the value of any factorial given to us?

\underbar{file 04055}
%(END_QUESTION)





%(BEGIN_ANSWER)

I'll let you research this on your own!  The algorithm isn't that difficult to figure out just by examining the sequence of factorials given in the question.

\vskip 10pt

Follow-up question: calculate $0!$

%(END_ANSWER)





%(BEGIN_NOTES)

In addition to being very useful in probability calculations, factorials are frequently seen in important mathematical {\it series}, including those used to calculate $e$, $\sin$, and $\cos$.

%INDEX% Factorial (mathematics), defined

%(END_NOTES)


