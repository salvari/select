
%(BEGIN_QUESTION)
% Copyright 2003, Tony R. Kuphaldt, released under the Creative Commons Attribution License (v 1.0)
% This means you may do almost anything with this work of mine, so long as you give me proper credit

$$\epsfbox{01656x01.eps}$$

\underbar{file 01656}
\vfil \eject
%(END_QUESTION)





%(BEGIN_ANSWER)

Use circuit simulation software to verify your predicted and measured parameter values.

%(END_ANSWER)





%(BEGIN_NOTES)

Use a sine-wave function generator for the AC voltage source.  I recommend against using line-power AC because of strong harmonic frequencies which may be present (due to nonlinear loads operating on the same power circuit).  Specify standard capacitor values.

If students are to use a multimeter to make their current and voltage measurements, be sure it is capable of accurate measurement at the circuit frequency!  Inexpensive digital multimeters often experience difficulty measuring AC voltage and current toward the high end of the audio-frequency range.

Students often become confused when learning about series and parallel capacitive reactances, because many think capacitors are inherently "opposite" of inductors and resistors in all respects.  In other words, they first learn that {\it capacitance} (measured in Farads) adds in parallel and diminishes in series, and that this is just the opposite of resistance (Ohms) and inductance (Henrys), and they mistakenly carry this thinking on through capacitive {\it reactance} (Ohms) as well.  The real lesson of this exercise is that reactances add in series, regardless of their nature.

%INDEX% Assessment, performance-based (Series AC capacitive reactances)

%(END_NOTES)


