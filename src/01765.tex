
%(BEGIN_QUESTION)
% Copyright 2003, Tony R. Kuphaldt, released under the Creative Commons Attribution License (v 1.0)
% This means you may do almost anything with this work of mine, so long as you give me proper credit

Calculate the voltage magnitude and polarity between points {\bf A} and {\bf D} in this circuit, assuming a power supply output voltage of 10.5 volts:

$$\epsfbox{01765x01.eps}$$

Also, calculate the total current output by the power supply as it energizes this resistor network.

\underbar{file 01765}
%(END_QUESTION)





%(BEGIN_ANSWER)

$V_{AD} = 7.31 \hbox{ volts}$, {\bf A} positive and {\bf D} negative.  The total power supply current is 4.36 mA.

\vskip 10pt

Follow-up question: explain why the voltage across the 4.7 k$\Omega$ resistor would go to zero if the 1.5 k$\Omega$ resistor were to fail open.

%(END_ANSWER)





%(BEGIN_NOTES)

Though some students might not realize it at first, there is no series-parallel analysis necessary to obtain the voltage drop $V_{AD}$.
 
\vskip 10pt

Students often have difficulty formulating a method of solution: determining what steps to take to get from the given conditions to a final answer.  While it is helpful at first for you (the instructor) to show them, it is bad for you to show them too often, lest they stop thinking for themselves and merely follow your lead.  A teaching technique I have found very helpful is to have students come up to the board (alone or in teams) in front of class to write their problem-solving strategies for all the others to see.  They don't have to actually do the math, but rather outline the steps they would take, in the order they would take them.

By having students \underbar{outline their problem-solving strategies}, everyone gets an opportunity to see multiple methods of solution, and you (the instructor) get to see how (and if!) your students are thinking.  An especially good point to emphasize in these "open thinking" activities is how to check your work to see if any mistakes were made.

\vskip 10pt

As a follow-up to the follow-up question, ask your students what {\it other} resistor in this circuit would completely lose voltage given an open failure of the 1.5 k$\Omega$ resistor.

%INDEX% Series-parallel circuit, voltage and current calculations in

%(END_NOTES)


