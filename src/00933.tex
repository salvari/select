
%(BEGIN_QUESTION)
% Copyright 2003, Tony R. Kuphaldt, released under the Creative Commons Attribution License (v 1.0)
% This means you may do almost anything with this work of mine, so long as you give me proper credit

Operational amplifier circuits employing negative feedback are sometimes referred to as "electronic levers," because their voltage gains may be understood through the mechanical analogy of a lever.  Explain this analogy in your own words, identifying how the lengths and fulcrum location of a lever relate to the component values of an op-amp circuit:

$$\epsfbox{00933x01.eps}$$

\underbar{file 00933}
%(END_QUESTION)





%(BEGIN_ANSWER)

The analogy of a lever works well to explain how the output voltage of an op-amp circuit relates to the input voltage, in terms of both magnitude and polarity.  Resistor values correspond to {\it moment arm} lengths, while direction of lever motion (up versus down) corresponds to polarity.  The position of the fulcrum represents the location of ground potential in the feedback network.

%(END_ANSWER)





%(BEGIN_NOTES)

I found this analogy in one of the best books I've ever read on op-amp circuits: John I. Smith's \underbar{Modern Operational Circuit Design}.  Unfortunately, this book is out of print, but if you can possibly obtain a copy for your library, I highly recommend it!

%INDEX% Opamp, inverting amplifier circuit ("lever" analogy)
%INDEX% Opamp, noninverting amplifier circuit ("lever" analogy)

%(END_NOTES)


