
%(BEGIN_QUESTION)
% Copyright 2003, Tony R. Kuphaldt, released under the Creative Commons Attribution License (v 1.0)
% This means you may do almost anything with this work of mine, so long as you give me proper credit

The ancient Mayans used a {\it vigesimal}, or base-twenty, numeration system in their mathematics.  Each "digit" was a actually a composite of dots and/or lines, as such:

$$\epsfbox{01197x01.eps}$$

To represent numbers larger than twenty, the Mayans combined multiple "digits" the same way we do to represent numbers larger than ten.  For example:

$$\epsfbox{01197x02.eps}$$

Based on the examples shown above, determine the place-weighting of each "digit" in the vigesimal numeration system.  For example, in our denary, or base-ten, system, we have a one's place, a ten's place, a hundred's place, and so on, each successive "place" having ten times the "weight" of the place before it.  What are the values of the respective "places" in the Mayan system?

Also, determine the values of these Mayan numbers:

$$\epsfbox{01197x03.eps}$$

\underbar{file 01197}
%(END_QUESTION)





%(BEGIN_ANSWER)

Place weights: one's, twenty's, four hundred's, eight thousand's, . . .

\vskip 10pt

$$\epsfbox{01197x04.eps}$$

%(END_ANSWER)





%(BEGIN_NOTES)

Although no one counts in "Mayan" anymore, this question is still relevant because it gets students thinking outside their accustomed numeration system.  Also, it has some cultural value, showing them that not everyone in the world counts the same way they do!

%INDEX% Base-20 numeration system
%INDEX% Place weighting, base-20 numeration system

%(END_NOTES)


