
%(BEGIN_QUESTION)
% Copyright 2005, Tony R. Kuphaldt, released under the Creative Commons Attribution License (v 1.0)
% This means you may do almost anything with this work of mine, so long as you give me proper credit

A digital computer network uses a two-conductor cable to send both power (DC) and data (AC pulses) from one computer to another, inductors and capacitors being used to filter the voltages to and from their proper destinations:

$$\epsfbox{03604x01.eps}$$

The system works fine for a while, but then one day the receiving computer stops receiving DC power and completely shuts down.  Identify the following:

\vskip 10pt

\medskip
\item{$\bullet$} \underbar{One} failed component in the circuit that could possibly account for the problem, and the type of fault (open or short) you suspect that component would have.
\vskip 40pt
\item{$\bullet$} Show where you would connect a voltmeter in the circuit to verify a fault in that one suspect component, and the voltage reading (AC, DC, or both) you would expect to get if indeed that one component had failed.
\medskip

\underbar{file 03604}
%(END_QUESTION)





%(BEGIN_ANSWER)

\goodbreak
\noindent
{\bf Possible component faults, and their respective validating tests}

\medskip
\item{$\bullet$} DC source $V_2$ failed (no signal); measure 0 DC voltage directly across its terminals.
\item{$\bullet$} Break in cable; measure 0 DC voltage and 0 AC voltage across cable at receiving end.
\item{$\bullet$} Inductor $L_1$ failed open; measure full DC voltage directly across its terminals.
\item{$\bullet$} Inductor $L_2$ failed open; measure full DC voltage directly across its terminals.
\item{$\bullet$} Inductor $L_3$ failed open; measure full DC voltage directly across its terminals.
\item{$\bullet$} Inductor $L_4$ failed open; measure full DC voltage directly across its terminals.
\medskip

%(END_ANSWER)





%(BEGIN_NOTES)

{\bf This question is intended for exams only and not worksheets!}.

%(END_NOTES)


