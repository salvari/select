
%(BEGIN_QUESTION)
% Copyright 2003, Tony R. Kuphaldt, released under the Creative Commons Attribution License (v 1.0)
% This means you may do almost anything with this work of mine, so long as you give me proper credit

What determines the color of an LED?

\underbar{file 01028}
%(END_QUESTION)





%(BEGIN_ANSWER)

The type of semiconductor materials used to make the PN junction determine the color of light emitted.

\vskip 10pt

Challenge question: describe the relationship between LED color and typical forward voltage, in terms of photon frequency, energy, and semiconductor band gap.

%(END_ANSWER)





%(BEGIN_NOTES)

Ask your students to identify some common LED materials and colors, and of course cite their sources as they do.  The challenge question may be readily answered through experimentation with different LED colors, although a physics-based explanation will take some additional research.  This kind of experiment is very easy to conduct in class, together.

If time permits, you might wish to mention Albert Einstein's contribution to this aspect of physics: his formulation for the energy carried by a photon (a {\it quantum}) of light:

$$E = hf$$

\noindent
Where,

$E$ = Energy carried by photon, in Joules

$h$ = Planck's constant, 6.63 $\times$ 10$^{-34}$ J$\cdot$s

$f$ = Frequency of light, in Hertz ($1 \over s$)

Typical frequencies for visible light colors range from $4 \times 10^{14}$ Hz for red, to $7.5 \times 10^{14}$ Hz for violet.

%INDEX% LED color

%(END_NOTES)


