
%(BEGIN_QUESTION)
% Copyright 2003, Tony R. Kuphaldt, released under the Creative Commons Attribution License (v 1.0)
% This means you may do almost anything with this work of mine, so long as you give me proper credit

It is relatively easy to design and build an electronic circuit to make square-wave voltage signals.  More difficult to engineer is a circuit that directly generates triangle-wave signals.  A common approach in electronic design when triangle waves are needed for an application is to connect a {\it passive integrator} circuit to the output of a square-wave oscillator, like this:

$$\epsfbox{01896x01.eps}$$

Anyone familiar with RC circuits will realize, however, that a passive integrator will not output a true triangle wave, but rather it will output a waveshape with "rounded" leading and trailing edges:

$$\epsfbox{01896x02.eps}$$

What can be done with the values of $R$ and $C$ to best approximate a true triangle wave?  What variable must be compromised to achieve the most linear edges on the integrator output waveform?  Explain why this is so.

\underbar{file 01896}
%(END_QUESTION)





%(BEGIN_ANSWER)

Maximum values of $R$ and $C$ will best approximate a true triangle wave.  The consequences of choosing extremely large values for $R$ and/or $C$ are not difficult to determine -- I leave that for you to explain!

%(END_ANSWER)





%(BEGIN_NOTES)

This question asks students to recognize conflicting design needs, and to balance one need against another.  Very practical skills here, as real-life applications almost always demand some form of practical compromise in the design stage.

If students cannot figure out what must be sacrificed to achieve waveshape linearity, tell them to build such a circuit and see for themselves!

%INDEX% Passive integrator circuit, choosing R and C values for

%(END_NOTES)


