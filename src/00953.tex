
%(BEGIN_QUESTION)
% Copyright 2004, Tony R. Kuphaldt, released under the Creative Commons Attribution License (v 1.0)
% This means you may do almost anything with this work of mine, so long as you give me proper credit

{\it Load lines} are useful tools for analyzing transistor amplifier circuits, but they may be hard to understand at first.  To help you understand what "load lines" are useful for and how they are determined, I will apply one to this simple two-resistor circuit:

$$\epsfbox{00953x01.eps}$$

We will have to plot a load line for this simple two-resistor circuit along with the "characteristic curve" for resistor $R_1$ in order to see the benefit of a load line.  Load lines really only have meaning when superimposed with other plots.  First, the characteristic curve for $R_1$, defined as the voltage/current relationship between terminals {\bf A} and {\bf B}:

$$\epsfbox{00953x02.eps}$$

Next, I will plot the load line as defined by the 1.5 k$\Omega$ load resistor.  This "load line" expresses the voltage available between the same two terminals ($V_{AB}$) as a function of the load current, to account for voltage dropped across the load:

$$\epsfbox{00953x03.eps}$$

At what value of current ($I_{R1}$) do the two lines intersect?  Explain what is significant about this value of current.

\underbar{file 00953}
%(END_QUESTION)





%(BEGIN_ANSWER)

$I_R$ = 8 mA is the same value of current you would calculate if you had analyzed this circuit as a simple series resistor network.

\vskip 10pt

Follow-up question: you might be wondering, "what is the point of plotting a 'characteristic curve' and a 'load line' in such a simple circuit, if all we had to do to solve for current was add the two resistances and divide that total resistance value into the total voltage?"  Well, to be honest, there is no point in analyzing such a simple circuit in this manner, except to illustrate {\it how} load lines work.  My follow-up question to you is this: where would plotting a load line actually be helpful in analyzing circuit behavior?  Can you think of any modifications to this two-resistor circuit that would require load line analysis in order to solve for current?

%(END_ANSWER)





%(BEGIN_NOTES)

While this approach to circuit analysis may seem silly -- using load lines to calculate the current in a two-resistor circuit -- it demonstrates the principle of load lines in a context that should be obvious to students at this point in their study.  Discuss with your students how the two lines are obtained (one for resistor $R_1$ and the other plotting the voltage available to $R_1$ based on the total source voltage and the load resistor's value).  

Also, discuss the significance of the two line intersecting.  Mathematically, what does the intersection of two graphs mean?  What do the coordinate values of the intersection point represent in a system of simultaneous functions?  How does this principle relate to an electronic circuit?

%INDEX% Load line, concept illustrated with two-resistor circuit

%(END_NOTES)


