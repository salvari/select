
%(BEGIN_QUESTION)
% Copyright 2003, Tony R. Kuphaldt, released under the Creative Commons Attribution License (v 1.0)
% This means you may do almost anything with this work of mine, so long as you give me proper credit

An important function in computer circuitry is {\it serial-to-parallel data conversion}, where a stream of serial data is "read" one bit at a time, then all bits output at once in parallel form.  A shift register circuit is ideal for this application.  Shown here is an eight-bit shift register circuit:

$$\epsfbox{01468x01.eps}$$

Draw any necessary wires and labels showing where serial data would enter the circuit, and where parallel data would exit.

\underbar{file 01468}
%(END_QUESTION)





%(BEGIN_ANSWER)

$$\epsfbox{01468x02.eps}$$

\vskip 10pt

Follow-up question: if we were to actually use this circuit for serial-to-parallel data conversion, we would have to be careful how fast we clocked the shift register.  Explain why.

%(END_ANSWER)





%(BEGIN_NOTES)

The subject of serial-to-parallel data conversion is much deeper than what is suggested by this disarmingly simple circuit.  Talk with your students about the need for clock synchronization (even in "asynchronous" serial data transmission).

%INDEX% Shift register, serial in / parallel out

%(END_NOTES)


