
%(BEGIN_QUESTION)
% Copyright 2003, Tony R. Kuphaldt, released under the Creative Commons Attribution License (v 1.0)
% This means you may do almost anything with this work of mine, so long as you give me proper credit

What is absolutely the best way to protect yourself from the dangers of arc blast while performing maintenance work on high energy electrical circuits?

\underbar{file 00565}
%(END_QUESTION)





%(BEGIN_ANSWER)

Ensure that the system is de-energized (lock-out / tag-out) prior to performing the work.

%(END_ANSWER)





%(BEGIN_NOTES)

Students have asked me on many occasions if maintenance work on high-voltage systems is dangerous.  My answer to this question is that "if you follow proper procedure, it's all 0 volts by the time you touch it!"  So, no, working on a de-energized 13.8 kV electrical system is no more dangerous than working on a de-energized 480 volt system or even a de-energized 120 volt system.  All these voltages are lethal under the right conditions.  The only real difference is psychological: high-voltage systems are {\it scarier} because the danger is more evident.

%INDEX% Arc blast
%INDEX% Lock-out / tag-out

%(END_NOTES)


