
%(BEGIN_QUESTION)
% Copyright 2003, Tony R. Kuphaldt, released under the Creative Commons Attribution License (v 1.0)
% This means you may do almost anything with this work of mine, so long as you give me proper credit

There are two basic Ohm's Law equations: one relating voltage, current, and resistance; and the other relating voltage, current, and power (the latter equation is sometimes known as {\it Joule's Law} rather than Ohm's Law):

$$E = I R$$

$$P = I E$$

In electronics textbooks and reference books, you will find twelve different variations of these two equations, one solving for each variable in terms of a unique pair of two other variables.  However, you need not memorize all twelve equations if you have the ability to algebraically manipulate the two simple equations shown above.

Demonstrate how algebra is used to derive the ten "other" forms of the two Ohm's Law / Joule's Law equations shown here.

\underbar{file 00088}
%(END_QUESTION)





%(BEGIN_ANSWER)

I won't show you how to do the algebraic manipulations, but I will show you the ten other equations.  First, those equations that may be derived strictly from $E = I R$:

$$I = {E \over R}$$

$$R = {E \over I}$$

Next, those equations that may be derived strictly from $P = I E$:

$$I = {P \over E}$$

$$E = {P \over I}$$

Next, those equations that may be derived by using algebraic {\it substitution} between the original two equations given in the question:

$$P = I^2 R$$

$$P = {E^2 \over R}$$

And finally, those equations which may be derived from manipulating the last two power equations:

$$R = {P \over I^2}$$

$$I = \sqrt{P \over R}$$

$$E = \sqrt{P R}$$

$$R = {E^2 \over P}$$

%(END_ANSWER)





%(BEGIN_NOTES)

Algebra is an extremely important tool in many technical fields.  One nice thing about the study of electronics is that it provides a relatively simple context in which fundamental algebraic principles may be learned (or at least illuminated).  

The same may be said for calculus concepts as well: basic principles of derivative and integral (with respect to time) may be easily applied to capacitor and inductor circuits, providing students with an accessible context in which these otherwise abstract concepts may be grasped.  But calculus is a topic for later worksheet questions . . .

%INDEX% Algebra, manipulating equations
%INDEX% Algebra, substitution
%INDEX% Joule's Law
%INDEX% Ohm's Law

%(END_NOTES)


