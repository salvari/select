
%(BEGIN_QUESTION)
% Copyright 2004, Tony R. Kuphaldt, released under the Creative Commons Attribution License (v 1.0)
% This means you may do almost anything with this work of mine, so long as you give me proper credit

A primitive form of communication long ago was the use of {\it smoke signals}: interrupting the rising stream of smoke from a fire by waving a blanket over it so that specific sequences of smoke "puffs" could be seen some distance away.  Explain how this is an example of {\it modulation}, albeit in a non-electronic form.

\underbar{file 02265}
%(END_QUESTION)





%(BEGIN_ANSWER)

{\it Modulation} is the impression of information onto an otherwise featureless stream of matter or energy.  In this case, the modulation of a smoke stream by blanket motions should be rather evident.

%(END_ANSWER)





%(BEGIN_NOTES)

It is important for students to understand that modulation is not limited to electronic media.  Stranger examples than this may be cited as proof.  I once spoke with an engineer specializing in vibration measurement who told me of a very odd application of modulation for data communication.  He worked on the design of a vibration sensor that would be embedded in the head of an oil well drill bit.  This sensor was supposed to transmit information to the surface, thousands of feet up, but could not use radio or any other "normal" data media because of the distances involved and the harsh environment.  The solution taken to this unique problem was to have the sensor activate a valve at the drill head which would modulate the flow of drilling mud up to the surface: a byproduct of the drilling process that had to be pumped up to the surface anyway.  By pulsing the normally steady mud flow, digital data could be communicated to pressure sensors at the surface, and then converted into binary data for a computer to archive and translate.  Granted, the bit rate was very slow, but the system worked.

An application like this shows how important it is for students to exercise creativity.  The really interesting problems in life do not yield to "tried and true" solutions, but can only be overcome through the exercise of creativity and skill.  Do everything you can to expose your students to such creative thinking within their discipline(s), and this will help them to become the problem-solvers of tomorrow!

%INDEX% Modulation, smoke signals as an example of

%(END_NOTES)


