
%(BEGIN_QUESTION)
% Copyright 2003, Tony R. Kuphaldt, released under the Creative Commons Attribution License (v 1.0)
% This means you may do almost anything with this work of mine, so long as you give me proper credit

What is a musical {\it chord}?  If viewed on an oscilloscope, what would the signal for a chord look like?

\underbar{file 00647}
%(END_QUESTION)





%(BEGIN_ANSWER)

A {\it chord} is a mixture of three of more notes.  On an oscilloscope, it would appear to be a very complex waveform, very non-sinusoidal.

\vskip 10pt

Note: if you want to see this form yourself without going through the trouble of setting up a musical keyboard (or piano) and oscilloscope, you may simulate it using a graphing calculator or computer program.  Simply graph the sum of three waveforms with the following frequencies:

\medskip
\item{$\bullet$} 261.63 Hz (middle "C")
\item{$\bullet$} 329.63 Hz ("E")
\item{$\bullet$} 392.00 Hz ("G")
\medskip

%(END_ANSWER)





%(BEGIN_NOTES)

Students with a musical background (especially piano) should be able to add substantially to the discussion on this question.  The important concept to discuss here is that multiple frequencies of any signal form (AC voltage, current, sound waves, light waves, etc.) are able to exist simultaneously along the same signal path without interference.

%INDEX% Chord (musical), defined

%(END_NOTES)


