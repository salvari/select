
%(BEGIN_QUESTION)
% Copyright 2008, Tony R. Kuphaldt, released under the Creative Commons Attribution License (v 1.0)
% This means you may do almost anything with this work of mine, so long as you give me proper credit

$$\epsfbox{01629x01.eps}$$

\underbar{file 01629}
\vfil \eject
%(END_QUESTION)





%(BEGIN_ANSWER)

Use circuit simulation software to verify your predicted and measured parameter values.

%(END_ANSWER)





%(BEGIN_NOTES)

Use a variable-voltage, regulated power supply to supply any amount of DC voltage below 30 volts.  Specify a standard resistor value, somewhere between 1 k$\Omega$ and 100 k$\Omega$ (1k5, 2k2, 2k7, 3k3, 4k7, 5k1, 6k8, 10k, 22k, 33k, 39k 47k, 68k, etc.).

An interesting "twist" on this exercise is to specify the value of resistor $R_1$ in colors.  For example: Red, Vio, Red, Gld instead of 2.7 k$\Omega$.

\vskip 10pt

When students set their power supplies for a certain amount of current, it is often helpful to have them do so while it is powering the resistor (rather than connecting their ammeter directly across the power supply to set its current output).  This helps avoid the possibility of blowing the fuse in their ammeter!

%INDEX% Assessment, performance-based (Ohm's Law in simple DC circuit)

%(END_NOTES)


