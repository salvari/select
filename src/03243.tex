
%(BEGIN_QUESTION)
% Copyright 2005, Tony R. Kuphaldt, released under the Creative Commons Attribution License (v 1.0)
% This means you may do almost anything with this work of mine, so long as you give me proper credit

Suppose a 12 volt lead-acid battery has an internal resistance of 20 milli-ohms (20 m$\Omega$):

$$\epsfbox{03243x01.eps}$$

If a short-circuit were placed across the terminals of this large battery, the fault current would be quite large: 600 amps!

Now suppose three of these batteries were connected directly in parallel with one another:

$$\epsfbox{03243x02.eps}$$

Reduce this network of parallel-connected batteries into either a Th\'evenin or a Norton equivalent circuit, and then re-calculate the fault current available at the terminals of the three-battery "bank" in the event of a direct short-circuit.

\underbar{file 03243}
%(END_QUESTION)





%(BEGIN_ANSWER)

$I_{fault}$ = 1800 amps

\vskip 10pt

Follow-up question: explain what practical importance this question has for parallel-connected batteries, and how either Th\'evenin's or Norton's theorems makes the concept easier to explain to someone else.  What safety issues might be raised by the parallel connection of large batteries such as these?

%(END_ANSWER)





%(BEGIN_NOTES)

Ask your students whether they used Th\'evenin's Theorem or Norton's theorem to solve for the fault current.  Have students demonstrate the analysis both ways to see which is easiest to understand.

%INDEX% Thevenin's Theorem, applied to parallel batteries

%(END_NOTES)


