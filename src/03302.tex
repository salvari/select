
%(BEGIN_QUESTION)
% Copyright 2005, Tony R. Kuphaldt, released under the Creative Commons Attribution License (v 1.0)
% This means you may do almost anything with this work of mine, so long as you give me proper credit

Examine the following illustration of a simple battery-switch-lamp circuit, connected together using screw-terminal blocks, each connection point on each terminal block identified by a unique number:

$$\epsfbox{03302x01.eps}$$

\goodbreak
Determine whether or not voltage should be present between the following pairs of terminal block points with the switch in the ON position:

\medskip
\item{$\bullet$} Points 1 and 5:
\item{$\bullet$} Points 6 and 7:
\item{$\bullet$} Points 4 and 10:
\item{$\bullet$} Points 9 and 12:
\item{$\bullet$} Points 6 and 12:
\item{$\bullet$} Points 9 and 10:
\item{$\bullet$} Points 4 and 7:
\medskip

\goodbreak
Now, determine whether or not voltage should be present between the following pairs of terminal block points with the switch in the OFF position:

\medskip
\item{$\bullet$} Points 1 and 5:
\item{$\bullet$} Points 6 and 7:
\item{$\bullet$} Points 4 and 10:
\item{$\bullet$} Points 9 and 12:
\item{$\bullet$} Points 6 and 12:
\item{$\bullet$} Points 9 and 10:
\item{$\bullet$} Points 4 and 7:
\medskip

\underbar{file 03302}
%(END_QUESTION)





%(BEGIN_ANSWER)

Switch ON:

\medskip
\goodbreak
\item{$\bullet$} Points 1 and 5: {\bf Voltage!}
\item{$\bullet$} Points 6 and 7: {\it No voltage}
\item{$\bullet$} Points 4 and 10: {\it No voltage}
\item{$\bullet$} Points 9 and 12: {\bf Voltage!}
\item{$\bullet$} Points 6 and 12: {\it No voltage}
\item{$\bullet$} Points 9 and 10: {\it No voltage}
\item{$\bullet$} Points 4 and 7: {\bf Voltage!}
\medskip

\vskip 10pt

Switch OFF:

\medskip
\goodbreak
\item{$\bullet$} Points 1 and 5: {\bf Voltage!}
\item{$\bullet$} Points 6 and 7: {\it No voltage}
\item{$\bullet$} Points 4 and 10: {\it No voltage}
\item{$\bullet$} Points 9 and 12: {\it No voltage}
\item{$\bullet$} Points 6 and 12: {\bf Voltage!}
\item{$\bullet$} Points 9 and 10: {\it No voltage}
\item{$\bullet$} Points 4 and 7: {\bf Voltage!}
\medskip

\vskip 10pt

Follow-up question: explain {\it why} there will be voltage or no voltage between each of these pairs of points for the two circuit conditions (switch on and switch off).

%(END_ANSWER)





%(BEGIN_NOTES)

This question is not really a troubleshooting question per se, but the principles involved in successfully determining the presence or absence of voltage are critically important to being able to troubleshoot simple circuits using a voltmeter.

I have found that the concept of {\it electrically common points} is most helpful when students first learn to relate voltage drop with continuity (breaks or non-breaks) in a circuit.  You might want them to identify which points in this circuit are electrically common to one another (in either or both switch positions).

%INDEX% Troubleshooting, simple circuit

%(END_NOTES)


