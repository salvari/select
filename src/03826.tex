
%(BEGIN_QUESTION)
% Copyright 2006, Tony R. Kuphaldt, released under the Creative Commons Attribution License (v 1.0)
% This means you may do almost anything with this work of mine, so long as you give me proper credit

Complete the truth table for the following relay logic circuit, and then complete a second truth table for the same circuit with relay coil CR2 failed open:

$$\epsfbox{03826x01.eps}$$

Explain {\it why} the truth table will be modified as a result of the fault.

\underbar{file 03826}
%(END_QUESTION)





%(BEGIN_ANSWER)

$$\epsfbox{03826x02.eps}$$

If you thought that the "faulted" truth table would be all 0's, you probably thought I said relay {\it contact} CR2 failed open.  The fault I proposed was relay CR2 {\bf coil} failed open.

%(END_ANSWER)





%(BEGIN_NOTES)

The purpose of this question is to approach the domain of circuit troubleshooting from a perspective of knowing what the fault is, rather than only knowing what the symptoms are.  Although this is not necessarily a realistic perspective, it helps students build the foundational knowledge necessary to diagnose a faulted circuit from empirical data.  Questions such as this should be followed (eventually) by other questions asking students to identify likely faults based on measurements.

%INDEX% Troubleshooting, predicting effects of fault in relay logic circuit

%(END_NOTES)


