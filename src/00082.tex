
%(BEGIN_QUESTION)
% Copyright 2003, Tony R. Kuphaldt, released under the Creative Commons Attribution License (v 1.0)
% This means you may do almost anything with this work of mine, so long as you give me proper credit

For a given amount of electric current, which resistor will dissipate the greatest amount of power: a small value (low-resistance) resistor, or a high value (high-resistance) resistor?  Explain your answer.

\underbar{file 00082}
%(END_QUESTION)





%(BEGIN_ANSWER)

A resistor with a high resistance rating (many "ohms" of resistance) will dissipate more heat power than a lower-valued resistor, given the same amount of electric current through it.

%(END_ANSWER)





%(BEGIN_NOTES)

This question is designed to make students think {\it qualitatively} about the relationship between current, resistance, and power.  I have found that qualitative (non-numeric) analysis is often more challenging than asking students to calculate answers quantitatively (with numbers).  Often, simple math is a kind of barrier behind which students seek refuge from true understanding of a topic.  In other words, it is easier to punch keys on a calculator (or even perform calculations with paper and pencil) than to really {\it think} about the inter-relationships of variables in a physical problem.  Yet, a qualitative understanding of electrical systems is vital to fast and efficient troubleshooting.

%INDEX% Joule's Law
%INDEX% Power dissipation, resistor

%(END_NOTES)


