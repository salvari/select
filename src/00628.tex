
%(BEGIN_QUESTION)
% Copyright 2003, Tony R. Kuphaldt, released under the Creative Commons Attribution License (v 1.0)
% This means you may do almost anything with this work of mine, so long as you give me proper credit

Define the following terms:

\medskip
\item{$\bullet$} Ferromagnetic
\item{$\bullet$} Paramagnetic
\item{$\bullet$} Diamagnetic
\medskip

\underbar{file 00628}
%(END_QUESTION)





%(BEGIN_ANSWER)

\medskip
\item{$\bullet$} Ferromagnetic: easily magnetized
\item{$\bullet$} Paramagnetic: only slightly magnetic
\item{$\bullet$} Diamagnetic: repels magnetic fields
\medskip

%(END_ANSWER)





%(BEGIN_NOTES)

Ask your students to give examples of a few ferromagnetic materials.  This should be quite easy.

\vskip 10pt

An interesting point of trivia to bring up in discussion is that oxygen, even in its gaseous state, is paramagnetic.  This principle is exploited in the design of some oxygen concentration analyzer equipment, which measure oxygen content by measuring the amount of force generated by a tiny glass container filled with a solution to be sampled!

%INDEX% Ferromagnetism
%INDEX% Paramagnetism
%INDEX% Diamagnetism

%(END_NOTES)


