
%(BEGIN_QUESTION)
% Copyright 2005, Tony R. Kuphaldt, released under the Creative Commons Attribution License (v 1.0)
% This means you may do almost anything with this work of mine, so long as you give me proper credit

When electricity and electronics students begin learning about series circuits and Kirchhoff's Voltage Law, they are often mercilessly subjected to circuits such as this:

$$\epsfbox{03007x01.eps}$$

A common (and legitimate!) question asked by students is, "Where would we ever encounter a circuit such as this, with batteries {\it opposing} each other?"  In practice, it is rare to find electrochemical batteries intentionally connected in such a manner, with some aiding and some opposing.  However, there are many practical applications where the voltages involved are not intentional, but rather are unavoidable potentials created by junctions of dissimilar materials.  Take for instance the application of EKG (electrocardiogram) measurements, where metal electrodes must be placed in contact with a human body to intercept tiny voltage signals from the contracting heart muscles:

$$\epsfbox{03007x02.eps}$$

A junction of metal wire to human skin is surprisingly complex from an electrical perspective, and may be approximately modeled by the following collection of idealized components:

$$\epsfbox{03007x03.eps}$$

Resistors $R_{contact1}$ and $R_{contact2}$ represent electrical resistance between the metal electrodes and skin at the point of contact.  Resistors $R_{tissue1}$ and $R_{tissue2}$ represent electrical resistance of human tissue between the points of electrode contact and the actual heart muscle.  The two series-opposing potentials ($E_1$ and $E_2$) are not intentional, but rather the result of electrochemical action between the metal electrode surfaces and human skin.  They cannot be eliminated, but their combined effect is minimal because they are approximately equal in magnitude.  Explain how Kirchhoff's Voltage Law applies to this equivalent circuit, especially how the biomedical instrument "sees" only the heart muscle voltage signal and neither of the skin-contact potentials.

\underbar{file 03007}
%(END_QUESTION)





%(BEGIN_ANSWER)

Applying KVL to the circuit loop, we find that equal contact potentials $E_1$ and $E_2$ cancel each other out, leaving only the heart muscle voltage to be present at the instrument terminals.

\vskip 10pt

Follow-up question: why is the heart muscle represented in the equivalent circuit by an {\it AC} voltage source symbol rather than by a DC voltage source symbol (battery)?  Does this matter when we apply KVL to the loop?  Why or why not?

%(END_ANSWER)





%(BEGIN_NOTES)

The question of "where will we ever see this?" is too often ignored by teachers, who forget the lack of context present in their new students' understanding.  Remember that this is all {\bf new} to most of your students, so they lack the years of experience you have working with circuits in practical scenarios.  A question like this deserves to be answered, and answered well.

Truth be known, the equivalent circuit for electrode-to-skin contact is far more complex than what is shown here (impedances everywhere!), but the point here is to simplify it enough so that students studying DC circuits and KVL would be able to grasp it.

Electrode half-cell potentials ($E_1$ and $E_2$ in the equivalent schematic diagram) caused by electrochemical action are not the only source of stray potentials in measurement circuits by any means!  Noise voltage is a consideration in many circumstances (whether induced by outside sources or generated by physical action such as Johnson noise), thermal voltages caused by junctions of dissimilar metals, and others.  If all we are doing is making crude measurements, these stray voltages will be of little concern.  If precision is necessary, which is often the case in medical and scientific measurements, these spurious voltages can be devastating.

%INDEX% EKG (electrocardiogram)
%INDEX% Electrode, EKG (simple equivalent circuit)
%INDEX% Kirchhoff's Voltage Law, applied to EKG measurement
%INDEX% Modeling electrode-skin contact as collection of passive components

%(END_NOTES)


