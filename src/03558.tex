
%(BEGIN_QUESTION)
% Copyright 2005, Tony R. Kuphaldt, released under the Creative Commons Attribution License (v 1.0)
% This means you may do almost anything with this work of mine, so long as you give me proper credit

Calculate the rate of change of voltage ($dv \over dt$) for the capacitor at the exact instant in time where the switch moves to the "charge" position.  Assume that prior to this motion the switch had been left in the "discharge" position for some time:

$$\epsfbox{03558x01.eps}$$

\underbar{file 03558}
%(END_QUESTION)





%(BEGIN_ANSWER)

$dv \over dt$ = 4.44 kV/s or 4.44 V/ms

%(END_ANSWER)





%(BEGIN_NOTES)

Some students may think that a rate of change of 4.44 {\it kilo}volts per second harbors shock hazard, because, well, 4.44 {\it thousand} volts is a lot of voltage!  Remind them that this is a {\it rate of change} and not an actual voltage figure.  This number simply tells us how fast the voltage is changing, not how far it will rise to.  It is the difference between saying that a car travels at 75 miles per hour and that a car will travel 75 miles.

%INDEX% RC time constant circuit
%INDEX% Rate-of-change calculation, RC circuit

%(END_NOTES)


