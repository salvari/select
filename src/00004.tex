
%(BEGIN_QUESTION)
% Copyright 2003, Tony R. Kuphaldt, released under the Creative Commons Attribution License (v 1.0)
% This means you may do almost anything with this work of mine, so long as you give me proper credit

It is sometimes said regarding electrical safety that, "It's not the {\it voltage} that will hurt you, it's the {\it current}."  Why then are there signs reading {\bf Danger: High Voltage} near electrical substations and on large pieces of electrical equipment, rather than signs reading {\bf Danger: High Current}?

\underbar{file 00004}
%(END_QUESTION)





%(BEGIN_ANSWER)

Yes, it is electric {\it current} that does the damage when enough of it passes through a body, but current exists only when there is sufficient {\it voltage} to push it through that body's {\it resistance}.  Thus, there is no shock hazard when there is too little voltage present to force dangerous levels of current through a human body.

%(END_ANSWER)





%(BEGIN_NOTES)

Ask students how this question relates to {\it Ohm's Law}.  If they don't know what Ohm's Law is yet, this is a really good place to begin exploring!

%INDEX% Safety, electrical
%INDEX% Voltage versus current
%INDEX% Ohm's Law, conceptual

%(END_NOTES)


