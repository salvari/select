
%(BEGIN_QUESTION)
% Copyright 2005, Tony R. Kuphaldt, released under the Creative Commons Attribution License (v 1.0)
% This means you may do almost anything with this work of mine, so long as you give me proper credit

Calculate the current through resistor $R_2$ in this opamp circuit for several different values of $R_2$:

$$\epsfbox{02511x01.eps}$$

% No blank lines allowed between lines of an \halign structure!
% I use comments (%) instead, so that TeX doesn't choke.

$$\vbox{\offinterlineskip
\halign{\strut
\vrule \quad\hfil # \ \hfil & 
\vrule \quad\hfil # \ \hfil \vrule \cr
\noalign{\hrule}
%
% First row
$R_2$ & $I_{R_2}$ \cr
%
\noalign{\hrule}
%
% Second row
1 k$\Omega$ &    \cr
%
\noalign{\hrule}
%
% Third row
2 k$\Omega$ &    \cr
%
\noalign{\hrule}
%
% Fourth row
3 k$\Omega$ &    \cr
%
\noalign{\hrule}
%
% Fifth row
4 k$\Omega$ &    \cr
%
\noalign{\hrule}
%
% Sixth row
5 k$\Omega$ &    \cr
%
\noalign{\hrule}
%
% Seventh row
6 k$\Omega$ &    \cr
%
\noalign{\hrule}
} % End of \halign 
}$$ % End of \vbox

For each value of $R_2$, what is it that establishes the amount of current through it?  Do you see any practical value for a circuit such as this?

\underbar{file 02511}
%(END_QUESTION)





%(BEGIN_ANSWER)

$$\vbox{\offinterlineskip
\halign{\strut
\vrule \quad\hfil # \ \hfil & 
\vrule \quad\hfil # \ \hfil \vrule \cr
\noalign{\hrule}
%
% First row
$R_2$ & $I_{R_2}$ \cr
%
\noalign{\hrule}
%
% Second row
1 k$\Omega$ & 3 mA \cr
%
\noalign{\hrule}
%
% Third row
2 k$\Omega$ & 3 mA \cr
%
\noalign{\hrule}
%
% Fourth row
3 k$\Omega$ & 3 mA \cr
%
\noalign{\hrule}
%
% Fifth row
4 k$\Omega$ & 3 mA \cr
%
\noalign{\hrule}
%
% Sixth row
5 k$\Omega$ & 3 mA \cr
%
\noalign{\hrule}
%
% Seventh row
6 k$\Omega$ & 3 mA \cr
%
\noalign{\hrule}
} % End of \halign 
}$$ % End of \vbox

This circuit acts like a {\it current mirror}, except much more precise.

\vskip 10pt

Follow-up question: what factor(s) limit the greatest resistance value of $R_2$ that the operational amplifier may sustain 3 milliamps of current through?

%(END_ANSWER)





%(BEGIN_NOTES)

Besides reviewing the purpose of a current mirror circuit, this question draws students' attention to the current-regulating capabilities of an operational amplifier by having them analyze it as though it were simply a noninverting voltage amplifier circuit.

%INDEX% Constant-current circuit, opamp
%INDEX% Voltage-to-current converter, opamp

%(END_NOTES)


